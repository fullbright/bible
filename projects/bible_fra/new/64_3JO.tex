\book[Troisième épître de Jean]{3 Jean}


\chapter[Troisième épître de Jean]

\chaptermark{Troisième épître de Jean}{}
\verse L`ancien, à Gaïus, le bien aimé, que j`aime dans la vérité. 
\verse Bien-aimé, je souhaite que tu prospères à tous égards et sois en bonne santé, comme prospère l`état de ton âme. 
\verse J`ai été fort réjoui, lorsque des frères sont arrivés et ont rendu témoignage de la vérité qui est en toi, de la manière dont tu marches dans la vérité. 
\verse Je n`ai pas de plus grande joie que d`apprendre que mes enfants marchent dans la vérité. 
\verse Bien-aimé, tu agis fidèlement dans ce que tu fais pour les frères, et même pour des frères étrangers, 
\verse lesquels ont rendu témoignage de ta charité, en présence de l`Église. Tu feras bien de pourvoir à leur voyage d`une manière digne de Dieu. 
\verse Car c`est pour le nom de Jésus Christ qu`ils sont partis, sans rien recevoir des païens. 
\verse Nous devons donc accueillir de tels hommes, afin d`être ouvriers avec eux pour la vérité. 
\verse J`ai écrit quelques mots à l`Église; mais Diotrèphe, qui aime à être le premier parmi eux, ne nous reçoit point. 
\verse C`est pourquoi, si je vais vous voir, je rappellerai les actes qu`il commet, en tenant contre nous de méchants propos; non content de cela, il ne reçoit pas les frères, et ceux qui voudraient le faire, il les en empêche et les chasse de l`Église. 
\verse Bien-aimé, n`imite pas le mal, mais le bien. Celui qui fait le bien est de Dieu; celui qui fait le mal n`a point vu Dieu. 
\verse Tous, et la vérité elle-même, rendent un bon témoignage à Démétrius; nous aussi, nous lui rendons témoignage, et tu sais que notre témoignage est vrai. 
\verse J`aurais beaucoup de choses à t`écrire, mais je ne veux pas le faire avec l`encre et la plume. 
\verse J`espère te voir bientôt, et nous parlerons de bouche à bouche. (1:15) Que la paix soit avec toi! Les amis te saluent. Salue les amis, chacun en particulier. 
