\book[Livre de la Genèse]{Genèse}


\chapter[Livre de la Genèse]

\chaptermark{Livre de la Genèse}{}
\verse Au commencement, Dieu créa les cieux et la terre. 
\verse La terre était informe et vide: il y avait des ténèbres à la surface de l`abîme, et l`esprit de Dieu se mouvait au-dessus des eaux. 
\verse Dieu dit: Que la lumière soit! Et la lumière fut. 
\verse Dieu vit que la lumière était bonne; et Dieu sépara la lumière d`avec les ténèbres. 
\verse Dieu appela la lumière jour, et il appela les ténèbres nuit. Ainsi, il y eut un soir, et il y eut un matin: ce fut le premier jour. 
\verse Dieu dit: Qu`il y ait une étendue entre les eaux, et qu`elle sépare les eaux d`avec les eaux. 
\verse Et Dieu fit l`étendue, et il sépara les eaux qui sont au-dessous de l`étendue d`avec les eaux qui sont au-dessus de l`étendue. Et cela fut ainsi. 
\verse Dieu appela l`étendue ciel. Ainsi, il y eut un soir, et il y eut un matin: ce fut le second jour. 
\verse Dieu dit: Que les eaux qui sont au-dessous du ciel se rassemblent en un seul lieu, et que le sec paraisse. Et cela fut ainsi. 
\verse Dieu appela le sec terre, et il appela l`amas des eaux mers. Dieu vit que cela était bon. 
\verse Dieu dit: Qu`il y ait des luminaires dans l`étendue du ciel, pour séparer le jour d`avec la nuit; que ce soient des signes pour marquer les époques, les jours et les années; 
\verse et qu`ils servent de luminaires dans l`étendue du ciel, pour éclairer la terre. Et cela fut ainsi. 
\verse Dieu fit les deux grands luminaires, le plus grand luminaire pour présider au jour, et le plus petit luminaire pour présider à la nuit; il fit aussi les étoiles. 
\verse pour présider au jour et à la nuit, et pour séparer la lumière d`avec les ténèbres. Dieu vit que cela était bon. 
\verse Ainsi, il y eut un soir, et il y eut un matin: ce fut le quatrième jour. 
\verse Dieu dit: Que les eaux produisent en abondance des animaux vivants, et que des oiseaux volent sur la terre vers l`étendue du ciel. 
\verse Dieu créa les grands poissons et tous les animaux vivants qui se meuvent, et que les eaux produisirent en abondance selon leur espèce; il créa aussi tout oiseau ailé selon son espèce. Dieu vit que cela était bon. 
\verse Dieu les bénit, en disant: Soyez féconds, multipliez, et remplissez les eaux des mers; et que les oiseaux multiplient sur la terre. 
\verse Ainsi, il y eut un soir, et il y eut un matin: ce fut le cinquième jour. 
\verse Dieu dit: Que la terre produise des animaux vivants selon leur espèce, du bétail, des reptiles et des animaux terrestres, selon leur espèce. Et cela fut ainsi. 
\verse Dieu fit les animaux de la terre selon leur espèce, le bétail selon son espèce, et tous les reptiles de la terre selon leur espèce. Dieu vit que cela était bon. 
\verse Puis Dieu dit: Faisons l`homme à notre image, selon notre ressemblance, et qu`il domine sur les poissons de la mer, sur les oiseaux du ciel, sur le bétail, sur toute la terre, et sur tous les reptiles qui rampent sur la terre. 
\verse Dieu créa l`homme à son image, il le créa à l`image de Dieu, il créa l`homme et la femme. 
\verse Dieu les bénit, et Dieu leur dit: Soyez féconds, multipliez, remplissez la terre, et l`assujettissez; et dominez sur les poissons de la mer, sur les oiseaux du ciel, et sur tout animal qui se meut sur la terre. 
\verse Et Dieu dit: Voici, je vous donne toute herbe portant de la semence et qui est à la surface de toute la terre, et tout arbre ayant en lui du fruit d`arbre et portant de la semence: ce sera votre nourriture. 
\verse Et à tout animal de la terre, à tout oiseau du ciel, et à tout ce qui se meut sur la terre, ayant en soi un souffle de vie, je donne toute herbe verte pour nourriture. Et cela fut ainsi. 

\chapter[Livre de la Genèse]

\chaptermark{Livre de la Genèse}{}
\verse Ainsi furent achevés les cieux et la terre, et toute leur armée. 
\verse Dieu acheva au septième jour son oeuvre, qu`il avait faite: et il se reposa au septième jour de toute son oeuvre, qu`il avait faite. 
\verse Dieu bénit le septième jour, et il le sanctifia, parce qu`en ce jour il se reposa de toute son oeuvre qu`il avait créée en la faisant. 
\verse Voici les origines des cieux et de la terre, quand ils furent créés. 
\verse Lorsque l`Éternel Dieu fit une terre et des cieux, aucun arbuste des champs n`était encore sur la terre, et aucune herbe des champs ne germait encore: car l`Éternel Dieu n`avait pas fait pleuvoir sur la terre, et il n`y avait point d`homme pour cultiver le sol. 
\verse Mais une vapeur s`éleva de la terre, et arrosa toute la surface du sol. 
\verse Puis l`Éternel Dieu planta un jardin en Éden, du côté de l`orient, et il y mit l`homme qu`il avait formé. 
\verse L`Éternel Dieu fit pousser du sol des arbres de toute espèce, agréables à voir et bons à manger, et l`arbre de la vie au milieu du jardin, et l`arbre de la connaissance du bien et du mal. 
\verse Un fleuve sortait d`Éden pour arroser le jardin, et de là il se divisait en quatre bras. 
\verse Le nom du premier est Pischon; c`est celui qui entoure tout le pays de Havila, où se trouve l`or. 
\verse L`or de ce pays est pur; on y trouve aussi le bdellium et la pierre d`onyx. 
\verse Le nom du second fleuve est Guihon; c`est celui qui entoure tout le pays de Cusch. 
\verse Le nom du troisième est Hiddékel; c`est celui qui coule à l`orient de l`Assyrie. Le quatrième fleuve, c`est l`Euphrate. 
\verse L`Éternel Dieu prit l`homme, et le plaça dans le jardin d`Éden pour le cultiver et pour le garder. 
\verse L`Éternel Dieu donna cet ordre à l`homme: Tu pourras manger de tous les arbres du jardin; 
\verse mais tu ne mangeras pas de l`arbre de la connaissance du bien et du mal, car le jour où tu en mangeras, tu mourras. 
\verse L`Éternel Dieu dit: Il n`est pas bon que l`homme soit seul; je lui ferai une aide semblable à lui. 
\verse L`Éternel Dieu forma de la terre tous les animaux des champs et tous les oiseaux du ciel, et il les fit venir vers l`homme, pour voir comment il les appellerait, et afin que tout être vivant portât le nom que lui donnerait l`homme. 
\verse Et l`homme donna des noms à tout le bétail, aux oiseaux du ciel et à tous les animaux des champs; mais, pour l`homme, il ne trouva point d`aide semblable à lui. 
\verse Alors l`Éternel Dieu fit tomber un profond sommeil sur l`homme, qui s`endormit; il prit une de ses côtes, et referma la chair à sa place. 
\verse L`Éternel Dieu forma une femme de la côte qu`il avait prise de l`homme, et il l`amena vers l`homme. 
\verse Et l`homme dit: Voici cette fois celle qui est os de mes os et chair de ma chair! on l`appellera femme, parce qu`elle a été prise de l`homme. 
\verse C`est pourquoi l`homme quittera son père et sa mère, et s`attachera à sa femme, et ils deviendront une seule chair. 
\verse L`homme et sa femme étaient tous deux nus, et ils n`en avaient point honte. 

\chapter[Livre de la Genèse]

\chaptermark{Livre de la Genèse}{}
\verse Le serpent était le plus rusé de tous les animaux des champs, que l`Éternel Dieu avait faits. Il dit à la femme: Dieu a-t-il réellement dit: Vous ne mangerez pas de tous les arbres du jardin? 
\verse La femme répondit au serpent: Nous mangeons du fruit des arbres du jardin. 
\verse Mais quant au fruit de l`arbre qui est au milieu du jardin, Dieu a dit: Vous n`en mangerez point et vous n`y toucherez point, de peur que vous ne mouriez. 
\verse Alors le serpent dit à la femme: Vous ne mourrez point; 
\verse mais Dieu sait que, le jour où vous en mangerez, vos yeux s`ouvriront, et que vous serez comme des dieux, connaissant le bien et le mal. 
\verse La femme vit que l`arbre était bon à manger et agréable à la vue, et qu`il était précieux pour ouvrir l`intelligence; elle prit de son fruit, et en mangea; elle en donna aussi à son mari, qui était auprès d`elle, et il en mangea. 
\verse Les yeux de l`un et de l`autre s`ouvrirent, ils connurent qu`ils étaient nus, et ayant cousu des feuilles de figuier, ils s`en firent des ceintures. 
\verse Alors ils entendirent la voix de l`Éternel Dieu, qui parcourait le jardin vers le soir, et l`homme et sa femme se cachèrent loin de la face de l`Éternel Dieu, au milieu des arbres du jardin. 
\verse Mais l`Éternel Dieu appela l`homme, et lui dit: Où es-tu? 
\verse Il répondit: J`ai entendu ta voix dans le jardin, et j`ai eu peur, parce que je suis nu, et je me suis caché. 
\verse Et l`Éternel Dieu dit: Qui t`a appris que tu es nu? Est-ce que tu as mangé de l`arbre dont je t`avais défendu de manger? 
\verse L`homme répondit: La femme que tu as mise auprès de moi m`a donné de l`arbre, et j`en ai mangé. 
\verse Et l`Éternel Dieu dit à la femme: Pourquoi as-tu fait cela? La femme répondit: Le serpent m`a séduite, et j`en ai mangé. 
\verse L`Éternel Dieu dit au serpent: Puisque tu as fait cela, tu seras maudit entre tout le bétail et entre tous les animaux des champs, tu marcheras sur ton ventre, et tu mangeras de la poussière tous les jours de ta vie. 
\verse Il dit à la femme: J`augmenterai la souffrance de tes grossesses, tu enfanteras avec douleur, et tes désirs se porteront vers ton mari, mais il dominera sur toi. 
\verse Il dit à l`homme: Puisque tu as écouté la voix de ta femme, et que tu as mangé de l`arbre au sujet duquel je t`avais donné cet ordre: Tu n`en mangeras point! le sol sera maudit à cause de toi. C`est à force de peine que tu en tireras ta nourriture tous les jours de ta vie, 
\verse il te produira des épines et des ronces, et tu mangeras de l`herbe des champs. 
\verse C`est à la sueur de ton visage que tu mangeras du pain, jusqu`à ce que tu retournes dans la terre, d`où tu as été pris; car tu es poussière, et tu retourneras dans la poussière. 
\verse Adam donna à sa femme le nom d`Eve: car elle a été la mère de tous les vivants. 
\verse L`Éternel Dieu fit à Adam et à sa femme des habits de peau, et il les en revêtit. 
\verse L`Éternel Dieu dit: Voici, l`homme est devenu comme l`un de nous, pour la connaissance du bien et du mal. Empêchons-le maintenant d`avancer sa main, de prendre de l`arbre de vie, d`en manger, et de vivre éternellement. 
\verse Et l`Éternel Dieu le chassa du jardin d`Éden, pour qu`il cultivât la terre, d`où il avait été pris. 
\verse C`est ainsi qu`il chassa Adam; et il mit à l`orient du jardin d`Éden les chérubins qui agitent une épée flamboyante, pour garder le chemin de l`arbre de vie. 

\chapter[Livre de la Genèse]

\chaptermark{Livre de la Genèse}{}
\verse Adam connut Eve, sa femme; elle conçut, et enfanta Caïn et elle dit: J`ai formé un homme avec l`aide de l`Éternel. 
\verse Elle enfanta encore son frère Abel. Abel fut berger, et Caïn fut laboureur. 
\verse Au bout de quelque temps, Caïn fit à l`Éternel une offrande des fruits de la terre; 
\verse et Abel, de son côté, en fit une des premiers-nés de son troupeau et de leur graisse. L`Éternel porta un regard favorable sur Abel et sur son offrande; 
\verse mais il ne porta pas un regard favorable sur Caïn et sur son offrande. Caïn fut très irrité, et son visage fut abattu. 
\verse Et l`Éternel dit à Caïn: Pourquoi es-tu irrité, et pourquoi ton visage est-il abattu? 
\verse Certainement, si tu agis bien, tu relèveras ton visage, et si tu agis mal, le péché se couche à la porte, et ses désirs se portent vers toi: mais toi, domine sur lui. 
\verse Cependant, Caïn adressa la parole à son frère Abel; mais, comme ils étaient dans les champs, Caïn se jeta sur son frère Abel, et le tua. 
\verse L`Éternel dit à Caïn: Où est ton frère Abel? Il répondit: Je ne sais pas; suis-je le gardien de mon frère? 
\verse Et Dieu dit: Qu`as-tu fait? La voix du sang de ton frère crie de la terre jusqu`à moi. 
\verse Maintenant, tu seras maudit de la terre qui a ouvert sa bouche pour recevoir de ta main le sang de ton frère. 
\verse Quand tu cultiveras le sol, il ne te donnera plus sa richesse. Tu seras errant et vagabond sur la terre. 
\verse Caïn dit à l`Éternel: Mon châtiment est trop grand pour être supporté. 
\verse Voici, tu me chasses aujourd`hui de cette terre; je serai caché loin de ta face, je serai errant et vagabond sur la terre, et quiconque me trouvera me tuera. 
\verse L`Éternel lui dit: Si quelqu`un tuait Caïn, Caïn serait vengé sept fois. Et l`Éternel mit un signe sur Caïn pour que quiconque le trouverait ne le tuât point. 
\verse Puis, Caïn s`éloigna de la face de l`Éternel, et habita dans la terre de Nod, à l`orient d`Éden. 
\verse Caïn connut sa femme; elle conçut, et enfanta Hénoc. Il bâtit ensuite une ville, et il donna à cette ville le nom de son fils Hénoc. 
\verse Hénoc engendra Irad, Irad engendra Mehujaël, Mehujaël engendra Metuschaël, et Metuschaël engendra Lémec. 
\verse Lémec prit deux femmes: le nom de l`une était Ada, et le nom de l`autre Tsilla. 
\verse Ada enfanta Jabal: il fut le père de ceux qui habitent sous des tentes et près des troupeaux. 
\verse Le nom de son frère était Jubal: il fut le père de tous ceux qui jouent de la harpe et du chalumeau. 
\verse Tsilla, de son côté, enfanta Tubal Caïn, qui forgeait tous les instruments d`airain et de fer. La soeur de Tubal Caïn était Naama. 
\verse Lémec dit à ses femmes: Ada et Tsilla, écoutez ma voix! Femmes de Lémec, écoutez ma parole! J`ai tué un homme pour ma blessure, Et un jeune homme pour ma meurtrissure. 
\verse Caïn sera vengé sept fois, Et Lémec soixante-dix-sept fois. 
\verse Adam connut encore sa femme; elle enfanta un fils, et l`appela du nom de Seth, car, dit-elle, Dieu m`a donnée un autre fils à la place d`Abel, que Caïn a tué. 
\verse Seth eut aussi un fils, et il l`appela du nom d`Énosch. C`est alors que l`on commença à invoquer le nom de l`Éternel. 

\chapter[Livre de la Genèse]

\chaptermark{Livre de la Genèse}{}
\verse Voici le livre de la postérité d`Adam. Lorsque Dieu créa l`homme, il le fit à la ressemblance de Dieu. 
\verse Il créa l`homme et la femme, il les bénit, et il les appela du nom d`homme, lorsqu`ils furent créés. 
\verse Adam, âgé de cent trente ans, engendra un fils à sa ressemblance, selon son image, et il lui donna le nom de Seth. 
\verse Les jours d`Adam, après la naissance de Seth, furent de huit cents ans; et il engendra des fils et des filles. 
\verse Tous les jours qu`Adam vécut furent de neuf cent trente ans; puis il mourut. 
\verse Seth, âgé de cent cinq ans, engendra Énosch. 
\verse Seth vécut, après la naissance d`Énosch, huit cent sept ans; et il engendra des fils et des filles. 
\verse Énosch, âgé de quatre-vingt-dix ans, engendra Kénan. 
\verse Énosch vécut, après la naissance de Kénan, huit cent quinze ans; et il engendra des fils et des filles. 
\verse Kénan, âgé de soixante-dix ans, engendra Mahalaleel. 
\verse Kénan vécut, après la naissance de Mahalaleel, huit cent quarante ans; et il engendra des fils et des filles. 
\verse Tous les jours de Kénan furent de neuf cent dix ans; puis il mourut. 
\verse Mahalaleel, âgé de soixante-cinq ans, engendra Jéred. 
\verse Mahalaleel vécut, après la naissance de Jéred, huit cent trente ans; et il engendra des fils et des filles. 
\verse Tous les jours de Mahalaleel furent de huit cent quatre-vingt-quinze ans; puis il mourut. 
\verse Jéred, âgé de cent soixante-deux ans, engendra Hénoc. 
\verse Jéred vécut, après la naissance d`Hénoc, huit cents ans; et il engendra des fils et des filles. 
\verse Tous les jours de Jéred furent de neuf cent soixante-deux ans; puis il mourut. 
\verse Hénoc, âgé de soixante-cinq ans, engendra Metuschélah. 
\verse Hénoc, après la naissance de Metuschélah, marcha avec Dieu trois cents ans; et il engendra des fils et des filles. 
\verse Tous les jours d`Hénoc furent de trois cent soixante-cinq ans. 
\verse Hénoc marcha avec Dieu; puis il ne fut plus, parce que Dieu le prit. 
\verse Metuschélah, âgé de cent quatre-vingt-sept ans, engendra Lémec. 
\verse Metuschélah vécut, après la naissance de Lémec, sept cent quatre-vingt deux ans; et il engendra des fils et des filles. 
\verse Tous les jours de Metuschélah furent de neuf cent soixante-neuf ans; puis il mourut. 
\verse Lémec, âgé de cent quatre-vingt-deux ans, engendra un fils. 
\verse Il lui donna le nom de Noé, en disant: Celui-ci nous consolera de nos fatigues et du travail pénible de nos mains, provenant de cette terre que l`Éternel a maudite. 
\verse Lémec vécut, après la naissance de Noé, cinq cent quatre-vingt-quinze ans; et il engendra des fils et des filles. 
\verse Tous les jours de Lémec furent de sept cent soixante-dix sept ans; puis il mourut. 
\verse Noé, âgé de cinq cents ans, engendra Sem, Cham et Japhet. 

\chapter[Livre de la Genèse]

\chaptermark{Livre de la Genèse}{}
\verse Lorsque les hommes eurent commencé à se multiplier sur la face de la terre, et que des filles leur furent nées, 
\verse les fils de Dieu virent que les filles des hommes étaient belles, et ils en prirent pour femmes parmi toutes celles qu`ils choisirent. 
\verse Alors l`Éternel dit: Mon esprit ne restera pas à toujours dans l`homme, car l`homme n`est que chair, et ses jours seront de cent vingt ans. 
\verse Les géants étaient sur la terre en ces temps-là, après que les fils de Dieu furent venus vers les filles des hommes, et qu`elles leur eurent donné des enfants: ce sont ces héros qui furent fameux dans l`antiquité. 
\verse Mais Noé trouva grâce aux yeux de l`Éternel. 
\verse Voici la postérité de Noé. Noé était un homme juste et intègre dans son temps; Noé marchait avec Dieu. 
\verse Noé engendra trois fils: Sem, Cham et Japhet. 
\verse La terre était corrompue devant Dieu, la terre était pleine de violence. 
\verse Dieu regarda la terre, et voici, elle était corrompue; car toute chair avait corrompu sa voie sur la terre. 
\verse Alors Dieu dit à Noé: La fin de toute chair est arrêtée par devers moi; car ils ont rempli la terre de violence; voici, je vais les détruire avec la terre. 
\verse Fais-toi une arche de bois de gopher; tu disposeras cette arche en cellules, et tu l`enduiras de poix en dedans et en dehors. 
\verse Voici comment tu la feras: l`arche aura trois cents coudées de longueur, cinquante coudées de largeur et trente coudées de hauteur. 
\verse Tu feras à l`arche une fenêtre, que tu réduiras à une coudée en haut; tu établiras une porte sur le côté de l`arche; et tu construiras un étage inférieur, un second et un troisième. 
\verse Mais j`établis mon alliance avec toi; tu entreras dans l`arche, toi et tes fils, ta femme et les femmes de tes fils avec toi. 
\verse De tout ce qui vit, de toute chair, tu feras entrer dans l`arche deux de chaque espèce, pour les conserver en vie avec toi: il y aura un mâle et une femelle. 
\verse Des oiseaux selon leur espèce, du bétail selon son espèce, et de tous les reptiles de la terre selon leur espèce, deux de chaque espèce viendront vers toi, pour que tu leur conserves la vie. 
\verse Et toi, prends de tous les aliments que l`on mange, et fais-en une provision auprès de toi, afin qu`ils te servent de nourriture ainsi qu`à eux. 
\verse C`est ce que fit Noé: il exécuta tout ce que Dieu lui avait ordonné. 
