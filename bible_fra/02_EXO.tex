\book[Livre de l'Exode]{Exode}


\chapter
\verse Voici les noms des fils d`Israël, venus en Égypte avec Jacob et la famille de chacun d`eux: 
\verse Ruben, Siméon, Lévi, Juda, 
\verse Issacar, Zabulon, Benjamin, 
\verse Dan, Nephthali, Gad et Aser. 
\verse Les personnes issues de Jacob étaient au nombre de soixante-dix en tout. Joseph était alors en Égypte. 
\verse Joseph mourut, ainsi que tous ses frères et toute cette génération-là. 
\verse Les enfants d`Israël furent féconds et multiplièrent, ils s`accrurent et devinrent de plus en plus puissants. Et le pays en fut rempli. 
\verse Il s`éleva sur l`Égypte un nouveau roi, qui n`avait point connu Joseph. 
\verse Il dit à son peuple: Voilà les enfants d`Israël qui forment un peuple plus nombreux et plus puissant que nous. 
\verse Allons! montrons-nous habiles à son égard; empêchons qu`il ne s`accroisse, et que, s`il survient une guerre, il ne se joigne à nos ennemis, pour nous combattre et sortir ensuite du pays. 
\verse Et l`on établit sur lui des chefs de corvées, afin de l`accabler de travaux pénibles. C`est ainsi qu`il bâtit les villes de Pithom et de Ramsès, pour servir de magasins à Pharaon. 
\verse Mais plus on l`accablait, plus il multipliait et s`accroissait; et l`on prit en aversion les enfants d`Israël. 
\verse Alors les Égyptiens réduisirent les enfants d`Israël à une dure servitude. 
\verse Ils leur rendirent la vie amère par de rudes travaux en argile et en briques, et par tous les ouvrages des champs: et c`était avec cruauté qu`ils leur imposaient toutes ces charges. 
\verse Le roi d`Égypte parla aussi aux sages-femmes des Hébreux, nommées l`une Schiphra, et l`autre Pua. 
\verse Il leur dit: Quand vous accoucherez les femmes des Hébreux et que vous les verrez sur les sièges, si c`est un garçon, faites-le mourir; si c`est une fille, laissez-la vivre. 
\verse Mais les sages-femmes craignirent Dieu, et ne firent point ce que leur avait dit le roi d`Égypte; elles laissèrent vivre les enfants. 
\verse Le roi d`Égypte appela les sages-femmes, et leur dit: Pourquoi avez-vous agi ainsi, et avez-vous laissé vivre les enfants? 
\verse Les sages-femmes répondirent à Pharaon: C`est que les femmes des Hébreux ne sont pas comme les Égyptiennes; elles sont vigoureuses et elles accouchent avant l`arrivée de la sage-femme. 
\verse Dieu fit du bien aux sages-femmes; et le peuple multiplia et devint très nombreux. 
\verse Parce que les sages-femmes avaient eu la crainte de Dieu, Dieu fit prospérer leurs maisons. 
\verse Alors Pharaon donna cet ordre à tout son peuple: Vous jetterez dans le fleuve tout garçon qui naîtra, et vous laisserez vivre toutes les filles. 

\chapter
\verse Un homme de la maison de Lévi avait pris pour femme une fille de Lévi. 
\verse Cette femme devint enceinte et enfanta un fils. Elle vit qu`il était beau, et elle le cacha pendant trois mois. 
\verse Ne pouvant plus le cacher, elle prit une caisse de jonc, qu`elle enduisit de bitume et de poix; elle y mit l`enfant, et le déposa parmi les roseaux, sur le bord du fleuve. 
\verse La soeur de l`enfant se tint à quelque distance, pour savoir ce qui lui arriverait. 
\verse La fille de Pharaon descendit au fleuve pour se baigner, et ses compagnes se promenèrent le long du fleuve. Elle aperçut la caisse au milieu des roseaux, et elle envoya sa servante pour la prendre. 
\verse Elle l`ouvrit, et vit l`enfant: c`était un petit garçon qui pleurait. Elle en eut pitié, et elle dit: C`est un enfant des Hébreux! 
\verse Alors la soeur de l`enfant dit à la fille de Pharaon: Veux-tu que j`aille te chercher une nourrice parmi les femmes des Hébreux, pour allaiter cet enfant? 
\verse Va, lui répondit la fille de Pharaon. Et la jeune fille alla chercher la mère de l`enfant. 
\verse La fille de Pharaon lui dit: Emporte cet enfant, et allaite-le-moi; je te donnerai ton salaire. La femme prit l`enfant, et l`allaita. 
\verse Quand il eut grandi, elle l`amena à la fille de Pharaon, et il fut pour elle comme un fils. Elle lui donna le nom de Moïse, car, dit-elle, je l`ai retiré des eaux. 
\verse En ce temps-là, Moïse, devenu grand, se rendit vers ses frères, et fut témoin de leurs pénibles travaux. Il vit un Égyptien qui frappait un Hébreu d`entre ses frères. 
\verse Il regarda de côté et d`autre, et, voyant qu`il n`y avait personne, il tua l`Égyptien, et le cacha dans le sable. 
\verse Il sortit le jour suivant; et voici, deux Hébreux se querellaient. Il dit à celui qui avait tort: Pourquoi frappes-tu ton prochain? 
\verse Et cet homme répondit: Qui t`a établi chef et juge sur nous? Penses-tu me tuer, comme tu as tué l`Égyptien? Moïse eut peur, et dit: Certainement la chose est connue. 
\verse Pharaon apprit ce qui s`était passé, et il cherchait à faire mourir Moïse. Mais Moïse s`enfuit de devant Pharaon, et il se retira dans le pays de Madian, où il s`arrêta près d`un puits. 
\verse Le sacrificateur de Madian avait sept filles. Elle vinrent puiser de l`eau, et elles remplirent les auges pour abreuver le troupeau de leur père. 
\verse Les bergers arrivèrent, et les chassèrent. Alors Moïse se leva, prit leur défense, et fit boire leur troupeau. 
\verse Quand elles furent de retour auprès de Réuel, leur père, il dit: Pourquoi revenez-vous si tôt aujourd`hui? 
\verse Elles répondirent: Un Égyptien nous a délivrées de la main des bergers, et même il nous a puisé de l`eau, et a fait boire le troupeau. 
\verse Et il dit à ses filles: Où est-il? Pourquoi avez-vous laissé cet homme? Appelez-le, pour qu`il prenne quelque nourriture. 
\verse Moïse se décida à demeurer chez cet homme, qui lui donna pour femme Séphora, sa fille. 
\verse Elle enfanta un fils, qu`il appela du nom de Guerschom, car, dit-il, j`habite un pays étranger. 
\verse Longtemps après, le roi d`Égypte mourut, et les enfants d`Israël gémissaient encore sous la servitude, et poussaient des cris. Ces cris, que leur arrachait la servitude, montèrent jusqu`à Dieu. 
\verse Dieu entendit leurs gémissements, et se souvint de son alliance avec Abraham, Isaac et Jacob. 
\verse Dieu regarda les enfants d`Israël, et il en eut compassion. 

\chapter
\verse Moïse faisait paître le troupeau de Jéthro, son beau-père, sacrificateur de Madian; et il mena le troupeau derrière le désert, et vint à la montagne de Dieu, à Horeb. 
\verse L`ange de l`Éternel lui apparut dans une flamme de feu, au milieu d`un buisson. Moïse regarda; et voici, le buisson était tout en feu, et le buisson ne se consumait point. 
\verse Moïse dit: Je veux me détourner pour voir quelle est cette grande vision, et pourquoi le buisson ne se consume point. 
\verse L`Éternel vit qu`il se détournait pour voir; et Dieu l`appela du milieu du buisson, et dit: Moïse! Moïse! Et il répondit: Me voici! 
\verse Dieu dit: N`approche pas d`ici, ôte tes souliers de tes pieds, car le lieu sur lequel tu te tiens est une terre sainte. 
\verse Et il ajouta: Je suis le Dieu de ton père, le Dieu d`Abraham, le Dieu d`Isaac et le Dieu de Jacob. Moïse se cacha le visage, car il craignait de regarder Dieu. 
\verse L`Éternel dit: J`ai vu la souffrance de mon peuple qui est en Égypte, et j`ai entendu les cris que lui font pousser ses oppresseurs, car je connais ses douleurs. 
\verse Je suis descendu pour le délivrer de la main des Égyptiens, et pour le faire monter de ce pays dans un bon et vaste pays, dans un pays où coulent le lait et le miel, dans les lieux qu`habitent les Cananéens, les Héthiens, les Amoréens, les Phéréziens, les Héviens et les Jébusiens. 
\verse Voici, les cris d`Israël sont venus jusqu`à moi, et j`ai vu l`oppression que leur font souffrir les Égyptiens. 
\verse Maintenant, va, je t`enverrai auprès de Pharaon, et tu feras sortir d`Égypte mon peuple, les enfants d`Israël. 
\verse Moïse dit à Dieu: Qui suis-je, pour aller vers Pharaon, et pour faire sortir d`Égypte les enfants d`Israël? 
\verse Dieu dit: Je serai avec toi; et ceci sera pour toi le signe que c`est moi qui t`envoie: quand tu auras fait sortir d`Égypte le peuple, vous servirez Dieu sur cette montagne. 
\verse Moïse dit à Dieu: J`irai donc vers les enfants d`Israël, et je leur dirai: Le Dieu de vos pères m`envoie vers vous. Mais, s`ils me demandent quel est son nom, que leur répondrai-je? 
\verse Dieu dit à Moïse: Je suis celui qui suis. Et il ajouta: C`est ainsi que tu répondras aux enfants d`Israël: Celui qui s`appelle `je suis`m`a envoyé vers vous. 
\verse Dieu dit encore à Moïse: Tu parleras ainsi aux enfants d`Israël: L`Éternel, le Dieu de vos pères, le Dieu d`Abraham, le Dieu d`Isaac et le Dieu de Jacob, m`envoie vers vous. Voilà mon nom pour l`éternité, voilà mon nom de génération en génération. 
\verse Va, rassemble les anciens d`Israël, et dis-leur: L`Éternel, le Dieu de vos pères, m`est apparu, le Dieu d`Abraham, d`Isaac et de Jacob. Il a dit: Je vous ai vus, et j`ai vu ce qu`on vous fait en Égypte, 
\verse et j`ai dit: Je vous ferai monter de l`Égypte, où vous souffrez, dans le pays des Cananéens, des Héthiens, des Amoréens, des Phéréziens, des Héviens et des Jébusiens, dans un pays où coulent le lait et le miel. 
\verse Ils écouteront ta voix; et tu iras, toi et les anciens d`Israël, auprès du roi d`Égypte, et vous lui direz: L`Éternel, le Dieu des Hébreux, nous est apparu. Permets-nous de faire trois journées de marche dans le désert, pour offrir des sacrifices à l`Éternel, notre Dieu. 
\verse Je sais que le roi d`Égypte ne vous laissera point aller, si ce n`est par une main puissante. 
\verse J`étendrai ma main, et je frapperai l`Égypte par toutes sortes de prodiges que je ferai au milieu d`elle. Après quoi, il vous laissera aller. 
\verse Je ferai même trouver grâce à ce peuple aux yeux des Égyptiens, et quand vous partirez, vous ne partirez point à vide. 
\verse Chaque femme demandera à sa voisine et à celle qui demeure dans sa maison des vases d`argent, des vases d`or, et des vêtements, que vous mettrez sur vos fils et vos filles. Et vous dépouillerez les Égyptiens. 

\chapter
\verse Moïse répondit, et dit: Voici, ils ne me croiront point, et ils n`écouteront point ma voix. Mais ils diront: L`Éternel ne t`est point apparu. 
\verse L`Éternel lui dit: Qu`y a-t-il dans ta main? Il répondit: Une verge. 
\verse L`Éternel dit: Jette-la par terre. Il la jeta par terre, et elle devint un serpent. Moïse fuyait devant lui. 
\verse L`Éternel dit à Moïse: Étends ta main, et saisis-le par la queue. Il étendit la main et le saisit et le serpent redevint une verge dans sa main. 
\verse C`est là, dit l`Éternel, ce que tu feras, afin qu`ils croient que l`Éternel, le Dieu de leurs pères, t`est apparu, le Dieu d`Abraham, le Dieu d`Isaac et le Dieu de Jacob. 
\verse L`Éternel lui dit encore: Mets ta main dans ton sein. Il mit sa main dans son sein; puis il la retira, et voici, sa main était couverte de lèpre, blanche comme la neige. 
\verse L`Éternel dit: Remets ta main dans ton sein. Il remit sa main dans son sein; puis il la retira de son sein, et voici, elle était redevenue comme sa chair. 
\verse S`ils ne te croient pas, dit l`Éternel, et n`écoutent pas la voix du premier signe, ils croiront à la voix du dernier signe. 
\verse S`ils ne croient pas même à ces deux signes, et n`écoutent pas ta voix, tu prendras de l`eau du fleuve, tu la répandras sur la terre, et l`eau que tu auras prise du fleuve deviendra du sang sur la terre. 
\verse Moïse dit à l`Éternel: Ah! Seigneur, je ne suis pas un homme qui ait la parole facile, et ce n`est ni d`hier ni d`avant-hier, ni même depuis que tu parles à ton serviteur; car j`ai la bouche et la langue embarrassées. 
\verse L`Éternel lui dit: Qui a fait la bouche de l`homme? et qui rend muet ou sourd, voyant ou aveugle? N`est-ce pas moi, l`Éternel? 
\verse Va donc, je serai avec ta bouche, et je t`enseignerai ce que tu auras à dire. 
\verse Moïse dit: Ah! Seigneur, envoie qui tu voudras envoyer. 
\verse Alors la colère de l`Éternel s`enflamma contre Moïse, et il dit: N`y a t-il pas ton frère Aaron, le Lévite? Je sais qu`il parlera facilement. Le voici lui-même, qui vient au-devant de toi; et, quand il te verra, il se réjouira dans son coeur. 
\verse Tu lui parleras, et tu mettras les paroles dans sa bouche; et moi, je serai avec ta bouche et avec sa bouche, et je vous enseignerai ce que vous aurez à faire. 
\verse Il parlera pour toi au peuple; il te servira de bouche, et tu tiendras pour lui la place de Dieu. 
\verse Prends dans ta main cette verge, avec laquelle tu feras les signes. 
\verse Moïse s`en alla; et de retour auprès de Jéthro, son beau-père, il lui dit: Laisse-moi, je te prie, aller rejoindre mes frères qui sont en Égypte, afin que je voie s`ils sont encore vivants. Jéthro dit à Moïse: Va en paix. 
\verse L`Éternel dit à Moïse, en Madian: Va, retourne en Égypte, car tous ceux qui en voulaient à ta vie sont morts. 
\verse Moïse prit sa femme et ses fils, les fit monter sur des ânes, et retourna dans le pays d`Égypte. Il prit dans sa main la verge de Dieu. 
\verse L`Éternel dit à Moïse: En partant pour retourner en Égypte, vois tous les prodiges que je mets en ta main: tu les feras devant Pharaon. Et moi, j`endurcirai son coeur, et il ne laissera point aller le peuple. 
\verse Tu diras à Pharaon: Ainsi parle l`Éternel: Israël est mon fils, mon premier-né. 
\verse Je te dis: Laisse aller mon fils, pour qu`il me serve; si tu refuses de le laisser aller, voici, je ferai périr ton fils, ton premier-né. 
\verse Pendant le voyage, en un lieu où Moïse passa la nuit, l`Éternel l`attaqua et voulut le faire mourir. 
\verse Séphora prit une pierre aiguë, coupa le prépuce de son fils, et le jeta aux pieds de Moïse, en disant: Tu es pour moi un époux de sang! 
\verse Et l`Éternel le laissa. C`est alors qu`elle dit: Époux de sang! à cause de la circoncision. 
\verse L`Éternel dit à Aaron: Va dans le désert au-devant de Moïse. Aaron partit; il rencontra Moïse à la montagne de Dieu, et il le baisa. 
\verse Moïse fit connaître à Aaron toutes les paroles de l`Éternel qui l`avait envoyé, et tous les signes qu`il lui avait ordonné de faire. 
\verse Moïse et Aaron poursuivirent leur chemin, et ils assemblèrent tous les anciens des enfants d`Israël. 
\verse Aaron rapporta toutes les paroles que l`Éternel avait dites à Moïse, et il exécuta les signes aux yeux du peuple. 
\verse Et le peuple crut. Ils apprirent que l`Éternel avait visité les enfants d`Israël, qu`il avait vu leur souffrance; et ils s`inclinèrent et se prosternèrent. 

\chapter
\verse Moïse et Aaron se rendirent ensuite auprès de Pharaon, et lui dirent: Ainsi parle l`Éternel, le Dieu d`Israël: Laisse aller mon peuple, pour qu`il célèbre au désert une fête en mon honneur. 
\verse Pharaon répondit: Qui est l`Éternel, pour que j`obéisse à sa voix, en laissant aller Israël? Je ne connais point l`Éternel, et je ne laisserai point aller Israël. 
\verse Ils dirent: Le Dieu des Hébreux nous est apparu. Permets-nous de faire trois journées de marche dans le désert, pour offrir des sacrifices à l`Éternel, afin qu`il ne nous frappe pas de la peste ou de l`épée. 
\verse Et le roi d`Égypte leur dit: Moïse et Aaron, pourquoi détournez-vous le peuple de son ouvrage? Allez à vos travaux. 
\verse Pharaon dit: Voici, ce peuple est maintenant nombreux dans le pays, et vous lui feriez interrompre ses travaux! 
\verse Et ce jour même, Pharaon donna cet ordre aux inspecteurs du peuple et aux commissaires: 
\verse Vous ne donnerez plus comme auparavant de la paille au peuple pour faire des briques; qu`ils aillent eux-mêmes ramasser de la paille. 
\verse Vous leur imposerez néanmoins la quantité de briques qu`ils faisaient auparavant, vous n`en retrancherez rien; car ce sont des paresseux; voilà pourquoi ils crient, en disant: Allons offrir des sacrifices à notre Dieu! 
\verse Que l`on charge de travail ces gens, qu`ils s`en occupent, et ils ne prendront plus garde à des paroles de mensonge. 
\verse Les inspecteurs du peuple et les commissaires vinrent dire au peuple: Ainsi parle Pharaon: Je ne vous donne plus de paille; 
\verse allez vous-mêmes vous procurer de la paille où vous en trouverez, car l`on ne retranche rien de votre travail. 
\verse Le peuple se répandit dans tout le pays d`Égypte, pour ramasser du chaume au lieu de paille. 
\verse Les inspecteurs les pressaient, en disant: Achevez votre tâche, jour par jour, comme quand il y avait de la paille. 
\verse On battit même les commissaires des enfants d`Israël, établis sur eux par les inspecteurs de Pharaon: Pourquoi, disait-on, n`avez-vous pas achevé hier et aujourd`hui, comme auparavant, la quantité de briques qui vous avait été fixée? 
\verse Les commissaires des enfants d`Israël allèrent se plaindre à Pharaon, et lui dirent: Pourquoi traites-tu ainsi tes serviteurs? 
\verse On ne donne point de paille à tes serviteurs, et l`on nous dit: Faites des briques! Et voici, tes serviteurs sont battus, comme si ton peuple était coupable. 
\verse Pharaon répondit: Vous êtes des paresseux, des paresseux! Voilà pourquoi vous dites: Allons offrir des sacrifices à l`Éternel! 
\verse Maintenant, allez travailler; on ne vous donnera point de paille, et vous livrerez la même quantité de briques. 
\verse Les commissaires des enfants d`Israël virent qu`on les rendait malheureux, en disant: Vous ne retrancherez rien de vos briques; chaque jour la tâche du jour. 
\verse En sortant de chez Pharaon, ils rencontrèrent Moïse et Aaron qui les attendaient. 
\verse Ils leur dirent: Que l`Éternel vous regarde, et qu`il juge! Vous nous avez rendus odieux à Pharaon et à ses serviteurs, vous avez mis une épée dans leurs mains pour nous faire périr. 
\verse Moïse retourna vers l`Éternel, et dit: Seigneur, pourquoi as-tu fait du mal à ce peuple? pourquoi m`as-tu envoyé? 
\verse Depuis que je suis allé vers Pharaon pour parler en ton nom, il fait du mal à ce peuple, et tu n`as point délivré ton peuple. 

\chapter
\verse L`Éternel dit à Moïse: Tu verras maintenant ce que je ferai à Pharaon; une main puissante le forcera à les laisser aller, une main puissante le forcera à les chasser de son pays. 
\verse Dieu parla encore à Moïse, et lui dit: Je suis l`Éternel. 
\verse Je suis apparu à Abraham, à Isaac et à Jacob, comme le Dieu tout puissant; mais je n`ai pas été connu d`eux sous mon nom, l`Éternel. 
\verse J`ai aussi établi mon alliance avec eux, pour leur donner le pays de Canaan, le pays de leurs pèlerinages, dans lequel ils ont séjourné. 
\verse J`ai entendu les gémissements des enfants d`Israël, que les Égyptiens tiennent dans la servitude, et je me suis souvenu de mon alliance. 
\verse C`est pourquoi dis aux enfants d`Israël: Je suis l`Éternel, je vous affranchirai des travaux dont vous chargent les Égyptiens, je vous délivrerai de leur servitude, et je vous sauverai à bras étendu et par de grands jugements. 
\verse Je vous prendrai pour mon peuple, je serai votre Dieu, et vous saurez que c`est moi, l`Éternel, votre Dieu, qui vous affranchis des travaux dont vous chargent les Égyptiens. 
\verse Je vous ferai entrer dans le pays que j`ai juré de donner à Abraham, à Isaac et à Jacob; je vous le donnerai en possession, moi l`Éternel. 
\verse Ainsi parla Moïse aux enfants d`Israël. Mais l`angoisse et la dure servitude les empêchèrent d`écouter Moïse. 
\verse L`Éternel parla à Moïse, et dit: 
\verse Va, parle à Pharaon, roi d`Égypte, pour qu`il laisse aller les enfants d`Israël hors de son pays. 
\verse Moïse répondit en présence de l`Éternel: Voici, les enfants d`Israël ne m`ont point écouté; comment Pharaon m`écouterait-il, moi qui n`ai pas la parole facile? 
\verse L`Éternel parla à Moïse et à Aaron, et leur donna des ordres au sujet des enfants d`Israël et au sujet de Pharaon, roi d`Égypte, pour faire sortir du pays d`Égypte les enfants d`Israël. 
\verse Voici les chefs de leurs familles. Fils de Ruben, premier-né d`Israël: Hénoc, Pallu, Hetsron et Carmi. Ce sont là les familles de Ruben. 
\verse Fils de Siméon: Jemuel, Jamin, Ohad, Jakin et Tsochar; et Saül, fils de la Cananéenne. Ce sont là les familles de Siméon. 
\verse Voici les noms des fils de Lévi, avec leur postérité: Guerschon, Kehath et Merari. Les années de la vie de Lévi furent de cent trente-sept ans. - 
\verse Fils de Guerschon: Libni et Schimeï, et leurs familles. - 
\verse Fils de Kehath: Amram, Jitsehar, Hébron et Uziel. Les années de la vie de Kehath furent de cent trente-trois ans. - 
\verse Fils de Merari: Machli et Muschi. -Ce sont là les familles de Lévi, avec leur postérité. 
\verse Amram prit pour femme Jokébed, sa tante; et elle lui enfanta Aaron, et Moïse. Les années de la vie d`Amram furent de cent trente-sept ans. - 
\verse Fils de Jitsehar: Koré, Népheg et Zicri. - 
\verse Fils d`Uziel: Mischaël, Eltsaphan et Sithri. 
\verse Aaron prit pour femme Élischéba, fille d`Amminadab, soeur de Nachschon; et elle lui enfanta Nadab, Abihu, Éléazar et Ithamar. 
\verse Fils de Koré: Assir, Elkana et Abiasaph. Ce sont là les familles des Korites. 
\verse Éléazar, fils d`Aaron, prit pour femme une des filles de Puthiel; et elle lui enfanta Phinées. Tels sont les chefs de famille des Lévites, avec leurs familles. 
\verse Ce sont là cet Aaron et ce Moïse, à qui l`Éternel dit: Faites sortir du pays d`Égypte les enfants d`Israël, selon leurs armées. 
\verse Ce sont eux qui parlèrent à Pharaon, roi d`Égypte, pour faire sortir d`Égypte les enfants d`Israël. Ce sont là ce Moïse et cet Aaron. 
\verse Lorsque l`Éternel parla à Moïse dans le pays d`Égypte, 
\verse l`Éternel dit à Moïse: Je suis l`Éternel. Dis à Pharaon, roi d`Égypte, tout ce que je te dis. 
\verse Et Moïse répondit en présence de l`Éternel: Voici, je n`ai pas la parole facile; comment Pharaon m`écouterait-il? 

\chapter
\verse L`Éternel dit à Moïse: Vois, je te fais Dieu pour Pharaon: et Aaron, ton frère, sera ton prophète. 
\verse Toi, tu diras tout ce que je t`ordonnerai; et Aaron, ton frère, parlera à Pharaon, pour qu`il laisse aller les enfants d`Israël hors de son pays. 
\verse Et moi, j`endurcirai le coeur de Pharaon, et je multiplierai mes signes et mes miracles dans le pays d`Égypte. 
\verse Pharaon ne vous écoutera point. Je mettrai ma main sur l`Égypte, et je ferai sortir du pays d`Égypte mes armées, mon peuple, les enfants d`Israël, par de grands jugements. 
\verse Les Égyptiens connaîtront que je suis l`Éternel, lorsque j`étendrai ma main sur l`Égypte, et que je ferai sortir du milieu d`eux les enfants d`Israël. 
\verse Moïse et Aaron firent ce que l`Éternel leur avait ordonné; ils firent ainsi. 
\verse Moïse était âgé de quatre-vingts ans, et Aaron de quatre-vingt-trois ans, lorsqu`ils parlèrent à Pharaon. 
\verse L`Éternel dit à Moïse et à Aaron: 
\verse Si Pharaon vous parle, et vous dit: Faites un miracle! tu diras à Aaron: Prends ta verge, et jette-la devant Pharaon. Elle deviendra un serpent. 
\verse Moïse et Aaron allèrent auprès de Pharaon, et ils firent ce que l`Éternel avait ordonné. Aaron jeta sa verge devant Pharaon et devant ses serviteurs; et elle devint un serpent. 
\verse Mais Pharaon appela des sages et des enchanteurs; et les magiciens d`Égypte, eux aussi, en firent autant par leurs enchantements. 
\verse Ils jetèrent tous leurs verges, et elles devinrent des serpents. Et la verge d`Aaron engloutit leurs verges. 
\verse Le coeur de Pharaon s`endurcit, et il n`écouta point Moïse et Aaron selon ce que l`Éternel avait dit. 
\verse L`Éternel dit à Moïse: Pharaon a le coeur endurci; il refuse de laisser aller le peuple. 
\verse Va vers Pharaon dès le matin; il sortira pour aller près de l`eau, et tu te présenteras devant lui au bord du fleuve. Tu prendras à ta main la verge qui a été changée en serpent, 
\verse et tu diras à Pharaon: L`Éternel, le Dieu des Hébreux, m`a envoyé auprès de toi, pour te dire: Laisse aller mon peuple, afin qu`il me serve dans le désert. Et voici, jusqu`à présent tu n`as point écouté. 
\verse Ainsi parle l`Éternel: A ceci tu connaîtras que je suis l`Éternel. Je vais frapper les eaux du fleuve avec la verge qui est dans ma main; et elles seront changées en sang. 
\verse Les poissons qui sont dans le fleuve périront, le fleuve se corrompra, et les Égyptiens s`efforceront en vain de boire l`eau du fleuve. 
\verse L`Éternel dit à Moïse: Dis à Aaron: Prends ta verge, et étends ta main sur les eaux des Égyptiens, sur leurs rivières, sur leurs ruisseaux, sur leurs étangs, et sur tous leurs amas d`eaux. Elles deviendront du sang: et il y aura du sang dans tout le pays d`Égypte, dans les vases de bois et dans les vases de pierre. 
\verse Moïse et Aaron firent ce que l`Éternel avait ordonné. Aaron leva la verge, et il frappa les eaux qui étaient dans le fleuve, sous les yeux de Pharaon et sous les yeux de ses serviteurs; et toutes les eaux du fleuve furent changées en sang. 
\verse Les poissons qui étaient dans le fleuve périrent, le fleuve se corrompit, les Égyptiens ne pouvaient plus boire l`eau du fleuve, et il y eut du sang dans tout le pays d`Égypte. 
\verse Mais les magiciens d`Égypte en firent autant par leurs enchantements. Le coeur de Pharaon s`endurcit, et il n`écouta point Moïse et Aaron, selon ce que l`Éternel avait dit. 
\verse Pharaon s`en retourna, et alla dans sa maison; et il ne prit pas même à coeur ces choses. 
\verse Tous les Égyptiens creusèrent aux environs du fleuve, pour trouver de l`eau à boire; car ils ne pouvaient boire de l`eau du fleuve. 
\verse Il s`écoula sept jours, après que l`Éternel eut frappé le fleuve. 

\chapter
\verse (7:26) L`Éternel dit à Moïse: Va vers Pharaon, et tu lui diras: Ainsi parle l`Éternel: Laisse aller mon peuple, afin qu`il me serve. 
\verse (7:27) Si tu refuses de le laisser aller, je vais frapper par des grenouilles toute l`étendue de ton pays. 
\verse (7:28) Le fleuve fourmillera de grenouilles; elles monteront, et elles entreront dans ta maison, dans ta chambre à coucher et dans ton lit, dans la maison de tes serviteurs et dans celles de ton peuple, dans tes fours et dans tes pétrins. 
\verse (7:29) Les grenouilles monteront sur toi, sur ton peuple, et sur tous tes serviteurs. 
\verse (8:1) L`Éternel dit à Moïse: Dis à Aaron: Étends ta main avec ta verge sur les rivières, sur les ruisseaux et sur les étangs, et fais monter les grenouilles sur le pays d`Égypte. 
\verse (8:2) Aaron étendit sa main sur les eaux de l`Égypte; et les grenouilles montèrent et couvrirent le pays d`Égypte. 
\verse (8:3) Mais les magiciens en firent autant par leurs enchantements. Ils firent monter les grenouilles sur le pays d`Égypte. 
\verse (8:4) Pharaon appela Moïse et Aaron, et dit: Priez l`Éternel, afin qu`il éloigne les grenouilles de moi et de mon peuple; et je laisserai aller le peuple, pour qu`il offre des sacrifices à l`Éternel. 
\verse (8:5) Moïse dit à Pharaon: Glorifie-toi sur moi! Pour quand prierai-je l`Éternel en ta faveur, en faveur de tes serviteurs et de ton peuple, afin qu`il retire les grenouilles loin de toi et de tes maisons? Il n`en restera que dans le fleuve. 
\verse (8:6) Il répondit: Pour demain. Et Moïse dit: Il en sera ainsi, afin que tu saches que nul n`est semblable à l`Éternel, notre Dieu. 
\verse (8:7) Les grenouilles s`éloigneront de toi et de tes maisons, de tes serviteurs et de ton peuple; il n`en restera que dans le fleuve. 
\verse (8:8) Moïse et Aaron sortirent de chez Pharaon. Et Moïse cria à l`Éternel au sujet des grenouilles dont il avait frappé Pharaon. 
\verse (8:9) L`Éternel fit ce que demandait Moïse; et les grenouilles périrent dans les maisons, dans les cours et dans les champs. 
\verse (8:10) On les entassa par monceaux, et le pays fut infecté. 
\verse (8:11) Pharaon, voyant qu`il y avait du relâche, endurcit son coeur, et il n`écouta point Moïse et Aaron, selon ce que l`Éternel avait dit. 
\verse (8:12) L`Éternel dit à Moïse: Dis à Aaron: Étends ta verge, et frappe la poussière de la terre. Elle se changera en poux, dans tout le pays d`Égypte. 
\verse (8:13) Ils firent ainsi. Aaron étendit sa main, avec sa verge, et il frappa la poussière de la terre; et elle fut changée en poux sur les hommes et sur les animaux. Toute la poussière de la terre fut changée en poux, dans tout le pays d`Égypte. 
\verse (8:14) Les magiciens employèrent leurs enchantements pour produire les poux; mais ils ne purent pas. Les poux étaient sur les hommes et sur les animaux. 
\verse (8:15) Et les magiciens dirent à Pharaon: C`est le doigt de Dieu! Le coeur de Pharaon s`endurcit, et il n`écouta point Moïse et Aaron, selon ce que l`Éternel avait dit. 
\verse (8:16) L`Éternel dit à Moïse: Lève-toi de bon matin, et présente-toi devant Pharaon; il sortira pour aller près de l`eau. Tu lui diras: Ainsi parle l`Éternel: Laisse aller mon peuple, afin qu`il me serve. 
\verse (8:17) Si tu ne laisses pas aller mon peuple, je vais envoyer les mouches venimeuses contre toi, contre tes serviteurs, contre ton peuple et contre tes maisons; les maisons des Égyptiens seront remplies de mouches, et le sol en sera couvert. 
\verse (8:18) Mais, en ce jour-là, je distinguerai le pays de Gosen où habite mon peuple, et là il n`y aura point de mouches, afin que tu saches que moi, l`Éternel, je suis au milieu de ce pays. 
\verse (8:19) J`établirai une distinction entre mon peuple et ton peuple. Ce signe sera pour demain. 
\verse (8:20) L`Éternel fit ainsi. Il vint une quantité de mouches venimeuses dans la maison de Pharaon et de ses serviteurs, et tout le pays d`Égypte fut dévasté par les mouches. 
\verse (8:21) Pharaon appela Moïse et Aaron et dit: Allez, offrez des sacrifices à votre Dieu dans le pays. 
\verse (8:22) Moïse répondit: Il n`est point convenable de faire ainsi; car nous offririons à l`Éternel, notre Dieu, des sacrifices qui sont en abomination aux Égyptiens. Et si nous offrons, sous leurs yeux, des sacrifices qui sont en abomination aux Égyptiens, ne nous lapideront-ils pas? 
\verse (8:23) Nous ferons trois journées de marche dans le désert, et nous offrirons des sacrifices à l`Éternel, notre Dieu, selon ce qu`il nous dira. 
\verse (8:24) Pharaon dit: Je vous laisserai aller, pour offrir à l`Éternel, votre Dieu, des sacrifices dans le désert: seulement, vous ne vous éloignerez pas, en y allant. Priez pour moi. 
\verse (8:25) Moïse répondit: Je vais sortir de chez toi, et je prierai l`Éternel. Demain, les mouches s`éloigneront de Pharaon, de ses serviteurs et de son peuple. Mais, que Pharaon ne trompe plus, en refusant de laisser aller le peuple, pour offrir des sacrifices à l`Éternel. 
\verse (8:26) Moïse sortit de chez Pharaon, et il pria l`Éternel. 
\verse (8:27) L`Éternel fit ce que demandait Moïse; et les mouches s`éloignèrent de Pharaon, de ses serviteurs et de son peuple. Il n`en resta pas une. 
\verse (8:28) Mais Pharaon, cette fois encore, endurcit son coeur, et il ne laissa point aller le peuple. 

\chapter
\verse L`Éternel dit à Moïse: Va vers Pharaon, et tu lui diras: Ainsi parle l`Éternel, le Dieu des Hébreux: Laisse aller mon peuple, afin qu`il me serve. 
\verse Si tu refuses de le laisser aller, et si tu le retiens encore, 
\verse voici, la main de l`Éternel sera sur tes troupeaux qui sont dans les champs, sur les chevaux, sur les ânes, sur les chameaux, sur les boeufs et sur les brebis; il y aura une mortalité très grande. 
\verse L`Éternel distinguera entre les troupeaux d`Israël et les troupeaux des Égyptiens, et il ne périra rien de tout ce qui est aux enfants d`Israël. 
\verse L`Éternel fixa le temps, et dit: Demain, l`Éternel fera cela dans le pays. 
\verse Et l`Éternel fit ainsi, dès le lendemain. Tous les troupeaux des Égyptiens périrent, et il ne périt pas une bête des troupeaux des enfants d`Israël. 
\verse Pharaon s`informa de ce qui était arrivé; et voici, pas une bête des troupeaux d`Israël n`avait péri. Mais le coeur de Pharaon s`endurcit, et il ne laissa point aller le peuple. 
\verse L`Éternel dit à Moïse et à Aaron: Remplissez vos mains de cendre de fournaise, et que Moïse la jette vers le ciel, sous les yeux de Pharaon. 
\verse Elle deviendra une poussière qui couvrira tout le pays d`Égypte; et elle produira, dans tout le pays d`Égypte, sur les hommes et sur les animaux, des ulcères formés par une éruption de pustules. 
\verse Ils prirent de la cendre de fournaise, et se présentèrent devant Pharaon; Moïse la jeta vers le ciel, et elle produisit sur les hommes et sur les animaux des ulcères formés par une éruption de pustules. 
\verse Les magiciens ne purent paraître devant Moïse, à cause des ulcères; car les ulcères étaient sur les magiciens, comme sur tous les Égyptiens. 
\verse L`Éternel endurcit le coeur de Pharaon, et Pharaon n`écouta point Moïse et Aaron, selon ce que l`Éternel avait dit à Moïse. 
\verse L`Éternel dit à Moïse: Lève-toi de bon matin, et présente-toi devant Pharaon. Tu lui diras: Ainsi parle l`Éternel, le Dieu des Hébreux: Laisse aller mon peuple, afin qu`il me serve. 
\verse Car, cette fois, je vais envoyer toutes mes plaies contre ton coeur, contre tes serviteurs et contre ton peuple, afin que tu saches que nul n`est semblable à moi sur toute la terre. 
\verse Si j`avais étendu ma main, et que je t`eusse frappé par la mortalité, toi et ton peuple, tu aurais disparu de la terre. 
\verse Mais, je t`ai laissé subsister, afin que tu voies ma puissance, et que l`on publie mon nom par toute la terre. 
\verse Si tu t`élèves encore contre mon peuple, et si tu ne le laisses point aller, 
\verse voici, je ferai pleuvoir demain, à cette heure, une grêle tellement forte, qu`il n`y en a point eu de semblable en Égypte depuis le jour où elle a été fondée jusqu`à présent. 
\verse Fais donc mettre en sûreté tes troupeaux et tout ce qui est à toi dans les champs. La grêle tombera sur tous les hommes et sur tous les animaux qui se trouveront dans les champs et qui n`auront pas été recueillis dans les maisons, et ils périront. 
\verse Ceux des serviteurs de Pharaon qui craignirent la parole de l`Éternel firent retirer dans les maisons leurs serviteurs et leurs troupeaux. 
\verse Mais ceux qui ne prirent point à coeur la parole de l`Éternel laissèrent leurs serviteurs et leurs troupeaux dans les champs. 
\verse L`Éternel dit à Moïse: Étends ta main vers le ciel; et qu`il tombe de la grêle dans tout le pays d`Égypte sur les hommes, sur les animaux, et sur toutes les herbes des champs, dans le pays d`Égypte. 
\verse Moïse étendit sa verge vers le ciel; et l`Éternel envoya des tonnerres et de la grêle, et le feu se promenait sur la terre. L`Éternel fit pleuvoir de la grêle sur le pays d`Égypte. 
\verse Il tomba de la grêle, et le feu se mêlait avec la grêle; elle était tellement forte qu`il n`y en avait point eu de semblable dans tout le pays d`Égypte depuis qu`il existe comme nation. 
\verse La grêle frappa, dans tout le pays d`Égypte, tout ce qui était dans les champs, depuis les hommes jusqu`aux animaux; la grêle frappa aussi toutes les herbes des champs, et brisa tous les arbres des champs. 
\verse Ce fut seulement dans le pays de Gosen, où étaient les enfants d`Israël, qu`il n`y eut point de grêle. 
\verse Pharaon fit appeler Moïse et Aaron, et leur dit: Cette fois, j`ai péché; c`est l`Éternel qui est le juste, et moi et mon peuple nous sommes les coupables. 
\verse Priez l`Éternel, pour qu`il n`y ait plus de tonnerres ni de grêle; et je vous laisserai aller, et l`on ne vous retiendra plus. 
\verse Moïse lui dit: Quand je sortirai de la ville, je lèverai mes mains vers l`Éternel, les tonnerres cesseront et il n`y aura plus de grêle, afin que tu saches que la terre est à l`Éternel. 
\verse Mais je sais que toi et tes serviteurs, vous ne craindrez pas encore l`Éternel Dieu. 
\verse Le lin et l`orge avaient été frappés, parce que l`orge était en épis et que c`était la floraison du lin; 
\verse le froment et l`épeautre n`avaient point été frappés, parce qu`ils sont tardifs. 
\verse Moïse sortit de chez Pharaon, pour aller hors de la ville; il leva ses mains vers l`Éternel, les tonnerres et la grêle cessèrent, et la pluie ne tomba plus sur la terre. 
\verse Pharaon, voyant que la pluie, la grêle et les tonnerres avaient cessé, continua de pécher, et il endurcit son coeur, lui et ses serviteurs. 
\verse Le coeur de Pharaon s`endurcit, et il ne laissa point aller les enfants d`Israël, selon ce que l`Éternel avait dit par l`intermédiaire de Moïse. 

\chapter
\verse L`Éternel dit à Moïse: Va vers Pharaon, car j`ai endurci son coeur et le coeur de ses serviteurs, pour faire éclater mes signes au milieu d`eux. 
\verse C`est aussi pour que tu racontes à ton fils et au fils de ton fils comment j`ai traité les Égyptiens, et quels signes j`ai fait éclater au milieu d`eux. Et vous saurez que je suis l`Éternel. 
\verse Moïse et Aaron allèrent vers Pharaon, et lui dirent: Ainsi parle l`Éternel, le Dieu des Hébreux: Jusqu`à quand refuseras-tu de t`humilier devant moi? Laisse aller mon peuple, afin qu`il me serve. 
\verse Si tu refuses de laisser aller mon peuple, voici, je ferai venir demain des sauterelles dans toute l`étendue de ton pays. 
\verse Elles couvriront la surface de la terre, et l`on ne pourra plus voir la terre; elles dévoreront le reste de ce qui est échappé, ce que vous a laissé la grêle, elles dévoreront tous les arbres qui croissent dans vos champs; 
\verse elles rempliront tes maisons, les maisons de tous tes serviteurs et les maisons de tous les Égyptiens. Tes pères et les pères de tes pères n`auront rien vu de pareil depuis qu`ils existent sur la terre jusqu`à ce jour. Moïse se retira, et sortit de chez Pharaon. 
\verse Les serviteurs de Pharaon lui dirent: Jusqu`à quand cet homme sera-t-il pour nous un piège? Laisse aller ces gens, et qu`ils servent l`Éternel, leur Dieu. Ne vois-tu pas encore que l`Égypte périt? 
\verse On fit revenir vers Pharaon Moïse et Aaron: Allez, leur dit-il, servez l`Éternel, votre Dieu. Qui sont ceux qui iront? 
\verse Moïse répondit: Nous irons avec nos enfants et nos vieillards, avec nos fils et nos filles, avec nos brebis et nos boeufs; car c`est pour nous une fête en l`honneur de l`Éternel. 
\verse Pharaon leur dit: Que l`Éternel soit avec vous, tout comme je vais vous laisser aller, vous et vos enfants! Prenez garde, car le malheur est devant vous! 
\verse Non, non: allez, vous les hommes, et servez l`Éternel, car c`est là ce que vous avez demandé. Et on les chassa de la présence de Pharaon. 
\verse L`Éternel dit à Moïse: Étends ta main sur le pays d`Égypte, et que les sauterelles montent sur le pays d`Égypte; qu`elles dévorent toute l`herbe de la terre, tout ce que la grêle a laissé. 
\verse Moïse étendit sa verge sur le pays d`Égypte; et l`Éternel fit souffler un vent d`orient sur le pays toute cette journée et toute la nuit. Quand ce fut le matin, le vent d`orient avait apporté les sauterelles. 
\verse Les sauterelles montèrent sur le pays d`Égypte, et se posèrent dans toute l`étendue de l`Égypte; elles étaient en si grande quantité qu`il n`y avait jamais eu et qu`il n`y aura jamais rien de semblable. 
\verse Elles couvrirent la surface de toute la terre, et la terre fut dans l`obscurité; elles dévorèrent toute l`herbe de la terre et tout le fruit des arbres, tout ce que la grêle avait laissé; et il ne resta aucune verdure aux arbres ni à l`herbe des champs, dans tout le pays d`Égypte. 
\verse Aussitôt Pharaon appela Moïse et Aaron, et dit: J`ai péché contre l`Éternel, votre Dieu, et contre vous. 
\verse Mais pardonne mon péché pour cette fois seulement; et priez l`Éternel, votre Dieu, afin qu`il éloigne de moi encore cette plaie mortelle. 
\verse Moïse sortit de chez Pharaon, et il pria l`Éternel. 
\verse L`Éternel fit souffler un vent d`occident très fort, qui emporta les sauterelles, et les précipita dans la mer Rouge; il ne resta pas une seule sauterelle dans toute l`étendue de l`Égypte. 
\verse L`Éternel endurcit le coeur de Pharaon, et Pharaon ne laissa point aller les enfants d`Israël. 
\verse L`Éternel dit à Moïse: Étends ta main vers le ciel, et qu`il y ait des ténèbres sur le pays d`Égypte, et que l`on puisse les toucher. 
\verse Moïse étendit sa main vers le ciel; et il y eut d`épaisses ténèbres dans tout le pays d`Égypte, pendant trois jours. 
\verse On ne se voyait pas les uns les autres, et personne ne se leva de sa place pendant trois jours. Mais il y avait de la lumière dans les lieux où habitaient tous les enfants d`Israël. 
\verse Pharaon appela Moïse, et dit: Allez, servez l`Éternel. Il n`y aura que vos brebis et vos boeufs qui resteront, et vos enfants pourront aller avec vous. 
\verse Moïse répondit: Tu mettras toi-même entre nos mains de quoi faire les sacrifices et les holocaustes que nous offrirons à l`Éternel, notre Dieu. 
\verse Nos troupeaux iront avec nous, et il ne restera pas un ongle; car c`est là que nous prendrons pour servir l`Éternel, notre Dieu; et jusqu`à ce que nous soyons arrivés, nous ne savons pas ce que nous choisirons pour offrir à l`Éternel. 
\verse L`Éternel endurcit le coeur de Pharaon, et Pharaon ne voulut point les laisser aller. 
\verse Pharaon dit à Moïse: Sors de chez moi! Garde-toi de paraître encore en ma présence, car le jour où tu paraîtras en ma présence, tu mourras. 
\verse Tu l`as dit! répliqua Moïse, je ne paraîtrai plus en ta présence. 

\chapter
\verse L`Éternel dit à Moïse: Je ferai venir encore une plaie sur Pharaon et sur l`Égypte. Après cela, il vous laissera partir d`ici. Lorsqu`il vous laissera tout à fait aller, il vous chassera même d`ici. 
\verse Parle au peuple, pour que chacun demande à son voisin et chacune à sa voisine des vases d`argent et des vases d`or. 
\verse L`Éternel fit trouver grâce au peuple aux yeux des Égyptiens; Moïse lui-même était très considéré dans le pays d`Égypte, aux yeux des serviteurs de Pharaon et aux yeux du peuple. 
\verse Moïse dit: Ainsi parle l`Éternel: Vers le milieu de la nuit, je passerai au travers de l`Égypte; 
\verse et tous les premiers-nés mourront dans le pays d`Égypte, depuis le premier-né de Pharaon assis sur son trône, jusqu`au premier-né de la servante qui est derrière la meule, et jusqu`à tous les premiers-nés des animaux. 
\verse Il y aura dans tout le pays d`Égypte de grands cris, tels qu`il n`y en a point eu et qu`il n`y en aura plus de semblables. 
\verse Mais parmi tous les enfants d`Israël, depuis les hommes jusqu`aux animaux, pas même un chien ne remuera sa langue, afin que vous sachiez quelle différence l`Éternel fait entre l`Égypte et Israël. 
\verse Alors tous tes serviteurs que voici descendront vers moi et se prosterneront devant moi, en disant: Sors, toi et tout le peuple qui s`attache à tes pas! Après cela, je sortirai. Moïse sortit de chez Pharaon, dans une ardente colère. 
\verse L`Éternel dit à Moïse: Pharaon ne vous écoutera point, afin que mes miracles se multiplient dans le pays d`Égypte. 
\verse Moïse et Aaron firent tous ces miracles devant Pharaon, et ne Pharaon ne laissa point aller les enfants d`Israël hors de son pays. 

\chapter
\verse L`Éternel dit à Moïse et à Aaron dans le pays d`Égypte: 
\verse Ce mois-ci sera pour vous le premier des mois; il sera pour vous le premier des mois de l`année. 
\verse Parlez à toute l`assemblée d`Israël, et dites: Le dixième jour de ce mois, on prendra un agneau pour chaque famille, un agneau pour chaque maison. 
\verse Si la maison est trop peu nombreuse pour un agneau, on le prendra avec son plus proche voisin, selon le nombre des personnes; vous compterez pour cet agneau d`après ce que chacun peut manger. 
\verse Ce sera un agneau sans défaut, mâle, âgé d`un an; vous pourrez prendre un agneau ou un chevreau. 
\verse Vous le garderez jusqu`au quatorzième jour de ce mois; et toute l`assemblée d`Israël l`immolera entre les deux soirs. 
\verse On prendra de son sang, et on en mettra sur les deux poteaux et sur le linteau de la porte des maisons où on le mangera. 
\verse Cette même nuit, on en mangera la chair, rôtie au feu; on la mangera avec des pains sans levain et des herbes amères. 
\verse Vous ne le mangerez point à demi cuit et bouilli dans l`eau; mais il sera rôti au feu, avec la tête, les jambes et l`intérieur. 
\verse Vous n`en laisserez rien jusqu`au matin; et, s`il en reste quelque chose le matin, vous le brûlerez au feu. 
\verse Quand vous le mangerez, vous aurez vos reins ceints, vos souliers aux pieds, et votre bâton à la main; et vous le mangerez à la hâte. C`est la Pâque de l`Éternel. 
\verse Cette nuit-là, je passerai dans le pays d`Égypte, et je frapperai tous les premiers-nés du pays d`Égypte, depuis les hommes jusqu`aux animaux, et j`exercerai des jugements contre tous les dieux de l`Égypte. Je suis l`Éternel. 
\verse Le sang vous servira de signe sur les maisons où vous serez; je verrai le sang, et je passerai par-dessus vous, et il n`y aura point de plaie qui vous détruise, quand je frapperai le pays d`Égypte. 
\verse Vous conserverez le souvenir de ce jour, et vous le célébrerez par une fête en l`honneur de l`Éternel; vous le célébrerez comme une loi perpétuelle pour vos descendants. 
\verse Pendant sept jours, vous mangerez des pains sans levain. Dès le premier jour, il n`y aura plus de levain dans vos maisons; car toute personne qui mangera du pain levé, du premier jour au septième jour, sera retranchée d`Israël. 
\verse Le premier jour, vous aurez une sainte convocation; et le septième jour, vous aurez une sainte convocation. On ne fera aucun travail ces jours-là; vous pourrez seulement préparer la nourriture de chaque personne. 
\verse Vous observerez la fête des pains sans levain, car c`est en ce jour même que j`aurai fait sortir vos armées du pays d`Égypte; vous observerez ce jour comme une loi perpétuelle pour vos descendants. 
\verse Le premier mois, le quatorzième jour du mois, au soir, vous mangerez des pains sans levain jusqu`au soir du vingt et unième jour. 
\verse Pendant sept jours, il ne se trouvera point de levain dans vos maisons; car toute personne qui mangera du pain levé sera retranchée de l`assemblée d`Israël, que ce soit un étranger ou un indigène. 
\verse Vous ne mangerez point de pain levé; dans toutes vos demeures, vous mangerez des pains sans levain. 
\verse Moïse appela tous les anciens d`Israël, et leur dit: Allez prendre du bétail pour vos familles, et immolez la Pâque. 
\verse Vous prendrez ensuite un bouquet d`hysope, vous le tremperez dans le sang qui sera dans le bassin, et vous toucherez le linteau et les deux poteaux de la porte avec le sang qui sera dans le bassin. Nul de vous ne sortira de sa maison jusqu`au matin. 
\verse Quand l`Éternel passera pour frapper l`Égypte, et verra le sang sur le linteau et sur les deux poteaux, l`Éternel passera par-dessus la porte, et il ne permettra pas au destructeur d`entrer dans vos maisons pour frapper. 
\verse Vous observerez cela comme une loi pour vous et pour vos enfants à perpétuité. 
\verse Quand vous serez entrés dans le pays que l`Éternel vous donnera, selon sa promesse, vous observerez cet usage sacré. 
\verse Et lorsque vos enfants vous diront: Que signifie pour vous cet usage? 
\verse vous répondrez: C`est le sacrifice de Pâque en l`honneur de l`Éternel, qui a passé par-dessus les maisons des enfants d`Israël en Égypte, lorsqu`il frappa l`Égypte et qu`il sauva nos maisons. Le peuple s`inclina et se prosterna. 
\verse Et les enfants d`Israël s`en allèrent, et firent ce que l`Éternel avait ordonné à Moïse et à Aaron; ils firent ainsi. 
\verse Au milieu de la nuit, l`Éternel frappa tous les premiers-nés dans le pays d`Égypte, depuis le premier-né de Pharaon assis sur son trône, jusqu`au premier-né du captif dans sa prison, et jusqu`à tous les premiers-nés des animaux. 
\verse Pharaon se leva de nuit, lui et tous ses serviteurs, et tous les Égyptiens; et il y eut de grands cris en Égypte, car il n`y avait point de maison où il n`y eût un mort. 
\verse Dans la nuit même, Pharaon appela Moïse et Aaron, et leur dit: Levez-vous, sortez du milieu de mon peuple, vous et les enfants d`Israël. Allez, servez l`Éternel, comme vous l`avez dit. 
\verse Prenez vos brebis et vos boeufs, comme vous l`avez dit; allez, et bénissez-moi. 
\verse Les Égyptiens pressaient le peuple, et avaient hâte de le renvoyer du pays, car ils disaient: Nous périrons tous. 
\verse Le peuple emporta sa pâte avant qu`elle fût levée. Ils enveloppèrent les pétrins dans leurs vêtements, et les mirent sur leurs épaules. 
\verse Les enfants d`Israël firent ce que Moïse avait dit, et ils demandèrent aux Égyptiens des vases d`argent, des vases d`or et des vêtements. 
\verse L`Éternel fit trouver grâce au peuple aux yeux des Égyptiens, qui se rendirent à leur demande. Et ils dépouillèrent les Égyptiens. 
\verse Les enfants d`Israël partirent de Ramsès pour Succoth au nombre d`environ six cent mille hommes de pied, sans les enfants. 
\verse Une multitude de gens de toute espèce montèrent avec eux; ils avaient aussi des troupeaux considérables de brebis et de boeufs. 
\verse Ils firent des gâteaux cuits sans levain avec la pâte qu`ils avaient emportée d`Égypte, et qui n`était pas levée; car ils avaient été chassés d`Égypte, sans pouvoir tarder, et sans prendre des provisions avec eux. 
\verse Le séjour des enfants d`Israël en Égypte fut de quatre cent trente ans. 
\verse Et au bout de quatre cent trente ans, le jour même, toutes les armées de l`Éternel sortirent du pays d`Égypte. 
\verse Cette nuit sera célébrée en l`honneur de l`Éternel, parce qu`il les fit sortir du pays d`Égypte; cette nuit sera célébrée en l`honneur de l`Éternel par tous les enfants d`Israël et par leurs descendants. 
\verse L`Éternel dit à Moïse et à Aaron: Voici une ordonnance au sujet de la Pâque: Aucun étranger n`en mangera. 
\verse Tu circonciras tout esclave acquis à prix d`argent; alors il en mangera. 
\verse L`habitant et le mercenaire n`en mangeront point. 
\verse On ne la mangera que dans la maison; vous n`emporterez point de chair hors de la maison, et vous ne briserez aucun os. 
\verse Toute l`assemblée d`Israël fera la Pâque. 
\verse Si un étranger en séjour chez toi veut faire la Pâque de l`Éternel, tout mâle de sa maison devra être circoncis; alors il s`approchera pour la faire, et il sera comme l`indigène; mais aucun incirconcis n`en mangera. 
\verse La même loi existera pour l`indigène comme pour l`étranger en séjour au milieu de vous. 
\verse Tous les enfants d`Israël firent ce que l`Éternel avait ordonné à Moïse et à Aaron; ils firent ainsi. 
\verse Et ce même jour l`Éternel fit sortir du pays d`Égypte les enfants d`Israël, selon leurs armées. 

\chapter
\verse L`Éternel parla à Moïse, et dit: 
\verse Consacre-moi tout premier-né, tout premier-né parmi les enfants d`Israël, tant des hommes que des animaux: il m`appartient. 
\verse Moïse dit au peuple: Souvenez-vous de ce jour, où vous êtes sortis d`Égypte, de la maison de servitude; car c`est par sa main puissante que l`Éternel vous en a fait sortir. On ne mangera point de pain levé. 
\verse Vous sortez aujourd`hui, dans le mois des épis. 
\verse Quand l`Éternel t`aura fait entrer dans le pays des Cananéens, des Héthiens, des Amoréens, des Héviens et des Jébusiens, qu`il a juré à tes pères de te donner, pays où coulent le lait et le miel, tu rendras ce culte à l`Éternel dans ce même mois. 
\verse Pendant sept jours, tu mangeras des pains sans levain; et le septième jour, il y aura une fête en l`honneur de l`Éternel. 
\verse On mangera des pains sans levain pendant les sept jours; on ne verra point chez toi de pain levé, et l`on ne verra point chez toi de levain, dans toute l`étendue de ton pays. 
\verse Tu diras alors à ton fils: C`est en mémoire de ce que l`Éternel a fait pour moi, lorsque je suis sorti d`Égypte. 
\verse Ce sera pour toi comme un signe sur ta main et comme un souvenir entre tes yeux, afin que la loi de l`Éternel soit dans ta bouche; car c`est par sa main puissante que l`Éternel t`a fait sortir d`Égypte. 
\verse Tu observeras cette ordonnance au temps fixé d`année en année. 
\verse Quand l`Éternel t`aura fait entrer dans le pays des Cananéens, comme il l`a juré à toi et à tes pères, et qu`il te l`aura donné, 
\verse tu consacreras à l`Éternel tout premier-né, même tout premier-né des animaux que tu auras: les mâles appartiennent à l`Éternel. 
\verse Tu rachèteras avec un agneau tout premier-né de l`âne; et, si tu ne le rachètes pas, tu lui briseras la nuque. Tu rachèteras aussi tout premier-né de l`homme parmi tes fils. 
\verse Et lorsque ton fils te demandera un jour: Que signifie cela? tu lui répondras: Par sa main puissante, l`Éternel nous a fait sortir d`Égypte, de la maison de servitude; 
\verse et, comme Pharaon s`obstinait à ne point nous laisser aller, l`Éternel fit mourir tous les premiers-nés dans le pays d`Égypte, depuis les premiers-nés des hommes jusqu`aux premiers-nés des animaux. Voilà pourquoi j`offre en sacrifice à l`Éternel tout premier-né des mâles, et je rachète tout premier-né de mes fils. 
\verse Ce sera comme un signe sur ta main et comme des fronteaux entre tes yeux; car c`est par sa main puissante que l`Éternel nous a fait sortir d`Égypte. 
\verse Lorsque Pharaon laissa aller le peuple, Dieu ne le conduisit point par le chemin du pays des Philistins, quoique le plus proche; car Dieu dit: Le peuple pourrait se repentir en voyant la guerre, et retourner en Égypte. 
\verse Mais Dieu fit faire au peuple un détour par le chemin du désert, vers la mer Rouge. Les enfants d`Israël montèrent en armes hors du pays d`Égypte. 
\verse Moïse prit avec lui les os de Joseph; car Joseph avait fait jurer les fils d`Israël, en disant: Dieu vous visitera, et vous ferez remonter avec vous mes os loin d`ici. 
\verse Ils partirent de Succoth, et ils campèrent à Étham, à l`extrémité du désert. 
\verse L`Éternel allait devant eux, le jour dans une colonne de nuée pour les guider dans leur chemin, et la nuit dans une colonne de feu pour les éclairer, afin qu`ils marchassent jour et nuit. 
\verse La colonne de nuée ne se retirait point de devant le peuple pendant le jour, ni la colonne de feu pendant la nuit. 

\chapter
\verse L`Éternel parla à Moïse, et dit: 
\verse Parle aux enfants d`Israël; qu`ils se détournent, et qu`ils campent devant Pi Hahiroth, entre Migdol et la mer, vis-à-vis de Baal Tsephon; c`est en face de ce lieu que vous camperez, près de la mer. 
\verse Pharaon dira des enfants d`Israël: Ils sont égarés dans le pays; le désert les enferme. 
\verse J`endurcirai le coeur de Pharaon, et il les poursuivra; mais Pharaon et toute son armée serviront à faire éclater ma gloire, et les Égyptiens sauront que je suis l`Éternel. Et les enfants d`Israël firent ainsi. 
\verse On annonça au roi d`Égypte que le peuple avait pris la fuite. Alors le coeur de Pharaon et celui de ses serviteurs furent changés à l`égard du peuple. Ils dirent: Qu`avons-nous fait, en laissant aller Israël, dont nous n`aurons plus les services? 
\verse Et Pharaon attela son char, et il prit son peuple avec lui. 
\verse Il prit six cent chars d`élite, et tous les chars de l`Égypte; il y avait sur tous des combattants. 
\verse L`Éternel endurcit le coeur de Pharaon, roi d`Égypte, et Pharaon poursuivit les enfants d`Israël. Les enfants d`Israël étaient sortis la main levée. 
\verse Les Égyptiens les poursuivirent; et tous les chevaux, les chars de Pharaon, ses cavaliers et son armée, les atteignirent campés près de la mer, vers Pi Hahiroth, vis-à-vis de Baal Tsephon. 
\verse Pharaon approchait. Les enfants d`Israël levèrent les yeux, et voici, les Égyptiens étaient en marche derrière eux. Et les enfants d`Israël eurent une grande frayeur, et crièrent à l`Éternel. 
\verse Ils dirent à Moïse: N`y avait-il pas des sépulcres en Égypte, sans qu`il fût besoin de nous mener mourir au désert? Que nous as-tu fait en nous faisant sortir d`Égypte? 
\verse N`est-ce pas là ce que nous te disions en Égypte: Laisse-nous servir les Égyptiens, car nous aimons mieux servir les Égyptiens que de mourir au désert? 
\verse Moïse répondit au peuple: Ne craignez rien, restez en place, et regardez la délivrance que l`Éternel va vous accorder en ce jour; car les Égyptiens que vous voyez aujourd`hui, vous ne les verrez plus jamais. 
\verse L`Éternel combattra pour vous; et vous, gardez le silence. 
\verse L`Éternel dit à Moïse: Pourquoi ces cris? Parle aux enfants d`Israël, et qu`ils marchent. 
\verse Toi, lève ta verge, étends ta main sur la mer, et fends-la; et les enfants d`Israël entreront au milieu de la mer à sec. 
\verse Et moi, je vais endurcir le coeur des Égyptiens, pour qu`ils y entrent après eux: et Pharaon et toute son armée, ses chars et ses cavaliers, feront éclater ma gloire. 
\verse Et les Égyptiens sauront que je suis l`Éternel, quand Pharaon, ses chars et ses cavaliers, auront fait éclater ma gloire. 
\verse L`ange de Dieu, qui allait devant le camp d`Israël, partit et alla derrière eux; et la colonne de nuée qui les précédait, partit et se tint derrière eux. 
\verse Elle se plaça entre le camp des Égyptiens et le camp d`Israël. Cette nuée était ténébreuse d`un côté, et de l`autre elle éclairait la nuit. Et les deux camps n`approchèrent point l`un de l`autre pendant toute la nuit. 
\verse Moïse étendit sa main sur la mer. Et l`Éternel refoula la mer par un vent d`orient, qui souffla avec impétuosité toute la nuit; il mit la mer à sec, et les eaux se fendirent. 
\verse Les enfants d`Israël entrèrent au milieu de la mer à sec, et les eaux formaient comme une muraille à leur droite et à leur gauche. 
\verse Les Égyptiens les poursuivirent; et tous les chevaux de Pharaon, ses chars et ses cavaliers, entrèrent après eux au milieu de la mer. 
\verse A la veille du matin, l`Éternel, de la colonne de feu et de nuée, regarda le camp des Égyptiens, et mit en désordre le camp des Égyptiens. 
\verse Il ôta les roues de leurs chars et en rendit la marche difficile. Les Égyptiens dirent alors: Fuyons devant Israël, car l`Éternel combat pour lui contre les Égyptiens. 
\verse L`Éternel dit à Moïse: Étends ta main sur la mer; et les eaux reviendront sur les Égyptiens, sur leurs chars et sur leurs cavaliers. 
\verse Moïse étendit sa main sur la mer. Et vers le matin, la mer reprit son impétuosité, et les Égyptiens s`enfuirent à son approche; mais l`Éternel précipita les Égyptiens au milieu de la mer. 
\verse Les eaux revinrent, et couvrirent les chars, les cavaliers et toute l`armée de Pharaon, qui étaient entrés dans la mer après les enfants d`Israël; et il n`en échappa pas un seul. 
\verse Mais les enfants d`Israël marchèrent à sec au milieu de la mer, et les eaux formaient comme une muraille à leur droite et à leur gauche. 
\verse En ce jour, l`Éternel délivra Israël de la main des Égyptiens; et Israël vit sur le rivage de la mer les Égyptiens qui étaient morts. 
\verse Israël vit la main puissante que l`Éternel avait dirigée contre les Égyptiens. Et le peuple craignit l`Éternel, et il crut en l`Éternel et en Moïse, son serviteur. 

\chapter
\verse Alors Moïse et les enfants d`Israël chantèrent ce cantique à l`Éternel. Ils dirent: Je chanterai à l`Éternel, car il a fait éclater sa gloire; Il a précipité dans la mer le cheval et son cavalier. 
\verse L`Éternel est ma force et le sujet de mes louanges; C`est lui qui m`a sauvé. Il est mon Dieu: je le célébrerai; Il est le Dieu de mon père: je l`exalterai. 
\verse L`Éternel est un vaillant guerrier; L`Éternel est son nom. 
\verse Il a lancé dans la mer les chars de Pharaon et son armée; Ses combattants d`élite ont été engloutis dans la mer Rouge. 
\verse Les flots les ont couverts: Ils sont descendus au fond des eaux, comme une pierre. 
\verse Ta droite, ô Éternel! a signalé sa force; Ta droite, ô Éternel! a écrasé l`ennemi. 
\verse Par la grandeur de ta majesté Tu renverses tes adversaires; Tu déchaînes ta colère: Elle les consume comme du chaume. 
\verse Au souffle de tes narines, les eaux se sont amoncelées, Les courants se sont dressés comme une muraille, Les flots se sont durcis au milieu de la mer. 
\verse L`ennemi disait: Je poursuivrai, j`atteindrai, Je partagerai le butin; Ma vengeance sera assouvie, Je tirerai l`épée, ma main les détruira. 
\verse Tu as soufflé de ton haleine: La mer les a couverts; Ils se sont enfoncés comme du plomb, Dans la profondeur des eaux. 
\verse Qui est comme toi parmi les dieux, ô Éternel? Qui est comme toi magnifique en sainteté, Digne de louanges, Opérant des prodiges? 
\verse Tu as étendu ta droite: La terre les a engloutis. 
\verse Par ta miséricorde tu as conduit, Tu as délivré ce peuple; Par ta puissance tu le diriges Vers la demeure de ta sainteté. 
\verse Les peuples l`apprennent, et ils tremblent: La terreur s`empare des Philistins; 
\verse Les chefs d`Édom s`épouvantent; Un tremblement saisit les guerriers de Moab; Tous les habitants de Canaan tombent en défaillance. 
\verse La crainte et la frayeur les surprendront; Par la grandeur de ton bras Ils deviendront muets comme une pierre, Jusqu`à ce que ton peuple soit passé, ô Éternel! Jusqu`à ce qu`il soit passé, Le peuple que tu as acquis. 
\verse Tu les amèneras et tu les établiras sur la montagne de ton héritage, Au lieu que tu as préparé pour ta demeure, ô Éternel! Au sanctuaire, Seigneur! que tes mains ont fondé. 
\verse L`Éternel régnera éternellement et à toujours. 
\verse Car les chevaux de Pharaon, ses chars et ses cavaliers sont entrés dans la mer, Et l`Éternel a ramené sur eux les eaux de la mer; Mais les enfants d`Israël ont marché à sec au milieu de la mer. 
\verse Marie, la prophétesse, soeur d`Aaron, prit à la main un tambourin, et toutes les femmes vinrent après elle, avec des tambourins et en dansant. 
\verse Marie répondait aux enfants d`Israël: Chantez à l`Éternel, car il a fait éclater sa gloire; Il a précipité dans la mer le cheval et son cavalier. 
\verse Moïse fit partir Israël de la mer Rouge. Ils prirent la direction du désert de Schur; et, après trois journées de marche dans le désert, ils ne trouvèrent point d`eau. 
\verse Ils arrivèrent à Mara; mais ils ne purent pas boire l`eau de Mara parce qu`elle était amère. C`est pourquoi ce lieu fut appelé Mara. 
\verse Le peuple murmura contre Moïse, en disant: Que boirons-nous? 
\verse Moïse cria à l`Éternel; et l`Éternel lui indiqua un bois, qu`il jeta dans l`eau. Et l`eau devint douce. Ce fut là que l`Éternel donna au peuple des lois et des ordonnances, et ce fut là qu`il le mit à l`épreuve. 
\verse Il dit: Si tu écoutes attentivement la voix de l`Éternel, ton Dieu, si tu fais ce qui est droit à ses yeux, si tu prêtes l`oreille à ses commandements, et si tu observes toutes ses lois, je ne te frapperai d`aucune des maladies dont j`ai frappé les Égyptiens; car je suis l`Éternel, qui te guérit. 
\verse Ils arrivèrent à Élim, où il y avait douze sources d`eau et soixante-dix palmiers. Ils campèrent là, près de l`eau. 

\chapter
\verse Toute l`assemblée des enfants d`Israël partit d`Élim, et ils arrivèrent au désert de Sin, qui est entre Élim et Sinaï, le quinzième jour du second mois après leur sortie du pays d`Égypte. 
\verse Et toute l`assemblée des enfants d`Israël murmura dans le désert contre Moïse et Aaron. 
\verse Les enfants d`Israël leur dirent: Que ne sommes-nous morts par la main de l`Éternel dans le pays d`Égypte, quand nous étions assis près des pots de viande, quand nous mangions du pain à satiété? car vous nous avez menés dans ce désert pour faire mourir de faim toute cette multitude. 
\verse L`Éternel dit à Moïse: Voici, je ferai pleuvoir pour vous du pain, du haut des cieux. Le peuple sortira, et en ramassera, jour par jour, la quantité nécessaire, afin que je le mette à l`épreuve, et que je voie s`il marchera, ou non, selon ma loi. 
\verse Le sixième jour, lorsqu`ils prépareront ce qu`ils auront apporté, il s`en trouvera le double de ce qu`ils ramasseront jour par jour. 
\verse Moïse et Aaron dirent à tous les enfants d`Israël: Ce soir, vous comprendrez que c`est l`Éternel qui vous a fait sortir du pays d`Égypte. 
\verse Et, au matin, vous verrez la gloire de l`Éternel, parce qu`il a entendu vos murmures contre l`Éternel; car que sommes-nous, pour que vous murmuriez contre nous? 
\verse Moïse dit: L`Éternel vous donnera ce soir de la viande à manger, et au matin du pain à satiété, parce que l`Éternel a entendu les murmures que vous avez proférés contre lui; car que sommes-nous? Ce n`est pas contre nous que sont vos murmures, c`est contre l`Éternel. 
\verse Moïse dit à Aaron: Dis à toute l`assemblée des enfants d`Israël: Approchez-vous devant l`Éternel, car il a entendu vos murmures. 
\verse Et tandis qu`Aaron parlait à toute l`assemblée des enfants d`Israël, ils se tournèrent du côté du désert, et voici, la gloire de l`Éternel parut dans la nuée. 
\verse L`Éternel, s`adressant à Moïse, dit: 
\verse J`ai entendu les murmures des enfants d`Israël. Dis-leur: Entre les deux soirs vous mangerez de la viande, et au matin vous vous rassasierez de pain; et vous saurez que je suis l`Éternel, votre Dieu. 
\verse Le soir, il survint des cailles qui couvrirent le camp; et, au matin, il y eut une couche de rosée autour du camp. 
\verse Quand cette rosée fut dissipée, il y avait à la surface du désert quelque chose de menu comme des grains, quelque chose de menu comme la gelée blanche sur la terre. 
\verse Les enfants d`Israël regardèrent et ils se dirent l`un à l`autre: Qu`est-ce que cela? car ils ne savaient pas ce que c`était. Moïse leur dit: C`est le pain que L`Éternel vous donne pour nourriture. 
\verse Voici ce que l`Éternel a ordonné: Que chacun de vous en ramasse ce qu`il faut pour sa nourriture, un omer par tête, suivant le nombre de vos personnes; chacun en prendra pour ceux qui sont dans sa tente. 
\verse Les Israélites firent ainsi; et ils en ramassèrent les uns en plus, les autres moins. 
\verse On mesurait ensuite avec l`omer; celui qui avait ramassé plus n`avait rien de trop, et celui qui avait ramassé moins n`en manquait pas. Chacun ramassait ce qu`il fallait pour sa nourriture. 
\verse Moïse leur dit: Que personne n`en laisse jusqu`au matin. 
\verse Ils n`écoutèrent pas Moïse, et il y eut des gens qui en laissèrent jusqu`au matin; mais il s`y mit des vers, et cela devint infect. Moïse fut irrité contre ces gens. 
\verse Tous les matins, chacun ramassait ce qu`il fallait pour sa nourriture; et quand venait la chaleur du soleil, cela fondait. 
\verse Le sixième jour, ils ramassèrent une quantité double de nourriture, deux omers pour chacun. Tous les principaux de l`assemblée vinrent le rapporter à Moïse. 
\verse Et Moïse leur dit: C`est ce que l`Éternel a ordonné. Demain est le jour du repos, le sabbat consacré à l`Éternel; faites cuire ce que vous avez à faire cuire, faites bouillir ce que vous avez à faire bouillir, et mettez en réserve jusqu`au matin tout ce qui restera. 
\verse Ils le laissèrent jusqu`au matin, comme Moïse l`avait ordonné; et cela ne devint point infect, et il ne s`y mit point de vers. 
\verse Moïse dit: Mangez-le aujourd`hui, car c`est le jour du sabbat; aujourd`hui vous n`en trouverez point dans la campagne. 
\verse Pendant six jours vous en ramasserez; mais le septième jour, qui est le sabbat, il n`y en aura point. 
\verse Le septième jour, quelques-uns du peuple sortirent pour en ramasser, et ils n`en trouvèrent point. 
\verse Alors l`Éternel dit à Moïse: Jusques à quand refuserez-vous d`observer mes commandements et mes lois? 
\verse Considérez que l`Éternel vous a donné le sabbat; c`est pourquoi il vous donne au sixième jour de la nourriture pour deux jours. Que chacun reste à sa place, et que personne ne sorte du lieu où il est au septième jour. 
\verse Et le peuple se reposa le septième jour. 
\verse La maison d`Israël donna à cette nourriture le nom de manne. Elle ressemblait à de la graine de coriandre; elle était blanche, et avait le goût d`un gâteau au miel. 
\verse Moïse dit: Voici ce que l`Éternel a ordonné: Qu`un omer rempli de manne soit conservé pour vos descendants, afin qu`ils voient le pain que je vous ai fait manger dans le désert, après vous avoir fait sortir du pays d`Égypte. 
\verse Et Moïse dit à Aaron: Prends un vase, mets-y de la manne plein un omer, et dépose-le devant l`Éternel, afin qu`il soit conservé pour vos descendants. 
\verse Suivant l`ordre donné par l`Éternel à Moïse, Aaron le déposa devant le témoignage, afin qu`il fût conservé. 
\verse Les enfants d`Israël mangèrent la manne pendant quarante ans, jusqu`à leur arrivée dans un pays habité; ils mangèrent la manne jusqu`à leur arrivée aux frontières du pays de Canaan. 
\verse L`omer est la dixième partie de l`épha. 

\chapter
\verse Toute l`assemblée des enfants d`Israël partit du désert de Sin, selon les marches que l`Éternel leur avait ordonnées; et ils campèrent à Rephidim, où le peuple ne trouva point d`eau à boire. 
\verse Alors le peuple chercha querelle à Moïse. Ils dirent: Donnez-nous de l`eau à boire. Moïse leur répondit: Pourquoi me cherchez-vous querelle? Pourquoi tentez-vous l`Éternel? 
\verse Le peuple était là, pressé par la soif, et murmurait contre Moïse. Il disait: Pourquoi nous as-tu fait monter hors d`Égypte, pour me faire mourir de soif avec mes enfants et mes troupeaux? 
\verse Moïse cria à l`Éternel, en disant: Que ferai-je à ce peuple? Encore un peu, et ils me lapideront. 
\verse L`Éternel dit à Moïse: Passe devant le peuple, et prends avec toi des anciens d`Israël; prends aussi dans ta main ta verge avec laquelle tu as frappé le fleuve, et marche! 
\verse Voici, je me tiendrai devant toi sur le rocher d`Horeb; tu frapperas le rocher, et il en sortira de l`eau, et le peuple boira. Et Moïse fit ainsi, aux yeux des anciens d`Israël. 
\verse Il donna à ce lieu le nom de Massa et Meriba, parce que les enfants d`Israël avaient contesté, et parce qu`ils avaient tenté l`Éternel, en disant: L`Éternel est-il au milieu de nous, ou n`y est-il pas? 
\verse Amalek vint combattre Israël à Rephidim. 
\verse Alors Moïse dit à Josué: Choisis-nous des hommes, sors, et combats Amalek; demain je me tiendrai sur le sommet de la colline, la verge de Dieu dans ma main. 
\verse Josué fit ce que lui avait dit Moïse, pour combattre Amalek. Et Moïse, Aaron et Hur montèrent au sommet de la colline. 
\verse Lorsque Moïse élevait sa main, Israël était le plus fort; et lorsqu`il baissait sa main, Amalek était le plus fort. 
\verse Les mains de Moïse étant fatiguées, ils prirent une pierre qu`ils placèrent sous lui, et il s`assit dessus. Aaron et Hur soutenaient ses mains, l`un d`un côté, l`autre de l`autre; et ses mains restèrent fermes jusqu`au coucher du soleil. 
\verse Et Josué vainquit Amalek et son peuple, au tranchant de l`épée. 
\verse L`Éternel dit à Moïse: Écris cela dans le livre, pour que le souvenir s`en conserve, et déclare à Josué que j`effacerai la mémoire d`Amalek de dessous les cieux. 
\verse Moïse bâtit un autel, et lui donna pour nom: l`Éternel ma bannière. 
\verse Il dit: Parce que la main a été levée sur le trône de l`Éternel, il y aura guerre de l`Éternel contre Amalek, de génération en génération. 

\chapter
\verse Jéthro, sacrificateur de Madian, beau-père de Moïse, apprit tout ce que Dieu avait fait en faveur de Moïse et d`Israël, son peuple; il apprit que l`Éternel avait fait sortir Israël d`Égypte. 
\verse Jéthro, beau-père de Moïse, prit Séphora, femme de Moïse, qui avait été renvoyée. 
\verse Il prit aussi les deux fils de Séphora; l`un se nommait Guerschom, car Moïse avait dit: J`habite un pays étranger; 
\verse l`autre se nommait Éliézer, car il avait dit: Le Dieu de mon père m`a secouru, et il m`a délivré de l`épée de Pharaon. 
\verse Jéthro, beau-père de Moïse, avec les fils et la femme de Moïse, vint au désert où il campait, à la montagne de Dieu. 
\verse Il fit dire à Moïse: Moi, ton beau-père Jéthro, je viens vers toi, avec ta femme et ses deux fils. 
\verse Moïse sortit au-devant de son beau-père, il se prosterna, et il le baisa. Ils s`informèrent réciproquement de leur santé, et ils entrèrent dans la tente de Moïse. 
\verse Moïse raconta à son beau-père tout ce que l`Éternel avait fait à Pharaon et à l`Égypte à cause d`Israël, toutes les souffrances qui leur étaient survenues en chemin, et comment l`Éternel les avait délivrés. 
\verse Jéthro se réjouit de tout le bien que l`Éternel avait fait à Israël, et de ce qu`il l`avait délivré de la main des Égyptiens. 
\verse Et Jéthro dit: Béni soit l`Éternel, qui vous a délivrés de la main des Égyptiens et de la main de Pharaon; qui a délivré le peuple de la main des Égyptiens! 
\verse Je reconnais maintenant que l`Éternel est plus grand que tous les dieux; car la méchanceté des Égyptiens est retombée sur eux. 
\verse Jéthro, beau-père de Moïse, offrit à Dieu un holocauste et des sacrifices. Aaron et tous les anciens d`Israël vinrent participer au repas avec le beau-père de Moïse, en présence de Dieu. 
\verse Le lendemain, Moïse s`assit pour juger le peuple, et le peuple se tint devant lui depuis le matin jusqu`au soir. 
\verse Le beau-père de Moïse vit tout ce qu`il faisait pour le peuple, et il dit: Que fais-tu là avec ce peuple? Pourquoi sièges-tu seul, et tout le peuple se tient-il devant toi, depuis le matin jusqu`au soir? 
\verse Moïse répondit à son beau-père: C`est que le peuple vient à moi pour consulter Dieu. 
\verse Quand ils ont quelque affaire, ils viennent à moi; je prononce entre eux, et je fais connaître les ordonnances de Dieu et ses lois. 
\verse Le beau-père de Moïse lui dit: Ce que tu fais n`est pas bien. 
\verse Tu t`épuiseras toi-même, et tu épuiseras ce peuple qui est avec toi; car la chose est au-dessus de tes forces, tu ne pourras pas y suffire seul. 
\verse Maintenant écoute ma voix; je vais te donner un conseil, et que Dieu soit avec toi! Sois l`interprète du peuple auprès de Dieu, et porte les affaires devant Dieu. 
\verse Enseigne-leur les ordonnances et les lois; et fais-leur connaître le chemin qu`ils doivent suivre, et ce qu`ils doivent faire. 
\verse Choisis parmi tout le peuple des hommes capables, craignant Dieu, des hommes intègres, ennemis de la cupidité; établis-les sur eux comme chefs de mille, chefs de cent, chefs de cinquante et chefs de dix. 
\verse Qu`ils jugent le peuple en tout temps; qu`ils portent devant toi toutes les affaires importantes, et qu`ils prononcent eux-mêmes sur les petites causes. Allège ta charge, et qu`ils la portent avec toi. 
\verse Si tu fais cela, et que Dieu te donne des ordres, tu pourras y suffire, et tout ce peuple parviendra heureusement à sa destination. 
\verse Moïse écouta la voix de son beau-père, et fit tout ce qu`il avait dit. 
\verse Moïse choisit des hommes capables parmi tout Israël, et il les établit chefs du peuple, chefs de mille, chefs de cent, chefs de cinquante et chefs de dix. 
\verse Ils jugeaient le peuple en tout temps; ils portaient devant Moïse les affaires difficiles, et ils prononçaient eux-mêmes sur toutes les petites causes. 
\verse Moïse laissa partir son beau-père, et Jéthro s`en alla dans son pays. 

\chapter
\verse Le troisième mois après leur sortie du pays d`Égypte, les enfants d`Israël arrivèrent ce jour-là au désert de Sinaï. 
\verse Étant partis de Rephidim, ils arrivèrent au désert de Sinaï, et ils campèrent dans le désert; Israël campa là, vis-à-vis de la montagne. 
\verse Moïse monta vers Dieu: et l`Éternel l`appela du haut de la montagne, en disant: Tu parleras ainsi à la maison de Jacob, et tu diras aux enfants d`Israël: 
\verse Vous avez vu ce que j`ai fait à l`Égypte, et comment je vous ai portés sur des ailes d`aigle et amenés vers moi. 
\verse Maintenant, si vous écoutez ma voix, et si vous gardez mon alliance, vous m`appartiendrez entre tous les peuples, car toute la terre est à moi; 
\verse vous serez pour moi un royaume de sacrificateurs et une nation sainte. Voilà les paroles que tu diras aux enfants d`Israël. 
\verse Moïse vint appeler les anciens du peuple, et il mit devant eux toutes ces paroles, comme l`Éternel le lui avait ordonné. 
\verse Le peuple tout entier répondit: Nous ferons tout ce que l`Éternel a dit. Moïse rapporta les paroles du peuple à l`Éternel. 
\verse Et l`Éternel dit à Moïse: Voici, je viendrai vers toi dans une épaisse nuée, afin que le peuple entende quand je te parlerai, et qu`il ait toujours confiance en toi. Moïse rapporta les paroles du peuple à l`Éternel. 
\verse Et l`Éternel dit à Moïse: Va vers le peuple; sanctifie-les aujourd`hui et demain, qu`ils lavent leurs vêtements. 
\verse Qu`ils soient prêts pour le troisième jour; car le troisième jour l`Éternel descendra, aux yeux de tout le peuple, sur la montagne de Sinaï. 
\verse Tu fixeras au peuple des limites tout à l`entour, et tu diras: Gardez-vous de monter sur la montagne, ou d`en toucher le bord. Quiconque touchera la montagne sera puni de mort. 
\verse On ne mettra pas la main sur lui, mais on le lapidera, ou on le percera de flèches: animal ou homme, il ne vivra point. Quand la trompette sonnera, ils s`avanceront près de la montagne. 
\verse Moïse descendit de la montagne vers le peuple; il sanctifia le peuple, et ils lavèrent leurs vêtements. 
\verse Et il dit au peuple: Soyez prêts dans trois jours; ne vous approchez d`aucune femme. 
\verse Le troisième jour au matin, il y eut des tonnerres, des éclairs, et une épaisse nuée sur la montagne; le son de la trompette retentit fortement; et tout le peuple qui était dans le camp fut saisi d`épouvante. 
\verse Moïse fit sortir le peuple du camp, à la rencontre de Dieu; et ils se placèrent au bas de la montagne. 
\verse La montagne de Sinaï était tout en fumée, parce que l`Éternel y était descendu au milieu du feu; cette fumée s`élevait comme la fumée d`une fournaise, et toute la montagne tremblait avec violence. 
\verse Le son de la trompette retentissait de plus en plus fortement. Moïse parlait, et Dieu lui répondait à haute voix. 
\verse Ainsi l`Éternel descendit sur la montagne de Sinaï, sur le sommet de la montagne; l`Éternel appela Moïse sur le sommet de la montagne. Et Moïse monta. 
\verse L`Éternel dit à Moïse: Descends, fais au peuple la défense expresse de se précipiter vers l`Éternel, pour regarder, de peur qu`un grand nombre d`entre eux ne périssent. 
\verse Que les sacrificateurs, qui s`approchent de l`Éternel, se sanctifient aussi, de peur que l`Éternel ne les frappe de mort. 
\verse Moïse dit à l`Éternel: Le peuple ne pourra pas monter sur la montagne de Sinaï, car tu nous en as fait la défense expresse, en disant: Fixe des limites autour de la montagne, et sanctifie-la. 
\verse L`Éternel lui dit: Va, descends; tu monteras ensuite avec Aaron; mais que les sacrificateurs et le peuple ne se précipitent point pour monter vers l`Éternel, de peur qu`il ne les frappe de mort. 
\verse Moïse descendit vers le peuple, et lui dit ces choses. 

\chapter
\verse Alors Dieu prononça toutes ces paroles, en disant: 
\verse Je suis l`Éternel, ton Dieu, qui t`ai fait sortir du pays d`Égypte, de la maison de servitude. 
\verse Tu n`auras pas d`autres dieux devant ma face. 
\verse Tu ne te feras point d`image taillée, ni de représentation quelconque des choses qui sont en haut dans les cieux, qui sont en bas sur la terre, et qui sont dans les eaux plus bas que la terre. 
\verse Tu ne te prosterneras point devant elles, et tu ne les serviras point; car moi, l`Éternel, ton Dieu, je suis un Dieu jaloux, qui punis l`iniquité des pères sur les enfants jusqu`à la troisième et la quatrième génération de ceux qui me haïssent, 
\verse et qui fais miséricorde jusqu`en mille générations à ceux qui m`aiment et qui gardent mes commandements. 
\verse Tu ne prendras point le nom de l`Éternel, ton Dieu, en vain; car l`Éternel ne laissera point impuni celui qui prendra son nom en vain. 
\verse Souviens-toi du jour du repos, pour le sanctifier. 
\verse Tu travailleras six jours, et tu feras tout ton ouvrage. 
\verse Mais le septième jour est le jour du repos de l`Éternel, ton Dieu: tu ne feras aucun ouvrage, ni toi, ni ton fils, ni ta fille, ni ton serviteur, ni ta servante, ni ton bétail, ni l`étranger qui est dans tes portes. 
\verse Car en six jours l`Éternel a fait les cieux, la terre et la mer, et tout ce qui y est contenu, et il s`est reposé le septième jour: c`est pourquoi l`Éternel a béni le jour du repos et l`a sanctifié. 
\verse Honore ton père et ta mère, afin que tes jours se prolongent dans le pays que l`Éternel, ton Dieu, te donne. 
\verse Tu ne tueras point. 
\verse Tu ne commettras point d`adultère. 
\verse Tu ne déroberas point. 
\verse Tu ne porteras point de faux témoignage contre ton prochain. 
\verse Tu ne convoiteras point la maison de ton prochain; tu ne convoiteras point la femme de ton prochain, ni son serviteur, ni sa servante, ni son boeuf, ni son âne, ni aucune chose qui appartienne à ton prochain. 
\verse Tout le peuple entendait les tonnerres et le son de la trompette; il voyait les flammes de la montagne fumante. A ce spectacle, le peuple tremblait, et se tenait dans l`éloignement. 
\verse Ils dirent à Moïse: Parle-nous toi-même, et nous écouterons; mais que Dieu ne nous parle point, de peur que nous ne mourions. 
\verse Moïse dit au peuple: Ne vous effrayez pas; car c`est pour vous mettre à l`épreuve que Dieu est venu, et c`est pour que vous ayez sa crainte devant les yeux, afin que vous ne péchiez point. 
\verse Le peuple restait dans l`éloignement; mais Moïse s`approcha de la nuée où était Dieu. 
\verse L`Éternel dit à Moïse: Tu parleras ainsi aux enfants d`Israël: Vous avez vu que je vous ai parlé depuis les cieux. 
\verse Vous ne ferez point des dieux d`argent et des dieux d`or, pour me les associer; vous ne vous en ferez point. 
\verse Tu m`élèveras un autel de terre, sur lequel tu offriras tes holocaustes et tes sacrifices d`actions de grâces, tes brebis et tes boeufs. Partout où je rappellerai mon nom, je viendrai à toi, et je te bénirai. 
\verse Si tu m`élèves un autel de pierre, tu ne le bâtiras point en pierres taillées; car en passant ton ciseau sur la pierre, tu la profanerais. 
\verse Tu ne monteras point à mon autel par des degrés, afin que ta nudité ne soit pas découverte. 

\chapter
\verse Voici les lois que tu leur présenteras. 
\verse Si tu achètes un esclave hébreu, il servira six années; mais la septième, il sortira libre, sans rien payer. 
\verse S`il est entré seul, il sortira seul; s`il avait une femme, sa femme sortira avec lui. 
\verse Si c`est son maître qui lui a donné une femme, et qu`il en ait eu des fils ou des filles, la femme et ses enfants seront à son maître, et il sortira seul. 
\verse Si l`esclave dit: J`aime mon maître, ma femme et mes enfants, je ne veux pas sortir libre, - 
\verse alors son maître le conduira devant Dieu, et le fera approcher de la porte ou du poteau, et son maître lui percera l`oreille avec un poinçon, et l`esclave sera pour toujours à son service. 
\verse Si un homme vend sa fille pour être esclave, elle ne sortira point comme sortent les esclaves. 
\verse Si elle déplaît à son maître, qui s`était proposé de la prendre pour femme, il facilitera son rachat; mais il n`aura pas le pouvoir de la vendre à des étrangers, après lui avoir été infidèle. 
\verse S`il la destine à son fils, il agira envers elle selon le droit des filles. 
\verse S`il prend une autre femme, il ne retranchera rien pour la première à la nourriture, au vêtement, et au droit conjugal. 
\verse Et s`il ne fait pas pour elle ces trois choses, elle pourra sortir sans rien payer, sans donner de l`argent. 
\verse Celui qui frappera un homme mortellement sera puni de mort. 
\verse S`il ne lui a point dressé d`embûches, et que Dieu l`ait fait tomber sous sa main, je t`établirai un lieu où il pourra se réfugier. 
\verse Mais si quelqu`un agit méchamment contre son prochain, en employant la ruse pour le tuer, tu l`arracheras même de mon autel, pour le faire mourir. 
\verse Celui qui frappera son père ou sa mère sera puni de mort. 
\verse Celui qui dérobera un homme, et qui l`aura vendu ou retenu entre ses mains, sera puni de mort. 
\verse Celui qui maudira son père ou sa mère sera puni de mort. 
\verse Si des hommes se querellent, et que l`un d`eux frappe l`autre avec une pierre ou avec le poing, sans causer sa mort, mais en l`obligeant à garder le lit, 
\verse celui qui aura frappé ne sera point puni, dans le cas où l`autre viendrait à se lever et à se promener dehors avec son bâton. Seulement, il le dédommagera de son interruption de travail, et il le fera soigner jusqu`à sa guérison. 
\verse Si un homme frappe du bâton son esclave, homme ou femme, et que l`esclave meure sous sa main, le maître sera puni. 
\verse Mais s`il survit un jour ou deux, le maître ne sera point puni; car c`est son argent. 
\verse Si des hommes se querellent, et qu`ils heurtent une femme enceinte, et la fasse accoucher, sans autre accident, ils seront punis d`une amende imposée par le mari de la femme, et qu`ils paieront devant les juges. 
\verse Mais s`il y a un accident, tu donneras vie pour vie, 
\verse oeil pour oeil, dent pour dent, main pour main, pied pour pied, 
\verse brûlure pour brûlure, blessure pour blessure, meurtrissure pour meurtrissure. 
\verse Si un homme frappe l`oeil de son esclave, homme ou femme, et qu`il lui fasse perdre l`oeil, il le mettra en liberté, pour prix de son oeil. 
\verse Et s`il fait tomber une dent à son esclave, homme ou femme, il le mettra en liberté, pour prix de sa dent. 
\verse Si un boeuf frappe de ses cornes un homme ou une femme, et que la mort en soit la suite, le boeuf sera lapidé, sa chair ne sera point mangée, et le maître du boeuf ne sera point puni. 
\verse Mais si le boeuf était auparavant sujet à frapper, et qu`on en ait averti le maître, qui ne l`a point surveillé, le boeuf sera lapidé, dans le cas où il tuerait un homme ou une femme, et son maître sera puni de mort. 
\verse Si on impose au maître un prix pour le rachat de sa vie, il paiera tout ce qui lui sera imposé. 
\verse Lorsque le boeuf frappera un fils ou une fille, cette loi recevra son application; 
\verse mais si le boeuf frappe un esclave, homme ou femme, on donnera trente sicles d`argent au maître de l`esclave, et le boeuf sera lapidé. 
\verse Si un homme met à découvert une citerne, ou si un homme en creuse une et ne la couvre pas, et qu`il y tombe un boeuf ou un âne, 
\verse le possesseur de la citerne paiera au maître la valeur de l`animal en argent, et aura pour lui l`animal mort. 
\verse Si le boeuf d`un homme frappe de ses cornes le boeuf d`un autre homme, et que la mort en soit la suite, ils vendront le boeuf vivant et en partageront le prix; ils partageront aussi le boeuf mort. 
\verse Mais s`il est connu que le boeuf était auparavant sujet à frapper, et que son maître ne l`ait point surveillé, ce maître rendra boeuf pour boeuf, et aura pour lui le boeuf mort. 

\chapter
\verse Si un homme dérobe un boeuf ou un agneau, et qu`il l`égorge ou le vende, il restituera cinq boeufs pour le boeuf et quatre agneaux pour l`agneau. 
\verse Si le voleur est surpris dérobant avec effraction, et qu`il soit frappé et meure, on ne sera point coupable de meurtre envers lui; 
\verse mais si le soleil est levé, on sera coupable de meurtre envers lui. Il fera restitution; s`il n`a rien, il sera vendu pour son vol; 
\verse si ce qu`il a dérobé, boeuf, âne, ou agneau, se trouve encore vivant entre ses mains, il fera une restitution au double. 
\verse Si un homme fait du dégât dans un champ ou dans une vigne, et qu`il laisse son bétail paître dans le champ d`autrui, il donnera en dédommagement le meilleur produit de son champ et de sa vigne. 
\verse Si un feu éclate et rencontre des épines, et que du blé en gerbes ou sur pied, ou bien le champ, soit consumé, celui qui a causé l`incendie sera tenu à un dédommagement. 
\verse Si un homme donne à un autre de l`argent ou des objets à garder, et qu`on les vole dans la maison de ce dernier, le voleur fera une restitution au double, dans le cas où il serait trouvé. 
\verse Si le voleur ne se trouve pas, le maître de la maison se présentera devant Dieu, pour déclarer qu`il n`a pas mis la main sur le bien de son prochain. 
\verse Dans toute affaire frauduleuse concernant un boeuf, un âne, un agneau, un vêtement, ou un objet perdu, au sujet duquel on dira: C`est cela! -la cause des deux parties ira jusqu`à Dieu; celui que Dieu condamnera fera à son prochain une restitution au double. 
\verse Si un homme donne à un autre un âne, un boeuf, un agneau, ou un animal quelconque à garder, et que l`animal meure, se casse un membre, ou soit enlevé, sans que personne l`ait vu, 
\verse le serment au nom de l`Éternel interviendra entre les deux parties, et celui qui a gardé l`animal déclarera qu`il n`a pas mis la main sur le bien de son prochain; le maître de l`animal acceptera ce serment, et l`autre ne sera point tenu à une restitution. 
\verse Mais si l`animal a été dérobé chez lui, il sera tenu vis-à-vis de son maître à une restitution. 
\verse Si l`animal a été déchiré, il le produira en témoignage, et il ne sera point tenu à une restitution pour ce qui a été déchiré. 
\verse Si un homme emprunte à un autre un animal, et que l`animal se casse un membre ou qu`il meure, en l`absence de son maître, il y aura lieu à restitution. 
\verse Si le maître est présent, il n`y aura pas lieu à restitution. Si l`animal a été loué, le prix du louage suffira. 
\verse Si un homme séduit une vierge qui n`est point fiancée, et qu`il couche avec elle, il paiera sa dot et la prendra pour femme. 
\verse Si le père refuse de la lui accorder, il paiera en argent la valeur de la dot des vierges. 
\verse Tu ne laisseras point vivre la magicienne. 
\verse Quiconque couche avec une bête sera puni de mort. 
\verse Celui qui offre des sacrifices à d`autres dieux qu`à l`Éternel seul sera voué à l`extermination. 
\verse Tu ne maltraiteras point l`étranger, et tu ne l`opprimeras point; car vous avez été étrangers dans le pays d`Égypte. 
\verse Tu n`affligeras point la veuve, ni l`orphelin. 
\verse Si tu les affliges, et qu`ils viennent à moi, j`entendrai leurs cris; 
\verse ma colère s`enflammera, et je vous détruirai par l`épée; vos femmes deviendront veuves, et vos enfants orphelins. 
\verse Si tu prêtes de l`argent à mon peuple, au pauvre qui est avec toi, tu ne seras point à son égard comme un créancier, tu n`exigeras de lui point d`intérêt. 
\verse Si tu prends en gage le vêtement de ton prochain, tu le lui rendras avant le coucher du soleil; 
\verse car c`est sa seule couverture, c`est le vêtement dont il s`enveloppe le corps: dans quoi coucherait-il? S`il crie à moi, je l`entendrai, car je suis miséricordieux. 
\verse Tu ne maudiras point Dieu, et tu ne maudiras point le prince de ton peuple. 
\verse Tu ne différeras point de m`offrir les prémices de ta moisson et de ta vendange. Tu me donneras le premier-né de tes fils. 
\verse Tu me donneras aussi le premier-né de ta vache et de ta brebis; il restera sept jours avec sa mère; le huitième jour, tu me le donneras. 
\verse Vous serez pour moi des hommes saints. Vous ne mangerez point de chair déchirée dans les champs: vous la jetterez aux chiens. 

\chapter
\verse Tu ne répandras point de faux bruit. Tu ne te joindras point au méchant pour faire un faux témoignage. 
\verse Tu ne suivras point la multitude pour faire le mal; et tu ne déposeras point dans un procès en te mettant du côté du grand nombre, pour violer la justice. 
\verse Tu ne favoriseras point le pauvre dans son procès. 
\verse Si tu rencontres le boeuf de ton ennemi ou son âne égaré, tu le lui ramèneras. 
\verse Si tu vois l`âne de ton ennemi succombant sous sa charge, et que tu hésites à le décharger, tu l`aideras à le décharger. 
\verse Tu ne porteras point atteinte au droit du pauvre dans son procès. 
\verse Tu ne prononceras point de sentence inique, et tu ne feras point mourir l`innocent et le juste; car je n`absoudrai point le coupable. 
\verse Tu ne recevras point de présent; car les présents aveuglent ceux qui ont les yeux ouverts et corrompent les paroles des justes. 
\verse Tu n`opprimeras point l`étranger; vous savez ce qu`éprouve l`étranger, car vous avez été étrangers dans le pays d`Égypte. 
\verse Pendant six années, tu ensemenceras la terre, et tu en recueilleras le produit. 
\verse Mais la septième, tu lui donneras du relâche et tu la laisseras en repos; les pauvres de ton peuple en jouiront, et les bêtes des champs mangeront ce qui restera. Tu feras de même pour ta vigne et pour tes oliviers. 
\verse Pendant six jours, tu feras ton ouvrage. Mais le septième jour, tu te reposeras, afin que ton boeuf et ton âne aient du repos, afin que le fils de ton esclave et l`étranger aient du relâche. 
\verse Vous observerez tout ce que je vous ai dit, et vous ne prononcerez point le nom d`autres dieux: qu`on ne l`entende point sortir de votre bouche. 
\verse Trois fois par année, tu célébreras des fêtes en mon honneur. 
\verse Tu observeras la fête des pains sans levain; pendant sept jours, au temps fixé dans le mois des épis, tu mangeras des pains sans levain, comme je t`en ai donné l`ordre, car c`est dans ce mois que tu es sorti d`Égypte; et l`on ne se présentera point à vide devant ma face. 
\verse Tu observeras la fête de la moisson, des prémices de ton travail, de ce que tu auras semé dans les champs; et la fête de la récolte, à la fin de l`année, quand tu recueilleras des champs le fruit de ton travail. 
\verse Trois fois par année, tous les mâles se présenteront devant le Seigneur, l`Éternel. 
\verse Tu n`offriras point avec du pain levé le sang de la victime sacrifiée en mon honneur; et sa graisse ne sera point gardée pendant la nuit jusqu`au matin. 
\verse Tu apporteras à la maison de l`Éternel, ton Dieu, les prémices des premiers fruits de la terre. Tu ne feras point cuire un chevreau dans le lait de sa mère. 
\verse Voici, j`envoie un ange devant toi, pour te protéger en chemin, et pour te faire arriver au lieu que j`ai préparé. 
\verse Tiens-toi sur tes gardes en sa présence, et écoute sa voix; ne lui résiste point, parce qu`il ne pardonnera pas vos péchés, car mon nom est en lui. 
\verse Mais si tu écoutes sa voix, et si tu fais tout ce que je te dirai, je serai l`ennemi de tes ennemis et l`adversaire de tes adversaires. 
\verse Mon ange marchera devant toi, et te conduira chez les Amoréens, les Héthiens, les Phéréziens, les Cananéens, les Héviens et les Jébusiens, et je les exterminerai. 
\verse Tu ne te prosterneras point devant leurs dieux, et tu ne les serviras point; tu n`imiteras point ces peuples dans leur conduite, mais tu les détruiras, et tu briseras leurs statues. 
\verse Vous servirez l`Éternel, votre Dieu, et il bénira votre pain et vos eaux, et j`éloignerai la maladie du milieu de toi. 
\verse Il n`y aura dans ton pays ni femme qui avorte, ni femme stérile. Je remplirai le nombre de tes jours. 
\verse J`enverrai ma terreur devant toi, je mettrai en déroute tous les peuples chez lesquels tu arriveras, et je ferai tourner le dos devant toi à tous tes ennemis. 
\verse J`enverrai les frelons devant toi, et ils chasseront loin de ta face les Héviens, les Cananéens et les Héthiens. 
\verse Je ne les chasserai pas en une seule année loin de ta face, de peur que le pays ne devienne un désert et que les bêtes des champs ne se multiplient contre toi. 
\verse Je les chasserai peu à peu loin de ta face, jusqu`à ce que tu augmentes en nombre et que tu puisses prendre possession du pays. 
\verse J`établirai tes limites depuis la mer Rouge jusqu`à la mer des Philistins, et depuis le désert jusqu`au fleuve; car je livrerai entre vos mains les habitants du pays, et tu les chasseras devant toi. 
\verse Tu ne feras point d`alliance avec eux, ni avec leurs dieux. 
\verse Ils n`habiteront point dans ton pays, de peur qu`ils ne te fassent pécher contre moi; car tu servirais leurs dieux, et ce serait un piège pour toi. 

\chapter
\verse Dieu dit à Moïse: Monte vers l`Éternel, toi et Aaron, Nadab et Abihu, et soixante-dix des anciens d`Israël, et vous vous prosternerez de loin. 
\verse Moïse s`approchera seul de l`Éternel; les autres ne s`approcheront pas, et le peuple ne montera point avec lui. 
\verse Moïse vint rapporter au peuple toutes les paroles de l`Éternel et toutes les lois. Le peuple entier répondit d`une même voix: Nous ferons tout ce que l`Éternel a dit. 
\verse Moïse écrivit toutes les paroles de l`Éternel. Puis il se leva de bon matin; il bâtit un autel au pied de la montagne, et dressa douze pierres pour les douze tribus d`Israël. 
\verse Il envoya des jeunes hommes, enfants d`Israël, pour offrir à l`Éternel des holocaustes, et immoler des taureaux en sacrifices d`actions de grâces. 
\verse Moïse prit la moitié du sang, qu`il mit dans des bassins, et il répandit l`autre moitié sur l`autel. 
\verse Il prit le livre de l`alliance, et le lut en présence du peuple; ils dirent: Nous ferons tout ce que l`Éternel a dit, et nous obéirons. 
\verse Moïse prit le sang, et il le répandit sur le peuple, en disant: Voici le sang de l`alliance que l`Éternel a faite avec vous selon toutes ces paroles. 
\verse Moïse monta avec Aaron, Nadab et Abihu, et soixante-dix anciens d`Israël. 
\verse Ils virent le Dieu d`Israël; sous ses pieds, c`était comme un ouvrage de saphir transparent, comme le ciel lui-même dans sa pureté. 
\verse Il n`étendit point sa main sur l`élite des enfants d`Israël. Ils virent Dieu, et ils mangèrent et burent. 
\verse L`Éternel dit à Moïse: Monte vers moi sur la montagne, et reste là; je te donnerai des tables de pierre, la loi et les ordonnances que j`ai écrites pour leur instruction. 
\verse Moïse se leva, avec Josué qui le servait, et Moïse monta sur la montagne de Dieu. 
\verse Il dit aux anciens: Attendez-nous ici, jusqu`à ce que nous revenions auprès de vous. Voici, Aaron et Hur resteront avec vous; si quelqu`un a un différend, c`est à eux qu`il s`adressera. 
\verse Moïse monta sur la montagne, et la nuée couvrit la montagne. 
\verse La gloire de l`Éternel reposa sur la montagne de Sinaï, et la nuée le couvrit pendant six jours. Le septième jour, l`Éternel appela Moïse du milieu de la nuée. 
\verse L`aspect de la gloire de l`Éternel était comme un feu dévorant sur le sommet de la montagne, aux yeux des enfants d`Israël. 
\verse Moïse entra au milieu de la nuée, et il monta sur la montagne. Moïse demeura sur la montagne quarante jours et quarante nuits. 

\chapter
\verse L`Éternel parla à Moïse, et dit: 
\verse Parle aux enfants d`Israël. Qu`ils m`apportent une offrande; vous la recevrez pour moi de tout homme qui la fera de bon coeur. 
\verse Voici ce que vous recevrez d`eux en offrande: de l`or, de l`argent et de l`airain; 
\verse des étoffes teintes en bleu, en pourpre, en cramoisi, du fin lin et du poil de chèvre; 
\verse des peaux de béliers teintes en rouge et des peaux de dauphins; du bois d`acacia; 
\verse de l`huile pour le chandelier, des aromates pour l`huile d`onction et pour le parfum odoriférant; 
\verse des pierres d`onyx et d`autres pierres pour la garniture de l`éphod et du pectoral. 
\verse Ils me feront un sanctuaire, et j`habiterai au milieu d`eux. 
\verse Vous ferez le tabernacle et tous ses ustensiles d`après le modèle que je vais te montrer. 
\verse Ils feront une arche de bois d`acacia, sa longueur sera de deux coudées et demie, sa largeur d`une coudée et demie, et sa hauteur d`une coudée et demie. 
\verse Tu la couvriras d`or pur, tu la couvriras en dedans et en dehors, et tu y feras une bordure d`or tout autour. 
\verse Tu fondras pour elle quatre anneaux d`or, et tu les mettras à ses quatre coins, deux anneaux d`un côté et deux anneaux de l`autre côté. 
\verse Tu feras des barres de bois d`acacia, et tu les couvriras d`or. 
\verse Tu passeras les barres dans les anneaux sur les côtés de l`arche, pour qu`elles servent à porter l`arche; 
\verse les barres resteront dans les anneaux de l`arche, et n`en seront point retirées. 
\verse Tu mettras dans l`arche le témoignage, que je te donnerai. 
\verse Tu feras un propitiatoire d`or pur; sa longueur sera de deux coudées et demie, et sa largeur d`une coudée et demie. 
\verse Tu feras deux chérubins d`or, tu les feras d`or battu, aux deux extrémités du propitiatoire; 
\verse fais un chérubin à l`une des extrémités et un chérubin à l`autre extrémité; vous ferez les chérubins sortant du propitiatoire à ses deux extrémités. 
\verse Les chérubins étendront les ailes par-dessus, couvrant de leurs ailes le propitiatoire, et se faisant face l`un à l`autre; les chérubins auront la face tournée vers le propitiatoire. 
\verse Tu mettras le propitiatoire sur l`arche, et tu mettras dans l`arche le témoignage, que je te donnerai. 
\verse C`est là que je me rencontrerai avec toi; du haut du propitiatoire, entre les deux chérubins placés sur l`arche du témoignage, je te donnerai tous mes ordres pour les enfants d`Israël. 
\verse Tu feras une table de bois d`acacia; sa longueur sera de deux coudées, sa largeur d`une coudée, et sa hauteur d`une coudée et demie. 
\verse Tu la couvriras d`or pur, et tu y feras une bordure d`or tout autour. 
\verse Tu y feras à l`entour un rebord de quatre doigts, sur lequel tu mettras une bordure d`or tout autour. 
\verse Tu feras pour la table quatre anneaux d`or, et tu mettras les anneaux aux quatre coins, qui seront à ses quatre pieds. 
\verse Les anneaux seront près du rebord, et recevront les barres pour porter la table. 
\verse Tu feras les barres de bois d`acacia, et tu les couvriras d`or; et elles serviront à porter la table. 
\verse Tu feras ses plats, ses coupes, ses calices et ses tasses, pour servir aux libations; tu les feras d`or pur. 
\verse Tu mettras sur la table les pains de proposition continuellement devant ma face. 
\verse Tu feras un chandelier d`or pur; ce chandelier sera fait d`or battu; son pied, sa tige, ses calices, ses pommes et ses fleurs seront d`une même pièce. 
\verse Six branches sortiront de ses côtés, trois branches du chandelier de l`un des côtés, et trois branches du chandelier de l`autre côté. 
\verse Il y aura sur une branche trois calices en forme d`amande, avec pommes et fleurs, et sur une autre branche trois calices en forme d`amande, avec pommes et fleurs; il en sera de même pour les six branches sortant du chandelier. 
\verse A la tige du chandelier, il y aura quatre calices en forme d`amande, avec leurs pommes et leurs fleurs. 
\verse Il y aura une pomme sous deux des branches sortant de la tige du chandelier, une pomme sous deux autres branches, et une pomme sous deux autres branches; il en sera de même pour les six branches sortant du chandelier. 
\verse Les pommes et les branches du chandelier seront d`une même pièce: il sera tout entier d`or battu, d`or pur. 
\verse Tu feras ses sept lampes, qui seront placées dessus, de manière à éclairer en face. 
\verse Ses mouchettes et ses vases à cendre seront d`or pur. 
\verse On emploiera un talent d`or pur pour faire le chandelier avec tous ses ustensiles. 
\verse Regarde, et fais d`après le modèle qui t`est montré sur la montagne. 

\chapter
\verse Tu feras le tabernacle de dix tapis de fin lin retors, et d`étoffes teintes en bleu, en pourpre et en cramoisi; tu y représenteras des chérubins artistement travaillés. 
\verse La longueur d`un tapis sera de vingt-huit coudées, et la largeur d`un tapis sera de quatre coudées; la mesure sera la même pour tous les tapis. 
\verse Cinq de ces tapis seront joints ensemble; les cinq autres seront aussi joints ensemble. 
\verse Tu feras des lacets bleus au bord du tapis terminant le premier assemblage; et tu feras de même au bord du tapis terminant le second assemblage. 
\verse Tu mettras cinquante lacets au premier tapis, et tu mettras cinquante lacets au bord du tapis terminant le second assemblage; ces lacets se correspondront les uns aux autres. 
\verse Tu feras cinquante agrafes d`or, et tu joindras les tapis l`un à l`autre avec les agrafes. Et le tabernacle formera un tout. 
\verse Tu feras des tapis de poil de chèvre, pour servir de tente sur le tabernacle; tu feras onze de ces tapis. 
\verse La longueur d`un tapis sera de trente coudées, et la largeur d`un tapis sera de quatre coudées; la mesure sera la même pour les onze tapis. 
\verse Tu joindras séparément cinq de ces tapis, et les six autres séparément, et tu redoubleras le sixième tapis sur le devant de la tente. 
\verse Tu mettras cinquante lacets au bord du tapis terminant le premier assemblage, et cinquante lacets au bord du tapis du second assemblage. 
\verse Tu feras cinquante agrafes d`airain, et tu feras entrer les agrafes dans les lacets. Tu assembleras ainsi la tente, qui fera un tout. 
\verse Comme il y aura du surplus dans les tapis de la tente, la moitié du tapis de reste retombera sur le derrière du tabernacle; 
\verse la coudée d`une part, et la coudée d`autre part, qui seront de reste sur la longueur des tapis de la tente, retomberont sur les deux côtés du tabernacle, pour le couvrir. 
\verse Tu feras pour la tente une couverture de peaux de béliers teintes en rouge, et une couverture de peaux de dauphins par-dessus. 
\verse Tu feras des planches pour le tabernacle; elles seront de bois d`acacia, placées debout. 
\verse La longueur d`une planche sera de dix coudées, et la largeur d`une planche sera d`une coudée et demie. 
\verse Il y aura à chaque planche deux tenons joints l`un à l`autre; tu feras de même pour toutes les planches du tabernacle. 
\verse Tu feras vingt planches pour le tabernacle, du côté du midi. 
\verse Tu mettras quarante bases d`argent sous les vingt planches, deux bases sous chaque planche pour ses deux tenons. 
\verse Tu feras vingt planches pour le second côté du tabernacle, le côté du nord, 
\verse et leurs quarante bases d`argent, deux bases sous chaque planche. 
\verse Tu feras six planches pour le fond du tabernacle, du côté de l`occident. 
\verse Tu feras deux planches pour les angles du tabernacle, dans le fond; 
\verse elles seront doubles depuis le bas, et bien liées à leur sommet par un anneau; il en sera de même pour toutes les deux, placées aux deux angles. 
\verse Il y aura ainsi huit planches, avec leurs bases d`argent, soit seize bases, deux bases sous chaque planche. 
\verse Tu feras cinq barres de bois d`acacia pour les planches de l`un des côtés du tabernacle, 
\verse cinq barres pour les planches du second côté du tabernacle, et cinq barres pour les planches du côté du tabernacle formant le fond vers l`occident. 
\verse La barre du milieu traversera les planches d`une extrémité à l`autre. 
\verse Tu couvriras d`or les planches, et tu feras d`or leurs anneaux qui recevront les barres, et tu couvriras d`or les barres. 
\verse Tu dresseras le tabernacle d`après le modèle qui t`est montré sur la montagne. 
\verse Tu feras un voile bleu, pourpre et cramoisi, et de fin lin retors; il sera artistement travaillé, et l`on y représentera des chérubins. 
\verse Tu le mettras sur quatre colonnes d`acacia, couvertes d`or; ces colonnes auront des crochets d`or, et poseront sur quatre bases d`argent. 
\verse Tu mettras le voile au-dessous des agrafes, et c`est là, en dedans du voile, que tu feras entrer l`arche du témoignage; le voile vous servira de séparation entre le lieu saint et le lieu très saint. 
\verse Tu mettras le propitiatoire sur l`arche du témoignage dans le lieu très saint. 
\verse Tu mettras la table en dehors du voile, et le chandelier en face de la table, au côté méridional du tabernacle; et tu mettras la table au côté septentrional. 
\verse Tu feras pour l`entrée de la tente un rideau bleu, pourpre et cramoisi, et de fin lin retors; ce sera un ouvrage de broderie. 
\verse Tu feras pour le rideau cinq colonnes d`acacia, et tu les couvriras d`or; elles auront des crochets d`or, et tu fondras pour elles cinq bases d`airain. 

\chapter
\verse Tu feras l`autel de bois d`acacia; sa longueur sera de cinq coudées, et sa largeur de cinq coudées. L`autel sera carré, et sa hauteur sera de trois coudées. 
\verse Tu feras, aux quatre coins, des cornes qui sortiront de l`autel; et tu le couvriras d`airain. 
\verse Tu feras pour l`autel des cendriers, des pelles, des bassins, des fourchettes et des brasiers; tu feras d`airain tous ses ustensiles. 
\verse Tu feras à l`autel une grille d`airain, en forme de treillis, et tu mettras quatre anneaux d`airain aux quatre coins du treillis. 
\verse Tu le placeras au-dessous du rebord de l`autel, à partir du bas, jusqu`à la moitié de la hauteur de l`autel. 
\verse Tu feras des barres pour l`autel, des barres de bois d`acacia, et tu les couvriras d`airain. 
\verse On passera les barres dans les anneaux; et les barres seront aux deux côtés de l`autel, quand on le portera. 
\verse Tu le feras creux, avec des planches; il sera fait tel qu`il t`est montré sur la montagne. 
\verse Tu feras le parvis du tabernacle. Du côté du midi, il y aura, pour former le parvis, des toiles de fin lin retors, sur une longueur de cent coudées pour ce premier côté, 
\verse avec vingt colonnes posant sur vingt bases d`airain; les crochets des colonnes et leurs tringles seront d`argent. 
\verse Du côté du nord, il y aura également des toiles sur une longueur de cent coudées, avec vingt colonnes et leurs vingt bases d`airain; les crochets des colonnes et leurs tringles seront d`argent. 
\verse Du côté de l`occident, il y aura pour la largeur du parvis cinquante coudées de toiles, avec dix colonnes et leurs dix bases. 
\verse Du côté de l`orient, sur les cinquante coudées de largeur du parvis, 
\verse il y aura quinze coudées de toiles pour une aile, avec trois colonnes et leurs trois bases, 
\verse et quinze coudées de toiles pour la seconde aile, avec trois colonnes et leurs trois bases. 
\verse Pour la porte du parvis il y aura un rideau de vingt coudées, bleu, pourpre et cramoisi, et de fin lin retors, en ouvrage de broderie, avec quatre colonnes et leurs quatre bases. 
\verse Toutes les colonnes formant l`enceinte du parvis auront des tringles d`argent, des crochets d`argent, et des bases d`airain. 
\verse La longueur du parvis sera de cent coudées, sa largeur de cinquante de chaque côté, et sa hauteur de cinq coudées; les toiles seront de fin lin retors, et les bases d`airain. 
\verse Tous les ustensiles destinés au service du tabernacle, tous ses pieux, et tous les pieux du parvis, seront d`airain. 
\verse Tu ordonneras aux enfants d`Israël de t`apporter pour le chandelier de l`huile pure d`olives concassées, afin d`entretenir les lampes continuellement. 
\verse C`est dans la tente d`assignation, en dehors du voile qui est devant le témoignage, qu`Aaron et ses fils la prépareront, pour que les lampes brûlent du soir au matin en présence de l`Éternel. C`est une loi perpétuelle pour leurs descendants, et que devront observer les enfants d`Israël. 

\chapter
\verse Fais approcher de toi Aaron, ton frère, et ses fils, et prends-les parmi les enfants d`Israël pour les consacrer à mon service dans le sacerdoce: Aaron et les fils d`Aaron, Nadab, Abihu, Éléazar et Ithamar. 
\verse Tu feras à Aaron, ton frère, des vêtements sacrés, pour marquer sa dignité et pour lui servir de parure. 
\verse Tu parleras à tous ceux qui sont habiles, à qui j`ai donné un esprit plein d`intelligence; et ils feront les vêtements d`Aaron, afin qu`il soit consacré et qu`il exerce mon sacerdoce. 
\verse Voici les vêtements qu`ils feront: un pectoral, un éphod, une robe, une tunique brodée, une tiare, et une ceinture. Ils feront des vêtements sacrés à Aaron, ton frère, et à ses fils, afin qu`ils exercent mon sacerdoce. 
\verse Ils emploieront de l`or, des étoffes teintes en bleu, en pourpre, en cramoisi, et de fin lin. 
\verse Ils feront l`éphod d`or, de fil bleu, pourpre et cramoisi, et de fin lin retors; il sera artistement travaillé. 
\verse On y fera deux épaulettes, qui le joindront par ses deux extrémités; et c`est ainsi qu`il sera joint. 
\verse La ceinture sera du même travail que l`éphod et fixée sur lui; elle sera d`or, de fil bleu, pourpre et cramoisi, et de fin lin retors. 
\verse Tu prendras deux pierres d`onyx, et tu y graveras les noms des fils d`Israël, 
\verse six de leurs noms sur une pierre, et les six autres sur la seconde pierre, d`après l`ordre des naissances. 
\verse Tu graveras sur les deux pierres les noms des fils d`Israël, comme on grave les pierres et les cachets; tu les entoureras de montures d`or. 
\verse Tu mettras les deux pierres sur les épaulettes de l`éphod, en souvenir des fils d`Israël; et c`est comme souvenir qu`Aaron portera leurs noms devant l`Éternel sur ses deux épaules. 
\verse Tu feras des montures d`or, 
\verse et deux chaînettes d`or pur, que tu tresseras en forme de cordons; et tu fixeras aux montures les chaînettes ainsi tressées. 
\verse Tu feras le pectoral du jugement, artistement travaillé; tu le feras du même travail que l`éphod, tu le feras d`or, de fil bleu, pourpre et cramoisi, et de fin lin retors. 
\verse Il sera carré et double; sa longueur sera d`un empan, et sa largeur d`un empan. 
\verse Tu y enchâsseras une garniture de pierres, quatre rangées de pierres: première rangée, une sardoine, une topaze, une émeraude; 
\verse seconde rangée, une escarboucle, un saphir, un diamant; 
\verse troisième rangée, une opale, une agate, une améthyste; 
\verse quatrième rangée, une chrysolithe, un onyx, un jaspe. Ces pierres seront enchâssées dans leurs montures d`or. 
\verse Il y en aura douze, d`après les noms des fils d`Israël; elles seront gravées comme des cachets, chacune avec le nom de l`une des douze tribus. - 
\verse Tu feras sur le pectoral des chaînettes d`or pur, tressées en forme de cordons. 
\verse Tu feras sur le pectoral deux anneaux d`or, et tu mettras ces deux anneaux aux deux extrémités du pectoral. 
\verse Tu passeras les deux cordons d`or dans les deux anneaux aux deux extrémités du pectoral; 
\verse et tu arrêteras par devant les bouts des deux cordons aux deux montures placées sur les épaulettes de l`éphod. 
\verse Tu feras encore deux anneaux d`or, que tu mettras aux deux extrémités du pectoral, sur le bord intérieur appliqué contre l`éphod. 
\verse Et tu feras deux autres anneaux d`or, que tu mettras au bas des deux épaulettes de l`éphod, sur le devant, près de la jointure, au-dessus de la ceinture de l`éphod. 
\verse On attachera le pectoral par ses anneaux de l`éphod avec un cordon bleu, afin que le pectoral soit au-dessus de la ceinture de l`éphod et qu`il ne puisse pas se séparer de l`éphod. 
\verse Lorsque Aaron entrera dans le sanctuaire, il portera sur son coeur les noms des fils d`Israël, gravés sur le pectoral du jugement, pour en conserver à toujours le souvenir devant l`Éternel. - 
\verse Tu joindras au pectoral du jugement l`urim et le thummim, et ils seront sur le coeur d`Aaron, lorsqu`il se présentera devant l`Éternel. Ainsi, Aaron portera constamment sur son coeur le jugement des enfants d`Israël, lorsqu`il se présentera devant l`Éternel. 
\verse Tu feras la robe de l`éphod entièrement d`étoffe bleue. 
\verse Il y aura, au milieu, une ouverture pour la tête; et cette ouverture aura tout autour un bord tissé, comme l`ouverture d`une cotte de mailles, afin que la robe ne se déchire pas. 
\verse Tu mettras autour de la bordure, en bas, des grenades de couleur bleue, pourpre et cramoisie, entremêlées de clochettes d`or: 
\verse une clochette d`or et une grenade, une clochette d`or et une grenade, sur tout le tour de la bordure de la robe. 
\verse Aaron s`en revêtira pour faire le service; quand il entrera dans le sanctuaire devant l`Éternel, et quand il en sortira, on entendra le son des clochettes, et il ne mourra point. 
\verse Tu feras une lame d`or pur, et tu y graveras, comme on grave un cachet: Sainteté à l`Éternel. 
\verse Tu l`attacheras avec un cordon bleu sur la tiare, sur le devant de la tiare. 
\verse Elle sera sur le front d`Aaron; et Aaron sera chargé des iniquités commises par les enfants d`Israël en faisant toutes leurs saintes offrandes; elle sera constamment sur son front devant l`Éternel, pour qu`il leur soit favorable. 
\verse Tu feras la tunique de fin lin; tu feras une tiare de fin lin, et tu feras une ceinture brodée. 
\verse Pour les fils d`Aaron tu feras des tuniques, tu leur feras des ceintures, et tu leur feras des bonnets, pour marquer leur dignité et pour leur servir de parure. 
\verse Tu en revêtiras Aaron, ton frère, et ses fils avec lui. Tu les oindras, tu les consacreras, tu les sanctifieras, et ils seront à mon service dans le sacerdoce. 
\verse Fais-leur des caleçons de lin, pour couvrir leur nudité; ils iront depuis les reins jusqu`aux cuisses. 
\verse Aaron et ses fils les porteront, quand ils entreront dans la tente d`assignation, ou quand ils s`approcheront de l`autel, pour faire le service dans le sanctuaire; ainsi ils ne se rendront point coupables, et ne mourront point. C`est une loi perpétuelle pour Aaron et pour ses descendants après lui. 

\chapter
\verse Voici ce que tu feras pour les sanctifier, afin qu`ils soient à mon service dans le sacerdoce. Prends un jeune taureau et deux béliers sans défaut. 
\verse Fais, avec de la fleur de farine de froment, des pains sans levain, des gâteaux sans levain pétris à l`huile, et des galettes sans levain arrosées d`huile. 
\verse Tu les mettras dans une corbeille, en offrant le jeune taureau et les deux béliers. 
\verse Tu feras avancer Aaron et ses fils vers l`entrée de la tente d`assignation, et tu les laveras avec de l`eau. 
\verse Tu prendras les vêtements; tu revêtiras Aaron de la tunique, de la robe de l`éphod, de l`éphod et du pectoral, et tu mettras sur lui la ceinture de l`éphod. 
\verse Tu poseras la tiare sur sa tête, et tu placeras le diadème de sainteté sur la tiare. 
\verse Tu prendras l`huile d`onction, tu en répandras sur sa tête, et tu l`oindras. 
\verse Tu feras approcher ses fils, et tu les revêtiras des tuniques. 
\verse Tu mettras une ceinture à Aaron et à ses fils, et tu attacheras des bonnets aux fils d`Aaron. Le sacerdoce leur appartiendra par une loi perpétuelle. Tu consacreras donc Aaron et ses fils. 
\verse Tu amèneras le taureau devant la tente d`assignation, et Aaron et ses fils poseront leurs mains sur la tête du taureau. 
\verse Tu égorgeras le taureau devant l`Éternel, à l`entrée de la tente d`assignation. 
\verse Tu prendras du sang du taureau, tu en mettras avec ton doigt sur les cornes de l`autel, et tu répandras tout le sang au pied de l`autel. 
\verse Tu prendras toute la graisse qui couvre les entrailles, le grand lobe du foie, les deux rognons et la graisse qui les entoure, et tu brûleras cela sur l`autel. 
\verse Mais tu brûleras au feu hors du camp la chair du taureau, sa peau et ses excréments: c`est un sacrifice pour le péché. 
\verse Tu prendras l`un des béliers, et Aaron et ses fils poseront leurs mains sur la tête du bélier. 
\verse Tu égorgeras le bélier; tu en prendras le sang, et tu le répandras sur l`autel tout autour. 
\verse Tu couperas le bélier par morceaux, et tu laveras les entrailles et les jambes, que tu mettras sur les morceaux et sur sa tête. 
\verse Tu brûleras tout le bélier sur l`autel; c`est un holocauste à l`Éternel, c`est un sacrifice consumé par le feu, d`une agréable odeur à l`Éternel. 
\verse Tu prendras l`autre bélier, et Aaron et ses fils poseront leurs mains sur la tête du bélier. 
\verse Tu égorgeras le bélier; tu prendras de son sang, tu en mettras sur le lobe de l`oreille droite d`Aaron et sur le lobe de l`oreille droite de ses fils, sur le pouce de leur main droite et sur le gros orteil de leur pied droit, et tu répandras le sang sur l`autel tout autour. 
\verse Tu prendras du sang qui sera sur l`autel et de l`huile d`onction, et tu en feras l`aspersion sur Aaron et sur ses vêtements, sur ses fils et sur leurs vêtements. Ainsi seront consacrés Aaron et ses vêtements, ses fils et leurs vêtements. 
\verse Tu prendras la graisse du bélier, la queue, la graisse qui couvre les entrailles, le grand lobe du foie, les deux rognons et la graisse qui les entoure, et l`épaule droite, car c`est un bélier de consécration; 
\verse tu prendras aussi dans la corbeille de pains sans levain, placée devant l`Éternel, un gâteau de pain, un gâteau à l`huile et une galette. 
\verse Tu mettras toutes ces choses sur les mains d`Aaron et sur les mains de ses fils, et tu les agiteras de côté et d`autre devant l`Éternel. 
\verse Tu les ôteras ensuite de leurs mains, et tu les brûleras sur l`autel, par-dessus l`holocauste; c`est un sacrifice consumé par le feu devant l`Éternel, d`une agréable odeur à l`Éternel. 
\verse Tu prendras la poitrine du bélier qui aura servi à la consécration d`Aaron, et tu l`agiteras de côté et d`autre devant l`Éternel: ce sera ta portion. 
\verse Tu sanctifieras la poitrine et l`épaule du bélier qui aura servi à la consécration d`Aaron et de ses fils, la poitrine en l`agitant de côté et d`autre, l`épaule en la présentant par élévation. 
\verse Elles appartiendront à Aaron et à ses fils, par une loi perpétuelle qu`observeront les enfants d`Israël, car c`est une offrande par élévation; et, dans les sacrifices d`actions de grâces des enfants d`Israël, l`offrande par élévation sera pour l`Éternel. 
\verse Les vêtements sacrés d`Aaron seront après lui pour ses fils, qui les mettront lorsqu`on les oindra et qu`on les consacrera. 
\verse Ils seront portés pendant sept jours par celui de ses fils qui lui succédera dans le sacerdoce, et qui entrera dans la tente d`assignation, pour faire le service dans le sanctuaire. 
\verse Tu prendras le bélier de consécration, et tu en feras cuire la chair dans un lieu saint. 
\verse Aaron et ses fils mangeront, à l`entrée de la tente d`assignation, la chair du bélier et le pain qui sera dans la corbeille. 
\verse Ils mangeront ainsi ce qui aura servi d`expiation afin qu`ils fussent consacrés et sanctifiés; nul étranger n`en mangera, car ce sont des choses saintes. 
\verse S`il reste de la chair de consécration et du pain jusqu`au matin, tu brûleras dans le feu ce qui restera; on ne le mangera point, car c`est une chose sainte. 
\verse Tu suivras à l`égard d`Aaron et de ses fils tous les ordres que je t`ai donnés. Tu emploieras sept jours à les consacrer. 
\verse Tu offriras chaque jour un taureau en sacrifice pour le péché, pour l`expiation; tu purifieras l`autel par cette expiation, et tu l`oindras pour le sanctifier. 
\verse Pendant sept jours, tu feras des expiations sur l`autel, et tu le sanctifieras; et l`autel sera très saint, et tout ce qui touchera l`autel sera sanctifié. 
\verse Voici ce que tu offriras sur l`autel: deux agneaux d`un an, chaque jour, à perpétuité. 
\verse Tu offriras l`un des agneaux le matin, et l`autre agneau entre les deux soirs. 
\verse Tu offriras, avec le premier agneau, un dixième d`épha de fleur de farine pétrie dans un quart de hin d`huile d`olives concassées, et une libation d`un quart de hin de vin. 
\verse Tu offriras le second agneau entre les deux soirs, avec une offrande et une libation semblables à celles du matin; c`est un sacrifice consumé par le feu, d`une agréable odeur à l`Éternel. 
\verse Voilà l`holocauste perpétuel qui sera offert par vos descendants, à l`entrée de la tente d`assignation, devant l`Éternel: c`est là que je me rencontrerai avec vous, et que je te parlerai. 
\verse Je me rencontrerai là avec les enfants d`Israël, et ce lieu sera sanctifié par ma gloire. 
\verse Je sanctifierai la tente d`assignation et l`autel; je sanctifierai Aaron et ses fils, pour qu`ils soient à mon service dans le sacerdoce. 
\verse J`habiterai au milieu des enfants d`Israël, et je serai leur Dieu. 
\verse Ils connaîtront que je suis l`Éternel, leur Dieu, qui les ai fait sortir du pays d`Égypte, pour habiter au milieu d`eux. Je suis l`Éternel, leur Dieu. 

\chapter
\verse Tu feras un autel pour brûler des parfums, tu le feras de bois d`acacia; 
\verse sa longueur sera d`une coudée, et sa largeur d`une coudée; il sera carré, et sa hauteur sera de deux coudées. Tu feras des cornes qui sortiront de l`autel. 
\verse Tu le couvriras d`or pur, le dessus, les côtés tout autour et les cornes, et tu y feras une bordure d`or tout autour. 
\verse Tu feras au-dessous de la bordure deux anneaux d`or aux deux côtés; tu en mettras aux deux côtés, pour recevoir les barres qui serviront à le porter. 
\verse Tu feras les barres de bois d`acacia, et tu les couvriras d`or. 
\verse Tu placeras l`autel en face du voile qui est devant l`arche du témoignage, en face du propitiatoire qui est sur le témoignage, et où je me rencontrerai avec toi. 
\verse Aaron y fera brûler du parfum odoriférant; il en fera brûler chaque matin, lorsqu`il préparera les lampes; 
\verse il en fera brûler aussi entre les deux soirs, lorsqu`il arrangera les lampes. C`est ainsi que l`on brûlera à perpétuité du parfum devant l`Éternel parmi vos descendants. 
\verse Vous n`offrirez sur l`autel ni parfum étranger, ni holocauste, ni offrande, et vous n`y répandrez aucune libation. 
\verse Une fois chaque année, Aaron fera des expiations sur les cornes de l`autel; avec le sang de la victime expiatoire, il y sera fait des expiations une fois chaque année parmi vos descendants. Ce sera une chose très sainte devant l`Éternel. 
\verse L`Éternel parla à Moïse, et dit: 
\verse Lorsque tu compteras les enfants d`Israël pour en faire le dénombrement, chacun d`eux paiera à l`Éternel le rachat de sa personne, afin qu`ils ne soient frappés d`aucune plaie lors de ce dénombrement. 
\verse Voici ce que donneront tous ceux qui seront compris dans le dénombrement: un demi-sicle, selon le sicle du sanctuaire, qui est de vingt guéras; un demi-sicle sera le don prélevé pour l`Éternel. 
\verse Tout homme compris dans le dénombrement, depuis l`âge de vingt ans et au-dessus, paiera le don prélevé pour l`Éternel. 
\verse Le riche ne paiera pas plus, et le pauvre ne paiera pas moins d`un demi-sicle, comme don prélevé pour l`Éternel, afin de racheter leurs personnes. 
\verse Tu recevras des enfants d`Israël l`argent du rachat, et tu l`appliqueras au travail de la tente d`assignation; ce sera pour les enfants d`Israël un souvenir devant l`Éternel pour le rachat de leurs personnes. 
\verse L`Éternel parla à Moïse, et dit: 
\verse Tu feras une cuve d`airain, avec sa base d`airain, pour les ablutions; tu la placeras entre la tente d`assignation et l`autel, et tu y mettras de l`eau, 
\verse avec laquelle Aaron et ses fils se laveront les mains et les pieds. 
\verse Lorsqu`ils entreront dans la tente d`assignation, ils se laveront avec cette eau, afin qu`ils ne meurent point; et aussi lorsqu`ils s`approcheront de l`autel, pour faire le service et pour offrir des sacrifices à l`Éternel. 
\verse Ils se laveront les mains et les pieds, afin qu`ils ne meurent point. Ce sera une loi perpétuelle pour Aaron, pour ses fils et pour leurs descendants. 
\verse L`Éternel parla à Moïse, et dit: 
\verse Prends des meilleurs aromates, cinq cents sicles de myrrhe, de celle qui coule d`elle-même; la moitié, soit deux cent cinquante sicles, de cinnamome aromatique, deux cent cinquante sicles de roseau aromatique, 
\verse cinq cents sicles de casse, selon le sicle du sanctuaire, et un hin d`huile d`olive. 
\verse Tu feras avec cela une huile pour l`onction sainte, composition de parfums selon l`art du parfumeur; ce sera l`huile pour l`onction sainte. 
\verse Tu en oindras la tente d`assignation et l`arche du témoignage, 
\verse la table et tous ses ustensiles, le chandelier et ses ustensiles, l`autel des parfums, 
\verse l`autel des holocaustes et tous ses ustensiles, la cuve avec sa base. 
\verse Tu sanctifieras ces choses, et elles seront très saintes, tout ce qui les touchera sera sanctifié. 
\verse Tu oindras Aaron et ses fils, et tu les sanctifieras, pour qu`ils soient à mon service dans le sacerdoce. 
\verse Tu parleras aux enfants d`Israël, et tu diras: Ce sera pour moi l`huile de l`onction sainte, parmi vos descendants. 
\verse On n`en répandra point sur le corps d`un homme, et vous n`en ferez point de semblable, dans les mêmes proportions; elle est sainte, et vous la regarderez comme sainte. 
\verse Quiconque en composera de semblable, ou en mettra sur un étranger, sera retranché de son peuple. 
\verse L`Éternel dit à Moïse: Prends des aromates, du stacté, de l`ongle odorant, du galbanum, et de l`encens pur, en parties égales. 
\verse Tu feras avec cela un parfum composé selon l`art du parfumeur; il sera salé, pur et saint. 
\verse Tu le réduiras en poudre, et tu le mettras devant le témoignage, dans la tente d`assignation, où je me rencontrerai avec toi. Ce sera pour vous une chose très sainte. 
\verse Vous ne ferez point pour vous de parfum semblable, dans les mêmes proportions; vous le regarderez comme saint, et réservé pour l`Éternel. 
\verse Quiconque en fera de semblable, pour le sentir, sera retranché de son peuple. 

\chapter
\verse L`Éternel parla à Moïse, et dit: 
\verse Sache que j`ai choisi Betsaleel, fils d`Uri, fils de Hur, de la tribu de Juda. 
\verse Je l`ai rempli de l`Esprit de Dieu, de sagesse, d`intelligence, et de savoir pour toutes sortes d`ouvrages, 
\verse je l`ai rendu capable de faire des inventions, de travailler l`or, l`argent et l`airain, 
\verse de graver les pierres à enchâsser, de travailler le bois, et d`exécuter toutes sortes d`ouvrages. 
\verse Et voici, je lui ai donné pour aide Oholiab, fils d`Ahisamac, de la tribu de Dan. J`ai mis de l`intelligence dans l`esprit de tous ceux qui sont habiles, pour qu`ils fassent tout ce que je t`ai ordonné: 
\verse la tente d`assignation, l`arche du témoignage, le propitiatoire qui sera dessus, et tous les ustensiles de la tente; 
\verse la table et ses ustensiles, le chandelier d`or pur et tous ses ustensiles, 
\verse l`autel des parfums; l`autel des holocaustes et tous ses ustensiles, la cuve avec sa base; 
\verse les vêtements d`office, les vêtements sacrés pour le sacrificateur Aaron, les vêtements de ses fils pour les fonctions du sacerdoce; 
\verse l`huile d`onction, et le parfum odoriférant pour le sanctuaire. Ils se conformeront à tous les ordres que j`ai donnés. 
\verse L`Éternel parla à Moïse, et dit: 
\verse Parle aux enfants d`Israël, et dis-leur: Vous ne manquerez pas d`observer mes sabbats, car ce sera entre moi et vous, et parmi vos descendants, un signe auquel on connaîtra que je suis l`Éternel qui vous sanctifie. 
\verse Vous observerez le sabbat, car il sera pour vous une chose sainte. Celui qui le profanera, sera puni de mort; celui qui fera quelque ouvrage ce jour-là, sera retranché du milieu de son peuple. 
\verse On travaillera six jours; mais le septième jour est le sabbat, le jour du repos, consacré à l`Éternel. Celui qui fera quelque ouvrage le jour du sabbat, sera puni de mort. 
\verse Les enfants d`Israël observeront le sabbat, en le célébrant, eux et leurs descendants, comme une alliance perpétuelle. 
\verse Ce sera entre moi et les enfants d`Israël un signe qui devra durer à perpétuité; car en six jours l`Éternel a fait les cieux et la terre, et le septième jour il a cessé son oeuvre et il s`est reposé. 
\verse Lorsque l`Éternel eut achevé de parler à Moïse sur la montagne de Sinaï, il lui donna les deux tables du témoignage, tables de pierre, écrites du doigt de Dieu. 

\chapter
\verse Le peuple, voyant que Moïse tardait à descendre de la montagne, s`assembla autour d`Aaron, et lui dit: Allons! fais-nous un dieu qui marche devant nous, car ce Moïse, cet homme qui nous a fait sortir du pays d`Égypte, nous ne savons ce qu`il est devenu. 
\verse Aaron leur dit: Otez les anneaux d`or qui sont aux oreilles de vos femmes, de vos fils et de vos filles, et apportez-les-moi. 
\verse Et tous ôtèrent les anneaux d`or qui étaient à leurs oreilles, et ils les apportèrent à Aaron. 
\verse Il les reçut de leurs mains, jeta l`or dans un moule, et fit un veau en fonte. Et ils dirent: Israël! voici ton dieu, qui t`a fait sortir du pays d`Égypte. 
\verse Lorsqu`Aaron vit cela, il bâtit un autel devant lui, et il s`écria: Demain, il y aura fête en l`honneur de l`Éternel! 
\verse Le lendemain, ils se levèrent de bon matin, et ils offrirent des holocaustes et des sacrifices d`actions de grâces. Le peuple s`assit pour manger et pour boire; puis ils se levèrent pour se divertir. 
\verse L`Éternel dit à Moïse: Va, descends; car ton peuple, que tu as fait sortir du pays d`Égypte, s`est corrompu. 
\verse Ils se sont promptement écartés de la voie que je leur avais prescrite; ils se sont fait un veau en fonte, ils se sont prosternés devant lui, ils lui ont offert des sacrifices, et ils ont dit: Israël! voici ton dieu, qui t`a fait sortir du pays d`Égypte. 
\verse L`Éternel dit à Moïse: Je vois que ce peuple est un peuple au cou roide. 
\verse Maintenant laisse-moi; ma colère va s`enflammer contre eux, et je les consumerai; mais je ferai de toi une grande nation. 
\verse Moïse implora l`Éternel, son Dieu, et dit: Pourquoi, ô Éternel! ta colère s`enflammerait-elle contre ton peuple, que tu as fait sortir du pays d`Égypte par une grande puissance et par une main forte? 
\verse Pourquoi les Égyptiens diraient-ils: C`est pour leur malheur qu`il les a fait sortir, c`est pour les tuer dans les montagnes, et pour les exterminer de dessus la terre? Reviens de l`ardeur de ta colère, et repens-toi du mal que tu veux faire à ton peuple. 
\verse Souviens-toi d`Abraham, d`Isaac et d`Israël, tes serviteurs, auxquels tu as dit, en jurant par toi-même: Je multiplierai votre postérité comme les étoiles du ciel, je donnerai à vos descendants tout ce pays dont j`ai parlé, et ils le posséderont à jamais. 
\verse Et l`Éternel se repentit du mal qu`il avait déclaré vouloir faire à son peuple. 
\verse Moïse retourna et descendit de la montagne, les deux tables du témoignage dans sa main; les tables étaient écrites des deux côtés, elles étaient écrites de l`un et de l`autre côté. 
\verse Les tables étaient l`ouvrage de Dieu, et l`écriture était l`écriture de Dieu, gravée sur les tables. 
\verse Josué entendit la voix du peuple, qui poussait des cris, et il dit à Moïse: Il y a un cri de guerre dans le camp. 
\verse Moïse répondit: Ce n`est ni un cri de vainqueurs, ni un cri de vaincus; ce que j`entends, c`est la voix de gens qui chantent. 
\verse Et, comme il approchait du camp, il vit le veau et les danses. La colère de Moïse s`enflamma; il jeta de ses mains les tables, et les brisa au pied de la montagne. 
\verse Il prit le veau qu`ils avaient fait, et le brûla au feu; il le réduisit en poudre, répandit cette poudre à la surface de l`eau, et fit boire les enfants d`Israël. 
\verse Moïse dit à Aaron: Que t`a fait ce peuple, pour que tu l`aies laissé commettre un si grand péché? 
\verse Aaron répondit: Que la colère de mon seigneur ne s`enflamme point! Tu sais toi-même que ce peuple est porté au mal. 
\verse Ils m`ont dit: Fais-nous un dieu qui marche devant nous; car ce Moïse, cet homme qui nous a fait sortir du pays d`Égypte, nous ne savons ce qu`il est devenu. 
\verse Je leur ai dit: Que ceux qui ont de l`or, s`en dépouillent! Et ils me l`ont donné; je l`ai jeté au feu, et il en est sorti ce veau. 
\verse Moïse vit que le peuple était livré au désordre, et qu`Aaron l`avait laissé dans ce désordre, exposé à l`opprobre parmi ses ennemis. 
\verse Moïse se plaça à la porte du camp, et dit: A moi ceux qui sont pour l`Éternel! Et tous les enfants de Lévi s`assemblèrent auprès de lui. 
\verse Il leur dit: Ainsi parle l`Éternel, le Dieu d`Israël: Que chacun de vous mette son épée au côté; traversez et parcourez le camp d`une porte à l`autre, et que chacun tue son frère, son parent. 
\verse Les enfants de Lévi firent ce qu`ordonnait Moïse; et environ trois mille hommes parmi le peuple périrent en cette journée. 
\verse Moïse dit: Consacrez-vous aujourd`hui à l`Éternel, même en sacrifiant votre fils et votre frère, afin qu`il vous accorde aujourd`hui une bénédiction. 
\verse Le lendemain, Moïse dit au peuple: Vous avez commis un grand péché. Je vais maintenant monter vers l`Éternel: j`obtiendrai peut-être le pardon de votre péché. 
\verse Moïse retourna vers l`Éternel et dit: Ah! ce peuple a commis un grand péché. Ils se sont fait un dieu d`or. 
\verse Pardonne maintenant leur péché! Sinon, efface-moi de ton livre que tu as écrit. 
\verse L`Éternel dit à Moïse: C`est celui qui a péché contre moi que j`effacerai de mon livre. 
\verse Va donc, conduis le peuple où je t`ai dit. Voici, mon ange marchera devant toi, mais au jour de ma vengeance, je les punirai de leur péché. 
\verse L`Éternel frappa le peuple, parce qu`il avait fait le veau, fabriqué par Aaron. 

\chapter
\verse L`Éternel dit à Moïse: Va, pars d`ici, toi et le peuple que tu as fait sortir du pays d`Égypte; monte vers le pays que j`ai juré de donner à Abraham, à Isaac et à Jacob, en disant: Je le donnerai à ta postérité. 
\verse J`enverrai devant toi un ange, et je chasserai les Cananéens, les Amoréens, les Héthiens, les Phéréziens, les Héviens et les Jébusiens. 
\verse Monte vers ce pays où coulent le lait et le miel. Mais je ne monterai point au milieu de toi, de peur que je ne te consume en chemin, car tu es un peuple au cou roide. 
\verse Lorsque le peuple eut entendu ces sinistres paroles, il fut dans la désolation, et personne ne mit ses ornements. 
\verse Et l`Éternel dit à Moïse: Dis aux enfants d`Israël: Vous êtes un peuple au cou roide; si je montais un seul instant au milieu de toi, je te consumerais. Ote maintenant tes ornements de dessus toi, et je verrai ce que je te ferai. 
\verse Les enfants d`Israël se dépouillèrent de leurs ornements, en s`éloignant du mont Horeb. 
\verse Moïse prit la tente et la dressa hors du camp, à quelque distance; il l`appela tente d`assignation; et tous ceux qui consultaient l`Éternel allaient vers la tente d`assignation, qui était hors du camp. 
\verse Lorsque Moïse se rendait à la tente, tout le peuple se levait; chacun se tenait à l`entrée de sa tente, et suivait des yeux Moïse, jusqu`à ce qu`il fût entré dans la tente. 
\verse Et lorsque Moïse était entré dans la tente, la colonne de nuée descendait et s`arrêtait à l`entrée de la tente, et l`Éternel parlait avec Moïse. 
\verse Tout le peuple voyait la colonne de nuée qui s`arrêtait à l`entrée de la tente, tout le peuple se levait et se prosternait à l`entrée de sa tente. 
\verse L`Éternel parlait avec Moïse face à face, comme un homme parle à son ami. Puis Moïse retournait au camp; mais son jeune serviteur, Josué, fils de Nun, ne sortait pas du milieu de la tente. 
\verse Moïse dit à l`Éternel: Voici, tu me dis: Fais monter ce peuple! Et tu ne me fais pas connaître qui tu enverras avec moi. Cependant, tu as dit: Je te connais par ton nom, et tu as trouvé grâce à mes yeux. 
\verse Maintenant, si j`ai trouvé grâce à tes yeux, fais-moi connaître tes voies; alors je te connaîtrai, et je trouverai encore grâce à tes yeux. Considère que cette nation est ton peuple. 
\verse L`Éternel répondit: Je marcherai moi-même avec toi, et je te donnerai du repos. 
\verse Moïse lui dit: Si tu ne marches pas toi-même avec nous, ne nous fais point partir d`ici. 
\verse Comment sera-t-il donc certain que j`ai trouvé grâce à tes yeux, moi et ton peuple? Ne sera-ce pas quand tu marcheras avec nous, et quand nous serons distingués, moi et ton peuple, de tous les peuples qui sont sur la face de la terre? 
\verse L`Éternel dit à Moïse: Je ferai ce que tu me demandes, car tu as trouvé grâce à mes yeux, et je te connais par ton nom. 
\verse Moïse dit: Fais-moi voir ta gloire! 
\verse L`Éternel répondit: Je ferai passer devant toi toute ma bonté, et je proclamerai devant toi le nom de l`Éternel; je fais grâce à qui je fais grâce, et miséricorde à qui je fais miséricorde. 
\verse L`Éternel dit: Tu ne pourras pas voir ma face, car l`homme ne peut me voir et vivre. 
\verse L`Éternel dit: Voici un lieu près de moi; tu te tiendras sur le rocher. 
\verse Quand ma gloire passera, je te mettrai dans un creux du rocher, et je te couvrirai de ma main jusqu`à ce que j`aie passé. 
\verse Et lorsque je retournerai ma main, tu me verras par derrière, mais ma face ne pourra pas être vue. 

\chapter
\verse L`Éternel dit à Moïse: Taille deux tables de pierre comme les premières, et j`y écrirai les paroles qui étaient sur les premières tables que tu as brisées. 
\verse Sois prêt de bonne heure, et tu monteras dès le matin sur la montagne de Sinaï; tu te tiendras là devant moi, sur le sommet de la montagne. 
\verse Que personne ne monte avec toi, et que personne ne paraisse sur toute la montagne; et même que ni brebis ni boeufs ne paissent près de cette montagne. 
\verse Moïse tailla deux tables de pierre comme les premières; il se leva de bon matin, et monta sur la montagne de Sinaï, selon l`ordre que l`Éternel lui avait donné, et il prit dans sa main les deux tables de pierre. 
\verse L`Éternel descendit dans une nuée, se tint là auprès de lui, et proclama le nom de l`Éternel. 
\verse Et l`Éternel passa devant lui, et s`écria: L`Éternel, l`Éternel, Dieu miséricordieux et compatissant, lent à la colère, riche en bonté et en fidélité, 
\verse qui conserve son amour jusqu`à mille générations, qui pardonne l`iniquité, la rébellion et le péché, mais qui ne tient point le coupable pour innocent, et qui punit l`iniquité des pères sur les enfants et sur les enfants des enfants jusqu`à la troisième et à la quatrième génération! 
\verse Aussitôt Moïse s`inclina à terre et se prosterna. 
\verse Il dit: Seigneur, si j`ai trouvé grâce à tes yeux, que le Seigneur marche au milieu de nous, car c`est un peuple au cou roide; pardonne nos iniquités et nos péchés, et prends-nous pour ta possession. 
\verse L`Éternel répondit: Voici, je traite une alliance. Je ferai, en présence de tout ton peuple, des prodiges qui n`ont eu lieu dans aucun pays et chez aucune nation; tout le peuple qui t`environne verra l`oeuvre de l`Éternel, et c`est par toi que j`accomplirai des choses terribles. 
\verse Prends garde à ce que je t`ordonne aujourd`hui. Voici, je chasserai devant toi les Amoréens, les Cananéens, les Héthiens, les Phéréziens, les Héviens et les Jébusiens. 
\verse Garde-toi de faire alliance avec les habitants du pays où tu dois entrer, de peur qu`ils ne soient un piège pour toi. 
\verse Au contraire, vous renverserez leurs autels, vous briserez leurs statues, et vous abattrez leurs idoles. 
\verse Tu ne te prosterneras point devant un autre dieu; car l`Éternel porte le nom de jaloux, il est un Dieu jaloux. 
\verse Garde-toi de faire alliance avec les habitants du pays, de peur que, se prostituant à leurs dieux et leur offrant des sacrifices, ils ne t`invitent, et que tu ne manges de leurs victimes; 
\verse de peur que tu ne prennes de leurs filles pour tes fils, et que leurs filles, se prostituant à leurs dieux, n`entraînent tes fils à se prostituer à leurs dieux. 
\verse Tu ne te feras point de dieu en fonte. 
\verse Tu observeras la fête des pains sans levain; pendant sept jours, au temps fixé dans le mois des épis, tu mangeras des pains sans levain, comme je t`en ai donné l`ordre, car c`est dans le mois des épis que tu es sorti d`Égypte. 
\verse Tout premier-né m`appartient, même tout mâle premier-né dans les troupeaux de gros et de menu bétail. 
\verse Tu rachèteras avec un agneau le premier-né de l`âne; et si tu ne le rachètes pas, tu lui briseras la nuque. Tu rachèteras tout premier-né de tes fils; et l`on ne se présentera point à vide devant ma face. 
\verse Tu travailleras six jours, et tu te reposeras le septième jour; tu te reposeras, même au temps du labourage et de la moisson. 
\verse Tu célébreras la fête des semaines, des prémices de la moisson du froment, et la fête de la récolte, à la fin de l`année. 
\verse Trois fois par an, tous les mâles se présenteront devant le Seigneur, l`Éternel, Dieu d`Israël. 
\verse Car je chasserai les nations devant toi, et j`étendrai tes frontières; et personne ne convoitera ton pays, pendant que tu monteras pour te présenter devant l`Éternel, ton Dieu, trois fois par an. 
\verse Tu n`offriras point avec du pain levé le sang de la victime immolée en mon honneur; et le sacrifice de la fête de Pâque ne sera point gardé pendant la nuit jusqu`au matin. 
\verse Tu apporteras à la maison de L`Éternel, ton Dieu, les prémices des premiers fruits de la terre. Tu ne feras point cuire un chevreau dans le lait de sa mère. 
\verse L`Éternel dit à Moïse: Écris ces paroles; car c`est conformément à ces paroles que je traite alliance avec toi et avec Israël. 
\verse Moïse fut là avec l`Éternel quarante jours et quarante nuits. Il ne mangea point de pain, et il ne but point d`eau. Et l`Éternel écrivit sur les tables les paroles de l`alliance, les dix paroles. 
\verse Moïse descendit de la montagne de Sinaï, ayant les deux tables du témoignage dans sa main, en descendant de la montagne; et il ne savait pas que la peau de son visage rayonnait, parce qu`il avait parlé avec l`Éternel. 
\verse Aaron et tous les enfants d`Israël regardèrent Moïse, et voici la peau de son visage rayonnait; et ils craignaient de s`approcher de lui. 
\verse Moïse les appela; Aaron et tous les principaux de l`assemblée vinrent auprès de lui, et il leur parla. 
\verse Après cela, tous les enfants d`Israël s`approchèrent, et il leur donna tous les ordres qu`il avait reçus de l`Éternel, sur la montagne de Sinaï. 
\verse Lorsque Moïse eut achevé de leur parler, il mit un voile sur son visage. 
\verse Quand Moïse entrait devant l`Éternel, pour lui parler, il ôtait le voile, jusqu`à ce qu`il sortît; et quand il sortait, il disait aux enfants d`Israël ce qui lui avait été ordonné. 
\verse Les enfants d`Israël regardaient le visage de Moïse, et voyait que la peau de son visage rayonnait; et Moïse remettait le voile sur son visage jusqu`à ce qu`il entrât, pour parler avec l`Éternel. 

\chapter
\verse Moïse convoqua toute l`assemblée des enfants d`Israël, et leur dit: Voici les choses que l`Éternel ordonne de faire. 
\verse On travaillera six jours; mais le septième jour sera pour vous une chose sainte; c`est le sabbat, le jour du repos, consacré à l`Éternel. Celui qui fera quelque ouvrage ce jour-là, sera puni de mort. 
\verse Vous n`allumerez point de feu, dans aucune de vos demeures, le jour du sabbat. 
\verse Moïse parla à toute l`assemblée des enfants d`Israël, et dit: Voici ce que l`Éternel a ordonné. 
\verse Prenez sur ce qui vous appartient une offrande pour l`Éternel. Tout homme dont le coeur est bien disposé apportera en offrande à l`Éternel: de l`or, de l`argent et de l`airain; 
\verse des étoffes teintes en bleu, en pourpre, en cramoisi, du fin lin et du poil de chèvre; 
\verse des peaux de béliers teintes en rouge et des peaux de dauphins; du bois d`acacia; 
\verse de l`huile pour le chandelier, des aromates pour l`huile d`onction et pour le parfum odoriférant; 
\verse des pierres d`onyx et d`autres pierres pour la garniture de l`éphod et du pectoral. 
\verse Que tous ceux d`entre vous qui ont de l`habileté viennent et exécutent tout ce que l`Éternel a ordonné: 
\verse le tabernacle, sa tente et sa couverture, ses agrafes, ses planches, ses barres, ses colonnes et ses bases; 
\verse l`arche et ses barres, le propitiatoire, et le voile pour couvrir l`arche; 
\verse la table et ses barres, et tous ses ustensiles, et les pains de proposition; 
\verse le chandelier et ses ustensiles, ses lampes, et l`huile pour le chandelier; 
\verse l`autel des parfums et ses barres, l`huile d`onction et le parfum odoriférant, et le rideau de la porte pour l`entrée du tabernacle; 
\verse l`autel des holocaustes, sa grille d`airain, ses barres, et tous ses ustensiles; la cuve avec sa base; 
\verse les toiles du parvis, ses colonnes, ses bases, et le rideau de la porte du parvis; 
\verse les pieux du tabernacle, les pieux du parvis, et leurs cordages; les vêtements d`office pour le service dans le sanctuaire, 
\verse les vêtements sacrés pour le sacrificateur Aaron, et les vêtements de ses fils pour les fonctions du sacerdoce. 
\verse Toute l`assemblée des enfants d`Israël sortit de la présence de Moïse. 
\verse Tous ceux qui furent entraînés par le coeur et animés de bonne volonté vinrent et apportèrent une offrande à l`Éternel pour l`oeuvre de la tente d`assignation, pour tout son service, et pour les vêtements sacrés. 
\verse Les hommes vinrent aussi bien que les femmes; tous ceux dont le coeur était bien disposé apportèrent des boucles, des anneaux, des bagues, des bracelets, toutes sortes d`objets d`or; chacun présenta l`offrande d`or qu`il avait consacrée à l`Éternel. 
\verse Tous ceux qui avaient des étoffes teintes en bleu, en pourpre, en cramoisi, du fin lin et du poil de chèvre, des peaux de béliers teintes en rouge et des peaux de dauphins, les apportèrent. 
\verse Tous ceux qui présentèrent par élévation une offrande d`argent et d`airain apportèrent l`offrande à l`Éternel. Tous ceux qui avaient du bois d`acacia pour les ouvrages destinés au service, l`apportèrent. 
\verse Toutes les femmes qui avaient de l`habileté filèrent de leurs mains, et elles apportèrent leur ouvrage, des fils teints en bleu, en pourpre, en cramoisi, et du fin lin. 
\verse Toutes les femmes dont le coeur était bien disposé, et qui avaient de l`habileté, filèrent du poil de chèvre. 
\verse Les principaux du peuple apportèrent des pierres d`onyx et d`autres pierres pour la garniture de l`éphod et du pectoral; 
\verse des aromates et de l`huile pour le chandelier, pour l`huile d`onction et pour le parfum odoriférant. 
\verse Tous les enfants d`Israël, hommes et femmes, dont le coeur était disposé à contribuer pour l`oeuvre que l`Éternel avait ordonnée par Moïse, apportèrent des offrandes volontaires à l`Éternel. 
\verse Moïse dit aux enfants d`Israël: Sachez que l`Éternel a choisi Betsaleel, fils d`Uri, fils de Hur, de la tribu de Juda. 
\verse Il l`a rempli de l`Esprit de Dieu, de sagesse, d`intelligence, et de savoir pour toutes sortes d`ouvrages. 
\verse Il l`a rendu capable de faire des inventions, de travailler l`or, l`argent et l`airain, 
\verse de graver les pierres à enchâsser, de travailler le bois, et d`exécuter toute sortes d`ouvrages d`art. 
\verse Il lui a accordé aussi le don d`enseigner, de même qu`à Oholiab, fils d`Ahisamac, de la tribu de Dan. 
\verse Il les a remplis d`intelligence, pour exécuter tous les ouvrages de sculpture et d`art, pour broder et tisser les étoffes teintes en bleu, en pourpre, en cramoisi, et le fin lin, pour faire toute espèce de travaux et d`inventions. 

\chapter
\verse Betsaleel, Oholiab, et tous les hommes habiles, en qui l`Éternel avait mis de la sagesse et de l`intelligence pour savoir et pour faire, exécutèrent les ouvrages destinés au service du sanctuaire, selon tout ce que l`Éternel avait ordonné. 
\verse Moïse appela Betsaleel, Oholiab, et tous les hommes habiles dans l`esprit desquels l`Éternel avait mis de l`intelligence, tous ceux dont le coeur était disposé à s`appliquer à l`oeuvre pour l`exécuter. 
\verse Ils prirent devant Moïse toutes les offrandes qu`avaient apportées les enfants d`Israël pour faire les ouvrages destinés au service du sanctuaire. Chaque matin, on apportait encore à Moïse des offrandes volontaires. 
\verse Alors tous les hommes habiles, occupés à tous les travaux du sanctuaire, quittèrent chacun l`ouvrage qu`ils faisaient, 
\verse et vinrent dire à Moïse: Le peuple apporte beaucoup plus qu`il ne faut pour exécuter les ouvrages que l`Éternel a ordonné de faire. 
\verse Moïse fit publier dans le camp que personne, homme ou femme, ne s`occupât plus d`offrandes pour le sanctuaire. On empêcha ainsi le peuple d`en apporter. 
\verse Les objets préparés suffisaient, et au delà, pour tous les ouvrages à faire. 
\verse Tous les hommes habiles, qui travaillèrent à l`oeuvre, firent le tabernacle avec dix tapis de fin lin retors et de fil bleu, pourpre et cramoisi; on y représenta des chérubins artistement travaillés. 
\verse La longueur d`un tapis était de vingt-huit coudées; et la largeur d`un tapis était de quatre coudées; la mesure était la même pour tous les tapis. 
\verse Cinq de ces tapis furent joints ensemble; les cinq autres furent aussi joints ensemble. 
\verse On fit des lacets bleus au bord du tapis terminant le premier assemblage; on fit de même au bord du tapis terminant le second assemblage. 
\verse On mit cinquante lacets au premier tapis, et l`on mit cinquante lacets au bord du tapis terminant le second assemblage; ces lacets se correspondaient les uns aux autres. 
\verse On fit cinquante agrafes d`or, et l`on joignit les tapis l`un à l`autre avec les agrafes. Et le tabernacle forma un tout. 
\verse On fit des tapis de poil de chèvre, pour servir de tente sur le tabernacle; on fit onze de ces tapis. 
\verse La longueur d`un tapis était de trente coudées, et la largeur d`un tapis était de quatre coudées; la mesure était la même pour les onze tapis. 
\verse On joignit séparément cinq de ces tapis, et les six autres séparément. 
\verse On mit cinquante lacets au bord du tapis terminant un assemblage, et l`on mit cinquante lacets au bord du tapis du second assemblage. 
\verse On fit cinquante agrafes d`airain, pour assembler la tente, afin qu`elle formât un tout. 
\verse On fit pour la tente une couverture de peaux de béliers teintes en rouge, et une couverture de peaux de dauphins, qui devait être mise par-dessus. 
\verse On fit les planches pour le tabernacle; elles étaient de bois d`acacia, placées debout. 
\verse La longueur d`une planche était de dix coudées, et la largeur d`une planche était d`une coudée et demie. 
\verse Il y avait pour chaque planche deux tenons, joints l`un à l`autre; l`on fit de même pour toutes les planches du tabernacle. 
\verse On fit vingt planches pour le tabernacle, du côté du midi. 
\verse On mit quarante bases d`argent sous les vingt planches, deux bases sous chaque planche pour ses deux tenons. 
\verse On fit vingt planches pour le second côté du tabernacle, le côté du nord, 
\verse et leur quarante bases d`argent, deux bases sous chaque planche. 
\verse On fit six planches pour le fond du tabernacle, du côté de l`occident. 
\verse On fit deux planches pour les angles du tabernacle dans le fond; 
\verse elles étaient doubles depuis le bas et bien liées à leur sommet par un anneau; on fit de même pour toutes aux deux angles. 
\verse Il y avait ainsi huit planches, avec leurs bases d`argent, soit seize bases, deux bases sous chaque planche. 
\verse On fit cinq barres de bois d`acacia pour les planches de l`un des côtés du tabernacle, 
\verse cinq barres pour les planches du second côté du tabernacle, et cinq barres pour les planches du côté du tabernacle formant le fond vers l`occident; 
\verse on fit la barre du milieu pour traverser les planches d`une extrémité à l`autre. 
\verse On couvrit d`or les planches, et l`on fit d`or leurs anneaux pour recevoir les barres, et l`on couvrit d`or les barres. 
\verse On fit le voile de fil bleu, pourpre et cramoisi, et de fin lin retors; on le fit artistement travaillé, et l`on y représenta des chérubins. 
\verse On fit pour lui quatre colonnes d`acacia, et on les couvrit d`or; elles avaient des crochets d`or, et l`on fondit pour elles quatre bases d`argent. 
\verse On fit pour l`entrée de la tente un rideau de fil bleu, pourpre et cramoisi, et de fin lin retors; c`était un ouvrage de broderie. 
\verse On fit ses cinq colonnes et leurs crochets, et l`on couvrit d`or leurs chapiteaux et leurs tringles; leurs cinq bases étaient d`airain. 

\chapter
\verse Betsaleel fit l`arche de bois d`acacia; sa longueur était de deux coudées et demie, sa largeur d`une coudée et demie, et sa hauteur d`une coudée et demie. 
\verse Il la couvrit d`or pur en dedans et en dehors, et il y fit une bordure d`or tout autour. 
\verse Il fondit pour elle quatre anneaux d`or, qu`il mit à ses quatre coins, deux anneaux d`un côté et deux anneaux de l`autre côté. 
\verse Il fit des barres de bois d`acacia, et les couvrit d`or. 
\verse Il passa les barres dans les anneaux sur les côtés de l`arche, pour porter l`arche. 
\verse Il fit un propitiatoire d`or pur; sa longueur était de deux coudées et demie, et sa largeur d`une coudée et demie. 
\verse Il fit deux chérubins d`or; il les fit d`or battu, aux deux extrémités du propitiatoire, 
\verse un chérubin à l`une des extrémités, et un chérubin à l`autre extrémité; il fit les chérubins sortant du propitiatoire à ses deux extrémités. 
\verse Les chérubins étendaient les ailes par-dessus, couvrant de leurs ailes le propitiatoire, et se regardant l`un l`autre; les chérubins avaient la face tournée vers le propitiatoire. 
\verse Il fit la table de bois d`acacia, sa longueur était de deux coudées, sa largeur d`une coudée, et sa hauteur d`une coudée et demie. 
\verse Il la couvrit d`or pur, et il y fit une bordure d`or tout autour. 
\verse Il y fit à l`entour un rebord de quatre doigts, sur lequel il mit une bordure d`or tout autour. 
\verse Il fondit pour la table quatre anneaux d`or, et mit les anneaux aux quatre coins, qui étaient à ses quatre pieds. 
\verse Les anneaux étaient près du rebord, et recevaient les barres pour porter la table. 
\verse Il fit les barres de bois d`acacia, et les couvrit d`or; et elles servaient à porter la table. 
\verse Il fit les ustensiles qu`on devait mettre sur la table, ses plats, ses coupes, ses calices et ses tasses pour servir aux libations; il les fit d`or pur. 
\verse Il fit le chandelier d`or pur; il fit le chandelier d`or battu; son pied, sa tige, ses calices, ses pommes et ses fleurs, étaient d`une même pièce. 
\verse Six branches sortaient de ses côtés, trois branches du chandelier de l`un des côtés, et trois branches du chandelier de l`autre côté. 
\verse Il y avait sur une branche trois calices en forme d`amande, avec pommes et fleurs, et sur une autre branche trois calices en forme d`amande, avec pommes et fleurs; il en était de même pour les six branches sortant du chandelier. 
\verse A la tige du chandelier il y avait quatre calices en forme d`amande, avec leurs pommes et leurs fleurs. 
\verse Il y avait une pomme sous deux des branches sortant du chandelier, une pomme sous deux autres branches, et une pomme sous deux autres branches; il en était de même pour les six branches sortant du chandelier. 
\verse Les pommes et les branches du chandelier étaient d`une même pièce; il était tout entier d`or battu, d`or pur. 
\verse Il fit ses sept lampes, ses mouchettes et ses vases à cendre d`or pur. 
\verse Il employa un talent d`or pur, pour faire le chandelier avec tous ses ustensiles. 
\verse Il fit l`autel des parfums de bois d`acacia; sa longueur était d`une coudée et sa largeur d`une coudée; il était carré, et sa hauteur était de deux coudées. Des cornes sortaient de l`autel. 
\verse Il le couvrit d`or pur, le dessus, les côtés tout autour et les cornes, et il y fit une bordure d`or tout autour. 
\verse Il fit au-dessous de la bordure deux anneaux d`or aux deux côtés; il en mit aux deux côtés, pour recevoir les barres qui servaient à le porter. 
\verse Il fit des barres de bois d`acacia, et les couvrit d`or. 
\verse Il fit l`huile pour l`onction sainte, et le parfum odoriférant, pur, composé selon l`art du parfumeur. 

\chapter
\verse Il fit l`autel des holocaustes de bois d`acacia; sa longueur était de cinq coudées, et sa largeur de cinq coudées; il était carré, et sa hauteur était de trois coudées. 
\verse Il fit, aux quatre coins, des cornes qui sortaient de l`autel, et il le couvrit d`airain. 
\verse Il fit tous les ustensiles de l`autel, les cendriers, les pelles, les bassins, les fourchettes et les brasiers; il fit d`airain tous ces ustensiles. 
\verse Il fit pour l`autel une grille d`airain, en forme de treillis, qu`il plaça au-dessous du rebord de l`autel, à partir du bas, jusqu`à la moitié de la hauteur de l`autel. 
\verse Il fondit quatre anneaux, qu`il mit aux quatre coins de la grille d`airain, pour recevoir les barres. 
\verse Il fit les barres de bois d`acacia, et les couvrit d`airain. 
\verse Il passa dans les anneaux aux côtés de l`autel les barres qui servaient à le porter. Il le fit creux, avec des planches. 
\verse Il fit la cuve d`airain, avec sa base d`airain, en employant les miroirs des femmes qui s`assemblaient à l`entrée de la tente d`assignation. 
\verse Il fit le parvis. Du côté du midi, il y avait, pour former le parvis, des toiles de fin lin retors, sur une longueur de cent coudées, 
\verse avec vingt colonnes posant sur vingt bases d`airain; les crochets des colonnes et leurs tringles étaient d`argent. 
\verse Du côté du nord, il y avait cent coudées de toiles, avec vingt colonnes et leurs vingt bases d`airain; les crochets des colonnes et leurs tringles étaient d`argent. 
\verse Du côté de l`occident, il y avait cinquante coudées de toiles, avec dix colonnes et leurs dix bases; les crochets des colonnes et leurs tringles étaient d`argent. 
\verse Du côté de l`orient, sur les cinquante coudées de largeur, 
\verse il y avait, pour une aile, quinze coudées de toiles, avec trois colonnes et leurs trois bases, 
\verse et, pour la seconde aile, qui lui correspondait de l`autre côté de la porte du parvis, quinze coudées de toiles, avec trois colonnes et leurs trois bases. 
\verse Toutes les toiles formant l`enceinte du parvis étaient de fin lin retors. 
\verse Les bases pour les colonnes étaient d`airain, les crochets des colonnes et leurs tringles étaient d`argent, et leurs chapiteaux étaient couverts d`argent. Toutes les colonnes du parvis étaient jointes par des tringles d`argent. 
\verse Le rideau de la porte du parvis était un ouvrage de broderie en fil bleu, pourpre et cramoisi, et en fin lin retors; il avait une longueur de vingt coudées, et sa hauteur était de cinq coudées, comme la largeur des toiles du parvis; 
\verse ses quatre colonnes et leurs quatre bases étaient d`airain, les crochets et leurs tringles étaient d`argent, et leurs chapiteaux étaient couverts d`argent. 
\verse Tous les pieux de l`enceinte du tabernacle et du parvis étaient d`airain. 
\verse Voici les comptes du tabernacle, du tabernacle d`assignation, révisés, d`après l`ordre de Moïse, par les soins des Lévites, sous la direction d`Ithamar, fils du sacrificateur Aaron. 
\verse Betsaleel, fils d`Uri, fils de Hur, de la tribu de Juda, fit tout ce que l`Éternel avait ordonné à Moïse; 
\verse il eut pour aide Oholiab, fils d`Ahisamac, de la tribu de Dan, habile à graver, à inventer, et à broder sur les étoffes teintes en bleu, en pourpre, en cramoisi, et sur le fin lin. 
\verse Le total de l`or employé à l`oeuvre pour tous les travaux du sanctuaire, or qui fut le produit des offrandes, montait à vingt-neuf talents et sept cent trente sicles, selon le sicle du sanctuaire. 
\verse L`argent de ceux de l`assemblée dont on fit le dénombrement montait à cent talents et mille sept cent soixante-quinze sicles, selon le sicle du sanctuaire. 
\verse C`était un demi-sicle par tête, la moitié d`un sicle, selon le sicle du sanctuaire, pour chaque homme compris dans le dénombrement, depuis l`âge de vingt ans et au-dessus, soit pour six cent trois mille cinq cent cinquante hommes. 
\verse Les cent talents d`argent servirent à fondre les bases du sanctuaire et les bases du voile, cent bases pour les cent talents, un talent par base. 
\verse Et avec les mille sept cent soixante-quinze sicles on fit les crochets et les tringles pour les colonnes, et on couvrit les chapiteaux. 
\verse L`airain des offrandes montait à soixante-dix talents et deux mille quatre cents sicles. 
\verse On en fit les bases de l`entrée de la tente d`assignation; l`autel d`airain avec sa grille, et tous les ustensiles de l`autel; 
\verse les bases du parvis, tout autour, et les bases de la porte du parvis; et tous les pieux de l`enceinte du tabernacle et du parvis. 

\chapter
\verse Avec les étoffes teintes en bleu, en pourpre et en cramoisi, on fit les vêtements d`office pour le service dans le sanctuaire, et on fit les vêtements sacrés pour Aaron, comme l`Éternel l`avait ordonné à Moïse. 
\verse On fit l`éphod d`or, de fil bleu, pourpre et cramoisi, et de fin lin retors. 
\verse On étendit des lames d`or, et on les coupa en fils, que l`on entrelaça dans les étoffes teintes en bleu, en pourpre et en cramoisi, et dans le fin lin; il était artistement travaillé. 
\verse On y fit des épaulettes qui le joignaient, et c`est ainsi qu`il était joint par ses deux extrémités. 
\verse La ceinture était du même travail que l`éphod et fixée sur lui; elle était d`or, de fil bleu, pourpre et cramoisi, et de fin lin retors, comme l`Éternel l`avait ordonné à Moïse. 
\verse On entoura de montures d`or des pierres d`onyx, sur lesquelles on grava les noms des fils d`Israël, comme on grave les cachets. 
\verse On les mit sur les épaulettes de l`éphod, en souvenir des fils d`Israël, comme l`Éternel l`avait ordonné à Moïse. 
\verse On fit le pectoral, artistement travaillé, du même travail que l`éphod, d`or, de fil bleu, pourpre et cramoisi, et de fin lin retors. 
\verse Il était carré; on fit le pectoral double: sa longueur était d`un empan, et sa largeur d`un empan; il était double. 
\verse On le garnit de quatre rangées de pierres: première rangée, une sardoine, une topaze, une émeraude; 
\verse seconde rangée, une escarboucle, un saphir, un diamant; 
\verse troisième rangée, une opale, une agate, une améthyste; 
\verse quatrième rangée, une chrysolithe, un onyx, un jaspe. Ces pierres étaient enchâssées dans leurs montures d`or. 
\verse Il y en avait douze, d`après les noms des fils d`Israël; elles étaient gravées comme des cachets, chacune avec le nom de l`une des douze tribus. - 
\verse On fit sur le pectoral des chaînettes d`or pur, tressées en forme de cordons. 
\verse On fit deux montures d`or et deux anneaux d`or, et on mit les deux anneaux aux deux extrémités du pectoral. 
\verse On passa les deux cordons d`or dans les deux anneaux aux deux extrémités du pectoral; 
\verse on arrêta par devant les bouts des deux cordons aux deux montures placées sur les épaulettes de l`éphod. - 
\verse On fit encore deux anneaux d`or, que l`on mit aux deux extrémités du pectoral, sur le bord intérieur appliqué contre l`éphod. 
\verse On fit deux autres anneaux d`or, que l`on mit au bas des deux épaulettes de l`éphod, sur le devant, près de la jointure, au-dessus de la ceinture de l`éphod. 
\verse On attacha le pectoral par ses anneaux aux anneaux de l`éphod avec un cordon bleu, afin que le pectoral fût au-dessus de la ceinture de l`éphod et qu`il ne pût pas se séparer de l`éphod, comme l`Éternel l`avait ordonné à Moïse. 
\verse On fit la robe de l`éphod, tissée entièrement d`étoffe bleue. 
\verse Il y avait, au milieu de la robe, une ouverture comme l`ouverture d`une cotte de mailles, et cette ouverture était bordée tout autour, afin que la robe ne se déchirât pas. 
\verse On mit sur la bordure de la robe des grenades de couleur bleue, pourpre et cramoisi, en fil retors; 
\verse on fit des clochettes d`or pur, et on mit les clochettes entre les grenades, sur tout le tour de la bordure de la robe, entre les grenades: 
\verse une clochette et une grenade, une clochette et une grenade, sur tout le tour de la bordure de la robe, pour le service, comme l`Éternel l`avait ordonné à Moïse. 
\verse On fit les tuniques de fin lin, tissées, pour Aaron et pour ses fils; 
\verse la tiare de fin lin, et les bonnets de fin lin servant de parure; les caleçons de lin, de fin lin retors; 
\verse la ceinture de fin lin retors, brodée, et de couleur bleue, pourpre et cramoisie, comme l`Éternel l`avait ordonné à Moïse. 
\verse On fit d`or pur la lame, diadème sacré, et l`on y écrivit, comme on grave un cachet: Sainteté à l`Éternel. 
\verse On l`attacha avec un cordon bleu à la tiare, en haut, comme l`Éternel l`avait ordonné à Moïse. 
\verse Ainsi furent achevés tous les ouvrages du tabernacle, de la tente d`assignation. Les enfants d`Israël firent tout ce que l`Éternel avait ordonné à Moïse; ils firent ainsi. 
\verse On amena le tabernacle à Moïse: la tente et tout ce qui en dépendait, les agrafes, les planches, les barres, les colonnes et les bases; 
\verse la couverture de peaux de béliers teintes en rouge, la couverture de peaux de dauphins, et le voile de séparation; 
\verse l`arche du témoignage et ses barres, et le propitiatoire; 
\verse la table, tous ses ustensiles, et les pains de proposition; 
\verse le chandelier d`or pur, ses lampes, les lampes préparées, tous ses ustensiles, et l`huile pour le chandelier; 
\verse l`autel d`or, l`huile d`onction et le parfum odoriférant, et le rideau de l`entrée de la tente; 
\verse l`autel d`airain, sa grille d`airain, ses barres, et tous ses ustensiles; la cuve avec sa base; 
\verse les toiles du parvis, ses colonnes, ses bases, et le rideau de la porte du parvis, ses cordages, ses pieux, et tous les ustensiles pour le service du tabernacle, pour la tente d`assignation; 
\verse les vêtements d`office pour le sanctuaire, les vêtements sacrés pour le sacrificateur Aaron, et les vêtements de ses fils pour les fonctions du sacerdoce. 
\verse Les enfants d`Israël firent tous ces ouvrages, en se conformant à tous les ordres que l`Éternel avait donnés à Moïse. 
\verse Moïse examina tout le travail; et voici, ils l`avaient fait comme l`Éternel l`avait ordonné, ils l`avaient fait ainsi. Et Moïse les bénit. 

\chapter
\verse L`Éternel parla à Moïse, et dit: 
\verse Le premier jour du premier mois, tu dresseras le tabernacle, la tente d`assignation. 
\verse Tu y placeras l`arche du témoignage, et tu couvriras l`arche avec le voile. 
\verse Tu apporteras la table, et tu la disposeras en ordre. Tu apporteras le chandelier, et tu en arrangeras les lampes. 
\verse Tu placeras l`autel d`or pour le parfum devant l`arche du témoignage, et tu mettras le rideau à l`entrée du tabernacle. 
\verse Tu placeras l`autel des holocaustes devant l`entrée du tabernacle, de la tente d`assignation. 
\verse Tu placeras la cuve entre la tente d`assignation et l`autel, et tu y mettras de l`eau. 
\verse Tu placeras le parvis à l`entour, et tu mettras le rideau à la porte du parvis. 
\verse Tu prendras l`huile d`onction, tu en oindras le tabernacle et tout ce qu`il renferme, et tu le sanctifieras, avec tous ses ustensiles; et il sera saint. 
\verse Tu oindras l`autel des holocaustes et tous ses ustensiles, et tu sanctifieras l`autel; et l`autel sera très saint. 
\verse Tu oindras la cuve avec sa base, et tu la sanctifieras. 
\verse Tu feras avancer Aaron et ses fils vers l`entrée de la tente d`assignation, et tu les laveras avec de l`eau. 
\verse Tu revêtiras Aaron des vêtements sacrés, tu l`oindras, et tu le sanctifieras, pour qu`il soit à mon service dans le sacerdoce. 
\verse Tu feras approcher ses fils, tu les revêtiras des tuniques, 
\verse et tu les oindras comme tu auras oint leur père, pour qu`ils soient à mon service dans le sacerdoce. Cette onction leur assurera à perpétuité le sacerdoce parmi leurs descendants. 
\verse Moïse fit tout ce que l`Éternel lui avait ordonné; il fit ainsi. 
\verse Le premier jour du premier mois de la seconde année, le tabernacle fut dressé. 
\verse Moïse dressa le tabernacle; il en posa les bases, plaça les planches et les barres, et éleva les colonnes. 
\verse Il étendit la tente sur le tabernacle, et il mit la couverture de la tente par-dessus, comme l`Éternel l`avait ordonné à Moïse. 
\verse Il prit le témoignage, et le plaça dans l`arche; il mit les barres à l`arche, et il posa le propitiatoire au-dessus de l`arche. 
\verse Il apporta l`arche dans le tabernacle; il mit le voile de séparation, et il en couvrit l`arche du témoignage, comme l`Éternel l`avait ordonné à Moïse. 
\verse Il plaça la table dans la tente d`assignation, au côté septentrional du tabernacle, en dehors du voile; 
\verse et il y déposa en ordre les pains, devant l`Éternel, comme l`Éternel l`avait ordonné à Moïse. 
\verse Il plaça le chandelier dans la tente d`assignation, en face de la table, au côté méridional du tabernacle; 
\verse et il en arrangea les lampes, devant l`Éternel, comme l`Éternel l`avait ordonné à Moïse. 
\verse Il plaça l`autel d`or dans la tente d`assignation, devant le voile; 
\verse et il y fit brûler le parfum odoriférant, comme l`Éternel l`avait ordonné à Moïse. 
\verse Il plaça le rideau à l`entrée du tabernacle. 
\verse Il plaça l`autel des holocaustes à l`entrée du tabernacle, de la tente d`assignation; et il y offrit l`holocauste et l`offrande, comme l`Éternel l`avait ordonné à Moïse. 
\verse Il plaça la cuve entre la tente d`assignation et l`autel, et il y mit de l`eau pour les ablutions; 
\verse Moïse, Aaron et ses fils, s`y lavèrent les mains et les pieds; 
\verse lorsqu`ils entrèrent dans la tente d`assignation et qu`ils s`approchèrent de l`autel, ils se lavèrent, comme l`Éternel l`avait ordonné à Moïse. 
\verse Il dressa le parvis autour du tabernacle et de l`autel, et il mit le rideau à la porte du parvis. Ce fut ainsi que Moïse acheva l`ouvrage. 
\verse Alors la nuée couvrit la tente d`assignation, et la gloire de l`Éternel remplit le tabernacle. 
\verse Moïse ne pouvait pas entrer dans la tente d`assignation, parce que la nuée restait dessus, et que la gloire de l`Éternel remplissait le tabernacle. 
\verse Aussi longtemps que durèrent leurs marches, les enfants d`Israël partaient, quand la nuée s`élevait de dessus le tabernacle. 
\verse Et quand la nuée ne s`élevait pas, ils ne partaient pas, jusqu`à ce qu`elle s`élevât. 
\verse La nuée de l`Éternel était de jour sur le tabernacle; et de nuit, il y avait un feu, aux yeux de toute la maison d`Israël, pendant toutes leurs marches. 
