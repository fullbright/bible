\book[b.GEN]{b.gen}


\chapter
\verse Au commencement, Dieu créa les cieux et la terre. 
\verse La terre était informe et vide: il y avait des ténèbres à la surface de l`abîme, et l`esprit de Dieu se mouvait au-dessus des eaux. 
\verse Dieu dit: Que la lumière soit! Et la lumière fut. 
\verse Dieu vit que la lumière était bonne; et Dieu sépara la lumière d`avec les ténèbres. 
\verse Dieu appela la lumière jour, et il appela les ténèbres nuit. Ainsi, il y eut un soir, et il y eut un matin: ce fut le premier jour. 
\verse Dieu dit: Qu`il y ait une étendue entre les eaux, et qu`elle sépare les eaux d`avec les eaux. 
\verse Et Dieu fit l`étendue, et il sépara les eaux qui sont au-dessous de l`étendue d`avec les eaux qui sont au-dessus de l`étendue. Et cela fut ainsi. 
\verse Dieu appela l`étendue ciel. Ainsi, il y eut un soir, et il y eut un matin: ce fut le second jour. 
\verse Dieu dit: Que les eaux qui sont au-dessous du ciel se rassemblent en un seul lieu, et que le sec paraisse. Et cela fut ainsi. 
\verse Dieu appela le sec terre, et il appela l`amas des eaux mers. Dieu vit que cela était bon. 
\verse Puis Dieu dit: Que la terre produise de la verdure, de l`herbe portant de la semence, des arbres fruitiers donnant du fruit selon leur espèce et ayant en eux leur semence sur la terre. Et cela fut ainsi. 
\verse La terre produisit de la verdure, de l`herbe portant de la semence selon son espèce, et des arbres donnant du fruit et ayant en eux leur semence selon leur espèce. Dieu vit que cela était bon. 
\verse Ainsi, il y eut un soir, et il y eut un matin: ce fut le troisième jour. 
\verse Dieu dit: Qu`il y ait des luminaires dans l`étendue du ciel, pour séparer le jour d`avec la nuit; que ce soient des signes pour marquer les époques, les jours et les années; 
\verse et qu`ils servent de luminaires dans l`étendue du ciel, pour éclairer la terre. Et cela fut ainsi. 
\verse Dieu fit les deux grands luminaires, le plus grand luminaire pour présider au jour, et le plus petit luminaire pour présider à la nuit; il fit aussi les étoiles. 
\verse Dieu les plaça dans l`étendue du ciel, pour éclairer la terre, 
\verse pour présider au jour et à la nuit, et pour séparer la lumière d`avec les ténèbres. Dieu vit que cela était bon. 
\verse Ainsi, il y eut un soir, et il y eut un matin: ce fut le quatrième jour. 
\verse Dieu dit: Que les eaux produisent en abondance des animaux vivants, et que des oiseaux volent sur la terre vers l`étendue du ciel. 
\verse Dieu créa les grands poissons et tous les animaux vivants qui se meuvent, et que les eaux produisirent en abondance selon leur espèce; il créa aussi tout oiseau ailé selon son espèce. Dieu vit que cela était bon. 
\verse Dieu les bénit, en disant: Soyez féconds, multipliez, et remplissez les eaux des mers; et que les oiseaux multiplient sur la terre. 
\verse Ainsi, il y eut un soir, et il y eut un matin: ce fut le cinquième jour. 
\verse Dieu dit: Que la terre produise des animaux vivants selon leur espèce, du bétail, des reptiles et des animaux terrestres, selon leur espèce. Et cela fut ainsi. 
\verse Dieu fit les animaux de la terre selon leur espèce, le bétail selon son espèce, et tous les reptiles de la terre selon leur espèce. Dieu vit que cela était bon. 
\verse Puis Dieu dit: Faisons l`homme à notre image, selon notre ressemblance, et qu`il domine sur les poissons de la mer, sur les oiseaux du ciel, sur le bétail, sur toute la terre, et sur tous les reptiles qui rampent sur la terre. 
\verse Dieu créa l`homme à son image, il le créa à l`image de Dieu, il créa l`homme et la femme. 
\verse Dieu les bénit, et Dieu leur dit: Soyez féconds, multipliez, remplissez la terre, et l`assujettissez; et dominez sur les poissons de la mer, sur les oiseaux du ciel, et sur tout animal qui se meut sur la terre. 
\verse Et Dieu dit: Voici, je vous donne toute herbe portant de la semence et qui est à la surface de toute la terre, et tout arbre ayant en lui du fruit d`arbre et portant de la semence: ce sera votre nourriture. 
\verse Et à tout animal de la terre, à tout oiseau du ciel, et à tout ce qui se meut sur la terre, ayant en soi un souffle de vie, je donne toute herbe verte pour nourriture. Et cela fut ainsi. 
\verse Dieu vit tout ce qu`il avait fait et voici, cela était très bon. Ainsi, il y eut un soir, et il y eut un matin: ce fut le sixième jour. 

\chapter
\verse Ainsi furent achevés les cieux et la terre, et toute leur armée. 
\verse Dieu acheva au septième jour son oeuvre, qu`il avait faite: et il se reposa au septième jour de toute son oeuvre, qu`il avait faite. 
\verse Dieu bénit le septième jour, et il le sanctifia, parce qu`en ce jour il se reposa de toute son oeuvre qu`il avait créée en la faisant. 
\verse Voici les origines des cieux et de la terre, quand ils furent créés. 
\verse Lorsque l`Éternel Dieu fit une terre et des cieux, aucun arbuste des champs n`était encore sur la terre, et aucune herbe des champs ne germait encore: car l`Éternel Dieu n`avait pas fait pleuvoir sur la terre, et il n`y avait point d`homme pour cultiver le sol. 
\verse Mais une vapeur s`éleva de la terre, et arrosa toute la surface du sol. 
\verse L`Éternel Dieu forma l`homme de la poussière de la terre, il souffla dans ses narines un souffle de vie et l`homme devint un être vivant. 
\verse Puis l`Éternel Dieu planta un jardin en Éden, du côté de l`orient, et il y mit l`homme qu`il avait formé. 
\verse L`Éternel Dieu fit pousser du sol des arbres de toute espèce, agréables à voir et bons à manger, et l`arbre de la vie au milieu du jardin, et l`arbre de la connaissance du bien et du mal. 
\verse Un fleuve sortait d`Éden pour arroser le jardin, et de là il se divisait en quatre bras. 
\verse Le nom du premier est Pischon; c`est celui qui entoure tout le pays de Havila, où se trouve l`or. 
\verse L`or de ce pays est pur; on y trouve aussi le bdellium et la pierre d`onyx. 
\verse Le nom du second fleuve est Guihon; c`est celui qui entoure tout le pays de Cusch. 
\verse Le nom du troisième est Hiddékel; c`est celui qui coule à l`orient de l`Assyrie. Le quatrième fleuve, c`est l`Euphrate. 
\verse L`Éternel Dieu prit l`homme, et le plaça dans le jardin d`Éden pour le cultiver et pour le garder. 
\verse L`Éternel Dieu donna cet ordre à l`homme: Tu pourras manger de tous les arbres du jardin; 
\verse mais tu ne mangeras pas de l`arbre de la connaissance du bien et du mal, car le jour où tu en mangeras, tu mourras. 
\verse L`Éternel Dieu dit: Il n`est pas bon que l`homme soit seul; je lui ferai une aide semblable à lui. 
\verse L`Éternel Dieu forma de la terre tous les animaux des champs et tous les oiseaux du ciel, et il les fit venir vers l`homme, pour voir comment il les appellerait, et afin que tout être vivant portât le nom que lui donnerait l`homme. 
\verse Et l`homme donna des noms à tout le bétail, aux oiseaux du ciel et à tous les animaux des champs; mais, pour l`homme, il ne trouva point d`aide semblable à lui. 
\verse Alors l`Éternel Dieu fit tomber un profond sommeil sur l`homme, qui s`endormit; il prit une de ses côtes, et referma la chair à sa place. 
\verse L`Éternel Dieu forma une femme de la côte qu`il avait prise de l`homme, et il l`amena vers l`homme. 
\verse Et l`homme dit: Voici cette fois celle qui est os de mes os et chair de ma chair! on l`appellera femme, parce qu`elle a été prise de l`homme. 
\verse C`est pourquoi l`homme quittera son père et sa mère, et s`attachera à sa femme, et ils deviendront une seule chair. 
\verse L`homme et sa femme étaient tous deux nus, et ils n`en avaient point honte. 

\chapter
\verse Le serpent était le plus rusé de tous les animaux des champs, que l`Éternel Dieu avait faits. Il dit à la femme: Dieu a-t-il réellement dit: Vous ne mangerez pas de tous les arbres du jardin? 
\verse La femme répondit au serpent: Nous mangeons du fruit des arbres du jardin. 
\verse Mais quant au fruit de l`arbre qui est au milieu du jardin, Dieu a dit: Vous n`en mangerez point et vous n`y toucherez point, de peur que vous ne mouriez. 
\verse Alors le serpent dit à la femme: Vous ne mourrez point; 
\verse mais Dieu sait que, le jour où vous en mangerez, vos yeux s`ouvriront, et que vous serez comme des dieux, connaissant le bien et le mal. 
\verse La femme vit que l`arbre était bon à manger et agréable à la vue, et qu`il était précieux pour ouvrir l`intelligence; elle prit de son fruit, et en mangea; elle en donna aussi à son mari, qui était auprès d`elle, et il en mangea. 
\verse Les yeux de l`un et de l`autre s`ouvrirent, ils connurent qu`ils étaient nus, et ayant cousu des feuilles de figuier, ils s`en firent des ceintures. 
\verse Alors ils entendirent la voix de l`Éternel Dieu, qui parcourait le jardin vers le soir, et l`homme et sa femme se cachèrent loin de la face de l`Éternel Dieu, au milieu des arbres du jardin. 
\verse Mais l`Éternel Dieu appela l`homme, et lui dit: Où es-tu? 
\verse Il répondit: J`ai entendu ta voix dans le jardin, et j`ai eu peur, parce que je suis nu, et je me suis caché. 
\verse Et l`Éternel Dieu dit: Qui t`a appris que tu es nu? Est-ce que tu as mangé de l`arbre dont je t`avais défendu de manger? 
\verse L`homme répondit: La femme que tu as mise auprès de moi m`a donné de l`arbre, et j`en ai mangé. 
\verse Et l`Éternel Dieu dit à la femme: Pourquoi as-tu fait cela? La femme répondit: Le serpent m`a séduite, et j`en ai mangé. 
\verse L`Éternel Dieu dit au serpent: Puisque tu as fait cela, tu seras maudit entre tout le bétail et entre tous les animaux des champs, tu marcheras sur ton ventre, et tu mangeras de la poussière tous les jours de ta vie. 
\verse Je mettrai inimitié entre toi et la femme, entre ta postérité et sa postérité: celle-ci t`écrasera la tête, et tu lui blesseras le talon. 
\verse Il dit à la femme: J`augmenterai la souffrance de tes grossesses, tu enfanteras avec douleur, et tes désirs se porteront vers ton mari, mais il dominera sur toi. 
\verse Il dit à l`homme: Puisque tu as écouté la voix de ta femme, et que tu as mangé de l`arbre au sujet duquel je t`avais donné cet ordre: Tu n`en mangeras point! le sol sera maudit à cause de toi. C`est à force de peine que tu en tireras ta nourriture tous les jours de ta vie, 
\verse il te produira des épines et des ronces, et tu mangeras de l`herbe des champs. 
\verse C`est à la sueur de ton visage que tu mangeras du pain, jusqu`à ce que tu retournes dans la terre, d`où tu as été pris; car tu es poussière, et tu retourneras dans la poussière. 
\verse Adam donna à sa femme le nom d`Eve: car elle a été la mère de tous les vivants. 
\verse L`Éternel Dieu fit à Adam et à sa femme des habits de peau, et il les en revêtit. 
\verse L`Éternel Dieu dit: Voici, l`homme est devenu comme l`un de nous, pour la connaissance du bien et du mal. Empêchons-le maintenant d`avancer sa main, de prendre de l`arbre de vie, d`en manger, et de vivre éternellement. 
\verse Et l`Éternel Dieu le chassa du jardin d`Éden, pour qu`il cultivât la terre, d`où il avait été pris. 
\verse C`est ainsi qu`il chassa Adam; et il mit à l`orient du jardin d`Éden les chérubins qui agitent une épée flamboyante, pour garder le chemin de l`arbre de vie. 

\chapter
\verse Adam connut Eve, sa femme; elle conçut, et enfanta Caïn et elle dit: J`ai formé un homme avec l`aide de l`Éternel. 
\verse Elle enfanta encore son frère Abel. Abel fut berger, et Caïn fut laboureur. 
\verse Au bout de quelque temps, Caïn fit à l`Éternel une offrande des fruits de la terre; 
\verse et Abel, de son côté, en fit une des premiers-nés de son troupeau et de leur graisse. L`Éternel porta un regard favorable sur Abel et sur son offrande; 
\verse mais il ne porta pas un regard favorable sur Caïn et sur son offrande. Caïn fut très irrité, et son visage fut abattu. 
\verse Et l`Éternel dit à Caïn: Pourquoi es-tu irrité, et pourquoi ton visage est-il abattu? 
\verse Certainement, si tu agis bien, tu relèveras ton visage, et si tu agis mal, le péché se couche à la porte, et ses désirs se portent vers toi: mais toi, domine sur lui. 
\verse Cependant, Caïn adressa la parole à son frère Abel; mais, comme ils étaient dans les champs, Caïn se jeta sur son frère Abel, et le tua. 
\verse L`Éternel dit à Caïn: Où est ton frère Abel? Il répondit: Je ne sais pas; suis-je le gardien de mon frère? 
\verse Et Dieu dit: Qu`as-tu fait? La voix du sang de ton frère crie de la terre jusqu`à moi. 
\verse Maintenant, tu seras maudit de la terre qui a ouvert sa bouche pour recevoir de ta main le sang de ton frère. 
\verse Quand tu cultiveras le sol, il ne te donnera plus sa richesse. Tu seras errant et vagabond sur la terre. 
\verse Caïn dit à l`Éternel: Mon châtiment est trop grand pour être supporté. 
\verse Voici, tu me chasses aujourd`hui de cette terre; je serai caché loin de ta face, je serai errant et vagabond sur la terre, et quiconque me trouvera me tuera. 
\verse L`Éternel lui dit: Si quelqu`un tuait Caïn, Caïn serait vengé sept fois. Et l`Éternel mit un signe sur Caïn pour que quiconque le trouverait ne le tuât point. 
\verse Puis, Caïn s`éloigna de la face de l`Éternel, et habita dans la terre de Nod, à l`orient d`Éden. 
\verse Caïn connut sa femme; elle conçut, et enfanta Hénoc. Il bâtit ensuite une ville, et il donna à cette ville le nom de son fils Hénoc. 
\verse Hénoc engendra Irad, Irad engendra Mehujaël, Mehujaël engendra Metuschaël, et Metuschaël engendra Lémec. 
\verse Lémec prit deux femmes: le nom de l`une était Ada, et le nom de l`autre Tsilla. 
\verse Ada enfanta Jabal: il fut le père de ceux qui habitent sous des tentes et près des troupeaux. 
\verse Le nom de son frère était Jubal: il fut le père de tous ceux qui jouent de la harpe et du chalumeau. 
\verse Tsilla, de son côté, enfanta Tubal Caïn, qui forgeait tous les instruments d`airain et de fer. La soeur de Tubal Caïn était Naama. 
\verse Lémec dit à ses femmes: Ada et Tsilla, écoutez ma voix! Femmes de Lémec, écoutez ma parole! J`ai tué un homme pour ma blessure, Et un jeune homme pour ma meurtrissure. 
\verse Caïn sera vengé sept fois, Et Lémec soixante-dix-sept fois. 
\verse Adam connut encore sa femme; elle enfanta un fils, et l`appela du nom de Seth, car, dit-elle, Dieu m`a donnée un autre fils à la place d`Abel, que Caïn a tué. 
\verse Seth eut aussi un fils, et il l`appela du nom d`Énosch. C`est alors que l`on commença à invoquer le nom de l`Éternel. 

\chapter
\verse Voici le livre de la postérité d`Adam. Lorsque Dieu créa l`homme, il le fit à la ressemblance de Dieu. 
\verse Il créa l`homme et la femme, il les bénit, et il les appela du nom d`homme, lorsqu`ils furent créés. 
\verse Adam, âgé de cent trente ans, engendra un fils à sa ressemblance, selon son image, et il lui donna le nom de Seth. 
\verse Les jours d`Adam, après la naissance de Seth, furent de huit cents ans; et il engendra des fils et des filles. 
\verse Tous les jours qu`Adam vécut furent de neuf cent trente ans; puis il mourut. 
\verse Seth, âgé de cent cinq ans, engendra Énosch. 
\verse Seth vécut, après la naissance d`Énosch, huit cent sept ans; et il engendra des fils et des filles. 
\verse Tous les jours de Seth furent de neuf cent douze ans; puis il mourut. 
\verse Énosch, âgé de quatre-vingt-dix ans, engendra Kénan. 
\verse Énosch vécut, après la naissance de Kénan, huit cent quinze ans; et il engendra des fils et des filles. 
\verse Tous les jours d`Énosch furent de neuf cent cinq ans; puis il mourut. 
\verse Kénan, âgé de soixante-dix ans, engendra Mahalaleel. 
\verse Kénan vécut, après la naissance de Mahalaleel, huit cent quarante ans; et il engendra des fils et des filles. 
\verse Tous les jours de Kénan furent de neuf cent dix ans; puis il mourut. 
\verse Mahalaleel, âgé de soixante-cinq ans, engendra Jéred. 
\verse Mahalaleel vécut, après la naissance de Jéred, huit cent trente ans; et il engendra des fils et des filles. 
\verse Tous les jours de Mahalaleel furent de huit cent quatre-vingt-quinze ans; puis il mourut. 
\verse Jéred, âgé de cent soixante-deux ans, engendra Hénoc. 
\verse Jéred vécut, après la naissance d`Hénoc, huit cents ans; et il engendra des fils et des filles. 
\verse Tous les jours de Jéred furent de neuf cent soixante-deux ans; puis il mourut. 
\verse Hénoc, âgé de soixante-cinq ans, engendra Metuschélah. 
\verse Hénoc, après la naissance de Metuschélah, marcha avec Dieu trois cents ans; et il engendra des fils et des filles. 
\verse Tous les jours d`Hénoc furent de trois cent soixante-cinq ans. 
\verse Hénoc marcha avec Dieu; puis il ne fut plus, parce que Dieu le prit. 
\verse Metuschélah, âgé de cent quatre-vingt-sept ans, engendra Lémec. 
\verse Metuschélah vécut, après la naissance de Lémec, sept cent quatre-vingt deux ans; et il engendra des fils et des filles. 
\verse Tous les jours de Metuschélah furent de neuf cent soixante-neuf ans; puis il mourut. 
\verse Lémec, âgé de cent quatre-vingt-deux ans, engendra un fils. 
\verse Il lui donna le nom de Noé, en disant: Celui-ci nous consolera de nos fatigues et du travail pénible de nos mains, provenant de cette terre que l`Éternel a maudite. 
\verse Lémec vécut, après la naissance de Noé, cinq cent quatre-vingt-quinze ans; et il engendra des fils et des filles. 
\verse Tous les jours de Lémec furent de sept cent soixante-dix sept ans; puis il mourut. 
\verse Noé, âgé de cinq cents ans, engendra Sem, Cham et Japhet. 

\chapter
\verse Lorsque les hommes eurent commencé à se multiplier sur la face de la terre, et que des filles leur furent nées, 
\verse les fils de Dieu virent que les filles des hommes étaient belles, et ils en prirent pour femmes parmi toutes celles qu`ils choisirent. 
\verse Alors l`Éternel dit: Mon esprit ne restera pas à toujours dans l`homme, car l`homme n`est que chair, et ses jours seront de cent vingt ans. 
\verse Les géants étaient sur la terre en ces temps-là, après que les fils de Dieu furent venus vers les filles des hommes, et qu`elles leur eurent donné des enfants: ce sont ces héros qui furent fameux dans l`antiquité. 
\verse L`Éternel vit que la méchanceté des hommes était grande sur la terre, et que toutes les pensées de leur coeur se portaient chaque jour uniquement vers le mal. 
\verse L`Éternel se repentit d`avoir fait l`homme sur la terre, et il fut affligé en son coeur. 
\verse Et l`Éternel dit: J`exterminerai de la face de la terre l`homme que j`ai créé, depuis l`homme jusqu`au bétail, aux reptiles, et aux oiseaux du ciel; car je me repens de les avoir faits. 
\verse Mais Noé trouva grâce aux yeux de l`Éternel. 
\verse Voici la postérité de Noé. Noé était un homme juste et intègre dans son temps; Noé marchait avec Dieu. 
\verse Noé engendra trois fils: Sem, Cham et Japhet. 
\verse La terre était corrompue devant Dieu, la terre était pleine de violence. 
\verse Dieu regarda la terre, et voici, elle était corrompue; car toute chair avait corrompu sa voie sur la terre. 
\verse Alors Dieu dit à Noé: La fin de toute chair est arrêtée par devers moi; car ils ont rempli la terre de violence; voici, je vais les détruire avec la terre. 
\verse Fais-toi une arche de bois de gopher; tu disposeras cette arche en cellules, et tu l`enduiras de poix en dedans et en dehors. 
\verse Voici comment tu la feras: l`arche aura trois cents coudées de longueur, cinquante coudées de largeur et trente coudées de hauteur. 
\verse Tu feras à l`arche une fenêtre, que tu réduiras à une coudée en haut; tu établiras une porte sur le côté de l`arche; et tu construiras un étage inférieur, un second et un troisième. 
\verse Et moi, je vais faire venir le déluge d`eaux sur la terre, pour détruire toute chair ayant souffle de vie sous le ciel; tout ce qui est sur la terre périra. 
\verse Mais j`établis mon alliance avec toi; tu entreras dans l`arche, toi et tes fils, ta femme et les femmes de tes fils avec toi. 
\verse De tout ce qui vit, de toute chair, tu feras entrer dans l`arche deux de chaque espèce, pour les conserver en vie avec toi: il y aura un mâle et une femelle. 
\verse Des oiseaux selon leur espèce, du bétail selon son espèce, et de tous les reptiles de la terre selon leur espèce, deux de chaque espèce viendront vers toi, pour que tu leur conserves la vie. 
\verse Et toi, prends de tous les aliments que l`on mange, et fais-en une provision auprès de toi, afin qu`ils te servent de nourriture ainsi qu`à eux. 
\verse C`est ce que fit Noé: il exécuta tout ce que Dieu lui avait ordonné. 

\chapter
\verse L`Éternel dit à Noé: Entre dans l`arche, toi et toute ta maison; car je t`ai vu juste devant moi parmi cette génération. 
\verse Tu prendras auprès de toi sept couples de tous les animaux purs, le mâle et sa femelle; une paire des animaux qui ne sont pas purs, le mâle et sa femelle; 
\verse sept couples aussi des oiseaux du ciel, mâle et femelle, afin de conserver leur race en vie sur la face de toute la terre. 
\verse Car, encore sept jours, et je ferai pleuvoir sur la terre quarante jours et quarante nuits, et j`exterminerai de la face de la terre tous les êtres que j`ai faits. 
\verse Noé exécuta tout ce que l`Éternel lui avait ordonné. 
\verse Noé avait six cents ans, lorsque le déluge d`eaux fut sur la terre. 
\verse Et Noé entra dans l`arche avec ses fils, sa femme et les femmes de ses fils, pour échapper aux eaux du déluge. 
\verse D`entre les animaux purs et les animaux qui ne sont pas purs, les oiseaux et tout ce qui se meut sur la terre, 
\verse il entra dans l`arche auprès de Noé, deux à deux, un mâle et une femelle, comme Dieu l`avait ordonné à Noé. 
\verse Sept jours après, les eaux du déluge furent sur la terre. 
\verse L`an six cent de la vie de Noé, le second mois, le dix-septième jour du mois, en ce jour-là toutes les sources du grand abîme jaillirent, et les écluses des cieux s`ouvrirent. 
\verse La pluie tomba sur la terre quarante jours et quarante nuits. 
\verse Ce même jour entrèrent dans l`arche Noé, Sem, Cham et Japhet, fils de Noé, la femme de Noé et les trois femmes de ses fils avec eux: 
\verse eux, et tous les animaux selon leur espèce, tout le bétail selon son espèce, tous les reptiles qui rampent sur la terre selon leur espèce, tous les oiseaux selon leur espèce, tous les petits oiseaux, tout ce qui a des ailes. 
\verse Ils entrèrent dans l`arche auprès de Noé, deux à deux, de toute chair ayant souffle de vie. 
\verse Il en entra, mâle et femelle, de toute chair, comme Dieu l`avait ordonné à Noé. Puis l`Éternel ferma la porte sur lui. 
\verse Le déluge fut quarante jours sur la terre. Les eaux crûrent et soulevèrent l`arche, et elle s`éleva au-dessus de la terre. 
\verse Les eaux grossirent et s`accrurent beaucoup sur la terre, et l`arche flotta sur la surface des eaux. 
\verse Les eaux grossirent de plus en plus, et toutes les hautes montagnes qui sont sous le ciel entier furent couvertes. 
\verse Les eaux s`élevèrent de quinze coudées au-dessus des montagnes, qui furent couvertes. 
\verse Tout ce qui se mouvait sur la terre périt, tant les oiseaux que le bétail et les animaux, tout ce qui rampait sur la terre, et tous les hommes. 
\verse Tout ce qui avait respiration, souffle de vie dans ses narines, et qui était sur la terre sèche, mourut. 
\verse Tous les êtres qui étaient sur la face de la terre furent exterminés, depuis l`homme jusqu`au bétail, aux reptiles et aux oiseaux du ciel: ils furent exterminés de la terre. Il ne resta que Noé, et ce qui était avec lui dans l`arche. 
\verse Les eaux furent grosses sur la terre pendant cent cinquante jours. 

\chapter
\verse Dieu se souvint de Noé, de tous les animaux et de tout le bétail qui étaient avec lui dans l`arche; et Dieu fit passer un vent sur la terre, et les eaux s`apaisèrent. 
\verse Les sources de l`abîme et les écluses des cieux furent fermées, et la pluie ne tomba plus du ciel. 
\verse Les eaux se retirèrent de dessus la terre, s`en allant et s`éloignant, et les eaux diminuèrent au bout de cent cinquante jours. 
\verse Le septième mois, le dix-septième jour du mois, l`arche s`arrêta sur les montagnes d`Ararat. 
\verse Les eaux allèrent en diminuant jusqu`au dixième mois. Le dixième mois, le premier jour du mois, apparurent les sommets des montagnes. 
\verse Au bout de quarante jours, Noé ouvrit la fenêtre qu`il avait faite à l`arche. 
\verse Il lâcha le corbeau, qui sortit, partant et revenant, jusqu`à ce que les eaux eussent séché sur la terre. 
\verse Il lâcha aussi la colombe, pour voir si les eaux avaient diminué à la surface de la terre. 
\verse Mais la colombe ne trouva aucun lieu pour poser la plante de son pied, et elle revint à lui dans l`arche, car il y avait des eaux à la surface de toute la terre. Il avança la main, la prit, et la fit rentrer auprès de lui dans l`arche. 
\verse Il attendit encore sept autres jours, et il lâcha de nouveau la colombe hors de l`arche. 
\verse La colombe revint à lui sur le soir; et voici, une feuille d`olivier arrachée était dans son bec. Noé connut ainsi que les eaux avaient diminué sur la terre. 
\verse Il attendit encore sept autres jours; et il lâcha la colombe. Mais elle ne revint plus à lui. 
\verse L`an six cent un, le premier mois, le premier jour du mois, les eaux avaient séché sur la terre. Noé ôta la couverture de l`arche: il regarda, et voici, la surface de la terre avait séché. 
\verse Le second mois, le vingt-septième jour du mois, la terre fut sèche. 
\verse Alors Dieu parla à Noé, en disant: 
\verse Sors de l`arche, toi et ta femme, tes fils et les femmes de tes fils avec toi. 
\verse Fais sortir avec toi tous les animaux de toute chair qui sont avec toi, tant les oiseaux que le bétail et tous les reptiles qui rampent sur la terre: qu`ils se répandent sur la terre, qu`ils soient féconds et multiplient sur la terre. 
\verse Et Noé sortit, avec ses fils, sa femme, et les femmes de ses fils. 
\verse Tous les animaux, tous les reptiles, tous les oiseaux, tout ce qui se meut sur la terre, selon leurs espèces, sortirent de l`arche. 
\verse Noé bâtit un autel à l`Éternel; il prit de toutes les bêtes pures et de tous les oiseaux purs, et il offrit des holocaustes sur l`autel. 
\verse L`Éternel sentit une odeur agréable, et l`Éternel dit en son coeur: Je ne maudirai plus la terre, à cause de l`homme, parce que les pensées du coeur de l`homme sont mauvaises dès sa jeunesse; et je ne frapperai plus tout ce qui est vivant, comme je l`ai fait. 
\verse Tant que la terre subsistera, les semailles et la moisson, le froid et la chaleur, l`été et l`hiver, le jour et la nuit ne cesseront point. 

\chapter
\verse Dieu bénit Noé et ses fils, et leur dit: Soyez féconds, multipliez, et remplissez la terre. 
\verse Vous serez un sujet de crainte et d`effroi pour tout animal de la terre, pour tout oiseau du ciel, pour tout ce qui se meut sur la terre, et pour tous les poissons de la mer: ils sont livrés entre vos mains. 
\verse Tout ce qui se meut et qui a vie vous servira de nourriture: je vous donne tout cela comme l`herbe verte. 
\verse Seulement, vous ne mangerez point de chair avec son âme, avec son sang. 
\verse Sachez-le aussi, je redemanderai le sang de vos âmes, je le redemanderai à tout animal; et je redemanderai l`âme de l`homme à l`homme, à l`homme qui est son frère. 
\verse Si quelqu`un verse le sang de l`homme, par l`homme son sang sera versé; car Dieu a fait l`homme à son image. 
\verse Et vous, soyez féconds et multipliez, répandez-vous sur la terre et multipliez sur elle. 
\verse Dieu parla encore à Noé et à ses fils avec lui, en disant: 
\verse Voici, j`établis mon alliance avec vous et avec votre postérité après vous; 
\verse avec tous les êtres vivants qui sont avec vous, tant les oiseaux que le bétail et tous les animaux de la terre, soit avec tous ceux qui sont sortis de l`arche, soit avec tous les animaux de la terre. 
\verse J`établis mon alliance avec vous: aucune chair ne sera plus exterminée par les eaux du déluge, et il n`y aura plus de déluge pour détruire la terre. 
\verse Et Dieu dit: C`est ici le signe de l`alliance que j`établis entre moi et vous, et tous les êtres vivants qui sont avec vous, pour les générations à toujours: 
\verse j`ai placé mon arc dans la nue, et il servira de signe d`alliance entre moi et la terre. 
\verse Quand j`aurai rassemblé des nuages au-dessus de la terre, l`arc paraîtra dans la nue; 
\verse et je me souviendrai de mon alliance entre moi et vous, et tous les êtres vivants, de toute chair, et les eaux ne deviendront plus un déluge pour détruire toute chair. 
\verse L`arc sera dans la nue; et je le regarderai, pour me souvenir de l`alliance perpétuelle entre Dieu et tous les êtres vivants, de toute chair qui est sur la terre. 
\verse Et Dieu dit à Noé: Tel est le signe de l`alliance que j`établis entre moi et toute chair qui est sur la terre. 
\verse Les fils de Noé, qui sortirent de l`arche, étaient Sem, Cham et Japhet. Cham fut le père de Canaan. 
\verse Ce sont là les trois fils de Noé, et c`est leur postérité qui peupla toute la terre. 
\verse Noé commença à cultiver la terre, et planta de la vigne. 
\verse Il but du vin, s`enivra, et se découvrit au milieu de sa tente. 
\verse Cham, père de Canaan, vit la nudité de son père, et il le rapporta dehors à ses deux frères. 
\verse Alors Sem et Japhet prirent le manteau, le mirent sur leurs épaules, marchèrent à reculons, et couvrirent la nudité de leur père; comme leur visage était détourné, ils ne virent point la nudité de leur père. 
\verse Lorsque Noé se réveilla de son vin, il apprit ce que lui avait fait son fils cadet. 
\verse Et il dit: Maudit soit Canaan! qu`il soit l`esclave des esclaves de ses frères! 
\verse Il dit encore: Béni soit l`Éternel, Dieu de Sem, et que Canaan soit leur esclave! 
\verse Que Dieu étende les possessions de Japhet, qu`il habite dans les tentes de Sem, et que Canaan soit leur esclave! 
\verse Noé vécut, après le déluge, trois cent cinquante ans. 
\verse Tous les jours de Noé furent de neuf cent cinquante ans; puis il mourut. 

\chapter
\verse Voici la postérité des fils de Noé, Sem, Cham et Japhet. Il leur naquit des fils après le déluge. 
\verse Les fils de Japhet furent: Gomer, Magog, Madaï, Javan, Tubal, Méschec et Tiras. 
\verse Les fils de Gomer: Aschkenaz, Riphat et Togarma. 
\verse Les fils de Javan: Élischa, Tarsis, Kittim et Dodanim. 
\verse C`est par eux qu`ont été peuplées les îles des nations selon leurs terres, selon la langue de chacun, selon leurs familles, selon leurs nations. 
\verse Les fils de Cham furent: Cusch, Mitsraïm, Puth et Canaan. 
\verse Les fils de Cusch: Saba, Havila, Sabta, Raema et Sabteca. Les fils de Raema: Séba et Dedan. 
\verse Cusch engendra aussi Nimrod; c`est lui qui commença à être puissant sur la terre. 
\verse Il fut un vaillant chasseur devant l`Éternel; c`est pourquoi l`on dit: Comme Nimrod, vaillant chasseur devant l`Éternel. 
\verse Il régna d`abord sur Babel, Érec, Accad et Calné, au pays de Schinear. 
\verse De ce pays-là sortit Assur; il bâtit Ninive, Rehoboth Hir, Calach, 
\verse et Résen entre Ninive et Calach; c`est la grande ville. 
\verse Mitsraïm engendra les Ludim, les Anamim, les Lehabim, les Naphtuhim, 
\verse les Patrusim, les Casluhim, d`où sont sortis les Philistins, et les Caphtorim. 
\verse Canaan engendra Sidon, son premier-né, et Heth; 
\verse et les Jébusiens, les Amoréens, les Guirgasiens, 
\verse les Héviens, les Arkiens, les Siniens, 
\verse les Arvadiens, les Tsemariens, les Hamathiens. Ensuite, les familles des Cananéens se dispersèrent. 
\verse Les limites des Cananéens allèrent depuis Sidon, du côté de Guérar, jusqu`à Gaza, et du côté de Sodome, de Gomorrhe, d`Adma et de Tseboïm, jusqu`à Léscha. 
\verse Ce sont là les fils de Cham, selon leurs familles, selon leurs langues, selon leurs pays, selon leurs nations. 
\verse Il naquit aussi des fils à Sem, père de tous les fils d`Héber, et frère de Japhet l`aîné. 
\verse Les fils de Sem furent: Élam, Assur, Arpacschad, Lud et Aram. 
\verse Les fils d`Aram: Uts, Hul, Guéter et Masch. 
\verse Arpacschad engendra Schélach; et Schélach engendra Héber. 
\verse Il naquit à Héber deux fils: le nom de l`un était Péleg, parce que de son temps la terre fut partagée, et le nom de son frère était Jokthan. 
\verse Jokthan engendra Almodad, Schéleph, Hatsarmaveth, Jérach, 
\verse Hadoram, Uzal, Dikla, 
\verse Obal, Abimaël, Séba, 
\verse Ophir, Havila et Jobab. Tous ceux-là furent fils de Jokthan. 
\verse Ils habitèrent depuis Méscha, du côté de Sephar, jusqu`à la montagne de l`orient. 
\verse Ce sont là les fils de Sem, selon leurs familles, selon leurs langues, selon leurs pays, selon leurs nations. 
\verse Telles sont les familles des fils de Noé, selon leurs générations, selon leurs nations. Et c`est d`eux que sont sorties les nations qui se sont répandues sur la terre après le déluge. 

\chapter
\verse Toute la terre avait une seule langue et les mêmes mots. 
\verse Comme ils étaient partis de l`orient, ils trouvèrent une plaine au pays de Schinear, et ils y habitèrent. 
\verse Ils se dirent l`un à l`autre: Allons! faisons des briques, et cuisons-les au feu. Et la brique leur servit de pierre, et le bitume leur servit de ciment. 
\verse Ils dirent encore: Allons! bâtissons-nous une ville et une tour dont le sommet touche au ciel, et faisons-nous un nom, afin que nous ne soyons pas dispersés sur la face de toute la terre. 
\verse L`Éternel descendit pour voir la ville et la tour que bâtissaient les fils des hommes. 
\verse Et l`Éternel dit: Voici, ils forment un seul peuple et ont tous une même langue, et c`est là ce qu`ils ont entrepris; maintenant rien ne les empêcherait de faire tout ce qu`ils auraient projeté. 
\verse Allons! descendons, et là confondons leur langage, afin qu`ils n`entendent plus la langue, les uns des autres. 
\verse Et l`Éternel les dispersa loin de là sur la face de toute la terre; et ils cessèrent de bâtir la ville. 
\verse C`est pourquoi on l`appela du nom de Babel, car c`est là que l`Éternel confondit le langage de toute la terre, et c`est de là que l`Éternel les dispersa sur la face de toute la terre. 
\verse Voici la postérité de Sem. Sem, âgé de cent ans, engendra Arpacschad, deux ans après le déluge. 
\verse Sem vécut, après la naissance d`Arpacschad, cinq cents ans; et il engendra des fils et des filles. 
\verse Arpacschad, âgé de trente-cinq ans, engendra Schélach. 
\verse Arpacschad vécut, après la naissance de Schélach, quatre cent trois ans; et il engendra des fils et des filles. 
\verse Schélach, âgé de trente ans, engendra Héber. 
\verse Schélach vécut, après la naissance d`Héber, quatre cent trois ans; et il engendra des fils et des filles. 
\verse Héber, âgé de trente-quatre ans, engendra Péleg. 
\verse Héber vécut, après la naissance de Péleg, quatre cent trente ans; et il engendra des fils et des filles. 
\verse Péleg, âgé de trente ans, engendra Rehu. 
\verse Péleg vécut, après la naissance de Rehu, deux cent neuf ans; et il engendra des fils et des filles. 
\verse Rehu, âgé de trente-deux ans, engendra Serug. 
\verse Rehu vécut, après la naissance de Serug, deux cent sept ans; et il engendra des fils et des filles. 
\verse Serug, âgé de trente ans, engendra Nachor. 
\verse Serug vécut, après la naissance de Nachor, deux cents ans; et il engendra des fils et des filles. 
\verse Nachor, âgé de vingt-neuf ans, engendra Térach. 
\verse Nachor vécut, après la naissance de Térach, cent dix-neuf ans; et il engendra des fils et des filles. 
\verse Térach, âgé de soixante-dix ans, engendra Abram, Nachor et Haran. 
\verse Voici la postérité de Térach. Térach engendra Abram, Nachor et Haran. -Haran engendra Lot. 
\verse Et Haran mourut en présence de Térach, son père, au pays de sa naissance, à Ur en Chaldée. - 
\verse Abram et Nachor prirent des femmes: le nom de la femme d`Abram était Saraï, et le nom de la femme de Nachor était Milca, fille d`Haran, père de Milca et père de Jisca. 
\verse Saraï était stérile: elle n`avait point d`enfants. 
\verse Térach prit Abram, son fils, et Lot, fils d`Haran, fils de son fils, et Saraï, sa belle-fille, femme d`Abram, son fils. Ils sortirent ensemble d`Ur en Chaldée, pour aller au pays de Canaan. Ils vinrent jusqu`à Charan, et ils y habitèrent. 
\verse Les jours de Térach furent de deux cent cinq ans; et Térach mourut à Charan. 

\chapter
\verse L`Éternel dit à Abram: Va-t-en de ton pays, de ta patrie, et de la maison de ton père, dans le pays que je te montrerai. 
\verse Je ferai de toi une grande nation, et je te bénirai; je rendrai ton nom grand, et tu seras une source de bénédiction. 
\verse Je bénirai ceux qui te béniront, et je maudirai ceux qui te maudiront; et toutes les familles de la terre seront bénies en toi. 
\verse Abram partit, comme l`Éternel le lui avait dit, et Lot partit avec lui. Abram était âgé de soixante-quinze ans, lorsqu`il sortit de Charan. 
\verse Abram prit Saraï, sa femme, et Lot, fils de son frère, avec tous les biens qu`ils possédaient et les serviteurs qu`ils avaient acquis à Charan. Ils partirent pour aller dans le pays de Canaan, et ils arrivèrent au pays de Canaan. 
\verse Abram parcourut le pays jusqu`au lieu nommé Sichem, jusqu`aux chênes de Moré. Les Cananéens étaient alors dans le pays. 
\verse L`Éternel apparut à Abram, et dit: Je donnerai ce pays à ta postérité. Et Abram bâtit là un autel à l`Éternel, qui lui était apparu. 
\verse Il se transporta de là vers la montagne, à l`orient de Béthel, et il dressa ses tentes, ayant Béthel à l`occident et Aï à l`orient. Il bâtit encore là un autel à l`Éternel, et il invoqua le nom de l`Éternel. 
\verse Abram continua ses marches, en s`avançant vers le midi. 
\verse Il y eut une famine dans le pays; et Abram descendit en Égypte pour y séjourner, car la famine était grande dans le pays. 
\verse Comme il était près d`entrer en Égypte, il dit à Saraï, sa femme: Voici, je sais que tu es une femme belle de figure. 
\verse Quand les Égyptiens te verront, ils diront: C`est sa femme! Et ils me tueront, et te laisseront la vie. 
\verse Dis, je te prie, que tu es ma soeur, afin que je sois bien traité à cause de toi, et que mon âme vive grâce à toi. 
\verse Lorsque Abram fut arrivé en Égypte, les Égyptiens virent que la femme était fort belle. 
\verse Les grands de Pharaon la virent aussi et la vantèrent à Pharaon; et la femme fut emmenée dans la maison de Pharaon. 
\verse Il traita bien Abram à cause d`elle; et Abram reçut des brebis, des boeufs, des ânes, des serviteurs et des servantes, des ânesses, et des chameaux. 
\verse Mais l`Éternel frappa de grandes plaies Pharaon et sa maison, au sujet de Saraï, femme d`Abram. 
\verse Alors Pharaon appela Abram, et dit: Qu`est-ce que tu m`as fait? Pourquoi ne m`as-tu pas déclaré que c`est ta femme? 
\verse Pourquoi as-tu dit: C`est ma soeur? Aussi l`ai-je prise pour ma femme. Maintenant, voici ta femme, prends-la, et va-t-en! 
\verse Et Pharaon donna ordre à ses gens de le renvoyer, lui et sa femme, avec tout ce qui lui appartenait. 

\chapter
\verse Abram remonta d`Égypte vers le midi, lui, sa femme, et tout ce qui lui appartenait, et Lot avec lui. 
\verse Abram était très riche en troupeaux, en argent et en or. 
\verse Il dirigea ses marches du midi jusqu`à Béthel, jusqu`au lieu où était sa tente au commencement, entre Béthel et Aï, 
\verse au lieu où était l`autel qu`il avait fait précédemment. Et là, Abram invoqua le nom de l`Éternel. 
\verse Lot, qui voyageait avec Abram, avait aussi des brebis, des boeufs et des tentes. 
\verse Et la contrée était insuffisante pour qu`ils demeurassent ensemble, car leurs biens étaient si considérables qu`ils ne pouvaient demeurer ensemble. 
\verse Il y eut querelle entre les bergers des troupeaux d`Abram et les bergers des troupeaux de Lot. Les Cananéens et les Phérésiens habitaient alors dans le pays. 
\verse Abram dit à Lot: Qu`il n`y ait point, je te prie, de dispute entre moi et toi, ni entre mes bergers et tes bergers; car nous sommes frères. 
\verse Tout le pays n`est-il pas devant toi? Sépare-toi donc de moi: si tu vas à gauche, j`irai à droite; si tu vas à droite, j`irai à gauche. 
\verse Lot leva les yeux, et vit toute la plaine du Jourdain, qui était entièrement arrosée. Avant que l`Éternel eût détruit Sodome et Gomorrhe, c`était, jusqu`à Tsoar, comme un jardin de l`Éternel, comme le pays d`Égypte. 
\verse Lot choisit pour lui toute la plaine du Jourdain, et il s`avança vers l`orient. C`est ainsi qu`ils se séparèrent l`un de l`autre. 
\verse Abram habita dans le pays de Canaan; et Lot habita dans les villes de la plaine, et dressa ses tentes jusqu`à Sodome. 
\verse Les gens de Sodome étaient méchants, et de grands pécheurs contre l`Éternel. 
\verse L`Éternel dit à Abram, après que Lot se fut séparé de lui: Lève les yeux, et, du lieu où tu es, regarde vers le nord et le midi, vers l`orient et l`occident; 
\verse car tout le pays que tu vois, je le donnerai à toi et à ta postérité pour toujours. 
\verse Je rendrai ta postérité comme la poussière de la terre, en sorte que, si quelqu`un peut compter la poussière de la terre, ta postérité aussi sera comptée. 
\verse Lève-toi, parcours le pays dans sa longueur et dans sa largeur; car je te le donnerai. 
\verse Abram leva ses tentes, et vint habiter parmi les chênes de Mamré, qui sont près d`Hébron. Et il bâtit là un autel à l`Éternel. 

\chapter
\verse Dans le temps d`Amraphel, roi de Schinear, d`Arjoc, roi d`Ellasar, de Kedorlaomer, roi d`Élam, et de Tideal, roi de Gojim, 
\verse il arriva qu`ils firent la guerre à Béra, roi de Sodome, à Birscha, roi de Gomorrhe, à Schineab, roi d`Adma, à Schémeéber, roi de Tseboïm, et au roi de Béla, qui est Tsoar. 
\verse Ces derniers s`assemblèrent tous dans la vallée de Siddim, qui est la mer Salée. 
\verse Pendant douze ans, ils avaient été soumis à Kedorlaomer; et la treizième année, ils s`étaient révoltés. 
\verse Mais, la quatorzième année, Kedorlaomer et les rois qui étaient avec lui se mirent en marche, et ils battirent les Rephaïm à Aschteroth Karnaïm, les Zuzim à Ham, les Émim à Schavé Kirjathaïm, 
\verse et les Horiens dans leur montagne de Séir, jusqu`au chêne de Paran, qui est près du désert. 
\verse Puis ils s`en retournèrent, vinrent à En Mischpath, qui est Kadès, et battirent les Amalécites sur tout leur territoire, ainsi que les Amoréens établis à Hatsatson Thamar. 
\verse Alors s`avancèrent le roi de Sodome, le roi de Gomorrhe, le roi d`Adma, le roi de Tseboïm, et le roi de Béla, qui est Tsoar; et ils se rangèrent en bataille contre eux, dans la vallée de Siddim, 
\verse contre Kedorlaomer, roi d`Élam, Tideal, roi de Gojim, Amraphel, roi de Schinear, et Arjoc, roi d`Ellasar: quatre rois contre cinq. 
\verse La vallée de Siddim était couverte de puits de bitume; le roi de Sodome et celui de Gomorrhe prirent la fuite, et y tombèrent; le reste s`enfuit vers la montagne. 
\verse Les vainqueurs enlevèrent toutes les richesses de Sodome et de Gomorrhe, et toutes leurs provisions; et ils s`en allèrent. 
\verse Ils enlevèrent aussi, avec ses biens, Lot, fils du frère d`Abram, qui demeurait à Sodome; et ils s`en allèrent. 
\verse Un fuyard vint l`annoncer à Abram, l`Hébreu; celui-ci habitait parmi les chênes de Mamré, l`Amoréen, frère d`Eschcol et frère d`Aner, qui avaient fait alliance avec Abram. 
\verse Dès qu`Abram eut appris que son frère avait été fait prisonnier, il arma trois cent dix-huit de ses plus braves serviteurs, nés dans sa maison, et il poursuivit les rois jusqu`à Dan. 
\verse Il divisa sa troupe, pour les attaquer de nuit, lui et ses serviteurs; il les battit, et les poursuivit jusqu`à Choba, qui est à la gauche de Damas. 
\verse Il ramena toutes les richesses; il ramena aussi Lot, son frère, avec ses biens, ainsi que les femmes et le peuple. 
\verse Après qu`Abram fut revenu vainqueur de Kedorlaomer et des rois qui étaient avec lui, le roi de Sodome sortit à sa rencontre dans la vallée de Schavé, qui est la vallée du roi. 
\verse Melchisédek, roi de Salem, fit apporter du pain et du vin: il était sacrificateur du Dieu Très Haut. 
\verse Il bénit Abram, et dit: Béni soit Abram par le Dieu Très Haut, maître du ciel et de la terre! 
\verse Béni soit le Dieu Très Haut, qui a livré tes ennemis entre tes mains! Et Abram lui donna la dîme de tout. 
\verse Le roi de Sodome dit à Abram: Donne-moi les personnes, et prends pour toi les richesses. 
\verse Abram répondit au roi de Sodome: Je lève la main vers l`Éternel, le Dieu Très Haut, maître du ciel et de la terre: 
\verse je ne prendrai rien de tout ce qui est à toi, pas même un fil, ni un cordon de soulier, afin que tu ne dises pas: J`ai enrichi Abram. Rien pour moi! 
\verse Seulement, ce qu`ont mangé les jeunes gens, et la part des hommes qui ont marché avec moi, Aner, Eschcol et Mamré: eux, ils prendront leur part. 

\chapter
\verse Après ces événements, la parole de l`Éternel fut adressée à Abram dans une vision, et il dit: Abram, ne crains point; je suis ton bouclier, et ta récompense sera très grande. 
\verse Abram répondit: Seigneur Éternel, que me donneras-tu? Je m`en vais sans enfants; et l`héritier de ma maison, c`est Éliézer de Damas. 
\verse Et Abram dit: Voici, tu ne m`as pas donné de postérité, et celui qui est né dans ma maison sera mon héritier. 
\verse Alors la parole de l`Éternel lui fut adressée ainsi: Ce n`est pas lui qui sera ton héritier, mais c`est celui qui sortira de tes entrailles qui sera ton héritier. 
\verse Et après l`avoir conduit dehors, il dit: Regarde vers le ciel, et compte les étoiles, si tu peux les compter. Et il lui dit: Telle sera ta postérité. 
\verse Abram eut confiance en l`Éternel, qui le lui imputa à justice. 
\verse L`Éternel lui dit encore: Je suis l`Éternel, qui t`ai fait sortir d`Ur en Chaldée, pour te donner en possession ce pays. 
\verse Abram répondit: Seigneur Éternel, à quoi connaîtrai-je que je le posséderai? 
\verse Et l`Éternel lui dit: Prends une génisse de trois ans, une chèvre de trois ans, un bélier de trois ans, une tourterelle et une jeune colombe. 
\verse Abram prit tous ces animaux, les coupa par le milieu, et mit chaque morceau l`un vis-à-vis de l`autre; mais il ne partagea point les oiseaux. 
\verse Les oiseaux de proie s`abattirent sur les cadavres; et Abram les chassa. 
\verse Au coucher du soleil, un profond sommeil tomba sur Abram; et voici, une frayeur et une grande obscurité vinrent l`assaillir. 
\verse Et l`Éternel dit à Abram: Sache que tes descendants seront étrangers dans un pays qui ne sera point à eux; ils y seront asservis, et on les opprimera pendant quatre cents ans. 
\verse Mais je jugerai la nation à laquelle ils seront asservis, et ils sortiront ensuite avec de grandes richesses. 
\verse Toi, tu iras en paix vers tes pères, tu seras enterré après une heureuse vieillesse. 
\verse A la quatrième génération, ils reviendront ici; car l`iniquité des Amoréens n`est pas encore à son comble. 
\verse Quand le soleil fut couché, il y eut une obscurité profonde; et voici, ce fut une fournaise fumante, et des flammes passèrent entre les animaux partagés. 
\verse En ce jour-là, l`Éternel fit alliance avec Abram, et dit: Je donne ce pays à ta postérité, depuis le fleuve d`Égypte jusqu`au grand fleuve, au fleuve d`Euphrate, 
\verse le pays des Kéniens, des Keniziens, des Kadmoniens, 
\verse des Héthiens, des Phéréziens, des Rephaïm, 
\verse des Amoréens, des Cananéens, des Guirgasiens et des Jébusiens. 

\chapter
\verse Saraï, femme d`Abram, ne lui avait point donné d`enfants. Elle avait une servante Égyptienne, nommée Agar. 
\verse Et Saraï dit à Abram: Voici, l`Éternel m`a rendue stérile; viens, je te prie, vers ma servante; peut-être aurai-je par elle des enfants. Abram écouta la voix de Saraï. 
\verse Alors Saraï, femme d`Abram, prit Agar, l`Égyptienne, sa servante, et la donna pour femme à Abram, son mari, après qu`Abram eut habité dix années dans le pays de Canaan. 
\verse Il alla vers Agar, et elle devint enceinte. Quand elle se vit enceinte, elle regarda sa maîtresse avec mépris. 
\verse Et Saraï dit à Abram: L`outrage qui m`est fait retombe sur toi. J`ai mis ma servante dans ton sein; et, quand elle a vu qu`elle était enceinte, elle m`a regardée avec mépris. Que l`Éternel soit juge entre moi et toi! 
\verse Abram répondit à Saraï: Voici, ta servante est en ton pouvoir, agis à son égard comme tu le trouveras bon. Alors Saraï la maltraita; et Agar s`enfuit loin d`elle. 
\verse L`ange de l`Éternel la trouva près d`une source d`eau dans le désert, près de la source qui est sur le chemin de Schur. 
\verse Il dit: Agar, servante de Saraï, d`où viens-tu, et où vas-tu? Elle répondit: Je fuis loin de Saraï, ma maîtresse. 
\verse L`ange de l`Éternel lui dit: Retourne vers ta maîtresse, et humilie-toi sous sa main. 
\verse L`ange de l`Éternel lui dit: Je multiplierai ta postérité, et elle sera si nombreuse qu`on ne pourra la compter. 
\verse L`ange de l`Éternel lui dit: Voici, tu es enceinte, et tu enfanteras un fils, à qui tu donneras le nom d`Ismaël; car l`Éternel t`a entendue dans ton affliction. 
\verse Il sera comme un âne sauvage; sa main sera contre tous, et la main de tous sera contre lui; et il habitera en face de tous ses frères. 
\verse Elle appela Atta El roï le nom de l`Éternel qui lui avait parlé; car elle dit: Ai-je rien vu ici, après qu`il m`a vue? 
\verse C`est pourquoi l`on a appelé ce puits le puits de Lachaï roï; il est entre Kadès et Bared. 
\verse Agar enfanta un fils à Abram; et Abram donna le nom d`Ismaël au fils qu`Agar lui enfanta. 
\verse Abram était âgé de quatre-vingt-six ans lorsqu`Agar enfanta Ismaël à Abram. 

\chapter
\verse Lorsque Abram fut âgé de quatre-vingt-dix-neuf ans, l`Éternel apparut à Abram, et lui dit: Je suis le Dieu tout puissant. Marche devant ma face, et sois intègre. 
\verse J`établirai mon alliance entre moi et toi, et je te multiplierai à l`infini. 
\verse Abram tomba sur sa face; et Dieu lui parla, en disant: 
\verse Voici mon alliance, que je fais avec toi. Tu deviendras père d`une multitude de nations. 
\verse On ne t`appellera plus Abram; mais ton nom sera Abraham, car je te rends père d`une multitude de nations. 
\verse Je te rendrai fécond à l`infini, je ferai de toi des nations; et des rois sortiront de toi. 
\verse J`établirai mon alliance entre moi et toi, et tes descendants après toi, selon leurs générations: ce sera une alliance perpétuelle, en vertu de laquelle je serai ton Dieu et celui de ta postérité après toi. 
\verse Je te donnerai, et à tes descendants après toi, le pays que tu habites comme étranger, tout le pays de Canaan, en possession perpétuelle, et je serai leur Dieu. 
\verse Dieu dit à Abraham: Toi, tu garderas mon alliance, toi et tes descendants après toi, selon leurs générations. 
\verse C`est ici mon alliance, que vous garderez entre moi et vous, et ta postérité après toi: tout mâle parmi vous sera circoncis. 
\verse Vous vous circoncirez; et ce sera un signe d`alliance entre moi et vous. 
\verse A l`âge de huit jours, tout mâle parmi vous sera circoncis, selon vos générations, qu`il soit né dans la maison, ou qu`il soit acquis à prix d`argent de tout fils d`étranger, sans appartenir à ta race. 
\verse On devra circoncire celui qui est né dans la maison et celui qui est acquis à prix d`argent; et mon alliance sera dans votre chair une alliance perpétuelle. 
\verse Un mâle incirconcis, qui n`aura pas été circoncis dans sa chair, sera exterminé du milieu de son peuple: il aura violé mon alliance. 
\verse Dieu dit à Abraham: Tu ne donneras plus à Saraï, ta femme, le nom de Saraï; mais son nom sera Sara. 
\verse Je la bénirai, et je te donnerai d`elle un fils; je la bénirai, et elle deviendra des nations; des rois de peuples sortiront d`elle. 
\verse Abraham tomba sur sa face; il rit, et dit en son coeur: Naîtrait-il un fils à un homme de cent ans? et Sara, âgée de quatre-vingt-dix ans, enfanterait-elle? 
\verse Et Abraham dit à Dieu: Oh! qu`Ismaël vive devant ta face! 
\verse Dieu dit: Certainement Sara, ta femme, t`enfantera un fils; et tu l`appelleras du nom d`Isaac. J`établirai mon alliance avec lui comme une alliance perpétuelle pour sa postérité après lui. 
\verse A l`égard d`Ismaël, je t`ai exaucé. Voici, je le bénirai, je le rendrai fécond, et je le multiplierai à l`infini; il engendrera douze princes, et je ferai de lui une grande nation. 
\verse J`établirai mon alliance avec Isaac, que Sara t`enfantera à cette époque-ci de l`année prochaine. 
\verse Lorsqu`il eut achevé de lui parler, Dieu s`éleva au-dessus d`Abraham. 
\verse Abraham prit Ismaël, son fils, tous ceux qui étaient nés dans sa maison et tous ceux qu`il avait acquis à prix d`argent, tous les mâles parmi les gens de la maison d`Abraham; et il les circoncit ce même jour, selon l`ordre que Dieu lui avait donné. 
\verse Abraham était âgé de quatre-vingt-dix-neuf ans, lorsqu`il fut circoncis. 
\verse Ismaël, son fils, était âgé de treize ans lorsqu`il fut circoncis. 
\verse Ce même jour, Abraham fut circoncis, ainsi qu`Ismaël, son fils. 
\verse Et tous les gens de sa maison, nés dans sa maison, ou acquis à prix d`argent des étrangers, furent circoncis avec lui. 

\chapter
\verse L`Éternel lui apparut parmi les chênes de Mamré, comme il était assis à l`entrée de sa tente, pendant la chaleur du jour. 
\verse Il leva les yeux, et regarda: et voici, trois hommes étaient debout près de lui. Quand il les vit, il courut au-devant d`eux, depuis l`entrée de sa tente, et se prosterna en terre. 
\verse Et il dit: Seigneur, si j`ai trouvé grâce à tes yeux, ne passe point, je te prie, loin de ton serviteur. 
\verse Permettez qu`on apporte un peu d`eau, pour vous laver les pieds; et reposez-vous sous cet arbre. 
\verse J`irai prendre un morceau de pain, pour fortifier votre coeur; après quoi, vous continuerez votre route; car c`est pour cela que vous passez près de votre serviteur. Ils répondirent: Fais comme tu l`as dit. 
\verse Abraham alla promptement dans sa tente vers Sara, et il dit: Vite, trois mesures de fleur de farine, pétris, et fais des gâteaux. 
\verse Et Abraham courut à son troupeau, prit un veau tendre et bon, et le donna à un serviteur, qui se hâta de l`apprêter. 
\verse Il prit encore de la crème et du lait, avec le veau qu`on avait apprêté, et il les mit devant eux. Il se tint lui-même à leurs côtés, sous l`arbre. Et ils mangèrent. 
\verse Alors ils lui dirent: Où est Sara, ta femme? Il répondit: Elle est là, dans la tente. 
\verse L`un d`entre eux dit: Je reviendrai vers toi à cette même époque; et voici, Sara, ta femme, aura un fils. Sara écoutait à l`entrée de la tente, qui était derrière lui. 
\verse Abraham et Sara étaient vieux, avancés en âge: et Sara ne pouvait plus espérer avoir des enfants. 
\verse Elle rit en elle-même, en disant: Maintenant que je suis vieille, aurais-je encore des désirs? Mon seigneur aussi est vieux. 
\verse L`Éternel dit à Abraham: Pourquoi donc Sara a-t-elle ri, en disant: Est-ce que vraiment j`aurais un enfant, moi qui suis vieille? 
\verse Y a-t-il rien qui soit étonnant de la part de l`Éternel? Au temps fixé je reviendrai vers toi, à cette même époque; et Sara aura un fils. 
\verse Sara mentit, en disant: Je n`ai pas ri. Car elle eut peur. Mais il dit: Au contraire, tu as ri. 
\verse Ces hommes se levèrent pour partir, et ils regardèrent du côté de Sodome. Abraham alla avec eux, pour les accompagner. 
\verse Alors l`Éternel dit: Cacherai-je à Abraham ce que je vais faire?... 
\verse Abraham deviendra certainement une nation grande et puissante, et en lui seront bénies toutes les nations de la terre. 
\verse Car je l`ai choisi, afin qu`il ordonne à ses fils et à sa maison après lui de garder la voie de l`Éternel, en pratiquant la droiture et la justice, et qu`ainsi l`Éternel accomplisse en faveur d`Abraham les promesses qu`il lui a faites... 
\verse Et l`Éternel dit: Le cri contre Sodome et Gomorrhe s`est accru, et leur péché est énorme. 
\verse C`est pourquoi je vais descendre, et je verrai s`ils ont agi entièrement selon le bruit venu jusqu`à moi; et si cela n`est pas, je le saurai. 
\verse Les hommes s`éloignèrent, et allèrent vers Sodome. Mais Abraham se tint encore en présence de l`Éternel. 
\verse Abraham s`approcha, et dit: Feras-tu aussi périr le juste avec le méchant? 
\verse Peut-être y a-t-il cinquante justes au milieu de la ville: les feras-tu périr aussi, et ne pardonneras-tu pas à la ville à cause des cinquante justes qui sont au milieu d`elle? 
\verse Faire mourir le juste avec le méchant, en sorte qu`il en soit du juste comme du méchant, loin de toi cette manière d`agir! loin de toi! Celui qui juge toute la terre n`exercera-t-il pas la justice? 
\verse Et l`Éternel dit: Si je trouve dans Sodome cinquante justes au milieu de la ville, je pardonnerai à toute la ville, à cause d`eux. 
\verse Abraham reprit, et dit: Voici, j`ai osé parler au Seigneur, moi qui ne suis que poudre et cendre. 
\verse Peut-être des cinquante justes en manquera-t-il cinq: pour cinq, détruiras-tu toute la ville? Et l`Éternel dit: Je ne la détruirai point, si j`y trouve quarante-cinq justes. 
\verse Abraham continua de lui parler, et dit: Peut-être s`y trouvera-t-il quarante justes. Et l`Éternel dit: Je ne ferai rien, à cause de ces quarante. 
\verse Abraham dit: Que le Seigneur ne s`irrite point, et je parlerai. Peut-être s`y trouvera-t-il trente justes. Et l`Éternel dit: Je ne ferai rien, si j`y trouve trente justes. 
\verse Abraham dit: Voici, j`ai osé parler au Seigneur. Peut-être s`y trouvera-t-il vingt justes. Et l`Éternel dit: Je ne la détruirai point, à cause de ces vingt. 
\verse Abraham dit: Que le Seigneur ne s`irrite point, et je ne parlerai plus que cette fois. Peut-être s`y trouvera-t-il dix justes. Et l`Éternel dit: Je ne la détruirai point, à cause de ces dix justes. 
\verse L`Éternel s`en alla lorsqu`il eut achevé de parler à Abraham. Et Abraham retourna dans sa demeure. 

\chapter
\verse Les deux anges arrivèrent à Sodome sur le soir; et Lot était assis à la porte de Sodome. Quand Lot les vit, il se leva pour aller au-devant d`eux, et se prosterna la face contre terre. 
\verse Puis il dit: Voici, mes seigneurs, entrez, je vous prie, dans la maison de votre serviteur, et passez-y la nuit; lavez-vous les pieds; vous vous lèverez de bon matin, et vous poursuivrez votre route. Non, répondirent-ils, nous passerons la nuit dans la rue. 
\verse Mais Lot les pressa tellement qu`ils vinrent chez lui et entrèrent dans sa maison. Il leur donna un festin, et fit cuire des pains sans levain. Et ils mangèrent. 
\verse Ils n`étaient pas encore couchés que les gens de la ville, les gens de Sodome, entourèrent la maison, depuis les enfants jusqu`aux vieillards; toute la population était accourue. 
\verse Ils appelèrent Lot, et lui dirent: Où sont les hommes qui sont entrés chez toi cette nuit? Fais-les sortir vers nous, pour que nous les connaissions. 
\verse Lot sortit vers eux à l`entrée de la maison, et ferma la porte derrière lui. 
\verse Et il dit: Mes frères, je vous prie, ne faites pas le mal! 
\verse Voici, j`ai deux filles qui n`ont point connu d`homme; je vous les amènerai dehors, et vous leur ferez ce qu`il vous plaira. Seulement, ne faites rien à ces hommes puisqu`ils sont venus à l`ombre de mon toit. 
\verse Ils dirent: Retire-toi! Ils dirent encore: Celui-ci est venu comme étranger, et il veut faire le juge! Eh bien, nous te ferons pis qu`à eux. Et, pressant Lot avec violence, ils s`avancèrent pour briser la porte. 
\verse Les hommes étendirent la main, firent rentrer Lot vers eux dans la maison, et fermèrent la porte. 
\verse Et ils frappèrent d`aveuglement les gens qui étaient à l`entrée de la maison, depuis le plus petit jusqu`au plus grand, de sorte qu`ils se donnèrent une peine inutile pour trouver la porte. 
\verse Les hommes dirent à Lot: Qui as-tu encore ici? Gendres, fils et filles, et tout ce qui t`appartient dans la ville, fais-les sortir de ce lieu. 
\verse Car nous allons détruire ce lieu, parce que le cri contre ses habitants est grand devant l`Éternel. L`Éternel nous a envoyés pour le détruire. 
\verse Lot sortit, et parla à ses gendres qui avaient pris ses filles: Levez-vous, dit-il, sortez de ce lieu; car l`Éternel va détruire la ville. Mais, aux yeux de ses gendres, il parut plaisanter. 
\verse Dès l`aube du jour, les anges insistèrent auprès de Lot, en disant: Lève-toi, prends ta femme et tes deux filles qui se trouvent ici, de peur que tu ne périsses dans la ruine de la ville. 
\verse Et comme il tardait, les hommes le saisirent par la main, lui, sa femme et ses deux filles, car l`Éternel voulait l`épargner; ils l`emmenèrent, et le laissèrent hors de la ville. 
\verse Après les avoir fait sortir, l`un d`eux dit: Sauve-toi, pour ta vie; ne regarde pas derrière toi, et ne t`arrête pas dans toute la plaine; sauve-toi vers la montagne, de peur que tu ne périsses. 
\verse Lot leur dit: Oh! non, Seigneur! 
\verse Voici, j`ai trouvé grâce à tes yeux, et tu as montré la grandeur de ta miséricorde à mon égard, en me conservant la vie; mais je ne puis me sauver à la montagne, avant que le désastre m`atteigne, et je périrai. 
\verse Voici, cette ville est assez proche pour que je m`y réfugie, et elle est petite. Oh! que je puisse m`y sauver,... n`est-elle pas petite?... et que mon âme vive! 
\verse Et il lui dit: Voici, je t`accorde encore cette grâce, et je ne détruirai pas la ville dont tu parles. 
\verse Hâte-toi de t`y réfugier, car je ne puis rien faire jusqu`à ce que tu y sois arrivé. C`est pour cela que l`on a donné à cette ville le nom de Tsoar. 
\verse Le soleil se levait sur la terre, lorsque Lot entra dans Tsoar. 
\verse Alors l`Éternel fit pleuvoir du ciel sur Sodome et sur Gomorrhe du soufre et du feu, de par l`Éternel. 
\verse Il détruisit ces villes, toute la plaine et tous les habitants des villes, et les plantes de la terre. 
\verse La femme de Lot regarda en arrière, et elle devint une statue de sel. 
\verse Abraham se leva de bon matin, pour aller au lieu où il s`était tenu en présence de l`Éternel. 
\verse Il porta ses regards du côté de Sodome et de Gomorrhe, et sur tout le territoire de la plaine; et voici, il vit s`élever de la terre une fumée, comme la fumée d`une fournaise. 
\verse Lorsque Dieu détruisit les villes de la plaine, il se souvint d`Abraham; et il fit échapper Lot du milieu du désastre, par lequel il bouleversa les villes où Lot avait établi sa demeure. 
\verse Lot quitta Tsoar pour la hauteur, et se fixa sur la montagne, avec ses deux filles, car il craignait de rester à Tsoar. Il habita dans une caverne, lui et ses deux filles. 
\verse L`aînée dit à la plus jeune: Notre père est vieux; et il n`y a point d`homme dans la contrée, pour venir vers nous, selon l`usage de tous les pays. 
\verse Viens, faisons boire du vin à notre père, et couchons avec lui, afin que nous conservions la race de notre père. 
\verse Elles firent donc boire du vin à leur père cette nuit-là; et l`aînée alla coucher avec son père: il ne s`aperçut ni quand elle se coucha, ni quand elle se leva. 
\verse Le lendemain, l`aînée dit à la plus jeune: Voici, j`ai couché la nuit dernière avec mon père; faisons-lui boire du vin encore cette nuit, et va coucher avec lui, afin que nous conservions la race de notre père. 
\verse Elles firent boire du vin à leur père encore cette nuit-là; et la cadette alla coucher avec lui: il ne s`aperçut ni quand elle se coucha, ni quand elle se leva. 
\verse Les deux filles de Lot devinrent enceintes de leur père. 
\verse L`aînée enfanta un fils, qu`elle appela du nom de Moab: c`est le père des Moabites, jusqu`à ce jour. 
\verse La plus jeune enfanta aussi un fils, qu`elle appela du nom de Ben Ammi: c`est le père des Ammonites, jusqu`à ce jour. 

\chapter
\verse Abraham partit de là pour la contrée du midi; il s`établit entre Kadès et Schur, et fit un séjour à Guérar. 
\verse Abraham disait de Sara, sa femme: C`est ma soeur. Abimélec, roi de Guérar, fit enlever Sara. 
\verse Alors Dieu apparut en songe à Abimélec pendant la nuit, et lui dit: Voici, tu vas mourir à cause de la femme que tu as enlevée, car elle a un mari. 
\verse Abimélec, qui ne s`était point approché d`elle, répondit: Seigneur, ferais-tu périr même une nation juste? 
\verse Ne m`a-t-il pas dit: C`est ma soeur? et elle-même n`a-t-elle pas dit: C`est mon frère? J`ai agi avec un coeur pur et avec des mains innocentes. 
\verse Dieu lui dit en songe: Je sais que tu as agi avec un coeur pur; aussi t`ai-je empêché de pécher contre moi. C`est pourquoi je n`ai pas permis que tu la touchasse. 
\verse Maintenant, rends la femme de cet homme; car il est prophète, il priera pour toi, et tu vivras. Mais, si tu ne la rends pas, sache que tu mourras, toi et tout ce qui t`appartient. 
\verse Abimélec se leva de bon matin, il appela tous ses serviteurs, et leur rapporta toutes ces choses; et ces gens furent saisis d`une grande frayeur. 
\verse Abimélec appela aussi Abraham, et lui dit: Qu`est-ce que tu nous as fait? Et en quoi t`ai-je offensé, que tu aies fait venir sur moi et sur mon royaume un si grand péché? Tu as commis à mon égard des actes qui ne doivent pas se commettre. 
\verse Et Abimélec dit à Abraham: Quelle intention avais-tu pour agir de la sorte? 
\verse Abraham répondit: Je me disais qu`il n`y avait sans doute aucune crainte de Dieu dans ce pays, et que l`on me tuerait à cause de ma femme. 
\verse De plus, il est vrai qu`elle est ma soeur, fille de mon père; seulement, elle n`est pas fille de ma mère; et elle est devenue ma femme. 
\verse Lorsque Dieu me fit errer loin de la maison de mon père, je dis à Sara: Voici la grâce que tu me feras; dans tous les lieux où nous irons, dis de moi: C`est mon frère. 
\verse Abimélec prit des brebis et des boeufs, des serviteurs et des servantes, et les donna à Abraham; et il lui rendit Sara, sa femme. 
\verse Abimélec dit: Voici, mon pays est devant toi; demeure où il te plaira. 
\verse Et il dit à Sara: Voici, je donne à ton frère mille pièces d`argent; cela te sera un voile sur les yeux pour tous ceux qui sont avec toi, et auprès de tous tu seras justifiée. 
\verse Abraham pria Dieu, et Dieu guérit Abimélec, sa femme et ses servantes; et elles purent enfanter. 
\verse Car l`Éternel avait frappé de stérilité toute la maison d`Abimélec, à cause de Sara, femme d`Abraham. 

\chapter
\verse L`Éternel se souvint de ce qu`il avait dit à Sara, et l`Éternel accomplit pour Sara ce qu`il avait promis. 
\verse Sara devint enceinte, et elle enfanta un fils à Abraham dans sa vieillesse, au temps fixé dont Dieu lui avait parlé. 
\verse Abraham donna le nom d`Isaac au fils qui lui était né, que Sara lui avait enfanté. 
\verse Abraham circoncit son fils Isaac, âgé de huit jours, comme Dieu le lui avait ordonné. 
\verse Abraham était âgé de cent ans, à la naissance d`Isaac, son fils. 
\verse Et Sara dit: Dieu m`a fait un sujet de rire; quiconque l`apprendra rira de moi. 
\verse Elle ajouta: Qui aurait dit à Abraham: Sara allaitera des enfants? Cependant je lui ai enfanté un fils dans sa vieillesse. 
\verse L`enfant grandit, et fut sevré; et Abraham fit un grand festin le jour où Isaac fut sevré. 
\verse Sara vit rire le fils qu`Agar, l`Égyptienne, avait enfanté à Abraham; 
\verse et elle dit à Abraham: Chasse cette servante et son fils, car le fils de cette servante n`héritera pas avec mon fils, avec Isaac. 
\verse Cette parole déplut fort aux yeux d`Abraham, à cause de son fils. 
\verse Mais Dieu dit à Abraham: Que cela ne déplaise pas à tes yeux, à cause de l`enfant et de ta servante. Accorde à Sara tout ce qu`elle te demandera; car c`est d`Isaac que sortira une postérité qui te sera propre. 
\verse Je ferai aussi une nation du fils de ta servante; car il est ta postérité. 
\verse Abraham se leva de bon matin; il prit du pain et une outre d`eau, qu`il donna à Agar et plaça sur son épaule; il lui remit aussi l`enfant, et la renvoya. Elle s`en alla, et s`égara dans le désert de Beer Schéba. 
\verse Quand l`eau de l`outre fut épuisée, elle laissa l`enfant sous un des arbrisseaux, 
\verse et alla s`asseoir vis-à-vis, à une portée d`arc; car elle disait: Que je ne voie pas mourir mon enfant! Elle s`assit donc vis-à-vis de lui, éleva la voix et pleura. 
\verse Dieu entendit la voix de l`enfant; et l`ange de Dieu appela du ciel Agar, et lui dit: Qu`as-tu, Agar? Ne crains point, car Dieu a entendu la voix de l`enfant dans le lieu où il est. 
\verse Lève-toi, prends l`enfant, saisis-le de ta main; car je ferai de lui une grande nation. 
\verse Et Dieu lui ouvrit les yeux, et elle vit un puits d`eau; elle alla remplir d`eau l`outre, et donna à boire à l`enfant. 
\verse Dieu fut avec l`enfant, qui grandit, habita dans le désert, et devint tireur d`arc. 
\verse Il habita dans le désert de Paran, et sa mère lui prit une femme du pays d`Égypte. 
\verse En ce temps-là, Abimélec, accompagné de Picol, chef de son armée, parla ainsi à Abraham: Dieu est avec toi dans tout ce que tu fais. 
\verse Jure-moi maintenant ici, par le nom de Dieu, que tu ne tromperas ni moi, ni mes enfants, ni mes petits-enfants, et que tu auras pour moi et le pays où tu séjournes la même bienveillance que j`ai eue pour toi. 
\verse Abraham dit: Je le jurerai. 
\verse Mais Abraham fit des reproches à Abimélec, au sujet d`un puits d`eau, dont s`étaient emparés de force les serviteurs d`Abimélec. 
\verse Abimélec répondit: J`ignore qui a fait cette chose-là; tu ne m`en as point informé, et moi, je ne l`apprends qu`aujourd`hui. 
\verse Et Abraham prit des brebis et des boeufs, qu`il donna à Abimélec; et ils firent tous deux alliance. 
\verse Abraham mit à part sept jeunes brebis. 
\verse Et Abimélec dit à Abraham: Qu`est-ce que ces sept jeunes brebis, que tu as mises à part? 
\verse Il répondit: Tu accepteras de ma main ces sept brebis, afin que cela me serve de témoignage que j`ai creusé ce puits. 
\verse C`est pourquoi on appelle ce lieu Beer Schéba; car c`est là qu`ils jurèrent l`un et l`autre. 
\verse Ils firent donc alliance à Beer Schéba. Après quoi, Abimélec se leva, avec Picol, chef de son armée; et ils retournèrent au pays des Philistins. 
\verse Abraham planta des tamariscs à Beer Schéba; et là il invoqua le nom de l`Éternel, Dieu de l`éternité. 
\verse Abraham séjourna longtemps dans le pays des Philistins. 

\chapter
\verse Après ces choses, Dieu mit Abraham à l`épreuve, et lui dit: Abraham! Et il répondit: Me voici! 
\verse Dieu dit: Prends ton fils, ton unique, celui que tu aimes, Isaac; va-t`en au pays de Morija, et là offre-le en holocauste sur l`une des montagnes que je te dirai. 
\verse Abraham se leva de bon matin, sella son âne, et prit avec lui deux serviteurs et son fils Isaac. Il fendit du bois pour l`holocauste, et partit pour aller au lieu que Dieu lui avait dit. 
\verse Le troisième jour, Abraham, levant les yeux, vit le lieu de loin. 
\verse Et Abraham dit à ses serviteurs: Restez ici avec l`âne; moi et le jeune homme, nous irons jusque-là pour adorer, et nous reviendrons auprès de vous. 
\verse Abraham prit le bois pour l`holocauste, le chargea sur son fils Isaac, et porta dans sa main le feu et le couteau. Et il marchèrent tous deux ensemble. 
\verse Alors Isaac, parlant à Abraham, son père, dit: Mon père! Et il répondit: Me voici, mon fils! Isaac reprit: Voici le feu et le bois; mais où est l`agneau pour l`holocauste? 
\verse Abraham répondit: Mon fils, Dieu se pourvoira lui-même de l`agneau pour l`holocauste. Et ils marchèrent tous deux ensemble. 
\verse Lorsqu`ils furent arrivés au lieu que Dieu lui avait dit, Abraham y éleva un autel, et rangea le bois. Il lia son fils Isaac, et le mit sur l`autel, par-dessus le bois. 
\verse Puis Abraham étendit la main, et prit le couteau, pour égorger son fils. 
\verse Alors l`ange de l`Éternel l`appela des cieux, et dit: Abraham! Abraham! Et il répondit: Me voici! 
\verse L`ange dit: N`avance pas ta main sur l`enfant, et ne lui fais rien; car je sais maintenant que tu crains Dieu, et que tu ne m`as pas refusé ton fils, ton unique. 
\verse Abraham leva les yeux, et vit derrière lui un bélier retenu dans un buisson par les cornes; et Abraham alla prendre le bélier, et l`offrit en holocauste à la place de son fils. 
\verse Abraham donna à ce lieu le nom de Jehova Jiré. C`est pourquoi l`on dit aujourd`hui: A la montagne de l`Éternel il sera pourvu. 
\verse L`ange de l`Éternel appela une seconde fois Abraham des cieux, 
\verse et dit: Je le jure par moi-même, parole de l`Éternel! parce que tu as fais cela, et que tu n`as pas refusé ton fils, ton unique, 
\verse je te bénirai et je multiplierai ta postérité, comme les étoiles du ciel et comme le sable qui est sur le bord de la mer; et ta postérité possédera la porte de ses ennemis. 
\verse Toutes les nations de la terre seront bénies en ta postérité, parce que tu as obéi à ma voix. 
\verse Abraham étant retourné vers ses serviteurs, ils se levèrent et s`en allèrent ensemble à Beer Schéba; car Abraham demeurait à Beer Schéba. 
\verse Après ces choses, on fit à Abraham un rapport, en disant: Voici, Milca a aussi enfanté des fils à Nachor, ton frère: 
\verse Uts, son premier-né, Buz, son frère, Kemuel, père d`Aram, 
\verse Késed, Hazo, Pildasch, Jidlaph et Bethuel. 
\verse Bethuel a engendré Rebecca. Ce sont là les huit fils que Milca a enfantés à Nachor, frère d`Abraham. 
\verse Sa concubine, nommée Réuma, a aussi enfanté Thébach, Gaham, Tahasch et Maaca. 

\chapter
\verse La vie de Sara fut de cent vingt-sept ans: telles sont les années de la vie de Sara. 
\verse Sara mourut à Kirjath Arba, qui est Hébron, dans le pays de Canaan; et Abraham vint pour mener deuil sur Sara et pour la pleurer. 
\verse Abraham se leva de devant son mort, et parla ainsi aux fils de Heth: 
\verse Je suis étranger et habitant parmi vous; donnez-moi la possession d`un sépulcre chez vous, pour enterrer mon mort et l`ôter de devant moi. 
\verse Les fils de Heth répondirent à Abraham, en lui disant: 
\verse Écoute-nous, mon seigneur! Tu es un prince de Dieu au milieu de nous; enterre ton mort dans celui de nos sépulcres que tu choisiras; aucun de nous ne te refusera son sépulcre pour enterrer ton mort. 
\verse Abraham se leva, et se prosterna devant le peuple du pays, devant les fils de Heth. 
\verse Et il leur parla ainsi: Si vous permettez que j`enterre mon mort et que je l`ôte de devant mes yeux, écoutez-moi, et priez pour moi Éphron, fils de Tsochar, 
\verse de me céder la caverne de Macpéla, qui lui appartient, à l`extrémité de son champ, de me la céder contre sa valeur en argent, afin qu`elle me serve de possession sépulcrale au milieu de vous. 
\verse Éphron était assis parmi les fils de Heth. Et Éphron, le Héthien, répondit à Abraham, en présence des fils de Heth et de tous ceux qui entraient par la porte de sa ville: 
\verse Non, mon seigneur, écoute-moi! Je te donne le champ, et je te donne la caverne qui y est. Je te les donne, aux yeux des fils de mon peuple: enterre ton mort. 
\verse Abraham se prosterna devant le peuple du pays. 
\verse Et il parla ainsi à Éphron, en présence du peuple du pays: Écoute-moi, je te prie! Je donne le prix du champ: accepte-le de moi; et j`y enterrerai mon mort. 
\verse Et Éphron répondit à Abraham, en lui disant: 
\verse Mon seigneur, écoute-moi! Une terre de quatre cents sicles d`argent, qu`est-ce que cela entre moi et toi? Enterre ton mort. 
\verse Abraham comprit Éphron; et Abraham pesa à Éphron l`argent qu`il avait dit, en présence des fils de Heth, quatre cents sicles d`argent ayant cours chez le marchand. 
\verse Le champ d`Éphron à Macpéla, vis-à-vis de Mamré, le champ et la caverne qui y est, et tous les arbres qui sont dans le champ et dans toutes ses limites alentour, 
\verse devinrent ainsi la propriété d`Abraham, aux yeux des fils de Heth et de tous ceux qui entraient par la porte de sa ville. 
\verse Après cela, Abraham enterra Sara, sa femme, dans la caverne du champ de Macpéla, vis-à-vis de Mamré, qui est Hébron, dans le pays de Canaan. 
\verse Le champ et la caverne qui y est demeurèrent à Abraham comme possession sépulcrale, acquise des fils de Heth. 

\chapter
\verse Abraham était vieux, avancé en âge; et l`Éternel avait béni Abraham en toute chose. 
\verse Abraham dit à son serviteur, le plus ancien de sa maison, l`intendant de tous ses biens: Mets, je te prie, ta main sous ma cuisse; 
\verse et je te ferai jurer par l`Éternel, le Dieu du ciel et le Dieu de la terre, de ne pas prendre pour mon fils une femme parmi les filles des Cananéens au milieu desquels j`habite, 
\verse mais d`aller dans mon pays et dans ma patrie prendre une femme pour mon fils Isaac. 
\verse Le serviteur lui répondit: Peut-être la femme ne voudra-t-elle pas me suivre dans ce pays-ci; devrai-je mener ton fils dans le pays d`où tu es sorti? 
\verse Abraham lui dit: Garde-toi d`y mener mon fils! 
\verse L`Éternel, le Dieu du ciel, qui m`a fait sortir de la maison de mon père et de ma patrie, qui m`a parlé et qui m`a juré, en disant: Je donnerai ce pays à ta postérité, lui-même enverra son ange devant toi; et c`est de là que tu prendras une femme pour mon fils. 
\verse Si la femme ne veut pas te suivre, tu seras dégagé de ce serment que je te fais faire. Seulement, tu n`y mèneras pas mon fils. 
\verse Le serviteur mit sa main sous la cuisse d`Abraham, son seigneur, et lui jura d`observer ces choses. 
\verse Le serviteur prit dix chameaux parmi les chameaux de son seigneur, et il partit, ayant à sa disposition tous les biens de son seigneur. Il se leva, et alla en Mésopotamie, à la ville de Nachor. 
\verse Il fit reposer les chameaux sur leurs genoux hors de la ville, près d`un puits, au temps du soir, au temps où sortent celles qui vont puiser de l`eau. 
\verse Et il dit: Éternel, Dieu de mon seigneur Abraham, fais-moi, je te prie, rencontrer aujourd`hui ce que je désire, et use de bonté envers mon seigneur Abraham! 
\verse Voici, je me tiens près de la source d`eau, et les filles des gens de la ville vont sortir pour puiser l`eau. 
\verse Que la jeune fille à laquelle je dirai: Penche ta cruche, je te prie, pour que je boive, et qui répondra: Bois, et je donnerai aussi à boire à tes chameaux, soit celle que tu as destinée à ton serviteur Isaac! Et par là je connaîtrai que tu uses de bonté envers mon seigneur. 
\verse Il n`avait pas encore fini de parler que sortit, sa cruche sur l`épaule, Rebecca, née de Bethuel, fils de Milca, femme de Nachor, frère d`Abraham. 
\verse C`était une jeune fille très belle de figure; elle était vierge, et aucun homme ne l`avait connue. Elle descendit à la source, remplit sa cruche, et remonta. 
\verse Le serviteur courut au-devant d`elle, et dit: Laisse-moi boire, je te prie, un peu d`eau de ta cruche. 
\verse Elle répondit: Bois, mon seigneur. Et elle s`empressa d`abaisser sa cruche sur sa main, et de lui donner à boire. 
\verse Quand elle eut achevé de lui donner à boire, elle dit: Je puiserai aussi pour tes chameaux, jusqu`à ce qu`ils aient assez bu. 
\verse Et elle s`empressa de vider sa cruche dans l`abreuvoir, et courut encore au puits pour puiser; et elle puisa pour tous les chameaux. 
\verse L`homme la regardait avec étonnement et sans rien dire, pour voir si l`Éternel faisait réussir son voyage, ou non. 
\verse Quand les chameaux eurent fini de boire, l`homme prit un anneau d`or, du poids d`un demi-sicle, et deux bracelets, du poids de dix sicles d`or. 
\verse Et il dit: De qui es-tu fille? dis-le moi, je te prie. Y a-t-il dans la maison de ton père de la place pour passer la nuit? 
\verse Elle répondit: Je suis fille de Bethuel, fils de Milca et de Nachor. 
\verse Elle lui dit encore: Il y a chez nous de la paille et du fourrage en abondance, et aussi de la place pour passer la nuit. 
\verse Alors l`homme s`inclina et se prosterna devant l`Éternel, 
\verse en disant: Béni soit l`Éternel, le Dieu de mon seigneur Abraham, qui n`a pas renoncé à sa miséricorde et à sa fidélité envers mon seigneur! Moi-même, l`Éternel m`a conduit à la maison des frères de mon seigneur. 
\verse La jeune fille courut raconter ces choses à la maison de sa mère. 
\verse Rebecca avait un frère, nommé Laban. Et Laban courut dehors vers l`homme, près de la source. 
\verse Il avait vu l`anneau et les bracelets aux mains de sa soeur, et il avait entendu les paroles de Rebecca, sa soeur, disant: Ainsi m`a parlé l`homme. Il vint donc à cet homme qui se tenait auprès des chameaux, vers la source, 
\verse et il dit: Viens, béni de l`Éternel! Pourquoi resterais-tu dehors? J`ai préparé la maison, et une place pour les chameaux. 
\verse L`homme arriva à la maison. Laban fit décharger les chameaux, et il donna de la paille et du fourrage aux chameaux, et de l`eau pour laver les pieds de l`homme et les pieds des gens qui étaient avec lui. 
\verse Puis, il lui servit à manger. Mais il dit: Je ne mangerai point, avant d`avoir dit ce que j`ai à dire. Parle! dit Laban. 
\verse Alors il dit: Je suis serviteur d`Abraham. 
\verse L`Éternel a comblé de bénédictions mon seigneur, qui est devenu puissant. Il lui a donné des brebis et des boeufs, de l`argent et de l`or, des serviteurs et des servantes, des chameaux et des ânes. 
\verse Sara, la femme de mon seigneur, a enfanté dans sa vieillesse un fils à mon seigneur; et il lui a donné tout ce qu`il possède. 
\verse Mon seigneur m`a fait jurer, en disant: Tu ne prendras pas pour mon fils une femme parmi les filles des Cananéens, dans le pays desquels j`habite; 
\verse mais tu iras dans la maison de mon père et de ma famille prendre une femme pour mon fils. 
\verse J`ai dit à mon seigneur: Peut-être la femme ne voudra-t-elle pas me suivre. 
\verse Et il m`a répondu: L`Éternel, devant qui j`ai marché, enverra son ange avec toi, et fera réussir ton voyage; et tu prendras pour mon fils une femme de la famille et de la maison de mon père. 
\verse Tu seras dégagé du serment que tu me fais, quand tu auras été vers ma famille; si on ne te l`accorde pas, tu seras dégagé du serment que tu me fais. 
\verse Je suis arrivé aujourd`hui à la source, et j`ai dit: Éternel, Dieu de mon seigneur Abraham, si tu daignes faire réussir le voyage que j`accomplis, 
\verse voici, je me tiens près de la source d`eau, et que la jeune fille qui sortira pour puiser, à qui je dirai: Laisse-moi boire, je te prie, un peu d`eau de ta cruche, et qui me répondra: 
\verse Bois toi-même, et je puiserai aussi pour tes chameaux, que cette jeune fille soit la femme que l`Éternel a destinée au fils de mon seigneur! 
\verse Avant que j`eusse fini de parler en mon coeur, voici, Rebecca est sortie, sa cruche sur l`épaule; elle est descendue à la source, et a puisé. Je lui ai dit: Donne-moi à boire, je te prie. 
\verse Elle s`est empressée d`abaisser sa cruche de dessus son épaule, et elle a dit: Bois, et je donnerai aussi à boire à tes chameaux. J`ai bu, et elle a aussi donné à boire à mes chameaux. 
\verse Je l`ai interrogée, et j`ai dit: De qui es-tu fille? Elle a répondu: Je suis fille de Bethuel, fils de Nachor et de Milca. J`ai mis l`anneau à son nez, et les bracelets à ses mains. 
\verse Puis je me suis incliné et prosterné devant l`Éternel, et j`ai béni L`Éternel, le Dieu de mon seigneur Abraham, qui m`a conduit fidèlement, afin que je prisse la fille du frère de mon seigneur pour son fils. 
\verse Maintenant, si vous voulez user de bienveillance et de fidélité envers mon seigneur, déclarez-le-moi; sinon, déclarez-le-moi, et je me tournerai à droite ou à gauche. 
\verse Laban et Bethuel répondirent, et dirent: C`est de l`Éternel que la chose vient; nous ne pouvons te parler ni en mal ni en bien. 
\verse Voici Rebecca devant toi; prends et va, et qu`elle soit la femme du fils de ton seigneur, comme l`Éternel l`a dit. 
\verse Lorsque le serviteur d`Abraham eut entendu leurs paroles, il se prosterna en terre devant l`Éternel. 
\verse Et le serviteur sortit des objets d`argent, des objets d`or, et des vêtements, qu`il donna à Rebecca; il fit aussi de riches présents à son frère et à sa mère. 
\verse Après quoi, ils mangèrent et burent, lui et les gens qui étaient avec lui, et ils passèrent la nuit. Le matin, quand ils furent levés, le serviteur dit: Laissez-moi retournez vers mon seigneur. 
\verse Le frère et la mère dirent: Que la jeune fille reste avec nous quelque temps encore, une dizaine de jours; ensuite, tu partiras. 
\verse Il leur répondit: Ne me retardez pas, puisque l`Éternel a fait réussir mon voyage; laissez-moi partir, et que j`aille vers mon seigneur. 
\verse Alors ils répondirent: Appelons la jeune fille et consultons-la. 
\verse Ils appelèrent donc Rebecca, et lui dirent: Veux-tu aller avec cet homme? Elle répondit: J`irai. 
\verse Et ils laissèrent partir Rebecca, leur soeur, et sa nourrice, avec le serviteur d`Abraham et ses gens. 
\verse Ils bénirent Rebecca, et lui dirent: O notre soeur, puisse-tu devenir des milliers de myriades, et que ta postérité possède la porte de ses ennemis! 
\verse Rebecca se leva, avec ses servantes; elles montèrent sur les chameaux, et suivirent l`homme. Et le serviteur emmena Rebecca, et partit. 
\verse Cependant Isaac était revenu du puits de Lachaï roï, et il habitait dans le pays du midi. 
\verse Un soir qu`Isaac était sorti pour méditer dans les champs, il leva les yeux, et regarda; et voici, des chameaux arrivaient. 
\verse Rebecca leva aussi les yeux, vit Isaac, et descendit de son chameau. 
\verse Elle dit au serviteur: Qui est cet homme, qui vient dans les champs à notre rencontre? Et le serviteur répondit: C`est mon seigneur. Alors elle prit son voile, et se couvrit. 
\verse Le serviteur raconta à Isaac toutes les choses qu`il avait faites. 
\verse Isaac conduisit Rebecca dans la tente de Sara, sa mère; il prit Rebecca, qui devint sa femme, et il l`aima. Ainsi fut consolé Isaac, après avoir perdu sa mère. 

\chapter
\verse Abraham prit encore une femme, nommée Ketura. 
\verse Elle lui enfanta Zimran, Jokschan, Medan, Madian, Jischbak et Schuach. 
\verse Jokschan engendra Séba et Dedan. Les fils de Dedan furent les Aschurim, les Letuschim et les Leummim. 
\verse Les fils de Madian furent Épha, Épher, Hénoc, Abida et Eldaa. -Ce sont là tous les fils de Ketura. 
\verse Abraham donna tous ses biens à Isaac. 
\verse Il fit des dons aux fils de ses concubines; et, tandis qu`il vivait encore, il les envoya loin de son fils Isaac du côté de l`orient, dans le pays d`Orient. 
\verse Voici les jours des années de la vie d`Abraham: il vécut cent soixante quinze ans. 
\verse Abraham expira et mourut, après une heureuse vieillesse, âgé et rassasié de jours, et il fut recueilli auprès de son peuple. 
\verse Isaac et Ismaël, ses fils, l`enterrèrent dans la caverne de Macpéla, dans le champ d`Éphron, fils de Tsochar, le Héthien, vis-à-vis de Mamré. 
\verse C`est le champ qu`Abraham avait acquis des fils de Heth. Là furent enterrés Abraham et Sara, sa femme. 
\verse Après la mort d`Abraham, Dieu bénit Isaac, son fils. Il habitait près du puits de Lachaï roï. 
\verse Voici la postérité d`Ismaël, fils d`Abraham, qu`Agar, l`Égyptienne, servante de Sara, avait enfanté à Abraham. 
\verse Voici les noms des fils d`Ismaël, par leurs noms, selon leurs générations: Nebajoth, premier-né d`Ismaël, Kédar, Adbeel, Mibsam, 
\verse Mischma, Duma, Massa, 
\verse Hadad, Théma, Jethur, Naphisch et Kedma. 
\verse Ce sont là les fils d`Ismaël; ce sont là leurs noms, selon leurs parcs et leurs enclos. Ils furent les douze chefs de leurs peuples. 
\verse Et voici les années de la vie d`Ismaël: cent trente-sept ans. Il expira et mourut, et il fut recueilli auprès de son peuple. 
\verse Ses fils habitèrent depuis Havila jusqu`à Schur, qui est en face de l`Égypte, en allant vers l`Assyrie. Il s`établit en présence de tous ses frères. 
\verse Voici la postérité d`Isaac, fils d`Abraham. 
\verse Abraham engendra Isaac. Isaac était âgé de quarante ans, quand il prit pour femme Rebecca, fille de Bethuel, l`Araméen, de Paddan Aram, et soeur de Laban, l`Araméen. 
\verse Isaac implora l`Éternel pour sa femme, car elle était stérile, et l`Éternel l`exauça: Rebecca, sa femme, devint enceinte. 
\verse Les enfants se heurtaient dans son sein; et elle dit: S`il en est ainsi, pourquoi suis-je enceinte? Elle alla consulter l`Éternel. 
\verse Et l`Éternel lui dit: Deux nations sont dans ton ventre, et deux peuples se sépareront au sortir de tes entrailles; un de ces peuples sera plus fort que l`autre, et le plus grand sera assujetti au plus petit. 
\verse Les jours où elle devait accoucher s`accomplirent; et voici, il y avait deux jumeaux dans son ventre. 
\verse Le premier sortit entièrement roux, comme un manteau de poil; et on lui donna le nom d`Ésaü. 
\verse Ensuite sortit son frère, dont la main tenait le talon d`Ésaü; et on lui donna le nom de Jacob. Isaac était âgé de soixante ans, lorsqu`ils naquirent. 
\verse Ces enfants grandirent. Ésaü devint un habile chasseur, un homme des champs; mais Jacob fut un homme tranquille, qui restait sous les tentes. 
\verse Isaac aimait Ésaü, parce qu`il mangeait du gibier; et Rebecca aimait Jacob. 
\verse Comme Jacob faisait cuire un potage, Ésaü revint des champs, accablé de fatigue. 
\verse Et Ésaü dit à Jacob: Laisse-moi, je te prie, manger de ce roux, de ce roux-là, car je suis fatigué. C`est pour cela qu`on a donné à Ésaü le nom d`Édom. 
\verse Jacob dit: Vends-moi aujourd`hui ton droit d`aînesse. 
\verse Ésaü répondit: Voici, je m`en vais mourir; à quoi me sert ce droit d`aînesse? 
\verse Et Jacob dit: Jure-le moi d`abord. Il le lui jura, et il vendit son droit d`aînesse à Jacob. 
\verse Alors Jacob donna à Ésaü du pain et du potage de lentilles. Il mangea et but, puis se leva et s`en alla. C`est ainsi qu`Ésaü méprisa le droit d`aînesse. 

\chapter
\verse Il y eut une famine dans le pays, outre la première famine qui eut lieu du temps d`Abraham; et Isaac alla vers Abimélec, roi des Philistins, à Guérar. 
\verse L`Éternel lui apparut, et dit: Ne descends pas en Égypte, demeure dans le pays que je te dirai. 
\verse Séjourne dans ce pays-ci: je serai avec toi, et je te bénirai, car je donnerai toutes ces contrées à toi et à ta postérité, et je tiendrai le serment que j`ai fait à Abraham, ton père. 
\verse Je multiplierai ta postérité comme les étoiles du ciel; je donnerai à ta postérité toutes ces contrées; et toutes les nations de la terre seront bénies en ta postérité, 
\verse parce qu`Abraham a obéi à ma voix, et qu`il a observé mes ordres, mes commandements, mes statuts et mes lois. 
\verse Et Isaac resta à Guérar. 
\verse Lorsque les gens du lieu faisaient des questions sur sa femme, il disait: C`est ma soeur; car il craignait, en disant ma femme, que les gens du lieu ne le tuassent, parce que Rebecca était belle de figure. 
\verse Comme son séjour se prolongeait, il arriva qu`Abimélec, roi des Philistins, regardant par la fenêtre, vit Isaac qui plaisantait avec Rebecca, sa femme. 
\verse Abimélec fit appeler Isaac, et dit: Certainement, c`est ta femme. Comment as-tu pu dire: C`est ma soeur? Isaac lui répondit: J`ai parlé ainsi, de peur de mourir à cause d`elle. 
\verse Et Abimélec dit: Qu`est-ce que tu nous as fait? Peu s`en est fallu que quelqu`un du peuple n`ait couché avec ta femme, et tu nous aurais rendus coupables. 
\verse Alors Abimélec fit cette ordonnance pour tout le peuple: Celui qui touchera à cet homme ou à sa femme sera mis à mort. 
\verse Isaac sema dans ce pays, et il recueillit cette année le centuple; car l`Éternel le bénit. 
\verse Cet homme devint riche, et il alla s`enrichissant de plus en plus, jusqu`à ce qu`il devint fort riche. 
\verse Il avait des troupeaux de menu bétail et des troupeaux de gros bétail, et un grand nombre de serviteurs: aussi les Philistins lui portèrent envie. 
\verse Tous les puits qu`avaient creusés les serviteurs de son père, du temps d`Abraham, son père, les Philistins les comblèrent et les remplirent de poussière. 
\verse Et Abimélec dit à Isaac: Va-t-en de chez nous, car tu es beaucoup plus puissant que nous. 
\verse Isaac partit de là, et campa dans la vallée de Guérar, où il s`établit. 
\verse Isaac creusa de nouveau les puits d`eau qu`on avait creusés du temps d`Abraham, son père, et qu`avaient comblés les Philistins après la mort d`Abraham; et il leur donna les mêmes noms que son père leur avait donnés. 
\verse Les serviteurs d`Isaac creusèrent encore dans la vallée, et y trouvèrent un puits d`eau vive. 
\verse Les bergers de Guérar querellèrent les bergers d`Isaac, en disant: L`eau est à nous. Et il donna au puits le nom d`Ések, parce qu`ils s`étaient disputés avec lui. 
\verse Ses serviteurs creusèrent un autre puits, au sujet duquel on chercha aussi une querelle; et il l`appela Sitna. 
\verse Il se transporta de là, et creusa un autre puits, pour lequel on ne chercha pas querelle; et il l`appela Rehoboth, car, dit-il, l`Éternel nous a maintenant mis au large, et nous prospérerons dans le pays. 
\verse Il remonta de là à Beer Schéba. 
\verse L`Éternel lui apparut dans la nuit, et dit: Je suis le Dieu d`Abraham, ton père; ne crains point, car je suis avec toi; je te bénirai, et je multiplierai ta postérité, à cause d`Abraham, mon serviteur. 
\verse Il bâtit là un autel, invoqua le nom de l`Éternel, et y dressa sa tente. Et les serviteurs d`Isaac y creusèrent un puits. 
\verse Abimélec vint de Guérar auprès de lui, avec Ahuzath, son ami, et Picol, chef de son armée. 
\verse Isaac leur dit: Pourquoi venez-vous vers moi, puisque vous me haïssez et que vous m`avez renvoyé de chez vous? 
\verse Ils répondirent: Nous voyons que l`Éternel est avec toi. C`est pourquoi nous disons: Qu`il y ait un serment entre nous, entre nous et toi, et que nous fassions alliance avec toi! 
\verse Jure que tu ne nous feras aucun mal, de même que nous ne t`avons point maltraité, que nous t`avons fait seulement du bien, et que nous t`avons laissé partir en paix. Tu es maintenant béni de l`Éternel. 
\verse Isaac leur fit un festin, et ils mangèrent et burent. 
\verse Ils se levèrent de bon matin, et se lièrent l`un à l`autre par un serment. Isaac les laissa partir, et ils le quittèrent en paix. 
\verse Ce même jour, des serviteurs d`Isaac vinrent lui parler du puits qu`ils creusaient, et lui dirent: Nous avons trouvé de l`eau. 
\verse Et il l`appela Schiba. C`est pourquoi on a donné à la ville le nom de Beer Schéba, jusqu`à ce jour. 
\verse Ésaü, âgé de quarante ans, prit pour femmes Judith, fille de Beéri, le Héthien, et Basmath, fille d`Élon, le Héthien. 
\verse Elles furent un sujet d`amertume pour le coeur d`Isaac et de Rebecca. 

\chapter
\verse Isaac devenait vieux, et ses yeux s`étaient affaiblis au point qu`il ne voyait plus. Alors il appela Ésaü, son fils aîné, et lui dit: Mon fils! Et il lui répondit: Me voici! 
\verse Isaac dit: Voici donc, je suis vieux, je ne connais pas le jour de ma mort. 
\verse Maintenant donc, je te prie, prends tes armes, ton carquois et ton arc, va dans les champs, et chasse-moi du gibier. 
\verse Fais-moi un mets comme j`aime, et apporte-le-moi à manger, afin que mon âme te bénisse avant que je meure. 
\verse Rebecca écouta ce qu`Isaac disait à Ésaü, son fils. Et Ésaü s`en alla dans les champs, pour chasser du gibier et pour le rapporter. 
\verse Puis Rebecca dit à Jacob, son fils: Voici, j`ai entendu ton père qui parlait ainsi à Ésaü, ton frère: 
\verse Apporte-moi du gibier et fais-moi un mets que je mangerai; et je te bénirai devant l`Éternel avant ma mort. 
\verse Maintenant, mon fils, écoute ma voix à l`égard de ce que je te commande. 
\verse Va me prendre au troupeau deux bons chevreaux; j`en ferai pour ton père un mets comme il aime; 
\verse et tu le porteras à manger à ton père, afin qu`il te bénisse avant sa mort. 
\verse Jacob répondit à sa mère: Voici, Ésaü, mon frère, est velu, et je n`ai point de poil. 
\verse Peut-être mon père me touchera-t-il, et je passerai à ses yeux pour un menteur, et je ferai venir sur moi la malédiction, et non la bénédiction. 
\verse Sa mère lui dit: Que cette malédiction, mon fils, retombe sur moi! Écoute seulement ma voix, et va me les prendre. 
\verse Jacob alla les prendre, et les apporta à sa mère, qui fit un mets comme son père aimait. 
\verse Ensuite, Rebecca prit les vêtements d`Ésaü, son fils aîné, les plus beaux qui se trouvaient à la maison, et elle les fit mettre à Jacob, son fils cadet. 
\verse Elle couvrit ses mains de la peau des chevreaux, et son cou qui était sans poil. 
\verse Et elle plaça dans la main de Jacob, son fils, le mets et le pain qu`elle avait préparés. 
\verse Il vint vers son père, et dit: Mon père! Et Isaac dit: Me voici! qui es-tu, mon fils? 
\verse Jacob répondit à son père: Je suis Ésaü, ton fils aîné; j`ai fait ce que tu m`as dit. Lève-toi, je te prie, assieds-toi, et mange de mon gibier, afin que ton âme me bénisse. 
\verse Isaac dit à son fils: Eh quoi! tu en as déjà trouvé, mon fils! Et Jacob répondit: C`est que l`Éternel, ton Dieu, l`a fait venir devant moi. 
\verse Isaac dit à Jacob: Approche donc, et que je te touche, mon fils, pour savoir si tu es mon fils Ésaü, ou non. 
\verse Jacob s`approcha d`Isaac, son père, qui le toucha, et dit: La voix est la voix de Jacob, mais les mains sont les mains d`Ésaü. 
\verse Il ne le reconnut pas, parce que ses mains étaient velues, comme les mains d`Ésaü, son frère; et il le bénit. 
\verse Il dit: C`est toi qui es mon fils Ésaü? Et Jacob répondit: C`est moi. 
\verse Isaac dit: Sers-moi, et que je mange du gibier de mon fils, afin que mon âme te bénisse. Jacob le servit, et il mangea; il lui apporta aussi du vin, et il but. 
\verse Alors Isaac, son père, lui dit: Approche donc, et baise-moi, mon fils. 
\verse Jacob s`approcha, et le baisa. Isaac sentit l`odeur de ses vêtements; puis il le bénit, et dit: Voici, l`odeur de mon fils est comme l`odeur d`un champ que l`Éternel a béni. 
\verse Que Dieu te donne de la rosée du ciel Et de la graisse de la terre, Du blé et du vin en abondance! 
\verse Que des peuples te soient soumis, Et que des nations se prosternent devant toi! Sois le maître de tes frères, Et que les fils de ta mère se prosternent devant toi! Maudit soit quiconque te maudira, Et béni soit quiconque te bénira. 
\verse Isaac avait fini de bénir Jacob, et Jacob avait à peine quitté son père Isaac, qu`Ésaü, son frère, revint de la chasse. 
\verse Il fit aussi un mets, qu`il porta à son père; et il dit à son père: Que mon père se lève et mange du gibier de son fils, afin que ton âme me bénisse! 
\verse Isaac, son père, lui dit: Qui es-tu? Et il répondit: Je suis ton fils aîné, Ésaü. 
\verse Isaac fut saisi d`une grande, d`une violente émotion, et il dit: Qui est donc celui qui a chassé du gibier, et me l`a apporté? J`ai mangé de tout avant que tu vinsses, et je l`ai béni. Aussi sera-t-il béni. 
\verse Lorsque Ésaü entendit les paroles de son père, il poussa de forts cris, pleins d`amertume, et il dit à son père: Bénis-moi aussi, mon père! 
\verse Isaac dit: Ton frère est venu avec ruse, et il a enlevé ta bénédiction. 
\verse Ésaü dit: Est-ce parce qu`on l`a appelé du nom de Jacob qu`il m`a supplanté deux fois? Il a enlevé mon droit d`aînesse, et voici maintenant qu`il vient d`enlever ma bénédiction. Et il dit: N`as-tu point réservé de bénédiction pour moi? 
\verse Isaac répondit, et dit à Ésaü: Voici, je l`ai établi ton maître, et je lui ai donné tous ses frères pour serviteurs, je l`ai pourvu de blé et de vin: que puis-je donc faire pour toi, mon fils? 
\verse Ésaü dit à son père: N`as-tu que cette seule bénédiction, mon père? Bénis-moi aussi, mon père! Et Ésaü éleva la voix, et pleura. 
\verse Isaac, son père, répondit, et lui dit: Voici! Ta demeure sera privée de la graisse de la terre Et de la rosée du ciel, d`en haut. 
\verse Tu vivras de ton épée, Et tu seras asservi à ton frère; Mais en errant librement çà et là, Tu briseras son joug de dessus ton cou. 
\verse Ésaü conçut de la haine contre Jacob, à cause de la bénédiction dont son père l`avait béni; et Ésaü disait en son coeur: Les jours du deuil de mon père vont approcher, et je tuerai Jacob, mon frère. 
\verse On rapporta à Rebecca les paroles d`Ésaü, son fils aîné. Elle fit alors appeler Jacob, son fils cadet, et elle lui dit: Voici, Ésaü, ton frère, veut tirer vengeance de toi, en te tuant. 
\verse Maintenant, mon fils, écoute ma voix! Lève-toi, fuis chez Laban, mon frère, à Charan; 
\verse et reste auprès de lui quelque temps, 
\verse jusqu`à ce que la fureur de ton frère s`apaise, jusqu`à ce que la colère de ton frère se détourne de toi, et qu`il oublie ce que tu lui as fait. Alors je te ferai revenir. Pourquoi serais-je privée de vous deux en un même jour? 
\verse Rebecca dit à Isaac: Je suis dégoûtée de la vie, à cause des filles de Heth. Si Jacob prend une femme, comme celles-ci, parmi les filles de Heth, parmi les filles du pays, à quoi me sert la vie? 

\chapter
\verse Isaac appela Jacob, le bénit, et lui donna cet ordre: Tu ne prendras pas une femme parmi les filles de Canaan. 
\verse Lève-toi, va à Paddan Aram, à la maison de Bethuel, père de ta mère, et prends-y une femme d`entre les filles de Laban, frère de ta mère. 
\verse Que le Dieu tout puissant te bénisse, te rende fécond et te multiplie, afin que tu deviennes une multitude de peuples! 
\verse Qu`il te donne la bénédiction d`Abraham, à toi et à ta postérité avec toi, afin que tu possèdes le pays où tu habites comme étranger, et qu`il a donné à Abraham! 
\verse Et Isaac fit partir Jacob, qui s`en alla à Paddan Aram, auprès de Laban, fils de Bethuel, l`Araméen, frère de Rebecca, mère de Jacob et d`Ésaü. 
\verse Ésaü vit qu`Isaac avait béni Jacob, et qu`il l`avait envoyé à Paddan Aram pour y prendre une femme, et qu`en le bénissant il lui avait donné cet ordre: Tu ne prendras pas une femme parmi les filles de Canaan. 
\verse Il vit que Jacob avait obéi à son père et à sa mère, et qu`il était parti pour Paddan Aram. 
\verse Ésaü comprit ainsi que les filles de Canaan déplaisaient à Isaac, son père. 
\verse Et Ésaü s`en alla vers Ismaël. Il prit pour femme, outre les femmes qu`il avait, Mahalath, fille d`Ismaël, fils d`Abraham, et soeur de Nebajoth. 
\verse Jacob partit de Beer Schéba, et s`en alla à Charan. 
\verse Il arriva dans un lieu où il passa la nuit; car le soleil était couché. Il y prit une pierre, dont il fit son chevet, et il se coucha dans ce lieu-là. 
\verse Il eut un songe. Et voici, une échelle était appuyée sur la terre, et son sommet touchait au ciel. Et voici, les anges de Dieu montaient et descendaient par cette échelle. 
\verse Et voici, l`Éternel se tenait au-dessus d`elle; et il dit: Je suis l`Éternel, le Dieu d`Abraham, ton père, et le Dieu d`Isaac. La terre sur laquelle tu es couché, je la donnerai à toi et à ta postérité. 
\verse Ta postérité sera comme la poussière de la terre; tu t`étendras à l`occident et à l`orient, au septentrion et au midi; et toutes les familles de la terre seront bénies en toi et en ta postérité. 
\verse Voici, je suis avec toi, je te garderai partout où tu iras, et je te ramènerai dans ce pays; car je ne t`abandonnerai point, que je n`aie exécuté ce que je te dis. 
\verse Jacob s`éveilla de son sommeil et il dit: Certainement, l`Éternel est en ce lieu, et moi, je ne le savais pas! 
\verse Il eut peur, et dit: Que ce lieu est redoutable! C`est ici la maison de Dieu, c`est ici la porte des cieux! 
\verse Et Jacob se leva de bon matin; il prit la pierre dont il avait fait son chevet, il la dressa pour monument, et il versa de l`huile sur son sommet. 
\verse Il donna à ce lieu le nom de Béthel; mais la ville s`appelait auparavant Luz. 
\verse Jacob fit un voeu, en disant: Si Dieu est avec moi et me garde pendant ce voyage que je fais, s`il me donne du pain à manger et des habits pour me vêtir, 
\verse et si je retourne en paix à la maison de mon père, alors l`Éternel sera mon Dieu; 
\verse cette pierre, que j`ai dressée pour monument, sera la maison de Dieu; et je te donnerai la dîme de tout ce que tu me donneras. 

\chapter
\verse Jacob se mit en marche, et s`en alla au pays des fils de l`Orient. 
\verse Il regarda. Et voici, il y avait un puits dans les champs; et voici, il y avait à côté trois troupeaux de brebis qui se reposaient, car c`était à ce puits qu`on abreuvait les troupeaux. Et la pierre sur l`ouverture du puits était grande. 
\verse Tous les troupeaux se rassemblaient là; on roulait la pierre de dessus l`ouverture du puits, on abreuvait les troupeaux, et l`on remettait la pierre à sa place sur l`ouverture du puits. 
\verse Jacob dit aux bergers: Mes frères, d`où êtes-vous? Ils répondirent: Nous sommes de Charan. 
\verse Il leur dit: Connaissez-vous Laban, fils de Nachor? Ils répondirent: Nous le connaissons. 
\verse Il leur dit: Est-il en bonne santé? Ils répondirent: Il est en bonne santé; et voici Rachel, sa fille, qui vient avec le troupeau. 
\verse Il dit: Voici, il est encore grand jour, et il n`est pas temps de rassembler les troupeaux; abreuvez les brebis, puis allez, et faites-les paître. 
\verse Ils répondirent: Nous ne le pouvons pas, jusqu`à ce que tous les troupeaux soient rassemblés; c`est alors qu`on roule la pierre de dessus l`ouverture du puits, et qu`on abreuve les brebis. 
\verse Comme il leur parlait encore, survint Rachel avec le troupeau de son père; car elle était bergère. 
\verse Lorsque Jacob vit Rachel, fille de Laban, frère de sa mère, et le troupeau de Laban, frère de sa mère, il s`approcha, roula la pierre de dessus l`ouverture du puits, et abreuva le troupeau de Laban, frère de sa mère. 
\verse Et Jacob baisa Rachel, il éleva la voix et pleura. 
\verse Jacob apprit à Rachel qu`il était parent de son père, qu`il était fils de Rebecca. Et elle courut l`annoncer à son père. 
\verse Dès que Laban eut entendu parler de Jacob, fils de sa soeur, il courut au-devant de lui, il l`embrassa et le baisa, et il le fit venir dans sa maison. Jacob raconta à Laban toutes ces choses. 
\verse Et Laban lui dit: Certainement, tu es mon os et ma chair. Jacob demeura un mois chez Laban. 
\verse Puis Laban dit à Jacob: Parce que tu es mon parent, me serviras-tu pour rien? Dis-moi quel sera ton salaire. 
\verse Or, Laban avait deux filles: l`aînée s`appelait Léa, et la cadette Rachel. 
\verse Léa avait les yeux délicats; mais Rachel était belle de taille et belle de figure. 
\verse Jacob aimait Rachel, et il dit: Je te servirai sept ans pour Rachel, ta fille cadette. 
\verse Et Laban dit: J`aime mieux te la donner que de la donner à un autre homme. Reste chez moi! 
\verse Ainsi Jacob servit sept années pour Rachel: et elles furent à ses yeux comme quelques jours, parce qu`il l`aimait. 
\verse Ensuite Jacob dit à Laban: Donne-moi ma femme, car mon temps est accompli: et j`irai vers elle. 
\verse Laban réunit tous les gens du lieu, et fit un festin. 
\verse Le soir, il prit Léa, sa fille, et l`amena vers Jacob, qui s`approcha d`elle. 
\verse Et Laban donna pour servante à Léa, sa fille, Zilpa, sa servante. 
\verse Le lendemain matin, voilà que c`était Léa. Alors Jacob dit à Laban: Qu`est-ce que tu m`as fait? N`est-ce pas pour Rachel que j`ai servi chez toi? Pourquoi m`as-tu trompé? 
\verse Laban dit: Ce n`est point la coutume dans ce lieu de donner la cadette avant l`aînée. 
\verse Achève la semaine avec celle-ci, et nous te donnerons aussi l`autre pour le service que tu feras encore chez moi pendant sept nouvelles années. 
\verse Jacob fit ainsi, et il acheva la semaine avec Léa; puis Laban lui donna pour femme Rachel, sa fille. 
\verse Et Laban donna pour servante à Rachel, sa fille, Bilha, sa servante. 
\verse Jacob alla aussi vers Rachel, qu`il aimait plus que Léa; et il servit encore chez Laban pendant sept nouvelles années. 
\verse L`Éternel vit que Léa n`était pas aimée; et il la rendit féconde, tandis que Rachel était stérile. 
\verse Léa devint enceinte, et enfanta un fils, à qui elle donna le nom de Ruben; car elle dit: L`Éternel a vu mon humiliation, et maintenant mon mari m`aimera. 
\verse Elle devint encore enceinte, et enfanta un fils, et elle dit: L`Éternel a entendu que je n`étais pas aimée, et il m`a aussi accordé celui-ci. Et elle lui donna le nom de Siméon. 
\verse Elle devint encore enceinte, et enfanta un fils, et elle dit: Pour cette fois, mon mari s`attachera à moi; car je lui ai enfanté trois fils. C`est pourquoi on lui donna le nom de Lévi. 
\verse Elle devint encore enceinte, et enfanta un fils, et elle dit: Cette fois, je louerai l`Éternel. C`est pourquoi elle lui donna le nom de Juda. Et elle cessa d`enfanter. 

\chapter
\verse Lorsque Rachel vit qu`elle ne donnait point d`enfants à Jacob, elle porta envie à sa soeur, et elle dit à Jacob: Donne-moi des enfants, ou je meurs! 
\verse La colère de Jacob s`enflamma contre Rachel, et il dit: Suis-je à la place de Dieu, qui t`empêche d`être féconde? 
\verse Elle dit: Voici ma servante Bilha; va vers elle; qu`elle enfante sur mes genoux, et que par elle j`aie aussi des fils. 
\verse Et elle lui donna pour femme Bilha, sa servante; et Jacob alla vers elle. 
\verse Bilha devint enceinte, et enfanta un fils à Jacob. 
\verse Rachel dit: Dieu m`a rendu justice, il a entendu ma voix, et il m`a donné un fils. C`est pourquoi elle l`appela du nom de Dan. 
\verse Bilha, servante de Rachel, devint encore enceinte, et enfanta un second fils à Jacob. 
\verse Rachel dit: J`ai lutté divinement contre ma soeur, et j`ai vaincu. Et elle l`appela du nom de Nephthali. 
\verse Léa voyant qu`elle avait cessé d`enfanter, prit Zilpa, sa servante, et la donna pour femme à Jacob. 
\verse Zilpa, servante de Léa, enfanta un fils à Jacob. 
\verse Léa dit: Quel bonheur! Et elle l`appela du nom de Gad. 
\verse Zilpa, servante de Léa, enfanta un second fils à Jacob. 
\verse Léa dit: Que je suis heureuse! car les filles me diront heureuse. Et elle l`appela du nom d`Aser. 
\verse Ruben sortit au temps de la moisson des blés, et trouva des mandragores dans les champs. Il les apporta à Léa, sa mère. Alors Rachel dit à Léa: Donne moi, je te prie, des mandragores de ton fils. 
\verse Elle lui répondit: Est-ce peu que tu aies pris mon mari, pour que tu prennes aussi les mandragores de mon fils? Et Rachel dit: Eh bien! il couchera avec toi cette nuit pour les mandragores de ton fils. 
\verse Le soir, comme Jacob revenait des champs, Léa sortit à sa rencontre, et dit: C`est vers moi que tu viendras, car je t`ai acheté pour les mandragores de mon fils. Et il coucha avec elle cette nuit. 
\verse Dieu exauça Léa, qui devint enceinte, et enfanta un cinquième fils à Jacob. 
\verse Léa dit: Dieu m`a donné mon salaire parce que j`ai donné ma servante à mon mari. Et elle l`appela du nom d`Issacar. 
\verse Léa devint encore enceinte, et enfanta un sixième fils à Jacob. 
\verse Léa dit: Dieu m`a fait un beau don; cette fois, mon mari habitera avec moi, car je lui ai enfanté six fils. Et elle l`appela du nom de Zabulon. 
\verse Ensuite, elle enfanta une fille, qu`elle appela du nom de Dina. 
\verse Dieu se souvint de Rachel, il l`exauça, et il la rendit féconde. 
\verse Elle devint enceinte, et enfanta un fils, et elle dit: Dieu a enlevé mon opprobre. 
\verse Et elle lui donna le nom de Joseph, en disant: Que l`Éternel m`ajoute un autre fils! 
\verse Lorsque Rachel eut enfanté Joseph, Jacob dit à Laban: Laisse-moi partir, pour que je m`en aille chez moi, dans mon pays. 
\verse Donne-moi mes femmes et mes enfants, pour lesquels je t`ai servi, et je m`en irai; car tu sais quel service j`ai fait pour toi. 
\verse Laban lui dit: Puissé-je trouver grâce à tes yeux! Je vois bien que l`Éternel m`a béni à cause de toi; 
\verse fixe-moi ton salaire, et je te le donnerai. 
\verse Jacob lui dit: Tu sais comment je t`ai servi, et ce qu`est devenu ton troupeau avec moi; 
\verse car le peu que tu avais avant moi s`est beaucoup accru, et l`Éternel t`a béni sur mes pas. Maintenant, quand travaillerai-je aussi pour ma maison? 
\verse Laban dit: Que te donnerai-je? Et Jacob répondit: Tu ne me donneras rien. Si tu consens à ce que je vais te dire, je ferai paître encore ton troupeau, et je le garderai. 
\verse Je parcourrai aujourd`hui tout ton troupeau; mets à part parmi les brebis tout agneau tacheté et marqueté et tout agneau noir, et parmi les chèvres tout ce qui est marqueté et tacheté. Ce sera mon salaire. 
\verse Ma droiture répondra pour moi demain, quand tu viendras voir mon salaire; tout ce qui ne sera pas tacheté et marqueté parmi les chèvres, et noir parmi les agneaux, ce sera de ma part un vol. 
\verse Laban dit: Eh bien! qu`il en soit selon ta parole. 
\verse Ce même jour, il mit à part les boucs rayés et marquetés, toutes les chèvres tachetées et marquetées, toutes celles où il y avait du blanc, et tout ce qui était noir parmi les brebis. Il les remit entre les mains de ses fils. 
\verse Puis il mit l`espace de trois journées de chemin entre lui et Jacob; et Jacob fit paître le reste du troupeau de Laban. 
\verse Jacob prit des branches vertes de peuplier, d`amandier et de platane; il y pela des bandes blanches, mettant à nu le blanc qui était sur les branches. 
\verse Puis il plaça les branches, qu`il avait pelées, dans les auges, dans les abreuvoirs, sous les yeux des brebis qui venaient boire, pour qu`elles entrassent en chaleur en venant boire. 
\verse Les brebis entraient en chaleur près des branches, et elles faisaient des petits rayés, tachetés et marquetés. 
\verse Jacob séparait les agneaux, et il mettait ensemble ce qui était rayé et tout ce qui était noir dans le troupeau de Laban. Il se fit ainsi des troupeaux à part, qu`il ne réunit point au troupeau de Laban. 
\verse Toutes les fois que les brebis vigoureuses entraient en chaleur, Jacob plaçait les branches dans les auges, sous les yeux des brebis, pour qu`elles entrassent en chaleur près des branches. 
\verse Quand les brebis étaient chétives, il ne les plaçait point; de sorte que les chétives étaient pour Laban, et les vigoureuses pour Jacob. 
\verse Cet homme devint de plus en plus riche; il eut du menu bétail en abondance, des servantes et des serviteurs, des chameaux et des ânes. 

\chapter
\verse Jacob entendit les propos des fils de Laban, qui disaient: Jacob a pris tout ce qui était à notre père, et c`est avec le bien de notre père qu`il s`est acquis toute cette richesse. 
\verse Jacob remarqua aussi le visage de Laban; et voici, il n`était plus envers lui comme auparavant. 
\verse Alors l`Éternel dit à Jacob: Retourne au pays de tes pères et dans ton lieu de naissance, et je serai avec toi. 
\verse Jacob fit appeler Rachel et Léa, qui étaient aux champs vers son troupeau. 
\verse Il leur dit: Je vois, au visage de votre père, qu`il n`est plus envers moi comme auparavant; mais le Dieu de mon père a été avec moi. 
\verse Vous savez vous-mêmes que j`ai servi votre père de tout mon pouvoir. 
\verse Et votre père s`est joué de moi, et a changé dix fois mon salaire; mais Dieu ne lui a pas permis de me faire du mal. 
\verse Quand il disait: Les tachetés seront ton salaire, toutes les brebis faisaient des petits tachetés. Et quand il disait: Les rayés seront ton salaire, toutes les brebis faisaient des petits rayés. 
\verse Dieu a pris à votre père son troupeau, et me l`a donné. 
\verse Au temps où les brebis entraient en chaleur, je levai les yeux, et je vis en songe que les boucs qui couvraient les brebis étaient rayés, tachetés et marquetés. 
\verse Et l`ange de Dieu me dit en songe: Jacob! Je répondis: Me voici! 
\verse Il dit: Lève les yeux, et regarde: tous les boucs qui couvrent les brebis sont rayés, tachetés et marquetés; car j`ai vu tout ce que te fait Laban. 
\verse Je suis le Dieu de Béthel, où tu as oint un monument, où tu m`as fait un voeu. Maintenant, lève-toi, sors de ce pays, et retourne au pays de ta naissance. 
\verse Rachel et Léa répondirent, et lui dirent: Avons-nous encore une part et un héritage dans la maison de notre père? 
\verse Ne sommes-nous pas regardées par lui comme des étrangères, puisqu`il nous a vendues, et qu`il a mangé notre argent? 
\verse Toute la richesse que Dieu a ôtée à notre père appartient à nous et à nos enfants. Fais maintenant tout ce que Dieu t`a dit. 
\verse Jacob se leva, et il fit monter ses enfants et ses femmes sur les chameaux. 
\verse Il emmena tout son troupeau et tous les biens qu`il possédait, le troupeau qui lui appartenait, qu`il avait acquis à Paddan Aram; et il s`en alla vers Isaac, son père, au pays de Canaan. 
\verse Tandis que Laban était allé tondre ses brebis, Rachel déroba les théraphim de son père; 
\verse et Jacob trompa Laban, l`Araméen, en ne l`avertissant pas de sa fuite. 
\verse Il s`enfuit, avec tout ce qui lui appartenait; il se leva, traversa le fleuve, et se dirigea vers la montagne de Galaad. 
\verse Le troisième jour, on annonça à Laban que Jacob s`était enfui. 
\verse Il prit avec lui ses frères, le poursuivit sept journées de marche, et l`atteignit à la montagne de Galaad. 
\verse Mais Dieu apparut la nuit en songe à Laban, l`Araméen, et lui dit: Garde-toi de parler à Jacob ni en bien ni en mal! 
\verse Laban atteignit donc Jacob. Jacob avait dressé sa tente sur la montagne; Laban dressa aussi la sienne, avec ses frères, sur la montagne de Galaad. 
\verse Alors Laban dit à Jacob: Qu`as-tu fait? Pourquoi m`as-tu trompé, et emmènes-tu mes filles comme des captives par l`épée? 
\verse Pourquoi as-tu pris la fuite en cachette, m`as-tu trompé, et ne m`as-tu point averti? Je t`aurais laissé partir au milieu des réjouissances et des chants, au son du tambourin et de la harpe. 
\verse Tu ne m`as pas permis d`embrasser mes fils et mes filles! C`est en insensé que tu as agi. 
\verse Ma main est assez forte pour vous faire du mal; mais le Dieu de votre père m`a dit hier: Garde-toi de parler à Jacob ni en bien ni en mal! 
\verse Maintenant que tu es parti, parce que tu languissais après la maison de ton père, pourquoi as-tu dérobé mes dieux? 
\verse Jacob répondit, et dit à Laban: J`avais de la crainte à la pensée que tu m`enlèverais peut-être tes filles. 
\verse Mais périsse celui auprès duquel tu trouveras tes dieux! En présence de nos frères, examine ce qui t`appartient chez moi, et prends-le. Jacob ne savait pas que Rachel les eût dérobés. 
\verse Laban entra dans la tente de Jacob, dans la tente de Léa, dans la tente des deux servantes, et il ne trouva rien. Il sortit de la tente de Léa, et entra dans la tente de Rachel. 
\verse Rachel avait pris les théraphim, les avait mis sous le bât du chameau, et s`était assise dessus. Laban fouilla toute la tente, et ne trouva rien. 
\verse Elle dit à son père: Que mon seigneur ne se fâche point, si je ne puis me lever devant toi, car j`ai ce qui est ordinaire aux femmes. Il chercha, et ne trouva point les théraphim. 
\verse Jacob s`irrita, et querella Laban. Il reprit la parole, et lui dit: Quel est mon crime, quel est mon péché, que tu me poursuives avec tant d`ardeur? 
\verse Quand tu as fouillé tous mes effets, qu`as-tu trouvé des effets de ta maison? Produis-le ici devant mes frères et tes frères, et qu`ils prononcent entre nous deux. 
\verse Voilà vingt ans que j`ai passés chez toi; tes brebis et tes chèvres n`ont point avorté, et je n`ai point mangé les béliers de ton troupeau. 
\verse Je ne t`ai point rapporté de bêtes déchirées, j`en ai payé le dommage; tu me redemandais ce qu`on me volait de jour et ce qu`on me volait de nuit. 
\verse La chaleur me dévorait pendant le jour, et le froid pendant la nuit, et le sommeil fuyait de mes yeux. 
\verse Voilà vingt ans que j`ai passés dans ta maison; je t`ai servi quatorze ans pour tes deux filles, et six ans pour ton troupeau, et tu as changé dix fois mon salaire. 
\verse Si je n`eusse pas eu pour moi le Dieu de mon père, le Dieu d`Abraham, celui que craint Isaac, tu m`aurais maintenant renvoyé à vide. Dieu a vu ma souffrance et le travail de mes mains, et hier il a prononcé son jugement. 
\verse Laban répondit, et dit à Jacob: Ces filles sont mes filles, ces enfants sont mes enfants, ce troupeau est mon troupeau, et tout ce que tu vois est à moi. Et que puis-je faire aujourd`hui pour mes filles, ou pour leurs enfants qu`elles ont mis au monde? 
\verse Viens, faisons alliance, moi et toi, et que cela serve de témoignage entre moi et toi! 
\verse Jacob prit une pierre, et il la dressa pour monument. 
\verse Jacob dit à ses frères: Ramassez des pierres. Ils prirent des pierres, et firent un monceau; et ils mangèrent là sur le monceau. 
\verse Laban l`appela Jegar Sahadutha, et Jacob l`appela Galed. 
\verse Laban dit: Que ce monceau serve aujourd`hui de témoignage entre moi et toi! C`est pourquoi on lui a donné le nom de Galed. 
\verse On l`appelle aussi Mitspa, parce que Laban dit: Que l`Éternel veille sur toi et sur moi, quand nous nous serons l`un et l`autre perdus de vue. 
\verse Si tu maltraites mes filles, et si tu prends encore d`autres femmes, ce n`est pas un homme qui sera avec nous, prends-y garde, c`est Dieu qui sera témoin entre moi et toi. 
\verse Laban dit à Jacob: Voici ce monceau, et voici ce monument que j`ai élevé entre moi et toi. 
\verse Que ce monceau soit témoin et que ce monument soit témoin que je n`irai point vers toi au delà de ce monceau, et que tu ne viendras point vers moi au delà de ce monceau et de ce monument, pour agir méchamment. 
\verse Que le Dieu d`Abraham et de Nachor, que le Dieu de leur père soit juge entre nous. Jacob jura par celui que craignait Isaac. 
\verse Jacob offrit un sacrifice sur la montagne, et il invita ses frères à manger; ils mangèrent donc, et passèrent la nuit sur la montagne. 
\verse Laban se leva de bon matin, baisa ses fils et ses filles, et les bénit. Ensuite il partit pour retourner dans sa demeure. 

\chapter
\verse Jacob poursuivit son chemin; et des anges de Dieu le rencontrèrent. 
\verse En les voyant, Jacob dit: C`est le camp de Dieu! Et il donna à ce lieu le nom de Mahanaïm. 
\verse Jacob envoya devant lui des messagers à Ésaü, son frère, au pays de Séir, dans le territoire d`Édom. 
\verse Il leur donna cet ordre: Voici ce que vous direz à mon seigneur Ésaü: Ainsi parle ton serviteur Jacob: J`ai séjourné chez Laban, et j`y suis resté jusqu`à présent; 
\verse j`ai des boeufs, des ânes, des brebis, des serviteurs et des servantes, et j`envoie l`annoncer à mon seigneur, pour trouver grâce à tes yeux. 
\verse Les messagers revinrent auprès de Jacob, en disant: Nous sommes allés vers ton frère Ésaü; et il marche à ta rencontre, avec quatre cents hommes. 
\verse Jacob fut très effrayé, et saisi d`angoisse. Il partagea en deux camps les gens qui étaient avec lui, les brebis, les boeufs et les chameaux; 
\verse et il dit: Si Ésaü vient contre l`un des camps et le bat, le camp qui restera pourra se sauver. 
\verse Jacob dit: Dieu de mon père Abraham, Dieu de mon père Isaac, Éternel, qui m`as dit: Retourne dans ton pays et dans ton lieu de naissance, et je te ferai du bien! 
\verse Je suis trop petit pour toutes les grâces et pour toute la fidélité dont tu as usé envers ton serviteur; car j`ai passé ce Jourdain avec mon bâton, et maintenant je forme deux camps. 
\verse Délivre-moi, je te prie, de la main de mon frère, de la main d`Ésaü! car je crains qu`il ne vienne, et qu`il ne me frappe, avec la mère et les enfants. 
\verse Et toi, tu as dit: Je te ferai du bien, et je rendrai ta postérité comme le sable de la mer, si abondant qu`on ne saurait le compter. 
\verse C`est dans ce lieu-là que Jacob passa la nuit. Il prit de ce qu`il avait sous la main, pour faire un présent à Ésaü, son frère: 
\verse deux cents chèvres et vingt boucs, deux cents brebis et vingt béliers, 
\verse trente femelles de chameaux avec leurs petits qu`elles allaitaient, quarante vaches et dix taureaux, vingt ânesses et dix ânes. 
\verse Il les remit à ses serviteurs, troupeau par troupeau séparément, et il dit à ses serviteurs: Passez devant moi, et mettez un intervalle entre chaque troupeau. 
\verse Il donna cet ordre au premier: Quand Ésaü, mon frère, te rencontrera, et te demandera: A qui es-tu? où vas-tu? et à qui appartient ce troupeau devant toi? 
\verse tu répondras: A ton serviteur Jacob; c`est un présent envoyé à mon seigneur Ésaü; et voici, il vient lui-même derrière nous. 
\verse Il donna le même ordre au second, au troisième, et à tous ceux qui suivaient les troupeaux: C`est ainsi que vous parlerez à mon seigneur Ésaü, quand vous le rencontrerez. 
\verse Vous direz: Voici, ton serviteur Jacob vient aussi derrière nous. Car il se disait: Je l`apaiserai par ce présent qui va devant moi; ensuite je le verrai en face, et peut-être m`accueillera-t-il favorablement. 
\verse Le présent passa devant lui; et il resta cette nuit-là dans le camp. 
\verse Il se leva la même nuit, prit ses deux femmes, ses deux servantes, et ses onze enfants, et passa le gué de Jabbok. 
\verse Il les prit, leur fit passer le torrent, et le fit passer à tout ce qui lui appartenait. 
\verse Jacob demeura seul. Alors un homme lutta avec lui jusqu`au lever de l`aurore. 
\verse Voyant qu`il ne pouvait le vaincre, cet homme le frappa à l`emboîture de la hanche; et l`emboîture de la hanche de Jacob se démit pendant qu`il luttait avec lui. 
\verse Il dit: Laisse-moi aller, car l`aurore se lève. Et Jacob répondit: Je ne te laisserai point aller, que tu ne m`aies béni. 
\verse Il lui dit: Quel est ton nom? Et il répondit: Jacob. 
\verse Il dit encore: ton nom ne sera plus Jacob, mais tu seras appelé Israël; car tu as lutté avec Dieu et avec des hommes, et tu as été vainqueur. 
\verse Jacob l`interrogea, en disant: Fais-moi je te prie, connaître ton nom. Il répondit: Pourquoi demandes-tu mon nom? Et il le bénit là. 
\verse Jacob appela ce lieu du nom de Peniel: car, dit-il, j`ai vu Dieu face à face, et mon âme a été sauvée. 
\verse Le soleil se levait, lorsqu`il passa Peniel. Jacob boitait de la hanche. 
\verse C`est pourquoi jusqu`à ce jour, les enfants d`Israël ne mangent point le tendon qui est à l`emboîture de la hanche; car Dieu frappa Jacob à l`emboîture de la hanche, au tendon. 

\chapter
\verse Jacob leva les yeux, et regarda; et voici, Ésaü arrivait, avec quatre cents hommes. Il répartit les enfants entre Léa, Rachel, et les deux servantes. 
\verse Il plaça en tête les servantes avec leurs enfants, puis Léa avec ses enfants, et enfin Rachel avec Joseph. 
\verse Lui-même passa devant eux; et il se prosterna en terre sept fois, jusqu`à ce qu`il fût près de son frère. 
\verse Ésaü courut à sa rencontre; il l`embrassa, se jeta à son cou, et le baisa. Et ils pleurèrent. 
\verse Ésaü, levant les yeux, vit les femmes et les enfants, et il dit: Qui sont ceux que tu as là? Et Jacob répondit: Ce sont les enfants que Dieu a accordés à ton serviteur. 
\verse Les servantes s`approchèrent, elles et leurs enfants, et se prosternèrent; 
\verse Léa et ses enfants s`approchèrent aussi, et se prosternèrent; ensuite Joseph et Rachel s`approchèrent, et se prosternèrent. 
\verse Ésaü dit: A quoi destines-tu tout ce camp que j`ai rencontré? Et Jacob répondit: A trouver grâce aux yeux de mon seigneur. 
\verse Ésaü dit: Je suis dans l`abondance, mon frère; garde ce qui est à toi. 
\verse Et Jacob répondit: Non, je te prie, si j`ai trouvé grâce à tes yeux, accepte de ma main mon présent; car c`est pour cela que j`ai regardé ta face comme on regarde la face de Dieu, et tu m`as accueilli favorablement. 
\verse Accepte donc mon présent qui t`a été offert, puisque Dieu m`a comblé de grâces, et que je ne manque de rien. Il insista auprès de lui, et Ésaü accepta. 
\verse Ésaü dit: Partons, mettons-nous en route; j`irai devant toi. 
\verse Jacob lui répondit: Mon seigneur sait que les enfants sont délicats, et que j`ai des brebis et des vaches qui allaitent; si l`on forçait leur marche un seul jour, tout le troupeau périrait. 
\verse Que mon seigneur prenne les devants sur son serviteur; et moi, je suivrai lentement, au pas du troupeau qui me précédera, et au pas des enfants, jusqu`à ce que j`arrive chez mon seigneur, à Séir. 
\verse Ésaü dit: Je veux au moins laisser avec toi une partie de mes gens. Et Jacob répondit: Pourquoi cela? Que je trouve seulement grâce aux yeux de mon seigneur! 
\verse Le même jour, Ésaü reprit le chemin de Séir. 
\verse Jacob partit pour Succoth. Il bâtit une maison pour lui, et il fit des cabanes pour ses troupeaux. C`est pourquoi l`on a appelé ce lieu de nom de Succoth. 
\verse A son retour de Paddan Aram, Jacob arriva heureusement à la ville de Sichem, dans le pays de Canaan, et il campa devant la ville. 
\verse Il acheta la portion du champ où il avait dressé sa tente, des fils d`Hamor, père de Sichem, pour cent kesita. 
\verse Et là, il éleva un autel, qu`il appela El Élohé Israël. 

\chapter
\verse Dina, la fille que Léa avait enfantée à Jacob, sortit pour voir les filles du pays. 
\verse Elle fut aperçue de Sichem, fils de Hamor, prince du pays. Il l`enleva, coucha avec elle, et la déshonora. 
\verse Son coeur s`attacha à Dina, fille de Jacob; il aima la jeune fille, et sut parler à son coeur. 
\verse Et Sichem dit à Hamor, son père: Donne-moi cette jeune fille pour femme. 
\verse Jacob apprit qu`il avait déshonoré Dina, sa fille; et, comme ses fils étaient aux champs avec son troupeau, Jacob garda le silence jusqu`à leur retour. 
\verse Hamor, père de Sichem, se rendit auprès de Jacob pour lui parler. 
\verse Et les fils de Jacob revenaient des champs, lorsqu`ils apprirent la chose; ces hommes furent irrités et se mirent dans une grande colère, parce que Sichem avait commis une infamie en Israël, en couchant avec la fille de Jacob, ce qui n`aurait pas dû se faire. 
\verse Hamor leur adressa ainsi la parole: Le coeur de Sichem, mon fils, s`est attaché à votre fille; donnez-la-lui pour femme, je vous prie. 
\verse Alliez-vous avec nous; vous nous donnerez vos filles, et vous prendrez pour vous les nôtres. 
\verse Vous habiterez avec nous, et le pays sera à votre disposition; restez, pour y trafiquer et y acquérir des propriétés. 
\verse Sichem dit au père et aux frères de Dina: Que je trouve grâce à vos yeux, et je donnerai ce que vous me direz. 
\verse Exigez de moi une forte dot et beaucoup de présents, et je donnerai ce que vous me direz; mais accordez-moi pour femme la jeune fille. 
\verse Les fils de Jacob répondirent et parlèrent avec ruse à Sichem et à Hamor, son père, parce que Sichem avait déshonoré Dina, leur soeur. 
\verse Ils leur dirent: C`est une chose que nous ne pouvons pas faire, que de donner notre soeur à un homme incirconcis; car ce serait un opprobre pour nous. 
\verse Nous ne consentirons à votre désir qu`à la condition que vous deveniez comme nous, et que tout mâle parmi vous soit circoncis. 
\verse Nous vous donnerons alors nos filles, et nous prendrons pour nous les vôtres; nous habiterons avec vous, et nous formerons un seul peuple. 
\verse Mais si vous ne voulez pas nous écouter et vous faire circoncire, nous prendrons notre fille, et nous nous en irons. 
\verse Leurs paroles eurent l`assentiment de Hamor et de Sichem, fils de Hamor. 
\verse Le jeune homme ne tarda pas à faire la chose, car il aimait la fille de Jacob. Il était considéré de tous dans la maison de son père. 
\verse Hamor et Sichem, son fils, se rendirent à la porte de leur ville, et ils parlèrent ainsi aux gens de leur ville: 
\verse Ces hommes sont paisibles à notre égard; qu`ils restent dans le pays, et qu`ils y trafiquent; le pays est assez vaste pour eux. Nous prendrons pour femmes leurs filles, et nous leur donnerons nos filles. 
\verse Mais ces hommes ne consentiront à habiter avec nous, pour former un seul peuple, qu`à la condition que tout mâle parmi nous soit circoncis, comme ils sont eux-mêmes circoncis. 
\verse Leurs troupeaux, leurs biens et tout leur bétail, ne seront-ils pas à nous? Acceptons seulement leur condition, pour qu`ils restent avec nous. 
\verse Tous ceux qui étaient venus à la porte de la ville écoutèrent Hamor et Sichem, son fils; et tous les mâles se firent circoncire, tous ceux qui étaient venus à la porte de la ville. 
\verse Le troisième jour, pendant qu`ils étaient souffrants, les deux fils de Jacob, Siméon et Lévi, frères de Dina, prirent chacun leur épée, tombèrent sur la ville qui se croyait en sécurité, et tuèrent tous les mâles. 
\verse Ils passèrent aussi au fil de l`épée Hamor et Sichem, son fils; ils enlevèrent Dina de la maison de Sichem, et sortirent. 
\verse Les fils de Jacob se jetèrent sur les morts, et pillèrent la ville, parce qu`on avait déshonoré leur soeur. 
\verse Ils prirent leurs troupeaux, leurs boeufs et leurs ânes, ce qui était dans la ville et ce qui était dans les champs; 
\verse ils emmenèrent comme butin toutes leurs richesses, leurs enfants et leurs femmes, et tout ce qui se trouvait dans les maisons. 
\verse Alors Jacob dit à Siméon et à Lévi: Vous me troublez, en me rendant odieux aux habitants du pays, aux Cananéens et aux Phérésiens. Je n`ai qu`un petit nombre d`hommes; et ils se rassembleront contre moi, ils me frapperont, et je serai détruit, moi et ma maison. 
\verse Ils répondirent: Traitera-t-on notre soeur comme une prostituée? 

\chapter
\verse Dieu dit à Jacob: Lève-toi, monte à Béthel, et demeures-y; là, tu dresseras un autel au Dieu qui t`apparut, lorsque tu fuyais Ésaü, ton frère. 
\verse Jacob dit à sa maison et à tous ceux qui étaient avec lui: Otez les dieux étrangers qui sont au milieu de vous, purifiez-vous, et changez de vêtements. 
\verse Nous nous lèverons, et nous monterons à Béthel; là, je dresserai un autel au Dieu qui m`a exaucé dans le jour de ma détresse, et qui a été avec moi pendant le voyage que j`ai fait. 
\verse Ils donnèrent à Jacob tous les dieux étrangers qui étaient entre leurs mains, et les anneaux qui étaient à leurs oreilles. Jacob les enfouit sous le térébinthe qui est près de Sichem. 
\verse Ensuite ils partirent. La terreur de Dieu se répandit sur les villes qui les entouraient, et l`on ne poursuivit point les fils de Jacob. 
\verse Jacob arriva, lui et tous ceux qui étaient avec lui, à Luz, qui est Béthel, dans le pays de Canaan. 
\verse Il bâtit là un autel, et il appela ce lieu El Béthel; car c`est là que Dieu s`était révélé à lui lorsqu`il fuyait son frère. 
\verse Débora, nourrice de Rebecca, mourut; et elle fut enterrée au-dessous de Béthel, sous le chêne auquel on a donné le nom de chêne des pleurs. 
\verse Dieu apparut encore à Jacob, après son retour de Paddan Aram, et il le bénit. 
\verse Dieu lui dit: Ton nom est Jacob; tu ne seras plus appelé Jacob, mais ton nom sera Israël. Et il lui donna le nom d`Israël. 
\verse Dieu lui dit: Je suis le Dieu tout puissant. Sois fécond, et multiplie: une nation et une multitude de nations naîtront de toi, et des rois sortiront de tes reins. 
\verse Je te donnerai le pays que j`ai donné à Abraham et à Isaac, et je donnerai ce pays à ta postérité après toi. 
\verse Dieu s`éleva au-dessus de lui, dans le lieu où il lui avait parlé. 
\verse Et Jacob dressa un monument dans le lieu où Dieu lui avait parlé, un monument de pierres, sur lequel il fit une libation et versa de l`huile. 
\verse Jacob donna le nom de Béthel au lieu où Dieu lui avait parlé. 
\verse Ils partirent de Béthel; et il y avait encore une certaine distance jusqu`à Éphrata, lorsque Rachel accoucha. Elle eut un accouchement pénible; 
\verse et pendant les douleurs de l`enfantement, la sage-femme lui dit: Ne crains point, car tu as encore un fils! 
\verse Et comme elle allait rendre l`âme, car elle était mourante, elle lui donna le nom de Ben Oni; mais le père l`appela Benjamin. 
\verse Rachel mourut, et elle fut enterrée sur le chemin d`Éphrata, qui est Bethléhem. 
\verse Jacob éleva un monument sur son sépulcre; c`est le monument du sépulcre de Rachel, qui existe encore aujourd`hui. 
\verse Israël partit; et il dressa sa tente au delà de Migdal Éder. 
\verse Pendant qu`Israël habitait cette contrée, Ruben alla coucher avec Bilha, concubine de son père. Et Israël l`apprit. Les fils de Jacob étaient au nombre de douze. 
\verse Fils de Léa: Ruben, premier-né de Jacob, Siméon, Lévi, Juda, Issacar et Zabulon. 
\verse Fils de Rachel: Joseph et Benjamin. 
\verse Fils de Bilha, servante de Rachel: Dan et Nephthali. 
\verse Fils de Zilpa, servante de Léa: Gad et Aser. Ce sont là les fils de Jacob, qui lui naquirent à Paddan Aram. 
\verse Jacob arriva auprès d`Isaac, son père, à Mamré, à Kirjath Arba, qui est Hébron, où avaient séjourné Abraham et Isaac. 
\verse Les jours d`Isaac furent de cent quatre-vingts ans. 
\verse Il expira et mourut, et il fut recueilli auprès de son peuple, âgé et rassasié de jours, et Ésaü et Jacob, ses fils, l`enterrèrent. 

\chapter
\verse Voici la postérité d`Ésaü, qui est Édom. 
\verse Ésaü prit ses femmes parmi les filles de Canaan: Ada, fille d`Élon, le Héthien; Oholibama, fille d`Ana, fille de Tsibeon, le Hévien; 
\verse et Basmath, fille d`Ismaël, soeur de Nebajoth. 
\verse Ada enfanta à Ésaü Éliphaz; Basmath enfanta Réuel; 
\verse et Oholibama enfanta Jéusch, Jaelam et Koré. Ce sont là les fils d`Ésaü, qui lui naquirent dans le pays de Canaan. 
\verse Ésaü prit ses femmes, ses fils et ses filles, toutes les personnes de sa maison, ses troupeaux, tout son bétail, et tout le bien qu`il avait acquis au pays de Canaan, et il s`en alla dans un autre pays, loin de Jacob, son frère. 
\verse Car leurs richesses étaient trop considérables pour qu`ils demeurassent ensemble, et la contrée où ils séjournaient ne pouvait plus leur suffire à cause de leurs troupeaux. 
\verse Ésaü s`établit dans la montagne de Séir. Ésaü, c`est Édom. 
\verse Voici la postérité d`Ésaü, père d`Édom, dans la montagne de Séir. 
\verse Voici les noms des fils d`Ésaü: Éliphaz, fils d`Ada, femme d`Ésaü; Réuel, fils de Basmath, femme d`Ésaü. 
\verse Les fils d`Éliphaz furent: Théman, Omar, Tsepho, Gaetham et Kenaz. 
\verse Et Thimna était la concubine d`Éliphaz, fils d`Ésaü: elle enfanta à Éliphaz Amalek. Ce sont là les fils d`Ada, femme d`Ésaü. 
\verse Voici les fils de Réuel: Nahath, Zérach, Schamma et Mizza. Ce sont là les fils de Basmath, femme d`Ésaü. 
\verse Voici les fils d`Oholibama, fille d`Ana, fille de Tsibeon, femme d`Ésaü: elle enfanta à Ésaü Jéusch, Jaelam et Koré. 
\verse Voici les chefs de tribus issues des fils d`Ésaü. -Voici les fils d`Éliphaz, premier-né d`Ésaü: le chef Théman, le chef Omar, le chef Tsepho, le chef Kenaz, 
\verse le chef Koré, le chef Gaetham, le chef Amalek. Ce sont là les chefs issus d`Éliphaz, dans le pays d`Édom. Ce sont les fils d`Ada. 
\verse Voici les fils de Réuel, fils d`Ésaü: le chef Nahath, le chef Zérach, le chef Schamma, le chef Mizza. Ce sont là les chefs issus de Réuel, dans le pays d`Édom. Ce sont là les fils de Basmath, femme d`Ésaü. 
\verse Voici les fils d`Oholibama, femme d`Ésaü: le chef Jéusch, le chef Jaelam, le chef Koré. Ce sont là les chefs issus d`Oholibama, fille d`Ana, femme d`Ésaü. 
\verse Ce sont là les fils d`Ésaü, et ce sont là leurs chefs de tribus. Ésaü, c`est Édom. 
\verse Voici les fils de Séir, le Horien, anciens habitants du pays: Lothan, Schobal, Tsibeon, Ana, 
\verse Dischon, Etser, et Dischan. Ce sont là les chefs des Horiens, fils de Séir, dans le pays d`Édom. 
\verse Les fils de Lothan furent: Hori et Hémam. La soeur de Lothan fut Thimna. 
\verse Voici les fils de Schobal: Alvan, Manahath, Ébal, Schepho et Onam. 
\verse Voici les fils de Tsibeon: Ajja et Ana. C`est cet Ana qui trouva les sources chaudes dans le désert, quand il faisait paître les ânes de Tsibeon, son père. 
\verse Voici les enfants d`Ana: Dischon, et Oholibama, fille d`Ana. 
\verse Voici les fils de Dischon: Hemdan, Eschban, Jithran et Karen. 
\verse Voici les fils d`Etser: Bilhan, Zaavan et Akan. 
\verse Voici les fils de Dischan: Uts et Aran. 
\verse Voici les chefs des Horiens: le chef Lothan, le chef Schobal, le chef Tsibeon, le chef Ana, 
\verse le chef Dischon, le chef Etser, le chef Dischan. Ce sont là les chefs des Horiens, les chefs qu`ils eurent dans le pays de Séir. 
\verse Voici les rois qui ont régné dans le pays d`Édom, avant qu`un roi régnât sur les enfants d`Israël. 
\verse Béla, fils de Béor, régna sur Édom; et le nom de sa ville était Dinhaba. 
\verse Béla mourut; et Jobab, fils de Zérach, de Botsra, régna à sa place. 
\verse Jobab mourut; et Huscham, du pays des Thémanites, régna à sa place. 
\verse Huscham mourut; et Hadad, fils de Bedad, régna à sa place. C`est lui qui frappa Madian dans les champs de Moab. Le nom de sa ville était Avith. 
\verse Hadad mourut; et Samla, de Masréka, régna à sa place. 
\verse Samla mourut; et Saül, de Rehoboth sur le fleuve, régna à sa place. 
\verse Saül mourut; et Baal Hanan, fils d`Acbor, régna à sa place. 
\verse Baal Hanan, fils d`Acbor, mourut; et Hadar régna à sa place. Le nom de sa ville était Pau; et le nom de sa femme Mehéthabeel, fille de Mathred, fille de Mézahab. 
\verse Voici les noms des chefs issus d`Ésaü, selon leurs tribus, selon leurs territoires, et d`après leurs noms: le chef Thimna, le chef Alva, le chef Jetheth, 
\verse le chef Oholibama, le chef Éla, le chef Pinon, 
\verse le chef Kenaz, le chef Théman, le chef Mibtsar, 
\verse le chef Magdiel, le chef Iram. Ce sont là les chefs d`Édom, selon leurs habitations dans le pays qu`ils possédaient. C`est là Ésaü, père d`Édom. 

\chapter
\verse Jacob demeura dans le pays de Canaan, où avait séjourné son père. 
\verse Voici la postérité de Jacob. Joseph, âgé de dix-sept ans, faisait paître le troupeau avec ses frères; cet enfant était auprès des fils de Bilha et des fils de Zilpa, femmes de son père. Et Joseph rapportait à leur père leurs mauvais propos. 
\verse Israël aimait Joseph plus que tous ses autres fils, parce qu`il l`avait eu dans sa vieillesse; et il lui fit une tunique de plusieurs couleurs. 
\verse Ses frères virent que leur père l`aimait plus qu`eux tous, et ils le prirent en haine. Ils ne pouvaient lui parler avec amitié. 
\verse Joseph eut un songe, et il le raconta à ses frères, qui le haïrent encore davantage. 
\verse Il leur dit: Écoutez donc ce songe que j`ai eu! 
\verse Nous étions à lier des gerbes au milieu des champs; et voici, ma gerbe se leva et se tint debout, et vos gerbes l`entourèrent et se prosternèrent devant elle. 
\verse Ses frères lui dirent: Est-ce que tu régneras sur nous? est-ce que tu nous gouverneras? Et ils le haïrent encore davantage, à cause de ses songes et à cause de ses paroles. 
\verse Il eut encore un autre songe, et il le raconta à ses frères. Il dit: J`ai eu encore un songe! Et voici, le soleil, la lune et onze étoiles se prosternaient devant moi. 
\verse Il le raconta à son père et à ses frères. Son père le réprimanda, et lui dit: Que signifie ce songe que tu as eu? Faut-il que nous venions, moi, ta mère et tes frères, nous prosterner en terre devant toi? 
\verse Ses frères eurent de l`envie contre lui, mais son père garda le souvenir de ces choses. 
\verse Les frères de Joseph étant allés à Sichem, pour faire paître le troupeau de leur père, 
\verse Israël dit à Joseph: Tes frères ne font-ils pas paître le troupeau à Sichem? Viens, je veux t`envoyer vers eux. Et il répondit: Me voici! 
\verse Israël lui dit: Va, je te prie, et vois si tes frères sont en bonne santé et si le troupeau est en bon état; et tu m`en rapporteras des nouvelles. Il l`envoya ainsi de la vallée d`Hébron; et Joseph alla à Sichem. 
\verse Un homme le rencontra, comme il errait dans les champs. Il le questionna, en disant: Que cherches-tu? 
\verse Joseph répondit: Je cherche mes frères; dis-moi, je te prie, où ils font paître leur troupeau. 
\verse Et l`homme dit: Ils sont partis d`ici; car je les ai entendus dire: Allons à Dothan. Joseph alla après ses frères, et il les trouva à Dothan. 
\verse Ils le virent de loin; et, avant qu`il fût près d`eux, ils complotèrent de le faire mourir. 
\verse Ils se dirent l`un à l`autre: Voici le faiseur de songes qui arrive. 
\verse Venez maintenant, tuons-le, et jetons-le dans une des citernes; nous dirons qu`une bête féroce l`a dévoré, et nous verrons ce que deviendront ses songes. 
\verse Ruben entendit cela, et il le délivra de leurs mains. Il dit: Ne lui ôtons pas la vie. 
\verse Ruben leur dit: Ne répandez point de sang; jetez-le dans cette citerne qui est au désert, et ne mettez pas la main sur lui. Il avait dessein de le délivrer de leurs mains pour le faire retourner vers son père. 
\verse Lorsque Joseph fut arrivé auprès de ses frères, ils le dépouillèrent de sa tunique, de la tunique de plusieurs couleurs, qu`il avait sur lui. 
\verse Ils le prirent, et le jetèrent dans la citerne. Cette citerne était vide; il n`y avait point d`eau. 
\verse Ils s`assirent ensuite pour manger. Ayant levé les yeux, ils virent une caravane d`Ismaélites venant de Galaad; leurs chameaux étaient chargés d`aromates, de baume et de myrrhe, qu`ils transportaient en Égypte. 
\verse Alors Juda dit à ses frères: Que gagnerons-nous à tuer notre frère et à cacher son sang? 
\verse Venez, vendons-le aux Ismaélites, et ne mettons pas la main sur lui, car il est notre frère, notre chair. Et ses frères l`écoutèrent. 
\verse Au passage des marchands madianites, ils tirèrent et firent remonter Joseph hors de la citerne; et ils le vendirent pour vingt sicles d`argent aux Ismaélites, qui l`emmenèrent en Égypte. 
\verse Ruben revint à la citerne; et voici, Joseph n`était plus dans la citerne. Il déchira ses vêtements, 
\verse retourna vers ses frères, et dit: L`enfant n`y est plus! Et moi, où irai-je? 
\verse Ils prirent alors la tunique de Joseph; et, ayant tué un bouc, ils plongèrent la tunique dans le sang. 
\verse Ils envoyèrent à leur père la tunique de plusieurs couleurs, en lui faisant dire: Voici ce que nous avons trouvé! reconnais si c`est la tunique de ton fils, ou non. 
\verse Jacob la reconnut, et dit: C`est la tunique de mon fils! une bête féroce l`a dévoré! Joseph a été mis en pièces! 
\verse Et il déchira ses vêtements, il mit un sac sur ses reins, et il porta longtemps le deuil de son fils. 
\verse Tous ses fils et toutes ses filles vinrent pour le consoler; mais il ne voulut recevoir aucune consolation. Il disait: C`est en pleurant que je descendrai vers mon fils au séjour des morts! Et il pleurait son fils. 
\verse Les Madianites le vendirent en Égypte à Potiphar, officier de Pharaon, chef des gardes. 

\chapter
\verse En ce temps-là, Juda s`éloigna de ses frères, et se retira vers un homme d`Adullam, nommé Hira. 
\verse Là, Juda vit la fille d`un Cananéen, nommé Schua; il la prit pour femme, et alla vers elle. 
\verse Elle devint enceinte, et enfanta un fils, qu`elle appela Er. 
\verse Elle devint encore enceinte, et enfanta un fils, qu`elle appela Onan. 
\verse Elle enfanta de nouveau un fils, qu`elle appela Schéla; Juda était à Czib quand elle l`enfanta. 
\verse Juda prit pour Er, son premier-né, une femme nommée Tamar. 
\verse Er, premier-né de Juda, était méchant aux yeux de l`Éternel; et l`Éternel le fit mourir. 
\verse Alors Juda dit à Onan: Va vers la femme de ton frère, prends-la, comme beau-frère, et suscite une postérité à ton frère. 
\verse Onan, sachant que cette postérité ne serait pas à lui, se souillait à terre lorsqu`il allait vers la femme de son frère, afin de ne pas donner de postérité à son frère. 
\verse Ce qu`il faisait déplut à l`Éternel, qui le fit aussi mourir. 
\verse Alors Juda dit à Tamar, sa belle-fille: Demeure veuve dans la maison de ton père, jusqu`à ce que Schéla, mon fils, soit grand. Il parlait ainsi dans la crainte que Schéla ne mourût comme ses frères. Tamar s`en alla, et elle habita dans la maison de son père. 
\verse Les jours s`écoulèrent, et la fille de Schua, femme de Juda, mourut. Lorsque Juda fut consolé, il monta à Thimna, vers ceux qui tondaient ses brebis, lui et son ami Hira, l`Adullamite. 
\verse On en informa Tamar, et on lui dit: Voici ton beau-père qui monte à Thimna, pour tondre ses brebis. 
\verse Alors elle ôta ses habits de veuve, elle se couvrit d`un voile et s`enveloppa, et elle s`assit à l`entrée d`Énaïm, sur le chemin de Thimna; car elle voyait que Schéla était devenu grand, et qu`elle ne lui était point donnée pour femme. 
\verse Juda la vit, et la prit pour une prostituée, parce qu`elle avait couvert son visage. 
\verse Il l`aborda sur le chemin, et dit: Laisse-moi aller vers toi. Car il ne connut pas que c`était sa belle-fille. Elle dit: Que me donneras-tu pour venir vers moi? 
\verse Il répondit: Je t`enverrai un chevreau de mon troupeau. Elle dit: Me donneras-tu un gage, jusqu`à ce que tu l`envoies? 
\verse Il répondit: Quel gage te donnerai-je? Elle dit: Ton cachet, ton cordon, et le bâton que tu as à la main. Il les lui donna. Puis il alla vers elle; et elle devint enceinte de lui. 
\verse Elle se leva, et s`en alla; elle ôta son voile, et remit ses habits de veuve. 
\verse Juda envoya le chevreau par son ami l`Adullamite, pour retirer le gage des mains de la femme. Mais il ne la trouva point. 
\verse Il interrogea les gens du lieu, en disant: Où est cette prostituée qui se tenait à Énaïm, sur le chemin? Ils répondirent: Il n`y a point eu ici de prostituée. 
\verse Il retourna auprès de Juda, et dit: Je ne l`ai pas trouvée, et même les gens du lieu ont dit: Il n`y a point eu ici de prostituée. 
\verse Juda dit: Qu`elle garde ce qu`elle a! Ne nous exposons pas au mépris. Voici, j`ai envoyé ce chevreau, et tu ne l`as pas trouvée. 
\verse Environ trois mois après, on vint dire à Juda: Tamar, ta belle-fille, s`est prostituée, et même la voilà enceinte à la suite de sa prostitution. Et Juda dit: Faites-la sortir, et qu`elle soit brûlée. 
\verse Comme on l`amenait dehors, elle fit dire à son beau-père: C`est de l`homme à qui ces choses appartiennent que je suis enceinte; reconnais, je te prie, à qui sont ce cachet, ces cordons et ce bâton. 
\verse Juda les reconnut, et dit: Elle est moins coupable que moi, puisque je ne l`ai pas donnée à Schéla, mon fils. Et il ne la connut plus. 
\verse Quand elle fut au moment d`accoucher, voici, il y avait deux jumeaux dans son ventre. 
\verse Et pendant l`accouchement il y en eut un qui présenta la main; la sage-femme la prit, et y attacha un fil cramoisi, en disant: Celui-ci sort le premier. 
\verse Mais il retira la main, et son frère sortit. Alors la sage-femme dit: Quelle brèche tu as faite! Et elle lui donna le nom de Pérets. 
\verse Ensuite sortit son frère, qui avait à la main le fil cramoisi; et on lui donna le nom de Zérach. 

\chapter
\verse On fit descendre Joseph en Égypte; et Potiphar, officier de Pharaon, chef des gardes, Égyptien, l`acheta des Ismaélites qui l`y avaient fait descendre. 
\verse L`Éternel fut avec lui, et la prospérité l`accompagna; il habitait dans la maison de son maître, l`Égyptien. 
\verse Son maître vit que l`Éternel était avec lui, et que l`Éternel faisait prospérer entre ses mains tout ce qu`il entreprenait. 
\verse Joseph trouva grâce aux yeux de son maître, qui l`employa à son service, l`établit sur sa maison, et lui confia tout ce qu`il possédait. 
\verse Dès que Potiphar l`eut établi sur sa maison et sur tout ce qu`il possédait, l`Éternel bénit la maison de l`Égyptien, à cause de Joseph; et la bénédiction de l`Éternel fut sur tout ce qui lui appartenait, soit à la maison, soit aux champs. 
\verse Il abandonna aux mains de Joseph tout ce qui lui appartenait, et il n`avait avec lui d`autre soin que celui de prendre sa nourriture. Or, Joseph était beau de taille et beau de figure. 
\verse Après ces choses, il arriva que la femme de son maître porta les yeux sur Joseph, et dit: Couche avec moi! 
\verse Il refusa, et dit à la femme de son maître: Voici, mon maître ne prend avec moi connaissance de rien dans la maison, et il a remis entre mes mains tout ce qui lui appartient. 
\verse Il n`est pas plus grand que moi dans cette maison, et il ne m`a rien interdit, excepté toi, parce que tu es sa femme. Comment ferais-je un aussi grand mal et pécherais-je contre Dieu? 
\verse Quoiqu`elle parlât tous les jours à Joseph, il refusa de coucher auprès d`elle, d`être avec elle. 
\verse Un jour qu`il était entré dans la maison pour faire son ouvrage, et qu`il n`y avait là aucun des gens de la maison, 
\verse elle le saisit par son vêtement, en disant: Couche avec moi! Il lui laissa son vêtement dans la main, et s`enfuit au dehors. 
\verse Lorsqu`elle vit qu`il lui avait laissé son vêtement dans la main, et qu`il s`était enfui dehors, 
\verse elle appela les gens de sa maison, et leur dit: Voyez, il nous a amené un Hébreu pour se jouer de nous. Cet homme est venu vers moi pour coucher avec moi; mais j`ai crié à haute voix. 
\verse Et quand il a entendu que j`élevais la voix et que je criais, il a laissé son vêtement à côté de moi et s`est enfui dehors. 
\verse Et elle posa le vêtement de Joseph à côté d`elle, jusqu`à ce que son maître rentrât à la maison. 
\verse Alors elle lui parla ainsi: L`esclave hébreu que tu nous as amené est venu vers moi pour se jouer de moi. 
\verse Et comme j`ai élevé la voix et que j`ai crié, il a laissé son vêtement à côté de moi et s`est enfui dehors. 
\verse Après avoir entendu les paroles de sa femme, qui lui disait: Voilà ce que m`a fait ton esclave! le maître de Joseph fut enflammé de colère. 
\verse Il prit Joseph, et le mit dans la prison, dans le lieu où les prisonniers du roi étaient enfermés: il fut là, en prison. 
\verse L`Éternel fut avec Joseph, et il étendit sur lui sa bonté. Il le mit en faveur aux yeux du chef de la prison. 
\verse Et le chef de la prison plaça sous sa surveillance tous les prisonniers qui étaient dans la prison; et rien ne s`y faisait que par lui. 
\verse Le chef de la prison ne prenait aucune connaissance de ce que Joseph avait en main, parce que l`Éternel était avec lui. Et l`Éternel donnait de la réussite à ce qu`il faisait. 

\chapter
\verse Après ces choses, il arriva que l`échanson et le panetier du roi d`Égypte, offensèrent leur maître, le roi d`Égypte. 
\verse Pharaon fut irrité contre ses deux officiers, le chef des échansons et le chef des panetiers. 
\verse Et il les fit mettre dans la maison du chef des gardes, dans la prison, dans le lieu où Joseph était enfermé. 
\verse Le chef des gardes les plaça sous la surveillance de Joseph, qui faisait le service auprès d`eux; et ils passèrent un certain temps en prison. 
\verse Pendant une même nuit, l`échanson et le panetier du roi d`Égypte, qui étaient enfermés dans la prison, eurent tous les deux un songe, chacun le sien, pouvant recevoir une explication distincte. 
\verse Joseph, étant venu le matin vers eux, les regarda; et voici, ils étaient tristes. 
\verse Alors il questionna les officiers de Pharaon, qui étaient avec lui dans la prison de son maître, et il leur dit: Pourquoi avez-vous mauvais visage aujourd`hui? 
\verse Ils lui répondirent: Nous avons eu un songe, et il n`y a personne pour l`expliquer. Joseph leur dit: N`est-ce pas à Dieu qu`appartiennent les explications? Racontez-moi donc votre songe. 
\verse Le chef des échansons raconta son songe à Joseph, et lui dit: Dans mon songe, voici, il y avait un cep devant moi. 
\verse Ce cep avait trois sarments. Quand il eut poussé, sa fleur se développa et ses grappes donnèrent des raisins mûrs. 
\verse La coupe de Pharaon était dans ma main. Je pris les raisins, je les pressai dans la coupe de Pharaon, et je mis la coupe dans la main de Pharaon. 
\verse Joseph lui dit: En voici l`explication. Les trois sarments sont trois jours. 
\verse Encore trois jours, et Pharaon relèvera ta tête et te rétablira dans ta charge; tu mettras la coupe dans la main de Pharaon, comme tu en avais l`habitude lorsque tu étais son échanson. 
\verse Mais souviens-toi de moi, quand tu seras heureux, et montre, je te prie, de la bonté à mon égard; parle en ma faveur à Pharaon, et fais-moi sortir de cette maison. 
\verse Car j`ai été enlevé du pays des Hébreux, et ici même je n`ai rien fait pour être mis en prison. 
\verse Le chef des panetiers, voyant que Joseph avait donné une explication favorable, dit: Voici, il y avait aussi, dans mon songe, trois corbeilles de pain blanc sur ma tête. 
\verse Dans la corbeille la plus élevée il y avait pour Pharaon des mets de toute espèce, cuits au four; et les oiseaux les mangeaient dans la corbeille au-dessus de ma tête. 
\verse Joseph répondit, et dit: En voici l`explication. Les trois corbeilles sont trois jours. 
\verse Encore trois jours, et Pharaon enlèvera ta tête de dessus toi, te fera pendre à un bois, et les oiseaux mangeront ta chair. 
\verse Le troisième jour, jour de la naissance de Pharaon, il fit un festin à tous ses serviteurs; et il éleva la tête du chef des échansons et la tête du chef des panetiers, au milieu de ses serviteurs: 
\verse il rétablit le chef des échansons dans sa charge d`échanson, pour qu`il mît la coupe dans la main de Pharaon; 
\verse mais il fit pendre le chef des panetiers, selon l`explication que Joseph leur avait donnée. 
\verse Le chef des échansons ne pensa plus à Joseph. Il l`oublia. 

\chapter
\verse Au bout de deux ans, Pharaon eut un songe. Voici, il se tenait près du fleuve. 
\verse Et voici, sept vaches belles à voir et grasses de chair montèrent hors du fleuve, et se mirent à paître dans la prairie. 
\verse Sept autres vaches laides à voir et maigres de chair montèrent derrière elles hors du fleuve, et se tinrent à leurs côtés sur le bord du fleuve. 
\verse Les vaches laides à voir et maigres de chair mangèrent les sept vaches belles à voir et grasses de chair. Et Pharaon s`éveilla. 
\verse Il se rendormit, et il eut un second songe. Voici, sept épis gras et beaux montèrent sur une même tige. 
\verse Et sept épis maigres et brûlés par le vent d`orient poussèrent après eux. 
\verse Les épis maigres engloutirent les sept épis gras et pleins. Et Pharaon s`éveilla. Voilà le songe. 
\verse Le matin, Pharaon eut l`esprit agité, et il fit appeler tous les magiciens et tous les sages de l`Égypte. Il leur raconta ses songes. Mais personne ne put les expliquer à Pharaon. 
\verse Alors le chef des échansons prit la parole, et dit à Pharaon: Je vais rappeler aujourd`hui le souvenir de ma faute. 
\verse Pharaon s`était irrité contre ses serviteurs; et il m`avait fait mettre en prison dans la maison du chef des gardes, moi et le chef des panetiers. 
\verse Nous eûmes l`un et l`autre un songe dans une même nuit; et chacun de nous reçut une explication en rapport avec le songe qu`il avait eu. 
\verse Il y avait là avec nous un jeune Hébreu, esclave du chef des gardes. Nous lui racontâmes nos songes, et il nous les expliqua. 
\verse Les choses sont arrivées selon l`explication qu`il nous avait donnée. Pharaon me rétablit dans ma charge, et il fit pendre le chef des panetiers. 
\verse Pharaon fit appeler Joseph. On le fit sortir en hâte de prison. Il se rasa, changea de vêtements, et se rendit vers Pharaon. 
\verse Pharaon dit à Joseph: J`ai eu un songe. Personne ne peut l`expliquer; et j`ai appris que tu expliques un songe, après l`avoir entendu. 
\verse Joseph répondit à Pharaon, en disant: Ce n`est pas moi! c`est Dieu qui donnera une réponse favorable à Pharaon. 
\verse Pharaon dit alors à Joseph: Dans mon songe, voici, je me tenais sur le bord du fleuve. 
\verse Et voici, sept vaches grasses de chair et belles d`apparence montèrent hors du fleuve, et se mirent à paître dans la prairie. 
\verse Sept autres vaches montèrent derrière elles, maigres, fort laides d`apparence, et décharnées: je n`en ai point vu d`aussi laides dans tout le pays d`Égypte. 
\verse Les vaches décharnées et laides mangèrent les sept premières vaches qui étaient grasses. 
\verse Elles les engloutirent dans leur ventre, sans qu`on s`aperçût qu`elles y fussent entrées; et leur apparence était laide comme auparavant. Et je m`éveillai. 
\verse Je vis encore en songe sept épis pleins et beaux, qui montèrent sur une même tige. 
\verse Et sept épis vides, maigres, brûlés par le vent d`orient, poussèrent après eux. 
\verse Les épis maigres engloutirent les sept beaux épis. Je l`ai dit aux magiciens, mais personne ne m`a donné l`explication. 
\verse Joseph dit à Pharaon: Ce qu`a songé Pharaon est une seule chose; Dieu a fait connaître à Pharaon ce qu`il va faire. 
\verse Les sept vaches belles sont sept années: et les sept épis beaux sont sept années: c`est un seul songe. 
\verse Les sept vaches décharnées et laides, qui montaient derrière les premières, sont sept années; et les sept épis vides, brûlés par le vent d`orient, seront sept années de famine. 
\verse Ainsi, comme je viens de le dire à Pharaon, Dieu a fait connaître à Pharaon ce qu`il va faire. 
\verse Voici, il y aura sept années de grande abondance dans tout le pays d`Égypte. 
\verse Sept années de famine viendront après elles; et l`on oubliera toute cette abondance au pays d`Égypte, et la famine consumera le pays. 
\verse Cette famine qui suivra sera si forte qu`on ne s`apercevra plus de l`abondance dans le pays. 
\verse Si Pharaon a vu le songe se répéter une seconde fois, c`est que la chose est arrêtée de la part de Dieu, et que Dieu se hâtera de l`exécuter. 
\verse Maintenant, que Pharaon choisisse un homme intelligent et sage, et qu`il le mette à la tête du pays d`Égypte. 
\verse Que Pharaon établisse des commissaires sur le pays, pour lever un cinquième des récoltes de l`Égypte pendant les sept années d`abondance. 
\verse Qu`ils rassemblent tous les produits de ces bonnes années qui vont venir; qu`ils fassent, sous l`autorité de Pharaon, des amas de blé, des approvisionnements dans les villes, et qu`ils en aient la garde. 
\verse Ces provisions seront en réserve pour le pays, pour les sept années de famine qui arriveront dans le pays d`Égypte, afin que le pays ne soit pas consumé par la famine. 
\verse Ces paroles plurent à Pharaon et à tous ses serviteurs. 
\verse Et Pharaon dit à ses serviteurs: Trouverions-nous un homme comme celui-ci, ayant en lui l`esprit de Dieu? 
\verse Et Pharaon dit à Joseph: Puisque Dieu t`a fait connaître toutes ces choses, il n`y a personne qui soit aussi intelligent et aussi sage que toi. 
\verse Je t`établis sur ma maison, et tout mon peuple obéira à tes ordres. Le trône seul m`élèvera au-dessus de toi. 
\verse Pharaon dit à Joseph: Vois, je te donne le commandement de tout le pays d`Égypte. 
\verse Pharaon ôta son anneau de la main, et le mit à la main de Joseph; il le revêtit d`habits de fin lin, et lui mit un collier d`or au cou. 
\verse Il le fit monter sur le char qui suivait le sien; et l`on criait devant lui: A genoux! C`est ainsi que Pharaon lui donna le commandement de tout le pays d`Égypte. 
\verse Il dit encore à Joseph: Je suis Pharaon! Et sans toi personne ne lèvera la main ni le pied dans tout le pays d`Égypte. 
\verse Pharaon appela Joseph du nom de Tsaphnath Paenéach; et il lui donna pour femme Asnath, fille de Poti Phéra, prêtre d`On. Et Joseph partit pour visiter le pays d`Égypte. 
\verse Joseph était âgé de trente ans lorsqu`il se présenta devant Pharaon, roi d`Égypte; et il quitta Pharaon, et parcourut tout le pays d`Égypte. 
\verse Pendant les sept années de fertilité, la terre rapporta abondamment. 
\verse Joseph rassembla tous les produits de ces sept années dans le pays d`Égypte; il fit des approvisionnements dans les villes, mettant dans l`intérieur de chaque ville les productions des champs d`alentour. 
\verse Joseph amassa du blé, comme le sable de la mer, en quantité si considérable que l`on cessa de compter, parce qu`il n`y avait plus de nombre. 
\verse Avant les années de famine, il naquit à Joseph deux fils, que lui enfanta Asnath, fille de Poti Phéra, prêtre d`On. 
\verse Joseph donna au premier-né le nom de Manassé, car, dit-il, Dieu m`a fait oublier toutes mes peines et toute la maison de mon père. 
\verse Et il donna au second le nom d`Éphraïm, car, dit-il, Dieu m`a rendu fécond dans le pays de mon affliction. 
\verse Les sept années d`abondance qu`il y eut au pays d`Égypte s`écoulèrent. 
\verse Et les sept années de famine commencèrent à venir, ainsi que Joseph l`avait annoncé. Il y eut famine dans tous les pays; mais dans tout le pays d`Égypte il y avait du pain. 
\verse Quand tout le pays d`Égypte fut aussi affamé, le peuple cria à Pharaon pour avoir du pain. Pharaon dit à tous les Égyptiens: Allez vers Joseph, et faites ce qu`il vous dira. 
\verse La famine régnait dans tout le pays. Joseph ouvrit tous les lieux d`approvisionnement, et vendit du blé aux Égyptiens. La famine augmentait dans le pays d`Égypte. 
\verse Et de tous les pays on arrivait en Égypte, pour acheter du blé auprès de Joseph; car la famine était forte dans tous les pays. 

\chapter
\verse Jacob, voyant qu`il y avait du blé en Égypte, dit à ses fils: Pourquoi vous regardez-vous les uns les autres? 
\verse Il dit: Voici, j`apprends qu`il y a du blé en Égypte; descendez-y, pour nous en acheter là, afin que nous vivions et que nous ne mourions pas. 
\verse Dix frères de Joseph descendirent en Égypte, pour acheter du blé. 
\verse Jacob n`envoya point avec eux Benjamin, frère de Joseph, dans la crainte qu`il ne lui arrivât quelque malheur. 
\verse Les fils d`Israël vinrent pour acheter du blé, au milieu de ceux qui venaient aussi; car la famine était dans le pays de Canaan. 
\verse Joseph commandait dans le pays; c`est lui qui vendait du blé à tout le peuple du pays. Les frères de Joseph vinrent, et se prosternèrent devant lui la face contre terre. 
\verse Joseph vit ses frères et les reconnut; mais il feignit d`être un étranger pour eux, il leur parla durement, et leur dit: D`où venez-vous? Ils répondirent: Du pays de Canaan, pour acheter des vivres. 
\verse Joseph reconnut ses frères, mais eux ne le reconnurent pas. 
\verse Joseph se souvint des songes qu`il avait eus à leur sujet, et il leur dit: Vous êtes des espions; c`est pour observer les lieux faibles du pays que vous êtes venus. 
\verse Ils lui répondirent: Non, mon seigneur, tes serviteurs sont venus pour acheter du blé. 
\verse Nous sommes tous fils d`un même homme; nous sommes sincères, tes serviteurs ne sont pas des espions. 
\verse Il leur dit: Nullement; c`est pour observer les lieux faibles du pays que vous êtes venus. 
\verse Ils répondirent: Nous, tes serviteurs, sommes douze frères, fils d`un même homme au pays de Canaan; et voici, le plus jeune est aujourd`hui avec notre père, et il y en a un qui n`est plus. 
\verse Joseph leur dit: Je viens de vous le dire, vous êtes des espions. 
\verse Voici comment vous serez éprouvés. Par la vie de Pharaon! vous ne sortirez point d`ici que votre jeune frère ne soit venu. 
\verse Envoyez l`un de vous pour chercher votre frère; et vous, restez prisonniers. Vos paroles seront éprouvées, et je saurai si la vérité est chez vous; sinon, par la vie de Pharaon! vous êtes des espions. 
\verse Et il les mit ensemble trois jours en prison. 
\verse Le troisième jour, Joseph leur dit: Faites ceci, et vous vivrez. Je crains Dieu! 
\verse Si vous êtes sincères, que l`un de vos frères reste enfermé dans votre prison; et vous, partez, emportez du blé pour nourrir vos familles, 
\verse et amenez-moi votre jeune frère, afin que vos paroles soient éprouvées et que vous ne mouriez point. Et ils firent ainsi. 
\verse Ils se dirent alors l`un à l`autre: Oui, nous avons été coupables envers notre frère, car nous avons vu l`angoisse de son âme, quand il nous demandait grâce, et nous ne l`avons point écouté! C`est pour cela que cette affliction nous arrive. 
\verse Ruben, prenant la parole, leur dit: Ne vous disais-je pas: Ne commettez point un crime envers cet enfant? Mais vous n`avez point écouté. Et voici, son sang est redemandé. 
\verse Ils ne savaient pas que Joseph comprenait, car il se servait avec eux d`un interprète. 
\verse Il s`éloigna d`eux, pour pleurer. Il revint, et leur parla; puis il prit parmi eux Siméon, et le fit enchaîner sous leurs yeux. 
\verse Joseph ordonna qu`on remplît de blé leurs sacs, qu`on remît l`argent de chacun dans son sac, et qu`on leur donnât des provisions pour la route. Et l`on fit ainsi. 
\verse Ils chargèrent le blé sur leurs ânes, et partirent. 
\verse L`un d`eux ouvrit son sac pour donner du fourrage à son âne, dans le lieu où ils passèrent la nuit, et il vit l`argent qui était à l`entrée du sac. 
\verse Il dit à ses frères: Mon argent a été rendu, et le voici dans mon sac. Alors leur coeur fut en défaillance; et ils se dirent l`un à l`autre, en tremblant: Qu`est-ce que Dieu nous a fait? 
\verse Ils revinrent auprès de Jacob, leur père, dans le pays de Canaan, et ils lui racontèrent tout ce qui leur était arrivé. Ils dirent: 
\verse L`homme, qui est le seigneur du pays, nous a parlé durement, et il nous a pris pour des espions. 
\verse Nous lui avons dit: Nous sommes sincères, nous ne sommes pas des espions. 
\verse Nous sommes douze frères, fils de notre père; l`un n`est plus, et le plus jeune est aujourd`hui avec notre père au pays de Canaan. 
\verse Et l`homme, qui est le seigneur du pays, nous a dit: Voici comment je saurai si vous êtes sincères. Laissez auprès de moi l`un de vos frères, prenez de quoi nourrir vos familles, partez, et amenez-moi votre jeune frère. 
\verse Je saurai ainsi que vous n`êtes pas des espions, que vous êtes sincères; je vous rendrai votre frère, et vous pourrez librement parcourir le pays. 
\verse Lorsqu`ils vidèrent leurs sacs, voici, le paquet d`argent de chacun était dans son sac. Ils virent, eux et leur père, leurs paquets d`argent, et ils eurent peur. 
\verse Jacob, leur père, leur dit: Vous me privez de mes enfants! Joseph n`est plus, Siméon n`est plus, et vous prendriez Benjamin! C`est sur moi que tout cela retombe. 
\verse Ruben dit à son père: Tu feras mourir mes deux fils si je ne te ramène pas Benjamin; remets-le entre mes mains, et je te le ramènerai. 
\verse Jacob dit: Mon fils ne descendra point avec vous; car son frère est mort, et il reste seul; s`il lui arrivait un malheur dans le voyage que vous allez faire, vous feriez descendre mes cheveux blancs avec douleur dans le séjour des morts. 

\chapter
\verse La famine s`appesantissait sur le pays. 
\verse Quand ils eurent fini de manger le blé qu`ils avaient apporté d`Égypte, Jacob dit à ses fils: Retournez, achetez-nous un peu de vivres. 
\verse Juda lui répondit: Cet homme nous a fait cette déclaration formelle: Vous ne verrez pas ma face, à moins que votre frère ne soit avec vous. 
\verse Si donc tu veux envoyer notre frère avec nous, nous descendrons, et nous t`achèterons des vivres. 
\verse Mais si tu ne veux pas l`envoyer, nous ne descendrons point, car cet homme nous a dit: Vous ne verrez pas ma face, à moins que votre frère ne soit avec vous. 
\verse Israël dit alors: Pourquoi avez-vous mal agi à mon égard, en disant à cet homme que vous aviez encore un frère? 
\verse Ils répondirent: Cet homme nous a interrogés sur nous et sur notre famille, en disant: Votre père vit-il encore? avez-vous un frère? Et nous avons répondu à ces questions. Pouvions-nous savoir qu`il dirait: Faites descendre votre frère? 
\verse Juda dit à Israël, son père: Laisse venir l`enfant avec moi, afin que nous nous levions et que nous partions; et nous vivrons et ne mourrons pas, nous, toi, et nos enfants. 
\verse Je réponds de lui; tu le redemanderas de ma main. Si je ne le ramène pas auprès de toi et si je ne le remets pas devant ta face, je serai pour toujours coupable envers toi. 
\verse Car si nous n`eussions pas tardé, nous serions maintenant deux fois de retour. 
\verse Israël, leur père, leur dit: Puisqu`il le faut, faites ceci. Prenez dans vos sacs des meilleures productions du pays, pour en porter un présent à cet homme, un peu de baume et un peu de miel, des aromates, de la myrrhe, des pistaches et des amandes. 
\verse Prenez avec vous de l`argent au double, et remportez l`argent qu`on avait mis à l`entrée de vos sacs: peut-être était-ce une erreur. 
\verse Prenez votre frère, et levez-vous; retournez vers cet homme. 
\verse Que le Dieu tout puissant vous fasse trouver grâce devant cet homme, et qu`il laisse revenir avec vous votre autre frère et Benjamin! Et moi, si je dois être privé de mes enfants, que j`en sois privé! 
\verse Ils prirent le présent; ils prirent avec eux de l`argent au double, ainsi que Benjamin; ils se levèrent, descendirent en Égypte, et se présentèrent devant Joseph. 
\verse Dès que Joseph vit avec eux Benjamin, il dit à son intendant: Fais entrer ces gens dans la maison, tue et apprête; car ces gens mangeront avec moi à midi. 
\verse Cet homme fit ce que Joseph avait ordonné, et il conduisit ces gens dans la maison de Joseph. 
\verse Ils eurent peur lorsqu`ils furent conduits à la maison de Joseph, et ils dirent: C`est à cause de l`argent remis l`autre fois dans nos sacs qu`on nous emmène; c`est pour se jeter sur nous, se précipiter sur nous; c`est pour nous prendre comme esclaves, et s`emparer de nos ânes. 
\verse Ils s`approchèrent de l`intendant de la maison de Joseph, et lui adressèrent la parole, à l`entrée de la maison. 
\verse Ils dirent: Pardon! mon seigneur, nous sommes déjà descendus une fois pour acheter des vivres. 
\verse Puis, quand nous arrivâmes, au lieu où nous devions passer la nuit, nous avons ouvert nos sacs; et voici, l`argent de chacun était à l`entrée de son sac, notre argent selon son poids: nous le rapportons avec nous. 
\verse Nous avons aussi apporté d`autre argent, pour acheter des vivres. Nous ne savons pas qui avait mis notre argent dans nos sacs. 
\verse L`intendant répondit: Que la paix soit avec vous! Ne craignez rien. C`est votre Dieu, le Dieu de votre père, qui vous a donné un trésor dans vos sacs. Votre argent m`est parvenu. Et il leur amena Siméon. 
\verse Cet homme les fit entrer dans la maison de Joseph; il leur donna de l`eau et ils se lavèrent les pieds; il donna aussi du fourrage à leurs ânes. 
\verse Ils préparèrent leur présent, en attendant que Joseph vienne à midi; car on les avait informés qu`ils mangeraient chez lui. 
\verse Quand Joseph fut arrivé à la maison, ils lui offrirent le présent qu`ils avaient apporté, et ils se prosternèrent en terre devant lui. 
\verse Il leur demanda comment ils se portaient; et il dit: Votre vieux père, dont vous avez parlé, est-il en bonne santé? vit-il encore? 
\verse Ils répondirent: Ton serviteur, notre père, est en bonne santé; il vit encore. Et ils s`inclinèrent et se prosternèrent. 
\verse Joseph leva les yeux; et, jetant un regard sur Benjamin, son frère, fils de sa mère, il dit: Est-ce là votre jeune frère, dont vous m`avez parlé? Et il ajouta: Dieu te fasse miséricorde, mon fils! 
\verse Ses entrailles étaient émues pour son frère, et il avait besoin de pleurer; il entra précipitamment dans une chambre, et il y pleura. 
\verse Après s`être lavé le visage, il en sortit; et, faisant des efforts pour se contenir, il dit: Servez à manger. 
\verse On servit Joseph à part, et ses frères à part; les Égyptiens qui mangeaient avec lui furent aussi servis à part, car les Égyptiens ne pouvaient pas manger avec les Hébreux, parce que c`est à leurs yeux une abomination. 
\verse Les frères de Joseph s`assirent en sa présence, le premier-né selon son droit d`aînesse, et le plus jeune selon son âge; et ils se regardaient les uns les autres avec étonnement. 
\verse Joseph leur fit porter des mets qui étaient devant lui, et Benjamin en eut cinq fois plus que les autres. Ils burent, et s`égayèrent avec lui. 

\chapter
\verse Joseph donna cet ordre à l`intendant de sa maison: Remplis de vivres les sacs de ces gens, autant qu`ils en pourront porter, et mets l`argent de chacun à l`entrée de son sac. 
\verse Tu mettras aussi ma coupe, la coupe d`argent, à l`entrée du sac du plus jeune, avec l`argent de son blé. L`intendant fit ce que Joseph lui avait ordonné. 
\verse Le matin, dès qu`il fit jour, on renvoya ces gens avec leur ânes. 
\verse Ils étaient sortis de la ville, et ils n`en étaient guère éloignés, lorsque Joseph dit à son intendant: Lève-toi, poursuis ces gens; et, quand tu les auras atteints, tu leur diras: Pourquoi avez-vous rendu le mal pour le bien? 
\verse N`avez-vous pas la coupe dans laquelle boit mon seigneur, et dont il se sert pour deviner? Vous avez mal fait d`agir ainsi. 
\verse L`intendant les atteignit, et leur dit ces mêmes paroles. 
\verse Ils lui répondirent: Pourquoi mon seigneur parle-t-il de la sorte? Dieu préserve tes serviteurs d`avoir commis une telle action! 
\verse Voici, nous t`avons rapporté du pays de Canaan l`argent que nous avons trouvé à l`entrée de nos sacs; comment aurions-nous dérobé de l`argent ou de l`or dans la maison de ton seigneur? 
\verse Que celui de tes serviteurs sur qui se trouvera la coupe meure, et que nous soyons nous-mêmes esclaves de mon seigneur! 
\verse Il dit: Qu`il en soit donc selon vos paroles! Celui sur qui se trouvera la coupe sera mon esclave; et vous, vous serez innocents. 
\verse Aussitôt, chacun descendit son sac à terre, et chacun ouvrit son sac. 
\verse L`intendant les fouilla, commençant par le plus âgé et finissant par le plus jeune; et la coupe fut trouvée dans le sac de Benjamin. 
\verse Ils déchirèrent leurs vêtements, chacun rechargea son âne, et ils retournèrent à la ville. 
\verse Juda et ses frères arrivèrent à la maison de Joseph, où il était encore, et ils se prosternèrent en terre devant lui. 
\verse Joseph leur dit: Quelle action avez-vous faite? Ne savez-vous pas qu`un homme comme moi a le pouvoir de deviner? 
\verse Juda répondit: Que dirons-nous à mon seigneur? comment parlerons-nous? comment nous justifierons-nous? Dieu a trouvé l`iniquité de tes serviteurs. Nous voici esclaves de mon seigneur, nous, et celui sur qui s`est trouvée la coupe. 
\verse Et Joseph dit: Dieu me garde de faire cela! L`homme sur qui la coupe a été trouvée sera mon esclave; mais vous, remontez en paix vers votre père. 
\verse Alors Juda s`approcha de Joseph, et dit: De grâce, mon seigneur, que ton serviteur puisse faire entendre une parole à mon seigneur, et que sa colère ne s`enflamme point contre ton serviteur! car tu es comme Pharaon. 
\verse Mon seigneur a interrogé ses serviteurs, en disant: Avez-vous un père, ou un frère? 
\verse Nous avons répondu: Nous avons un vieux père, et un jeune frère, enfant de sa vieillesse; cet enfant avait un frère qui est mort, et qui était de la même mère; il reste seul, et son père l`aime. 
\verse Tu as dit à tes serviteurs: Faites-le descendre vers moi, et que je le voie de mes propres yeux. 
\verse Nous avons répondu à mon seigneur: L`enfant ne peut pas quitter son père; s`il le quitte, son père mourra. 
\verse Tu as dit à tes serviteurs: Si votre jeune frère ne descend pas avec vous, vous ne reverrez pas ma face. 
\verse Lorsque nous sommes remontés auprès de ton serviteur, mon père, nous lui avons rapporté les paroles de mon seigneur. 
\verse Notre père a dit: Retournez, achetez-nous un peu de vivres. 
\verse Nous avons répondu: Nous ne pouvons pas descendre; mais, si notre jeune frère est avec nous, nous descendrons, car nous ne pouvons pas voir la face de cet homme, à moins que notre jeune frère ne soit avec nous. 
\verse Ton serviteur, notre père, nous a dit: Vous savez que ma femme m`a enfanté deux fils. 
\verse L`un étant sorti de chez moi, je pense qu`il a été sans doute déchiré, car je ne l`ai pas revu jusqu`à présent. 
\verse Si vous me prenez encore celui-ci, et qu`il lui arrive un malheur, vous ferez descendre mes cheveux blancs avec douleur dans le séjour des morts. 
\verse Maintenant, si je retourne auprès de ton serviteur, mon père, sans avoir avec nous l`enfant à l`âme duquel son âme est attachée, 
\verse il mourra, en voyant que l`enfant n`y est pas; et tes serviteurs feront descendre avec douleur dans le séjour des morts les cheveux blancs de ton serviteur, notre père. 
\verse Car ton serviteur a répondu pour l`enfant, en disant à mon père: Si je ne le ramène pas auprès de toi, je serai pour toujours coupable envers mon père. 
\verse Permets donc, je te prie, à ton serviteur de rester à la place de l`enfant, comme esclave de mon seigneur; et que l`enfant remonte avec ses frères. 
\verse Comment pourrai-je remonter vers mon père, si l`enfant n`est pas avec moi? Ah! que je ne voie point l`affliction de mon père! 

\chapter
\verse Joseph ne pouvait plus se contenir devant tous ceux qui l`entouraient. Il s`écria: Faites sortir tout le monde. Et il ne resta personne avec Joseph, quand il se fit connaître à ses frères. 
\verse Il éleva la voix, en pleurant. Les Égyptiens l`entendirent, et la maison de Pharaon l`entendit. 
\verse Joseph dit à ses frères: Je suis Joseph! Mon père vit-il encore? Mais ses frères ne purent lui répondre, car ils étaient troublés en sa présence. 
\verse Joseph dit à ses frères: Approchez-vous de moi. Et ils s`approchèrent. Il dit: Je suis Joseph, votre frère, que vous avez vendu pour être mené en Égypte. 
\verse Maintenant, ne vous affligez pas, et ne soyez pas fâchés de m`avoir vendu pour être conduit ici, car c`est pour vous sauver la vie que Dieu m`a envoyé devant vous. 
\verse Voilà deux ans que la famine est dans le pays; et pendant cinq années encore, il n`y aura ni labour, ni moisson. 
\verse Dieu m`a envoyé devant vous pour vous faire subsister dans le pays, et pour vous faire vivre par une grande délivrance. 
\verse Ce n`est donc pas vous qui m`avez envoyé ici, mais c`est Dieu; il m`a établi père de Pharaon, maître de toute sa maison, et gouverneur de tout le pays d`Égypte. 
\verse Hâtez-vous de remonter auprès de mon père, et vous lui direz: Ainsi a parlé ton fils Joseph: Dieu m`a établi seigneur de toute l`Égypte; descends vers moi, ne tarde pas! 
\verse Tu habiteras dans le pays de Gosen, et tu seras près de moi, toi, tes fils, et les fils de tes fils, tes brebis et tes boeufs, et tout ce qui est à toi. 
\verse Là, je te nourrirai, car il y aura encore cinq années de famine; et ainsi tu ne périras point, toi, ta maison, et tout ce qui est à toi. 
\verse Vous voyez de vos yeux, et mon frère Benjamin voit de ses yeux que c`est moi-même qui vous parle. 
\verse Racontez à mon père toute ma gloire en Égypte, et tout ce que vous avez vu; et vous ferez descendre ici mon père au plus tôt. 
\verse Il se jeta au cou de Benjamin, son frère, et pleura; et Benjamin pleura sur son cou. 
\verse Il embrassa aussi tous ses frères, en pleurant. Après quoi, ses frères s`entretinrent avec lui. 
\verse Le bruit se répandit dans la maison de Pharaon que les frères de Joseph étaient arrivés: ce qui fut agréable à Pharaon et à ses serviteurs. 
\verse Pharaon dit à Joseph: Dis à tes frères: Faites ceci. Chargez vos bêtes, et partez pour le pays de Canaan; 
\verse prenez votre père et vos familles, et venez auprès de moi. Je vous donnerai ce qu`il y a de meilleur au pays d`Égypte, et vous mangerez la graisse du pays. 
\verse Tu as ordre de leur dire: Faites ceci. Prenez dans le pays d`Égypte des chars pour vos enfants et pour vos femmes; amenez votre père, et venez. 
\verse Ne regrettez point ce que vous laisserez, car ce qu`il a de meilleur dans tout le pays d`Égypte sera pour vous. 
\verse Les fils d`Israël firent ainsi. Joseph leur donna des chars, selon l`ordre de Pharaon; il leur donna aussi des provisions pour la route. 
\verse Il leur donna à tous des vêtements de rechange, et il donna à Benjamin trois cents sicles d`argent et cinq vêtements de rechange. 
\verse Il envoya à son père dix ânes chargés de ce qu`il y avait de meilleur en Égypte, et dix ânesses chargées de blé, de pain et de vivres, pour son père pendant le voyage. 
\verse Puis il congédia ses frères, qui partirent; et il leur dit: Ne vous querellez pas en chemin. 
\verse Ils remontèrent de l`Égypte, et ils arrivèrent dans le pays de Canaan, auprès de Jacob, leur père. 
\verse Ils lui dirent: Joseph vit encore, et même c`est lui qui gouverne tout le pays d`Égypte. Mais le coeur de Jacob resta froid, parce qu`il ne les croyait pas. 
\verse Ils lui rapportèrent toutes les paroles que Joseph leur avait dites. Il vit les chars que Joseph avait envoyés pour le transporter. C`est alors que l`esprit de Jacob, leur père, se ranima; 
\verse et Israël dit: C`est assez! Joseph, mon fils, vit encore! J`irai, et je le verrai avant que je meure. 

\chapter
\verse Israël partit, avec tout ce qui lui appartenait. Il arriva à Beer Schéba, et il offrit des sacrifices au Dieu de son père Isaac. 
\verse Dieu parla à Israël dans une vision pendant la nuit, et il dit: Jacob! Jacob! Israël répondit: Me voici! 
\verse Et Dieu dit: Je suis le Dieu, le Dieu de ton père. Ne crains point de descendre en Égypte, car là je te ferai devenir une grande nation. 
\verse Moi-même je descendrai avec toi en Égypte, et moi-même je t`en ferai remonter; et Joseph te fermera les yeux. 
\verse Jacob quitta Beer Schéba; et les fils d`Israël mirent Jacob, leur père, avec leurs enfants et leurs femmes, sur les chars que Pharaon avait envoyés pour les transporter. 
\verse Ils prirent aussi leurs troupeaux et les biens qu`ils avaient acquis dans le pays de Canaan. Et Jacob se rendit en Égypte, avec toute sa famille. 
\verse Il emmena avec lui en Égypte ses fils et les fils de ses fils, ses filles et les filles de ses fils, et toute sa famille. 
\verse Voici les noms des fils d`Israël, qui vinrent en Égypte. Jacob et ses fils. Premier-né de Jacob: Ruben. 
\verse Fils de Ruben: Hénoc, Pallu, Hetsron et Carmi. 
\verse Fils de Siméon: Jemuel, Jamin, Ohad, Jakin et Tsochar; et Saul, fils de la Cananéenne. 
\verse Fils de Lévi: Guerschon, Kehath et Merari. 
\verse Fils de Juda: Er, Onan, Schéla, Pérets et Zarach; mais Er et Onan moururent au pays de Canaan. Les fils de Pérets furent Hetsron et Hamul. 
\verse Fils d`Issacar: Thola, Puva, Job et Schimron. 
\verse Fils de Zabulon: Séred, Élon et Jahleel. 
\verse Ce sont là les fils que Léa enfanta à Jacob à Paddan Aram, avec sa fille Dina. Ses fils et ses filles formaient en tout trente-trois personnes. 
\verse Fils de Gad: Tsiphjon, Haggi, Schuni, Etsbon, Éri, Arodi et Areéli. 
\verse Fils d`Aser: Jimna, Jischva, Jischvi et Beria; et Sérach, leur soeur. Et les fils de Beria: Héber et Malkiel. 
\verse Ce sont là les fils de Zilpa, que Laban avait donnée à Léa, sa fille; et elle les enfanta à Jacob. En tout, seize personnes. 
\verse Fils de Rachel, femme de Jacob: Joseph et Benjamin. 
\verse Il naquit à Joseph, au pays d`Égypte, Manassé et Éphraïm, que lui enfanta Asnath, fille de Poti Phéra, prêtre d`On. 
\verse Fils de Benjamin: Béla, Béker, Aschbel, Guéra, Naaman, Éhi, Rosch, Muppim, Huppim et Ard. 
\verse Ce sont là les fils de Rachel, qui naquirent à Jacob. En tout, quatorze personnes. 
\verse Fils de Dan: Huschim. 
\verse Fils de Nephthali: Jathtseel, Guni, Jetser et Schillem. 
\verse Ce sont là les fils de Bilha, que Laban avait donnée à Rachel, sa fille; et elle les enfanta à Jacob. En tout, sept personnes. 
\verse Les personnes qui vinrent avec Jacob en Égypte, et qui étaient issues de lui, étaient au nombre de soixante-six en tout, sans compter les femmes des fils de Jacob. 
\verse Et Joseph avait deux fils qui lui étaient nés en Égypte. Le total des personnes de la famille de Jacob qui vinrent en Égypte était de soixante-dix. 
\verse Jacob envoya Juda devant lui vers Joseph, pour l`informer qu`il se rendait en Gosen. 
\verse Joseph attela son char et y monta, pour aller en Gosen, à la rencontre d`Israël, son père. Dès qu`il le vit, il se jeta à son cou, et pleura longtemps sur son cou. 
\verse Israël dit à Joseph: Que je meure maintenant, puisque j`ai vu ton visage et que tu vis encore! 
\verse Joseph dit à ses frères et à la famille de son père: Je vais avertir Pharaon, et je lui dirai: Mes frères et la famille de mon père, qui étaient au pays de Canaan, sont arrivés auprès de moi. 
\verse Ces hommes sont bergers, car ils élèvent des troupeaux; ils ont amené leurs brebis et leurs boeufs, et tout ce qui leur appartient. 
\verse Et quand Pharaon vous appellera, et dira: 
\verse Quelle est votre occupation? vous répondrez: Tes serviteurs ont élevé des troupeaux, depuis notre jeunesse jusqu`à présent, nous et nos pères. De cette manière, vous habiterez dans le pays de Gosen, car tous les bergers sont en abomination aux Égyptiens. 

\chapter
\verse Joseph alla avertir Pharaon, et lui dit: Mes frères et mon père sont arrivés du pays de Canaan, avec leurs brebis et leurs boeufs, et tout ce qui leur appartient; et les voici dans le pays de Gosen. 
\verse Il prit cinq de ses frères, et les présenta à Pharaon. 
\verse Pharaon leur dit: Quelle est votre occupation? Ils répondirent à Pharaon: Tes serviteurs sont bergers, comme l`étaient nos pères. 
\verse Ils dirent encore à Pharaon: Nous sommes venus pour séjourner dans le pays, parce qu`il n`y a plus de pâturage pour les brebis de tes serviteurs, car la famine s`appesantit sur le pays de Canaan; permets donc à tes serviteurs d`habiter au pays de Gosen. 
\verse Pharaon dit à Joseph: Ton père et tes frères sont venus auprès de toi. 
\verse Le pays d`Égypte est devant toi; établis ton père et tes frères dans la meilleure partie du pays. Qu`ils habitent dans le pays de Gosen; et, si tu trouves parmi eux des hommes capables, mets-les à la tête de mes troupeaux. 
\verse Joseph fit venir Jacob, son père, et le présenta à Pharaon. Et Jacob bénit Pharaon. 
\verse Pharaon dit à Jacob: Quel est le nombre de jours des années de ta vie? 
\verse Jacob répondit à Pharaon: Les jours des années de mon pèlerinage sont de cent trente ans. Les jours des années de ma vie ont été peu nombreux et mauvais, et ils n`ont point atteint les jours des années de la vie de mes pères durant leur pèlerinage. 
\verse Jacob bénit encore Pharaon, et se retira de devant Pharaon. 
\verse Joseph établit son père et ses frères, et leur donna une propriété dans le pays d`Égypte, dans la meilleure partie du pays, dans la contrée de Ramsès, comme Pharaon l`avait ordonné. 
\verse Joseph fournit du pain à son père et à ses frères, et à toute la famille de son père, selon le nombre des enfants. 
\verse Il n`y avait plus de pain dans tout le pays, car la famine était très grande; le pays d`Égypte et le pays de Canaan languissaient, à cause de la famine. 
\verse Joseph recueillit tout l`argent qui se trouvait dans le pays d`Égypte et dans le pays de Canaan, contre le blé qu`on achetait; et il fit entrer cet argent dans la maison de Pharaon. 
\verse Quand l`argent du pays d`Égypte et du pays de Canaan fut épuisé, tous les Égyptiens vinrent à Joseph, en disant: Donne-nous du pain! Pourquoi mourrions-nous en ta présence? car l`argent manque. 
\verse Joseph dit: Donnez vos troupeaux, et je vous donnerai du pain contre vos troupeaux, si l`argent manque. 
\verse Ils amenèrent leurs troupeaux à Joseph, et Joseph leur donna du pain contre les chevaux, contre les troupeaux de brebis et de boeufs, et contre les ânes. Il leur fournit ainsi du pain cette année-là contre tous leurs troupeaux. 
\verse Lorsque cette année fut écoulée, ils vinrent à Joseph l`année suivante, et lui dirent: Nous ne cacherons point à mon seigneur que l`argent est épuisé, et que les troupeaux de bétail ont été amenés à mon seigneur; il ne reste devant mon seigneur que nos corps et nos terres. 
\verse Pourquoi mourrions-nous sous tes yeux, nous et nos terres? Achète-nous avec nos terres contre du pain, et nous appartiendrons à mon seigneur, nous et nos terres. Donne-nous de quoi semer, afin que nous vivions et que nous ne mourions pas, et que nos terres ne soient pas désolées. 
\verse Joseph acheta toutes les terres de l`Égypte pour Pharaon; car les Égyptiens vendirent chacun leur champ, parce que la famine les pressait. Et le pays devint la propriété de Pharaon. 
\verse Il fit passer le peuple dans les villes, d`un bout à l`autre des frontières de l`Égypte. 
\verse Seulement, il n`acheta point les terres des prêtres, parce qu`il y avait une loi de Pharaon en faveur des prêtres, qui vivaient du revenu que leur assurait Pharaon: c`est pourquoi ils ne vendirent point leurs terres. 
\verse Joseph dit au peuple: Je vous ai achetés aujourd`hui avec vos terres, pour Pharaon; voici pour vous de la semence, et vous pourrez ensemencer le sol. 
\verse A la récolte, vous donnerez un cinquième à Pharaon, et vous aurez les quatre autres parties, pour ensemencer les champs, et pour vous nourrir avec vos enfants et ceux qui sont dans vos maisons. 
\verse Ils dirent: Tu nous sauves la vie! que nous trouvions grâce aux yeux de mon seigneur, et nous serons esclaves de Pharaon. 
\verse Joseph fit de cela une loi, qui a subsisté jusqu`à ce jour, et d`après laquelle un cinquième du revenu des terres de l`Égypte appartient à Pharaon; il n`y a que les terres des prêtres qui ne soient point à Pharaon. 
\verse Israël habita dans le pays d`Égypte, dans le pays de Gosen. Ils eurent des possessions, ils furent féconds et multiplièrent beaucoup. 
\verse Jacob vécut dix-sept ans dans le pays d`Égypte; et les jours des années de la vie de Jacob furent de cent quarante-sept ans. 
\verse Lorsqu`Israël approcha du moment de sa mort, il appela son fils Joseph, et lui dit: Si j`ai trouvé grâce à tes yeux, mets, je te prie, ta main sous ma cuisse, et use envers moi de bonté et de fidélité: ne m`enterre pas en Égypte! 
\verse Quand je serai couché avec mes pères, tu me transporteras hors de l`Égypte, et tu m`enterreras dans leur sépulcre. Joseph répondit: Je ferai selon ta parole. 
\verse Jacob dit: Jure-le-moi. Et Joseph le lui jura. Puis Israël se prosterna sur le chevet de son lit. 

\chapter
\verse Après ces choses, l`on vint dire à Joseph: Voici, ton père est malade. Et il prit avec lui ses deux fils, Manassé et Éphraïm. 
\verse On avertit Jacob, et on lui dit: Voici ton fils Joseph qui vient vers toi. Et Israël rassembla ses forces, et s`assit sur son lit. 
\verse Jacob dit à Joseph: Le Dieu tout puissant m`est apparu à Luz, dans le pays de Canaan, et il m`a béni. 
\verse Il m`a dit: Je te rendrai fécond, je te multiplierai, et je ferai de toi une multitude de peuples; je donnerai ce pays à ta postérité après toi, pour qu`elle le possède à toujours. 
\verse Maintenant, les deux fils qui te sont nés au pays d`Égypte, avant mon arrivée vers toi en Égypte, seront à moi; Éphraïm et Manassé seront à moi, comme Ruben et Siméon. 
\verse Mais les enfants que tu as engendrés après eux seront à toi; ils seront appelés du nom de leurs frères dans leur héritage. 
\verse A mon retour de Paddan, Rachel mourut en route auprès de moi, dans le pays de Canaan, à quelque distance d`Éphrata; et c`est là que je l`ai enterrée, sur le chemin d`Éphrata, qui est Bethléhem. 
\verse Israël regarda les fils de Joseph, et dit: Qui sont ceux-ci? 
\verse Joseph répondit à son père: Ce sont mes fils, que Dieu m`a donnés ici. Israël dit: Fais-les, je te prie, approcher de moi, pour que je les bénisse. 
\verse Les yeux d`Israël étaient appesantis par la vieillesse; il ne pouvait plus voir. Joseph les fit approcher de lui; et Israël leur donna un baiser, et les embrassa. 
\verse Israël dit à Joseph: Je ne pensais pas revoir ton visage, et voici que Dieu me fait voir même ta postérité. 
\verse Joseph les retira des genoux de son père, et il se prosterna en terre devant lui. 
\verse Puis Joseph les prit tous deux, Éphraïm de sa main droite à la gauche d`Israël, et Manassé de sa main gauche à la droite d`Israël, et il les fit approcher de lui. 
\verse Israël étendit sa main droite et la posa sur la tête d`Éphraïm qui était le plus jeune, et il posa sa main gauche sur la tête de Manassé: ce fut avec intention qu`il posa ses mains ainsi, car Manassé était le premier-né. 
\verse Il bénit Joseph, et dit: Que le Dieu en présence duquel ont marché mes pères, Abraham et Isaac, que le Dieu qui m`a conduit depuis que j`existe jusqu`à ce jour, 
\verse que l`ange qui m`a délivré de tout mal, bénisse ces enfants! Qu`ils soient appelés de mon nom et du nom de mes pères, Abraham et Isaac, et qu`ils multiplient en abondance au milieu du pays! 
\verse Joseph vit avec déplaisir que son père posait sa main droite sur la tête d`Éphraïm; il saisit la main de son père, pour la détourner de dessus la tête d`Éphraïm, et la diriger sur celle de Manassé. 
\verse Et Joseph dit à son père: Pas ainsi, mon père, car celui-ci est le premier-né; pose ta main droite sur sa tête. 
\verse Son père refusa, et dit: Je le sais, mon fils, je le sais; lui aussi deviendra un peuple, lui aussi sera grand; mais son frère cadet sera plus grand que lui, et sa postérité deviendra une multitude de nations. 
\verse Il les bénit ce jour-là, et dit: C`est par toi qu`Israël bénira, en disant: Que Dieu te traite comme Éphraïm et comme Manassé! Et il mit Éphraïm avant Manassé. 
\verse Israël dit à Joseph: Voici, je vais mourir! Mais Dieu sera avec vous, et il vous fera retourner dans le pays de vos pères. 
\verse Je te donne, de plus qu`à tes frères, une part que j`ai prise de la main des Amoréens avec mon épée et avec mon arc. 

\chapter
\verse Jacob appela ses fils, et dit: Assemblez-vous, et je vous annoncerai ce qui vous arrivera dans la suite des temps. 
\verse Rassemblez-vous, et écoutez, fils de Jacob! Écoutez Israël, votre père! 
\verse Ruben, toi, mon premier-né, Ma force et les prémices de ma vigueur, Supérieur en dignité et supérieur en puissance, 
\verse Impétueux comme les eaux, tu n`auras pas la prééminence! Car tu es monté sur la couche de ton père, Tu as souillé ma couche en y montant. 
\verse Siméon et Lévi sont frères; Leurs glaives sont des instruments de violence. 
\verse Que mon âme n`entre point dans leur conciliabule, Que mon esprit ne s`unisse point à leur assemblée! Car, dans leur colère, ils ont tué des hommes, Et, dans leur méchanceté, ils ont coupé les jarrets des taureaux. 
\verse Maudite soit leur colère, car elle est violente, Et leur fureur, car elle est cruelle! Je les séparerai dans Jacob, Et je les disperserai dans Israël. 
\verse Juda, tu recevras les hommages de tes frères; Ta main sera sur la nuque de tes ennemis. Les fils de ton père se prosterneront devant toi. 
\verse Juda est un jeune lion. Tu reviens du carnage, mon fils! Il ploie les genoux, il se couche comme un lion, Comme une lionne: qui le fera lever? 
\verse Le sceptre ne s`éloignera point de Juda, Ni le bâton souverain d`entre ses pieds, Jusqu`à ce que vienne le Schilo, Et que les peuples lui obéissent. 
\verse Il attache à la vigne son âne, Et au meilleur cep le petit de son ânesse; Il lave dans le vin son vêtement, Et dans le sang des raisins son manteau. 
\verse Il a les yeux rouges de vin, Et les dents blanches de lait. 
\verse Zabulon habitera sur la côte des mers, Il sera sur la côte des navires, Et sa limite s`étendra du côté de Sidon. 
\verse Issacar est un âne robuste, Qui se couche dans les étables. 
\verse Il voit que le lieu où il repose est agréable, Et que la contrée est magnifique; Et il courbe son épaule sous le fardeau, Il s`assujettit à un tribut. 
\verse Dan jugera son peuple, Comme l`une des tribus d`Israël. 
\verse Dan sera un serpent sur le chemin, Une vipère sur le sentier, Mordant les talons du cheval, Pour que le cavalier tombe à la renverse. 
\verse J`espère en ton secours, ô Éternel! 
\verse Gad sera assailli par des bandes armées, Mais il les assaillira et les poursuivra. 
\verse Aser produit une nourriture excellente; Il fournira les mets délicats des rois. 
\verse Nephthali est une biche en liberté; Il profère de belles paroles. 
\verse Joseph est le rejeton d`un arbre fertile, Le rejeton d`un arbre fertile près d`une source; Les branches s`élèvent au-dessus de la muraille. 
\verse Ils l`ont provoqué, ils ont lancé des traits; Les archers l`ont poursuivi de leur haine. 
\verse Mais son arc est demeuré ferme, Et ses mains ont été fortifiées Par les mains du Puissant de Jacob: Il est ainsi devenu le berger, le rocher d`Israël. 
\verse C`est l`oeuvre du Dieu de ton père, qui t`aidera; C`est l`oeuvre du Tout puissant, qui te bénira Des bénédictions des cieux en haut, Des bénédictions des eaux en bas, Des bénédictions des mamelles et du sein maternel. 
\verse Les bénédictions de ton père s`élèvent Au-dessus des bénédictions de mes pères Jusqu`à la cime des collines éternelles: Qu`elles soient sur la tête de Joseph, Sur le sommet de la tête du prince de ses frères! 
\verse Benjamin est un loup qui déchire; Le matin, il dévore la proie, Et le soir, il partage le butin. 
\verse Ce sont là tous ceux qui forment les douze tribus d`Israël. Et c`est là ce que leur dit leur père, en les bénissant. Il les bénit, chacun selon sa bénédiction. 
\verse Puis il leur donna cet ordre: Je vais être recueilli auprès de mon peuple; enterrez-moi avec mes pères, dans la caverne qui est au champ d`Éphron, le Héthien, 
\verse dans la caverne du champ de Macpéla, vis-à-vis de Mamré, dans le pays de Canaan. C`est le champ qu`Abraham a acheté d`Éphron, le Héthien, comme propriété sépulcrale. 
\verse Là on a enterré Abraham et Sara, sa femme; là on a enterré Isaac et Rebecca, sa femme; et là j`ai enterré Léa. 
\verse Le champ et la caverne qui s`y trouve ont été achetés des fils de Heth. 
\verse Lorsque Jacob eut achevé de donner ses ordres à ses fils, il retira ses pieds dans le lit, il expira, et fut recueilli auprès de son peuple. 

\chapter
\verse Joseph se jeta sur le visage de son père, pleura sur lui, et le baisa. 
\verse Il ordonna aux médecins à son service d`embaumer son père, et les médecins embaumèrent Israël. 
\verse Quarante jours s`écoulèrent ainsi, et furent employés à l`embaumer. Et les Égyptiens le pleurèrent soixante-dix jours. 
\verse Quand les jours du deuil furent passés, Joseph s`adressa aux gens de la maison de Pharaon, et leur dit: Si j`ai trouvé grâce à vos yeux, rapportez, je vous prie, à Pharaon ce que je vous dis. 
\verse Mon père m`a fait jurer, en disant: Voici, je vais mourir! Tu m`enterreras dans le sépulcre que je me suis acheté au pays de Canaan. Je voudrais donc y monter, pour enterrer mon père; et je reviendrai. 
\verse Pharaon répondit: Monte, et enterre ton père, comme il te l`a fait jurer. 
\verse Joseph monta, pour enterrer son père. Avec lui montèrent tous les serviteurs de Pharaon, anciens de sa maison, tous les anciens du pays d`Égypte, 
\verse toute la maison de Joseph, ses frères, et la maison de son père: on ne laissa dans le pays de Gosen que les enfants, les brebis et les boeufs. 
\verse Il y avait encore avec Joseph des chars et des cavaliers, en sorte que le cortège était très nombreux. 
\verse Arrivés à l`aire d`Athad, qui est au delà du Jourdain, ils firent entendre de grandes et profondes lamentations; et Joseph fit en l`honneur de son père un deuil de sept jours. 
\verse Les habitants du pays, les Cananéens, furent témoins de ce deuil dans l`aire d`Athad, et ils dirent: Voilà un grand deuil parmi les Égyptiens! C`est pourquoi l`on a donné le nom d`Abel Mitsraïm à cette aire qui est au delà du Jourdain. 
\verse C`est ainsi que les fils de Jacob exécutèrent les ordres de leur père. 
\verse Ils le transportèrent au pays de Canaan, et l`enterrèrent dans la caverne du champ de Macpéla, qu`Abraham avait achetée d`Éphron, le Héthien, comme propriété sépulcrale, et qui est vis-à-vis de Mamré. 
\verse Joseph, après avoir enterré son père, retourna en Égypte, avec ses frères et tous ceux qui étaient montés avec lui pour enterrer son père. 
\verse Quand les frères de Joseph virent que leur père était mort, ils dirent: Si Joseph nous prenait en haine, et nous rendait tout le mal que nous lui avons fait! 
\verse Et ils firent dire à Joseph: Ton père a donné cet ordre avant de mourir: 
\verse Vous parlerez ainsi à Joseph: Oh! pardonne le crime de tes frères et leur péché, car ils t`ont fait du mal! Pardonne maintenant le péché des serviteurs du Dieu de ton père! Joseph pleura, en entendant ces paroles. 
\verse Ses frères vinrent eux-mêmes se prosterner devant lui, et ils dirent: Nous sommes tes serviteurs. 
\verse Joseph leur dit: Soyez sans crainte; car suis-je à la place de Dieu? 
\verse Vous aviez médité de me faire du mal: Dieu l`a changé en bien, pour accomplir ce qui arrive aujourd`hui, pour sauver la vie à un peuple nombreux. 
\verse Soyez donc sans crainte; je vous entretiendrai, vous et vos enfants. Et il les consola, en parlant à leur coeur. 
\verse Joseph demeura en Égypte, lui et la maison de son père. Il vécut cent dix ans. 
\verse Joseph vit les fils d`Éphraïm jusqu`à la troisième génération; et les fils de Makir, fils de Manassé, naquirent sur ses genoux. 
\verse Joseph dit à ses frères: Je vais mourir! Mais Dieu vous visitera, et il vous fera remonter de ce pays-ci dans le pays qu`il a juré de donner à Abraham, à Isaac et à Jacob. 
\verse Joseph fit jurer les fils d`Israël, en disant: Dieu vous visitera; et vous ferez remonter mes os loin d`ici. 
\verse Joseph mourut, âgé de cent dix ans. On l`embauma, et on le mit dans un cercueil en Égypte. 
