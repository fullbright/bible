\book[Livre des Nombres]{Nombres}


\chapter
\verse L`Éternel parla à Moïse dans le désert de Sinaï, dans la tente d`assignation, le premier jour du second mois, la seconde année après leur sortie du pays d`Égypte. Il dit: 
\verse Faites le dénombrement de toute l`assemblée des enfants d`Israël, selon leurs familles, selon les maisons de leurs pères, en comptant par tête les noms de tous les mâles, 
\verse depuis l`âge de vingt ans et au-dessus, tous ceux d`Israël en état de porter les armes; vous en ferez le dénombrement selon leurs divisions, toi et Aaron. 
\verse Il y aura avec vous un homme par tribu, chef de la maison de ses pères. 
\verse Voici les noms des hommes qui se tiendront avec vous. Pour Ruben: Élitsur, fils de Schedéur; 
\verse pour Siméon: Schelumiel, fils de Tsurischaddaï; 
\verse pour Juda: Nachschon, fils d`Amminadab; 
\verse pour Issacar: Nethaneel, fils de Tsuar; 
\verse pour Zabulon: Éliab, fils de Hélon; 
\verse pour les fils de Joseph, -pour Éphraïm: Élischama, fils d`Ammihud; -pour Manassé: Gamliel, fils de Pedahtsur; 
\verse pour Benjamin: Abidan, fils de Guideoni; 
\verse pour Dan: Ahiézer, fils d`Ammischaddaï; 
\verse pour Aser: Paguiel, fils d`Ocran; 
\verse pour Gad: Éliasaph, fils de Déuel; 
\verse pour Nephthali: Ahira, fils d`Énan. 
\verse Tels sont ceux qui furent convoqués à l`assemblée, princes des tribus de leurs pères, chefs des milliers d`Israël. 
\verse Moïse et Aaron prirent ces hommes, qui avaient été désignés par leurs noms, 
\verse et ils convoquèrent toute l`assemblée, le premier jour du second mois. On les enregistra selon leurs familles, selon les maisons de leurs pères, en comptant par tête les noms depuis l`âge de vingt ans et au-dessus. 
\verse Moïse en fit le dénombrement dans le désert de Sinaï, comme l`Éternel le lui avait ordonné. 
\verse On enregistra les fils de Ruben, premier-né d`Israël, selon leurs familles, selon les maisons de leurs pères, en comptant par tête les noms de tous les mâles, depuis l`âge de vingt ans et au-dessus, tous ceux en état de porter les armes: 
\verse les hommes de la tribu de Ruben dont on fit le dénombrement furent quarante-six mille cinq cents. 
\verse On enregistra les fils de Siméon, selon leurs familles, selon les maisons de leurs pères; on en fit le dénombrement, en comptant par tête les noms de tous les mâles depuis l`âge de vingt ans et au-dessus, tous ceux en état de porter les armes: 
\verse les hommes de la tribu de Siméon dont on fit le dénombrement furent cinquante-neuf mille trois cents. 
\verse On enregistra les fils de Gad, selon leurs familles, selon les maisons de leurs pères, en comptant les noms depuis l`âge de vingt ans et au-dessus, tous ceux en état de porter les armes: 
\verse les hommes de la tribu de Gad dont on fit le dénombrement furent quarante-cinq mille six cent cinquante. 
\verse On enregistra les fils de Juda, selon leurs familles, selon les maisons de leurs pères, en comptant les noms depuis l`âge de vingt ans et au-dessus, tous ceux en état de porter les armes: 
\verse les hommes de la tribu de Juda dont on fit le dénombrement furent soixante-quatorze mille six cents. 
\verse On enregistra les fils d`Issacar, selon leurs familles, selon les maisons de leurs pères, en comptant les noms depuis l`âge de vingt ans et au-dessus, tous ceux en état de porter les armes: 
\verse les hommes de la tribu d`Issacar dont on fit le dénombrement furent cinquante-quatre mille quatre cents. 
\verse On enregistra les fils de Zabulon, selon leurs familles, selon les maisons de leurs pères, en comptant les noms depuis l`âge de vingt ans et au-dessus, tous ceux en état de porter les armes: 
\verse les hommes de la tribu de Zabulon dont on fit le dénombrement furent cinquante-sept mille quatre cents. 
\verse On enregistra, d`entre les fils de Joseph, les fils d`Éphraïm, selon leurs familles, selon les maisons de leurs pères, en comptant les noms depuis l`âge de vingt ans et au-dessus, tous ceux en état de porter les armes: 
\verse les hommes de la tribu d`Éphraïm dont on fit le dénombrement furent quarante mille cinq cents. 
\verse On enregistra les fils de Manassé, selon leurs familles, selon les maisons de leurs pères, en comptant les noms depuis l`âge de vingt ans et au-dessus, tous ceux en état de porter les armes: 
\verse les hommes de la tribu de Manassé dont on fit le dénombrement furent trente-deux mille deux cents. 
\verse On enregistra les fils de Benjamin, selon leurs familles, selon les maisons de leurs pères, en comptant les noms depuis l`âge de vingt ans et au-dessus, tous ceux en état de porter les armes: 
\verse les hommes de la tribu de Benjamin dont on fit le dénombrement furent trente-cinq mille quatre cents. 
\verse On enregistra les fils de Dan, selon leurs familles, selon les maisons de leurs pères, en comptant les noms depuis l`âge de vingt ans et au-dessus, tous ceux en état de porter les armes: 
\verse les hommes de la tribu de Dan dont on fit le dénombrement furent soixante-deux mille sept cents. 
\verse On enregistra les fils d`Aser, selon leurs familles, selon les maisons de leurs pères, en comptant les noms depuis l`âge de vingt ans et au-dessus, tous ceux en état de porter les armes: 
\verse les hommes de la tribu d`Aser dont on fit le dénombrement furent quarante et un mille cinq cents. 
\verse On enregistra les fils de Nephthali, selon leurs familles, selon les maisons de leurs pères, en comptant les noms depuis l`âge de vingt ans et au-dessus, tous ceux en état de porter les armes: 
\verse les hommes de la tribu de Nephthali dont on fit le dénombrement furent cinquante-trois mille quatre cents. 
\verse Tels sont ceux dont le dénombrement fut fait par Moïse et Aaron, et par les douze hommes, princes d`Israël; il y avait un homme pour chacune des maisons de leurs pères. 
\verse Tous ceux des enfants d`Israël dont on fit le dénombrement, selon les maisons de leurs pères, depuis l`âge de vingt ans et au-dessus, tous ceux d`Israël en état de porter les armes, 
\verse tous ceux dont on fit le dénombrement furent six cent trois mille cinq cent cinquante. 
\verse Les Lévites, selon la tribu de leurs pères, ne firent point partie de ce dénombrement. 
\verse L`Éternel parla à Moïse, et dit: 
\verse Tu ne feras point le dénombrement de la tribu de Lévi, et tu n`en compteras point les têtes au milieu des enfants d`Israël. 
\verse Remets aux soins des Lévites le tabernacle du témoignage, tous ses ustensiles et tout ce qui lui appartient. Ils porteront le tabernacle et tous ses ustensiles, ils en feront le service, et ils camperont autour du tabernacle. 
\verse Quand le tabernacle partira, les Lévites le démonteront; quand le tabernacle campera, les Lévites le dresseront; et l`étranger qui en approchera sera puni de mort. 
\verse Les enfants d`Israël camperont chacun dans son camp, chacun près de sa bannière, selon leurs divisions. 
\verse Mais les Lévites camperont autour du tabernacle du témoignage, afin que ma colère n`éclate point sur l`assemblée des enfants d`Israël; et les Lévites auront la garde du tabernacle du témoignage. 
\verse Les enfants d`Israël se conformèrent à tous les ordres que l`Éternel avait donnés à Moïse; ils firent ainsi. 

\chapter
\verse L`Éternel parla à Moïse et à Aaron, et dit: 
\verse Les enfants d`Israël camperont chacun près de sa bannière, sous les enseignes de la maison de ses pères; ils camperont vis-à-vis et tout autour de la tente d`assignation. 
\verse A l`orient, le camp de Juda, avec sa bannière, et avec ses corps d`armée. Là camperont le prince des fils de Juda, Nachschon, fils d`Amminadab, 
\verse et son corps d`armée composé de soixante-quatorze mille six cents hommes, d`après le dénombrement. 
\verse A ses côtés camperont la tribu d`Issacar, le prince des fils d`Issacar, Nethaneel, fils de Tsuar, 
\verse et son corps d`armée composé de cinquante-quatre mille quatre cents hommes, d`après le dénombrement; 
\verse puis la tribu de Zabulon, le prince des fils de Zabulon, Éliab, fils de Hélon, 
\verse et son corps d`armée composé de cinquante-sept mille quatre cents hommes, d`après le dénombrement. 
\verse Total pour le camp de Juda, d`après le dénombrement: cent quatre-vingt six mille quatre cents hommes, selon leurs corps d`armée. Ils seront les premiers dans la marche. 
\verse Au midi, le camp de Ruben, avec sa bannière, et avec ses corps d`armée. Là camperont le prince des fils de Ruben, Élitsur, fils de Schedéur, 
\verse et son corps d`armée composé de quarante-six mille cinq cents hommes, d`après le dénombrement. 
\verse A ses côtés camperont la tribu de Siméon, le prince des fils de Siméon, Schelumiel, fils de Tsurischaddaï, 
\verse et son corps d`armée composé de cinquante-neuf mille trois cents hommes, d`après le dénombrement; 
\verse puis la tribu de Gad, le prince des fils de Gad, Éliasaph, fils de Déuel, 
\verse et son corps d`armée composé de quarante-cinq mille six cent cinquante hommes, d`après le dénombrement. 
\verse Total pour le camp de Ruben, d`après le dénombrement: cent cinquante et un mille quatre cent cinquante hommes, selon leurs corps d`armée. Ils seront les seconds dans la marche. 
\verse Ensuite partira la tente d`assignation, avec le camp des Lévites placé au milieu des autres camps: Ils suivront dans la marche l`ordre dans lequel ils auront campé, chacun dans son rang, selon sa bannière. 
\verse A l`occident, le camp d`Éphraïm, avec sa bannière, et avec ses corps d`armée. Là camperont le prince des fils d`Éphraïm, Élischama, fils d`Ammihud, 
\verse et son corps d`armée composé de quarante mille cinq cents hommes, d`après le dénombrement. 
\verse A ses côtés camperont la tribu de Manassé, le prince des fils de Manassé, Gamliel, fils de Pedahtsur, 
\verse et son corps d`armée composé de trente-deux mille deux cents hommes, d`après le dénombrement; 
\verse puis la tribu de Benjamin, le prince des fils de Benjamin, Abidan, fils de Guideoni, 
\verse et son corps d`armée composé de trente-cinq mille quatre cents hommes, d`après le dénombrement. 
\verse Total pour le camp d`Éphraïm, d`après le dénombrement: cent huit mille et cent hommes, selon leurs corps d`armée. Ils seront les troisièmes dans la marche. 
\verse Au nord, le camp de Dan, avec sa bannière, et avec ses corps d`armée. Là camperont le prince des fils de Dan, Ahiézer, fils d`Ammischaddaï, 
\verse et son corps d`armée composé de soixante-deux mille sept cents hommes, d`après le dénombrement. 
\verse A ses côtés camperont la tribu d`Aser, le prince des fils d`Aser, Paguiel, fils d`Ocran, 
\verse et son corps d`armée composé de quarante et un mille cinq cents hommes, d`après le dénombrement; 
\verse puis la tribu de Nephthali, le prince des fils de Nephthali, Ahira, fils d`Énan, 
\verse et son corps d`armée composé de cinquante-trois mille quatre cents hommes, d`après le dénombrement. 
\verse Total pour le camp de Dan, d`après le dénombrement: cent cinquante-sept mille six cents hommes. Ils seront les derniers dans la marche, selon leur bannière. 
\verse Tels sont ceux des enfants d`Israël dont on fit le dénombrement, selon les maisons de leurs pères. Tous ceux dont on fit le dénombrement, et qui formèrent les camps, selon leurs corps d`armée, furent six cent trois mille cinq cent cinquante. 
\verse Les Lévites, suivant l`ordre que l`Éternel avait donné à Moïse, ne firent point partie du dénombrement au milieu des enfants d`Israël. 
\verse Et les enfants d`Israël se conformèrent à tous les ordres que l`Éternel avait donnés à Moïse. C`est ainsi qu`ils campaient, selon leurs bannières; et c`est ainsi qu`ils se mettaient en marche, chacun selon sa famille, selon la maison de ses pères. 

\chapter
\verse Voici la postérité d`Aaron et de Moïse, au temps où l`Éternel parla à Moïse, sur la montagne de Sinaï. 
\verse Voici les noms des fils d`Aaron: Nadab, le premier-né, Abihu, Éléazar et Ithamar. 
\verse Ce sont là les noms des fils d`Aaron, qui reçurent l`onction comme sacrificateurs, et qui furent consacrés pour l`exercice du sacerdoce. 
\verse Nadab et Abihu moururent devant l`Éternel, lorsqu`ils apportèrent devant l`Éternel du feu étranger, dans le désert de Sinaï; ils n`avaient point de fils. Éléazar et Ithamar exercèrent le sacerdoce, en présence d`Aaron, leur père. 
\verse L`Éternel parla à Moïse, et dit: 
\verse Fais approcher la tribu de Lévi, et tu la placeras devant le sacrificateur Aaron, pour qu`elle soit à son service. 
\verse Ils auront le soin de ce qui est remis à sa garde et à la garde de toute l`assemblée, devant la tente d`assignation: ils feront le service du tabernacle. 
\verse Ils auront le soin de tous les ustensiles de la tente d`assignation, et de ce qui est remis à la garde des enfants d`Israël: ils feront le service du tabernacle. 
\verse Tu donneras les Lévites à Aaron et à ses fils; ils lui seront entièrement donnés, de la part des enfants d`Israël. 
\verse Tu établiras Aaron et ses fils pour qu`ils observent les fonctions de leur sacerdoce; et l`étranger qui approchera sera puni de mort. 
\verse L`Éternel parla à Moïse, et dit: 
\verse Voici, j`ai pris les Lévites du milieu des enfants d`Israël, à la place de tous les premiers-nés, des premiers-nés des enfants d`Israël; et les Lévites m`appartiendront. 
\verse Car tout premier-né m`appartient; le jour où j`ai frappé tous les premiers-nés dans le pays d`Égypte, je me suis consacré tous les premiers-nés en Israël, tant des hommes que des animaux: ils m`appartiendront. Je suis l`Éternel. 
\verse L`Éternel parla à Moïse, dans le désert de Sinaï, et dit: 
\verse Fais le dénombrement des enfants de Lévi, selon les maisons de leurs pères, selon leurs familles; tu feras le dénombrement de tous les mâles, depuis l`âge d`un mois et au-dessus. 
\verse Moïse en fit le dénombrement sur l`ordre de l`Éternel, en se conformant à l`ordre qui lui fut donné. 
\verse Ce sont ici les fils de Lévi, d`après leurs noms: Guerschon, Kehath et Merari. - 
\verse Voici les noms des fils de Guerschon, selon leurs familles: Libni et Schimeï. 
\verse Voici les fils de Kehath, selon leurs familles: Amram, Jitsehar, Hébron et Uziel; 
\verse et les fils de Merari, selon leurs familles: Machli et Muschi. Ce sont là les familles de Lévi, selon les maisons de leurs pères. 
\verse De Guerschon descendent la famille de Libni et la famille de Schimeï, formant les familles des Guerschonites. 
\verse Ceux dont on fit le dénombrement, en comptant tous les mâles depuis l`âge d`un mois et au-dessus, furent sept mille cinq cents. 
\verse Les familles des Guerschonites campaient derrière le tabernacle à l`occident. 
\verse Le chef de la maison paternelle des Guerschonites était Éliasaph, fils de Laël. 
\verse Pour ce qui concerne la tente d`assignation, on remit aux soins des fils de Guerschon le tabernacle et la tente, la couverture, le rideau qui est à l`entrée de la tente d`assignation; 
\verse les toiles du parvis et le rideau de l`entrée du parvis, tout autour du tabernacle et de l`autel, et tous les cordages pour le service du tabernacle. 
\verse De Kehath descendent la famille des Amramites, la famille des Jitseharites, la famille des Hébronites et la famille des Uziélites, formant les familles des Kehathites. 
\verse En comptant tous les mâles depuis l`âge d`un mois et au-dessus, il y en eut huit mille six cents, qui furent chargés des soins du sanctuaire. 
\verse Les familles des fils de Kehath campaient au côté méridional du tabernacle. 
\verse Le chef de la maison paternelle des familles des Kehathites était Élitsaphan, fils d`Uziel. 
\verse On remit à leurs soins l`arche, la table, le chandelier, les autels, les ustensiles du sanctuaire, avec lesquels on fait le service, le voile et tout ce qui en dépend. 
\verse Le chef des chefs des Lévites était Éléazar, fils du sacrificateur Aaron; il avait la surveillance de ceux qui étaient chargés des soins du sanctuaire. 
\verse De Merari descendent la famille de Machli et la famille de Muschi, formant les familles des Merarites. 
\verse Ceux dont on fit le dénombrement, en comptant tous les mâles depuis l`âge d`un mois et au-dessus, furent six mille deux cents. 
\verse Le chef de la maison paternelle des familles de Merari était Tsuriel, fils d`Abihaïl. Ils campaient du côté septentrional du tabernacle. 
\verse On remit à la garde et aux soins des fils de Merari les planches du tabernacle, ses barres, ses colonnes et leurs bases, tous ses ustensiles et tout ce qui en dépend; 
\verse les colonnes du parvis tout autour, leurs bases, leurs pieux et leurs cordages. 
\verse Moïse, Aaron et ses fils campaient devant le tabernacle, à l`orient, devant la tente d`assignation, au levant; ils avaient la garde et le soin du sanctuaire, remis à la garde des enfants d`Israël; et l`étranger qui s`approchera sera puni de mort. 
\verse Tous les Lévites dont Moïse et Aaron firent le dénombrement sur l`ordre de l`Éternel, selon leurs familles, tous les mâles depuis l`âge d`un mois et au-dessus, furent vingt-deux mille. 
\verse L`Éternel dit à Moïse: Fais le dénombrement de tous les premiers-nés mâles parmi les enfants d`Israël, depuis l`âge d`un mois et au-dessus, et compte les d`après leurs noms. 
\verse Tu prendras les Lévites pour moi, l`Éternel, à la place de tous les premiers-nés des enfants d`Israël, et le bétail des Lévites à la place de tous les premiers-nés du bétail des enfants d`Israël. 
\verse Moïse fit le dénombrement de tous les premiers-nés parmi les enfants d`Israël, selon l`ordre que l`Éternel lui avait donné. 
\verse Tous les premiers-nés, dont on fit le dénombrement, en comptant les noms, depuis l`âge d`un mois et au-dessus, furent vingt-deux mille deux cent soixante-treize. 
\verse L`Éternel parla à Moïse, et dit: 
\verse Prends les Lévites à la place de tous les premiers-nés des enfants d`Israël, et le bétail des Lévites à la place de leur bétail; et les Lévites m`appartiendront. Je suis l`Éternel. 
\verse Pour le rachat des deux cent soixante-treize qui dépassent le nombre des Lévites, parmi les premiers-nés des enfants d`Israël, 
\verse tu prendras cinq sicles par tête; tu les prendras selon le sicle du sanctuaire, qui est de vingt guéras. 
\verse Tu donneras l`argent à Aaron et à ses fils, pour le rachat de ceux qui dépassent le nombre des Lévites. 
\verse Moïse prit l`argent pour le rachat de ceux qui dépassaient le nombre des rachetés par les Lévites; 
\verse il prit l`argent des premiers-nés des enfants d`Israël: mille trois cent soixante-cinq sicles, selon le sicle du sanctuaire. 
\verse Et Moïse donna l`argent du rachat à Aaron et à ses fils, sur l`ordre de l`Éternel, en se conformant à l`ordre que l`Éternel avait donné à Moïse. 

\chapter
\verse L`Éternel parla à Moïse et à Aaron, et dit: 
\verse Compte les fils de Kehath parmi les enfants de Lévi, selon leurs familles, selon les maisons de leurs pères, 
\verse depuis l`âge de trente ans et au-dessus jusqu`à l`âge de cinquante ans, tous ceux qui sont propres à exercer quelque fonction dans la tente d`assignation. 
\verse Voici les fonctions des fils de Kehath, dans la tente d`assignation: elles concernent le lieu très saint. 
\verse Au départ du camp, Aaron et ses fils viendront démonter le voile, et ils en couvriront l`arche du témoignage; 
\verse ils mettront dessus une couverture de peaux de dauphins, et ils étendront par-dessus un drap entièrement d`étoffe bleue; puis ils placeront les barres de l`arche. 
\verse Ils étendront un drap bleu sur la table des pains de proposition, et ils mettront dessus les plats, les coupes, les tasses et les calices pour les libations; le pain y sera toujours; 
\verse ils étendront sur ces choses un drap de cramoisi, et ils l`envelopperont d`une couverture de peaux de dauphins; puis ils placeront les barres de la table. 
\verse Ils prendront un drap bleu, et ils couvriront le chandelier, ses lampes, ses mouchettes, ses vases à cendre et tous ses vases à l`huile, destinés à son service; 
\verse ils le mettront, avec tous ses ustensiles, dans une couverture de peaux de dauphins; puis ils le placeront sur le brancard. 
\verse Ils étendront un drap bleu sur l`autel d`or, et ils l`envelopperont d`une couverture de peaux de dauphins; puis ils placeront les barres de l`autel. 
\verse Ils prendront tous les ustensiles dont on se sert pour le service dans le sanctuaire, et ils les mettront dans un drap bleu, et ils les envelopperont d`une couverture de peaux de dauphins; puis ils les placeront sur le brancard. 
\verse Ils ôteront les cendres de l`autel, et ils étendront sur l`autel un drap de pourpre; 
\verse ils mettront dessus tous les ustensiles destinés à son service, les brasiers, les fourchettes, les pelles, les bassins, tous les ustensiles de l`autel, et ils étendront par-dessus une couverture de peaux de dauphins; puis ils placeront les barres de l`autel. 
\verse Après qu`Aaron et ses fils auront achevé de couvrir le sanctuaire et tous les ustensiles du sanctuaire, les fils de Kehath viendront, au départ du camp, pour les porter; mais ils ne toucheront point les choses saintes, de peur qu`ils ne meurent. Telles sont les fonctions de porteurs, imposées aux fils de Kehath dans la tente d`assignation. 
\verse Éléazar, fils du sacrificateur Aaron, aura sous sa surveillance l`huile du chandelier, le parfum odoriférant, l`offrande perpétuelle et l`huile d`onction; il aura sous sa surveillance tout le tabernacle et tout ce qu`il contient, le sanctuaire et ses ustensiles. 
\verse L`Éternel parla à Moïse et à Aaron, et dit: 
\verse N`exposez point la race des familles des Kehathites à être retranchée du milieu des Lévites. 
\verse Faites ceci pour eux, afin qu`ils vivent et qu`ils ne meurent point, quand ils s`approcheront du lieu très saint: Aaron et ses fils viendront, et ils placeront chacun d`eux à son service et à sa charge. 
\verse Ils n`entreront point pour voir envelopper les choses saintes, de peur qu`ils ne meurent. 
\verse L`Éternel parla à Moïse, et dit: 
\verse Compte aussi les fils de Guerschon, selon les maisons de leurs pères, selon leurs familles; 
\verse tu feras le dénombrement, depuis l`âge de trente ans et au-dessus jusqu`à l`âge de cinquante ans, de tous ceux qui sont propres à exercer quelque fonction dans la tente d`assignation. 
\verse Voici les fonctions des familles des Guerschonites, le service qu`ils devront faire et ce qu`ils devront porter. 
\verse Ils porteront les tapis du tabernacle et la tente d`assignation, sa couverture et la couverture de peaux de dauphins qui se met par-dessus, le rideau qui est à l`entrée de la tente d`assignation; 
\verse les toiles du parvis et le rideau de l`entrée de la porte du parvis, tout autour du tabernacle et de l`autel, leurs cordages et tous les ustensiles qui en dépendent. Et ils feront tout le service qui s`y rapporte. 
\verse Dans leurs fonctions, les fils des Guerschonites seront sous les ordres d`Aaron et de ses fils, pour tout ce qu`ils porteront et pour tout le service qu`ils devront faire; vous remettrez à leurs soins tout ce qu`ils ont à porter. 
\verse Telles sont les fonctions des familles des fils des Guerschonites dans la tente d`assignation, et ce qu`ils ont à garder sous la direction d`Ithamar, fils du sacrificateur Aaron. 
\verse Tu feras le dénombrement des fils de Merari, selon leurs familles, selon les maisons de leurs pères; 
\verse tu feras le dénombrement, depuis l`âge de trente ans et au-dessus jusqu`à l`âge de cinquante ans, de tous ceux qui sont propres à exercer quelque fonction dans la tente d`assignation. 
\verse Voici ce qui est remis à leurs soins et ce qu`ils ont à porter, pour toutes leurs fonctions dans la tente d`assignation: les planches du tabernacle, ses barres, ses colonnes, ses bases, 
\verse les colonnes du parvis formant l`enceinte, leurs bases, leurs pieux, leurs cordages, tous les ustensiles qui en dépendent et tout ce qui est destiné à leur service. Vous désignerez par leurs noms les objets qui sont remis à leurs soins et qu`ils ont à porter. 
\verse Telles sont les fonctions des familles des fils de Merari, toutes leurs fonctions dans la tente d`assignation, sous la direction d`Ithamar, fils du sacrificateur Aaron. 
\verse Moïse, Aaron et les princes de l`assemblée firent le dénombrement des fils des Kehathites, selon leurs familles et selon les maisons de leurs pères, 
\verse de tous ceux qui, depuis l`âge de trente ans et au-dessus jusqu`à l`âge de cinquante ans, étaient propres à exercer quelque fonction dans la tente d`assignation. 
\verse Ceux dont ils firent le dénombrement, selon leurs familles, furent deux mille sept cent cinquante. 
\verse Tels sont ceux des familles des Kehathites dont on fit le dénombrement, tous ceux qui exerçaient des fonctions dans la tente d`assignation; Moïse et Aaron en firent le dénombrement sur l`ordre de l`Éternel par Moïse. 
\verse Les fils de Guerschon dont on fit le dénombrement, selon leurs familles et selon les maisons de leurs pères, 
\verse depuis l`âge de trente ans et au-dessus jusqu`à l`âge de cinquante ans, tous ceux qui étaient propres à exercer quelque fonction dans la tente d`assignation, 
\verse ceux dont on fit le dénombrement, selon leurs familles, selon les maisons de leurs pères, furent deux mille six cent trente. 
\verse Tels sont ceux des familles des fils de Guerschon dont on fit le dénombrement, tous ceux qui exerçaient des fonctions dans la tente d`assignation; Moïse et Aaron en firent le dénombrement sur l`ordre de l`Éternel. 
\verse Ceux des familles des fils de Merari dont on fit le dénombrement, selon leurs familles, selon les maisons de leurs pères, 
\verse depuis l`âge de trente ans et au-dessus jusqu`à l`âge de cinquante ans, tous ceux qui étaient propres à exercer quelque fonction dans la tente d`assignation, 
\verse ceux dont on fit le dénombrement, selon leurs familles, furent trois mille deux cents. 
\verse Tels sont ceux des familles des fils de Merari dont on fit le dénombrement; Moïse et Aaron en firent le dénombrement sur l`ordre de l`Éternel par Moïse. 
\verse Tous ceux des Lévites dont Moïse, Aaron et les princes d`Israël firent le dénombrement, selon leurs familles et selon les maisons de leurs pères, 
\verse depuis l`âge de trente ans et au-dessus jusqu`à l`âge de cinquante ans, tous ceux qui étaient propres à exercer quelque fonction et à servir de porteurs dans la tente d`assignation, 
\verse tous ceux dont on fit le dénombrement furent huit mille cinq cent quatre-vingts. 
\verse On en fit le dénombrement sur l`ordre de l`Éternel par Moïse, en indiquant à chacun le service qu`il devait faire et ce qu`il devait porter; on en fit le dénombrement selon l`ordre que l`Éternel avait donné à Moïse. 

\chapter
\verse L`Éternel parla à Moïse, et dit: 
\verse Ordonne aux enfants d`Israël de renvoyer du camp tout lépreux, et quiconque a une gonorrhée ou est souillé par un mort. 
\verse Hommes ou femmes, vous les renverrez, vous les renverrez hors du camp, afin qu`ils ne souillent pas le camp au milieu duquel j`ai ma demeure. 
\verse Les enfants d`Israël firent ainsi, et ils les renvoyèrent hors du camp; comme l`Éternel l`avait ordonné à Moïse, ainsi firent les enfants d`Israël. 
\verse L`Éternel parla à Moïse, et dit: 
\verse Parle aux enfants d`Israël: Lorsqu`un homme ou une femme péchera contre son prochain en commettant une infidélité à l`égard de l`Éternel, et qu`il se rendra ainsi coupable, 
\verse il confessera son péché, et il restituera dans son entier l`objet mal acquis, en y ajoutant un cinquième; il le remettra à celui envers qui il s`est rendu coupable. 
\verse S`il n`y a personne qui ait droit à la restitution de l`objet mal acquis, cet objet revient à l`Éternel, au sacrificateur, outre le bélier expiatoire avec lequel on fera l`expiation pour le coupable. 
\verse Toute offrande de choses consacrées par les enfants d`Israël appartiendra au sacrificateur à qui elles seront présentées. 
\verse Les choses qu`on aura consacrées lui appartiendront, ce qu`on lui aura remis lui appartiendra. 
\verse L`Éternel parla à Moïse, et dit: 
\verse Parle aux enfants d`Israël, et tu leur diras: Si une femme se détourne de son mari, et lui devient infidèle; 
\verse si un autre a commerce avec elle, et que la chose soit cachée aux yeux de son mari; si elle s`est souillée en secret, sans qu`il y ait de témoin contre elle, et sans qu`elle ait été prise sur le fait; - 
\verse et si le mari est saisi d`un esprit de jalousie et a des soupçons sur sa femme, qui s`est souillée, ou bien s`il est saisi d`un esprit de jalousie et a des soupçons sur sa femme, qui ne s`est point souillée; - 
\verse cet homme amènera sa femme au sacrificateur, et apportera en offrande pour elle un dixième d`épha de farine d`orge; il n`y répandra point d`huile, et n`y mettra point d`encens, car c`est une offrande de jalousie, une offrande de souvenir, qui rappelle une iniquité. 
\verse Le sacrificateur la fera approcher, et la fera tenir debout devant l`Éternel. 
\verse Le sacrificateur prendra de l`eau sainte dans un vase de terre; il prendra de la poussière sur le sol du tabernacle, et la mettra dans l`eau. 
\verse Le sacrificateur fera tenir la femme debout devant l`Éternel; il découvrira la tête de la femme, et lui posera sur les mains l`offrande de souvenir, l`offrande de jalousie; le sacrificateur aura dans sa main les eaux amères qui apportent la malédiction. 
\verse Le sacrificateur fera jurer la femme, et lui dira: Si aucun homme n`a couché avec toi, et si, étant sous la puissance de ton mari, tu ne t`en es point détournée pour te souiller, ces eaux amères qui apportent la malédiction ne te seront point funestes. 
\verse Mais si, étant sous la puissance de ton mari, tu t`en es détournée et que tu te sois souillée, et si un autre homme que ton mari a couché avec toi, - 
\verse et le sacrificateur fera jurer la femme avec un serment d`imprécation, et lui dira: -Que l`Éternel te livre à la malédiction et à l`exécration au milieu de ton peuple, en faisant dessécher ta cuisse et enfler ton ventre, 
\verse et que ces eaux qui apportent la malédiction entrent dans tes entrailles pour te faire enfler le ventre et dessécher la cuisse! Et la femme dira: Amen! Amen! 
\verse Le sacrificateur écrira ces imprécations dans un livre, puis les effacera avec les eaux amères. 
\verse Et il fera boire à la femme les eaux amères qui apportent la malédiction, et les eaux qui apportent la malédiction entreront en elle pour produire l`amertume. 
\verse Le sacrificateur prendra des mains de la femme l`offrande de jalousie, il agitera l`offrande de côté et d`autre devant l`Éternel, et il l`offrira sur l`autel; 
\verse le sacrificateur prendra une poignée de cette offrande comme souvenir, et il la brûlera sur l`autel. C`est après cela qu`il fera boire les eaux à la femme. 
\verse Quand il aura fait boire les eaux, il arrivera, si elle s`est souillée et a été infidèle à son mari, que les eaux qui apportent la malédiction entreront en elle pour produire l`amertume; son ventre s`enflera, sa cuisse se desséchera, et cette femme sera en malédiction au milieu de son peuple. 
\verse Mais si la femme ne s`est point souillée et qu`elle soit pure, elle sera reconnue innocente et aura des enfants. 
\verse Telle est la loi sur la jalousie, pour le cas où une femme sous la puissance de son mari se détourne et se souille, 
\verse et pour le cas où un mari saisi d`un esprit de jalousie a des soupçons sur sa femme: le sacrificateur la fera tenir debout devant l`Éternel, et lui appliquera cette loi dans son entier. 
\verse Le mari sera exempt de faute, mais la femme portera la peine de son iniquité. 

\chapter
\verse L`Éternel parla à Moïse, et dit: 
\verse Parle aux enfants d`Israël, et tu leur diras: Lorsqu`un homme ou une femme se séparera des autres en faisant voeu de naziréat, pour se consacrer à l`Éternel, 
\verse il s`abstiendra de vin et de boisson enivrante; il ne boira ni vinaigre fait avec du vin, ni vinaigre fait avec une boisson enivrante; il ne boira d`aucune liqueur tirée des raisins, et il ne mangera point de raisins frais ni de raisins secs. 
\verse Pendant tout le temps de son naziréat, il ne mangera rien de ce qui provient de la vigne, depuis les pépins jusqu`à la peau du raisin. 
\verse Pendant tout le temps de son naziréat, le rasoir ne passera point sur sa tête; jusqu`à l`accomplissement des jours pour lesquels il s`est consacré à l`Éternel, il sera saint, il laissera croître librement ses cheveux. 
\verse Pendant tout le temps qu`il a voué à l`Éternel, il ne s`approchera point d`une personne morte; 
\verse il ne se souillera point à la mort de son père, de sa mère, de son frère ou de sa soeur, car il porte sur sa tête la consécration de son Dieu. 
\verse Pendant tout le temps de son naziréat, il sera consacré à l`Éternel. 
\verse Si quelqu`un meurt subitement près de lui, et que sa tête consacrée devienne ainsi souillée, il se rasera la tête le jour de sa purification, il se la rasera le septième jour. 
\verse Le huitième jour, il apportera au sacrificateur deux tourterelles ou deux jeunes pigeons, à l`entrée de la tente d`assignation. 
\verse Le sacrificateur sacrifiera l`un comme victime expiatoire, et l`autre comme holocauste, et il fera pour lui l`expiation de son péché à l`occasion du mort. Le naziréen sanctifiera ainsi sa tête ce jour-là 
\verse Il consacrera de nouveau à l`Éternel les jours de son naziréat, et il offrira un agneau d`un an en sacrifice de culpabilité; les jours précédents ne seront point comptés, parce que son naziréat a été souillé. 
\verse Voici la loi du naziréen. Le jour où il aura accompli le temps de son naziréat, on le fera venir à l`entrée de la tente d`assignation. 
\verse Il présentera son offrande à l`Éternel: un agneau d`un an et sans défaut pour l`holocauste, une brebis d`un an et sans défaut pour le sacrifice d`expiation, et un bélier sans défaut pour le sacrifice d`actions de grâces; 
\verse une corbeille de pains sans levain, de gâteaux de fleur de farine pétris à l`huile, et de galettes sans levain arrosées d`huile, avec l`offrande et la libation ordinaires. 
\verse Le sacrificateur présentera ces choses devant l`Éternel, et il offrira sa victime expiatoire et son holocauste; 
\verse il offrira le bélier en sacrifice d`actions de grâces à l`Éternel, outre la corbeille de pains sans levain, avec l`offrande et la libation. 
\verse Le naziréen rasera, à l`entrée de la tente d`assignation, sa tête consacrée; il prendra les cheveux de sa tête consacrée, et il les mettra sur le feu qui est sous le sacrifice d`actions de grâces. 
\verse Le sacrificateur prendra l`épaule cuite du bélier, un gâteau sans levain de la corbeille, et une galette sans levain; et il les posera sur les mains du naziréen, après qu`il aura rasé sa tête consacrée. 
\verse Le sacrificateur les agitera de côté et d`autre devant l`Éternel: c`est une chose sainte, qui appartient au sacrificateur, avec la poitrine agitée et l`épaule offerte par élévation. Ensuite, le naziréen pourra boire du vin. 
\verse Telle est la loi pour celui qui fait voeu de naziréat; telle est son offrande à l`Éternel pour son naziréat, outre ce que lui permettront ses ressources. Il accomplira ce qui est ordonné pour le voeu qu`il a fait, selon la loi de son naziréat. 
\verse L`Éternel parla à Moïse, et dit: 
\verse Parle à Aaron et à ses fils, et dis: Vous bénirez ainsi les enfants d`Israël, vous leur direz: 
\verse Que l`Éternel te bénisse, et qu`il te garde! 
\verse Que l`Éternel fasse luire sa face sur toi, et qu`il t`accorde sa grâce! 
\verse Que l`Éternel tourne sa face vers toi, et qu`il te donne la paix! 
\verse C`est ainsi qu`ils mettront mon nom sur les enfants d`Israël, et je les bénirai. 

\chapter
\verse Lorsque Moïse eut achevé de dresser le tabernacle, il l`oignit et le sanctifia avec tous ses ustensiles, de même que l`autel avec tous ses ustensiles; il les oignit et les sanctifia. 
\verse Alors les princes d`Israël, chefs des maisons de leurs pères, présentèrent leur offrande: c`étaient les princes des tribus, ceux qui avaient présidé au dénombrement. 
\verse Ils amenèrent leur offrande devant l`Éternel: six chars en forme de litières et douze boeufs, soit un char pour deux princes et un boeuf pour chaque prince; et ils les offrirent devant le tabernacle. 
\verse L`Éternel parla à Moïse, et dit: 
\verse Prends d`eux ces choses, afin de les employer pour le service de la tente d`assignation; tu les donneras aux Lévites, à chacun selon ses fonctions. 
\verse Moïse prit les chars et les boeufs, et il les remit aux Lévites. 
\verse Il donna deux chars et quatre boeufs aux fils de Guerschon, selon leurs fonctions; 
\verse il donna quatre chars et huit boeufs aux fils de Merari, selon leurs fonctions, sous la conduite d`Ithamar, fils du sacrificateur Aaron. 
\verse Mais il n`en donna point aux fils de Kehath, parce que, selon leurs fonctions, ils devaient porter les choses saintes sur les épaules. 
\verse Les princes présentèrent leur offrande pour la dédicace de l`autel, le jour où on l`oignit; les princes présentèrent leur offrande devant l`autel. 
\verse L`Éternel dit à Moïse: Les princes viendront un à un, et à des jours différents, présenter leur offrande pour la dédicace de l`autel. 
\verse Celui qui présenta son offrande le premier jour fut Nachschon, fils d`Amminadab, de la tribu de Juda. 
\verse Il offrit: un plat d`argent du poids de cent trente sicles, un bassin d`argent de soixante-dix sicles, selon le sicle du sanctuaire, tous deux pleins de fleur de farine pétrie à l`huile, pour l`offrande; 
\verse une coupe d`or de dix sicles, pleine de parfum; 
\verse un jeune taureau, un bélier, un agneau d`un an, pour l`holocauste; 
\verse un bouc, pour le sacrifice d`expiation; 
\verse et, pour le sacrifice d`actions de grâces, deux boeufs, cinq béliers, cinq boucs, cinq agneaux d`un an. Telle fut l`offrande de Nachschon, fils d`Amminadab. 
\verse Le second jour, Nethaneel, fils de Tsuar, prince d`Issacar, présenta son offrande. 
\verse Il offrit: un plat d`argent du poids de cent trente sicles, un bassin d`argent de soixante-dix sicles, selon le sicle du sanctuaire, tous deux pleins de fleur de farine pétrie à l`huile, pour l`offrande; 
\verse une coupe d`or de dix sicles, pleine de parfum; 
\verse un jeune taureau, un bélier, un agneau d`un an, pour l`holocauste; 
\verse un bouc, pour le sacrifice d`expiation; 
\verse et, pour le sacrifice d`actions de grâces, deux boeufs, cinq béliers, cinq boucs, cinq agneaux d`un an. Telle fut l`offrande de Nethaneel, fils de Tsuar. 
\verse Le troisième jour, le prince des fils de Zabulon, Éliab, fils de Hélon, 
\verse offrit: un plat d`argent du poids de cent trente sicles, un bassin d`argent de soixante-dix sicles, selon le sicle du sanctuaire, tous deux pleins de fleur de farine pétrie à l`huile, pour l`offrande; 
\verse une coupe d`or de dix sicles, pleine de parfum; 
\verse un jeune taureau, un bélier, un agneau d`un an, pour l`holocauste; 
\verse un bouc, pour le sacrifice d`expiation; 
\verse et, pour le sacrifice d`actions de grâces, deux boeufs, cinq béliers, cinq boucs, cinq agneaux d`un an. Telle fut l`offrande d`Éliab, fils de Hélon. 
\verse Le quatrième jour, le prince des fils de Ruben, Élitsur, fils de Schedéur, 
\verse offrit: un plat d`argent du poids de cent trente sicles, un bassin d`argent de soixante-dix sicles, selon le sicle du sanctuaire, tous deux pleins de fleur de farine pétrie à l`huile, pour l`offrande; 
\verse une coupe d`or de dix sicles, pleine de parfum; 
\verse un jeune taureau, un bélier, un agneau d`un an, pour l`holocauste; 
\verse un bouc, pour le sacrifice d`expiation; 
\verse et, pour le sacrifice d`actions de grâces, deux boeufs, cinq béliers, cinq boucs, cinq agneaux d`un an. Telle fut l`offrande d`Élitsur, fils de Schedéur. 
\verse Le cinquième jour, le prince des fils de Siméon, Schelumiel, fils de Tsurischaddaï, 
\verse offrit: un plat d`argent du poids de cent trente sicles, un bassin d`argent de soixante-dix sicles, selon le sicle du sanctuaire, tous deux pleins de fleur de farine pétrie à l`huile, pour l`offrande; 
\verse une coupe d`or de dix sicles, pleine de parfum; 
\verse un jeune taureau, un bélier, un agneau d`un an, pour l`holocauste; 
\verse un bouc, pour le sacrifice d`expiation; 
\verse et, pour le sacrifice d`actions de grâces, deux boeufs, cinq béliers, cinq boucs, cinq agneaux d`un an. Telle fut l`offrande de Schelumiel, fils de Tsurischaddaï. 
\verse Le sixième jour, le prince des fils de Gad, Éliasaph, fils de Déuel, 
\verse offrit: un plat d`argent du poids de cent trente sicles, un bassin d`argent de soixante-dix sicles, selon le sicle du sanctuaire, tous deux pleins de fleur de farine pétrie à l`huile, pour l`offrande; 
\verse une coupe d`or de dix sicles, pleine de parfum; 
\verse un jeune taureau, un bélier, un agneau d`un an, pour l`holocauste; 
\verse un bouc, pour le sacrifice d`expiation; 
\verse et, pour le sacrifice d`actions de grâces, deux boeufs, cinq béliers, cinq boucs, cinq agneaux d`un an. Telle fut l`offrande d`Éliasaph, fils de Déuel. 
\verse Le septième jour, le prince des fils d`Éphraïm, Élischama, fils d`Ammihud, 
\verse offrit: un plat d`argent du poids de cent trente sicles, un bassin d`argent de soixante-dix sicles, selon le sicle du sanctuaire, tous deux pleins de fleur de farine pétrie à l`huile, pour l`offrande; 
\verse une coupe d`or de dix sicles, pleine de parfum; 
\verse un jeune taureau, un bélier, un agneau d`un an, pour l`holocauste; 
\verse un bouc, pour le sacrifice d`expiation; 
\verse et, pour le sacrifice d`actions de grâces, deux boeufs, cinq béliers, cinq boucs, cinq agneaux d`un an. Telle fut l`offrande d`Élischama, fils d`Ammihud. 
\verse Le huitième jour, le prince des fils de Manassé, Gamliel, fils de Pedahtsur, 
\verse offrit: un plat d`argent du poids de cent trente sicles, un bassin d`argent de soixante-dix sicles, selon le sicle du sanctuaire, tous deux pleins de fleur de farine pétrie à l`huile, pour l`offrande; 
\verse une coupe d`or de dix sicles, pleine de parfum; 
\verse un jeune taureau, un bélier, un agneau d`un an, pour l`holocauste; 
\verse un bouc, pour le sacrifice d`expiation; 
\verse et, pour le sacrifice d`actions de grâces, deux boeufs, cinq béliers, cinq boucs, cinq agneaux d`un an. Telle fut l`offrande de Gamliel, fils de Pedahtsur. 
\verse Le neuvième jour, le prince des fils de Benjamin, Abidan, fils de Guideoni, 
\verse offrit: un plat d`argent du poids de cent trente sicles, un bassin d`argent de soixante-dix sicles, selon le sicle du sanctuaire, tous deux pleins de fleur de farine pétrie à l`huile, pour l`offrande; 
\verse une coupe d`or de dix sicles, pleine de parfum; 
\verse un jeune taureau, un bélier, un agneau d`un an, pour l`holocauste; 
\verse un bouc, pour le sacrifice d`expiation; 
\verse et, pour le sacrifice d`actions de grâces, deux boeufs, cinq béliers, cinq boucs, cinq agneaux d`un an. Telle fut l`offrande d`Abidan, fils de Guideoni. 
\verse Le dixième jour, le prince des fils de Dan, Ahiézer, fils d`Ammischaddaï, 
\verse offrit: un plat d`argent du poids de cent trente sicles, un bassin d`argent de soixante-dix sicles, selon le sicle du sanctuaire, tous deux pleins de fleur de farine pétrie à l`huile, pour l`offrande; 
\verse une coupe d`or de dix sicles, pleine de parfum; 
\verse un jeune taureau, un bélier, un agneau d`un an, pour l`holocauste; 
\verse un bouc, pour le sacrifice d`expiation; 
\verse et, pour le sacrifice d`actions de grâces, deux boeufs, cinq béliers, cinq boucs, cinq agneaux d`un an. Telle fut l`offrande d`Ahiézer, fils d`Ammischaddaï. 
\verse Le onzième jour, le prince des fils d`Aser, Paguiel fils d`Ocran, 
\verse offrit: un plat d`argent du poids de cent trente sicles, un bassin d`argent de soixante-dix sicles, selon le sicle du sanctuaire, tous deux pleins de fleur de farine pétrie à l`huile, pour l`offrande; 
\verse une coupe d`or de dix sicles, pleine de parfum; 
\verse un jeune taureau, un bélier, un agneau d`un an, pour l`holocauste; 
\verse un bouc, pour le sacrifice d`expiation; 
\verse et, pour le sacrifice d`actions de grâces, deux boeufs, cinq béliers, cinq boucs, cinq agneaux d`un an. Telle fut l`offrande de Paguiel, fils d`Ocran. 
\verse Le douzième jour, le prince des fils de Nephthali, Ahira, fils d`Énan, 
\verse offrit: un plat d`argent du poids de cent trente sicles, un bassin d`argent de soixante-dix sicles selon le sicle du sanctuaire, tous deux pleins de fleur de farine pétrie à l`huile, pour l`offrande; 
\verse une coupe d`or de dix sicles, pleine de parfum; 
\verse un jeune taureau, un bélier, un agneau d`un an, pour l`holocauste; 
\verse un bouc, pour le sacrifice d`expiation; 
\verse et, pour le sacrifice d`actions de grâces, deux boeufs, cinq béliers, cinq boucs, cinq agneaux d`un an. Telle fut l`offrande d`Ahira, fils d`Énan. 
\verse Tels furent les dons des princes d`Israël pour la dédicace de l`autel, le jour où on l`oignit. Douze plats d`argent, douze bassins d`argent, douze coupes d`or; 
\verse chaque plat d`argent pesait cent trente sicles, et chaque bassin soixante-dix, ce qui fit pour l`argent de ces ustensiles un total de deux mille quatre cents sicles, selon le sicle du sanctuaire; 
\verse les douze coupes d`or pleines de parfum, à dix sicles la coupe, selon le sicle du sanctuaire, firent pour l`or des coupes un total de cent vingt sicles. 
\verse Total des animaux pour l`holocauste: douze taureaux, douze béliers, douze agneaux d`un an, avec les offrandes ordinaires. Douze boucs, pour le sacrifice d`expiation. 
\verse Total des animaux pour le sacrifice d`actions de grâces: vingt-quatre boeufs, soixante béliers, soixante boucs, soixante agneaux d`un an. Tels furent les dons pour la dédicace de l`autel, après qu`on l`eut oint. 
\verse Lorsque Moïse entrait dans la tente d`assignation pour parler avec l`Éternel, il entendait la voix qui lui parlait du haut du propitiatoire placé sur l`arche du témoignage, entre les deux chérubins. Et il parlait avec l`Éternel. 

\chapter
\verse L`Éternel parla à Moïse, et dit: 
\verse Parle à Aaron, et tu lui diras: Lorsque tu placeras les lampes sur le chandelier, les sept lampes devront éclairer en face. 
\verse Aaron fit ainsi; il plaça les lampes sur le devant du chandelier, comme l`Éternel l`avait ordonné à Moïse. 
\verse Le chandelier était d`or battu; jusqu`à son pied, jusqu`à ses fleurs, il était d`or battu; Moïse avait fait le chandelier d`après le modèle que l`Éternel lui avait montré. 
\verse L`Éternel parla à Moïse, et dit: 
\verse Prends les Lévites du milieu des enfants d`Israël, et purifie-les. 
\verse Voici comment tu les purifieras. Fais sur eux une aspersion d`eau expiatoire; qu`ils fassent passer le rasoir sur tout leur corps, qu`ils lavent leurs vêtements, et qu`ils se purifient. 
\verse Ils prendront ensuite un jeune taureau, avec l`offrande ordinaire de fleur de farine pétrie à l`huile; et tu prendras un autre jeune taureau pour le sacrifice d`expiation. 
\verse Tu feras approcher les Lévites devant la tente d`assignation, et tu convoqueras toute l`assemblée des enfants d`Israël. 
\verse Tu feras approcher les Lévites devant l`Éternel; et les enfants d`Israël poseront leurs mains sur les Lévites. 
\verse Aaron fera tourner de côté et d`autre les Lévites devant l`Éternel, comme une offrande de la part des enfants d`Israël; et ils seront consacrés au service de l`Éternel. 
\verse Les Lévites poseront leurs mains sur la tête des taureaux; et tu offriras l`un en sacrifice d`expiation, et l`autre en holocauste, afin de faire l`expiation pour les Lévites. 
\verse Tu feras tenir les Lévites debout devant Aaron et devant ses fils, et tu les feras tourner de côté et d`autre comme une offrande à l`Éternel. 
\verse Tu sépareras les Lévites du milieu des enfants d`Israël; et les Lévites m`appartiendront. 
\verse Après cela, les Lévites viendront faire le service dans la tente d`assignation. C`est ainsi que tu les purifieras, et que tu les feras tourner de côté et d`autre comme une offrande. 
\verse Car ils me sont entièrement donnés du milieu des enfants d`Israël: je les ai pris pour moi à la place des premiers-nés, de tous les premiers-nés des enfants d`Israël. 
\verse Car tout premier-né des enfants d`Israël m`appartient, tant des hommes que des animaux; le jour où j`ai frappé tous les premiers-nés dans le pays d`Égypte, je me les suis consacrés. 
\verse Et j`ai pris les Lévites à la place de tous les premiers-nés des enfants d`Israël. 
\verse J`ai donné les Lévites entièrement à Aaron et à ses fils, du milieu des enfants d`Israël, pour qu`ils fassent le service des enfants d`Israël dans la tente d`assignation, pour qu`ils fassent l`expiation pour les enfants d`Israël, et pour que les enfants d`Israël ne soient frappés d`aucune plaie, en s`approchant du sanctuaire. 
\verse Moïse, Aaron et toute l`assemblée des enfants d`Israël, firent à l`égard des Lévites tout ce que l`Éternel avait ordonné à Moïse touchant les Lévites; ainsi firent à leur égard les enfants d`Israël. 
\verse Les Lévites se purifièrent, et lavèrent leurs vêtements; Aaron les fit tourner de côté et d`autre comme une offrande devant l`Éternel, et il fit l`expiation pour eux, afin de les purifier. 
\verse Après cela, les Lévites vinrent faire leur service dans la tente d`assignation, en présence d`Aaron et de ses fils, selon ce que l`Éternel avait ordonné à Moïse touchant les Lévites; ainsi fut-il fait à leur égard. 
\verse L`Éternel parla à Moïse, et dit: 
\verse Voici ce qui concerne les Lévites. Depuis l`âge de vingt-cinq ans et au-dessus, tout Lévite entrera au service de la tente d`assignation pour y exercer une fonction. 
\verse Depuis l`âge de cinquante ans, il sortira de fonction, et ne servira plus. 
\verse Il aidera ses frères dans la tente d`assignation, pour garder ce qui est remis à leurs soins; mais il ne fera plus de service. Tu agiras ainsi à l`égard des Lévites pour ce qui concerne leurs fonctions. 

\chapter
\verse L`Éternel parla à Moïse, dans le désert de Sinaï, le premier mois de la seconde année après leur sortie du pays d`Égypte. 
\verse Il dit: Que les enfants d`Israël célèbrent la Pâque au temps fixé. 
\verse Vous la célébrerez au temps fixé, le quatorzième jour de ce mois, entre les deux soirs; vous la célébrerez selon toutes les lois et toutes les ordonnances qui s`y rapportent. 
\verse Moïse parla aux enfants d`Israël, afin qu`ils célébrassent la Pâque. 
\verse Et ils célébrèrent la Pâque le quatorzième jour du premier mois, entre les deux soirs, dans le désert de Sinaï; les enfants d`Israël se conformèrent à tous les ordres que l`Éternel avait donnés à Moïse. 
\verse Il y eut des hommes qui, se trouvant impurs à cause d`un mort, ne pouvaient pas célébrer la Pâque ce jour-là. Ils se présentèrent le même jour devant Moïse et Aaron; 
\verse et ces hommes dirent à Moïse: Nous sommes impurs à cause d`un mort; pourquoi serions-nous privés de présenter au temps fixé l`offrande de l`Éternel au milieu des enfants d`Israël? 
\verse Moïse leur dit: Attendez que je sache ce que l`Éternel vous ordonne. 
\verse Et l`Éternel parla à Moïse, et dit: 
\verse Parle aux enfants d`Israël, et dis-leur: Si quelqu`un d`entre vous ou de vos descendants est impur à cause d`un mort, ou est en voyage dans le lointain, il célébrera la Pâque en l`honneur de l`Éternel. 
\verse C`est au second mois qu`ils la célébreront, le quatorzième jour, entre les deux soirs; ils la mangeront avec des pains sans levain et des herbes amères. 
\verse Ils n`en laisseront rien jusqu`au matin, et ils n`en briseront aucun os. Ils la célébreront selon toutes les ordonnances de la Pâque. 
\verse Si celui qui est pur et qui n`est pas en voyage s`abstient de célébrer la Pâque, celui-là sera retranché de son peuple; parce qu`il n`a pas présenté l`offrande de l`Éternel au temps fixé, cet homme-là portera la peine de son péché. 
\verse Si un étranger en séjour chez vous célèbre la Pâque de l`Éternel, il se conformera aux lois et aux ordonnances de la Pâque. Il y aura une même loi parmi vous, pour l`étranger comme pour l`indigène. 
\verse Le jour où le tabernacle fut dressé, la nuée couvrit le tabernacle, la tente d`assignation; et, depuis le soir jusqu`au matin, elle eut sur le tabernacle l`apparence d`un feu. 
\verse Il en fut continuellement ainsi: la nuée couvrait le tabernacle, et elle avait de nuit l`apparence d`un feu. 
\verse Quand la nuée s`élevait de dessus la tente, les enfants d`Israël partaient; et les enfants d`Israël campaient dans le lieu où s`arrêtait la nuée. 
\verse Les enfants d`Israël partaient sur l`ordre de l`Éternel, et ils campaient sur l`ordre de l`Éternel; ils campaient aussi longtemps que la nuée restait sur le tabernacle. 
\verse Quand la nuée restait longtemps sur le tabernacle, les enfants d`Israël obéissaient au commandement de l`Éternel, et ne partaient point. 
\verse Quand la nuée restait peu de jours sur le tabernacle, ils campaient sur l`ordre de l`Éternel, et ils partaient sur l`ordre de l`Éternel. 
\verse Si la nuée s`arrêtait du soir au matin, et s`élevait le matin, ils partaient. Si la nuée s`élevait après un jour et une nuit, ils partaient. 
\verse Si la nuée s`arrêtait sur le tabernacle deux jours, ou un mois, ou une année, les enfants d`Israël restaient campés, et ne partaient point; et quand elle s`élevait, ils partaient. 
\verse Ils campaient sur l`ordre de l`Éternel, et ils partaient sur l`ordre de l`Éternel; ils obéissaient au commandement de l`Éternel, sur l`ordre de l`Éternel par Moïse. 

\chapter
\verse L`Éternel parla à Moïse, et dit: 
\verse Fais-toi deux trompettes d`argent; tu les feras d`argent battu. Elles te serviront pour la convocation de l`assemblée et pour le départ des camps. 
\verse Quand on en sonnera, toute l`assemblée se réunira auprès de toi, à l`entrée de la tente d`assignation. 
\verse Si l`on ne sonne que d`une trompette, les princes, les chefs des milliers d`Israël, se réuniront auprès de toi. 
\verse Quand vous sonnerez avec éclat, ceux qui campent à l`orient partiront; 
\verse quand vous sonnerez avec éclat pour la seconde fois, ceux qui campent au midi partiront: on sonnera avec éclat pour leur départ. 
\verse Vous sonnerez aussi pour convoquer l`assemblée, mais vous ne sonnerez pas avec éclat. 
\verse Les fils d`Aaron, les sacrificateurs, sonneront des trompettes. Ce sera une loi perpétuelle pour vous et pour vos descendants. 
\verse Lorsque, dans votre pays, vous irez à la guerre contre l`ennemi qui vous combattra, vous sonnerez des trompettes avec éclat, et vous serez présents au souvenir de l`Éternel, votre Dieu, et vous serez délivrés de vos ennemis. 
\verse Dans vos jours de joie, dans vos fêtes, et à vos nouvelles lunes, vous sonnerez des trompettes, en offrant vos holocaustes et vos sacrifices d`actions de grâces, et elles vous mettront en souvenir devant votre Dieu. Je suis l`Éternel, votre Dieu. 
\verse Le vingtième jour du second mois de la seconde année, la nuée s`éleva de dessus le tabernacle du témoignage. 
\verse Et les enfants d`Israël partirent du désert de Sinaï, selon l`ordre fixé pour leur marche. La nuée s`arrêta dans le désert de Paran. 
\verse Ils firent ce premier départ sur l`ordre de l`Éternel par Moïse. 
\verse La bannière du camp des fils de Juda partit la première, avec ses corps d`armée. Le corps d`armée de Juda était commandé par Nachschon, fils d`Amminadab; 
\verse le corps d`armée de la tribu des fils d`Issacar, par Nethaneel, fils de Tsuar; 
\verse le corps d`armée de la tribu des fils de Zabulon, par Éliab, fils de Hélon. 
\verse Le tabernacle fut démonté; et les fils de Guerschon et les fils de Merari partirent, portant le tabernacle. 
\verse La bannière du camp de Ruben partit, avec ses corps d`armée. Le corps d`armée de Ruben était commandé par Élitsur, fils de Schedéur; 
\verse le corps d`armée de la tribu des fils de Siméon, par Schelumiel, fils de Tsurischaddaï; 
\verse le corps d`armée de la tribu des fils de Gad, par Éliasaph, fils de Déuel. 
\verse Les Kehathites partirent, portant le sanctuaire; et l`on dressait le tabernacle en attendant leur arrivée. 
\verse La bannière du camp des fils d`Éphraïm partit, avec ses corps d`armée. Le corps d`armée d`Éphraïm était commandé par Élischama, fils d`Ammihud; 
\verse le corps d`armée de la tribu des fils de Manassé, par Gamliel, fils de Pedahtsur; 
\verse le corps d`armée de la tribu des fils de Benjamin, par Abidan, fils de Guideoni. 
\verse La bannière du camp des fils de Dan partit, avec ses corps d`armée: elle formait l`arrière-garde de tous les camps. Le corps d`armée de Dan était commandé par Ahiézer, fils d`Ammischaddaï; 
\verse le corps d`armée de la tribu des fils d`Aser, par Paguiel, fils d`Ocran; 
\verse le corps d`armée de la tribu des fils de Nephthali, par Ahira, fils d`Énan. 
\verse Tel fut l`ordre d`après lequel les enfants d`Israël se mirent en marche selon leur corps d`armée; et c`est ainsi qu`ils partirent. 
\verse Moïse dit à Hobab, fils de Réuel, le Madianite, beau-père de Moïse: Nous partons pour le lieu dont l`Éternel a dit: Je vous le donnerai. Viens avec nous, et nous te ferons du bien, car l`Éternel a promis de faire du bien à Israël. 
\verse Hobab lui répondit: Je n`irai point; mais j`irai dans mon pays et dans ma patrie. 
\verse Et Moïse dit: Ne nous quitte pas, je te prie; puisque tu connais les lieux où nous campons dans le désert, tu nous serviras de guide. 
\verse Et si tu viens avec nous, nous te ferons jouir du bien que l`Éternel nous fera. 
\verse Ils partirent de la montagne de l`Éternel, et marchèrent trois jours; l`arche de l`alliance de l`Éternel partit devant eux, et fit une marche de trois jours, pour leur chercher un lieu de repos. 
\verse La nuée de l`Éternel était au-dessus d`eux pendant le jour, lorsqu`ils partaient du camp. 
\verse Quand l`arche partait, Moïse disait: Lève-toi, Éternel! et que tes ennemis soient dispersés! que ceux qui te haïssent fuient devant ta face! 
\verse Et quand on la posait, il disait: Reviens, Éternel, aux myriades des milliers d`Israël! 

\chapter
\verse Le peuple murmura et cela déplut aux oreilles l`Éternel. Lorsque l`Éternel l`entendit, sa colère s`enflamma; le feu de l`Éternel s`alluma parmi eux, et dévora l`extrémité du camp. 
\verse Le peuple cria à Moïse. Moïse pria l`Éternel, et le feu s`arrêta. 
\verse On donna à ce lieu le nom de Tabeéra, parce que le feu de l`Éternel s`était allumé parmi eux. 
\verse Le ramassis de gens qui se trouvaient au milieu d`Israël fut saisi de convoitise; et même les enfants d`Israël recommencèrent à pleurer et dirent: Qui nous donnera de la viande à manger? 
\verse Nous nous souvenons des poissons que nous mangions en Égypte, et qui ne nous coûtaient rien, des concombres, des melons, des poireaux, des oignons et des aulx. 
\verse Maintenant, notre âme est desséchée: plus rien! Nos yeux ne voient que de la manne. 
\verse La manne ressemblait à de la graine de coriandre, et avait l`apparence du bdellium. 
\verse Le peuple se dispersait pour la ramasser; il la broyait avec des meules, ou la pilait dans un mortier; il la cuisait au pot, et en faisait des gâteaux. Elle avait le goût d`un gâteau à l`huile. 
\verse Quand la rosée descendait la nuit sur le camp, la manne y descendait aussi. 
\verse Moïse entendit le peuple qui pleurait, chacun dans sa famille et à l`entrée de sa tente. La colère de l`Éternel s`enflamma fortement. 
\verse Moïse fut attristé, et il dit à l`Éternel: Pourquoi affliges-tu ton serviteur, et pourquoi n`ai-je pas trouvé grâce à tes yeux, que tu aies mis sur moi la charge de tout ce peuple? 
\verse Est-ce moi qui ai conçu ce peuple? est-ce moi qui l`ai enfanté, pour que tu me dises: Porte-le sur ton sein, comme le nourricier porte un enfant, jusqu`au pays que tu as juré à ses pères de lui donner? 
\verse Où prendrai-je de la viande pour donner à tout ce peuple? Car ils pleurent auprès de moi, en disant: Donne-nous de la viande à manger! 
\verse Je ne puis pas, à moi seul, porter tout ce peuple, car il est trop pesant pour moi. 
\verse Plutôt que de me traiter ainsi, tue-moi, je te prie, si j`ai trouvé grâce à tes yeux, et que je ne voie pas mon malheur. 
\verse L`Éternel dit à Moïse: Assemble auprès de moi soixante-dix hommes des anciens d`Israël, de ceux que tu connais comme anciens du peuple et ayant autorité sur lui; amène-les à la tente d`assignation, et qu`ils s`y présentent avec toi. 
\verse Je descendrai, et là je te parlerai; je prendrai de l`esprit qui est sur toi, et je le mettrai sur eux, afin qu`ils portent avec toi la charge du peuple, et que tu ne la portes pas à toi seul. 
\verse Tu diras au peuple: Sanctifiez-vous pour demain, et vous mangerez de la viande, puisque vous avez pleuré aux oreilles de l`Éternel, en disant: Qui nous fera manger de la viande? car nous étions bien en Égypte. L`Éternel vous donnera de la viande, et vous en mangerez. 
\verse Vous en mangerez non pas un jour, ni deux jours, ni cinq jours, ni dix jours, ni vingt jours, 
\verse mais un mois entier, jusqu`à ce qu`elle vous sorte par les narines et que vous en ayez du dégoût, parce que vous avez rejeté l`Éternel qui est au milieu de vous, et parce que vous avez pleuré devant lui, en disant: Pourquoi donc sommes-nous sortis d`Égypte? 
\verse Moïse dit: Six cent mille hommes de pied forment le peuple au milieu duquel je suis, et tu dis: Je leur donnerai de la viande, et ils en mangeront un mois entier! 
\verse Égorgera-t-on pour eux des brebis et des boeufs, en sorte qu`ils en aient assez? ou rassemblera-t-on pour eux tous les poissons de la mer, en sorte qu`ils en aient assez? 
\verse L`Éternel répondit à Moïse: La main de l`Éternel serait-elle trop courte? Tu verras maintenant si ce que je t`ai dit arrivera ou non. 
\verse Moïse sortit, et rapporta au peuple les paroles de l`Éternel. Il assembla soixante-dix hommes des anciens du peuple, et les plaça autour de la tente. 
\verse L`Éternel descendit dans la nuée, et parla à Moïse; il prit de l`esprit qui était sur lui, et le mit sur les soixante-dix anciens. Et dès que l`esprit reposa sur eux, ils prophétisèrent; mais ils ne continuèrent pas. 
\verse Il y eut deux hommes, l`un appelé Eldad, et l`autre Médad, qui étaient restés dans le camp, et sur lesquels l`esprit reposa; car ils étaient parmi les inscrits, quoiqu`ils ne fussent point allés à la tente; et ils prophétisèrent dans le camp. 
\verse Un jeune garçon courut l`annoncer à Moïse, et dit: Eldad et Médad prophétisent dans le camp. 
\verse Et Josué, fils de Nun, serviteur de Moïse depuis sa jeunesse, prit la parole et dit: Moïse, mon seigneur, empêche-les! 
\verse Moïse lui répondit: Es-tu jaloux pour moi? Puisse tout le peuple de l`Éternel être composé de prophètes; et veuille l`Éternel mettre son esprit sur eux! 
\verse Et Moïse se retira au camp, lui et les anciens d`Israël. 
\verse L`Éternel fit souffler de la mer un vent, qui amena des cailles, et les répandit sur le camp, environ une journée de chemin d`un côté et environ une journée de chemin de l`autre côté, autour du camp. Il y en avait près de deux coudées au-dessus de la surface de la terre. 
\verse Pendant tout ce jour et toute la nuit, et pendant toute la journée du lendemain, le peuple se leva et ramassa les cailles; celui qui en avait ramassé le moins en avait dix homers. Ils les étendirent pour eux autour du camp. 
\verse Comme la chair était encore entre leurs dents sans être mâchée, la colère de l`Éternel s`enflamma contre le peuple, et l`Éternel frappa le peuple d`une très grande plaie. 
\verse On donna à ce lieu le nom de Kibroth Hattaava, parce qu`on y enterra le peuple que la convoitise avait saisi. 
\verse De Kibroth Hattaava le peuple partit pour Hatséroth, et il s`arrêta à Hatséroth. 

\chapter
\verse Marie et Aaron parlèrent contre Moïse au sujet de la femme éthiopienne qu`il avait prise, car il avait pris une femme éthiopienne. 
\verse Ils dirent: Est-ce seulement par Moïse que l`Éternel parle? N`est-ce pas aussi par nous qu`il parle? 
\verse Et l`Éternel l`entendit. Or, Moïse était un homme fort patient, plus qu`aucun homme sur la face de la terre. 
\verse Soudain l`Éternel dit à Moïse, à Aaron et à Marie: Allez, vous trois, à la tente d`assignation. Et ils y allèrent tous les trois. 
\verse L`Éternel descendit dans la colonne de nuée, et il se tint à l`entrée de la tente. Il appela Aaron et Marie, qui s`avancèrent tous les deux. 
\verse Et il dit: Écoutez bien mes paroles! Lorsqu`il y aura parmi vous un prophète, c`est dans une vision que moi, l`Éternel, je me révélerai à lui, c`est dans un songe que je lui parlerai. 
\verse Il n`en est pas ainsi de mon serviteur Moïse. Il est fidèle dans toute ma maison. 
\verse Je lui parle bouche à bouche, je me révèle à lui sans énigmes, et il voit une représentation de l`Éternel. Pourquoi donc n`avez-vous pas craint de parler contre mon serviteur, contre Moïse? 
\verse La colère de l`Éternel s`enflamma contre eux. Et il s`en alla. 
\verse La nuée se retira de dessus la tente. Et voici, Marie était frappée d`une lèpre, blanche comme la neige. Aaron se tourna vers Marie; et voici, elle avait la lèpre. 
\verse Alors Aaron dit à Moïse: De grâce, mon seigneur, ne nous fais pas porter la peine du péché que nous avons commis en insensés, et dont nous nous sommes rendus coupables! 
\verse Oh! qu`elle ne soit pas comme l`enfant mort-né, dont la chair est à moitié consumée quand il sort du sein de sa mère! 
\verse Moïse cria à l`Éternel, en disant: O Dieu, je te prie, guéris-la! 
\verse Et l`Éternel dit à Moïse: Si son père lui avait craché au visage, ne serait-elle pas pendant sept jours un objet de honte? Qu`elle soit enfermée sept jours en dehors du camp; après quoi, elle y sera reçue. 
\verse Marie fut enfermée sept jours en dehors du camp; et le peuple ne partit point, jusqu`à ce que Marie y fut rentrée. 
\verse Après cela, le peuple partit de Hatséroth, et il campa dans le désert de Paran. 

\chapter
\verse L`Éternel parla à Moïse, et dit: 
\verse Envoie des hommes pour explorer le pays de Canaan, que je donne aux enfants d`Israël. Tu enverras un homme de chacune des tribus de leurs pères; tous seront des principaux d`entre eux. 
\verse Moïse les envoya du désert de Paran, d`après l`ordre de l`Éternel; tous ces hommes étaient chefs des enfants d`Israël. 
\verse Voici leurs noms. Pour la tribu de Ruben: Schammua, fils de Zaccur; 
\verse pour la tribu de Siméon: Schaphath, fils de Hori; 
\verse pour la tribu de Juda: Caleb, fils de Jephunné; 
\verse pour la tribu d`Issacar: Jigual, fils de Joseph; 
\verse pour la tribu d`Éphraïm: Hosée, fils de Nun; 
\verse pour la tribu de Benjamin: Palthi, fils de Raphu; 
\verse pour la tribu de Zabulon: Gaddiel, fils de Sodi; 
\verse pour la tribu de Joseph, la tribu de Manassé: Gaddi, fils de Susi; 
\verse pour la tribu de Dan: Ammiel, fils de Guemalli; 
\verse pour la tribu d`Aser: Sethur, fils de Micaël; 
\verse pour la tribu de Nephthali: Nachbi, fils de Vophsi; 
\verse pour la tribu de Gad: Guéuel, fils de Maki. 
\verse Tels sont les noms des hommes que Moïse envoya pour explorer le pays. Moïse donna à Hosée, fils de Nun, le nom de Josué. 
\verse Moïse les envoya pour explorer le pays de Canaan. Il leur dit: Montez ici, par le midi; et vous monterez sur la montagne. 
\verse Vous verrez le pays, ce qu`il est, et le peuple qui l`habite, s`il est fort ou faible, s`il est en petit ou en grand nombre; 
\verse ce qu`est le pays où il habite, s`il est bon ou mauvais; ce que sont les villes où il habite, si elles sont ouvertes ou fortifiées; 
\verse ce qu`est le terrain, s`il est gras ou maigre, s`il y a des arbres ou s`il n`y en a point. Ayez bon courage, et prenez des fruits du pays. C`était le temps des premiers raisins. 
\verse Ils montèrent, et ils explorèrent le pays, depuis le désert de Tsin jusqu`à Rehob, sur le chemin de Hamath. 
\verse Ils montèrent, par le midi, et ils allèrent jusqu`à Hébron, où étaient Ahiman, Schéschaï et Talmaï, enfants d`Anak. Hébron avait été bâtie sept ans avant Tsoan en Égypte. 
\verse Ils arrivèrent jusqu`à la vallée d`Eschcol, où ils coupèrent une branche de vigne avec une grappe de raisin, qu`ils portèrent à deux au moyen d`une perche; ils prirent aussi des grenades et des figues. 
\verse On donna à ce lieu le nom de vallée d`Eschcol, à cause de la grappe que les enfants d`Israël y coupèrent. 
\verse Ils furent de retour de l`exploration du pays au bout de quarante jours. 
\verse A leur arrivée, ils se rendirent auprès de Moïse et d`Aaron, et de toute l`assemblée des enfants d`Israël, à Kadès dans le désert de Paran. Ils leur firent un rapport, ainsi qu`à toute l`assemblée, et ils leur montrèrent les fruits du pays. 
\verse Voici ce qu`ils racontèrent à Moïse: Nous sommes allés dans le pays où tu nous as envoyés. A la vérité, c`est un pays où coulent le lait et le miel, et en voici les fruits. 
\verse Mais le peuple qui habite ce pays est puissant, les villes sont fortifiées, très grandes; nous y avons vu des enfants d`Anak. 
\verse Les Amalécites habitent la contrée du midi; les Héthiens, les Jébusiens et les Amoréens habitent la montagne; et les Cananéens habitent près de la mer et le long du Jourdain. 
\verse Caleb fit taire le peuple, qui murmurait contre Moïse. Il dit: Montons, emparons-nous du pays, nous y serons vainqueurs! 
\verse Mais les hommes qui y étaient allés avec lui dirent: Nous ne pouvons pas monter contre ce peuple, car il est plus fort que nous. 
\verse Et ils décrièrent devant les enfants d`Israël le pays qu`ils avaient exploré. Ils dirent: Le pays que nous avons parcouru, pour l`explorer, est un pays qui dévore ses habitants; tous ceux que nous y avons vus sont des hommes d`une haute taille; 
\verse et nous y avons vu les géants, enfants d`Anak, de la race des géants: nous étions à nos yeux et aux leurs comme des sauterelles. 

\chapter
\verse Toute l`assemblée éleva la voix et poussa des cris, et le peuple pleura pendant la nuit. 
\verse Tous les enfants d`Israël murmurèrent contre Moïse et Aaron, et toute l`assemblée leur dit: Que ne sommes-nous morts dans le pays d`Égypte, ou que ne sommes-nous morts dans ce désert! 
\verse Pourquoi l`Éternel nous fait-il aller dans ce pays, où nous tomberons par l`épée, où nos femmes et nos petits enfants deviendront une proie? Ne vaut-il pas mieux pour nous retourner en Égypte? 
\verse Et ils se dirent l`un à l`autre: Nommons un chef, et retournons en Égypte. 
\verse Moïse et Aaron tombèrent sur leur visage, en présence de toute l`assemblée réunie des enfants d`Israël. 
\verse Et, parmi ceux qui avaient exploré le pays, Josué, fils de Nun, et Caleb, fils de Jephunné, déchirèrent leurs vêtements, 
\verse et parlèrent ainsi à toute l`assemblée des enfants d`Israël: Le pays que nous avons parcouru, pour l`explorer, est un pays très bon, excellent. 
\verse Si l`Éternel nous est favorable, il nous mènera dans ce pays, et nous le donnera: c`est un pays où coulent le lait et le miel. 
\verse Seulement, ne soyez point rebelles contre l`Éternel, et ne craignez point les gens de ce pays, car ils nous serviront de pâture, ils n`ont plus d`ombrage pour les couvrir, l`Éternel est avec nous, ne les craignez point! 
\verse Toute l`assemblée parlait de les lapider, lorsque la gloire de l`Éternel apparut sur la tente d`assignation, devant tous les enfants d`Israël. 
\verse Et l`Éternel dit à Moïse: Jusqu`à quand ce peuple me méprisera-t-il? Jusqu`à quand ne croira-t-il pas en moi, malgré tous les prodiges que j`ai faits au milieu de lui? 
\verse Je le frapperai par la peste, et je le détruirai; mais je ferai de toi une nation plus grande et plus puissante que lui. 
\verse Moïse dit à l`Éternel: Les Égyptiens l`apprendront, eux du milieu desquels tu as fait monter ce peuple par ta puissance, 
\verse et ils le diront aux habitants de ce pays. Ils savaient que toi, l`Éternel, tu es au milieu de ce peuple; que tu apparais visiblement, toi, l`Éternel; que ta nuée se tient sur lui; que tu marches devant lui le jour dans une colonne de nuée, et la nuit dans une colonne de feu. 
\verse Si tu fais mourir ce peuple comme un seul homme, les nations qui ont entendu parler de toi diront: 
\verse L`Éternel n`avait pas le pouvoir de mener ce peuple dans le pays qu`il avait juré de lui donner: c`est pour cela qu`il l`a égorgé dans le désert. 
\verse Maintenant, que la puissance du Seigneur se montre dans sa grandeur, comme tu l`as déclaré en disant: 
\verse L`Éternel est lent à la colère et riche en bonté, il pardonne l`iniquité et la rébellion; mais il ne tient point le coupable pour innocent, et il punit l`iniquité des pères sur les enfants jusqu`à la troisième et la quatrième génération. 
\verse Pardonne l`iniquité de ce peuple, selon la grandeur de ta miséricorde, comme tu as pardonné à ce peuple depuis l`Égypte jusqu`ici. 
\verse Et l`Éternel dit: Je pardonne, comme tu l`as demandé. 
\verse Mais, je suis vivant! et la gloire de l`Éternel remplira toute la terre. 
\verse Tous ceux qui ont vu ma gloire, et les prodiges que j`ai faits en Égypte et dans le désert, qui m`ont tenté déjà dix fois, et qui n`ont point écouté ma voix, 
\verse tous ceux-là ne verront point le pays que j`ai juré à leurs pères de leur donner, tous ceux qui m`ont méprisé ne le verront point. 
\verse Et parce que mon serviteur Caleb a été animé d`un autre esprit, et qu`il a pleinement suivi ma voie, je le ferai entrer dans le pays où il est allé, et ses descendants le posséderont. 
\verse Les Amalécites et les Cananéens habitent la vallée: demain, tournez-vous, et partez pour le désert, dans la direction de la mer Rouge. 
\verse L`Éternel parla à Moïse et à Aaron, et dit: 
\verse Jusqu`à quand laisserai-je cette méchante assemblée murmurer contre moi? J`ai entendu les murmures des enfants d`Israël qui murmuraient contre moi. 
\verse Dis-leur: Je suis vivant! dit l`Éternel, je vous ferai ainsi que vous avez parlé à mes oreilles. 
\verse Vos cadavres tomberont dans ce désert. Vous tous, dont on a fait le dénombrement, en vous comptant depuis l`âge de vingt ans et au-dessus, et qui avez murmuré contre moi, 
\verse vous n`entrerez point dans le pays que j`avais juré de vous faire habiter, excepté Caleb, fils de Jephunné, et Josué, fils de Nun. 
\verse Et vos petits enfants, dont vous avez dit: Ils deviendront une proie! je les y ferai entrer, et ils connaîtront le pays que vous avez dédaigné. 
\verse Vos cadavres, à vous, tomberont dans le désert; 
\verse et vos enfants paîtront quarante années dans le désert, et porteront la peine de vos infidélités, jusqu`à ce que vos cadavres soient tous tombés dans le désert. 
\verse De même que vous avez mis quarante jours à explorer le pays, vous porterez la peine de vos iniquités quarante années, une année pour chaque jour; et vous saurez ce que c`est que d`être privé de ma présence. 
\verse Moi, l`Éternel, j`ai parlé! et c`est ainsi que je traiterai cette méchante assemblée qui s`est réunie contre moi; ils seront consumés dans ce désert, ils y mourront. 
\verse Les hommes que Moïse avait envoyés pour explorer le pays, et qui, à leur retour, avaient fait murmurer contre lui toute l`assemblée, en décriant le pays; 
\verse ces hommes, qui avaient décrié le pays, moururent frappés d`une plaie devant l`Éternel. 
\verse Josué, fils de Nun, et Caleb, fils de Jephunné, restèrent seuls vivants parmi ces hommes qui étaient allés pour explorer le pays. 
\verse Moïse rapporta ces choses à tous les enfants d`Israël, et le peuple fut dans une grande désolation. 
\verse Ils se levèrent de bon matin, et montèrent au sommet de la montagne, en disant: Nous voici! nous monterons au lieu dont a parlé l`Éternel, car nous avons péché. 
\verse Moïse dit: Pourquoi transgressez-vous l`ordre de l`Éternel? Cela ne réussira point. 
\verse Ne montez pas! car l`Éternel n`est pas au milieu de vous. Ne vous faites pas battre par vos ennemis. 
\verse Car les Amalécites et les Cananéens sont là devant vous, et vous tomberez par l`épée. Parce que vous vous êtes détournés de l`Éternel, l`Éternel ne sera point avec vous. 
\verse Ils s`obstinèrent à monter au sommet de la montagne; mais l`arche de l`alliance et Moïse ne sortirent point du milieu du camp. 
\verse Alors descendirent les Amalécites et les Cananéens qui habitaient cette montagne; ils les battirent, et les taillèrent en pièces jusqu`à Horma. 

\chapter
\verse L`Éternel parla à Moïse, et dit: 
\verse Parle aux enfants d`Israël, et dis-leur: Quand vous serez entrés dans le pays que je vous donne pour y établir vos demeures, 
\verse et que vous offrirez à l`Éternel un sacrifice consumé par le feu, soit un holocauste, soit un sacrifice en accomplissement d`un voeu ou en offrande volontaire, ou bien dans vos fêtes, pour produire avec votre gros ou votre menu bétail une agréable odeur à l`Éternel, - 
\verse celui qui fera son offrande à l`Éternel présentera en offrande un dixième de fleur de farine pétrie dans un quart de hin d`huile, 
\verse et tu feras une libation d`un quart de hin de vin, avec l`holocauste ou le sacrifice, pour chaque agneau. 
\verse Pour un bélier, tu présenteras en offrande deux dixièmes de fleur de farine pétrie dans un tiers de hin d`huile, 
\verse et tu feras une libation d`un tiers de hin de vin, comme offrande d`une agréable odeur à l`Éternel. 
\verse Si tu offres un veau, soit comme holocauste, soit comme sacrifice en accomplissement d`un voeu, ou comme sacrifice d`actions de grâces à l`Éternel, 
\verse on présentera en offrande, avec le veau, trois dixièmes de fleur de farine pétrie dans un demi-hin d`huile, 
\verse et tu feras une libation d`un demi-hin de vin: c`est un sacrifice consumé par le feu, d`une agréable odeur à l`Éternel. 
\verse On fera ainsi pour chaque boeuf, pour chaque bélier, pour chaque petit des brebis ou des chèvres. 
\verse Suivant le nombre des victimes, vous ferez ainsi pour chacune, d`après leur nombre. 
\verse Tout indigène fera ces choses ainsi, lorsqu`il offrira un sacrifice consumé par le feu, d`une agréable odeur à l`Éternel. 
\verse Si un étranger séjournant chez vous, ou se trouvant à l`avenir au milieu de vous, offre un sacrifice consumé par le feu, d`une agréable odeur à l`Éternel, il l`offrira de la même manière que vous. 
\verse Il y aura une seule loi pour toute l`assemblée, pour vous et pour l`étranger en séjour au milieu de vous; ce sera une loi perpétuelle parmi vos descendants: il en sera de l`étranger comme de vous, devant l`Éternel. 
\verse Il y aura une seule loi et une seule ordonnance pour vous et pour l`étranger en séjour parmi vous. 
\verse L`Éternel parla à Moïse, et dit: 
\verse Parle aux enfants d`Israël, et dis-leur: Quand vous serez arrivés dans le pays où je vous ferai entrer, 
\verse et que vous mangerez du pain de ce pays, vous prélèverez une offrande pour l`Éternel. 
\verse Vous présenterez par élévation un gâteau, les prémices de votre pâte; vous le présenterez comme l`offrande qu`on prélève de l`aire. 
\verse Vous prélèverez pour l`Éternel une offrande des prémices de votre pâte, dans les temps à venir. 
\verse Si vous péchez involontairement, en n`observant pas tous ces commandements que l`Éternel a fait connaître à Moïse, 
\verse tout ce que l`Éternel vous a ordonné par Moïse, depuis le jour où l`Éternel a donné des commandements et plus tard dans les temps à venir; 
\verse si l`on a péché involontairement, sans que l`assemblée s`en soit aperçue, toute l`assemblée offrira un jeune taureau en holocauste d`une agréable odeur à l`Éternel, avec l`offrande et la libation, d`après les règles établies; elle offrira encore un bouc en sacrifice d`expiation. 
\verse Le sacrificateur fera l`expiation pour toute l`assemblée des enfants d`Israël, et il leur sera pardonné; car ils ont péché involontairement, et ils ont apporté leur offrande, un sacrifice consumé par le feu en l`honneur de l`Éternel et une victime expiatoire devant l`Éternel, à cause du péché qu`ils ont involontairement commis. 
\verse Il sera pardonné à toute l`assemblée des enfants d`Israël et à l`étranger en séjour au milieu d`eux, car c`est involontairement que tout le peuple a péché. 
\verse Si c`est une seule personne qui a péché involontairement, elle offrira une chèvre d`un an en sacrifice pour le péché. 
\verse Le sacrificateur fera l`expiation pour la personne qui a péché involontairement devant l`Éternel: quand il aura fait l`expiation pour elle, il lui sera pardonné. 
\verse Pour l`indigène parmi les enfants d`Israël et pour l`étranger en séjour au milieu d`eux, il y aura pour vous une même loi, quand on péchera involontairement. 
\verse Mais si quelqu`un, indigène ou étranger, agit la main levée, il outrage l`Éternel; celui-là sera retranché du milieu de son peuple. 
\verse Il a méprisé la parole de l`Éternel, et il a violé son commandement: celui-là sera retranché, il portera la peine de son iniquité. 
\verse Comme les enfants d`Israël étaient dans le désert, on trouva un homme qui ramassait du bois le jour du sabbat. 
\verse Ceux qui l`avaient trouvé ramassant du bois l`amenèrent à Moïse, à Aaron, et à toute l`assemblée. 
\verse On le mit en prison, car ce qu`on devait lui faire n`avait pas été déclaré. 
\verse L`Éternel dit à Moïse: Cet homme sera puni de mort, toute l`assemblée le lapidera hors du camp. 
\verse Toute l`assemblée le fit sortir du camp et le lapida, et il mourut, comme l`Éternel l`avait ordonné à Moïse. 
\verse L`Éternel dit à Moïse: 
\verse Parle aux enfants d`Israël, et dis-leur qu`ils se fassent, de génération en génération, une frange au bord de leurs vêtements, et qu`ils mettent un cordon bleu sur cette frange du bord de leurs vêtements. 
\verse Quand vous aurez cette frange, vous la regarderez, et vous vous souviendrez de tous les commandements de l`Éternel pour les mettre en pratique, et vous ne suivrez pas les désirs de vos coeurs et de vos yeux pour vous laisser entraîner à l`infidélité. 
\verse Vous vous souviendrez ainsi de mes commandements, vous les mettrez en pratique, et vous serez saints pour votre Dieu. 
\verse Je suis l`Éternel, votre Dieu, qui vous ai fait sortir du pays d`Égypte, pour être votre Dieu. Je suis l`Éternel, votre Dieu. 

\chapter
\verse Koré, fils de Jitsehar, fils de Kehath, fils de Lévi, se révolta avec Dathan et Abiram, fils d`Éliab, et On, fils de Péleth, tous trois fils de Ruben. 
\verse Ils se soulevèrent contre Moïse, avec deux cent cinquante hommes des enfants d`Israël, des principaux de l`assemblée, de ceux que l`on convoquait à l`assemblée, et qui étaient des gens de renom. 
\verse Ils s`assemblèrent contre Moïse et Aaron, et leur dirent: C`en est assez! car toute l`assemblée, tous sont saints, et l`Éternel est au milieu d`eux. Pourquoi vous élevez-vous au-dessus de l`assemblée de l`Éternel? 
\verse Quand Moïse eut entendu cela, il tomba sur son visage. 
\verse Il parla à Koré et à toute sa troupe, en disant: Demain, l`Éternel fera connaître qui est à lui et qui est saint, et il le fera approcher de lui; il fera approcher de lui celui qu`il choisira. 
\verse Faites ceci. Prenez des brasiers, Koré et toute sa troupe. 
\verse Demain, mettez-y du feu, et posez-y du parfum devant l`Éternel; celui que l`Éternel choisira, c`est celui-là qui sera saint. C`en est assez, enfants de Lévi! 
\verse Moïse dit à Koré: Écoutez donc, enfants de Lévi: 
\verse Est-ce trop peu pour vous que le Dieu d`Israël vous ait choisis dans l`assemblée d`Israël, en vous faisant approcher de lui, afin que vous soyez employés au service du tabernacle de l`Éternel, et que vous vous présentiez devant l`assemblée pour la servir? 
\verse Il vous a fait approcher de lui, toi, et tous tes frères, les enfants de Lévi, et vous voulez encore le sacerdoce! 
\verse C`est à cause de cela que toi et toute ta troupe, vous vous assemblez contre l`Éternel! car qui est Aaron, pour que vous murmuriez contre lui? 
\verse Moïse envoya appeler Dathan et Abiram, fils d`Éliab. Mais ils dirent: Nous ne monterons pas. 
\verse N`est-ce pas assez que tu nous aies fait sortir d`un pays où coulent le lait et le miel pour nous faire mourir au désert, sans que tu continues à dominer sur nous? 
\verse Et ce n`est pas dans un pays où coulent le lait et le miel que tu nous as menés, ce ne sont pas des champs et des vignes que tu nous as donnés en possession. Penses-tu crever les yeux de ces gens? Nous ne monterons pas. 
\verse Moïse fut très irrité, et il dit à l`Éternel: N`aie point égard à leur offrande. Je ne leur ai pas même pris un âne, et je n`ai fait de mal à aucun d`eux. 
\verse Moïse dit à Koré: Toi et toute ta troupe, trouvez-vous demain devant l`Éternel, toi et eux, avec Aaron. 
\verse Prenez chacun votre brasier, mettez-y du parfum, et présentez devant l`Éternel chacun votre brasier: il y aura deux cent cinquante brasiers; toi et Aaron, vous prendrez aussi chacun votre brasier. 
\verse Ils prirent chacun leur brasier, y mirent du feu et y posèrent du parfum, et ils se tinrent à l`entrée de la tente d`assignation, avec Moïse et Aaron. 
\verse Et Koré convoqua toute l`assemblée contre Moïse et Aaron, à l`entrée de la tente d`assignation. Alors la gloire de l`Éternel apparut à toute l`assemblée. 
\verse Et l`Éternel parla à Moïse et à Aaron, et dit: 
\verse Séparez-vous du milieu de cette assemblée, et je les consumerai en un seul instant. 
\verse Ils tombèrent sur leur visage, et dirent: O Dieu, Dieu des esprits de toute chair! un seul homme a péché, et tu t`irriterais contre toute l`assemblée? 
\verse L`Éternel parla à Moïse, et dit: 
\verse Parle à l`assemblée, et dis: Retirez-vous de toutes parts loin de la demeure de Koré, de Dathan et d`Abiram. 
\verse Moïse se leva, et alla vers Dathan et Abiram; et les anciens d`Israël le suivirent. 
\verse Il parla à l`assemblée, et dit: Éloignez-vous des tentes de ces méchants hommes, et ne touchez à rien de ce qui leur appartient, de peur que vous ne périssiez en même temps qu`ils seront punis pour tous leurs péchés. 
\verse Ils se retirèrent de toutes parts loin de la demeure de Koré, de Dathan et d`Abiram. Dathan et Abiram sortirent, et se tinrent à l`entrée de leurs tentes, avec leurs femmes, leurs fils et leurs petits-enfants. 
\verse Moïse dit: A ceci vous connaîtrez que l`Éternel m`a envoyé pour faire toutes ces choses, et que je n`agis pas de moi-même. 
\verse Si ces gens meurent comme tous les hommes meurent, s`ils subissent le sort commun à tous les hommes, ce n`est pas l`Éternel qui m`a envoyé; 
\verse mais si l`Éternel fait une chose inouïe, si la terre ouvre sa bouche pour les engloutir avec tout ce qui leur appartient, et qu`ils descendent vivants dans le séjour des morts, vous saurez alors que ces gens ont méprisé l`Éternel. 
\verse Comme il achevait de prononcer toutes ces paroles, la terre qui était sous eux se fendit. 
\verse La terre ouvrit sa bouche, et les engloutit, eux et leurs maisons, avec tous les gens de Koré et tous leurs biens. 
\verse Ils descendirent vivants dans le séjour des morts, eux et tout ce qui leur appartenait; la terre les recouvrit, et ils disparurent au milieu de l`assemblée. 
\verse Tout Israël, qui était autour d`eux, s`enfuit à leur cri; car ils disaient: Fuyons, de peur que la terre ne nous engloutisse! 
\verse Un feu sortit d`auprès de l`Éternel, et consuma les deux cent cinquante hommes qui offraient le parfum. 
\verse L`Éternel parla à Moïse, et dit: 
\verse Dis à Éléazar, fils du sacrificateur Aaron, de retirer de l`incendie les brasiers et d`en répandre au loin le feu, car ils sont sanctifiés. 
\verse Avec les brasiers de ces gens qui ont péché au péril de leur vie, que l`on fasse des lames étendues dont on couvrira l`autel. Puisqu`ils ont été présentés devant l`Éternel et qu`ils sont sanctifiés, ils serviront de souvenir aux enfants d`Israël. 
\verse Le sacrificateur Éléazar prit les brasiers d`airain qu`avaient présentés les victimes de l`incendie, et il en fit des lames pour couvrir l`autel. 
\verse C`est un souvenir pour les enfants d`Israël, afin qu`aucun étranger à la race d`Aaron ne s`approche pour offrir du parfum devant l`Éternel et ne soit comme Koré et comme sa troupe, selon ce que l`Éternel avait déclaré par Moïse. 
\verse Dès le lendemain, toute l`assemblée des enfants d`Israël murmura contre Moïse et Aaron, en disant: Vous avez fait mourir le peuple de l`Éternel. 
\verse Comme l`assemblée se formait contre Moïse et Aaron, et comme ils tournaient les regards vers la tente d`assignation, voici, la nuée la couvrit, et la gloire de l`Éternel apparut. 
\verse Moïse et Aaron arrivèrent devant la tente d`assignation. 
\verse Et l`Éternel parla à Moïse, et dit: 
\verse Retirez-vous du milieu de cette assemblée, et je les consumerai en un instant. Ils tombèrent sur leur visage; 
\verse et Moïse dit à Aaron: Prends le brasier, mets-y du feu de dessus l`autel, poses-y du parfum, va promptement vers l`assemblée, et fais pour eux l`expiation; car la colère de l`Éternel a éclaté, la plaie a commencé. 
\verse Aaron prit le brasier, comme Moïse avait dit, et courut au milieu de l`assemblée; et voici, la plaie avait commencé parmi le peuple. Il offrit le parfum, et il fit l`expiation pour le peuple. 
\verse Il se plaça entre les morts et les vivants, et la plaie fut arrêtée. 
\verse Il y eut quatorze mille sept cents personnes qui moururent de cette plaie, outre ceux qui étaient morts à cause de Koré. 
\verse Aaron retourna auprès de Moïse, à l`entrée de la tente d`assignation. La plaie était arrêtée. 

\chapter
\verse L`Éternel parla à Moïse, et dit: 
\verse Parle aux enfants d`Israël, et prend d`eux une verge selon les maisons de leurs pères, soit douze verges de la part de tous leurs princes selon les maisons de leurs pères. 
\verse Tu écriras le nom de chacun sur sa verge, et tu écriras le nom d`Aaron sur la verge de Lévi; car il y aura une verge pour chaque chef des maisons de leurs pères. 
\verse Tu les déposeras dans la tente d`assignation, devant le témoignage, où je me rencontre avec vous. 
\verse L`homme que je choisirai sera celui dont la verge fleurira, et je ferai cesser de devant moi les murmures que profèrent contre vous les enfants d`Israël. 
\verse Moïse parla aux enfants d`Israël; et tous leurs princes lui donnèrent une verge, chaque prince une verge, selon les maisons de leurs pères, soit douze verges; la verge d`Aaron était au milieu des leurs. 
\verse Moïse déposa les verges devant l`Éternel, dans la tente du témoignage. 
\verse Le lendemain, lorsque Moïse entra dans la tente du témoignage, voici, la verge d`Aaron, pour la maison de Lévi, avait fleuri, elle avait poussé des boutons, produit des fleurs, et mûri des amandes. 
\verse Moïse ôta de devant l`Éternel toutes les verges, et les porta à tous les enfants d`Israël, afin qu`ils les vissent et qu`ils prissent chacun leur verge. 
\verse L`Éternel dit à Moïse: Reporte la verge d`Aaron devant le témoignage, pour être conservée comme un signe pour les enfants de rébellion, afin que tu fasses cesser de devant moi leurs murmures et qu`ils ne meurent point. 
\verse Moïse fit ainsi; il se conforma à l`ordre que l`Éternel lui avait donné. 
\verse Les enfants d`Israël dirent à Moïse: Voici, nous expirons, nous périssons, nous périssons tous! 
\verse Quiconque s`approche du tabernacle de l`Éternel, meurt. Nous faudra-t-il tous expirer? 

\chapter
\verse L`Éternel dit à Aaron: Toi et tes fils, et la maison de ton père avec toi, vous porterez la peine des iniquités commises dans le sanctuaire; toi et tes fils avec toi, vous porterez la peine des iniquités commises dans l`exercice de votre sacerdoce. 
\verse Fais aussi approcher de toi tes frères, la tribu de Lévi, la tribu de ton père, afin qu`ils te soient attachés et qu`ils te servent, lorsque toi, et tes fils avec toi, vous serez devant la tente du témoignage. 
\verse Ils observeront ce que tu leur ordonneras et ce qui concerne toute la tente; mais ils ne s`approcheront ni des ustensiles du sanctuaire, ni de l`autel, de peur que vous ne mouriez, eux et vous. 
\verse Ils te seront attachés, et ils observeront ce qui concerne la tente d`assignation pour tout le service de la tente. Aucun étranger n`approchera de vous. 
\verse Vous observerez ce qui concerne le sanctuaire et l`autel, afin qu`il n`y ait plus de colère contre les enfants d`Israël. 
\verse Voici, j`ai pris vos frères les Lévites du milieu des enfants d`Israël: donnés à l`Éternel, ils vous sont remis en don pour faire le service de la tente d`assignation. 
\verse Toi, et tes fils avec toi, vous observerez les fonctions de votre sacerdoce pour tout ce qui concerne l`autel et pour ce qui est en dedans du voile: c`est le service que vous ferez. Je vous accorde en pur don l`exercice du sacerdoce. L`étranger qui approchera sera mis à mort. 
\verse L`Éternel dit à Aaron: Voici, de toutes les choses que consacrent les enfants d`Israël, je te donne celles qui me sont offertes par élévation; je te les donne, à toi et à tes fils, comme droit d`onction, par une loi perpétuelle. 
\verse Voici ce qui t`appartiendra parmi les choses très saintes qui ne sont pas consumées par le feu: toutes leurs offrandes, tous leurs dons, tous leurs sacrifices d`expiation, et tous les sacrifices de culpabilité qu`ils m`offriront; ces choses très saintes seront pour toi et pour tes fils. 
\verse Vous les mangerez dans un lieu très saint; tout mâle en mangera; vous les regarderez comme saintes. 
\verse Voici encore ce qui t`appartiendra: tous les dons que les enfants d`Israël présenteront par élévation et en les agitant de côté et d`autre, je te les donne à toi, à tes fils et à tes filles avec toi, par une loi perpétuelle. Quiconque sera pur dans ta maison en mangera. 
\verse Je te donne les prémices qu`ils offriront à l`Éternel: tout ce qu`il y aura de meilleur en huile, tout ce qu`il y aura de meilleur en moût et en blé. 
\verse Les premiers produits de leur terre, qu`ils apporteront à l`Éternel, seront pour toi. Quiconque sera pur dans ta maison en mangera. 
\verse Tout ce qui sera dévoué par interdit en Israël sera pour toi. 
\verse Tout premier-né de toute chair, qu`ils offriront à l`Éternel, tant des hommes que des animaux, sera pour toi. Seulement, tu feras racheter le premier-né de l`homme, et tu feras racheter le premier-né d`un animal impur. 
\verse Tu les feras racheter dès l`âge d`un mois, d`après ton estimation, au prix de cinq sicles d`argent, selon le sicle du sanctuaire, qui est de vingt guéras. 
\verse Mais tu ne feras point racheter le premier-né du boeuf, ni le premier-né de la brebis, ni le premier-né de la chèvre: ce sont des choses saintes. Tu répandras leur sang sur l`autel, et tu brûleras leur graisse: ce sera un sacrifice consumé par le feu, d`une agréable odeur à l`Éternel. 
\verse Leur chair sera pour toi, comme la poitrine qu`on agite de côté et d`autre et comme l`épaule droite. 
\verse Je te donne, à toi, à tes fils et à tes filles avec toi, par une loi perpétuelle, toutes les offrandes saintes que les enfants d`Israël présenteront à l`Éternel par élévation. C`est une alliance inviolable et à perpétuité devant l`Éternel, pour toi et pour ta postérité avec toi. 
\verse L`Éternel dit à Aaron: Tu ne posséderas rien dans leur pays, et il n`y aura point de part pour toi au milieu d`eux; c`est moi qui suis ta part et ta possession, au milieu des enfants d`Israël. 
\verse Je donne comme possession aux fils de Lévi toute dîme en Israël, pour le service qu`ils font, le service de la tente d`assignation. 
\verse Les enfants d`Israël n`approcheront plus de la tente d`assignation, de peur qu`ils ne se chargent d`un péché et qu`ils ne meurent. 
\verse Les Lévites feront le service de la tente d`assignation, et ils resteront chargés de leurs iniquités. Ils n`auront point de possession au milieu des enfants d`Israël: ce sera une loi perpétuelle parmi vos descendants. 
\verse Je donne comme possession aux Lévites les dîmes que les enfants d`Israël présenteront à l`Éternel par élévation; c`est pourquoi je dis à leur égard: Ils n`auront point de possession au milieu des enfants d`Israël. 
\verse L`Éternel parla à Moïse, et dit: 
\verse Tu parleras aux Lévites, et tu leur diras: Lorsque vous recevrez des enfants d`Israël la dîme que je vous donne de leur part comme votre possession, vous en prélèverez une offrande pour l`Éternel, une dîme de la dîme; 
\verse et votre offrande vous sera comptée comme le blé qu`on prélève de l`aire et comme le moût qu`on prélève de la cuve. 
\verse C`est ainsi que vous prélèverez une offrande pour l`Éternel sur toutes les dîmes que vous recevrez des enfants d`Israël, et vous donnerez au sacrificateur Aaron l`offrande que vous en aurez prélevée pour l`Éternel. 
\verse Sur tous les dons qui vous seront faits, vous prélèverez toutes les offrandes pour l`Éternel; sur tout ce qu`il y aura de meilleur, vous prélèverez la portion consacrée. 
\verse Tu leur diras: Quand vous en aurez prélevé le meilleur, la dîme sera comptée aux Lévites comme le revenu de l`aire et comme le revenu de la cuve. 
\verse Vous la mangerez en un lieu quelconque, vous et votre maison; car c`est votre salaire pour le service que vous faites dans la tente d`assignation. 
\verse Vous ne serez chargés pour cela d`aucun péché, quand vous en aurez prélevé le meilleur, vous ne profanerez point les offrandes saintes des enfants d`Israël, et vous ne mourrez point. 

\chapter
\verse L`Éternel parla à Moïse et à Aaron, et dit: 
\verse Voici ce qui est ordonné par la loi que l`Éternel a prescrite, en disant: Parle aux enfants d`Israël, et qu`ils t`amènent une vache rousse, sans tache, sans défaut corporel, et qui n`ait point porté le joug. 
\verse Vous la remettrez au sacrificateur Éléazar, qui la fera sortir du camp, et on l`égorgera devant lui. 
\verse Le sacrificateur Éléazar prendra du sang de la vache avec le doigt, et il en fera sept fois l`aspersion sur le devant de la tente d`assignation. 
\verse On brûlera la vache sous ses yeux; on brûlera sa peau, sa chair et son sang, avec ses excréments. 
\verse Le sacrificateur prendra du bois de cèdre, de l`hysope et du cramoisi, et il les jettera au milieu des flammes qui consumeront la vache. 
\verse Le sacrificateur lavera ses vêtements, et lavera son corps dans l`eau; puis il rentrera dans le camp, et sera impur jusqu`au soir. 
\verse Celui qui aura brûlé la vache lavera ses vêtements dans l`eau, et lavera son corps dans l`eau; et il sera impur jusqu`au soir. 
\verse Un homme pur recueillera la cendre de la vache, et la déposera hors du camp, dans un lieu pur; on la conservera pour l`assemblée des enfants d`Israël, afin d`en faire l`eau de purification. C`est une eau expiatoire. 
\verse Celui qui aura recueilli la cendre de la vache lavera ses vêtements, et sera impur jusqu`au soir. Ce sera une loi perpétuelle pour les enfants d`Israël et pour l`étranger en séjour au milieu d`eux. 
\verse Celui qui touchera un mort, un corps humain quelconque, sera impur pendant sept jours. 
\verse Il se purifiera avec cette eau le troisième jour et le septième jour, et il sera pur; mais, s`il ne se purifie pas le troisième jour et le septième jour, il ne sera pas pur. 
\verse Celui qui touchera un mort, le corps d`un homme qui sera mort, et qui ne se purifiera pas, souille le tabernacle de l`Éternel; celui-là sera retranché d`Israël. Comme l`eau de purification n`a pas été répandue sur lui, il est impur, et son impureté est encore sur lui. 
\verse Voici la loi. Lorsqu`un homme mourra dans une tente, quiconque entrera dans la tente, et quiconque se trouvera dans la tente, sera impur pendant sept jours. 
\verse Tout vase découvert, sur lequel il n`y aura point de couvercle attaché, sera impur. 
\verse Quiconque touchera, dans les champs, un homme tué par l`épée, ou un mort, ou des ossements humains, ou un sépulcre, sera impur pendant sept jours. 
\verse On prendra, pour celui qui est impur, de la cendre de la victime expiatoire qui a été brûlée, et on mettra dessus de l`eau vive dans un vase. 
\verse Un homme pur prendra de l`hysope, et la trempera dans l`eau; puis il en fera l`aspersion sur la tente, sur tous les ustensiles, sur les personnes qui sont là, sur celui qui a touché des ossements, ou un homme tué, ou un mort, ou un sépulcre. 
\verse Celui qui est pur fera l`aspersion sur celui qui est impur, le troisième jour et le septième jour, et il le purifiera le septième jour. Il lavera ses vêtements, et se lavera dans l`eau; et le soir, il sera pur. 
\verse Un homme qui sera impur, et qui ne se purifiera pas, sera retranché du milieu de l`assemblée, car il a souillé le sanctuaire de l`Éternel; comme l`eau de purification n`a pas été répandue sur lui, il est impur. 
\verse Ce sera pour eux une loi perpétuelle. Celui qui fera l`aspersion de l`eau de purification lavera ses vêtements, et celui qui touchera l`eau de purification sera impur jusqu`au soir. 
\verse Tout ce que touchera celui qui est impur sera souillé, et la personne qui le touchera sera impure jusqu`au soir. 

\chapter
\verse Toute l`assemblée des enfants d`Israël arriva dans le désert de Tsin le premier mois, et le peuple s`arrêta à Kadès. C`est là que mourut Marie, et qu`elle fut enterrée. 
\verse Il n`y avait point d`eau pour l`assemblée; et l`on se souleva contre Moïse et Aaron. 
\verse Le peuple chercha querelle à Moïse. Ils dirent: Que n`avons-nous expiré, quand nos frères expirèrent devant l`Éternel? 
\verse Pourquoi avez-vous fait venir l`assemblée de l`Éternel dans ce désert, pour que nous y mourions, nous et notre bétail? 
\verse Pourquoi nous avez-vous fait monter hors d`Égypte, pour nous amener dans ce méchant lieu? Ce n`est pas un lieu où l`on puisse semer, et il n`y a ni figuier, ni vigne, ni grenadier, ni d`eau à boire. 
\verse Moïse et Aaron s`éloignèrent de l`assemblée pour aller à l`entrée de la tente d`assignation. Ils tombèrent sur leur visage; et la gloire de l`Éternel leur apparut. 
\verse L`Éternel parla à Moïse, et dit: 
\verse Prends la verge, et convoque l`assemblée, toi et ton frère Aaron. Vous parlerez en leur présence au rocher, et il donnera ses eaux; tu feras sortir pour eux de l`eau du rocher, et tu abreuveras l`assemblée et leur bétail. 
\verse Moïse prit la verge qui était devant l`Éternel, comme l`Éternel le lui avait ordonné. 
\verse Moïse et Aaron convoquèrent l`assemblée en face du rocher. Et Moïse leur dit: Écoutez donc, rebelles! Est-ce de ce rocher que nous vous ferons sortir de l`eau? 
\verse Puis Moïse leva la main et frappa deux fois le rocher avec sa verge. Il sortit de l`eau en abondance. L`assemblée but, et le bétail aussi. 
\verse Alors l`Éternel dit à Moïse et à Aaron: Parce que vous n`avez pas cru en moi, pour me sanctifier aux yeux des enfants d`Israël, vous ne ferez point entrer cette assemblée dans le pays que je lui donne. 
\verse Ce sont les eaux de Meriba, où les enfants d`Israël contestèrent avec l`Éternel, qui fut sanctifié en eux. 
\verse De Kadès, Moïse envoya des messagers au roi d`Édom, pour lui dire: Ainsi parle ton frère Israël: Tu sais toutes les souffrances que nous avons éprouvées. 
\verse Nos pères descendirent en Égypte, et nous y demeurâmes longtemps. Mais les Égyptiens nous ont maltraités, nous et nos pères. 
\verse Nous avons crié à l`Éternel, et il a entendu notre voix. Il a envoyé un ange, et nous a fait sortir de l`Égypte. Et voici, nous sommes à Kadès, ville à l`extrémité de ton territoire. 
\verse Laisse-nous passer par ton pays; nous ne traverserons ni les champs, ni les vignes, et nous ne boirons pas l`eau des puits; nous suivrons la route royale, sans nous détourner à droite ou à gauche, jusqu`à ce que nous ayons franchi ton territoire. 
\verse Édom lui dit: Tu ne passeras point chez moi, sinon je sortirai à ta rencontre avec l`épée. 
\verse Les enfants d`Israël lui dirent: Nous monterons par la grande route; et, si nous buvons de ton eau, moi et mes troupeaux, j`en paierai le prix; je ne ferai que passer avec mes pieds, pas autre chose. 
\verse Il répondit: Tu ne passeras pas! Et Édom sortit à sa rencontre avec un peuple nombreux et à main forte. 
\verse Ainsi Édom refusa de donner passage à Israël par son territoire. Et Israël se détourna de lui. 
\verse Toute l`assemblée des enfants d`Israël partit de Kadès, et arriva à la montagne de Hor. 
\verse L`Éternel dit à Moïse et à Aaron, vers la montagne de Hor, sur la frontière du pays d`Édom: 
\verse Aaron va être recueilli auprès de son peuple; car il n`entrera point dans le pays que je donne aux enfants d`Israël, parce que vous avez été rebelles à mon ordre, aux eaux de Meriba. 
\verse Prends Aaron et son fils Éléazar, et fais-les monter sur la montagne de Hor. 
\verse Dépouille Aaron de ses vêtements, et fais-les revêtir à Éléazar, son fils. C`est là qu`Aaron sera recueilli et qu`il mourra. 
\verse Moïse fit ce que l`Éternel avait ordonné. Ils montèrent sur la montagne de Hor, aux yeux de toute l`assemblée. 
\verse Moïse dépouilla Aaron de ses vêtements, et les fit revêtir à Éléazar, son fils. Aaron mourut là, au sommet de la montagne. Moïse et Éléazar descendirent de la montagne. 
\verse Toute l`assemblée vit qu`Aaron avait expiré, et toute la maison d`Israël pleura Aaron pendant trente jours. 

\chapter
\verse Le roi d`Arad, Cananéen, qui habitait le midi, apprit qu`Israël venait par le chemin d`Atharim. Il combattit Israël, et emmena des prisonniers. 
\verse Alors Israël fit un voeu à l`Éternel, et dit: Si tu livres ce peuple entre mes mains, je dévouerai ses villes par interdit. 
\verse L`Éternel entendit la voix d`Israël, et livra les Cananéens. On les dévoua par interdit, eux et leurs villes; et l`on nomma ce lieu Horma. 
\verse Ils partirent de la montagne de Hor par le chemin de la mer Rouge, pour contourner le pays d`Édom. Le peuple s`impatienta en route, 
\verse et parla contre Dieu et contre Moïse: Pourquoi nous avez-vous fait monter hors d`Égypte, pour que nous mourions dans le désert? car il n`y a point de pain, et il n`y a point d`eau, et notre âme est dégoûtée de cette misérable nourriture. 
\verse Alors l`Éternel envoya contre le peuple des serpents brûlants; ils mordirent le peuple, et il mourut beaucoup de gens en Israël. 
\verse Le peuple vint à Moïse, et dit: Nous avons péché, car nous avons parlé contre l`Éternel et contre toi. Prie l`Éternel, afin qu`il éloigne de nous ces serpents. Moïse pria pour le peuple. 
\verse L`Éternel dit à Moïse: Fais-toi un serpent brûlant, et place-le sur une perche; quiconque aura été mordu, et le regardera, conservera la vie. 
\verse Moïse fit un serpent d`airain, et le plaça sur une perche; et quiconque avait été mordu par un serpent, et regardait le serpent d`airain, conservait la vie. 
\verse Les enfants d`Israël partirent, et ils campèrent à Oboth. 
\verse Ils partirent d`Oboth et ils campèrent à Ijjé Abarim, dans le désert qui est vis-à-vis de Moab, vers le soleil levant. 
\verse De là ils partirent, et ils campèrent dans la vallée de Zéred. 
\verse De là ils partirent, et ils campèrent de l`autre côté de l`Arnon, qui coule dans le désert en sortant du territoire des Amoréens; car l`Arnon est la frontière de Moab, entre Moab et les Amoréens. 
\verse C`est pourquoi il est dit dans le livre des Guerres de l`Éternel: ...Vaheb en Supha, et les torrents de l`Arnon, 
\verse et le cours des torrents, qui s`étend du côté d`Ar et touche à la frontière de Moab. 
\verse De là ils allèrent à Beer. C`est ce Beer, où l`Éternel dit à Moïse: Rassemble le peuple, et je leur donnerai de l`eau. 
\verse Alors Israël chanta ce cantique: Monte, puits! Chantez en son honneur! 
\verse Puits, que des princes ont creusé, Que les grands du peuple ont creusé, Avec le sceptre, avec leurs bâtons! 
\verse Du désert ils allèrent à Matthana; de Matthana, à Nahaliel; de Nahaliel, à Bamoth; 
\verse de Bamoth, à la vallée qui est dans le territoire de Moab, au sommet du Pisga, en regard du désert. 
\verse Israël envoya des messagers à Sihon, roi des Amoréens, pour lui dire: 
\verse Laisse-moi passer par ton pays; nous n`entrerons ni dans les champs, ni dans les vignes, et nous ne boirons pas l`eau des puits; nous suivrons la route royale, jusqu`à ce que nous ayons franchi ton territoire. 
\verse Sihon n`accorda point à Israël le passage sur son territoire; il rassembla tout son peuple, et sortit à la rencontre d`Israël, dans le désert; il vint à Jahats, et combattit Israël. 
\verse Israël le frappa du tranchant de l`épée et s`empara de son pays depuis l`Arnon jusqu`au Jabbok, jusqu`à la frontière des enfants d`Ammon; car la frontière des enfants d`Ammon était fortifiée. 
\verse Israël prit toutes les villes, et s`établit dans toutes les villes des Amoréens, à Hesbon et dans toutes les villes de son ressort. 
\verse Car Hesbon était la ville de Sihon, roi des Amoréens; il avait fait la guerre au précédent roi de Moab, et lui avait enlevé tout son pays jusqu`à l`Arnon. 
\verse C`est pourquoi les poètes disent: Venez à Hesbon! Que la ville de Sihon soit rebâtie et fortifiée! 
\verse Car il est sorti un feu de Hesbon, Une flamme de la ville de Sihon; Elle a dévoré Ar Moab, Les habitants des hauteurs de l`Arnon. 
\verse Malheur à toi, Moab! Tu es perdu, peuple de Kemosch! Il a fait de ses fils des fuyards, Et il a livré ses filles captives A Sihon, roi des Amoréens. 
\verse Nous avons lancé sur eux nos traits: De Hesbon à Dibon tout est détruit; Nous avons étendu nos ravages jusqu`à Nophach, Jusqu`à Médeba. 
\verse Israël s`établit dans le pays des Amoréens. 
\verse Moïse envoya reconnaître Jaezer; et ils prirent les villes de son ressort, et chassèrent les Amoréens qui y étaient. 
\verse Ils changèrent ensuite de direction, et montèrent par le chemin de Basan. Og, roi de Basan, sortit à leur rencontre, avec tout son peuple, pour les combattre à Édréi. 
\verse L`Éternel dit à Moïse: Ne le crains point; car je le livre entre tes mains, lui et tout son peuple, et son pays; tu le traiteras comme tu as traité Sihon, roi des Amoréens, qui habitait à Hesbon. 
\verse Et ils le battirent, lui et ses fils, et tout son peuple, sans en laisser échapper un seul, et ils s`emparèrent de son pays. 

\chapter
\verse Les enfants d`Israël partirent, et ils campèrent dans les plaines de Moab, au delà du Jourdain, vis-à-vis de Jéricho. 
\verse Balak, fils de Tsippor, vit tout ce qu`Israël avait fait aux Amoréens. 
\verse Et Moab fut très effrayé en face d`un peuple aussi nombreux, il fut saisi de terreur en face des enfants d`Israël. 
\verse Moab dit aux anciens de Madian: Cette multitude va dévorer tout ce qui nous entoure, comme le boeuf broute la verdure des champs. Balak, fils de Tsippor, était alors roi de Moab. 
\verse Il envoya des messagers auprès de Balaam, fils de Beor, à Pethor sur le fleuve, dans le pays des fils de son peuple, afin de l`appeler et de lui dire: Voici, un peuple est sorti d`Égypte, il couvre la surface de la terre, et il habite vis-à-vis de moi. 
\verse Viens, je te prie, maudis-moi ce peuple, car il est plus puissant que moi; peut-être ainsi pourrai-je le battre et le chasserai-je du pays, car je sais que celui que tu bénis est béni, et que celui que tu maudis est maudit. 
\verse Les anciens de Moab et les anciens de Madian partirent, ayant avec eux des présents pour le devin. Ils arrivèrent auprès de Balaam, et lui rapportèrent les paroles de Balak. 
\verse Balaam leur dit: Passez ici la nuit, et je vous donnerai réponse, d`après ce que l`Éternel me dira. Et les chefs de Moab restèrent chez Balaam. 
\verse Dieu vint à Balaam, et dit: Qui sont ces hommes que tu as chez toi? 
\verse Balaam répondit à Dieu: Balak, fils de Tsippor, roi de Moab, les a envoyés pour me dire: 
\verse Voici, un peuple est sorti d`Égypte, et il couvre la surface de la terre; viens donc, maudis-le; peut-être ainsi pourrai-je le combattre, et le chasserai-je. 
\verse Dieu dit à Balaam: Tu n`iras point avec eux; tu ne maudiras point ce peuple, car il est béni. 
\verse Balaam se leva le matin, et il dit aux chefs de Balak: Allez dans votre pays, car l`Éternel refuse de me laisser aller avec vous. 
\verse Et les princes de Moab se levèrent, retournèrent auprès de Balak, et dirent: Balaam a refusé de venir avec nous. 
\verse Balak envoya de nouveau des chefs en plus grand nombre et plus considérés que les précédents. 
\verse Ils arrivèrent auprès de Balaam, et lui dirent: Ainsi parle Balak, fils de Tsippor: Que l`on ne t`empêche donc pas de venir vers moi; 
\verse car je te rendrai beaucoup d`honneurs, et je ferai tout ce que tu me diras; viens, je te prie, maudis-moi ce peuple. 
\verse Balaam répondit et dit aux serviteurs de Balak: Quand Balak me donnerait sa maison pleine d`argent et d`or, je ne pourrais faire aucune chose, ni petite ni grande, contre l`ordre de l`Éternel, mon Dieu. 
\verse Maintenant, je vous prie, restez ici cette nuit, et je saurai ce que l`Éternel me dira encore. 
\verse Dieu vint à Balaam pendant la nuit, et lui dit: Puisque ces hommes sont venus pour t`appeler, lève-toi, va avec eux; mais tu feras ce que je te dirai. 
\verse Balaam se leva le matin, sella son ânesse, et partit avec les chefs de Moab. 
\verse La colère de Dieu s`enflamma, parce qu`il était parti; et l`ange de l`Éternel se plaça sur le chemin, pour lui résister. Balaam était monté sur son ânesse, et ses deux serviteurs étaient avec lui. 
\verse L`ânesse vit l`ange de l`Éternel qui se tenait sur le chemin, son épée nue dans la main; elle se détourna du chemin et alla dans les champs. Balaam frappa l`ânesse pour la ramener dans le chemin. 
\verse L`ange de l`Éternel se plaça dans un sentier entre les vignes; il y avait un mur de chaque côté. 
\verse L`ânesse vit l`ange de l`Éternel; elle se serra contre le mur, et pressa le pied de Balaam contre le mur. Balaam la frappa de nouveau. 
\verse L`ange de l`Éternel passa plus loin, et se plaça dans un lieu où il n`y avait point d`espace pour se détourner à droite ou à gauche. 
\verse L`ânesse vit l`ange de l`Éternel, et elle s`abattit sous Balaam. La colère de Balaam s`enflamma, et il frappa l`ânesse avec un bâton. 
\verse L`Éternel ouvrit la bouche de l`ânesse, et elle dit à Balaam: Que t`ai je fait, pour que tu m`aies frappée déjà trois fois? 
\verse Balaam répondit à l`ânesse: C`est parce que tu t`es moquée de moi; si j`avais une épée dans la main, je te tuerais à l`instant. 
\verse L`ânesse dit à Balaam: Ne suis-je pas ton ânesse, que tu as de tout temps montée jusqu`à ce jour? Ai-je l`habitude de te faire ainsi? Et il répondit: Non. 
\verse L`Éternel ouvrit les yeux de Balaam, et Balaam vit l`ange de l`Éternel qui se tenait sur le chemin, son épée nue dans la main; et il s`inclina, et se prosterna sur son visage. 
\verse L`ange de l`Éternel lui dit: Pourquoi as-tu frappé ton ânesse déjà trois fois? Voici, je suis sorti pour te résister, car c`est un chemin de perdition qui est devant moi. 
\verse L`ânesse m`a vu, et elle s`est détournée devant moi déjà trois fois; si elle ne fût pas détournée de moi, je t`aurais même tué, et je lui aurais laissé la vie. 
\verse Balaam dit à l`ange de l`Éternel: J`ai péché, car je ne savais pas que tu te fusses placé au-devant de moi sur le chemin; et maintenant, si tu me désapprouves, je m`en retournerai. 
\verse L`ange de l`Éternel dit à Balaam: Va avec ces hommes; mais tu ne feras que répéter les paroles que je te dirai. Et Balaam alla avec les chefs de Balak. 
\verse Balak apprit que Balaam arrivait, et il sortit à sa rencontre jusqu`à la ville de Moab qui est sur la limite de l`Arnon, à l`extrême frontière. 
\verse Balak dit à Balaam: N`ai-je pas envoyé auprès de toi pour t`appeler? Pourquoi n`es-tu pas venu vers moi? Ne puis-je donc pas te traiter avec honneur? 
\verse Balaam dit à Balak: Voici, je suis venu vers toi; maintenant, me sera-t-il permis de dire quoi que ce soit? Je dirai les paroles que Dieu mettra dans ma bouche. 
\verse Balaam alla avec Balak, et ils arrivèrent à Kirjath Hutsoth. 
\verse Balak sacrifia des boeufs et des brebis, et il en envoya à Balaam et aux chefs qui étaient avec lui. 
\verse Le matin, Balak prit Balaam, et le fit monter à Bamoth Baal, d`où Balaam vit une partie du peuple. 

\chapter
\verse Balaam dit à Balak: Bâtis-moi ici sept autels, et prépare-moi ici sept taureaux et sept béliers. 
\verse Balak fit ce que Balaam avait dit; et Balak et Balaam offrirent un taureau et un bélier sur chaque autel. 
\verse Balaam dit à Balak: Tiens-toi près de ton holocauste, et je m`éloignerai; peut-être que l`Éternel viendra à ma rencontre, et je te dirai ce qu`il me révélera. Et il alla sur un lieu élevé. 
\verse Dieu vint au-devant de Balaam, et Balaam lui dit: J`ai dressé sept autels, et j`ai offert un taureau et un bélier sur chaque autel. 
\verse L`Éternel mit des paroles dans la bouche de Balaam, et dit: Retourne vers Balak, et tu parleras ainsi. 
\verse Il retourna vers lui; et voici, Balak se tenait près de son holocauste, lui et tous les chefs de Moab. 
\verse Balaam prononça son oracle, et dit: Balak m`a fait descendre d`Aram, Le roi de Moab m`a fait descendre des montagnes de l`Orient. -Viens, maudis-moi Jacob! Viens, sois irrité contre Israël! 
\verse Comment maudirais-je celui que Dieu n`a point maudit? Comment serais-je irrité quand l`Éternel n`est point irrité? 
\verse Je le vois du sommet des rochers, Je le contemple du haut des collines: C`est un peuple qui a sa demeure à part, Et qui ne fait point partie des nations. 
\verse Qui peut compter la poussière de Jacob, Et dire le nombre du quart d`Israël? Que je meure de la mort des justes, Et que ma fin soit semblable à la leur! 
\verse Balak dit à Balaam: Que m`as-tu fait? Je t`ai pris pour maudire mon ennemi, et voici, tu le bénis! 
\verse Il répondit, et dit: N`aurai-je pas soin de dire ce que l`Éternel met dans ma bouche? 
\verse Balak lui dit: Viens donc avec moi dans un autre lieu, d`où tu le verras; tu n`en verras qu`une partie, tu n`en verras pas la totalité. Et de là maudis-le-moi. 
\verse Il le mena au champ de Tsophim, sur le sommet du Pisga; il bâtit sept autels, et offrit un taureau et un bélier sur chaque autel. 
\verse Balaam dit à Balak: Tiens-toi ici, près de ton holocauste, et j`irai à la rencontre de Dieu. 
\verse L`Éternel vint au-devant de Balaam; il mit des paroles dans sa bouche, et dit: Retourne vers Balak, et tu parleras ainsi. 
\verse Il retourna vers lui; et voici, Balak se tenait près de son holocauste, avec les chefs de Moab. Balak lui dit: Qu`est-ce que l`Éternel a dit? 
\verse Balaam prononça son oracle, et dit: Lève-toi, Balak, écoute! Prête-moi l`oreille, fils de Tsippor! 
\verse Dieu n`est point un homme pour mentir, Ni fils d`un homme pour se repentir. Ce qu`il a dit, ne le fera-t-il pas? Ce qu`il a déclaré, ne l`exécutera-t il pas? 
\verse Voici, j`ai reçu l`ordre de bénir: Il a béni, je ne le révoquerai point. 
\verse Il n`aperçoit point d`iniquité en Jacob, Il ne voit point d`injustice en Israël; L`Éternel, son Dieu, est avec lui, Il est son roi, l`objet de son allégresse. 
\verse Dieu les a fait sortir d`Égypte, Il est pour eux comme la vigueur du buffle. 
\verse L`enchantement ne peut rien contre Jacob, Ni la divination contre Israël; Au temps marqué, il sera dit à Jacob et à Israël: Quelle est l`oeuvre de Dieu. 
\verse C`est un peuple qui se lève comme une lionne, Et qui se dresse comme un lion; Il ne se couche point jusqu`à ce qu`il ait dévoré la proie, Et qu`il ait bu le sang des blessés. 
\verse Balak dit à Balaam: Ne le maudis pas, mais du moins ne le bénis pas. 
\verse Balaam répondit, et dit à Balak: Ne t`ai-je pas parlé ainsi: Je ferai tout ce que l`Éternel dira? 
\verse Balak dit à Balaam: Viens donc, je te mènerai dans un autre lieu; peut être Dieu trouvera-t-il bon que de là tu me maudisses ce peuple. 
\verse Balak mena Balaam sur le sommet du Peor, en regard du désert. 
\verse Balaam dit à Balak: Bâtis-moi ici sept autels, et prépare-moi ici sept taureaux et sept béliers. 
\verse Balak fit ce que Balaam avait dit, et il offrit un taureau et un bélier sur chaque autel. 

\chapter
\verse Balaam vit que l`Éternel trouvait bon de bénir Israël, et il n`alla point comme les autres fois, à la rencontre des enchantements; mais il tourna son visage du côté du désert. 
\verse Balaam leva les yeux, et vit Israël campé selon ses tribus. Alors l`esprit de Dieu fut sur lui. 
\verse Balaam prononça son oracle, et dit: Parole de Balaam, fils de Beor, Parole de l`homme qui a l`oeil ouvert, 
\verse Parole de celui qui entend les paroles de Dieu, De celui qui voit la vision du Tout Puissant, De celui qui se prosterne et dont les yeux s`ouvrent. 
\verse Qu`elles sont belles, tes tentes, ô Jacob! Tes demeures, ô Israël! 
\verse Elles s`étendent comme des vallées, Comme des jardins près d`un fleuve, Comme des aloès que l`Éternel a plantés, Comme des cèdres le long des eaux. 
\verse L`eau coule de ses seaux, Et sa semence est fécondée par d`abondantes eaux. Son roi s`élève au-dessus d`Agag, Et son royaume devient puissant. 
\verse Dieu l`a fait sortir d`Égypte, Il est pour lui comme la vigueur du buffle. Il dévore les nations qui s`élèvent contre lui, Il brise leurs os, et les abat de ses flèches. 
\verse Il ploie les genoux, il se couche comme un lion, comme une lionne: Qui le fera lever? Béni soit quiconque te bénira, Et maudit soit quiconque te maudira! 
\verse La colère de Balak s`enflamma contre Balaam; il frappa des mains, et dit à Balaam: C`est pour maudire mes ennemis que je t`ai appelé, et voici, tu les as bénis déjà trois fois. 
\verse Fuis maintenant, va-t`en chez toi! J`avais dit que je te rendrais des honneurs, mais l`Éternel t`empêche de les recevoir. 
\verse Balaam répondit à Balak: Eh! n`ai-je pas dit aux messagers que tu m`as envoyés: 
\verse Quand Balak me donnerait sa maison pleine d`argent et d`or, je ne pourrais faire de moi-même ni bien ni mal contre l`ordre de l`Éternel; je répéterai ce que dira l`Éternel? 
\verse Et maintenant voici, je m`en vais vers mon peuple. Viens, je t`annoncerai ce que ce peuple fera à ton peuple dans la suite des temps. 
\verse Balaam prononça son oracle, et dit: Parole de Balaam, fils de Beor, Parole de l`homme qui a l`oeil ouvert, 
\verse Parole de celui qui entend les paroles de Dieu, De celui qui connaît les desseins du Très Haut, De celui qui voit la vision du Tout Puissant, De celui qui se prosterne et dont les yeux s`ouvrent. 
\verse Je le vois, mais non maintenant, Je le contemple, mais non de près. Un astre sort de Jacob, Un sceptre s`élève d`Israël. Il perce les flancs de Moab, Et il abat tous les enfants de Seth. 
\verse Il se rend maître d`Édom, Il se rend maître de Séir, ses ennemis. Israël manifeste sa force. 
\verse Celui qui sort de Jacob règne en souverain, Il fait périr ceux qui s`échappent des villes. 
\verse Balaam vit Amalek. Il prononça son oracle, et dit: Amalek est la première des nations, Mais un jour il sera détruit. 
\verse Balaam vit les Kéniens. Il prononça son oracle, et dit: Ta demeure est solide, Et ton nid posé sur le roc. 
\verse Mais le Kénien sera chassé, Quand l`Assyrien t`emmènera captif. 
\verse Balaam prononça son oracle, et dit: Hélas! qui vivra après que Dieu l`aura établi? 
\verse Mais des navires viendront de Kittim, Ils humilieront l`Assyrien, ils humilieront l`Hébreu; Et lui aussi sera détruit. 
\verse Balaam se leva, partit, et retourna chez lui. Balak s`en alla aussi de son côté. 

\chapter
\verse Israël demeurait à Sittim; et le peuple commença à se livrer à la débauche avec les filles de Moab. 
\verse Elles invitèrent le peuple aux sacrifices de leurs dieux; et le peuple mangea, et se prosterna devant leurs dieux. 
\verse Israël s`attacha à Baal Peor, et la colère de l`Éternel s`enflamma contre Israël. 
\verse L`Éternel dit à Moïse: Assemble tous les chefs du peuple, et fais pendre les coupables devant l`Éternel en face du soleil, afin que la colère ardente de l`Éternel se détourne d`Israël. 
\verse Moïse dit aux juges d`Israël: Que chacun de vous tue ceux de ses gens qui se sont attachés à Baal Peor. 
\verse Et voici, un homme des enfants d`Israël vint et amena vers ses frères une Madianite, sous les yeux de Moïse et sous les yeux de toute l`assemblée des enfants d`Israël, tandis qu`ils pleuraient à l`entrée de la tente d`assignation. 
\verse A cette vue, Phinées, fils d`Éléazar, fils du sacrificateur Aaron, se leva du milieu de l`assemblée, et prit une lance, dans sa main. 
\verse Il suivit l`homme d`Israël dans sa tente, et il les perça tous les deux, l`homme d`Israël, puis la femme, par le bas-ventre. Et la plaie s`arrêta parmi les enfants d`Israël. 
\verse Il y en eut vingt-quatre mille qui moururent de la plaie. 
\verse L`Éternel parla à Moïse, et dit: 
\verse Phinées, fils d`Éléazar, fils du sacrificateur Aaron, a détourné ma fureur de dessus les enfants d`Israël, parce qu`il a été animé de mon zèle au milieu d`eux; et je n`ai point, dans ma colère, consumé les enfants d`Israël. 
\verse C`est pourquoi tu diras que je traite avec lui une alliance de paix. 
\verse Ce sera pour lui et pour sa postérité après lui l`alliance d`un sacerdoce perpétuel, parce qu`il a été zélé pour son Dieu, et qu`il a fait l`expiation pour les enfants d`Israël. 
\verse L`homme d`Israël, qui fut tué avec la Madianite, s`appelait Zimri, fils de Salu; il était chef d`une maison paternelle des Siméonites. 
\verse La femme qui fut tuée, la Madianite, s`appelait Cozbi, fille de Tsur, chef des peuplades issues d`une maison paternelle en Madian. 
\verse L`Éternel parla à Moïse, et dit: 
\verse Traite les Madianites en ennemis, et tuez-les; 
\verse car ils se sont montrés vos ennemis, en vous séduisant par leurs ruses, dans l`affaire de Peor, et dans l`affaire de Cozbi, fille d`un chef de Madian, leur soeur, tuée le jour de la plaie qui eut lieu à l`occasion de Peor. 

\chapter
\verse A la suite de cette plaie, l`Éternel dit à Moïse et à Éléazar, fils du sacrificateur Aaron: 
\verse Faites le dénombrement de toute l`assemblée des enfants d`Israël, depuis l`âge de vingt ans et au-dessus, selon les maisons de leurs pères, de tous ceux d`Israël en état de porter les armes. 
\verse Moïse et le sacrificateur Éléazar leur parlèrent dans les plaines de Moab, près du Jourdain, vis-à-vis de Jéricho. Ils dirent: 
\verse On fera le dénombrement, depuis l`âge de vingt ans et au-dessus, comme l`Éternel l`avait ordonné à Moïse et aux enfants d`Israël, quand ils furent sortis du pays d`Égypte. 
\verse Ruben, premier-né d`Israël. Fils de Ruben: Hénoc de qui descend la famille des Hénokites; Pallu, de qui descend la famille des Palluites; 
\verse Hetsron, de qui descend la famille des Hetsronites; Carmi, de qui descend la famille des Carmites. 
\verse Ce sont là les familles des Rubénites: ceux dont on fit le dénombrement furent quarante-trois mille sept cent trente. - 
\verse Fils de Pallu: Éliab. 
\verse Fils d`Éliab: Nemuel, Dathan et Abiram. C`est ce Dathan et cet Abiram, qui étaient de ceux que l`on convoquait à l`assemblée, et qui se soulevèrent contre Moïse et Aaron, dans l`assemblée de Koré, lors de leur révolte contre l`Éternel. 
\verse La terre ouvrit sa bouche, et les engloutit avec Koré, quand moururent ceux qui s`étaient assemblés, et que le feu consuma les deux cent cinquante hommes: ils servirent au peuple d`avertissement. 
\verse Les fils de Koré ne moururent pas. 
\verse Fils de Siméon, selon leurs familles: de Nemuel descend la famille des Nemuélites; de Jamin, la famille des Jaminites; de Jakin, la famille des Jakinites; 
\verse de Zérach, la famille des Zérachites; de Saül, la famille des Saülites. 
\verse Ce sont là les familles des Siméonites; vingt-deux mille deux cents. 
\verse Fils de Gad, selon leurs familles: de Tsephon descend la famille des Tsephonites; de Haggi, la famille des Haggites; de Schuni, la famille des Schunites; 
\verse d`Ozni, la famille des Oznites; d`Éri, la famille des Érites; 
\verse d`Arod, la famille des Arodites; d`Areéli, la famille des Areélites. 
\verse Ce sont là les familles des fils de Gad, d`après leur dénombrement: quarante mille cinq cents. 
\verse Fils de Juda: Er et Onan; mais Er et Onan moururent au pays de Canaan. 
\verse Voici les fils de Juda, selon leurs familles: de Schéla descend la famille des Schélanites; de Pérets, la famille des Péretsites; de Zérach, la famille des Zérachites. 
\verse Les fils de Pérets furent: Hetsron, de qui descend la famille des Hetsronites; Hamul, de qui descend la famille des Hamulites. 
\verse Ce sont là les familles de Juda, d`après leur dénombrement: soixante-seize mille cinq cents. 
\verse Fils d`Issacar, selon leurs familles: de Thola descend la famille des Tholaïtes; de Puva, la famille des Puvites; 
\verse de Jaschub, la famille des Jaschubites; de Schimron, la famille des Schimronites. 
\verse Ce sont là les familles d`Issacar, d`après leur dénombrement: soixante quatre mille trois cents. 
\verse Fils de Zabulon, selon leurs familles: de Séred descend la famille des Sardites; d`Élon, la famille des Élonites; de Jahleel, la famille des Jahleélites. 
\verse Ce sont là les familles des Zabulonites, d`après leur dénombrement: soixante mille cinq cents. 
\verse Fils de Joseph, selon leurs familles: Manassé et Éphraïm. 
\verse Fils de Manassé: de Makir descend la famille des Makirites. -Makir engendra Galaad. De Galaad descend la famille des Galaadites. 
\verse Voici les fils de Galaad: Jézer, de qui descend la famille des Jézerites; Hélek, la famille des Hélekites; 
\verse Asriel, la famille des Asriélites; Sichem, la famille des Sichémites; 
\verse Schemida, la famille des Schemidaïtes; Hépher, la famille des Héphrites. 
\verse Tselophchad, fils de Hépher, n`eut point de fils, mais il eut des filles. Voici les noms des filles de Tselophchad: Machla, Noa, Hogla, Milca et Thirsta. 
\verse Ce sont là les familles de Manassé, d`après leur dénombrement: cinquante-deux mille sept cents. 
\verse Voici les fils d`Éphraïm, selon leurs familles: de Schutélach descend la famille des Schutalchites; de Béker, la famille des Bakrites; de Thachan, la famille des Thachanites. - 
\verse Voici les fils de Schutélach: d`Éran est descendue la famille des Éranites. 
\verse Ce sont là les familles des fils d`Éphraïm, d`après leur dénombrement: trente-deux mille cinq cents. Ce sont là les fils de Joseph, selon leurs familles. 
\verse Fils de Benjamin, selon leurs familles: de Béla descend la famille des Balites; d`Aschbel, la famille des Aschbélites; d`Achiram, la famille des Achiramites; 
\verse de Schupham, la famille des Schuphamites; de Hupham, la famille des Huphamites. - 
\verse Les fils de Béla furent: Ard et Naaman. D`Ard descend la famille des Ardites; de Naaman, la famille des Naamanites. 
\verse Ce sont là les fils de Benjamin, selon leurs familles et d`après leur dénombrement; quarante-cinq mille six cents. 
\verse Voici les fils de Dan, selon leurs familles: de Schucham descend la famille des Schuchamites. Ce sont là les familles de Dan, selon leurs familles. 
\verse Total pour les familles des Schuchamites, d`après leur dénombrement: soixante-quatre mille quatre cents. 
\verse Fils d`Aser, selon leurs familles: de Jimna descend la famille des Jimnites; de Jischvi, la famille des Jischvites; de Beria, la famille des Beriites. 
\verse Des fils de Beria descendent: de Héber, la famille des Hébrites; de Malkiel, la famille des Malkiélites. 
\verse Le nom de la fille d`Aser était Sérach. 
\verse Ce sont là les familles des fils d`Aser, d`après leur dénombrement: cinquante-trois mille quatre cents. 
\verse Fils de Nephthali, selon leurs familles: de Jahtseel descend la famille des Jahtseélites; de Guni, la famille des Gunites; 
\verse de Jetser, la famille des Jitsrites; de Schillem, la famille des Schillémites. 
\verse Ce sont là les familles de Nephthali, selon leurs familles et d`après leur dénombrement: quarante-cinq mille quatre cents. 
\verse Tels sont ceux des enfants d`Israël dont on fit le dénombrement: six cent un mille sept cent trente. 
\verse L`Éternel parla à Moïse, et dit: 
\verse Le pays sera partagé entre eux, pour être leur propriété, selon le nombre des noms. 
\verse A ceux qui sont en plus grand nombre tu donneras une portion plus grande, et à ceux qui sont en plus petit nombre tu donneras une portion plus petite; on donnera à chacun sa portion d`après le dénombrement. 
\verse Mais le partage du pays aura lieu par le sort; ils le recevront en propriété selon les noms des tribus de leurs pères. 
\verse C`est par le sort que le pays sera partagé entre ceux qui sont en grand nombre et ceux qui sont en petit nombre. 
\verse Voici les Lévites dont on fit le dénombrement, selon leurs familles: de Guerschon descend la famille des Guerschonites; de Kehath, la famille des Kehathites; de Merari, la famille des Merarites. 
\verse Voici les familles de Lévi: la famille des Libnites, la famille des Hébronites, la famille des Machlites, la famille des Muschites, la famille des Korites. Kehath engendra Amram. 
\verse Le nom de la femme d`Amram était Jokébed, fille de Lévi, laquelle naquit à Lévi, en Égypte; elle enfanta à Amram: Aaron, Moïse, et Marie, leur soeur. 
\verse Il naquit à Aaron: Nadab et Abihu, Éléazar et Ithamar. 
\verse Nadab et Abihu moururent, lorsqu`ils apportèrent devant l`Éternel du feu étranger. 
\verse Ceux dont on fit le dénombrement, tous les mâles depuis l`âge d`un mois et au-dessus, furent vingt-trois mille. Ils ne furent pas compris dans le dénombrement des enfants d`Israël, parce qu`il ne leur fut point donné de possession au milieu des enfants d`Israël. 
\verse Tels sont ceux des enfants d`Israël dont Moïse et le sacrificateur Éléazar firent le dénombrement dans les plaines de Moab, près du Jourdain, vis-à-vis de Jéricho. 
\verse Parmi eux, il n`y avait aucun des enfants d`Israël dont Moïse et le sacrificateur Aaron avaient fait le dénombrement dans le désert de Sinaï. 
\verse Car l`Éternel avait dit: ils mourront dans le désert, et il n`en restera pas un, excepté Caleb, fils de Jephunné, et Josué, fils de Nun. 

\chapter
\verse Les filles de Tselophchad, fils de Hépher, fils de Galaad, fils de Makir, fils de Manassé, des familles de Manassé, fils de Joseph, et dont les noms étaient Machla, Noa, Hogla, Milca et Thirsta, 
\verse s`approchèrent et se présentèrent devant Moïse, devant le sacrificateur Éléazar, et devant les princes et toute l`assemblée, à l`entrée de la tente d`assignation. Elles dirent: 
\verse Notre père est mort dans le désert; il n`était pas au milieu de l`assemblée de ceux qui se révoltèrent contre l`Éternel, de l`assemblée de Koré, mais il est mort pour son péché, et il n`avait point de fils. 
\verse Pourquoi le nom de notre père serait-il retranché du milieu de sa famille, parce qu`il n`avait point eu de fils? Donne-nous une possession parmi les frères de notre père. 
\verse Moïse porta la cause devant l`Éternel. 
\verse Et l`Éternel dit à Moïse: 
\verse Les filles de Tselophchad ont raison. Tu leur donneras en héritage une possession parmi les frères de leur père, et c`est à elles que tu feras passer l`héritage de leur père. 
\verse Tu parleras aux enfants d`Israël, et tu diras: Lorsqu`un homme mourra sans laisser de fils, vous ferez passer son héritage à sa fille. 
\verse S`il n`a point de fille, vous donnerez son héritage à ses frères. 
\verse S`il n`a point de frères, vous donnerez son héritage aux frères de son père. 
\verse S`il n`y a point de frères de son père, vous donnerez son héritage au plus proche parent dans sa famille, et c`est lui qui le possédera. Ce sera pour les enfants d`Israël une loi et un droit, comme l`Éternel l`a ordonné à Moïse. 
\verse L`Éternel dit à Moïse: Monte sur cette montagne d`Abarim, et regarde le pays que je donne aux enfants d`Israël. 
\verse Tu le regarderas; mais toi aussi, tu sera recueilli auprès de ton peuple, comme Aaron, ton frère, a été recueilli; 
\verse parce que vous avez été rebelles à mon ordre, dans le désert de Tsin, lors de la contestation de l`assemblée, et que vous ne m`avez point sanctifié à leurs yeux à l`occasion des eaux. Ce sont les eaux de contestation, à Kadès, dans le désert de Tsin. 
\verse Moïse parla à l`Éternel, et dit: 
\verse Que l`Éternel, le Dieu des esprits de toute chair, établisse sur l`assemblée un homme 
\verse qui sorte devant eux et qui entre devant eux, qui les fasse sortir et qui les fasse entrer, afin que l`assemblée de l`Éternel ne soit pas comme des brebis qui n`ont point de berger. 
\verse L`Éternel dit à Moïse: Prends Josué, fils de Nun, homme en qui réside l`esprit; et tu poseras ta main sur lui. 
\verse Tu le placeras devant le sacrificateur Éléazar et devant toute l`assemblée, et tu lui donneras des ordres sous leurs yeux. 
\verse Tu le rendras participant de ta dignité, afin que toute l`assemblée des enfants d`Israël l`écoute. 
\verse Il se présentera devant le sacrificateur Éléazar, qui consultera pour lui le jugement de l`urim devant l`Éternel; et Josué, tous les enfants d`Israël avec lui, et toute l`assemblée, sortiront sur l`ordre d`Éléazar et entreront sur son ordre. 
\verse Moïse fit ce que l`Éternel lui avait ordonné. Il prit Josué, et il le plaça devant le sacrificateur Éléazar et devant toute l`assemblée. 
\verse Il posa ses mains sur lui, et lui donna des ordres, comme l`Éternel l`avait dit par Moïse. 

\chapter
\verse L`Éternel parla à Moïse, et dit: Donne cet ordre aux enfants d`Israël, et dis-leur: 
\verse Vous aurez soin de me présenter, au temps fixé, mon offrande, l`aliment de mes sacrifices consumés par le feu, et qui me sont d`une agréable odeur. 
\verse Tu leur diras: Voici le sacrifice consumé par le feu que vous offrirez à l`Éternel: chaque jour, deux agneaux d`un an sans défaut, comme holocauste perpétuel. 
\verse Tu offriras l`un des agneaux le matin, et l`autre agneau entre les deux soirs, 
\verse et, pour l`offrande, un dixième d`épha de fleur de farine pétrie dans un quart de hin d`huile d`olives concassées. 
\verse C`est l`holocauste perpétuel, qui a été offert à la montagne de Sinaï; c`est un sacrifice consumé par le feu, d`une agréable odeur à l`Éternel. 
\verse La libation sera d`un quart de hin pour chaque agneau: c`est dans le lieu saint que tu feras la libation de vin à l`Éternel. 
\verse Tu offriras le second agneau entre les deux soirs, avec une offrande et une libation semblables à celles du matin; c`est un sacrifice consumé par le feu, d`une agréable odeur à l`Éternel. 
\verse Le jour du sabbat, vous offrirez deux agneaux d`un an sans défaut, et, pour l`offrande, deux dixièmes de fleur de farine pétrie à l`huile, avec la libation. 
\verse C`est l`holocauste du sabbat, pour chaque sabbat, outre l`holocauste perpétuel et la libation. 
\verse Au commencement de vos mois, vous offrirez en holocauste à l`Éternel deux jeunes taureaux, un bélier, et sept agneaux d`un an sans défaut; 
\verse et, comme offrande pour chaque taureau, trois dixièmes de fleur de farine pétrie à l`huile; comme offrande pour le bélier, deux dixièmes de fleur de farine pétrie à l`huile; 
\verse comme offrande pour chaque agneau, un dixième de fleur de farine pétrie à l`huile. C`est un holocauste, un sacrifice consumé par le feu, d`une agréable odeur à l`Éternel. 
\verse Les libations seront d`un demi-hin de vin pour un taureau, d`un tiers de hin pour un bélier, et d`un quart de hin pour un agneau. C`est l`holocauste du commencement du mois, pour chaque mois, pour tous les mois de l`année. 
\verse On offrira à l`Éternel un bouc, en sacrifice d`expiation, outre l`holocauste perpétuel et la libation. 
\verse Le premier mois, le quatorzième jour du mois, ce sera la Pâque de l`Éternel. 
\verse Le quinzième jour de ce mois sera un jour de fête. On mangera pendant sept jours des pains sans levain. 
\verse Le premier jour, il y aura une sainte convocation: vous ne ferez aucune oeuvre servile. 
\verse Vous offrirez en holocauste à l`Éternel un sacrifice consumé par le feu: deux jeunes taureaux, un bélier, et sept agneaux d`un an sans défaut. 
\verse Vous y joindrez l`offrande de fleur de farine pétrie à l`huile, trois dixièmes pour un taureau, deux dixièmes pour un bélier, 
\verse et un dixième pour chacun des sept agneaux. 
\verse Vous offrirez un bouc en sacrifice d`expiation, afin de faire pour vous l`expiation. 
\verse Vous offrirez ces sacrifices, outre l`holocauste du matin, qui est un holocauste perpétuel. 
\verse Vous les offrirez chaque jour, pendant sept jours, comme l`aliment d`un sacrifice consumé par le feu, d`une agréable odeur à l`Éternel. On les offrira, outre l`holocauste perpétuel et la libation. 
\verse Le septième jour, vous aurez une sainte convocation: vous ne ferez aucune oeuvre servile. 
\verse Le jour des prémices, où vous présenterez à l`Éternel une offrande, à votre fête des semaines, vous aurez une sainte convocation: vous ne ferez aucune oeuvre servile. 
\verse Vous offrirez en holocauste, d`une agréable odeur à l`Éternel, deux jeunes taureaux, un bélier, et sept agneaux d`un an. 
\verse Vous y joindrez l`offrande de fleur de farine pétrie à l`huile, trois dixièmes pour chaque taureau, deux dixièmes pour le bélier, 
\verse et un dixième pour chacun des sept agneaux. 
\verse Vous offrirez un bouc, afin de faire pour vous l`expiation. 
\verse Vous offrirez ces sacrifices, outre l`holocauste perpétuel et l`offrande. Vous aurez des agneaux sans défaut, et vous joindrez les libations. 

\chapter
\verse Le septième mois, le premier jour du mois, vous aurez une sainte convocation: vous ne ferez aucune oeuvre servile. Ce jour sera publié parmi vous au son des trompettes. 
\verse Vous offrirez en holocauste, d`une agréable odeur à l`Éternel, un jeune taureau, un bélier, et sept agneaux d`un an sans défaut. 
\verse Vous y joindrez l`offrande de fleur de farine pétrie à l`huile, trois dixièmes pour le taureau, deux dixièmes pour le bélier, 
\verse et un dixième pour chacun des sept agneaux. 
\verse Vous offrirez un bouc en sacrifice d`expiation, afin de faire pour vous l`expiation. 
\verse Vous offrirez ces sacrifices, outre l`holocauste et l`offrande de chaque mois, l`holocauste perpétuel et l`offrande, et les libations qui s`y joignent, d`après les règles établies. Ce sont des sacrifices consumés par le feu, d`une agréable odeur à l`Éternel. 
\verse Le dixième jour de ce septième mois, vous aurez une sainte convocation, et vous humilierez vos âmes: vous ne ferez aucun ouvrage. 
\verse Vous offrirez en holocauste, d`une agréable odeur à l`Éternel, un jeune taureau, un bélier, et sept agneaux d`un an sans défaut. 
\verse Vous y joindrez l`offrande de fleur de farine pétrie à l`huile, trois dixièmes pour le taureau, 
\verse deux dixièmes pour le bélier, et un dixième pour chacun des sept agneaux. 
\verse Vous offrirez un bouc en sacrifice d`expiation, outre le sacrifice des expiations, l`holocauste perpétuel et l`offrande, et les libations ordinaires. 
\verse Le quinzième jour du septième mois, vous aurez une sainte convocation: vous ne ferez aucune oeuvre servile. Vous célébrerez une fête en l`honneur de l`Éternel, pendant sept jours. 
\verse Vous offrirez en holocauste un sacrifice consumé par le feu, d`une agréable odeur à l`Éternel: treize jeunes taureaux, deux béliers, et quatorze agneaux d`un an sans défaut. 
\verse Vous y joindrez l`offrande de fleur de farine pétrie à l`huile, trois dixièmes pour chacun des treize taureaux, deux dixièmes pour chacun des deux béliers, 
\verse et un dixième pour chacun des quatorze agneaux. 
\verse Vous offrirez un bouc en sacrifice d`expiation, outre l`holocauste perpétuel, l`offrande et la libation. - 
\verse Le second jour, vous offrirez douze jeunes taureaux, deux béliers, et quatorze agneaux d`un an sans défaut, 
\verse avec l`offrande et les libations pour les taureaux, les béliers et les agneaux, selon leur nombre, d`après les règles établies. 
\verse Vous offrirez un bouc en sacrifice d`expiation, outre l`holocauste perpétuel, l`offrande et la libation. - 
\verse Le troisième jour, vous offrirez onze taureaux, deux béliers, et quatorze agneaux d`un an sans défaut, 
\verse avec l`offrande et les libations pour les taureaux, les béliers et les agneaux, selon leur nombre, d`après les règles établies. 
\verse Vous offrirez un bouc en sacrifice d`expiation, outre l`holocauste perpétuel, l`offrande et la libation. - 
\verse Le quatrième jour, vous offrirez dix taureaux, deux béliers, et quatorze agneaux d`un an sans défaut, 
\verse avec l`offrande et les libations pour les taureaux, les béliers et les agneaux, selon leur nombre, d`après les règles établies. 
\verse Vous offrirez un bouc en sacrifice d`expiation, outre l`holocauste perpétuel, l`offrande et la libation. - 
\verse Le cinquième jour, vous offrirez neuf taureaux, deux béliers, et quatorze agneaux d`un an sans défaut, 
\verse avec l`offrande et les libations pour les taureaux, les béliers et les agneaux selon leur nombre, d`après les règles établies. 
\verse Vous offrirez un bouc en sacrifice d`expiation, outre l`holocauste perpétuel, l`offrande et la libation. - 
\verse Le sixième jour, vous offrirez huit taureaux, deux béliers et quatorze agneaux d`un an sans défaut, 
\verse avec l`offrande et les libations pour les taureaux, les béliers et les agneaux, selon leur nombre, d`après les règles établies. 
\verse Vous offrirez un bouc en sacrifice d`expiation, outre l`holocauste perpétuel, l`offrande et la libation. - 
\verse Le septième jour, vous offrirez sept taureaux, deux béliers, et quatorze agneaux d`un an sans défaut, 
\verse avec l`offrande et les libations pour les taureaux, les béliers et les agneaux, selon leur nombre, d`après les règles établies. 
\verse Vous offrirez un bouc en sacrifice d`expiation, outre l`holocauste perpétuel, l`offrande et la libation. - 
\verse Le huitième jour, vous aurez une assemblée solennelle: vous ne ferez aucune oeuvre servile. 
\verse Vous offrirez en holocauste un sacrifice consumé par le feu, d`une agréable odeur à l`Éternel: un taureau, un bélier, et sept agneaux d`un an sans défaut, 
\verse avec l`offrande et les libations pour le taureau, le bélier et les agneaux, selon leur nombre, d`après les règles établies. 
\verse Vous offrirez un bouc en sacrifice d`expiation, outre l`holocauste perpétuel, l`offrande et la libation. 
\verse Tels sont les sacrifices que vous offrirez à l`Éternel dans vos fêtes, outre vos holocaustes, vos offrandes et vos libations, et vos sacrifices de prospérité, en accomplissement d`un voeu ou en offrandes volontaires. 
\verse (30:1) Moïse dit aux enfants d`Israël tout ce que l`Éternel lui avait ordonné. 

\chapter
\verse (30:2) Moïse parla aux chefs des tribus des enfants d`Israël, et dit: Voici ce que l`Éternel ordonne. 
\verse (30:3) Lorsqu`un homme fera un voeu à l`Éternel, ou un serment pour se lier par un engagement, il ne violera point sa parole, il agira selon tout ce qui est sorti de sa bouche. 
\verse (30:4) Lorsqu`une femme, dans sa jeunesse et à la maison de son père, fera un voeu à l`Éternel et se liera par un engagement, 
\verse (30:5) et que son père aura connaissance du voeu qu`elle a fait et de l`engagement par lequel elle s`est liée, -si son père garde le silence envers elle, tout voeu qu`elle aura fait sera valable, et tout engagement par lequel elle se sera liée sera valable; 
\verse (30:6) mais si son père la désapprouve le jour où il en a connaissance, tous ses voeux et tous les engagements par lesquels elle se sera liée n`auront aucune valeur; et l`Éternel lui pardonnera, parce qu`elle a été désapprouvée de son père. 
\verse (30:7) Lorsqu`elle sera mariée, après avoir fait des voeux, ou s`être liée par une parole échappée de ses lèvres, 
\verse (30:8) et que son mari en aura connaissance, -s`il garde le silence envers elle le jour où il en a connaissance, ses voeux seront valables, et les engagements par lesquels elle se sera liée seront valables; 
\verse (30:9) mais si son mari la désapprouve le jour où il en a connaissance, il annulera le voeu qu`elle a fait et la parole échappée de ses lèvres, par laquelle elle s`est liée; et l`Éternel lui pardonnera. 
\verse (30:10) Le voeu d`une femme veuve ou répudiée, l`engagement quelconque par lequel elle se sera liée, sera valable pour elle. 
\verse (30:11) Lorsqu`une femme, dans la maison de son mari, fera des voeux ou se liera par un serment, 
\verse (30:12) et que son mari en aura connaissance, -s`il garde le silence envers elle et ne la désapprouve pas, tous ses voeux seront valables, et tous les engagements par lesquels elle se sera liée seront valables; 
\verse (30:13) mais si son mari les annule le jour où il en a connaissance, tout voeu et tout engagement sortis de ses lèvres n`auront aucune valeur, son mari les a annulés; et l`Éternel lui pardonnera. 
\verse (30:14) Son mari peut ratifier et son mari peut annuler tout voeu, tout serment par lequel elle s`engage à mortifier sa personne. 
\verse (30:15) S`il garde de jour en jour le silence envers elle, il ratifie ainsi tous les voeux ou tous les engagements par lesquels elle s`est liée; il les ratifie, parce qu`il a gardé le silence envers elle le jour où il en a eu connaissance. 
\verse (30:16) Mais s`il les annule après le jour où il en a eu connaissance, il sera coupable du péché de sa femme. 
\verse (30:17) Telles sont les lois que l`Éternel prescrivit à Moïse, entre un mari et sa femme, entre un père et sa fille, lorsqu`elle est dans sa jeunesse et à la maison de son père. 

\chapter
\verse L`Éternel parla à Moïse, et dit: 
\verse Venge les enfants d`Israël sur les Madianites; tu seras ensuite recueilli auprès de ton peuple. 
\verse Moïse parla au peuple, et dit: Équipez d`entre vous des hommes pour l`armée, et qu`ils marchent contre Madian, afin d`exécuter la vengeance de l`Éternel sur Madian. 
\verse Vous enverrez à l`armée mille hommes par tribu, de toutes les tribus d`Israël. 
\verse On leva d`entre les milliers d`Israël mille hommes par tribu, soit douze mille hommes équipés pour l`armée. 
\verse Moïse envoya à l`armée ces mille hommes par tribu, et avec eux le fils du sacrificateur Éléazar, Phinées, qui portait les instruments sacrés et les trompettes retentissantes. 
\verse Ils s`avancèrent contre Madian, selon l`ordre que l`Éternel avait donné à Moïse; et ils tuèrent tous les mâles. 
\verse Ils tuèrent les rois de Madian avec tous les autres, Évi, Rékem, Tsur, Hur et Réba, cinq rois de Madian; ils tuèrent aussi par l`épée Balaam, fils de Beor. 
\verse Les enfants d`Israël firent prisonnières les femmes des Madianites avec leurs petits enfants, et ils pillèrent tout leur bétail, tous leurs troupeaux et toutes leurs richesses. 
\verse Ils incendièrent toutes les villes qu`ils habitaient et tous leurs enclos. 
\verse Ils prirent toutes les dépouilles et tout le butin, personnes et bestiaux; 
\verse et ils amenèrent les captifs, le butin et les dépouilles, à Moïse, au sacrificateur Éléazar, et à l`assemblée des enfants d`Israël, campés dans les plaines de Moab, près du Jourdain, vis-à-vis de Jéricho. 
\verse Moïse, le sacrificateur Éléazar, et tous les princes de l`assemblée, sortirent au-devant d`eux, hors du camp. 
\verse Et Moïse s`irrita contre les commandants de l`armée, les chefs de milliers et les chefs de centaines, qui revenaient de l`expédition. 
\verse Il leur dit: Avez-vous laissé la vie à toutes les femmes? 
\verse Voici, ce sont elles qui, sur la parole de Balaam, ont entraîné les enfants d`Israël à l`infidélité envers l`Éternel, dans l`affaire de Peor; et alors éclata la plaie dans l`assemblée de l`Éternel. 
\verse Maintenant, tuez tout mâle parmi les petits enfants, et tuez toute femme qui a connu un homme en couchant avec lui; 
\verse mais laissez en vie pour vous toutes les filles qui n`ont point connu la couche d`un homme. 
\verse Et vous, campez pendant sept jours hors du camp; tous ceux d`entre vous qui ont tué quelqu`un, et tous ceux qui ont touché un mort, se purifieront le troisième et le septième jour, eux et vos prisonniers. 
\verse Vous purifierez aussi tout vêtement, tout objet de peau, tout ouvrage de poil de chèvre et tout ustensile de bois. 
\verse Le sacrificateur Éléazar dit aux soldats qui étaient allés à la guerre: Voici ce qui est ordonné par la loi que l`Éternel a prescrite à Moïse. 
\verse L`or, l`argent, l`airain, le fer, l`étain et le plomb, 
\verse tout objet qui peut aller au feu, vous le ferez passer par le feu pour le rendre pur. Mais c`est par l`eau de purification que sera purifié tout ce qui ne peut aller au feu; vous le ferez passer dans l`eau. 
\verse Vous laverez vos vêtements le septième jour, et vous serez purs; ensuite, vous pourrez entrer dans le camp. 
\verse L`Éternel dit à Moïse: 
\verse Fais, avec le sacrificateur Éléazar et les chefs de maison de l`assemblée, le compte du butin, de ce qui a été pris, personnes et bestiaux. 
\verse Partage le butin entre les combattants qui sont allés à l`armée et toute l`assemblée. 
\verse Tu prélèveras sur la portion des soldats qui sont allés à l`armée un tribut pour l`Éternel, savoir: un sur cinq cents, tant des personnes que des boeufs, des ânes et des brebis. 
\verse Vous le prendrez sur leur moitié, et tu le donneras au sacrificateur Éléazar comme une offrande à l`Éternel. 
\verse Et sur la moitié qui revient aux enfants d`Israël tu prendras un sur cinquante, tant des personnes que des boeufs, des ânes et des brebis, de tout animal; et tu le donneras aux Lévites, qui ont la garde du tabernacle de l`Éternel. 
\verse Moïse et le sacrificateur Éléazar firent ce que l`Éternel avait ordonné à Moïse. 
\verse Le butin, reste du pillage de ceux qui avaient fait partie de l`armée, était de six cent soixante-quinze mille brebis, 
\verse soixante-douze mille boeufs, 
\verse soixante et un mille ânes, 
\verse et trente-deux mille personnes ou femmes qui n`avaient point connu la couche d`un homme. - 
\verse La moitié, formant la part de ceux qui étaient allés à l`armée, fut de trois cent trente-sept mille cinq cents brebis, 
\verse dont six cent soixante-quinze pour le tribut à l`Éternel; 
\verse trente-six mille boeufs, dont soixante-douze pour le tribut à l`Éternel; 
\verse trente mille cinq cents ânes, dont soixante et un pour le tribut à l`Éternel; 
\verse et seize mille personnes, dont trente-deux pour le tribut à l`Éternel. 
\verse Moïse donna au sacrificateur Éléazar le tribut réservé comme offrande à l`Éternel, selon ce que l`Éternel lui avait ordonné. - 
\verse La moitié qui revenait aux enfants d`Israël, séparée par Moïse de celle des hommes de l`armée, 
\verse et formant la part de l`assemblée, fut de trois cent trente-sept mille cinq cents brebis, 
\verse trente-six mille boeufs, 
\verse trente mille cinq cents ânes, 
\verse et seize mille personnes. 
\verse Sur cette moitié qui revenait aux enfants d`Israël, Moïse prit un sur cinquante, tant des personnes que des animaux; et il le donna aux Lévites, qui ont la garde du tabernacle de l`Éternel, selon ce que l`Éternel lui avait ordonné. 
\verse Les commandants des milliers de l`armée, les chefs de milliers et les chefs de centaines, s`approchèrent de Moïse, 
\verse et lui dirent: Tes serviteurs ont fait le compte des soldats qui étaient sous nos ordres, et il ne manque pas un homme d`entre nous. 
\verse Nous apportons, comme offrande à l`Éternel, chacun les objets d`or que nous avons trouvés, chaînettes, bracelets, anneaux, pendants d`oreilles, et colliers, afin de faire pour nos personnes l`expiation devant l`Éternel. 
\verse Moïse et le sacrificateur Éléazar reçurent d`eux tous ces objets travaillés en or. 
\verse Tout l`or, que les chefs de milliers et les chefs de centaines présentèrent à l`Éternel en offrande par élévation, pesait seize mille sept cent cinquante sicles. 
\verse Les hommes de l`armée gardèrent chacun le butin qu`ils avaient fait. 
\verse Moïse et le sacrificateur Éléazar prirent l`or des chefs de milliers et des chefs de centaines, et l`apportèrent à la tente d`assignation, comme souvenir pour les enfants d`Israël devant l`Éternel. 

\chapter
\verse Les fils de Ruben et les fils de Gad avaient une quantité considérable de troupeaux, et ils virent que le pays de Jaezer et le pays de Galaad étaient un lieu propre pour des troupeaux. 
\verse Alors les fils de Gad et les fils de Ruben vinrent auprès de Moïse, du sacrificateur Éléazar et des princes de l`assemblée, et ils leur dirent: 
\verse Atharoth, Dibon, Jaezer, Nimra, Hesbon, Élealé, Sebam, Nebo et Beon, 
\verse ce pays que l`Éternel a frappé devant l`assemblée d`Israël, est un lieu propre pour des troupeaux, et tes serviteurs ont des troupeaux. 
\verse Ils ajoutèrent: Si nous avons trouvé grâce à tes yeux, que la possession de ce pays soit accordée à tes serviteurs, et ne nous fais point passer le Jourdain. 
\verse Moïse répondit aux fils de Gad et aux fils de Ruben: Vos frères iront-ils à la guerre, et vous, resterez-vous ici? 
\verse Pourquoi voulez-vous décourager les enfants d`Israël de passer dans le pays que l`Éternel leur donne? 
\verse Ainsi firent vos pères, quand je les envoyai de Kadès Barnéa pour examiner le pays. 
\verse Ils montèrent jusqu`à la vallée d`Eschcol, et, après avoir examiné le pays, ils découragèrent les enfants d`Israël d`aller dans le pays que l`Éternel leur donnait. 
\verse La colère de l`Éternel s`enflamma ce jour-là, et il jura en disant: 
\verse Ces hommes qui sont montés d`Égypte, depuis l`âge de vingt ans et au-dessus, ne verront point le pays que j`ai juré de donner à Abraham, à Isaac et à Jacob, car ils n`ont pas suivi pleinement ma voie, 
\verse excepté Caleb, fils de Jephunné, le Kenizien, et Josué, fils de Nun, qui ont pleinement suivi la voie de l`Éternel. 
\verse La colère de l`Éternel s`enflamma contre Israël, et il les fit errer dans le désert pendant quarante années, jusqu`à l`anéantissement de toute la génération qui avait fait le mal aux yeux de l`Éternel. 
\verse Et voici, vous prenez la place de vos pères comme des rejetons d`hommes pécheurs, pour rendre la colère de l`Éternel encore plus ardente contre Israël. 
\verse Car, si vous vous détournez de lui, il continuera de laisser Israël au désert, et vous causerez la perte de tout ce peuple. 
\verse Ils s`approchèrent de Moïse, et ils dirent: Nous construirons ici des parcs pour nos troupeaux et des villes pour nos petits enfants; 
\verse puis nous nous équiperons en hâte pour marcher devant les enfants d`Israël, jusqu`à ce que nous les ayons introduits dans le lieu qui leur est destiné; et nos petits enfants demeureront dans les villes fortes, à cause des habitants du pays. 
\verse Nous ne retournerons point dans nos maisons avant que les enfants d`Israël aient pris possession chacun de son héritage; 
\verse et nous ne posséderons rien avec eux de l`autre côté du Jourdain, ni plus loin, puisque nous aurons notre héritage de ce côté-ci du Jourdain, à l`orient. 
\verse Moïse leur dit: Si vous faites cela, si vous vous armez pour combattre devant l`Éternel, 
\verse si tous ceux de vous qui s`armeront passent le Jourdain devant l`Éternel jusqu`à ce qu`il ait chassé ses ennemis loin de sa face, 
\verse et si vous revenez seulement après que le pays aura été soumis devant l`Éternel, -vous serez alors sans reproche vis-à-vis de l`Éternel et vis-à-vis d`Israël, et cette contrée-ci sera votre propriété devant l`Éternel. 
\verse Mais si vous ne faites pas ainsi, vous péchez contre l`Éternel; sachez que votre péché vous atteindra. 
\verse Construisez des villes pour vos petits enfants et des parcs pour vos troupeaux, et faites ce que votre bouche a déclaré. 
\verse Les fils de Gad et les fils de Ruben dirent à Moïse: Tes serviteurs feront ce que mon seigneur ordonne. 
\verse Nos petits enfants, nos femmes, nos troupeaux et tout notre bétail, resteront dans les villes de Galaad; 
\verse et tes serviteurs, tous armés pour la guerre, iront combattre devant l`Éternel, comme dit mon seigneur. 
\verse Moïse donna des ordres à leur sujet au sacrificateur Éléazar, à Josué, fils de Nun, et aux chefs de famille dans les tribus des enfants d`Israël. 
\verse Il leur dit: Si les fils de Gad et les fils de Ruben passent avec vous le Jourdain, tous armés pour combattre devant l`Éternel, et que le pays soit soumis devant vous, vous leur donnerez en propriété la contrée de Galaad. 
\verse Mais s`ils ne marchent point en armes avec vous, qu`ils s`établissent au milieu de vous dans le pays de Canaan. 
\verse Les fils de Gad et les fils de Ruben répondirent: Nous ferons ce que l`Éternel a dit à tes serviteurs. 
\verse Nous passerons en armes devant l`Éternel au pays de Canaan; mais que nous possédions notre héritage de ce côté-ci du Jourdain. 
\verse Moïse donna aux fils de Gad et aux fils de Ruben, et à la moitié de la tribu de Manassé, fils de Joseph, le royaume de Sihon, roi des Amoréens, et le royaume d`Og, roi de Basan, le pays avec ses villes, avec les territoires des villes du pays tout alentour. 
\verse Les fils de Gad bâtirent Dibon, Atharoth, Aroër, 
\verse Athroth Schophan, Jaezer, Jogbeha, 
\verse Beth Nimra et Beth Haran, villes fortes, et ils firent des parcs pour les troupeaux. 
\verse Les fils de Ruben bâtirent Hesbon, Élealé et Kirjathaïm, 
\verse Nebo et Baal Meon, dont les noms furent changés, et Sibma, et ils donnèrent des noms aux villes qu`ils bâtirent. 
\verse Les fils de Makir, fils de Manassé, marchèrent contre Galaad, et s`en emparèrent; ils chassèrent les Amoréens qui y étaient. 
\verse Moïse donna Galaad à Makir, fils de Manassé, qui s`y établit. 
\verse Jaïr, fils de Manassé, se mit en marche, prit les bourgs, et les appela bourgs de Jaïr. 
\verse Nobach se mit en marche, prit Kenath avec les villes de son ressort, et l`appela Nobach, d`après son nom. 

\chapter
\verse Voici les stations des enfants d`Israël qui sortirent du pays d`Égypte, selon leurs corps d`armée, sous la conduite de Moïse et d`Aaron. 
\verse Moïse écrivit leurs marches de station en station, d`après l`ordre de l`Éternel. Et voici leurs stations, selon leurs marches. 
\verse Ils partirent de Ramsès le premier mois, le quinzième jour du premier mois. Le lendemain de la Pâque, les enfants d`Israël sortirent la main levée, à la vue de tous les Égyptiens. 
\verse Et les Égyptiens enterraient ceux que l`Éternel avait frappés parmi eux, tous les premiers-nés; l`Éternel exerçait aussi des jugements contre leurs dieux. 
\verse Les enfants d`Israël partirent de Ramsès, et campèrent à Succoth. 
\verse Ils partirent de Succoth, et campèrent à Étham, qui est à l`extrémité du désert. 
\verse Ils partirent d`Étham, se détournèrent vers Pi Hahiroth, vis-à-vis de Baal Tsephon, et campèrent devant Migdol. 
\verse Ils partirent de devant Pi Hahiroth, et passèrent au milieu de la mer dans la direction du désert; ils firent trois journées de marche dans le désert d`Étham, et campèrent à Mara. 
\verse Ils partirent de Mara, et arrivèrent à Élim; il y avait à Élim douze sources d`eau et soixante-dix palmiers: ce fut là qu`ils campèrent. 
\verse Ils partirent d`Élim, et campèrent près de la mer Rouge. 
\verse Ils partirent de la mer Rouge, et campèrent dans le désert de Sin. 
\verse Ils partirent du désert de Sin, et campèrent à Dophka. 
\verse Ils partirent de Dophka, et campèrent à Alusch. 
\verse Ils partirent d`Alusch, et campèrent à Rephidim, où le peuple ne trouva point d`eau à boire. 
\verse Ils partirent de Rephidim, et campèrent dans le désert de Sinaï. 
\verse Ils partirent de désert du Sinaï, et campèrent à Kibroth Hattaava. 
\verse Ils partirent de Kibroth Hattaava, et campèrent à Hatséroth. 
\verse Ils partirent de Hatséroth, et campèrent à Rithma. 
\verse Ils partirent de Rithma, et campèrent à Rimmon Pérets. 
\verse Ils partirent de Rimmon Pérets, et campèrent à Libna. 
\verse Ils partirent de Libna, et campèrent à Rissa. 
\verse Ils partirent de Rissa, et campèrent à Kehélatha. 
\verse Ils partirent de Kehélatha, et campèrent à la montagne de Schapher. 
\verse Ils partirent de la montagne de Schapher, et campèrent à Harada. 
\verse Ils partirent de Harada, et campèrent à Makhéloth. 
\verse Ils partirent de Makhéloth, et campèrent à Tahath. 
\verse Ils partirent de Tahath, et campèrent à Tarach. 
\verse Ils partirent de Tarach, et campèrent à Mithka. 
\verse Ils partirent de Mithka, et campèrent à Haschmona. 
\verse Ils partirent de Haschmona, et campèrent à Moséroth. 
\verse Ils partirent de Moséroth, et campèrent à Bené Jaakan. 
\verse Ils partirent de Bené Jaakan, et campèrent à Hor Guidgad. 
\verse Ils partirent de Hor Guidgad, et campèrent à Jothbatha. 
\verse Ils partirent de Jothbatha, et campèrent à Abrona. 
\verse Ils partirent d`Abrona, et campèrent à Etsjon Guéber. 
\verse Ils partirent d`Etsjon Guéber, et campèrent dans le désert de Tsin: c`est Kadès. 
\verse Ils partirent de Kadès, et campèrent à la montagne de Hor, à l`extrémité du pays d`Édom. 
\verse Le sacrificateur Aaron monta sur la montagne de Hor, suivant l`ordre de l`Éternel; et il y mourut, la quarantième année après la sortie des enfants d`Israël du pays d`Égypte, le cinquième mois, le premier jour du mois. 
\verse Aaron était âgé de cent vingt-trois ans lorsqu`il mourut sur la montagne de Hor. 
\verse Le roi d`Arad, Cananéen, qui habitait le midi du pays de Canaan, apprit l`arrivée des enfants d`Israël. 
\verse Ils partirent de la montagne de Hor, et campèrent à Tsalmona. 
\verse Ils partirent de Tsalmona, et campèrent à Punon. 
\verse Ils partirent de Punon, et campèrent à Oboth. 
\verse Ils partirent d`Oboth, et campèrent à Ijjé Abarim, sur la frontière de Moab. 
\verse Ils partirent d`Ijjé Abarim, et campèrent à Dibon Gad. 
\verse Ils partirent de Dibon Gad, et campèrent à Almon Diblathaïm. 
\verse Ils partirent d`Almon Diblathaïm, et campèrent aux montagnes d`Abarim, devant Nebo. 
\verse Ils partirent des montagnes d`Abarim, et campèrent dans les plaines de Moab, près du Jourdain, vis-à-vis de Jéricho. 
\verse Ils campèrent près du Jourdain, depuis Beth Jeschimoth jusqu`à Abel Sittim, dans les plaines de Moab. 
\verse L`Éternel parla à Moïse dans les plaines de Moab, près du Jourdain, vis-à-vis de Jéricho. Il dit: 
\verse Parle aux enfants d`Israël, et dis-leur: Lorsque vous aurez passé le Jourdain et que vous serez entrés dans le pays de Canaan, 
\verse vous chasserez devant vous tous les habitants du pays, vous détruirez toutes leurs idoles de pierre, vous détruirez toutes leurs images de fonte, et vous détruirez tous leurs hauts lieux. 
\verse Vous prendrez possession du pays, et vous vous y établirez; car je vous ai donné le pays, pour qu`il soit votre propriété. 
\verse Vous partagerez le pays par le sort, selon vos familles. A ceux qui sont en plus grand nombre vous donnerez une portion plus grande, et à ceux qui sont en plus petit nombre vous donnerez une portion plus petite. Chacun possédera ce qui lui sera échu par le sort: vous le recevrez en propriété, selon les tribus de vos pères. 
\verse Mais si vous ne chassez pas devant vous les habitants du pays, ceux d`entre eux que vous laisserez seront comme des épines dans vos yeux et des aiguillons dans vos côtés, ils seront vos ennemis dans le pays où vous allez vous établir. 
\verse Et il arrivera que je vous traiterai comme j`avais résolu de les traiter. 

\chapter
\verse L`Éternel parla à Moïse, et dit: 
\verse Donne cet ordre aux enfants d`Israël, et dis-leur: Quand vous serez entrés dans le pays de Canaan, ce pays deviendra votre héritage, le pays de Canaan, dont voici les limites. 
\verse Le côté du midi commencera au désert de Tsin près d`Édom. Ainsi, votre limite méridionale partira de l`extrémité de la mer Salée, vers l`orient; 
\verse elle tournera au sud de la montée d`Akrabbim, passera par Tsin, et s`étendra jusqu`au midi de Kadès Barnéa; elle continuera par Hatsar Addar, et passera vers Atsmon; 
\verse depuis Atsmon, elle tournera jusqu`au torrent d`Égypte, pour aboutir à la mer. 
\verse Votre limite occidentale sera la grande mer: ce sera votre limite à l`occident. 
\verse Voici quelle sera votre limite septentrionale: à partir de la grande mer, vous la tracerez jusqu`à la montagne de Hor; 
\verse depuis la montagne de Hor, vous la ferez passer par Hamath, et arriver à Tsedad; 
\verse elle continuera par Ziphron, pour aboutir à Hatsar Énan: ce sera votre limite au septentrion. 
\verse Vous tracerez votre limite orientale de Hatsar Énan à Schepham; 
\verse elle descendra de Schepham vers Ribla, à l`orient d`Aïn; elle descendra, et s`étendra le long de la mer de Kinnéreth, à l`orient; 
\verse elle descendra encore vers le Jourdain, pour aboutir à la mer Salée. Tel sera votre pays avec ses limites tout autour. 
\verse Moïse transmit cet ordre aux enfants d`Israël, et dit: C`est là le pays que vous partagerez par le sort, et que l`Éternel a résolu de donner aux neuf tribus et à la demi-tribu. 
\verse Car la tribu des fils de Ruben et la tribu des fils de Gad ont pris leur héritage, selon les maisons de leurs pères; la demi-tribu de Manassé a aussi pris son héritage. 
\verse Ces deux tribus et la demi-tribu ont pris leur héritage en deçà du Jourdain, vis-à-vis de Jéricho, du côté de l`orient. 
\verse L`Éternel parla à Moïse, et dit: 
\verse Voici les noms des hommes qui partageront entre vous le pays: le sacrificateur Éléazar, et Josué, fils de Nun. 
\verse Vous prendrez encore un prince de chaque tribu, pour faire le partage du pays. 
\verse Voici les noms de ces hommes. Pour la tribu de Juda: Caleb, fils de Jephunné; 
\verse pour la tribu des fils de Siméon: Samuel, fils d`Ammihud; 
\verse pour la tribu de Benjamin: Élidad, fils de Kislon; 
\verse pour la tribu des fils de Dan: le prince Buki, fils de Jogli; 
\verse pour les fils de Joseph, -pour la tribu des fils de Manassé: le prince Hanniel, fils d`Éphod; - 
\verse et pour la tribu des fils d`Éphraïm: le prince Kemuel, fils de Schiphtan; 
\verse pour la tribu des fils de Zabulon: le prince Élitsaphan, fils de Parnac; 
\verse pour la tribu des fils d`Issacar: le prince Paltiel, fils d`Azzan; 
\verse pour la tribu des fils d`Aser: le prince Ahihud, fils de Schelomi; 
\verse pour la tribu des fils de Nephthali: le prince Pedahel, fils d`Ammihud. 
\verse Tels sont ceux à qui l`Éternel ordonna de partager le pays de Canaan entre les enfants d`Israël. 

\chapter
\verse L`Éternel parla à Moïse, dans les plaines de Moab, près du Jourdain, vis-à-vis de Jéricho. Il dit: 
\verse Ordonne aux enfants d`Israël d`accorder aux Lévites, sur l`héritage qu`ils posséderont, des villes où ils puissent habiter. Vous donnerez aussi aux Lévites une banlieue autour de ces villes. 
\verse Ils auront les villes pour y habiter; et les banlieues seront pour leur bétail, pour leurs biens et pour tous leurs animaux. 
\verse Les banlieues des villes que vous donnerez aux Lévites auront, à partir du mur de la ville et au dehors, mille coudées tout autour. 
\verse Vous mesurerez, en dehors de la ville, deux mille coudées pour le côté oriental, deux mille coudées pour le côté méridional, deux mille coudées pour le côté occidental, et deux mille coudées pour le côté septentrional. La ville sera au milieu. Telles seront les banlieues de leurs villes. 
\verse Parmi les villes que vous donnerez aux Lévites, il y aura six villes de refuge où pourra s`enfuir le meurtrier, et quarante-deux autres villes. 
\verse Total des villes que vous donnerez aux Lévites: quarante-huit villes, avec leurs banlieues. 
\verse Les villes que vous donnerez sur les propriétés des enfants d`Israël seront livrées en plus grand nombre par ceux qui en ont le plus, et en plus petit nombre par ceux qui en ont moins; chacun donnera de ses villes aux Lévites à proportion de l`héritage qu`il possédera. 
\verse L`Éternel parla à Moïse, et dit: 
\verse Parle aux enfants d`Israël, et dis-leur: Lorsque vous aurez passé le Jourdain et que vous serez entrés dans le pays de Canaan, 
\verse vous vous établirez des villes qui soient pour vous des villes de refuge, où pourra s`enfuir le meurtrier qui aura tué quelqu`un involontairement. 
\verse Ces villes vous serviront de refuge contre le vengeur du sang, afin que le meurtrier ne soit point mis à mort avant d`avoir comparu devant l`assemblée pour être jugé. 
\verse Des villes que vous donnerez, six seront pour vous des villes de refuge. 
\verse Vous donnerez trois villes au delà du Jourdain, et vous donnerez trois villes dans le pays de Canaan: ce seront des villes de refuge. 
\verse Ces six villes serviront de refuge aux enfants d`Israël, à l`étranger et à celui qui demeure au milieu de vous: là pourra s`enfuir tout homme qui aura tué quelqu`un involontairement. 
\verse Si un homme frappe son prochain avec un instrument de fer, et que la mort en soit la suite, c`est un meurtrier: le meurtrier sera puni de mort. 
\verse S`il le frappe, tenant à la main une pierre qui puisse causer la mort, et que la mort en soit la suite, c`est un meurtrier: le meurtrier sera puni de mort. 
\verse S`il le frappe, tenant à la main un instrument de bois qui puisse causer la mort, et que la mort en soit la suite, c`est un meurtrier: le meurtrier sera puni de mort. 
\verse Le vengeur du sang fera mourir le meurtrier; quand il le rencontrera, il le tuera. 
\verse Si un homme pousse son prochain par un mouvement de haine, ou s`il jette quelque chose sur lui avec préméditation, et que la mort en soit la suite, 
\verse ou s`il le frappe de sa main par inimitié, et que la mort en soit la suite, celui qui a frappé sera puni de mort, c`est un meurtrier: le vengeur du sang tuera le meurtrier, quand il le rencontrera. 
\verse Mais si un homme pousse son prochain subitement et non par inimitié, ou s`il jette quelque chose sur lui sans préméditation, 
\verse ou s`il fait tomber sur lui par mégarde une pierre qui puisse causer la mort, et que la mort en soit la suite, sans qu`il ait de la haine contre lui et qu`il lui cherche du mal, 
\verse voici les lois d`après lesquelles l`assemblée jugera entre celui qui a frappé et le vengeur du sang. 
\verse L`assemblée délivrera le meurtrier de la main du vengeur du sang, et le fera retourner dans la ville de refuge où il s`était enfui. Il y demeurera jusqu`à la mort du souverain sacrificateur qu`on a oint de l`huile sainte. 
\verse Si le meurtrier sort du territoire de la ville de refuge où il s`est enfui, 
\verse et si le vengeur du sang le rencontre hors du territoire de la ville de refuge et qu`il tue le meurtrier, il ne sera point coupable de meurtre. 
\verse Car le meurtrier doit demeurer dans sa ville de refuge jusqu`à la mort du souverain sacrificateur; et après la mort du souverain sacrificateur, il pourra retourner dans sa propriété. 
\verse Voici des ordonnances de droit pour vous et pour vos descendants, dans tous les lieux où vous habiterez. 
\verse Si un homme tue quelqu`un, on ôtera la vie au meurtrier, sur la déposition de témoins. Un seul témoin ne suffira pas pour faire condamner une personne à mort. 
\verse Vous n`accepterez point de rançon pour la vie d`un meurtrier qui mérite la mort, car il sera puni de mort. 
\verse Vous n`accepterez point de rançon, qui lui permette de s`enfuir dans sa ville de refuge, et de retourner habiter dans le pays après la mort du sacrificateur. 
\verse Vous ne souillerez point le pays où vous serez, car le sang souille le pays; et il ne sera fait pour le pays aucune expiation du sang qui y sera répandu que par le sang de celui qui l`aura répandu. 
\verse Vous ne souillerez point le pays où vous allez demeurer, et au milieu duquel j`habiterai; car je suis l`Éternel, qui habite au milieu des enfants d`Israël. 

\chapter
\verse Les chefs de la famille de Galaad, fils de Makir, fils de Manassé, d`entre les familles des fils de Joseph, s`approchèrent et parlèrent devant Moïse et devant les princes, chefs de famille des enfants d`Israël. 
\verse Ils dirent: L`Éternel a ordonné à mon seigneur de donner le pays en héritage par le sort aux enfants d`Israël. Mon seigneur a aussi reçu de l`Éternel l`ordre de donner l`héritage de Tselophchad, notre frère, à ses filles. 
\verse Si elles se marient à l`un des fils d`une autre tribu des enfants d`Israël, leur héritage sera retranché de l`héritage de nos pères et ajouté à celui de la tribu à laquelle elles appartiendront; ainsi sera diminué l`héritage qui nous est échu par le sort. 
\verse Et quand viendra le jubilé pour les enfants d`Israël, leur héritage sera ajouté à celui de la tribu à laquelle elles appartiendront, et il sera retranché de celui de la tribu de nos pères. 
\verse Moïse transmit aux enfants d`Israël les ordres de l`Éternel. Il dit: La tribu des fils de Joseph a raison. 
\verse Voici ce que l`Éternel ordonne au sujet des filles de Tselophchad: elles se marieront à qui elles voudront, pourvu qu`elles se marient dans une famille de la tribu de leurs pères. 
\verse Aucun héritage parmi les enfants d`Israël ne passera d`une tribu à une autre tribu, mais les enfants d`Israël s`attacheront chacun à l`héritage de la tribu de ses pères. 
\verse Et toute fille, possédant un héritage dans les tribus des enfants d`Israël, se mariera à quelqu`un d`une famille de la tribu de son père, afin que les enfants d`Israël possèdent chacun l`héritage de leurs pères. 
\verse Aucun héritage ne passera d`une tribu à une autre tribu, mais les tribus des enfants d`Israël s`attacheront chacune à son héritage. 
\verse Les filles de Tselophchad se conformèrent à l`ordre que l`Éternel avait donné à Moïse. 
\verse Machla, Thirtsa, Hogla, Milca et Noa, filles de Tselophchad, se marièrent aux fils de leurs oncles; 
\verse elles se marièrent dans les familles des fils de Manassé, fils de Joseph, et leur héritage resta dans la tribu de la famille de leur père. 
\verse Tels sont les commandements et les lois que l`Éternel donna par Moïse aux enfants d`Israël, dans les plaines de Moab, près du Jourdain, vis-à-vis de Jéricho. 
