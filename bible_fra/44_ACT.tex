\book[Actes des Apôtres]{Actes}


\chapter
\verse Théophile, j`ai parlé, dans mon premier livre, de tout ce que Jésus a commencé de faire et d`enseigner dès le commencement 
\verse jusqu`au jour où il fut enlevé au ciel, après avoir donné ses ordres, par le Saint Esprit, aux apôtres qu`il avait choisis. 
\verse Après qu`il eut souffert, il leur apparut vivant, et leur en donna plusieurs preuves, se montrant à eux pendant quarante jours, et parlant des choses qui concernent le royaume de Dieu. 
\verse Comme il se trouvait avec eux, il leur recommanda de ne pas s`éloigner de Jérusalem, mais d`attendre ce que le Père avait promis, ce que je vous ai annoncé, leur dit-il; 
\verse car Jean a baptisé d`eau, mais vous, dans peu de jours, vous serez baptisés du Saint Esprit. 
\verse Alors les apôtres réunis lui demandèrent: Seigneur, est-ce en ce temps que tu rétabliras le royaume d`Israël? 
\verse Il leur répondit: Ce n`est pas à vous de connaître les temps ou les moments que le Père a fixés de sa propre autorité. 
\verse Mais vous recevrez une puissance, le Saint Esprit survenant sur vous, et vous serez mes témoins à Jérusalem, dans toute la Judée, dans la Samarie, et jusqu`aux extrémités de la terre. 
\verse Après avoir dit cela, il fut élevé pendant qu`ils le regardaient, et une nuée le déroba à leurs yeux. 
\verse Et comme ils avaient les regards fixés vers le ciel pendant qu`il s`en allait, voici, deux hommes vêtus de blanc leur apparurent, 
\verse et dirent: Hommes Galiléens, pourquoi vous arrêtez-vous à regarder au ciel? Ce Jésus, qui a été enlevé au ciel du milieu de vous, viendra de la même manière que vous l`avez vu allant au ciel. 
\verse Alors ils retournèrent à Jérusalem, de la montagne appelée des oliviers, qui est près de Jérusalem, à la distance d`un chemin de sabbat. 
\verse Quand ils furent arrivés, ils montèrent dans la chambre haute où ils se tenaient d`ordinaire; c`étaient Pierre, Jean, Jacques, André, Philippe, Thomas, Barthélemy, Matthieu, Jacques, fils d`Alphée, Simon le Zélote, et Jude, fils de Jacques. 
\verse Tous d`un commun accord persévéraient dans la prière, avec les femmes, et Marie, mère de Jésus, et avec les frères de Jésus. 
\verse En ces jours-là, Pierre se leva au milieu des frères, le nombre des personnes réunies étant d`environ cent vingt. Et il dit: 
\verse Hommes frères, il fallait que s`accomplît ce que le Saint Esprit, dans l`Écriture, a annoncé d`avance, par la bouche de David, au sujet de Judas, qui a été le guide de ceux qui ont saisi Jésus. 
\verse Il était compté parmi nous, et il avait part au même ministère. 
\verse Cet homme, ayant acquis un champ avec le salaire du crime, est tombé, s`est rompu par le milieu du corps, et toutes ses entrailles se sont répandues. 
\verse La chose a été si connue de tous les habitants de Jérusalem que ce champ a été appelé dans leur langue Hakeldama, c`est-à-dire, champ du sang. 
\verse Or, il est écrit dans le livre des Psaumes: Que sa demeure devienne déserte, Et que personne ne l`habite! Et: Qu`un autre prenne sa charge! 
\verse Il faut donc que, parmi ceux qui nous ont accompagnés tout le temps que le Seigneur Jésus a vécu avec nous, 
\verse depuis le baptême de Jean jusqu`au jour où il a été enlevé du milieu de nous, il y en ait un qui nous soit associé comme témoin de sa résurrection. 
\verse Ils en présentèrent deux: Joseph appelé Barsabbas, surnommé Justus, et Matthias. 
\verse Puis ils firent cette prière: Seigneur, toi qui connais les coeurs de tous, désigne lequel de ces deux tu as choisi, 
\verse afin qu`il ait part à ce ministère et à cet apostolat, que Judas a abandonné pour aller en son lieu. 
\verse Ils tirèrent au sort, et le sort tomba sur Matthias, qui fut associé aux onze apôtres. 

\chapter
\verse Le jour de la Pentecôte, ils étaient tous ensemble dans le même lieu. 
\verse Tout à coup il vint du ciel un bruit comme celui d`un vent impétueux, et il remplit toute la maison où ils étaient assis. 
\verse Des langues, semblables à des langues de feu, leur apparurent, séparées les unes des autres, et se posèrent sur chacun d`eux. 
\verse Et ils furent tous remplis du Saint Esprit, et se mirent à parler en d`autres langues, selon que l`Esprit leur donnait de s`exprimer. 
\verse Or, il y avait en séjour à Jérusalem des Juifs, hommes pieux, de toutes les nations qui sont sous le ciel. 
\verse Au bruit qui eut lieu, la multitude accourut, et elle fut confondue parce que chacun les entendait parler dans sa propre langue. 
\verse Ils étaient tous dans l`étonnement et la surprise, et ils se disaient les uns aux autres: Voici, ces gens qui parlent ne sont-ils pas tous Galiléens? 
\verse Et comment les entendons-nous dans notre propre langue à chacun, dans notre langue maternelle? 
\verse Parthes, Mèdes, Élamites, ceux qui habitent la Mésopotamie, la Judée, la Cappadoce, le Pont, l`Asie, 
\verse la Phrygie, la Pamphylie, l`Égypte, le territoire de la Libye voisine de Cyrène, et ceux qui sont venus de Rome, Juifs et prosélytes, 
\verse Crétois et Arabes, comment les entendons-nous parler dans nos langues des merveilles de Dieu? 
\verse Ils étaient tous dans l`étonnement, et, ne sachant que penser, ils se disaient les uns aux autres: Que veut dire ceci? 
\verse Mais d`autres se moquaient, et disaient: Ils sont pleins de vin doux. 
\verse Alors Pierre, se présentant avec les onze, éleva la voix, et leur parla en ces termes: Hommes Juifs, et vous tous qui séjournez à Jérusalem, sachez ceci, et prêtez l`oreille à mes paroles! 
\verse Ces gens ne sont pas ivres, comme vous le supposez, car c`est la troisième heure du jour. 
\verse Mais c`est ici ce qui a été dit par le prophète Joël: 
\verse Dans les derniers jours, dit Dieu, je répandrai de mon Esprit sur toute chair; Vos fils et vos filles prophétiseront, Vos jeunes gens auront des visions, Et vos vieillards auront des songes. 
\verse Oui, sur mes serviteurs et sur mes servantes, Dans ces jours-là, je répandrai de mon Esprit; et ils prophétiseront. 
\verse Je ferai paraître des prodiges en haut dans le ciel et des miracles en bas sur la terre, Du sang, du feu, et une vapeur de fumée; 
\verse Le soleil se changera en ténèbres, Et la lune en sang, Avant l`arrivée du jour du Seigneur, De ce jour grand et glorieux. 
\verse Alors quiconque invoquera le nom du Seigneur sera sauvé. 
\verse Hommes Israélites, écoutez ces paroles! Jésus de Nazareth, cet homme à qui Dieu a rendu témoignage devant vous par les miracles, les prodiges et les signes qu`il a opérés par lui au milieu de vous, comme vous le savez vous-mêmes; 
\verse cet homme, livré selon le dessein arrêté et selon la prescience de Dieu, vous l`avez crucifié, vous l`avez fait mourir par la main des impies. 
\verse Dieu l`a ressuscité, en le délivrant des liens de la mort, parce qu`il n`était pas possible qu`il fût retenu par elle. 
\verse Car David dit de lui: Je voyais constamment le Seigneur devant moi, Parce qu`il est à ma droite, afin que je ne sois point ébranlé. 
\verse Aussi mon coeur est dans la joie, et ma langue dans l`allégresse; Et même ma chair reposera avec espérance, 
\verse Car tu n`abandonneras pas mon âme dans le séjour des morts, Et tu ne permettras pas que ton Saint voie la corruption. 
\verse Tu m`as fait connaître les sentiers de la vie, Tu me rempliras de joie par ta présence. 
\verse Hommes frères, qu`il me soit permis de vous dire librement, au sujet du patriarche David, qu`il est mort, qu`il a été enseveli, et que son sépulcre existe encore aujourd`hui parmi nous. 
\verse Comme il était prophète, et qu`il savait que Dieu lui avait promis avec serment de faire asseoir un de ses descendants sur son trône, 
\verse c`est la résurrection du Christ qu`il a prévue et annoncée, en disant qu`il ne serait pas abandonné dans le séjour des morts et que sa chair ne verrait pas la corruption. 
\verse C`est ce Jésus que Dieu a ressuscité; nous en sommes tous témoins. 
\verse Élevé par la droite de Dieu, il a reçu du Père le Saint Esprit qui avait été promis, et il l`a répandu, comme vous le voyez et l`entendez. 
\verse Car David n`est point monté au ciel, mais il dit lui-même: Le Seigneur a dit à mon Seigneur: Assieds-toi à ma droite, 
\verse Jusqu`à ce que je fasse de tes ennemis ton marchepied. 
\verse Que toute la maison d`Israël sache donc avec certitude que Dieu a fait Seigneur et Christ ce Jésus que vous avez crucifié. 
\verse Après avoir entendu ce discours, ils eurent le coeur vivement touché, et ils dirent à Pierre et aux autres apôtres: Hommes frères, que ferons-nous? 
\verse Pierre leur dit: Repentez-vous, et que chacun de vous soit baptisé au nom de Jésus Christ, pour le pardon de vos péchés; et vous recevrez le don du Saint Esprit. 
\verse Car la promesse est pour vous, pour vos enfants, et pour tous ceux qui sont au loin, en aussi grand nombre que le Seigneur notre Dieu les appellera. 
\verse Et, par plusieurs autres paroles, il les conjurait et les exhortait, disant: Sauvez-vous de cette génération perverse. 
\verse Ceux qui acceptèrent sa parole furent baptisés; et, en ce jour-là, le nombre des disciples s`augmenta d`environ trois mille âmes. 
\verse Ils persévéraient dans l`enseignement des apôtres, dans la communion fraternelle, dans la fraction du pain, et dans les prières. 
\verse La crainte s`emparait de chacun, et il se faisait beaucoup de prodiges et de miracles par les apôtres. 
\verse Tous ceux qui croyaient étaient dans le même lieu, et ils avaient tout en commun. 
\verse Ils vendaient leurs propriétés et leurs biens, et ils en partageaient le produit entre tous, selon les besoins de chacun. 
\verse Ils étaient chaque jour tous ensemble assidus au temple, ils rompaient le pain dans les maisons, et prenaient leur nourriture avec joie et simplicité de coeur, 
\verse louant Dieu, et trouvant grâce auprès de tout le peuple. Et le Seigneur ajoutait chaque jour à l`Église ceux qui étaient sauvés. 

\chapter
\verse Pierre et Jean montaient ensemble au temple, à l`heure de la prière: c`était la neuvième heure. 
\verse Il y avait un homme boiteux de naissance, qu`on portait et qu`on plaçait tous les jours à la porte du temple appelée la Belle, pour qu`il demandât l`aumône à ceux qui entraient dans le temple. 
\verse Cet homme, voyant Pierre et Jean qui allaient y entrer, leur demanda l`aumône. 
\verse Pierre, de même que Jean, fixa les yeux sur lui, et dit: Regarde-nous. 
\verse Et il les regardait attentivement, s`attendant à recevoir d`eux quelque chose. 
\verse Alors Pierre lui dit: Je n`ai ni argent, ni or; mais ce que j`ai, je te le donne: au nom de Jésus Christ de Nazareth, lève-toi et marche. 
\verse Et le prenant par la main droite, il le fit lever. Au même instant, ses pieds et ses chevilles devinrent fermes; 
\verse d`un saut il fut debout, et il se mit à marcher. Il entra avec eux dans le temple, marchant, sautant, et louant Dieu. 
\verse Tout le monde le vit marchant et louant Dieu. 
\verse Ils reconnaissaient que c`était celui qui était assis à la Belle porte du temple pour demander l`aumône, et ils furent remplis d`étonnement et de surprise au sujet de ce qui lui était arrivé. 
\verse Comme il ne quittait pas Pierre et Jean, tout le peuple étonné accourut vers eux, au portique dit de Salomon. 
\verse Pierre, voyant cela, dit au peuple: Hommes Israélites, pourquoi vous étonnez-vous de cela? Pourquoi avez-vous les regards fixés sur nous, comme si c`était par notre propre puissance ou par notre piété que nous eussions fait marcher cet homme? 
\verse Le Dieu d`Abraham, d`Isaac et de Jacob, le Dieu de nos pères, a glorifié son serviteur Jésus, que vous avez livré et renié devant Pilate, qui était d`avis qu`on le relâchât. 
\verse Vous avez renié le Saint et le Juste, et vous avez demandé qu`on vous accordât la grâce d`un meurtrier. 
\verse Vous avez fait mourir le Prince de la vie, que Dieu a ressuscité des morts; nous en sommes témoins. 
\verse C`est par la foi en son nom que son nom a raffermi celui que vous voyez et connaissez; c`est la foi en lui qui a donné à cet homme cette entière guérison, en présence de vous tous. 
\verse Et maintenant, frères, je sais que vous avez agi par ignorance, ainsi que vos chefs. 
\verse Mais Dieu a accompli de la sorte ce qu`il avait annoncé d`avance par la bouche de tous ses prophètes, que son Christ devait souffrir. 
\verse Repentez-vous donc et convertissez-vous, pour que vos péchés soient effacés, 
\verse afin que des temps de rafraîchissement viennent de la part du Seigneur, et qu`il envoie celui qui vous a été destiné, Jésus Christ, 
\verse que le ciel doit recevoir jusqu`aux temps du rétablissement de toutes choses, dont Dieu a parlé anciennement par la bouche de ses saints prophètes. 
\verse Moïse a dit: Le Seigneur votre Dieu vous suscitera d`entre vos frères un prophète comme moi; vous l`écouterez dans tout ce qu`il vous dira, 
\verse et quiconque n`écoutera pas ce prophète sera exterminé du milieu du peuple. 
\verse Tous les prophètes qui ont successivement parlé, depuis Samuel, ont aussi annoncé ces jours-là. 
\verse Vous êtes les fils des prophètes et de l`alliance que Dieu a traitée avec nos pères, en disant à Abraham: Toutes les familles de la terre seront bénies en ta postérité. 
\verse C`est à vous premièrement que Dieu, ayant suscité son serviteur, l`a envoyé pour vous bénir, en détournant chacun de vous de ses iniquités. 

\chapter
\verse Tandis que Pierre et Jean parlaient au peuple, survinrent les sacrificateurs, le commandant du temple, et les sadducéens, 
\verse mécontents de ce qu`ils enseignaient le peuple, et annonçaient en la personne de Jésus la résurrection des morts. 
\verse Ils mirent les mains sur eux, et ils les jetèrent en prison jusqu`au lendemain; car c`était déjà le soir. 
\verse Cependant, beaucoup de ceux qui avaient entendu la parole crurent, et le nombre des hommes s`éleva à environ cinq mille. 
\verse Le lendemain, les chefs du peuple, les anciens et les scribes, s`assemblèrent à Jérusalem, 
\verse avec Anne, le souverain sacrificateur, Caïphe, Jean, Alexandre, et tous ceux qui étaient de la race des principaux sacrificateurs. 
\verse Ils firent placer au milieu d`eux Pierre et Jean, et leur demandèrent: Par quel pouvoir, ou au nom de qui avez-vous fait cela? 
\verse Alors Pierre, rempli du Saint Esprit, leur dit: Chefs du peuple, et anciens d`Israël, 
\verse puisque nous sommes interrogés aujourd`hui sur un bienfait accordé à un homme malade, afin que nous disions comment il a été guéri, 
\verse sachez-le tous, et que tout le peuple d`Israël le sache! C`est par le nom de Jésus Christ de Nazareth, que vous avez été crucifié, et que Dieu a ressuscité des morts, c`est par lui que cet homme se présente en pleine santé devant vous. 
\verse Jésus est La pierre rejetée par vous qui bâtissez, Et qui est devenue la principale de l`angle. 
\verse Il n`y a de salut en aucun autre; car il n`y a sous le ciel aucun autre nom qui ait été donné parmi les hommes, par lequel nous devions être sauvés. 
\verse Lorsqu`ils virent l`assurance de Pierre et de Jean, ils furent étonnés, sachant que c`étaient des hommes du peuple sans instruction; et ils les reconnurent pour avoir été avec Jésus. 
\verse Mais comme ils voyaient là près d`eux l`homme qui avait été guéri, ils n`avaient rien à répliquer. 
\verse Ils leur ordonnèrent de sortir du sanhédrin, et ils délibérèrent entre eux, disant: Que ferons-nous à ces hommes? 
\verse Car il est manifeste pour tous les habitants de Jérusalem qu`un miracle signalé a été accompli par eux, et nous ne pouvons pas le nier. 
\verse Mais, afin que la chose ne se répande pas davantage parmi le peuple, défendons-leur avec menaces de parler désormais à qui que ce soit en ce nom-là. 
\verse Et les ayant appelés, ils leur défendirent absolument de parler et d`enseigner au nom de Jésus. 
\verse Pierre et Jean leur répondirent: Jugez s`il est juste, devant Dieu, de vous obéir plutôt qu`à Dieu; 
\verse car nous ne pouvons pas ne pas parler de ce que nous avons vu et entendu. 
\verse Ils leur firent de nouvelles menaces, et les relâchèrent, ne sachant comment les punir, à cause du peuple, parce que tous glorifiaient Dieu de ce qui était arrivé. 
\verse Car l`homme qui avait été l`objet de cette guérison miraculeuse était âgé de plus de quarante ans. 
\verse Après avoir été relâchés, ils allèrent vers les leurs, et racontèrent tout ce que les principaux sacrificateurs et les anciens leur avaient dit. 
\verse Lorsqu`ils l`eurent entendu, ils élevèrent à Dieu la voix tous ensemble, et dirent: Seigneur, toi qui as fait le ciel, la terre, la mer, et tout ce qui s`y trouve, 
\verse c`est toi qui as dit par le Saint Esprit, par la bouche de notre père, ton serviteur David: Pourquoi ce tumulte parmi les nations, Et ces vaines pensées parmi les peuples? 
\verse Les rois de la terre se sont soulevés, Et les princes se sont ligués Contre le Seigneur et contre son Oint. 
\verse En effet, contre ton saint serviteur Jésus, que tu as oint, Hérode et Ponce Pilate se sont ligués dans cette ville avec les nations et avec les peuples d`Israël, 
\verse pour faire tout ce que ta main et ton conseil avaient arrêté d`avance. 
\verse Et maintenant, Seigneur, vois leurs menaces, et donne à tes serviteurs d`annoncer ta parole avec une pleine assurance, 
\verse en étendant ta main, pour qu`il se fasse des guérisons, des miracles et des prodiges, par le nom de ton saint serviteur Jésus. 
\verse Quand ils eurent prié, le lieu où ils étaient assemblés trembla; ils furent tous remplis du Saint Esprit, et ils annonçaient la parole de Dieu avec assurance. 
\verse La multitude de ceux qui avaient cru n`était qu`un coeur et qu`une âme. Nul ne disait que ses biens lui appartinssent en propre, mais tout était commun entre eux. 
\verse Les apôtres rendaient avec beaucoup de force témoignage de la résurrection du Seigneur Jésus. Et une grande grâce reposait sur eux tous. 
\verse Car il n`y avait parmi eux aucun indigent: tous ceux qui possédaient des champs ou des maisons les vendaient, apportaient le prix de ce qu`ils avaient vendu, 
\verse et le déposaient aux pieds des apôtres; et l`on faisait des distributions à chacun selon qu`il en avait besoin. 
\verse Joseph, surnommé par les apôtres Barnabas, ce qui signifie fils d`exhortation, Lévite, originaire de Chypre, 
\verse vendit un champ qu`il possédait, apporta l`argent, et le déposa aux pieds des apôtres. 

\chapter
\verse Mais un homme nommé Ananias, avec Saphira sa femme, vendit une propriété, 
\verse et retint une partie du prix, sa femme le sachant; puis il apporta le reste, et le déposa aux pieds des apôtres. 
\verse Pierre lui dit: Ananias, pourquoi Satan a-t-il rempli ton coeur, au point que tu mentes au Saint Esprit, et que tu aies retenu une partie du prix du champ? 
\verse S`il n`eût pas été vendu, ne te restait-il pas? Et, après qu`il a été vendu, le prix n`était-il pas à ta disposition? Comment as-tu pu mettre en ton coeur un pareil dessein? Ce n`est pas à des hommes que tu as menti, mais à Dieu. 
\verse Ananias, entendant ces paroles, tomba, et expira. Une grande crainte saisit tous les auditeurs. 
\verse Les jeunes gens, s`étant levés, l`enveloppèrent, l`emportèrent, et l`ensevelirent. 
\verse Environ trois heures plus tard, sa femme entra, sans savoir ce qui était arrivé. 
\verse Pierre lui adressa la parole: Dis-moi, est-ce à un tel prix que vous avez vendu le champ? Oui, répondit-elle, c`est à ce prix-là. 
\verse Alors Pierre lui dit: Comment vous êtes-vous accordés pour tenter l`Esprit du Seigneur? Voici, ceux qui ont enseveli ton mari sont à la porte, et ils t`emporteront. 
\verse Au même instant, elle tomba aux pieds de l`apôtre, et expira. Les jeunes gens, étant entrés, la trouvèrent morte; ils l`emportèrent, et l`ensevelirent auprès de son mari. 
\verse Une grande crainte s`empara de toute l`assemblée et de tous ceux qui apprirent ces choses. 
\verse Beaucoup de miracles et de prodiges se faisaient au milieu du peuple par les mains des apôtres. Ils se tenaient tous ensemble au portique de Salomon, 
\verse et aucun des autres n`osait se joindre à eux; mais le peuple les louait hautement. 
\verse Le nombre de ceux qui croyaient au Seigneur, hommes et femmes, s`augmentait de plus en plus; 
\verse en sorte qu`on apportait les malades dans les rues et qu`on les plaçait sur des lits et des couchettes, afin que, lorsque Pierre passerait, son ombre au moins couvrît quelqu`un d`eux. 
\verse La multitude accourait aussi des villes voisines à Jérusalem, amenant des malades et des gens tourmentés par des esprits impurs; et tous étaient guéris. 
\verse Cependant le souverain sacrificateur et tous ceux qui étaient avec lui, savoir le parti des sadducéens, se levèrent, remplis de jalousie, 
\verse mirent les mains sur les apôtres, et les jetèrent dans la prison publique. 
\verse Mais un ange du Seigneur, ayant ouvert pendant la nuit les portes de la prison, les fit sortir, et leur dit: 
\verse Allez, tenez-vous dans le temple, et annoncez au peuple toutes les paroles de cette vie. 
\verse Ayant entendu cela, ils entrèrent dès le matin dans le temple, et se mirent à enseigner. Le souverain sacrificateur et ceux qui étaient avec lui étant survenus, ils convoquèrent le sanhédrin et tous les anciens des fils d`Israël, et ils envoyèrent chercher les apôtres à la prison. 
\verse Les huissiers, à leur arrivée, ne les trouvèrent point dans la prison. Ils s`en retournèrent, et firent leur rapport, 
\verse en disant: Nous avons trouvé la prison soigneusement fermée, et les gardes qui étaient devant les portes; mais, après avoir ouvert, nous n`avons trouvé personne dedans. 
\verse Lorsqu`ils eurent entendu ces paroles, le commandant du temple et les principaux sacrificateurs ne savaient que penser des apôtres et des suites de cette affaire. 
\verse Quelqu`un vint leur dire: Voici, les hommes que vous avez mis en prison sont dans le temple, et ils enseignent le peuple. 
\verse Alors le commandant partit avec les huissiers, et les conduisit sans violence, car ils avaient peur d`être lapidés par le peuple. 
\verse Après qu`ils les eurent amenés en présence du sanhédrin, le souverain sacrificateur les interrogea en ces termes: 
\verse Ne vous avons-nous pas défendu expressément d`enseigner en ce nom-là? Et voici, vous avez rempli Jérusalem de votre enseignement, et vous voulez faire retomber sur nous le sang de cet homme! 
\verse Pierre et les apôtres répondirent: Il faut obéir à Dieu plutôt qu`aux hommes. 
\verse Le Dieu de nos pères a ressuscité Jésus, que vous avez tué, en le pendant au bois. 
\verse Dieu l`a élevé par sa droite comme Prince et Sauveur, pour donner à Israël la repentance et le pardon des péchés. 
\verse Nous sommes témoins de ces choses, de même que le Saint Esprit, que Dieu a donné à ceux qui lui obéissent. 
\verse Furieux de ces paroles, ils voulaient les faire mourir. 
\verse Mais un pharisien, nommé Gamaliel, docteur de la loi, estimé de tout le peuple, se leva dans le sanhédrin, et ordonna de faire sortir un instant les apôtres. 
\verse Puis il leur dit: Hommes Israélites, prenez garde à ce que vous allez faire à l`égard de ces gens. 
\verse Car, il n`y a pas longtemps que parut Theudas, qui se donnait pour quelque chose, et auquel se rallièrent environ quatre cents hommes: il fut tué, et tous ceux qui l`avaient suivi furent mis en déroute et réduits à rien. 
\verse Après lui, parut Judas le Galiléen, à l`époque du recensement, et il attira du monde à son parti: il périt aussi, et tous ceux qui l`avaient suivi furent dispersés. 
\verse Et maintenant, je vous le dis ne vous occupez plus de ces hommes, et laissez-les aller. Si cette entreprise ou cette oeuvre vient des hommes, elle se détruira; 
\verse mais si elle vient de Dieu, vous ne pourrez la détruire. Ne courez pas le risque d`avoir combattu contre Dieu. 
\verse Ils se rangèrent à son avis. Et ayant appelé les apôtres, ils les firent battre de verges, ils leur défendirent de parler au nom de Jésus, et ils les relâchèrent. 
\verse Les apôtres se retirèrent de devant le sanhédrin, joyeux d`avoir été jugés dignes de subir des outrages pour le nom de Jésus. 
\verse Et chaque jour, dans le temple et dans les maisons, ils ne cessaient d`enseigner, et d`annoncer la bonne nouvelle de Jésus Christ. 

\chapter
\verse En ce temps-là, le nombre des disciples augmentant, les Hellénistes murmurèrent contre les Hébreux, parce que leurs veuves étaient négligées dans la distribution qui se faisait chaque jour. 
\verse Les douze convoquèrent la multitude des disciples, et dirent: Il n`est pas convenable que nous laissions la parole de Dieu pour servir aux tables. 
\verse C`est pourquoi, frères, choisissez parmi vous sept hommes, de qui l`on rende un bon témoignage, qui soient pleins d`Esprit Saint et de sagesse, et que nous chargerons de cet emploi. 
\verse Et nous, nous continuerons à nous appliquer à la prière et au ministère de la parole. 
\verse Cette proposition plut à toute l`assemblée. Ils élurent Étienne, homme plein de foi et d`Esprit Saint, Philippe, Prochore, Nicanor, Timon, Parménas, et Nicolas, prosélyte d`Antioche. 
\verse Ils les présentèrent aux apôtres, qui, après avoir prié, leur imposèrent les mains. 
\verse La parole de Dieu se répandait de plus en plus, le nombre des disciples augmentait beaucoup à Jérusalem, et une grande foule de sacrificateurs obéissaient à la foi. 
\verse Étienne, plein de grâce et de puissance, faisait des prodiges et de grands miracles parmi le peuple. 
\verse Quelques membres de la synagogue dite des Affranchis, de celle des Cyrénéens et de celle des Alexandrins, avec des Juifs de Cilicie et d`Asie, se mirent à discuter avec lui; 
\verse mais ils ne pouvaient résister à sa sagesse et à l`Esprit par lequel il parlait. 
\verse Alors ils subornèrent des hommes qui dirent: Nous l`avons entendu proférer des paroles blasphématoires contre Moïse et contre Dieu. 
\verse Ils émurent le peuple, les anciens et les scribes, et, se jetant sur lui, ils le saisirent, et l`emmenèrent au sanhédrin. 
\verse Ils produisirent de faux témoins, qui dirent: Cet homme ne cesse de proférer des paroles contre le lieu saint et contre la loi; 
\verse car nous l`avons entendu dire que Jésus, ce Nazaréen, détruira ce lieu, et changera les coutumes que Moïse nous a données. 
\verse Tous ceux qui siégeaient au sanhédrin ayant fixé les regards sur Étienne, son visage leur parut comme celui d`un ange. 

\chapter
\verse Le souverain sacrificateur dit: Les choses sont-elles ainsi? 
\verse Étienne répondit: Hommes frères et pères, écoutez! Le Dieu de gloire apparut à notre père Abraham, lorsqu`il était en Mésopotamie, avant qu`il s`établît à Charran; et il lui dit: 
\verse Quitte ton pays et ta famille, et va dans le pays que je te montrerai. 
\verse Il sortit alors du pays des Chaldéens, et s`établit à Charran. De là, après la mort de son père, Dieu le fit passer dans ce pays que vous habitez maintenant; 
\verse il ne lui donna aucune propriété en ce pays, pas même de quoi poser le pied, mais il promit de lui en donner la possession, et à sa postérité après lui, quoiqu`il n`eût point d`enfant. 
\verse Dieu parla ainsi: Sa postérité séjournera dans un pays étranger; on la réduira en servitude et on la maltraitera pendant quatre cents ans. 
\verse Mais la nation à laquelle ils auront été asservis, c`est moi qui la jugerai, dit Dieu. Après cela, ils sortiront, et ils me serviront dans ce lieu-ci. 
\verse Puis Dieu donna à Abraham l`alliance de la circoncision; et ainsi, Abraham, ayant engendré Isaac, le circoncit le huitième jour; Isaac engendra et circoncit Jacob, et Jacob les douze patriarches. 
\verse Les patriarches, jaloux de Joseph, le vendirent pour être emmené en Égypte. 
\verse Mais Dieu fut avec lui, et le délivra de toutes ses tribulations; il lui donna de la sagesse et lui fit trouver grâce devant Pharaon, roi d`Égypte, qui l`établit gouverneur d`Égypte et de toute sa maison. 
\verse Il survint une famine dans tout le pays d`Égypte, et dans celui de Canaan. La détresse était grande, et nos pères ne trouvaient pas de quoi se nourrir. 
\verse Jacob apprit qu`il y avait du blé en Égypte, et il y envoya nos pères une première fois. 
\verse Et la seconde fois, Joseph fut reconnu par ses frères, et Pharaon sut de quelle famille il était. 
\verse Puis Joseph envoya chercher son père Jacob, et toute sa famille, composée de soixante-quinze personnes. 
\verse Jacob descendit en Égypte, où il mourut, ainsi que nos pères; 
\verse et ils furent transportés à Sichem, et déposés dans le sépulcre qu`Abraham avait acheté, à prix d`argent, des fils d`Hémor, père de Sichem. 
\verse Le temps approchait où devait s`accomplir la promesse que Dieu avait faite à Abraham, et le peuple s`accrut et se multiplia en Égypte, 
\verse jusqu`à ce que parut un autre roi, qui n`avait pas connu Joseph. 
\verse Ce roi, usant d`artifice contre notre race, maltraita nos pères, au point de leur faire exposer leurs enfants, pour qu`ils ne vécussent pas. 
\verse A cette époque, naquit Moïse, qui était beau aux yeux de Dieu. Il fut nourri trois mois dans la maison de son père; 
\verse et, quand il eut été exposé, la fille de Pharaon le recueillit, et l`éleva comme son fils. 
\verse Moïse fut instruit dans toute la sagesse des Égyptiens, et il était puissant en paroles et en oeuvres. 
\verse Il avait quarante ans, lorsqu`il lui vint dans le coeur de visiter ses frères, les fils d`Israël. 
\verse Il en vit un qu`on outrageait, et, prenant sa défense, il vengea celui qui était maltraité, et frappa l`Égyptien. 
\verse Il pensait que ses frères comprendraient que Dieu leur accordait la délivrance par sa main; mais ils ne comprirent pas. 
\verse Le jour suivant, il parut au milieu d`eux comme ils se battaient, et il les exhorta à la paix: Hommes, dit-il, vous êtes frères; pourquoi vous maltraitez-vous l`un l`autre? 
\verse Mais celui qui maltraitait son prochain le repoussa, en disant: Qui t`a établi chef et juge sur nous? 
\verse Veux-tu me tuer, comme tu as tué hier l`Égyptien? 
\verse A cette parole, Moïse prit la fuite, et il alla séjourner dans le pays de Madian, où il engendra deux fils. 
\verse Quarante ans plus tard, un ange lui apparut, au désert de la montagne de Sinaï, dans la flamme d`un buisson en feu. 
\verse Moïse, voyant cela, fut étonné de cette apparition; et, comme il s`approchait pour examiner, la voix du Seigneur se fit entendre: 
\verse Je suis le Dieu de tes pères, le Dieu d`Abraham, d`Isaac et de Jacob. Et Moïse, tout tremblant, n`osait regarder. 
\verse Le Seigneur lui dit: Ote tes souliers de tes pieds, car le lieu sur lequel tu te tiens est une terre sainte. 
\verse J`ai vu la souffrance de mon peuple qui est en Égypte, j`ai entendu ses gémissements, et je suis descendu pour le délivrer. Maintenant, va, je t`enverrai en Égypte. 
\verse Ce Moïse, qu`ils avaient renié, en disant: Qui t`a établi chef et juge? c`est lui que Dieu envoya comme chef et comme libérateur avec l`aide de l`ange qui lui était apparu dans le buisson. 
\verse C`est lui qui les fit sortir d`Égypte, en opérant des prodiges et des miracles au pays d`Égypte, au sein de la mer Rouge, et au désert, pendant quarante ans. 
\verse C`est ce Moïse qui dit aux fils d`Israël: Dieu vous suscitera d`entre vos frères un prophète comme moi. 
\verse C`est lui qui, lors de l`assemblée au désert, étant avec l`ange qui lui parlait sur la montagne de Sinaï et avec nos pères, reçut des oracles vivants, pour nous les donner. 
\verse Nos pères ne voulurent pas lui obéir, ils le repoussèrent, et ils tournèrent leur coeur vers l`Égypte, 
\verse en disant à Aaron: Fais-nous des dieux qui marchent devant nous; car ce Moïse qui nous a fait sortir du pays d`Égypte, nous ne savons ce qu`il est devenu. 
\verse Et, en ces jours-là, ils firent un veau, ils offrirent un sacrifice à l`idole, et se réjouirent de l`oeuvre de leurs mains. 
\verse Alors Dieu se détourna, et les livra au culte de l`armée du ciel, selon qu`il est écrit dans le livre des prophètes: M`avez-vous offert des victimes et des sacrifices Pendant quarante ans au désert, maison d`Israël?... 
\verse Vous avez porté la tente de Moloch Et l`étoile du dieu Remphan, Ces images que vous avez faites pour les adorer! Aussi vous transporterai-je au delà de Babylone. 
\verse Nos pères avaient au désert le tabernacle du témoignage, comme l`avait ordonné celui qui dit à Moïse de le faire d`après le modèle qu`il avait vu. 
\verse Et nos pères, l`ayant reçu, l`introduisirent, sous la conduite de Josué, dans le pays qui était possédé par les nations que Dieu chassa devant eux, et il y resta jusqu`aux jours de David. 
\verse David trouva grâce devant Dieu, et demanda d`élever une demeure pour le Dieu de Jacob; 
\verse et ce fut Salomon qui lui bâtit une maison. 
\verse Mais le Très Haut n`habite pas dans ce qui est fait de main d`homme, comme dit le prophète: 
\verse Le ciel est mon trône, Et la terre mon marchepied. Quelle maison me bâtirez-vous, dit le Seigneur, Ou quel sera le lieu de mon repos? 
\verse N`est-ce pas ma main qui a fait toutes ces choses?... 
\verse Hommes au cou raide, incirconcis de coeur et d`oreilles! vous vous opposez toujours au Saint Esprit. Ce que vos pères ont été, vous l`êtes aussi. 
\verse Lequel des prophètes vos pères n`ont-ils pas persécuté? Ils ont tué ceux qui annonçaient d`avance la venue du Juste, que vous avez livré maintenant, et dont vous avez été les meurtriers, 
\verse vous qui avez reçu la loi d`après des commandements d`anges, et qui ne l`avez point gardée!... 
\verse En entendant ces paroles, ils étaient furieux dans leur coeur, et ils grinçaient des dents contre lui. 
\verse Mais Étienne, rempli du Saint Esprit, et fixant les regards vers le ciel, vit la gloire de Dieu et Jésus debout à la droite de Dieu. 
\verse Et il dit: Voici, je vois les cieux ouverts, et le Fils de l`homme debout à la droite de Dieu. 
\verse Ils poussèrent alors de grands cris, en se bouchant les oreilles, et ils se précipitèrent tous ensemble sur lui, 
\verse le traînèrent hors de la ville, et le lapidèrent. Les témoins déposèrent leurs vêtements aux pieds d`un jeune homme nommé Saul. 
\verse Et ils lapidaient Étienne, qui priait et disait: Seigneur Jésus, reçois mon esprit! 
\verse Puis, s`étant mis à genoux, il s`écria d`une voix forte: Seigneur, ne leur impute pas ce péché! Et, après ces paroles, il s`endormit. 

\chapter
\verse Saul avait approuvé le meurtre d`Étienne. Il y eut, ce jour-là, une grande persécution contre l`Église de Jérusalem; et tous, excepté les apôtres, se dispersèrent dans les contrées de la Judée et de la Samarie. 
\verse Des hommes pieux ensevelirent Étienne, et le pleurèrent à grand bruit. 
\verse Saul, de son côté, ravageait l`Église; pénétrant dans les maisons, il en arrachait hommes et femmes, et les faisait jeter en prison. 
\verse Ceux qui avaient été dispersés allaient de lieu en lieu, annonçant la bonne nouvelle de la parole. 
\verse Philippe, étant descendu dans la ville de Samarie, y prêcha le Christ. 
\verse Les foules tout entières étaient attentives à ce que disait Philippe, lorsqu`elles apprirent et virent les miracles qu`il faisait. 
\verse Car des esprits impurs sortirent de plusieurs démoniaques, en poussant de grands cris, et beaucoup de paralytiques et de boiteux furent guéris. 
\verse Et il y eut une grande joie dans cette ville. 
\verse Il y avait auparavant dans la ville un homme nommé Simon, qui, se donnant pour un personnage important, exerçait la magie et provoquait l`étonnement du peuple de la Samarie. 
\verse Tous, depuis le plus petit jusqu`au plus grand, l`écoutaient attentivement, et disaient: Celui-ci est la puissance de Dieu, celle qui s`appelle la grande. 
\verse Ils l`écoutaient attentivement, parce qu`il les avait longtemps étonnés par ses actes de magie. 
\verse Mais, quand ils eurent cru à Philippe, qui leur annonçait la bonne nouvelle du royaume de Dieu et du nom de Jésus Christ, hommes et femmes se firent baptiser. 
\verse Simon lui-même crut, et, après avoir été baptisé, il ne quittait plus Philippe, et il voyait avec étonnement les miracles et les grands prodiges qui s`opéraient. 
\verse Les apôtres, qui étaient à Jérusalem, ayant appris que la Samarie avait reçu la parole de Dieu, y envoyèrent Pierre et Jean. 
\verse Ceux-ci, arrivés chez les Samaritains, prièrent pour eux, afin qu`ils reçussent le Saint Esprit. 
\verse Car il n`était encore descendu sur aucun d`eux; ils avaient seulement été baptisés au nom du Seigneur Jésus. 
\verse Alors Pierre et Jean leur imposèrent les mains, et ils reçurent le Saint Esprit. 
\verse Lorsque Simon vit que le Saint Esprit était donné par l`imposition des mains des apôtres, il leur offrit de l`argent, 
\verse en disant: Accordez-moi aussi ce pouvoir, afin que celui à qui j`imposerai les mains reçoive le Saint Esprit. 
\verse Mais Pierre lui dit: Que ton argent périsse avec toi, puisque tu as cru que le don de Dieu s`acquérait à prix d`argent! 
\verse Il n`y a pour toi ni part ni lot dans cette affaire, car ton coeur n`est pas droit devant Dieu. 
\verse Repens-toi donc de ta méchanceté, et prie le Seigneur pour que la pensée de ton coeur te soit pardonnée, s`il est possible; 
\verse car je vois que tu es dans un fiel amer et dans les liens de l`iniquité. 
\verse Simon répondit: Priez vous-mêmes le Seigneur pour moi, afin qu`il ne m`arrive rien de ce que vous avez dit. 
\verse Après avoir rendu témoignage à la parole du Seigneur, et après l`avoir prêchée, Pierre et Jean retournèrent à Jérusalem, en annonçant la bonne nouvelle dans plusieurs villages des Samaritains. 
\verse Un ange du Seigneur, s`adressant à Philippe, lui dit: Lève-toi, et va du côté du midi, sur le chemin qui descend de Jérusalem à Gaza, celui qui est désert. 
\verse Il se leva, et partit. Et voici, un Éthiopien, un eunuque, ministre de Candace, reine d`Éthiopie, et surintendant de tous ses trésors, venu à Jérusalem pour adorer, 
\verse s`en retournait, assis sur son char, et lisait le prophète Ésaïe. 
\verse L`Esprit dit à Philippe: Avance, et approche-toi de ce char. 
\verse Philippe accourut, et entendit l`Éthiopien qui lisait le prophète Ésaïe. Il lui dit: Comprends-tu ce que tu lis? 
\verse Il répondit: Comment le pourrais-je, si quelqu`un ne me guide? Et il invita Philippe à monter et à s`asseoir avec lui. 
\verse Le passage de l`Écriture qu`il lisait était celui-ci: Il a été mené comme une brebis à la boucherie; Et, comme un agneau muet devant celui qui le tond, Il n`a point ouvert la bouche. 
\verse Dans son humiliation, son jugement a été levé. Et sa postérité, qui la dépeindra? Car sa vie a été retranchée de la terre. 
\verse L`eunuque dit à Philippe: Je te prie, de qui le prophète parle-t-il ainsi? Est-ce de lui-même, ou de quelque autre? 
\verse Alors Philippe, ouvrant la bouche et commençant par ce passage, lui annonça la bonne nouvelle de Jésus. 
\verse Comme ils continuaient leur chemin, ils rencontrèrent de l`eau. Et l`eunuque dit: Voici de l`eau; qu`est-ce qui empêche que je ne sois baptisé? 
\verse Philippe dit: Si tu crois de tout ton coeur, cela est possible. L`eunuque répondit: Je crois que Jésus Christ est le Fils de Dieu. 
\verse Il fit arrêter le char; Philippe et l`eunuque descendirent tous deux dans l`eau, et Philippe baptisa l`eunuque. 
\verse Quand ils furent sortis de l`eau, l`Esprit du Seigneur enleva Philippe, et l`eunuque ne le vit plus. Tandis que, joyeux, il poursuivait sa route, 
\verse Philippe se trouva dans Azot, d`où il alla jusqu`à Césarée, en évangélisant toutes les villes par lesquelles il passait. 

\chapter
\verse Cependant Saul, respirant encore la menace et le meurtre contre les disciples du Seigneur, se rendit chez le souverain sacrificateur, 
\verse et lui demanda des lettres pour les synagogues de Damas, afin que, s`il trouvait des partisans de la nouvelle doctrine, hommes ou femmes, il les amenât liés à Jérusalem. 
\verse Comme il était en chemin, et qu`il approchait de Damas, tout à coup une lumière venant du ciel resplendit autour de lui. 
\verse Il tomba par terre, et il entendit une voix qui lui disait: Saul, Saul, pourquoi me persécutes-tu? 
\verse Il répondit: Qui es-tu, Seigneur? Et le Seigneur dit: Je suis Jésus que tu persécutes. Il te serait dur de regimber contre les aiguillons. 
\verse Tremblant et saisi d`effroi, il dit: Seigneur, que veux-tu que je fasse? Et le Seigneur lui dit: Lève-toi, entre dans la ville, et on te dira ce que tu dois faire. 
\verse Les hommes qui l`accompagnaient demeurèrent stupéfaits; ils entendaient bien la voix, mais ils ne voyaient personne. 
\verse Saul se releva de terre, et, quoique ses yeux fussent ouverts, il ne voyait rien; on le prit par la main, et on le conduisit à Damas. 
\verse Il resta trois jours sans voir, et il ne mangea ni ne but. 
\verse Or, il y avait à Damas un disciple nommé Ananias. Le Seigneur lui dit dans une vision: Ananias! Il répondit: Me voici, Seigneur! 
\verse Et le Seigneur lui dit: Lève-toi, va dans la rue qu`on appelle la droite, et cherche, dans la maison de Judas, un nommé Saul de Tarse. 
\verse Car il prie, et il a vu en vision un homme du nom d`Ananias, qui entrait, et qui lui imposait les mains, afin qu`il recouvrât la vue. Ananias répondit: 
\verse Seigneur, j`ai appris de plusieurs personnes tous les maux que cet homme a faits à tes saints dans Jérusalem; 
\verse et il a ici des pouvoirs, de la part des principaux sacrificateurs, pour lier tous ceux qui invoquent ton nom. 
\verse Mais le Seigneur lui dit: Va, car cet homme est un instrument que j`ai choisi, pour porter mon nom devant les nations, devant les rois, et devant les fils d`Israël; 
\verse et je lui montrerai tout ce qu`il doit souffrir pour mon nom. 
\verse Ananias sortit; et, lorsqu`il fut arrivé dans la maison, il imposa les mains à Saul, en disant: Saul, mon frère, le Seigneur Jésus, qui t`est apparu sur le chemin par lequel tu venais, m`a envoyé pour que tu recouvres la vue et que tu sois rempli du Saint Esprit. 
\verse Au même instant, il tomba de ses yeux comme des écailles, et il recouvra la vue. Il se leva, et fut baptisé; 
\verse et, après qu`il eut pris de la nourriture, les forces lui revinrent. Saul resta quelques jours avec les disciples qui étaient à Damas. 
\verse Et aussitôt il prêcha dans les synagogues que Jésus est le Fils de Dieu. 
\verse Tous ceux qui l`entendaient étaient dans l`étonnement, et disaient: N`est-ce pas celui qui persécutait à Jérusalem ceux qui invoquent ce nom, et n`est-il pas venu ici pour les emmener liés devant les principaux sacrificateurs? 
\verse Cependant Saul se fortifiait de plus en plus, et il confondait les Juifs qui habitaient Damas, démontrant que Jésus est le Christ. 
\verse Au bout d`un certain temps, les Juifs se concertèrent pour le tuer, 
\verse et leur complot parvint à la connaissance de Saul. On gardait les portes jour et nuit, afin de lui ôter la vie. 
\verse Mais, pendant une nuit, les disciples le prirent, et le descendirent par la muraille, dans une corbeille. 
\verse Lorsqu`il se rendit à Jérusalem, Saul tâcha de se joindre à eux; mais tous le craignaient, ne croyant pas qu`il fût un disciple. 
\verse Alors Barnabas, l`ayant pris avec lui, le conduisit vers les apôtres, et leur raconta comment sur le chemin Saul avait vu le Seigneur, qui lui avait parlé, et comment à Damas il avait prêché franchement au nom de Jésus. 
\verse Il allait et venait avec eux dans Jérusalem, et s`exprimait en toute assurance au nom du Seigneur. 
\verse Il parlait aussi et disputait avec les Hellénistes; mais ceux-ci cherchaient à lui ôter la vie. 
\verse Les frères, l`ayant su, l`emmenèrent à Césarée, et le firent partir pour Tarse. 
\verse L`Église était en paix dans toute la Judée, la Galilée et la Samarie, s`édifiant et marchant dans la crainte du Seigneur, et elle s`accroissait par l`assistance du Saint Esprit. 
\verse Comme Pierre visitait tous les saints, il descendit aussi vers ceux qui demeuraient à Lydde. 
\verse Il y trouva un homme nommé Énée, couché sur un lit depuis huit ans, et paralytique. 
\verse Pierre lui dit: Énée, Jésus Christ te guérit; lève-toi, et arrange ton lit. Et aussitôt il se leva. 
\verse Tous les habitants de Lydde et du Saron le virent, et ils se convertirent au Seigneur. 
\verse Il y avait à Joppé, parmi les disciples, une femme nommée Tabitha, ce qui signifie Dorcas: elle faisait beaucoup de bonnes oeuvres et d`aumônes. 
\verse Elle tomba malade en ce temps-là, et mourut. Après l`avoir lavée, on la déposa dans une chambre haute. 
\verse Comme Lydde est près de Joppé, les disciples, ayant appris que Pierre s`y trouvait, envoyèrent deux hommes vers lui, pour le prier de venir chez eux sans tarder. 
\verse Pierre se leva, et partit avec ces hommes. Lorsqu`il fut arrivé, on le conduisit dans la chambre haute. Toutes les veuves l`entourèrent en pleurant, et lui montrèrent les tuniques et les vêtements que faisait Dorcas pendant qu`elle était avec elles. 
\verse Pierre fit sortir tout le monde, se mit à genoux, et pria; puis, se tournant vers le corps, il dit: Tabitha, lève-toi! Elle ouvrit les yeux, et ayant vu Pierre, elle s`assit. 
\verse Il lui donna la main, et la fit lever. Il appela ensuite les saints et les veuves, et la leur présenta vivante. 
\verse Cela fut connu de tout Joppé, et beaucoup crurent au Seigneur. 
\verse Pierre demeura quelque temps à Joppé, chez un corroyeur nommé Simon. 

\chapter
\verse Il y avait à Césarée un homme nommé Corneille, centenier dans la cohorte dite italienne. 
\verse Cet homme était pieux et craignait Dieu, avec toute sa maison; il faisait beaucoup d`aumônes au peuple, et priait Dieu continuellement. 
\verse Vers la neuvième heure du jour, il vit clairement dans une vision un ange de Dieu qui entra chez lui, et qui lui dit: Corneille! 
\verse Les regards fixés sur lui, et saisi d`effroi, il répondit: Qu`est-ce, Seigneur? Et l`ange lui dit: Tes prières et tes aumônes sont montées devant Dieu, et il s`en est souvenu. 
\verse Envoie maintenant des hommes à Joppé, et fais venir Simon, surnommé Pierre; 
\verse il est logé chez un certain Simon, corroyeur, dont la maison est près de la mer. 
\verse Dès que l`ange qui lui avait parlé fut parti, Corneille appela deux de ses serviteurs, et un soldat pieux d`entre ceux qui étaient attachés à sa personne; 
\verse et, après leur avoir tout raconté, il les envoya à Joppé. 
\verse Le lendemain, comme ils étaient en route, et qu`ils approchaient de la ville, Pierre monta sur le toit, vers la sixième heure, pour prier. 
\verse Il eut faim, et il voulut manger. Pendant qu`on lui préparait à manger, il tomba en extase. 
\verse Il vit le ciel ouvert, et un objet semblable à une grande nappe attachée par les quatre coins, qui descendait et s`abaissait vers la terre, 
\verse et où se trouvaient tous les quadrupèdes et les reptiles de la terre et les oiseaux du ciel. 
\verse Et une voix lui dit: Lève-toi, Pierre, tue et mange. 
\verse Mais Pierre dit: Non, Seigneur, car je n`ai jamais rien mangé de souillé ni d`impur. 
\verse Et pour la seconde fois la voix se fit encore entendre à lui: Ce que Dieu a déclaré pur, ne le regarde pas comme souillé. 
\verse Cela arriva jusqu`à trois fois; et aussitôt après, l`objet fut retiré dans le ciel. 
\verse Tandis que Pierre ne savait en lui-même que penser du sens de la vision qu`il avait eue, voici, les hommes envoyés par Corneille, s`étant informés de la maison de Simon, se présentèrent à la porte, 
\verse et demandèrent à haute voix si c`était là que logeait Simon, surnommé Pierre. 
\verse Et comme Pierre était à réfléchir sur la vision, l`Esprit lui dit: Voici, trois hommes te demandent; 
\verse lève-toi, descends, et pars avec eux sans hésiter, car c`est moi qui les ai envoyés. 
\verse Pierre donc descendit, et il dit à ces hommes: Voici, je suis celui que vous cherchez; quel est le motif qui vous amène? 
\verse Ils répondirent: Corneille, centenier, homme juste et craignant Dieu, et de qui toute la nation des Juifs rend un bon témoignage, a été divinement averti par un saint ange de te faire venir dans sa maison et d`entendre tes paroles. 
\verse Pierre donc les fit entrer, et les logea. Le lendemain, il se leva, et partit avec eux. Quelques-uns des frères de Joppé l`accompagnèrent. 
\verse Ils arrivèrent à Césarée le jour suivant. Corneille les attendait, et avait invité ses parents et ses amis intimes. 
\verse Lorsque Pierre entra, Corneille, qui était allé au-devant de lui, tomba à ses pieds et se prosterna. 
\verse Mais Pierre le releva, en disant: Lève-toi; moi aussi, je suis un homme. 
\verse Et conversant avec lui, il entra, et trouva beaucoup de personnes réunies. 
\verse Vous savez, leur dit-il, qu`il est défendu à un Juif de se lier avec un étranger ou d`entrer chez lui; mais Dieu m`a appris à ne regarder aucun homme comme souillé et impur. 
\verse C`est pourquoi je n`ai pas eu d`objection à venir, puisque vous m`avez appelé; je vous demande donc pour quel motif vous m`avez envoyé chercher. 
\verse Corneille dit: Il y a quatre jours, à cette heure-ci, je priais dans ma maison à la neuvième heure; et voici, un homme vêtu d`un habit éclatant se présenta devant moi, et dit: 
\verse Corneille, ta prière a été exaucée, et Dieu s`est souvenu de tes aumônes. 
\verse Envoie donc à Joppé, et fais venir Simon, surnommé Pierre; il est logé dans la maison de Simon, corroyeur, près de la mer. 
\verse Aussitôt j`ai envoyé vers toi, et tu as bien fait de venir. Maintenant donc nous sommes tous devant Dieu, pour entendre tout ce que le Seigneur t`a ordonné de nous dire. 
\verse Alors Pierre, ouvrant la bouche, dit: En vérité, je reconnais que Dieu ne fait point acception de personnes, 
\verse mais qu`en toute nation celui qui le craint et qui pratique la justice lui est agréable. 
\verse Il a envoyé la parole aux fils d`Israël, en leur annonçant la paix par Jésus Christ, qui est le Seigneur de tous. 
\verse Vous savez ce qui est arrivé dans toute la Judée, après avoir commencé en Galilée, à la suite du baptême que Jean a prêché; 
\verse vous savez comment Dieu a oint du Saint Esprit et de force Jésus de Nazareth, qui allait de lieu en lieu faisant du bien et guérissant tous ceux qui étaient sous l`empire du diable, car Dieu était avec lui. 
\verse Nous sommes témoins de tout ce qu`il a fait dans le pays des Juifs et à Jérusalem. Ils l`ont tué, en le pendant au bois. 
\verse Dieu l`a ressuscité le troisième jour, et il a permis qu`il apparût, 
\verse non à tout le peuple, mais aux témoins choisis d`avance par Dieu, à nous qui avons mangé et bu avec lui, après qu`il fut ressuscité des morts. 
\verse Et Jésus nous a ordonné de prêcher au peuple et d`attester que c`est lui qui a été établi par Dieu juge des vivants et des morts. 
\verse Tous les prophètes rendent de lui le témoignage que quiconque croit en lui reçoit par son nom le pardon des péchés. 
\verse Comme Pierre prononçait encore ces mots, le Saint Esprit descendit sur tous ceux qui écoutaient la parole. 
\verse Tous les fidèles circoncis qui étaient venus avec Pierre furent étonnés de ce que le don du Saint Esprit était aussi répandu sur les païens. 
\verse Car ils les entendaient parler en langues et glorifier Dieu. 
\verse Alors Pierre dit: Peut-on refuser l`eau du baptême à ceux qui ont reçu le Saint Esprit aussi bien que nous? 
\verse Et il ordonna qu`ils fussent baptisés au nom du Seigneur. Sur quoi ils le prièrent de rester quelques jours auprès d`eux. 

\chapter
\verse Les apôtres et les frères qui étaient dans la Judée apprirent que les païens avaient aussi reçu la parole de Dieu. 
\verse Et lorsque Pierre fut monté à Jérusalem, les fidèles circoncis lui adressèrent des reproches, 
\verse en disant: Tu es entré chez des incirconcis, et tu as mangé avec eux. 
\verse Pierre se mit à leur exposer d`une manière suivie ce qui s`était passé. 
\verse Il dit: J`étais dans la ville de Joppé, et, pendant que je priais, je tombai en extase et j`eus une vision: un objet, semblable à une grande nappe attachée par les quatre coins, descendait du ciel et vint jusqu`à moi. 
\verse Les regards fixés sur cette nappe, j`examinai, et je vis les quadrupèdes de la terre, les bêtes sauvages, les reptiles, et les oiseaux du ciel. 
\verse Et j`entendis une voix qui me disait: Lève-toi, Pierre, tue et mange. 
\verse Mais je dis: Non, Seigneur, car jamais rien de souillé ni d`impur n`est entré dans ma bouche. 
\verse Et pour la seconde fois la voix se fit entendre du ciel: Ce que Dieu a déclaré pur, ne le regarde pas comme souillé. 
\verse Cela arriva jusqu`à trois fois; puis tout fut retiré dans le ciel. 
\verse Et voici, aussitôt trois hommes envoyés de Césarée vers moi se présentèrent devant la porte de la maison où j`étais. 
\verse L`Esprit me dit de partir avec eux sans hésiter. Les six hommes que voici m`accompagnèrent, et nous entrâmes dans la maison de Corneille. 
\verse Cet homme nous raconta comment il avait vu dans sa maison l`ange se présentant à lui et disant: Envoie à Joppé, et fais venir Simon, surnommé Pierre, 
\verse qui te dira des choses par lesquelles tu seras sauvé, toi et toute ta maison. 
\verse Lorsque je me fus mis à parler, le Saint Esprit descendit sur eux, comme sur nous au commencement. 
\verse Et je me souvins de cette parole du Seigneur: Jean a baptisé d`eau, mais vous, vous serez baptisés du Saint Esprit. 
\verse Or, puisque Dieu leur a accordé le même don qu`à nous qui avons cru au Seigneur Jésus Christ, pouvais-je, moi, m`opposer à Dieu? 
\verse Après avoir entendu cela, ils se calmèrent, et ils glorifièrent Dieu, en disant: Dieu a donc accordé la repentance aussi aux païens, afin qu`ils aient la vie. 
\verse Ceux qui avaient été dispersés par la persécution survenue à l`occasion d`Étienne allèrent jusqu`en Phénicie, dans l`île de Chypre, et à Antioche, annonçant la parole seulement aux Juifs. 
\verse Il y eut cependant parmi eux quelques hommes de Chypre et de Cyrène, qui, étant venus à Antioche, s`adressèrent aussi aux Grecs, et leur annoncèrent la bonne nouvelle du Seigneur Jésus. 
\verse La main du Seigneur était avec eux, et un grand nombre de personnes crurent et se convertirent au Seigneur. 
\verse Le bruit en parvint aux oreilles des membres de l`Église de Jérusalem, et ils envoyèrent Barnabas jusqu`à Antioche. 
\verse Lorsqu`il fut arrivé, et qu`il eut vu la grâce de Dieu, il s`en réjouit, et il les exhorta tous à rester d`un coeur ferme attachés au Seigneur. 
\verse Car c`était un homme de bien, plein d`Esprit Saint et de foi. Et une foule assez nombreuse se joignit au Seigneur. 
\verse Barnabas se rendit ensuite à Tarse, pour chercher Saul; 
\verse et, l`ayant trouvé, il l`amena à Antioche. Pendant toute une année, ils se réunirent aux assemblées de l`Église, et ils enseignèrent beaucoup de personnes. Ce fut à Antioche que, pour la première fois, les disciples furent appelés chrétiens. 
\verse En ce temps-là, des prophètes descendirent de Jérusalem à Antioche. 
\verse L`un deux, nommé Agabus, se leva, et annonça par l`Esprit qu`il y aurait une grande famine sur toute la terre. Elle arriva, en effet, sous Claude. 
\verse Les disciples résolurent d`envoyer, chacun selon ses moyens, un secours aux frères qui habitaient la Judée. 
\verse Ils le firent parvenir aux anciens par les mains de Barnabas et de Saul. 

\chapter
\verse Vers le même temps, le roi Hérode se mit à maltraiter quelques membres de l`Église, 
\verse et il fit mourir par l`épée Jacques, frère de Jean. 
\verse Voyant que cela était agréable aux Juifs, il fit encore arrêter Pierre. -C`était pendant les jours des pains sans levain. - 
\verse Après l`avoir saisi et jeté en prison, il le mit sous la garde de quatre escouades de quatre soldats chacune, avec l`intention de le faire comparaître devant le peuple après la Pâque. 
\verse Pierre donc était gardé dans la prison; et l`Église ne cessait d`adresser pour lui des prières à Dieu. 
\verse La nuit qui précéda le jour où Hérode allait le faire comparaître, Pierre, lié de deux chaînes, dormait entre deux soldats; et des sentinelles devant la porte gardaient la prison. 
\verse Et voici, un ange du Seigneur survint, et une lumière brilla dans la prison. L`ange réveilla Pierre, en le frappant au côté, et en disant: Lève-toi promptement! Les chaînes tombèrent de ses mains. 
\verse Et l`ange lui dit: Mets ta ceinture et tes sandales. Et il fit ainsi. L`ange lui dit encore: Enveloppe-toi de ton manteau, et suis-moi. 
\verse Pierre sortit, et le suivit, ne sachant pas que ce qui se faisait par l`ange fût réel, et s`imaginant avoir une vision. 
\verse Lorsqu`ils eurent passé la première garde, puis la seconde, ils arrivèrent à la porte de fer qui mène à la ville, et qui s`ouvrit d`elle-même devant eux; ils sortirent, et s`avancèrent dans une rue. Aussitôt l`ange quitta Pierre. 
\verse Revenu à lui-même, Pierre dit: Je vois maintenant d`une manière certaine que le Seigneur a envoyé son ange, et qu`il m`a délivré de la main d`Hérode et de tout ce que le peuple juif attendait. 
\verse Après avoir réfléchi, il se dirigea vers la maison de Marie, mère de Jean, surnommé Marc, où beaucoup de personnes étaient réunies et priaient. 
\verse Il frappa à la porte du vestibule, et une servante, nommée Rhode, s`approcha pour écouter. 
\verse Elle reconnut la voix de Pierre; et, dans sa joie, au lieu d`ouvrir, elle courut annoncer que Pierre était devant la porte. 
\verse Ils lui dirent: Tu es folle. Mais elle affirma que la chose était ainsi. 
\verse Et ils dirent: C`est son ange. Cependant Pierre continuait à frapper. Ils ouvrirent, et furent étonnés de le voir. 
\verse Pierre, leur ayant de la main fait signe de se taire, leur raconta comment le Seigneur l`avait tiré de la prison, et il dit: Annoncez-le à Jacques et aux frères. Puis il sortit, et s`en alla dans un autre lieu. 
\verse Quand il fit jour, les soldats furent dans une grande agitation, pour savoir ce que Pierre était devenu. 
\verse Hérode, s`étant mis à sa recherche et ne l`ayant pas trouvé, interrogea les gardes, et donna l`ordre de les mener au supplice. Ensuite il descendit de la Judée à Césarée, pour y séjourner. 
\verse Hérode avait des dispositions hostiles à l`égard des Tyriens et des Sidoniens. Mais ils vinrent le trouver d`un commun accord; et, après avoir gagné Blaste, son chambellan, ils sollicitèrent la paix, parce que leur pays tirait sa subsistance de celui du roi. 
\verse A un jour fixé, Hérode, revêtu de ses habits royaux, et assis sur son trône, les harangua publiquement. 
\verse Le peuple s`écria: Voix d`un dieu, et non d`un homme! 
\verse Au même instant, un ange du Seigneur le frappa, parce qu`il n`avait pas donné gloire à Dieu. Et il expira, rongé des vers. 
\verse Cependant la parole de Dieu se répandait de plus en plus, et le nombre des disciples augmentait. 
\verse Barnabas et Saul, après s`être acquittés de leur message, s`en retournèrent de Jérusalem, emmenant avec eux Jean, surnommé Marc. 

\chapter
\verse Il y avait dans l`Église d`Antioche des prophètes et des docteurs: Barnabas, Siméon appelé Niger, Lucius de Cyrène, Manahen, qui avait été élevé avec Hérode le tétrarque, et Saul. 
\verse Pendant qu`ils servaient le Seigneur dans leur ministère et qu`ils jeûnaient, le Saint Esprit dit: Mettez-moi à part Barnabas et Saul pour l`oeuvre à laquelle je les ai appelés. 
\verse Alors, après avoir jeûné et prié, ils leur imposèrent les mains, et les laissèrent partir. 
\verse Barnabas et Saul, envoyés par le Saint Esprit, descendirent à Séleucie, et de là ils s`embarquèrent pour l`île de Chypre. 
\verse Arrivés à Salamine, ils annoncèrent la parole de Dieu dans les synagogues des Juifs. Ils avaient Jean pour aide. 
\verse Ayant ensuite traversé toute l`île jusqu`à Paphos, ils trouvèrent un certain magicien, faux prophète juif, nommé Bar Jésus, 
\verse qui était avec le proconsul Sergius Paulus, homme intelligent. Ce dernier fit appeler Barnabas et Saul, et manifesta le désir d`entendre la parole de Dieu. 
\verse Mais Élymas, le magicien, -car c`est ce que signifie son nom, -leur faisait opposition, cherchant à détourner de la foi le proconsul. 
\verse Alors Saul, appelé aussi Paul, rempli du Saint Esprit, fixa les regards sur lui, et dit: 
\verse Homme plein de toute espèce de ruse et de fraude, fils du diable, ennemi de toute justice, ne cesseras-tu point de pervertir les voies droites du Seigneur? 
\verse Maintenant voici, la main du Seigneur est sur toi, tu seras aveugle, et pour un temps tu ne verras pas le soleil. Aussitôt l`obscurité et les ténèbres tombèrent sur lui, et il cherchait, en tâtonnant, des personnes pour le guider. 
\verse Alors le proconsul, voyant ce qui était arrivé, crut, étant frappé de la doctrine du Seigneur. 
\verse Paul et ses compagnons, s`étant embarqués à Paphos, se rendirent à Perge en Pamphylie. Jean se sépara d`eux, et retourna à Jérusalem. 
\verse De Perge ils poursuivirent leur route, et arrivèrent à Antioche de Pisidie. Étant entrés dans la synagogue le jour du sabbat, ils s`assirent. 
\verse Après la lecture de la loi et des prophètes, les chefs de la synagogue leur envoyèrent dire: Hommes frères, si vous avez quelque exhortation à adresser au peuple, parlez. 
\verse Paul se leva, et, ayant fait signe de la main, il dit: Hommes Israélites, et vous qui craignez Dieu, écoutez! 
\verse Le Dieu de ce peuple d`Israël a choisi nos pères. Il mit ce peuple en honneur pendant son séjour au pays d`Égypte, et il l`en fit sortir par son bras puissant. 
\verse Il les nourrit près de quarante ans dans le désert; 
\verse et, ayant détruit sept nations au pays de Canaan, il leur en accorda le territoire comme propriété. 
\verse Après cela, durant quatre cent cinquante ans environ, il leur donna des juges, jusqu`au prophète Samuel. 
\verse Ils demandèrent alors un roi. Et Dieu leur donna, pendant quarante ans, Saül, fils de Kis, de la tribu de Benjamin; 
\verse puis, l`ayant rejeté, il leur suscita pour roi David, auquel il a rendu ce témoignage: J`ai trouvé David, fils d`Isaï, homme selon mon coeur, qui accomplira toutes mes volontés. 
\verse C`est de la postérité de David que Dieu, selon sa promesse, a suscité à Israël un Sauveur, qui est Jésus. 
\verse Avant sa venue, Jean avait prêché le baptême de repentance à tout le peuple d`Israël. 
\verse Et lorsque Jean achevait sa course, il disait: Je ne suis pas celui que vous pensez; mais voici, après moi vient celui des pieds duquel je ne suis pas digne de délier les souliers. 
\verse Hommes frères, fils de la race d`Abraham, et vous qui craignez Dieu, c`est à vous que cette parole de salut a été envoyée. 
\verse Car les habitants de Jérusalem et leurs chefs ont méconnu Jésus, et, en le condamnant, ils ont accompli les paroles des prophètes qui se lisent chaque sabbat. 
\verse Quoiqu`ils ne trouvassent en lui rien qui fût digne de mort, ils ont demandé à Pilate de le faire mourir. 
\verse Et, après qu`ils eurent accompli tout ce qui est écrit de lui, ils le descendirent de la croix et le déposèrent dans un sépulcre. 
\verse Mais Dieu l`a ressuscité des morts. 
\verse Il est apparu pendant plusieurs jours à ceux qui étaient montés avec lui de la Galilée à Jérusalem, et qui sont maintenant ses témoins auprès du peuple. 
\verse Et nous, nous vous annonçons cette bonne nouvelle que la promesse faite à nos pères, 
\verse Dieu l`a accomplie pour nous leurs enfants, en ressuscitant Jésus, selon ce qui est écrit dans le Psaume deuxième: Tu es mon Fils, Je t`ai engendré aujourd`hui. 
\verse Qu`il l`ait ressuscité des morts, de telle sorte qu`il ne retournera pas à la corruption, c`est ce qu`il a déclaré, en disant: Je vous donnerai Les grâces saintes promises à David, ces grâces qui sont assurées. 
\verse C`est pourquoi il dit encore ailleurs: Tu ne permettras pas que ton Saint voie la corruption. 
\verse Or, David, après avoir en son temps servi au dessein de Dieu, est mort, a été réuni à ses pères, et a vu la corruption. 
\verse Mais celui que Dieu a ressuscité n`a pas vu la corruption. 
\verse Sachez donc, hommes frères, que c`est par lui que le pardon des péchés vous est annoncé, 
\verse et que quiconque croit est justifié par lui de toutes les choses dont vous ne pouviez être justifiés par la loi de Moïse. 
\verse Ainsi, prenez garde qu`il ne vous arrive ce qui est dit dans les prophètes: 
\verse Voyez, contempteurs, Soyez étonnés et disparaissez; Car je vais faire en vos jours une oeuvre, Une oeuvre que vous ne croiriez pas si on vous la racontait. 
\verse Lorsqu`ils sortirent, on les pria de parler le sabbat suivant sur les mêmes choses; 
\verse et, à l`issue de l`assemblée, beaucoup de Juifs et de prosélytes pieux suivirent Paul et Barnabas, qui s`entretinrent avec eux, et les exhortèrent à rester attachés à la grâce de Dieu. 
\verse Le sabbat suivant, presque toute la ville se rassembla pour entendre la parole de Dieu. 
\verse Les Juifs, voyant la foule, furent remplis de jalousie, et ils s`opposaient à ce que disait Paul, en le contredisant et en l`injuriant. 
\verse Paul et Barnabas leur dirent avec assurance: C`est à vous premièrement que la parole de Dieu devait être annoncée; mais, puisque vous la repoussez, et que vous vous jugez vous-mêmes indignes de la vie éternelle, voici, nous nous tournons vers les païens. 
\verse Car ainsi nous l`a ordonné le Seigneur: Je t`ai établi pour être la lumière des nations, Pour porter le salut jusqu`aux extrémités de la terre. 
\verse Les païens se réjouissaient en entendant cela, ils glorifiaient la parole du Seigneur, et tous ceux qui étaient destinés à la vie éternelle crurent. 
\verse La parole du Seigneur se répandait dans tout le pays. 
\verse Mais les Juifs excitèrent les femmes dévotes de distinction et les principaux de la ville; ils provoquèrent une persécution contre Paul et Barnabas, et ils les chassèrent de leur territoire. 
\verse Paul et Barnabas secouèrent contre eux la poussière de leurs pieds, et allèrent à Icone, 
\verse tandis que les disciples étaient remplis de joie et du Saint Esprit. 

\chapter
\verse A Icone, Paul et Barnabas entrèrent ensemble dans la synagogue des Juifs, et ils parlèrent de telle manière qu`une grande multitude de Juifs et de Grecs crurent. 
\verse Mais ceux des Juifs qui ne crurent point excitèrent et aigrirent les esprits des païens contre les frères. 
\verse Ils restèrent cependant assez longtemps à Icone, parlant avec assurance, appuyés sur le Seigneur, qui rendait témoignage à la parole de sa grâce et permettait qu`il se fît par leurs mains des prodiges et des miracles. 
\verse La population de la ville se divisa: les uns étaient pour les Juifs, les autres pour les apôtres. 
\verse Et comme les païens et les Juifs, de concert avec leurs chefs, se mettaient en mouvement pour les outrager et les lapider, 
\verse Paul et Barnabas, en ayant eu connaissance, se réfugièrent dans les villes de la Lycaonie, à Lystre et à Derbe, et dans la contrée d`alentour. 
\verse Et ils y annoncèrent la bonne nouvelle. 
\verse A Lystre, se tenait assis un homme impotent des pieds, boiteux de naissance, et qui n`avait jamais marché. 
\verse Il écoutait parler Paul. Et Paul, fixant les regards sur lui et voyant qu`il avait la foi pour être guéri, 
\verse dit d`une voix forte: Lève-toi droit sur tes pieds. Et il se leva d`un bond et marcha. 
\verse A la vue de ce que Paul avait fait, la foule éleva la voix, et dit en langue lycaonienne: Les dieux sous une forme humaine sont descendus vers nous. 
\verse Ils appelaient Barnabas Jupiter, et Paul Mercure, parce que c`était lui qui portait la parole. 
\verse Le prêtre de Jupiter, dont le temple était à l`entrée de la ville, amena des taureaux avec des bandelettes vers les portes, et voulait, de même que la foule, offrir un sacrifice. 
\verse Les apôtres Barnabas et Paul, ayant appris cela, déchirèrent leurs vêtements, et se précipitèrent au milieu de la foule, 
\verse en s`écriant: O hommes, pourquoi agissez-vous de la sorte? Nous aussi, nous sommes des hommes de la même nature que vous; et, vous apportant une bonne nouvelle, nous vous exhortons à renoncer à ces choses vaines, pour vous tourner vers le Dieu vivant, qui a fait le ciel, la terre, la mer, et tout ce qui s`y trouve. 
\verse Ce Dieu, dans les âges passés, a laissé toutes les nations suivre leurs propres voies, 
\verse quoiqu`il n`ait cessé de rendre témoignage de ce qu`il est, en faisant du bien, en vous dispensant du ciel les pluies et les saisons fertiles, en vous donnant la nourriture avec abondance et en remplissant vos coeurs de joie. 
\verse A peine purent-ils, par ces paroles, empêcher la foule de leur offrir un sacrifice. 
\verse Alors survinrent d`Antioche et d`Icone des Juifs qui gagnèrent la foule, et qui, après avoir lapidé Paul, le traînèrent hors de la ville, pensant qu`il était mort. 
\verse Mais, les disciples l`ayant entouré, il se leva, et entra dans la ville. Le lendemain, il partit pour Derbe avec Barnabas. 
\verse Quand ils eurent évangélisé cette ville et fait un certain nombre de disciples, ils retournèrent à Lystre, à Icone et à Antioche, 
\verse fortifiant l`esprit des disciples, les exhortant à persévérer dans la foi, et disant que c`est par beaucoup de tribulations qu`il nous faut entrer dans le royaume de Dieu. 
\verse Ils firent nommer des anciens dans chaque Église, et, après avoir prié et jeûné, ils les recommandèrent au Seigneur, en qui ils avaient cru. 
\verse Traversant ensuite la Pisidie, ils vinrent en Pamphylie, 
\verse annoncèrent la parole à Perge, et descendirent à Attalie. 
\verse De là ils s`embarquèrent pour Antioche, d`où ils avaient été recommandés à la grâce de Dieu pour l`oeuvre qu`ils venaient d`accomplir. 
\verse Après leur arrivée, ils convoquèrent l`Église, et ils racontèrent tout ce que Dieu avait fait avec eux, et comment il avait ouvert aux nations la porte de la foi. 
\verse Et ils demeurèrent assez longtemps avec les disciples. 

\chapter
\verse Quelques hommes, venus de la Judée, enseignaient les frères, en disant: Si vous n`êtes circoncis selon le rite de Moïse, vous ne pouvez être sauvés. 
\verse Paul et Barnabas eurent avec eux un débat et une vive discussion; et les frères décidèrent que Paul et Barnabas, et quelques-uns des leurs, monteraient à Jérusalem vers les apôtres et les anciens, pour traiter cette question. 
\verse Après avoir été accompagnés par l`Église, ils poursuivirent leur route à travers la Phénicie et la Samarie, racontant la conversion des païens, et ils causèrent une grande joie à tous les frères. 
\verse Arrivés à Jérusalem, ils furent reçus par l`Église, les apôtres et les anciens, et ils racontèrent tout ce que Dieu avait fait avec eux. 
\verse Alors quelques-uns du parti des pharisiens, qui avaient cru, se levèrent, en disant qu`il fallait circoncire les païens et exiger l`observation de la loi de Moïse. 
\verse Les apôtres et les anciens se réunirent pour examiner cette affaire. 
\verse Une grande discussion s`étant engagée, Pierre se leva, et leur dit: Hommes frères, vous savez que dès longtemps Dieu a fait un choix parmi vous, afin que, par ma bouche, les païens entendissent la parole de l`Évangile et qu`ils crussent. 
\verse Et Dieu, qui connaît les coeurs, leur a rendu témoignage, en leur donnant le Saint Esprit comme à nous; 
\verse il n`a fait aucune différence entre nous et eux, ayant purifié leurs coeurs par la foi. 
\verse Maintenant donc, pourquoi tentez-vous Dieu, en mettant sur le cou des disciples un joug que ni nos pères ni nous n`avons pu porter? 
\verse Mais c`est par la grâce du Seigneur Jésus que nous croyons être sauvés, de la même manière qu`eux. 
\verse Toute l`assemblée garda le silence, et l`on écouta Barnabas et Paul, qui racontèrent tous les miracles et les prodiges que Dieu avait faits par eux au milieu des païens. 
\verse Lorsqu`ils eurent cessé de parler, Jacques prit la parole, et dit: Hommes frères, écoutez-moi! 
\verse Simon a raconté comment Dieu a d`abord jeté les regards sur les nations pour choisir du milieu d`elles un peuple qui portât son nom. 
\verse Et avec cela s`accordent les paroles des prophètes, selon qu`il est écrit: 
\verse Après cela, je reviendrai, et je relèverai de sa chute la tente de David, J`en réparerai les ruines, et je la redresserai, 
\verse Afin que le reste des hommes cherche le Seigneur, Ainsi que toutes les nations sur lesquelles mon nom est invoqué, Dit le Seigneur, qui fait ces choses, 
\verse Et à qui elles sont connues de toute éternité. 
\verse C`est pourquoi je suis d`avis qu`on ne crée pas des difficultés à ceux des païens qui se convertissent à Dieu, 
\verse mais qu`on leur écrive de s`abstenir des souillures des idoles, de l`impudicité, des animaux étouffés et du sang. 
\verse Car, depuis bien des générations, Moïse a dans chaque ville des gens qui le prêchent, puisqu`on le lit tous les jours de sabbat dans les synagogues. 
\verse Alors il parut bon aux apôtres et aux anciens, et à toute l`Église, de choisir parmi eux et d`envoyer à Antioche, avec Paul et Barsabas, Jude appelé Barnabas et Silas, hommes considérés entre les frères. 
\verse Ils les chargèrent d`une lettre ainsi conçue: Les apôtres, les anciens, et les frères, aux frères d`entre les païens, qui sont à Antioche, en Syrie, et en Cilicie, salut! 
\verse Ayant appris que quelques hommes partis de chez nous, et auxquels nous n`avions donné aucun ordre, vous ont troublés par leurs discours et ont ébranlé vos âmes, 
\verse nous avons jugé à propos, après nous être réunis tous ensemble, de choisir des délégués et de vous les envoyer avec nos bien-aimés Barnabas et Paul, 
\verse ces hommes qui ont exposé leur vie pour le nom de notre Seigneur Jésus Christ. 
\verse Nous avons donc envoyé Jude et Silas, qui vous annonceront de leur bouche les mêmes choses. 
\verse Car il a paru bon au Saint Esprit et à nous de ne vous imposer d`autre charge que ce qui est nécessaire, 
\verse savoir, de vous abstenir des viandes sacrifiées aux idoles, du sang, des animaux étouffés, et de l`impudicité, choses contre lesquelles vous vous trouverez bien de vous tenir en garde. Adieu. 
\verse Eux donc, ayant pris congé de l`Église, allèrent à Antioche, où ils remirent la lettre à la multitude assemblée. 
\verse Après l`avoir lue, les frères furent réjouis de l`encouragement qu`elle leur apportait. 
\verse Jude et Silas, qui étaient eux-mêmes prophètes, les exhortèrent et les fortifièrent par plusieurs discours. 
\verse Au bout de quelque temps, les frères les laissèrent en paix retourner vers ceux qui les avaient envoyés. 
\verse Toutefois Silas trouva bon de rester. 
\verse Paul et Barnabas demeurèrent à Antioche, enseignant et annonçant, avec plusieurs autres, la bonne nouvelle de la parole du Seigneur. 
\verse Quelques jours s`écoulèrent, après lesquels Paul dit à Barnabas: Retournons visiter les frères dans toutes les villes où nous avons annoncé la parole du Seigneur, pour voir en quel état ils sont. 
\verse Barnabas voulait emmener aussi Jean, surnommé Marc; 
\verse mais Paul jugea plus convenable de ne pas prendre avec eux celui qui les avait quittés depuis la Pamphylie, et qui ne les avait point accompagnés dans leur oeuvre. 
\verse Ce dissentiment fut assez vif pour être cause qu`ils se séparèrent l`un de l`autre. Et Barnabas, prenant Marc avec lui, s`embarqua pour l`île de Chypre. 
\verse Paul fit choix de Silas, et partit, recommandé par les frères à la grâce du Seigneur. 
\verse Il parcourut la Syrie et la Cilicie, fortifiant les Églises. 

\chapter
\verse Il se rendit ensuite à Derbe et à Lystre. Et voici, il y avait là un disciple nommé Timothée, fils d`une femme juive fidèle et d`un père grec. 
\verse Les frères de Lystre et d`Icone rendaient de lui un bon témoignage. 
\verse Paul voulut l`emmener avec lui; et, l`ayant pris, il le circoncit, à cause des Juifs qui étaient dans ces lieux-là, car tous savaient que son père était grec. 
\verse En passant par les villes, ils recommandaient aux frères d`observer les décisions des apôtres et des anciens de Jérusalem. 
\verse Les Églises se fortifiaient dans la foi, et augmentaient en nombre de jour en jour. 
\verse Ayant été empêchés par le Saint Esprit d`annoncer la parole dans l`Asie, ils traversèrent la Phrygie et le pays de Galatie. 
\verse Arrivés près de la Mysie, ils se disposaient à entrer en Bithynie; mais l`Esprit de Jésus ne le leur permit pas. 
\verse Ils franchirent alors la Mysie, et descendirent à Troas. 
\verse Pendant la nuit, Paul eut une vision: un Macédonien lui apparut, et lui fit cette prière: Passe en Macédoine, secours-nous! 
\verse Après cette vision de Paul, nous cherchâmes aussitôt à nous rendre en Macédoine, concluant que le Seigneur nous appelait à y annoncer la bonne nouvelle. 
\verse Étant partis de Troas, nous fîmes voile directement vers la Samothrace, et le lendemain nous débarquâmes à Néapolis. 
\verse De là nous allâmes à Philippes, qui est la première ville d`un district de Macédoine, et une colonie. Nous passâmes quelques jours dans cette ville. 
\verse Le jour du sabbat, nous nous rendîmes, hors de la porte, vers une rivière, où nous pensions que se trouvait un lieu de prière. Nous nous assîmes, et nous parlâmes aux femmes qui étaient réunies. 
\verse L`une d`elles, nommée Lydie, marchande de pourpre, de la ville de Thyatire, était une femme craignant Dieu, et elle écoutait. Le Seigneur lui ouvrit le coeur, pour qu`elle fût attentive à ce que disait Paul. 
\verse Lorsqu`elle eut été baptisée, avec sa famille, elle nous fit cette demande: Si vous me jugez fidèle au Seigneur, entrez dans ma maison, et demeurez-y. Et elle nous pressa par ses instances. 
\verse Comme nous allions au lieu de prière, une servante qui avait un esprit de Python, et qui, en devinant, procurait un grand profit à ses maîtres, vint au-devant de nous, 
\verse et se mit à nous suivre, Paul et nous. Elle criait: Ces hommes sont les serviteurs du Dieu Très Haut, et ils vous annoncent la voie du salut. 
\verse Elle fit cela pendant plusieurs jours. Paul fatigué se retourna, et dit à l`esprit: Je t`ordonne, au nom de Jésus Christ, de sortir d`elle. Et il sortit à l`heure même. 
\verse Les maîtres de la servante, voyant disparaître l`espoir de leur gain, se saisirent de Paul et de Silas, et les traînèrent sur la place publique devant les magistrats. 
\verse Ils les présentèrent aux préteurs, en disant: Ces hommes troublent notre ville; 
\verse ce sont des Juifs, qui annoncent des coutumes qu`il ne nous est permis ni de recevoir ni de suivre, à nous qui sommes Romains. 
\verse La foule se souleva aussi contre eux, et les préteurs, ayant fait arracher leurs vêtements, ordonnèrent qu`on les battît de verges. 
\verse Après qu`on les eut chargés de coups, ils les jetèrent en prison, en recommandant au geôlier de les garder sûrement. 
\verse Le geôlier, ayant reçu cet ordre, les jeta dans la prison intérieure, et leur mit les ceps aux pieds. 
\verse Vers le milieu de la nuit, Paul et Silas priaient et chantaient les louanges de Dieu, et les prisonniers les entendaient. 
\verse Tout à coup il se fit un grand tremblement de terre, en sorte que les fondements de la prison furent ébranlés; au même instant, toutes les portes s`ouvrirent, et les liens de tous les prisonniers furent rompus. 
\verse Le geôlier se réveilla, et, lorsqu`il vit les portes de la prison ouvertes, il tira son épée et allait se tuer, pensant que les prisonniers s`étaient enfuis. 
\verse Mais Paul cria d`une voix forte: Ne te fais point de mal, nous sommes tous ici. 
\verse Alors le geôlier, ayant demandé de la lumière, entra précipitamment, et se jeta tout tremblant aux pieds de Paul et de Silas; 
\verse il les fit sortir, et dit: Seigneurs, que faut-il que je fasse pour être sauvé? 
\verse Paul et Silas répondirent: Crois au Seigneur Jésus, et tu seras sauvé, toi et ta famille. 
\verse Et ils lui annoncèrent la parole du Seigneur, ainsi qu`à tous ceux qui étaient dans sa maison. 
\verse Il les prit avec lui, à cette heure même de la nuit, il lava leurs plaies, et aussitôt il fut baptisé, lui et tous les siens. 
\verse Les ayant conduits dans son logement, il leur servit à manger, et il se réjouit avec toute sa famille de ce qu`il avait cru en Dieu. 
\verse Quand il fit jour, les préteurs envoyèrent les licteurs pour dire au geôlier: Relâche ces hommes. 
\verse Et le geôlier annonça la chose à Paul: Les préteurs ont envoyé dire qu`on vous relâchât; maintenant donc sortez, et allez en paix. 
\verse Mais Paul dit aux licteurs: Après nous avoir battus de verges publiquement et sans jugement, nous qui sommes Romains, ils nous ont jetés en prison, et maintenant ils nous font sortir secrètement! Il n`en sera pas ainsi. Qu`ils viennent eux-mêmes nous mettre en liberté. 
\verse Les licteurs rapportèrent ces paroles aux préteurs, qui furent effrayés en apprenant qu`ils étaient Romains. 
\verse Ils vinrent les apaiser, et ils les mirent en liberté, en les priant de quitter la ville. 
\verse Quand ils furent sortis de la prison, ils entrèrent chez Lydie, et, après avoir vu et exhorté les frères, ils partirent. 

\chapter
\verse Paul et Silas passèrent par Amphipolis et Apollonie, et ils arrivèrent à Thessalonique, où les Juifs avaient une synagogue. 
\verse Paul y entra, selon sa coutume. Pendant trois sabbats, il discuta avec eux, d`après les Écritures, 
\verse expliquant et établissant que le Christ devait souffrir et ressusciter des morts. Et Jésus que je vous annonce, disait-il, c`est lui qui est le Christ. 
\verse Quelques-uns d`entre eux furent persuadés, et se joignirent à Paul et à Silas, ainsi qu`une grande multitude de Grecs craignant Dieu, et beaucoup de femmes de qualité. 
\verse Mais les Juifs, jaloux prirent avec eux quelques méchants hommes de la populace, provoquèrent des attroupements, et répandirent l`agitation dans la ville. Ils se portèrent à la maison de Jason, et ils cherchèrent Paul et Silas, pour les amener vers le peuple. 
\verse Ne les ayant pas trouvés, ils traînèrent Jason et quelques frères devant les magistrats de la ville, en criant: Ces gens, qui ont bouleversé le monde, sont aussi venus ici, et Jason les a reçus. 
\verse Ils agissent tous contre les édits de César, disant qu`il y a un autre roi, Jésus. 
\verse Par ces paroles ils émurent la foule et les magistrats, 
\verse qui ne laissèrent aller Jason et les autres qu`après avoir obtenu d`eux une caution. 
\verse Aussitôt les frères firent partir de nuit Paul et Silas pour Bérée. Lorsqu`ils furent arrivés, ils entrèrent dans la synagogue des Juifs. 
\verse Ces Juifs avaient des sentiments plus nobles que ceux de Thessalonique; ils reçurent la parole avec beaucoup d`empressement, et ils examinaient chaque jour les Écritures, pour voir si ce qu`on leur disait était exact. 
\verse Plusieurs d`entre eux crurent, ainsi que beaucoup de femmes grecques de distinction, et beaucoup d`hommes. 
\verse Mais, quand les Juifs de Thessalonique surent que Paul annonçait aussi à Bérée la parole de Dieu, ils vinrent y agiter la foule. 
\verse Alors les frères firent aussitôt partir Paul du côté de la mer; Silas et Timothée restèrent à Bérée. 
\verse Ceux qui accompagnaient Paul le conduisirent jusqu`à Athènes. Puis ils s`en retournèrent, chargés de transmettre à Silas et à Timothée l`ordre de le rejoindre au plus tôt. 
\verse Comme Paul les attendait à Athènes, il sentait au dedans de lui son esprit s`irriter, à la vue de cette ville pleine d`idoles. 
\verse Il s`entretenait donc dans la synagogue avec les Juifs et les hommes craignant Dieu, et sur la place publique chaque jour avec ceux qu`il rencontrait. 
\verse Quelques philosophes épicuriens et stoïciens se mirent à parler avec lui. Et les uns disaient: Que veut dire ce discoureur? D`autres, l`entendant annoncer Jésus et la résurrection, disaient: Il semble qu`il annonce des divinités étrangères. 
\verse Alors ils le prirent, et le menèrent à l`Aréopage, en disant: Pourrions-nous savoir quelle est cette nouvelle doctrine que tu enseignes? 
\verse Car tu nous fais entendre des choses étranges. Nous voudrions donc savoir ce que cela peut être. 
\verse Or, tous les Athéniens et les étrangers demeurant à Athènes ne passaient leur temps qu`à dire ou à écouter des nouvelles. 
\verse Paul, debout au milieu de l`Aréopage, dit: Hommes Athéniens, je vous trouve à tous égards extrêmement religieux. 
\verse Car, en parcourant votre ville et en considérant les objets de votre dévotion, j`ai même découvert un autel avec cette inscription: A un dieu inconnu! Ce que vous révérez sans le connaître, c`est ce que je vous annonce. 
\verse Le Dieu qui a fait le monde et tout ce qui s`y trouve, étant le Seigneur du ciel et de la terre, n`habite point dans des temples faits de main d`homme; 
\verse il n`est point servi par des mains humaines, comme s`il avait besoin de quoi que ce soit, lui qui donne à tous la vie, la respiration, et toutes choses. 
\verse Il a fait que tous les hommes, sortis d`un seul sang, habitassent sur toute la surface de la terre, ayant déterminé la durée des temps et les bornes de leur demeure; 
\verse il a voulu qu`ils cherchassent le Seigneur, et qu`ils s`efforçassent de le trouver en tâtonnant, bien qu`il ne soit pas loin de chacun de nous, 
\verse car en lui nous avons la vie, le mouvement, et l`être. C`est ce qu`ont dit aussi quelques-uns de vos poètes: De lui nous sommes la race... 
\verse Ainsi donc, étant la race de Dieu, nous ne devons pas croire que la divinité soit semblable à de l`or, à de l`argent, ou à de la pierre, sculptés par l`art et l`industrie de l`homme. 
\verse Dieu, sans tenir compte des temps d`ignorance, annonce maintenant à tous les hommes, en tous lieux, qu`ils aient à se repentir, 
\verse parce qu`il a fixé un jour où il jugera le monde selon la justice, par l`homme qu`il a désigné, ce dont il a donné à tous une preuve certaine en le ressuscitant des morts... 
\verse Lorsqu`ils entendirent parler de résurrection des morts, les uns se moquèrent, et les autres dirent: Nous t`entendrons là-dessus une autre fois. 
\verse Ainsi Paul se retira du milieu d`eux. 
\verse Quelques-uns néanmoins s`attachèrent à lui et crurent, Denys l`aréopagite, une femme nommée Damaris, et d`autres avec eux. 

\chapter
\verse Après cela, Paul partit d`Athènes, et se rendit à Corinthe. 
\verse Il y trouva un Juif nommé Aquilas, originaire du Pont, récemment arrivé d`Italie avec sa femme Priscille, parce que Claude avait ordonné à tous les Juifs de sortir de Rome. Il se lia avec eux; 
\verse et, comme il avait le même métier, il demeura chez eux et y travailla: ils étaient faiseurs de tentes. 
\verse Paul discourait dans la synagogue chaque sabbat, et il persuadait des Juifs et des Grecs. 
\verse Mais quand Silas et Timothée furent arrivés de la Macédoine, il se donna tout entier à la parole, attestant aux Juifs que Jésus était le Christ. 
\verse Les Juifs faisant alors de l`opposition et se livrant à des injures, Paul secoua ses vêtements, et leur dit: Que votre sang retombe sur votre tête! J`en suis pur. Dès maintenant, j`irai vers les païens. 
\verse Et sortant de là, il entra chez un nommé Justus, homme craignant Dieu, et dont la maison était contiguë à la synagogue. 
\verse Cependant Crispus, le chef de la synagogue, crut au Seigneur avec toute sa famille. Et plusieurs Corinthiens, qui avaient entendu Paul, crurent aussi, et furent baptisés. 
\verse Le Seigneur dit à Paul en vision pendant la nuit: Ne crains point; mais parle, et ne te tais point, 
\verse Car je suis avec toi, et personne ne mettra la main sur toi pour te faire du mal: parle, car j`ai un peuple nombreux dans cette ville. 
\verse Il y demeura un an et six mois, enseignant parmi les Corinthiens la parole de Dieu. 
\verse Du temps que Gallion était proconsul de l`Achaïe, les Juifs se soulevèrent unanimement contre Paul, et le menèrent devant le tribunal, 
\verse en disant: Cet homme excite les gens à servir Dieu d`une manière contraire à la loi. 
\verse Paul allait ouvrir la bouche, lorsque Gallion dit aux Juifs: S`il s`agissait de quelque injustice ou de quelque méchante action, je vous écouterais comme de raison, ô Juifs; 
\verse mais, s`il s`agit de discussions sur une parole, sur des noms, et sur votre loi, cela vous regarde: je ne veux pas être juge de ces choses. 
\verse Et il les renvoya du tribunal. 
\verse Alors tous, se saisissant de Sosthène, le chef de la synagogue, le battirent devant le tribunal, sans que Gallion s`en mît en peine. 
\verse Paul resta encore assez longtemps à Corinthe. Ensuite il prit congé des frères, et s`embarqua pour la Syrie, avec Priscille et Aquilas, après s`être fait raser la tête à Cenchrées, car il avait fait un voeu. 
\verse Ils arrivèrent à Éphèse, et Paul y laissa ses compagnons. Étant entré dans la synagogue, il s`entretint avec les Juifs, 
\verse qui le prièrent de prolonger son séjour. 
\verse Mais il n`y consentit point, et il prit congé d`eux, en disant: Il faut absolument que je célèbre la fête prochaine à Jérusalem. Je reviendrai vers vous, si Dieu le veut. Et il partit d`Éphèse. 
\verse Étant débarqué à Césarée, il monta à Jérusalem, et, après avoir salué l`Église, il descendit à Antioche. 
\verse Lorsqu`il eut passé quelque temps à Antioche, Paul se mit en route, et parcourut successivement la Galatie et la Phrygie, fortifiant tous les disciples. 
\verse Un Juif nommé Apollos, originaire d`Alexandrie, homme éloquent et versé dans les Écritures, vint à Éphèse. 
\verse Il était instruit dans la voie du Seigneur, et, fervent d`esprit, il annonçait et enseignait avec exactitude ce qui concerne Jésus, bien qu`il ne connût que le baptême de Jean. 
\verse Il se mit à parler librement dans la synagogue. Aquilas et Priscille, l`ayant entendu, le prirent avec eux, et lui exposèrent plus exactement la voie de Dieu. 
\verse Comme il voulait passer en Achaïe, les frères l`y encouragèrent, et écrivirent aux disciples de le bien recevoir. Quand il fut arrivé, il se rendit, par la grâce de Dieu, très utile à ceux qui avaient cru; 
\verse Car il réfutait vivement les Juifs en public, démontrant par les Écritures que Jésus est le Christ. 

\chapter
\verse Pendant qu`Apollos était à Corinthe, Paul, après avoir parcouru les hautes provinces de l`Asie, arriva à Éphèse. Ayant rencontré quelques disciples, il leur dit: 
\verse Avez-vous reçu le Saint Esprit, quand vous avez cru? Ils lui répondirent: Nous n`avons pas même entendu dire qu`il y ait un Saint Esprit. 
\verse Il dit: De quel baptême avez-vous donc été baptisés? Et ils répondirent: Du baptême de Jean. 
\verse Alors Paul dit: Jean a baptisé du baptême de repentance, disant au peuple de croire en celui qui venait après lui, c`est-à-dire, en Jésus. 
\verse Sur ces paroles, ils furent baptisés au nom du Seigneur Jésus. 
\verse Lorsque Paul leur eut imposé les mains, le Saint Esprit vint sur eux, et ils parlaient en langues et prophétisaient. 
\verse Ils étaient en tout environ douze hommes. 
\verse Ensuite Paul entra dans la synagogue, où il parla librement. Pendant trois mois, il discourut sur les choses qui concernent le royaume de Dieu, s`efforçant de persuader ceux qui l`écoutaient. 
\verse Mais, comme quelques-uns restaient endurcis et incrédules, décriant devant la multitude la voie du Seigneur, il se retira d`eux, sépara les disciples, et enseigna chaque jour dans l`école d`un nommé Tyrannus. 
\verse Cela dura deux ans, de sorte que tous ceux qui habitaient l`Asie, Juifs et Grecs, entendirent la parole du Seigneur. 
\verse Et Dieu faisait des miracles extraordinaires par les mains de Paul, 
\verse au point qu`on appliquait sur les malades des linges ou des mouchoirs qui avaient touché son corps, et les maladies les quittaient, et les esprits malins sortaient. 
\verse Quelques exorcistes juifs ambulants essayèrent d`invoquer sur ceux qui avaient des esprits malins le nom du Seigneur Jésus, en disant: Je vous conjure par Jésus que Paul prêche! 
\verse Ceux qui faisaient cela étaient sept fils de Scéva, Juif, l`un des principaux sacrificateurs. 
\verse L`esprit malin leur répondit: Je connais Jésus, et je sais qui est Paul; mais vous, qui êtes-vous? 
\verse Et l`homme dans lequel était l`esprit malin s`élança sur eux, se rendit maître de tous deux, et les maltraita de telle sorte qu`ils s`enfuirent de cette maison nus et blessés. 
\verse Cela fut connu de tous les Juifs et de tous les Grecs qui demeuraient à Éphèse, et la crainte s`empara d`eux tous, et le nom du Seigneur Jésus était glorifié. 
\verse Plusieurs de ceux qui avaient cru venaient confesser et déclarer ce qu`ils avaient fait. 
\verse Et un certain nombre de ceux qui avaient exercé les arts magiques, ayant apporté leurs livres, les brûlèrent devant tout le monde: on en estima la valeur à cinquante mille pièces d`argent. 
\verse C`est ainsi que la parole du Seigneur croissait en puissance et en force. 
\verse Après que ces choses se furent passées, Paul forma le projet d`aller à Jérusalem, en traversant la Macédoine et l`Achaïe. Quand j`aurai été là, se disait-il, il faut aussi que je voie Rome. 
\verse Il envoya en Macédoine deux de ses aides, Timothée et Éraste, et il resta lui-même quelque temps encore en Asie. 
\verse Il survint, à cette époque, un grand trouble au sujet de la voie du Seigneur. 
\verse Un nommé Démétrius, orfèvre, fabriquait en argent des temples de Diane, et procurait à ses ouvriers un gain considérable. 
\verse Il les rassembla, avec ceux du même métier, et dit: O hommes, vous savez que notre bien-être dépend de cette industrie; 
\verse et vous voyez et entendez que, non seulement à Éphèse, mais dans presque toute l`Asie, ce Paul a persuadé et détourné une foule de gens, en disant que les dieux faits de main d`homme ne sont pas des dieux. 
\verse Le danger qui en résulte, ce n`est pas seulement que notre industrie ne tombe en discrédit; c`est encore que le temple de la grande déesse Diane ne soit tenu pour rien, et même que la majesté de celle qui est révérée dans toute l`Asie et dans le monde entier ne soit réduite à néant. 
\verse Ces paroles les ayant remplis de colère, ils se mirent à crier: Grande est la Diane des Éphésiens! 
\verse Toute la ville fut dans la confusion. Ils se précipitèrent tous ensemble au théâtre, entraînant avec eux Gaïus et Aristarque, Macédoniens, compagnons de voyage de Paul. 
\verse Paul voulait se présenter devant le peuple, mais les disciples l`en empêchèrent; 
\verse quelques-uns même des Asiarques, qui étaient ses amis, envoyèrent vers lui, pour l`engager à ne pas se rendre au théâtre. 
\verse Les uns criaient d`une manière, les autres d`une autre, car le désordre régnait dans l`assemblée, et la plupart ne savaient pas pourquoi ils s`étaient réunis. 
\verse Alors on fit sortir de la foule Alexandre, que les Juifs poussaient en avant; et Alexandre, faisant signe de la main, voulait parler au peuple. 
\verse Mais quand ils reconnurent qu`il était Juif, tous d`une seule voix crièrent pendant près de deux heures: Grande est la Diane des Éphésiens! 
\verse Cependant le secrétaire, ayant apaisé la foule, dit: Hommes Éphésiens, quel est celui qui ignore que la ville d`Éphèse est la gardienne du temple de la grande Diane et de son simulacre tombé du ciel? 
\verse Cela étant incontestable, vous devez vous calmer, et ne rien faire avec précipitation. 
\verse Car vous avez amené ces hommes, qui ne sont coupables ni de sacrilège, ni de blasphème envers notre déesse. 
\verse Si donc Démétrius et ses ouvriers ont à se plaindre de quelqu`un, il y a des jours d`audience et des proconsuls; qu`ils s`appellent en justice les uns les autres. 
\verse Et si vous avez en vue d`autres objets, ils se régleront dans une assemblée légale. 
\verse Nous risquons, en effet, d`être accusés de sédition pour ce qui s`est passé aujourd`hui, puisqu`il n`existe aucun motif qui nous permette de justifier cet attroupement. 
\verse (19:40b) Après ces paroles, il congédia l`assemblée. 

\chapter
\verse Lorsque le tumulte eut cessé, Paul réunit les disciples, et, après les avoir exhortés, prit congé d`eux, et partit pour aller en Macédoine. 
\verse Il parcourut cette contrée, en adressant aux disciples de nombreuses exhortations. 
\verse Puis il se rendit en Grèce, où il séjourna trois mois. Il était sur le point de s`embarquer pour la Syrie, quand les Juifs lui dressèrent des embûches. Alors il se décida à reprendre la route de la Macédoine. 
\verse Il avait pour l`accompagner jusqu`en Asie: Sopater de Bérée, fils de Pyrrhus, Aristarque et Second de Thessalonique, Gaïus de Derbe, Timothée, ainsi que Tychique et Trophime, originaires d`Asie. 
\verse Ceux-ci prirent les devants, et nous attendirent à Troas. 
\verse Pour nous, après les jours des pains sans levain, nous nous embarquâmes à Philippes, et, au bout de cinq jours, nous les rejoignîmes à Troas, où nous passâmes sept jours. 
\verse Le premier jour de la semaine, nous étions réunis pour rompre le pain. Paul, qui devait partir le lendemain, s`entretenait avec les disciples, et il prolongea son discours jusqu`à minuit. 
\verse Il y avait beaucoup de lampes dans la chambre haute où nous étions assemblés. 
\verse Or, un jeune homme nommé Eutychus, qui était assis sur la fenêtre, s`endormit profondément pendant le long discours de Paul; entraîné par le sommeil, il tomba du troisième étage en bas, et il fut relevé mort. 
\verse Mais Paul, étant descendu, se pencha sur lui et le prit dans ses bras, en disant: Ne vous troublez pas, car son âme est en lui. 
\verse Quand il fut remonté, il rompit le pain et mangea, et il parla longtemps encore jusqu`au jour. Après quoi il partit. 
\verse Le jeune homme fut ramené vivant, et ce fut le sujet d`une grande consolation. 
\verse Pour nous, nous précédâmes Paul sur le navire, et nous fîmes voile pour Assos, où nous avions convenu de le reprendre, parce qu`il devait faire la route à pied. 
\verse Lorsqu`il nous eut rejoints à Assos, nous le prîmes à bord, et nous allâmes à Mytilène. 
\verse De là, continuant par mer, nous arrivâmes le lendemain vis-à-vis de Chios. Le jour suivant, nous cinglâmes vers Samos, et le jour d`après nous vînmes à Milet. 
\verse Paul avait résolu de passer devant Éphèse sans s`y arrêter, afin de ne pas perdre de temps en Asie; car il se hâtait pour se trouver, si cela lui était possible, à Jérusalem le jour de la Pentecôte. 
\verse Cependant, de Milet Paul envoya chercher à Éphèse les anciens de l`Église. 
\verse Lorsqu`ils furent arrivés vers lui, il leur dit: Vous savez de quelle manière, depuis le premier jour où je suis entré en Asie, je me suis sans cesse conduit avec vous, 
\verse servant le Seigneur en toute humilité, avec larmes, et au milieu des épreuves que me suscitaient les embûches des Juifs. 
\verse Vous savez que je n`ai rien caché de ce qui vous était utile, et que je n`ai pas craint de vous prêcher et de vous enseigner publiquement et dans les maisons, 
\verse annonçant aux Juifs et aux Grecs la repentance envers Dieu et la foi en notre Seigneur Jésus Christ. 
\verse Et maintenant voici, lié par l`Esprit, je vais à Jérusalem, ne sachant pas ce qui m`y arrivera; 
\verse seulement, de ville en ville, l`Esprit Saint m`avertit que des liens et des tribulations m`attendent. 
\verse Mais je ne fais pour moi-même aucun cas de ma vie, comme si elle m`était précieuse, pourvu que j`accomplisse ma course avec joie, et le ministère que j`ai reçu du Seigneur Jésus, d`annoncer la bonne nouvelle de la grâce de Dieu. 
\verse Et maintenant voici, je sais que vous ne verrez plus mon visage, vous tous au milieu desquels j`ai passé en prêchant le royaume de Dieu. 
\verse C`est pourquoi je vous déclare aujourd`hui que je suis pur du sang de vous tous, 
\verse car je vous ai annoncé tout le conseil de Dieu, sans en rien cacher. 
\verse Prenez donc garde à vous-mêmes, et à tout le troupeau sur lequel le Saint Esprit vous a établis évêques, pour paître l`Église du Seigneur, qu`il s`est acquise par son propre sang. 
\verse Je sais qu`il s`introduira parmi vous, après mon départ, des loups cruels qui n`épargneront pas le troupeau, 
\verse et qu`il s`élèvera du milieu de vous des hommes qui enseigneront des choses pernicieuses, pour entraîner les disciples après eux. 
\verse Veillez donc, vous souvenant que, durant trois années, je n`ai cessé nuit et jour d`exhorter avec larmes chacun de vous. 
\verse Et maintenant je vous recommande à Dieu et à la parole de sa grâce, à celui qui peut édifier et donner l`héritage avec tous les sanctifiés. 
\verse Je n`ai désiré ni l`argent, ni l`or, ni les vêtements de personne. 
\verse Vous savez vous-mêmes que ces mains ont pourvu à mes besoins et à ceux des personnes qui étaient avec moi. 
\verse Je vous ai montré de toutes manières que c`est en travaillant ainsi qu`il faut soutenir les faibles, et se rappeler les paroles du Seigneur, qui a dit lui-même: Il y a plus de bonheur à donner qu`à recevoir. 
\verse Après avoir ainsi parlé, il se mit à genoux, et il pria avec eux tous. 
\verse Et tous fondirent en larmes, et, se jetant au cou de Paul, 
\verse ils l`embrassaient, affligés surtout de ce qu`il avait dit qu`ils ne verraient plus son visage. Et ils l`accompagnèrent jusqu`au navire. 

\chapter
\verse Nous nous embarquâmes, après nous être séparés d`eux, et nous allâmes directement à Cos, le lendemain à Rhodes, et de là à Patara. 
\verse Et ayant trouvé un navire qui faisait la traversée vers la Phénicie, nous montâmes et partîmes. 
\verse Quand nous fûmes en vue de l`île de Chypre, nous la laissâmes à gauche, poursuivant notre route du côté de la Syrie, et nous abordâmes à Tyr, où le bâtiment devait décharger sa cargaison. 
\verse Nous trouvâmes les disciples, et nous restâmes là sept jours. Les disciples, poussés par l`Esprit, disaient à Paul de ne pas monter à Jérusalem. 
\verse Mais, lorsque nous fûmes au terme des sept jours, nous nous acheminâmes pour partir, et tous nous accompagnèrent avec leur femme et leurs enfants jusque hors de la ville. Nous nous mîmes à genoux sur le rivage, et nous priâmes. 
\verse Puis, ayant pris congé les uns des autres, nous montâmes sur le navire, et ils retournèrent chez eux. 
\verse Achevant notre navigation, nous allâmes de Tyr à Ptolémaïs, où nous saluâmes les frères, et passâmes un jour avec eux. 
\verse Nous partîmes le lendemain, et nous arrivâmes à Césarée. Étant entrés dans la maison de Philippe l`évangéliste, qui était l`un des sept, nous logeâmes chez lui. 
\verse Il avait quatre filles vierges qui prophétisaient. 
\verse Comme nous étions là depuis plusieurs jours, un prophète, nommé Agabus, descendit de Judée, 
\verse et vint nous trouver. Il prit la ceinture de Paul, se lia les pieds et les mains, et dit: Voici ce que déclare le Saint Esprit: L`homme à qui appartient cette ceinture, les Juifs le lieront de la même manière à Jérusalem, et le livreront entre les mains des païens. 
\verse Quand nous entendîmes cela, nous et ceux de l`endroit, nous priâmes Paul de ne pas monter à Jérusalem. 
\verse Alors il répondit: Que faites-vous, en pleurant et en me brisant le coeur? Je suis prêt, non seulement à être lié, mais encore à mourir à Jérusalem pour le nom du Seigneur Jésus. 
\verse Comme il ne se laissait pas persuader, nous n`insistâmes pas, et nous dîmes: Que la volonté du Seigneur se fasse! 
\verse Après ces jours-là, nous fîmes nos préparatifs, et nous montâmes à Jérusalem. 
\verse Quelques disciples de Césarée vinrent aussi avec nous, et nous conduisirent chez un nommé Mnason, de l`île de Chypre, ancien disciple, chez qui nous devions loger. 
\verse Lorsque nous arrivâmes à Jérusalem, les frères nous reçurent avec joie. 
\verse Le lendemain, Paul se rendit avec nous chez Jacques, et tous les anciens s`y réunirent. 
\verse Après les avoir salués, il raconta en détail ce que Dieu avait fait au milieu des païens par son ministère. 
\verse Quand ils l`eurent entendu, ils glorifièrent Dieu. Puis ils lui dirent: Tu vois, frère, combien de milliers de Juifs ont cru, et tous sont zélés pour la loi. 
\verse Or, ils ont appris que tu enseignes à tous les Juifs qui sont parmi les païens à renoncer à Moïse, leur disant de ne pas circoncire les enfants et de ne pas se conformer aux coutumes. 
\verse Que faire donc? Sans aucun doute la multitude se rassemblera, car on saura que tu es venu. 
\verse C`est pourquoi fais ce que nous allons te dire. Il y a parmi nous quatre hommes qui ont fait un voeu; 
\verse prends-les avec toi, purifie-toi avec eux, et pourvois à leur dépense, afin qu`ils se rasent la tête. Et ainsi tous sauront que ce qu`ils ont entendu dire sur ton compte est faux, mais que toi aussi tu te conduis en observateur de la loi. 
\verse A l`égard des païens qui ont cru, nous avons décidé et nous leur avons écrit qu`ils eussent à s`abstenir des viandes sacrifiées aux idoles, du sang, des animaux étouffés, et de l`impudicité. 
\verse Alors Paul prit ces hommes, se purifia, et entra le lendemain dans le temple avec eux, pour annoncer à quel jour la purification serait accomplie et l`offrande présentée pour chacun d`eux. 
\verse Sur la fin des sept jours, les Juifs d`Asie, ayant vu Paul dans le temple, soulevèrent toute la foule, et mirent la main sur lui, 
\verse en criant: Hommes Israélites, au secours! Voici l`homme qui prêche partout et à tout le monde contre le peuple, contre la loi et contre ce lieu; il a même introduit des Grecs dans le temple, et a profané ce saint lieu. 
\verse Car ils avaient vu auparavant Trophime d`Éphèse avec lui dans la ville, et ils croyaient que Paul l`avait fait entrer dans le temple. 
\verse Toute la ville fut émue, et le peuple accourut de toutes parts. Ils se saisirent de Paul, et le traînèrent hors du temple, dont les portes furent aussitôt fermées. 
\verse Comme ils cherchaient à le tuer, le bruit vint au tribun de la cohorte que tout Jérusalem était en confusion. 
\verse A l`instant il prit des soldats et des centeniers, et courut à eux. Voyant le tribun et les soldats, ils cessèrent de frapper Paul. 
\verse Alors le tribun s`approcha, se saisit de lui, et le fit lier de deux chaînes. Puis il demanda qui il était, et ce qu`il avait fait. 
\verse Mais dans la foule les uns criaient d`une manière, les autres d`une autre; ne pouvant donc rien apprendre de certain, à cause du tumulte, il ordonna de le mener dans la forteresse. 
\verse Lorsque Paul fut sur les degrés, il dut être porté par les soldats, à cause de la violence de la foule; 
\verse car la multitude du peuple suivait, en criant: Fais-le mourir! 
\verse Au moment d`être introduit dans la forteresse, Paul dit au tribun: M`est-il permis de te dire quelque chose? Le tribun répondit: Tu sais le grec? 
\verse Tu n`es donc pas cet Égyptien qui s`est révolté dernièrement, et qui a emmené dans le désert quatre mille brigands? 
\verse Je suis Juif, reprit Paul, de Tarse en Cilicie, citoyen d`une ville qui n`est pas sans importance. Permets-moi, je te prie, de parler au peuple. 
\verse Le tribun le lui ayant permis, Paul, debout sur les degrés, fit signe de la main au peuple. Un profond silence s`établit, et Paul, parlant en langue hébraïque, dit: 

\chapter
\verse Hommes frères et pères, écoutez ce que j`ai maintenant à vous dire pour ma défense! 
\verse Lorsqu`ils entendirent qu`il leur parlait en langue hébraïque, ils redoublèrent de silence. Et Paul dit: 
\verse je suis Juif, né à Tarse en Cilicie; mais j`ai été élevé dans cette ville-ci, et instruit aux pieds de Gamaliel dans la connaissance exacte de la loi de nos pères, étant plein de zèle pour Dieu, comme vous l`êtes tous aujourd`hui. 
\verse J`ai persécuté à mort cette doctrine, liant et mettant en prison hommes et femmes. 
\verse Le souverain sacrificateur et tout le collège des anciens m`en sont témoins. J`ai même reçu d`eux des lettres pour les frères de Damas, où je me rendis afin d`amener liés à Jérusalem ceux qui se trouvaient là et de les faire punir. 
\verse Comme j`étais en chemin, et que j`approchais de Damas, tout à coup, vers midi, une grande lumière venant du ciel resplendit autour de moi. 
\verse Je tombai par terre, et j`entendis une voix qui me disait: Saul, Saul, pourquoi me persécutes-tu? 
\verse Je répondis: Qui es-tu, Seigneur? Et il me dit: Je suis Jésus de Nazareth, que tu persécutes. 
\verse Ceux qui étaient avec moi virent bien la lumière, mais ils n`entendirent pas la voix de celui qui parlait. Alors je dis: Que ferai-je, Seigneur? 
\verse Et le Seigneur me dit: Lève-toi, va à Damas, et là on te dira tout ce que tu dois faire. 
\verse Comme je ne voyais rien, à cause de l`éclat de cette lumière, ceux qui étaient avec moi me prirent par la main, et j`arrivai à Damas. 
\verse Or, un nommé Ananias, homme pieux selon la loi, et de qui tous les Juifs demeurant à Damas rendaient un bon témoignage, vint se présenter à moi, 
\verse et me dit: Saul, mon frère, recouvre la vue. Au même instant, je recouvrai la vue et je le regardai. 
\verse Il dit: Le Dieu de nos pères t`a destiné à connaître sa volonté, à voir le Juste, et à entendre les paroles de sa bouche; 
\verse car tu lui serviras de témoin, auprès de tous les hommes, des choses que tu as vues et entendues. 
\verse Et maintenant, que tardes-tu? Lève-toi, sois baptisé, et lavé de tes péchés, en invoquant le nom du Seigneur. 
\verse De retour à Jérusalem, comme je priais dans le temple, je fus ravi en extase, 
\verse et je vis le Seigneur qui me disait: Hâte-toi, et sors promptement de Jérusalem, parce qu`ils ne recevront pas ton témoignage sur moi. 
\verse Et je dis: Seigneur, ils savent eux-mêmes que je faisais mettre en prison et battre de verges dans les synagogues ceux qui croyaient en toi, et que, 
\verse lorsqu`on répandit le sang d`Étienne, ton témoin, j`étais moi-même présent, joignant mon approbation à celle des autres, et gardant les vêtements de ceux qui le faisaient mourir. 
\verse Alors il me dit: Va, je t`enverrai au loin vers les nations... 
\verse Ils l`écoutèrent jusqu`à cette parole. Mais alors ils élevèrent la voix, disant: Ote de la terre un pareil homme! Il n`est pas digne de vivre. 
\verse Et ils poussaient des cris, jetaient leurs vêtements, lançaient de la poussière en l`air. 
\verse Le tribun commanda de faire entrer Paul dans la forteresse, et de lui donner la question par le fouet, afin de savoir pour quel motif ils criaient ainsi contre lui. 
\verse Lorsqu`on l`eut exposé au fouet, Paul dit au centenier qui était présent: Vous est-il permis de battre de verges un citoyen romain, qui n`est pas même condamné? 
\verse A ces mots, le centenier alla vers le tribun pour l`avertir, disant: Que vas-tu faire? Cet homme est Romain. 
\verse Et le tribun, étant venu, dit à Paul: Dis-moi, es-tu Romain? Oui, répondit-il. 
\verse Le tribun reprit: C`est avec beaucoup d`argent que j`ai acquis ce droit de citoyen. Et moi, dit Paul, je l`ai par ma naissance. 
\verse Aussitôt ceux qui devaient lui donner la question se retirèrent, et le tribun, voyant que Paul était Romain, fut dans la crainte parce qu`il l`avait fait lier. 
\verse Le lendemain, voulant savoir avec certitude de quoi les Juifs l`accusaient, le tribun lui fit ôter ses liens, et donna l`ordre aux principaux sacrificateurs et à tout le sanhédrin de se réunir; puis, faisant descendre Paul, il le plaça au milieu d`eux. 

\chapter
\verse Paul, les regards fixés sur le sanhédrin, dit: Hommes frères, c`est en toute bonne conscience que je me suis conduit jusqu`à ce jour devant Dieu... 
\verse Le souverain sacrificateur Ananias ordonna à ceux qui étaient près de lui de le frapper sur la bouche. 
\verse Alors Paul lui dit: Dieu te frappera, muraille blanchie! Tu es assis pour me juger selon la loi, et tu violes la loi en ordonnant qu`on me frappe! 
\verse Ceux qui étaient près de lui dirent: Tu insultes le souverain sacrificateur de Dieu! 
\verse Et Paul dit: Je ne savais pas, frères, que ce fût le souverain sacrificateur; car il est écrit: Tu ne parleras pas mal du chef de ton peuple. 
\verse Paul, sachant qu`une partie de l`assemblée était composée de sadducéens et l`autre de pharisiens, s`écria dans le sanhédrin: Hommes frères, je suis pharisien, fils de pharisien; c`est à cause de l`espérance et de la résurrection des morts que je suis mis en jugement. 
\verse Quand il eut dit cela, il s`éleva une discussion entre les pharisiens et les sadducéens, et l`assemblée se divisa. 
\verse Car les sadducéens disent qu`il n`y a point de résurrection, et qu`il n`existe ni ange ni esprit, tandis que les pharisiens affirment les deux choses. 
\verse Il y eut une grande clameur, et quelques scribes du parti des pharisiens, s`étant levés, engagèrent un vif débat, et dirent: Nous ne trouvons aucun mal en cet homme; peut-être un esprit ou un ange lui a-t-il parlé. 
\verse Comme la discorde allait croissant, le tribun craignant que Paul ne fût mis en pièces par ces gens, fit descendre les soldats pour l`enlever du milieu d`eux et le conduire à la forteresse. 
\verse La nuit suivante, le Seigneur apparut à Paul, et dit: Prends courage; car, de même que tu as rendu témoignage de moi dans Jérusalem, il faut aussi que tu rendes témoignage dans Rome. 
\verse Quand le jour fut venu, les Juifs formèrent un complot, et firent des imprécations contre eux-mêmes, en disant qu`ils s`abstiendraient de manger et de boire jusqu`à ce qu`ils eussent tué Paul. 
\verse Ceux qui formèrent ce complot étaient plus de quarante, 
\verse et ils allèrent trouver les principaux sacrificateurs et les anciens, auxquels ils dirent: Nous nous sommes engagés, avec des imprécations contre nous-mêmes, à ne rien manger jusqu`à ce que nous ayons tué Paul. 
\verse Vous donc, maintenant, adressez-vous avec le sanhédrin au tribun, pour qu`il l`amène devant vous, comme si vous vouliez examiner sa cause plus exactement; et nous, avant qu`il approche, nous sommes prêts à le tuer. 
\verse Le fils de la soeur de Paul, ayant eu connaissance du guet-apens, alla dans la forteresse en informer Paul. 
\verse Paul appela l`un des centeniers, et dit: Mène ce jeune homme vers le tribun, car il a quelque chose à lui rapporter. 
\verse Le centenier prit le jeune homme avec lui, le conduisit vers le tribun, et dit: Le prisonnier Paul m`a appelé, et il m`a prié de t`amener ce jeune homme, qui a quelque chose à te dire. 
\verse Le tribun, prenant le jeune homme par la main, et se retirant à l`écart, lui demanda: Qu`as-tu à m`annoncer? 
\verse Il répondit: Les Juifs sont convenus de te prier d`amener Paul demain devant le sanhédrin, comme si tu devais t`enquérir de lui plus exactement. 
\verse Ne les écoute pas, car plus de quarante d`entre eux lui dressent un guet-apens, et se sont engagés, avec des imprécations contre eux-mêmes, à ne rien manger ni boire jusqu`à ce qu`ils l`aient tué; maintenant ils sont prêts, et n`attendent que ton consentement. 
\verse Le tribun renvoya le jeune homme, après lui avoir recommandé de ne parler à personne de ce rapport qu`il lui avait fait. 
\verse Ensuite il appela deux des centeniers, et dit: Tenez prêts, dès la troisième heure de la nuit, deux cents soldats, soixante-dix cavaliers et deux cents archers, pour aller jusqu`à Césarée. 
\verse Qu`il y ait aussi des montures pour Paul, afin qu`on le mène sain et sauf au gouverneur Félix. 
\verse Il écrivit une lettre ainsi conçue: 
\verse Claude Lysias au très excellent gouverneur Félix, salut! 
\verse Cet homme, dont les Juifs s`étaient saisis, allait être tué par eux, lorsque je survins avec des soldats et le leur enlevai, ayant appris qu`il était Romain. 
\verse Voulant connaître le motif pour lequel ils l`accusaient, je l`amenai devant leur sanhédrin. 
\verse J`ai trouvé qu`il était accusé au sujet de questions relatives à leur loi, mais qu`il n`avait commis aucun crime qui mérite la mort ou la prison. 
\verse Informé que les Juifs lui dressaient des embûches, je te l`ai aussitôt envoyé, en faisant savoir à ses accusateurs qu`ils eussent à s`adresser eux-mêmes à toi. Adieu. 
\verse Les soldats, selon l`ordre qu`ils avaient reçu, prirent Paul, et le conduisirent pendant la nuit jusqu`à Antipatris. 
\verse Le lendemain, laissant les cavaliers poursuivre la route avec lui, ils retournèrent à la forteresse. 
\verse Arrivés à Césarée, les cavaliers remirent la lettre au gouverneur, et lui présentèrent Paul. 
\verse Le gouverneur, après avoir lu la lettre, demanda de quelle province était Paul. Ayant appris qu`il était de la Cilicie: 
\verse Je t`entendrai, dit-il, quand tes accusateurs seront venus. Et il ordonna qu`on le gardât dans le prétoire d`Hérode. 

\chapter
\verse Cinq jours après, arriva le souverain sacrificateur Ananias, avec des anciens et un orateur nommé Tertulle. Ils portèrent plainte au gouverneur contre Paul. 
\verse Paul fut appelé, et Tertulle se mit à l`accuser, en ces termes: 
\verse Très excellent Félix, tu nous fais jouir d`une paix profonde, et cette nation a obtenu de salutaires réformes par tes soins prévoyants; c`est ce que nous reconnaissons en tout et partout avec une entière gratitude. 
\verse Mais, pour ne pas te retenir davantage, je te prie d`écouter, dans ta bonté, ce que nous avons à dire en peu de mots. 
\verse Nous avons trouvé cet homme, qui est une peste, qui excite des divisions parmi tous les Juifs du monde, qui est chef de la secte des Nazaréens, 
\verse et qui même a tenté de profaner le temple. Et nous l`avons arrêté. Nous avons voulu le juger selon notre loi; 
\verse mais le tribun Lysias étant survenu, l`a arraché de nos mains avec une grande violence, 
\verse en ordonnant à ses accusateurs de venir devant toi. Tu pourras toi-même, en l`interrogeant, apprendre de lui tout ce dont nous l`accusons. 
\verse Les Juifs se joignirent à l`accusation, soutenant que les choses étaient ainsi. 
\verse Après que le gouverneur lui eut fait signe de parler, Paul répondit: Sachant que, depuis plusieurs années, tu es juge de cette nation, c`est avec confiance que je prends la parole pour défendre ma cause. 
\verse Il n`y a pas plus de douze jours, tu peux t`en assurer, que je suis monté à Jérusalem pour adorer. 
\verse On ne m`a trouvé ni dans le temple, ni dans les synagogues, ni dans la ville, disputant avec quelqu`un, ou provoquant un rassemblement séditieux de la foule. 
\verse Et ils ne sauraient prouver ce dont ils m`accusent maintenant. 
\verse Je t`avoue bien que je sers le Dieu de mes pères selon la voie qu`ils appellent une secte, croyant tout ce qui est écrit dans la loi et dans les prophètes, 
\verse et ayant en Dieu cette espérance, comme ils l`ont eux-mêmes, qu`il y aura une résurrection des justes et des injustes. 
\verse C`est pourquoi je m`efforce d`avoir constamment une conscience sans reproche devant Dieu et devant les hommes. 
\verse Après une absence de plusieurs années, je suis venu pour faire des aumônes à ma nation, et pour présenter des offrandes. 
\verse C`est alors que quelques Juifs d`Asie m`ont trouvé purifié dans le temple, sans attroupement ni tumulte. 
\verse C`était à eux de paraître en ta présence et de se porter accusateurs, s`ils avaient quelque chose contre moi. 
\verse Ou bien, que ceux-ci déclarent de quel crime ils m`ont trouvé coupable, lorsque j`ai comparu devant le sanhédrin, 
\verse à moins que ce ne soit uniquement de ce cri que j`ai fait entendre au milieu d`eux: C`est à cause de la résurrection des morts que je suis aujourd`hui mis en jugement devant vous. 
\verse Félix, qui savait assez exactement ce qui concernait cette doctrine, les ajourna, en disant: Quand le tribun Lysias sera venu, j`examinerai votre affaire. 
\verse Et il donna l`ordre au centenier de garder Paul, en lui laissant une certaine liberté, et en n`empêchant aucun des siens de lui rendre des services. 
\verse Quelques jours après, Félix vint avec Drusille, sa femme, qui était Juive, et il fit appeler Paul. Il l`entendit sur la foi en Christ. 
\verse Mais, comme Paul discourait sur la justice, sur la tempérance, et sur le jugement à venir, Félix, effrayé, dit: Pour le moment retire-toi; quand j`en trouverai l`occasion, je te rappellerai. 
\verse Il espérait en même temps que Paul lui donnerait de l`argent; aussi l`envoyait-il chercher assez fréquemment, pour s`entretenir avec lui. 
\verse Deux ans s`écoulèrent ainsi, et Félix eut pour successeur Porcius Festus. Dans le désir de plaire aux Juifs, Félix laissa Paul en prison. 

\chapter
\verse Festus, étant arrivé dans la province, monta trois jours après de Césarée à Jérusalem. 
\verse Les principaux sacrificateurs et les principaux d`entre les Juifs lui portèrent plainte contre Paul. Ils firent des instances auprès de lui, et, dans des vues hostiles, 
\verse lui demandèrent comme une faveur qu`il le fît venir à Jérusalem. Ils préparaient un guet-apens, pour le tuer en chemin. 
\verse Festus répondit que Paul était gardé à Césarée, et que lui-même devait partir sous peu. 
\verse Que les principaux d`entre vous descendent avec moi, dit-il, et s`il y a quelque chose de coupable en cet homme, qu`ils l`accusent. 
\verse Festus ne passa que huit à dix jours parmi eux, puis il descendit à Césarée. Le lendemain, s`étant assis sur son tribunal, il donna l`ordre qu`on amenât Paul. 
\verse Quand il fut arrivé, les Juifs qui étaient venus de Jérusalem l`entourèrent, et portèrent contre lui de nombreuses et graves accusations, qu`ils n`étaient pas en état de prouver. 
\verse Paul entreprit sa défense, en disant: Je n`ai rien fait de coupable, ni contre la loi des Juifs, ni contre le temple, ni contre César. 
\verse Festus, désirant plaire aux Juifs, répondit à Paul: Veux-tu monter à Jérusalem, et y être jugé sur ces choses en ma présence? 
\verse Paul dit: C`est devant le tribunal de César que je comparais, c`est là que je dois être jugé. Je n`ai fait aucun tort aux Juifs, comme tu le sais fort bien. 
\verse Si j`ai commis quelque injustice, ou quelque crime digne de mort, je ne refuse pas de mourir; mais, si les choses dont ils m`accusent sont fausses, personne n`a le droit de me livrer à eux. J`en appelle à César. 
\verse Alors Festus, après avoir délibéré avec le conseil, répondit: Tu en as appelé à César; tu iras devant César. 
\verse Quelques jours après, le roi Agrippa et Bérénice arrivèrent à Césarée, pour saluer Festus. 
\verse Comme ils passèrent là plusieurs jours, Festus exposa au roi l`affaire de Paul, et dit: Félix a laissé prisonnier un homme 
\verse contre lequel, lorsque j`étais à Jérusalem, les principaux sacrificateurs et les anciens des Juifs ont porté plainte, en demandant sa condamnation. 
\verse Je leur ai répondu que ce n`est pas la coutume des Romains de livrer un homme avant que l`inculpé ait été mis en présence de ses accusateurs, et qu`il ait eu la faculté de se défendre sur les choses dont on l`accuse. 
\verse Ils sont donc venus ici, et, sans différer, je m`assis le lendemain sur mon tribunal, et je donnai l`ordre qu`on amenât cet homme. 
\verse Les accusateurs, s`étant présentés, ne lui imputèrent rien de ce que je supposais; 
\verse ils avaient avec lui des discussions relatives à leur religion particulière, et à un certain Jésus qui est mort, et que Paul affirmait être vivant. 
\verse Ne sachant quel parti prendre dans ce débat, je lui demandai s`il voulait aller à Jérusalem, et y être jugé sur ces choses. 
\verse Mais Paul en ayant appelé, pour que sa cause fût réservée à la connaissance de l`empereur, j`ai ordonné qu`on le gardât jusqu`à ce que je l`envoyasse à César. 
\verse Agrippa dit à Festus: Je voudrais aussi entendre cet homme. Demain, répondit Festus, tu l`entendras. 
\verse Le lendemain donc, Agrippa et Bérénice vinrent en grande pompe, et entrèrent dans le lieu de l`audience avec les tribuns et les principaux de la ville. Sur l`ordre de Festus, Paul fut amené. 
\verse Alors Festus dit: Roi Agrippa, et vous tous qui êtes présents avec nous, vous voyez cet homme au sujet duquel toute la multitude des Juifs s`est adressée à moi, soit à Jérusalem, soit ici, en s`écriant qu`il ne devait plus vivre. 
\verse Pour moi, ayant reconnu qu`il n`a rien fait qui mérite la mort, et lui-même en ayant appelé à l`empereur, j`ai résolu de le faire partir. 
\verse Je n`ai rien de certain à écrire à l`empereur sur son compte; c`est pourquoi je l`ai fait paraître devant vous, et surtout devant toi, roi Agrippa, afin de savoir qu`écrire, après qu`il aura été examiné. 
\verse Car il me semble absurde d`envoyer un prisonnier sans indiquer de quoi on l`accuse. 

\chapter
\verse Agrippa dit à Paul: Il t`est permis de parler pour ta défense. Et Paul, ayant étendu la main, se justifia en ces termes: 
\verse Je m`estime heureux, roi Agrippa, d`avoir aujourd`hui à me justifier devant toi de toutes les choses dont je suis accusé par les Juifs, 
\verse car tu connais parfaitement leurs coutumes et leurs discussions. Je te prie donc de m`écouter avec patience. 
\verse Ma vie, dès les premiers temps de ma jeunesse, est connue de tous les Juifs, puisqu`elle s`est passée à Jérusalem, au milieu de ma nation. 
\verse Ils savent depuis longtemps, s`ils veulent le déclarer, que j`ai vécu pharisien, selon la secte la plus rigide de notre religion. 
\verse Et maintenant, je suis mis en jugement parce que j`espère l`accomplissement de la promesse que Dieu a faite à nos pères, 
\verse et à laquelle aspirent nos douze tribus, qui servent Dieu continuellement nuit et jour. C`est pour cette espérance, ô roi, que je suis accusé par des Juifs! 
\verse Quoi! vous semble-t-il incroyable que Dieu ressuscite les morts? 
\verse Pour moi, j`avais cru devoir agir vigoureusement contre le nom de Jésus de Nazareth. 
\verse C`est ce que j`ai fait à Jérusalem. J`ai jeté en prison plusieurs des saints, ayant reçu ce pouvoir des principaux sacrificateurs, et, quand on les mettait à mort, je joignais mon suffrage à celui des autres. 
\verse je les ai souvent châtiés dans toutes les synagogues, et je les forçais à blasphémer. Dans mes excès de fureur contre eux, je les persécutais même jusque dans les villes étrangères. 
\verse C`est dans ce but que je me rendis à Damas, avec l`autorisation et la permission des principaux sacrificateurs. 
\verse Vers le milieu du jour, ô roi, je vis en chemin resplendir autour de moi et de mes compagnons une lumière venant du ciel, et dont l`éclat surpassait celui du soleil. 
\verse Nous tombâmes tous par terre, et j`entendis une voix qui me disait en langue hébraïque: Saul, Saul, pourquoi me persécutes-tu? Il te serait dur de regimber contre les aiguillons. 
\verse Je répondis: Qui es-tu, Seigneur? Et le Seigneur dit: Je suis Jésus que tu persécutes. 
\verse Mais lève-toi, et tiens-toi sur tes pieds; car je te suis apparu pour t`établir ministre et témoin des choses que tu as vues et de celles pour lesquelles je t`apparaîtrai. 
\verse Je t`ai choisi du milieu de ce peuple et du milieu des païens, vers qui je t`envoie, 
\verse afin que tu leur ouvres les yeux, pour qu`ils passent des ténèbres à la lumière et de la puissance de Satan à Dieu, pour qu`ils reçoivent, par la foi en moi, le pardon des péchés et l`héritage avec les sanctifiés. 
\verse En conséquence, roi Agrippa, je n`ai point résisté à la vision céleste: 
\verse à ceux de Damas d`abord, puis à Jérusalem, dans toute la Judée, et chez les païens, j`ai prêché la repentance et la conversion à Dieu, avec la pratique d`oeuvres dignes de la repentance. 
\verse Voilà pourquoi les Juifs se sont saisis de moi dans le temple, et ont tâché de me faire périr. 
\verse Mais, grâce au secours de Dieu, j`ai subsisté jusqu`à ce jour, rendant témoignage devant les petits et les grands, sans m`écarter en rien de ce que les prophètes et Moïse ont déclaré devoir arriver, 
\verse savoir que le Christ souffrirait, et que, ressuscité le premier d`entre les morts, il annoncerait la lumière au peuple et aux nations. 
\verse Comme il parlait ainsi pour sa justification, Festus dit à haute voix: Tu es fou, Paul! Ton grand savoir te fait déraisonner. 
\verse Je ne suis point fou, très excellent Festus, répliqua Paul; ce sont, au contraire, des paroles de vérité et de bon sens que je prononce. 
\verse Le roi est instruit de ces choses, et je lui en parle librement; car je suis persuadé qu`il n`en ignore aucune, puisque ce n`est pas en cachette qu`elles se sont passées. 
\verse Crois-tu aux prophètes, roi Agrippa?... Je sais que tu y crois. 
\verse Et Agrippa dit à Paul: Tu vas bientôt me persuader de devenir chrétien! 
\verse Paul répondit: Que ce soit bientôt ou que ce soit tard, plaise à Dieu que non seulement toi, mais encore tous ceux qui m`écoutent aujourd`hui, vous deveniez tels que je suis, à l`exception de ces liens! 
\verse Le roi, le gouverneur, Bérénice, et tous ceux qui étaient assis avec eux se levèrent, 
\verse et, en se retirant, ils se disaient les uns aux autres: Cet homme n`a rien fait qui mérite la mort ou la prison. 
\verse Et Agrippa dit à Festus: Cet homme pouvait être relâché, s`il n`en eût pas appelé à César. 

\chapter
\verse Lorsqu`il fut décidé que nous nous embarquerions pour l`Italie, on remit Paul et quelques autres prisonniers à un centenier de la cohorte Auguste, nommé Julius. 
\verse Nous montâmes sur un navire d`Adramytte, qui devait côtoyer l`Asie, et nous partîmes, ayant avec nous Aristarque, Macédonien de Thessalonique. 
\verse Le jour suivant, nous abordâmes à Sidon; et Julius, qui traitait Paul avec bienveillance, lui permit d`aller chez ses amis et de recevoir leurs soins. 
\verse Partis de là, nous longeâmes l`île de Chypre, parce que les vents étaient contraires. 
\verse Après avoir traversé la mer qui baigne la Cilicie et la Pamphylie, nous arrivâmes à Myra en Lycie. 
\verse Et là, le centenier, ayant trouvé un navire d`Alexandrie qui allait en Italie, nous y fit monter. 
\verse Pendant plusieurs jours nous naviguâmes lentement, et ce ne fut pas sans difficulté que nous atteignîmes la hauteur de Cnide, où le vent ne nous permit pas d`aborder. Nous passâmes au-dessous de l`île de Crète, du côté de Salmone. 
\verse Nous la côtoyâmes avec peine, et nous arrivâmes à un lieu nommé Beaux Ports, près duquel était la ville de Lasée. 
\verse Un temps assez long s`était écoulé, et la navigation devenait dangereuse, car l`époque même du jeûne était déjà passée. 
\verse C`est pourquoi Paul avertit les autres, en disant: O hommes, je vois que la navigation ne se fera pas sans péril et sans beaucoup de dommage, non seulement pour la cargaison et pour le navire, mais encore pour nos personnes. 
\verse Le centenier écouta le pilote et le patron du navire plutôt que les paroles de Paul. 
\verse Et comme le port n`était pas bon pour hiverner, la plupart furent d`avis de le quitter pour tâcher d`atteindre Phénix, port de Crète qui regarde le sud-ouest et le nord-ouest, afin d`y passer l`hiver. 
\verse Un léger vent du sud vint à souffler, et, se croyant maîtres de leur dessein, ils levèrent l`ancre et côtoyèrent de près l`île de Crète. 
\verse Mais bientôt un vent impétueux, qu`on appelle Euraquilon, se déchaîna sur l`île. 
\verse Le navire fut entraîné, sans pouvoir lutter contre le vent, et nous nous laissâmes aller à la dérive. 
\verse Nous passâmes au-dessous d`une petite île nommée Clauda, et nous eûmes de la peine à nous rendre maîtres de la chaloupe; 
\verse après l`avoir hissée, on se servit des moyens de secours pour ceindre le navire, et, dans la crainte de tomber sur la Syrte, on abaissa les voiles. C`est ainsi qu`on se laissa emporter par le vent. 
\verse Comme nous étions violemment battus par la tempête, le lendemain on jeta la cargaison à la mer, 
\verse et le troisième jour nous y lançâmes de nos propres mains les agrès du navire. 
\verse Le soleil et les étoiles ne parurent pas pendant plusieurs jours, et la tempête était si forte que nous perdîmes enfin toute espérance de nous sauver. 
\verse On n`avait pas mangé depuis longtemps. Alors Paul, se tenant au milieu d`eux, leur dit: O hommes, il fallait m`écouter et ne pas partir de Crète, afin d`éviter ce péril et ce dommage. 
\verse Maintenant je vous exhorte à prendre courage; car aucun de vous ne périra, et il n`y aura de perte que celle du navire. 
\verse Un ange du Dieu à qui j`appartiens et que je sers m`est apparu cette nuit, 
\verse et m`a dit: Paul, ne crains point; il faut que tu comparaisses devant César, et voici, Dieu t`a donné tous ceux qui naviguent avec toi. 
\verse C`est pourquoi, ô hommes, rassurez-vous, car j`ai cette confiance en Dieu qu`il en sera comme il m`a été dit. 
\verse Mais nous devons échouer sur une île. 
\verse La quatorzième nuit, tandis que nous étions ballottés sur l`Adriatique, les matelots, vers le milieu de la nuit, soupçonnèrent qu`on approchait de quelque terre. 
\verse Ayant jeté la sonde, ils trouvèrent vingt brasses; un peu plus loin, ils la jetèrent de nouveau, et trouvèrent quinze brasses. 
\verse Dans la crainte de heurter contre des écueils, ils jetèrent quatre ancres de la poupe, et attendirent le jour avec impatience. 
\verse Mais, comme les matelots cherchaient à s`échapper du navire, et mettaient la chaloupe à la mer sous prétexte de jeter les ancres de la proue, 
\verse Paul dit au centenier et aux soldats: Si ces hommes ne restent pas dans le navire, vous ne pouvez être sauvés. 
\verse Alors les soldats coupèrent les cordes de la chaloupe, et la laissèrent tomber. 
\verse Avant que le jour parût, Paul exhorta tout le monde à prendre de la nourriture, disant: C`est aujourd`hui le quatorzième jour que vous êtes dans l`attente et que vous persistez à vous abstenir de manger. 
\verse Je vous invite donc à prendre de la nourriture, car cela est nécessaire pour votre salut, et il ne se perdra pas un cheveux de la tête d`aucun de vous. 
\verse Ayant ainsi parlé, il prit du pain, et, après avoir rendu grâces à Dieu devant tous, il le rompit, et se mit à manger. 
\verse Et tous, reprenant courage, mangèrent aussi. 
\verse Nous étions, dans le navire, deux cent soixante-seize personnes en tout. 
\verse Quand ils eurent mangé suffisamment, ils allégèrent le navire en jetant le blé à la mer. 
\verse Lorsque le jour fut venu, ils ne reconnurent point la terre; mais, ayant aperçu un golfe avec une plage, ils résolurent d`y pousser le navire, s`ils le pouvaient. 
\verse Ils délièrent les ancres pour les laisser aller dans la mer, et ils relâchèrent en même temps les attaches des gouvernails; puis ils mirent au vent la voile d`artimon, et se dirigèrent vers le rivage. 
\verse Mais ils rencontrèrent une langue de terre, où ils firent échouer le navire; et la proue, s`étant engagée, resta immobile, tandis que la poupe se brisait par la violence des vagues. 
\verse Les soldats furent d`avis de tuer les prisonniers, de peur que quelqu`un d`eux ne s`échappât à la nage. 
\verse Mais le centenier, qui voulait sauver Paul, les empêcha d`exécuter ce dessein. Il ordonna à ceux qui savaient nager de se jeter les premiers dans l`eau pour gagner la terre, 
\verse et aux autres de se mettre sur des planches ou sur des débris du navire. Et ainsi tous parvinrent à terre sains et saufs. 

\chapter
\verse Après nous être sauvés, nous reconnûmes que l`île s`appelait Malte. 
\verse Les barbares nous témoignèrent une bienveillance peu commune; ils nous recueillirent tous auprès d`un grand feu, qu`ils avaient allumé parce que la pluie tombait et qu`il faisait grand froid. 
\verse Paul ayant ramassé un tas de broussailles et l`ayant mis au feu, une vipère en sortit par l`effet de la chaleur et s`attacha à sa main. 
\verse Quand les barbares virent l`animal suspendu à sa main, ils se dirent les uns aux autres: Assurément cet homme est un meurtrier, puisque la Justice n`a pas voulu le laisser vivre, après qu`il a été sauvé de la mer. 
\verse Paul secoua l`animal dans le feu, et ne ressentit aucun mal. 
\verse Ces gens s`attendaient à le voir enfler ou tomber mort subitement; mais, après avoir longtemps attendu, voyant qu`il ne lui arrivait aucun mal, ils changèrent d`avis et dirent que c`était un dieu. 
\verse Il y avait, dans les environs, des terres appartenant au principal personnage de l`île, nommé Publius, qui nous reçut et nous logea pendant trois jours de la manière la plus amicale. 
\verse Le père de Publius était alors au lit, malade de la fièvre et de la dysenterie; Paul, s`étant rendu vers lui, pria, lui imposa les mains, et le guérit. 
\verse Là-dessus, vinrent les autres malades de l`île, et ils furent guéris. 
\verse On nous rendit de grands honneurs, et, à notre départ, on nous fournit les choses dont nous avions besoin. 
\verse Après un séjour de trois mois, nous nous embarquâmes sur un navire d`Alexandrie, qui avait passé l`hiver dans l`île, et qui portait pour enseigne les Dioscures. 
\verse Ayant abordé à Syracuse, nous y restâmes trois jours. 
\verse De là, en suivant la côte, nous atteignîmes Reggio; et, le vent du midi s`étant levé le lendemain, nous fîmes en deux jours le trajet jusqu`à Pouzzoles, 
\verse où nous trouvâmes des frères qui nous prièrent de passer sept jours avec eux. Et c`est ainsi que nous allâmes à Rome. 
\verse De Rome vinrent à notre rencontre, jusqu`au Forum d`Appius et aux Trois Tavernes, les frères qui avaient entendu parler de nous. Paul, en les voyant, rendit grâces à Dieu, et prit courage. 
\verse Lorsque nous fûmes arrivés à Rome, on permit à Paul de demeurer en son particulier, avec un soldat qui le gardait. 
\verse Au bout de trois jours, Paul convoqua les principaux des Juifs; et, quand ils furent réunis, il leur adressa ces paroles: Hommes frères, sans avoir rien fait contre le peuple ni contre les coutumes de nos pères, j`ai été mis en prison à Jérusalem et livré de là entre les mains des Romains. 
\verse Après m`avoir interrogé, ils voulaient me relâcher, parce qu`il n`y avait en moi rien qui méritât la mort. 
\verse Mais les Juifs s`y opposèrent, et j`ai été forcé d`en appeler à César, n`ayant du reste aucun dessein d`accuser ma nation. 
\verse Voilà pourquoi j`ai demandé à vous voir et à vous parler; car c`est à cause de l`espérance d`Israël que je porte cette chaîne. 
\verse Ils lui répondirent: Nous n`avons reçu de Judée aucune lettre à ton sujet, et il n`est venu aucun frère qui ait rapporté ou dit du mal de toi. 
\verse Mais nous voudrions apprendre de toi ce que tu penses, car nous savons que cette secte rencontre partout de l`opposition. 
\verse Ils lui fixèrent un jour, et plusieurs vinrent le trouver dans son logis. Paul leur annonça le royaume de Dieu, en rendant témoignage, et en cherchant, par la loi de Moïse et par les prophètes, à les persuader de ce qui concerne Jésus. L`entretien dura depuis le matin jusqu`au soir. 
\verse Les uns furent persuadés par ce qu`il disait, et les autres ne crurent point. 
\verse Comme ils se retiraient en désaccord, Paul n`ajouta que ces mots: C`est avec raison que le Saint Esprit, parlant à vos pères par le prophète Ésaïe, a dit: 
\verse Va vers ce peuple, et dis: Vous entendrez de vos oreilles, et vous ne comprendrez point; Vous regarderez de vos yeux, et vous ne verrez point. 
\verse Car le coeur de ce peuple est devenu insensible; Ils ont endurci leurs oreilles, et ils ont fermé leurs yeux, De peur qu`ils ne voient de leurs yeux, qu`ils n`entendent de leurs oreilles, Qu`ils ne comprennent de leur coeur, Qu`ils ne se convertissent, et que je ne les guérisse. 
\verse Sachez donc que ce salut de Dieu a été envoyé aux païens, et qu`ils l`écouteront. 
\verse Lorsqu`il eut dit cela, les Juifs s`en allèrent, discutant vivement entre eux. 
\verse Paul demeura deux ans entiers dans une maison qu`il avait louée. Il recevait tous ceux qui venaient le voir, 
\verse prêchant le royaume de Dieu et enseignant ce qui concerne le Seigneur Jésus Christ, en toute liberté et sans obstacle. 
