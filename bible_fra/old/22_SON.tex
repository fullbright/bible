\book[Cantique des cantiques]{Cantique des cantiques}


\chapter[Cantique des cantiques]

\chaptermark{Cantique des cantiques}{}
\verse Cantique des cantiques, de Salomon. 
\verse Qu`il me baise des baisers de sa bouche! Car ton amour vaut mieux que le vin, 
\verse Tes parfums ont une odeur suave; Ton nom est un parfum qui se répand; C`est pourquoi les jeunes filles t`aiment. 
\verse Entraîne-moi après toi! Nous courrons! Le roi m`introduit dans ses appartements... Nous nous égaierons, nous nous réjouirons à cause de toi; Nous célébrerons ton amour plus que le vin. C`est avec raison que l`on t`aime. 
\verse Je suis noire, mais je suis belle, filles de Jérusalem, Comme les tentes de Kédar, comme les pavillons de Salomon. 
\verse Ne prenez pas garde à mon teint noir: C`est le soleil qui m`a brûlée. Les fils de ma mère se sont irrités contre moi, Ils m`ont faite gardienne des vignes. Ma vigne, à moi, je ne l`ai pas gardée. 
\verse Dis-moi, ô toi que mon coeur aime, Où tu fais paître tes brebis, Où tu les fais reposer à midi; Car pourquoi serais-je comme une égarée Près des troupeaux de tes compagnons? - 
\verse Si tu ne le sais pas, ô la plus belle des femmes, Sors sur les traces des brebis, Et fais paître tes chevreaux Près des demeures des bergers. - 
\verse A ma jument qu`on attelle aux chars de Pharaon Je te compare, ô mon amie. 
\verse Tes joues sont belles au milieu des colliers, Ton cou est beau au milieu des rangées de perles. 
\verse Nous te ferons des colliers d`or, Avec des points d`argent. - 
\verse Tandis que le roi est dans son entourage, Mon nard exhale son parfum. 
\verse Mon bien-aimé est pour moi un bouquet de myrrhe, Qui repose entre mes seins. 
\verse Mon bien-aimé est pour moi une grappe de troëne Des vignes d`En Guédi. - 
\verse Que tu es belle, mon amie, que tu es belle! Tes yeux sont des colombes. - 
\verse Que tu es beau, mon bien-aimé, que tu es aimable! Notre lit, c`est la verdure. - 
\verse Les solives de nos maisons sont des cèdres, Nos lambris sont des cyprès. - 

\chapter[Cantique des cantiques]

\chaptermark{Cantique des cantiques}{}
\verse Je suis un narcisse de Saron, Un lis des vallées. - 
\verse Comme un lis au milieu des épines, Telle est mon amie parmi les jeunes filles. - 
\verse Comme un pommier au milieu des arbres de la forêt, Tel est mon bien-aimé parmi les jeunes hommes. J`ai désiré m`asseoir à son ombre, Et son fruit est doux à mon palais. 
\verse Il m`a fait entrer dans la maison du vin; Et la bannière qu`il déploie sur moi, c`est l`amour. 
\verse Soutenez-moi avec des gâteaux de raisins, Fortifiez-moi avec des pommes; Car je suis malade d`amour. 
\verse Que sa main gauche soit sous ma tête, Et que sa droite m`embrasse! - 
\verse Je vous en conjure, filles de Jérusalem, Par les gazelles et les biches des champs, Ne réveillez pas, ne réveillez pas l`amour, Avant qu`elle le veuille. - 
\verse C`est la voix de mon bien-aimé! Le voici, il vient, Sautant sur les montagnes, Bondissant sur les collines. 
\verse Mon bien-aimé est semblable à la gazelle Ou au faon des biches. Le voici, il est derrière notre mur, Il regarde par la fenêtre, Il regarde par le treillis. 
\verse Mon bien-aimé parle et me dit: Lève-toi, mon amie, ma belle, et viens! 
\verse Car voici, l`hiver est passé; La pluie a cessé, elle s`en est allée. 
\verse Les fleurs paraissent sur la terre, Le temps de chanter est arrivé, Et la voix de la tourterelle se fait entendre dans nos campagnes. 
\verse Le figuier embaume ses fruits, Et les vignes en fleur exhalent leur parfum. Lève-toi, mon amie, ma belle, et viens! 
\verse Ma colombe, qui te tiens dans les fentes du rocher, Qui te caches dans les parois escarpées, Fais-moi voir ta figure, Fais-moi entendre ta voix; Car ta voix est douce, et ta figure est agréable. 
\verse Prenez-nous les renards, Les petits renards qui ravagent les vignes; Car nos vignes sont en fleur. 
\verse Mon bien-aimé est à moi, et je suis à lui; Il fait paître son troupeau parmi les lis. 
\verse Avant que le jour se rafraîchisse, Et que les ombres fuient, Reviens!... sois semblable, mon bien-aimé, A la gazelle ou au faon des biches, Sur les montagnes qui nous séparent. 

\chapter[Cantique des cantiques]

\chaptermark{Cantique des cantiques}{}
\verse Sur ma couche, pendant les nuits, J`ai cherché celui que mon coeur aime; Je l`ai cherché, et je ne l`ai point trouvé... 
\verse Je me lèverai, et je ferai le tour de la ville, Dans les rues et sur les places; Je chercherai celui que mon coeur aime... Je l`ai cherché, et je ne l`ai point trouvé. 
\verse Les gardes qui font la ronde dans la ville m`ont rencontrée: Avez-vous vu celui que mon coeur aime? 
\verse A peine les avais-je passés, Que j`ai trouvé celui que mon coeur aime; Je l`ai saisi, et je ne l`ai point lâché Jusqu`à ce que je l`aie amené dans la maison de ma mère, Dans la chambre de celle qui m`a conçue. - 
\verse Je vous en conjure, filles de Jérusalem, Par les gazelles et les biches des champs, Ne réveillez pas, ne réveillez pas l`amour, Avant qu`elle le veuille. - 
\verse Qui est celle qui monte du désert, Comme des colonnes de fumée, Au milieu des vapeurs de myrrhe et d`encens Et de tous les aromates des marchands? - 
\verse Voici la litière de Salomon, Et autour d`elle soixante vaillants hommes, Des plus vaillants d`Israël. 
\verse Tous sont armés de l`épée, Sont exercés au combat; Chacun porte l`épée sur sa hanche, En vue des alarmes nocturnes. 
\verse Le roi Salomon s`est fait une litière De bois du Liban. 
\verse Il en a fait les colonnes d`argent, Le dossier d`or, Le siège de pourpre; Au milieu est une broderie, oeuvre d`amour Des filles de Jérusalem. 
\verse Sortez, filles de Sion, regardez Le roi Salomon, Avec la couronne dont sa mère l`a couronné Le jour de ses fiançailles, Le jour de la joie de son coeur. - 

\chapter[Cantique des cantiques]

\chaptermark{Cantique des cantiques}{}
\verse Que tu es belle, mon amie, que tu es belle! Tes yeux sont des colombes, Derrière ton voile. Tes cheveux sont comme un troupeau de chèvres, Suspendues aux flancs de la montagne de Galaad. 
\verse Tes dents sont comme un troupeau de brebis tondues, Qui remontent de l`abreuvoir; Toutes portent des jumeaux, Aucune d`elles n`est stérile. 
\verse Tes lèvres sont comme un fil cramoisi, Et ta bouche est charmante; Ta joue est comme une moitié de grenade, Derrière ton voile. 
\verse Ton cou est comme la tour de David, Bâtie pour être un arsenal; Mille boucliers y sont suspendus, Tous les boucliers des héros. 
\verse Tes deux seins sont comme deux faons, Comme les jumeaux d`une gazelle, Qui paissent au milieu des lis. 
\verse Avant que le jour se rafraîchisse, Et que les ombres fuient, J`irai à la montagne de la myrrhe Et à la colline de l`encens. 
\verse Tu es toute belle, mon amie, Et il n`y a point en toi de défaut. 
\verse Viens avec moi du Liban, ma fiancée, Viens avec moi du Liban! Regarde du sommet de l`Amana, Du sommet du Senir et de l`Hermon, Des tanières des lions, Des montagnes des léopards. 
\verse Tu me ravis le coeur, ma soeur, ma fiancée, Tu me ravis le coeur par l`un de tes regards, Par l`un des colliers de ton cou. 
\verse Que de charmes dans ton amour, ma soeur, ma fiancée! Comme ton amour vaut mieux que le vin, Et combien tes parfums sont plus suaves que tous les aromates! 
\verse Tes lèvres distillent le miel, ma fiancée; Il y a sous ta langue du miel et du lait, Et l`odeur de tes vêtements est comme l`odeur du Liban. 
\verse Tu es un jardin fermé, ma soeur, ma fiancée, Une source fermée, une fontaine scellée. 
\verse Tes jets forment un jardin, où sont des grenadiers, Avec les fruits les plus excellents, Les troënes avec le nard; 
\verse Le nard et le safran, le roseau aromatique et le cinnamome, Avec tous les arbres qui donnent l`encens; La myrrhe et l`aloès, Avec tous les principaux aromates; 
\verse Une fontaine des jardins, Une source d`eaux vives, Des ruisseaux du Liban. 
\verse Lève-toi, aquilon! viens, autan! Soufflez sur mon jardin, et que les parfums s`en exhalent! -Que mon bien-aimé entre dans son jardin, Et qu`il mange de ses fruits excellents! - 

\chapter[Cantique des cantiques]

\chaptermark{Cantique des cantiques}{}
\verse J`entre dans mon jardin, ma soeur, ma fiancée; Je cueille ma myrrhe avec mes aromates, Je mange mon rayon de miel avec mon miel, Je bois mon vin avec mon lait... -Mangez, amis, buvez, enivrez-vous d`amour! - 
\verse J`étais endormie, mais mon coeur veillait... C`est la voix de mon bien-aimé, qui frappe: -Ouvre-moi, ma soeur, mon amie, Ma colombe, ma parfaite! Car ma tête est couverte de rosée, Mes boucles sont pleines des gouttes de la nuit. - 
\verse J`ai ôté ma tunique; comment la remettrais-je? J`ai lavé mes pieds; comment les salirais-je? 
\verse Mon bien-aimé a passé la main par la fenêtre, Et mes entrailles se sont émues pour lui. 
\verse Je me suis levée pour ouvrir à mon bien-aimé; Et de mes mains a dégoutté la myrrhe, De mes doigts, la myrrhe répandue Sur la poignée du verrou. 
\verse J`ai ouvert à mon bien-aimé; Mais mon bien-aimé s`en était allé, il avait disparu. J`étais hors de moi, quand il me parlait. Je l`ai cherché, et je ne l`ai point trouvé; Je l`ai appelé, et il ne m`a point répondu. 
\verse Les gardes qui font la ronde dans la ville m`ont rencontrée; Ils m`ont frappée, ils m`ont blessée; Ils m`ont enlevé mon voile, les gardes des murs. 
\verse Je vous en conjure, filles de Jérusalem, Si vous trouvez mon bien-aimé, Que lui direz-vous?... Que je suis malade d`amour. - 
\verse Qu`a ton bien-aimé de plus qu`un autre, O la plus belle des femmes? Qu`a ton bien-aimé de plus qu`un autre, Pour que tu nous conjures ainsi? - 
\verse Mon bien-aimé est blanc et vermeil; Il se distingue entre dix mille. 
\verse Sa tête est de l`or pur; Ses boucles sont flottantes, Noires comme le corbeau. 
\verse Ses yeux sont comme des colombes au bord des ruisseaux, Se baignant dans le lait, Reposant au sein de l`abondance. 
\verse Ses joues sont comme un parterre d`aromates, Une couche de plantes odorantes; Ses lèvres sont des lis, D`où découle la myrrhe. 
\verse Ses mains sont des anneaux d`or, Garnis de chrysolithes; Son corps est de l`ivoire poli, Couvert de saphirs; 
\verse Ses jambes sont des colonnes de marbre blanc, Posées sur des bases d`or pur. Son aspect est comme le Liban, Distingué comme les cèdres. 
\verse Son palais n`est que douceur, Et toute sa personne est pleine de charme. Tel est mon bien-aimé, tel est mon ami, Filles de Jérusalem! - 

\chapter[Cantique des cantiques]

\chaptermark{Cantique des cantiques}{}
\verse Où est allé ton bien-aimé, O la plus belle des femmes? De quel côté ton bien-aimé s`est-il dirigé? Nous le chercherons avec toi. 
\verse Mon bien-aimé est descendu à son jardin, Au parterre d`aromates, Pour faire paître son troupeau dans les jardins, Et pour cueillir des lis. 
\verse Je suis à mon bien-aimé, et mon bien-aimé est à moi; Il fait paître son troupeau parmi les lis. - 
\verse Tu es belle, mon amie, comme Thirtsa, Agréable comme Jérusalem, Mais terrible comme des troupes sous leurs bannières. 
\verse Détourne de moi tes yeux, car ils me troublent. Tes cheveux sont comme un troupeau de chèvres, Suspendues aux flancs de Galaad. 
\verse Tes dents sont comme un troupeau de brebis, Qui remontent de l`abreuvoir; Toutes portent des jumeaux, Aucune d`elles n`est stérile. 
\verse Ta joue est comme une moitié de grenade, Derrière ton voile... 
\verse Il y a soixante reines, quatre-vingts concubines, Et des jeunes filles sans nombre. 
\verse Une seule est ma colombe, ma parfaite; Elle est l`unique de sa mère, La préférée de celle qui lui donna le jour. Les jeunes filles la voient, et la disent heureuse; Les reines et les concubines aussi, et elles la louent. - 
\verse Qui est celle qui apparaît comme l`aurore, Belle comme la lune, pure comme le soleil, Mais terrible comme des troupes sous leurs bannières? - 
\verse Je suis descendue au jardin des noyers, Pour voir la verdure de la vallée, Pour voir si la vigne pousse, Si les grenadiers fleurissent. 
\verse Je ne sais, mais mon désir m`a rendue semblable Aux chars de mon noble peuple. - 
\verse (7:1) Reviens, reviens, Sulamithe! Reviens, reviens, afin que nous te regardions. -Qu`avez-vous à regarder la Sulamithe Comme une danse de deux choeurs? 

\chapter[Cantique des cantiques]

\chaptermark{Cantique des cantiques}{}
\verse (7:2) Que tes pieds sont beaux dans ta chaussure, fille de prince! Les contours de ta hanche sont comme des colliers, Oeuvre des mains d`un artiste. 
\verse (7:3) Ton sein est une coupe arrondie, Où le vin parfumé ne manque pas; Ton corps est un tas de froment, Entouré de lis. 
\verse (7:4) Tes deux seins sont comme deux faons, Comme les jumeaux d`une gazelle. 
\verse (7:5) Ton cou est comme une tour d`ivoire; Tes yeux sont comme les étangs de Hesbon, Près de la porte de Bath Rabbim; Ton nez est comme la tour du Liban, Qui regarde du côté de Damas. 
\verse (7:6) Ta tête est élevée comme le Carmel, Et les cheveux de ta tête sont comme la pourpre; Un roi est enchaîné par des boucles!... 
\verse (7:7) Que tu es belle, que tu es agréable, O mon amour, au milieu des délices! 
\verse (7:8) Ta taille ressemble au palmier, Et tes seins à des grappes. 
\verse (7:9) Je me dis: Je monterai sur le palmier, J`en saisirai les rameaux! Que tes seins soient comme les grappes de la vigne, Le parfum de ton souffle comme celui des pommes, 
\verse (7:10) Et ta bouche comme un vin excellent,... -Qui coule aisément pour mon bien-aimé, Et glisse sur les lèvres de ceux qui s`endorment! 
\verse (7:11) Je suis à mon bien-aimé, Et ses désirs se portent vers moi. 
\verse (7:12) Viens, mon bien-aimé, sortons dans les champs, Demeurons dans les villages! 
\verse (7:13) Dès le matin nous irons aux vignes, Nous verrons si la vigne pousse, si la fleur s`ouvre, Si les grenadiers fleurissent. Là je te donnerai mon amour. 
\verse (7:14) Les mandragores répandent leur parfum, Et nous avons à nos portes tous les meilleurs fruits, Nouveaux et anciens: Mon bien-aimé, je les ai gardés pour toi. 

\chapter[Cantique des cantiques]

\chaptermark{Cantique des cantiques}{}
\verse Oh! Que n`es-tu mon frère, Allaité des mamelles de ma mère! Je te rencontrerais dehors, je t`embrasserais, Et l`on ne me mépriserait pas. 
\verse Je veux te conduire, t`amener à la maison de ma mère; Tu me donneras tes instructions, Et je te ferai boire du vin parfumé, Du moût de mes grenades. 
\verse Que sa main gauche soit sous ma tête, Et que sa droite m`embrasse! - 
\verse Je vous en conjure, filles de Jérusalem, Ne réveillez pas, ne réveillez pas l`amour, Avant qu`elle le veuille. - 
\verse Qui est celle qui monte du désert, Appuyée sur son bien-aimé? -Je t`ai réveillée sous le pommier; Là ta mère t`a enfantée, C`est là qu`elle t`a enfantée, qu`elle t`a donné le jour. - 
\verse Mets-moi comme un sceau sur ton coeur, Comme un sceau sur ton bras; Car l`amour est fort comme la mort, La jalousie est inflexible comme le séjour des morts; Ses ardeurs sont des ardeurs de feu, Une flamme de l`Éternel. 
\verse Les grandes eaux ne peuvent éteindre l`amour, Et les fleuves ne le submergeraient pas; Quand un homme offrirait tous les biens de sa maison contre l`amour, Il ne s`attirerait que le mépris. 
\verse Nous avons une petite soeur, Qui n`a point encore de mamelles; Que ferons-nous de notre soeur, Le jour où on la recherchera? 
\verse Si elle est un mur, Nous bâtirons sur elle des créneaux d`argent; Si elle est une porte, Nous la fermerons avec une planche de cèdre. - 
\verse Je suis un mur, Et mes seins sont comme des tours; J`ai été à ses yeux comme celle qui trouve la paix. 
\verse Salomon avait une vigne à Baal Hamon; Il remit la vigne à des gardiens; Chacun apportait pour son fruit mille sicles d`argent. 
\verse Ma vigne, qui est à moi, je la garde. A toi, Salomon, les mille sicles, Et deux cents à ceux qui gardent le fruit! - 
\verse Habitante des jardins! Des amis prêtent l`oreille à ta voix. Daigne me la faire entendre! - 
\verse Fuis, mon bien-aimé! Sois semblable à la gazelle ou au faon des biches, Sur les montagnes des aromates! 
