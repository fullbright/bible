\book[Première épître à Timothée]{1 Timothée}


\chapter
\verse Paul, apôtre de Jésus Christ, par ordre de Dieu notre Sauveur et de Jésus Christ notre espérance, 
\verse à Timothée, mon enfant légitime en la foi: que la grâce, la miséricorde et la paix, te soient données de la part de Dieu le Père et de Jésus Christ notre Seigneur! 
\verse Je te rappelle l`exhortation que je te fis, à mon départ pour la Macédoine, lorsque je t`engageai à rester à Éphèse, afin de recommander à certaines personnes de ne pas enseigner d`autres doctrines, 
\verse et de ne pas s`attacher à des fables et à des généalogies sans fin, qui produisent des discussions plutôt qu`elles n`avancent l`oeuvre de Dieu dans la foi. 
\verse Le but du commandement, c`est une charité venant d`un coeur pur, d`une bonne conscience, et d`une foi sincère. 
\verse Quelques-uns, s`étant détournés de ces choses, se sont égarés dans de vains discours; 
\verse ils veulent être docteurs de la loi, et ils ne comprennent ni ce qu`ils disent, ni ce qu`ils affirment. 
\verse Nous n`ignorons pas que la loi est bonne, pourvu qu`on en fasse un usage légitime, 
\verse sachant bien que la loi n`est pas faite pour le juste, mais pour les méchants et les rebelles, les impies et les pécheurs, les irréligieux et les profanes, les parricides, les meurtriers, 
\verse les impudiques, les infâmes, les voleurs d`hommes, les menteurs, les parjures, et tout ce qui est contraire à la saine doctrine, - 
\verse conformément à l`Évangile de la gloire du Dieu bienheureux, Évangile qui m`a été confié. 
\verse Je rends grâces à celui qui m`a fortifié, à Jésus Christ notre Seigneur, de ce qu`il m`a jugé fidèle, 
\verse en m`établissant dans le ministère, moi qui étais auparavant un blasphémateur, un persécuteur, un homme violent. Mais j`ai obtenu miséricorde, parce que j`agissais par ignorance, dans l`incrédulité; 
\verse et la grâce de notre Seigneur a surabondé, avec la foi et la charité qui est en Jésus Christ. 
\verse C`est une parole certaine et entièrement digne d`être reçue, que Jésus Christ est venu dans le monde pour sauver les pécheurs, dont je suis le premier. 
\verse Mais j`ai obtenu miséricorde, afin que Jésus Christ fît voir en moi le premier toute sa longanimité, pour que je servisse d`exemple à ceux qui croiraient en lui pour la vie éternelle. 
\verse Au roi des siècles, immortel, invisible, seul Dieu, soient honneur et gloire, aux siècles des siècles! Amen! 
\verse Le commandement que je t`adresse, Timothée, mon enfant, selon les prophéties faites précédemment à ton sujet, c`est que, d`après elles, tu combattes le bon combat, 
\verse en gardant la foi et une bonne conscience. Cette conscience, quelques-uns l`ont perdue, et ils ont fait naufrage par rapport à la foi. 
\verse De ce nombre son Hyménée et Alexandre, que j`ai livrés à Satan, afin qu`ils apprennent à ne pas blasphémer. 

\chapter
\verse J`exhorte donc, avant toutes choses, à faire des prières, des supplications, des requêtes, des actions de grâces, pour tous les hommes, 
\verse pour les rois et pour tous ceux qui sont élevés en dignité, afin que nous menions une vie paisible et tranquille, en toute piété et honnêteté. 
\verse Cela est bon et agréable devant Dieu notre Sauveur, 
\verse qui veut que tous les hommes soient sauvés et parviennent à la connaissance de la vérité. 
\verse Car il y a un seul Dieu, et aussi un seul médiateur entre Dieu et les hommes, Jésus Christ homme, 
\verse qui s`est donné lui-même en rançon pour tous. C`est là le témoignage rendu en son propre temps, 
\verse et pour lequel j`ai été établi prédicateur et apôtre, -je dis la vérité, je ne mens pas, -chargé d`instruire les païens dans la foi et la vérité. 
\verse Je veux donc que les hommes prient en tout lieu, en élevant des mains pures, sans colère ni mauvaises pensées. 
\verse Je veux aussi que les femmes, vêtues d`une manière décente, avec pudeur et modestie, ne se parent ni de tresses, ni d`or, ni de perles, ni d`habits somptueux, 
\verse mais qu`elles se parent de bonnes oeuvres, comme il convient à des femmes qui font profession de servir Dieu. 
\verse Que la femme écoute l`instruction en silence, avec une entière soumission. 
\verse Je ne permets pas à la femme d`enseigner, ni de prendre de l`autorité sur l`homme; mais elle doit demeurer dans le silence. 
\verse Car Adam a été formé le premier, Eve ensuite; 
\verse et ce n`est pas Adam qui a été séduit, c`est la femme qui, séduite, s`est rendue coupable de transgression. 
\verse Elle sera néanmoins sauvée en devenant mère, si elle persévère avec modestie dans la foi, dans la charité, et dans la sainteté. 

\chapter
\verse Cette parole est certaine: Si quelqu`un aspire à la charge d`évêque, il désire une oeuvre excellente. 
\verse Il faut donc que l`évêque soit irréprochable, mari d`une seul femme, sobre, modéré, réglé dans sa conduite, hospitalier, propre à l`enseignement. 
\verse Il faut qu`il ne soit ni adonné au vin, ni violent, mais indulgent, pacifique, désintéressé. 
\verse Il faut qu`il dirige bien sa propre maison, et qu`il tienne ses enfants dans la soumission et dans une parfaite honnêteté; 
\verse car si quelqu`un ne sait pas diriger sa propre maison, comment prendra-t-il soin de l`Église de Dieu? 
\verse Il ne faut pas qu`il soit un nouveau converti, de peur qu`enflé d`orgueil il ne tombe sous le jugement du diable. 
\verse Il faut aussi qu`il reçoive un bon témoignage de ceux du dehors, afin de ne pas tomber dans l`opprobre et dans les pièges du diable. 
\verse Les diacres aussi doivent être honnêtes, éloignés de la duplicité, des excès du vin, d`un gain sordide, 
\verse conservant le mystère de la foi dans une conscience pure. 
\verse Qu`on les éprouve d`abord, et qu`ils exercent ensuite leur ministère, s`ils sont sans reproche. 
\verse Les femmes, de même, doivent être honnêtes, non médisantes, sobres, fidèles en toutes choses. 
\verse Les diacres doivent être maris d`une seule femme, et diriger bien leurs enfants et leurs propres maisons; 
\verse car ceux qui remplissent convenablement leur ministère s`acquièrent un rang honorable, et une grande assurance dans la foi en Jésus Christ. 
\verse Je t`écris ces choses, avec l`espérance d`aller bientôt vers toi, 
\verse mais afin que tu saches, si je tarde, comment il faut se conduire dans la maison de Dieu, qui est l`Église du Dieu vivant, la colonne et l`appui de la vérité. 
\verse Et, sans contredit, le mystère de la piété est grand: celui qui a été manifesté en chair, justifié par l`Esprit, vu des anges, prêché aux Gentils, cru dans le monde, élevé dans la gloire. 

\chapter
\verse Mais l`Esprit dit expressément que, dans les derniers temps, quelques-uns abandonneront la foi, pour s`attacher à des esprits séducteurs et à des doctrines de démons, 
\verse par l`hypocrisie de faux docteurs portant la marque de la flétrissure dans leur propre conscience, 
\verse prescrivant de ne pas se marier, et de s`abstenir d`aliments que Dieu a créés pour qu`ils soient pris avec actions de grâces par ceux qui sont fidèles et qui ont connu la vérité. 
\verse Car tout ce que Dieu a créé est bon, et rien ne doit être rejeté, pourvu qu`on le prenne avec actions de grâces, 
\verse parce que tout est sanctifié par la parole de Dieu et par la prière. 
\verse En exposant ces choses au frères, tu seras un bon ministre de Jésus Christ, nourri des paroles de la foi et de la bonne doctrine que tu as exactement suivie. 
\verse Repousse les contes profanes et absurdes. 
\verse Exerce-toi à la piété; car l`exercice corporel est utile à peu de chose, tandis que la piété est utile à tout, ayant la promesse de la vie présente et de celle qui est à venir. 
\verse C`est là une parole certaine et entièrement digne d`être reçue. 
\verse Nous travaillons, en effet, et nous combattons, parce que nous mettons notre espérance dans le Dieu vivant, qui est le Sauveur de tous les hommes, principalement des croyants. 
\verse Déclare ces choses, et enseigne-les. 
\verse Que personne ne méprise ta jeunesse; mais sois un modèle pour les fidèles, en parole, en conduite, en charité, en foi, en pureté. 
\verse Jusqu`à ce que je vienne, applique-toi à la lecture, à l`exhortation, à l`enseignement. 
\verse Ne néglige pas le don qui est en toi, et qui t`a été donné par prophétie avec l`imposition des mains de l`assemblée des anciens. 
\verse Occupe-toi de ces choses, donne-toi tout entier à elles, afin que tes progrès soient évidents pour tous. 
\verse Veille sur toi-même et sur ton enseignement; persévère dans ces choses, car, en agissant ainsi, tu te sauveras toi-même, et tu sauveras ceux qui t`écoutent. 

\chapter
\verse Ne réprimande pas rudement le vieillard, mais exhorte-le comme un père; exhorte les jeunes gens comme des frères, 
\verse les femmes âgées comme des mères, celles qui sont jeunes comme des soeurs, en toute pureté. 
\verse Honore les veuves qui sont véritablement veuves. 
\verse Si une veuve a des enfants ou des petits-enfants, qu`ils apprennent avant tout à exercer la piété envers leur propre famille, et à rendre à leurs parents ce qu`ils ont reçu d`eux; car cela est agréable à Dieu. 
\verse Celle qui est véritablement veuve, et qui est demeurée dans l`isolement, met son espérance en Dieu et persévère nuit et jour dans les supplications et les prières. 
\verse Mais celle qui vit dans les plaisirs est morte, quoique vivante. 
\verse Déclare-leur ces choses, afin qu`elles soient irréprochables. 
\verse Si quelqu`un n`a pas soin des siens, et principalement de ceux de sa famille, il a renié la foi, et il est pire qu`un infidèle. 
\verse Qu`une veuve, pour être inscrite sur le rôle, n`ait pas moins de soixante ans, qu`elle ait été femme d`un seul mari, 
\verse qu`elle soit recommandable par de bonnes oeuvres, ayant élevé des enfants, exercé l`hospitalité, lavé les pieds des saints, secouru les malheureux, pratiqué toute espèce de bonne oeuvre. 
\verse Mais refuse les jeunes veuves; car, lorsque la volupté les détache du Christ, elles veulent se marier, 
\verse et se rendent coupables en ce qu`elles violent leur premier engagement. 
\verse Avec cela, étant oisives, elles apprennent à aller de maison en maison; et non seulement elles sont oisives, mais encore causeuses et intrigantes, disant ce qu`il ne faut pas dire. 
\verse Je veux donc que les jeunes se marient, qu`elles aient des enfants, qu`elles dirigent leur maison, qu`elles ne donnent à l`adversaire aucune occasion de médire; 
\verse car déjà quelques-unes se sont détournées pour suivre Satan. 
\verse Si quelque fidèle, homme ou femme, a des veuves, qu`il les assiste, et que l`Église n`en soit point chargée, afin qu`elle puisse assister celles qui sont véritablement veuves. 
\verse Que les anciens qui dirigent bien soient jugés dignes d`un double honneur, surtout ceux qui travaillent à la prédication et à l`enseignement. 
\verse Car l`Écriture dit: Tu n`emmuselleras point le boeuf quand il foule le grain. Et l`ouvrier mérite son salaire. 
\verse Ne reçois point d`accusation contre un ancien, si ce n`est sur la déposition de deux ou trois témoins. 
\verse Ceux qui pèchent, reprends-les devant tous, afin que les autres aussi éprouvent de la crainte. 
\verse Je te conjure devant Dieu, devant Jésus Christ, et devant les anges élus, d`observer ces choses sans prévention, et de ne rien faire par faveur. 
\verse N`impose les mains à personne avec précipitation, et ne participe pas aux péchés d`autrui; toi-même, conserve-toi pur. 
\verse Ne continue pas à ne boire que de l`eau; mais fais usage d`un peu de vin, à cause de ton estomac et de tes fréquentes indispositions. 
\verse Les péchés de certains hommes sont manifestes, même avant qu`on les juge, tandis que chez d`autres, ils ne se découvrent que dans la suite. 
\verse De même, les bonnes oeuvres sont manifestes, et celles qui ne le sont pas ne peuvent rester cachées. 

\chapter
\verse Que tous ceux qui sont sous le joug de la servitude regardent leurs maîtres comme dignes de tout honneur, afin que le nom de Dieu et la doctrine ne soient pas blasphémés. 
\verse Et que ceux qui ont des fidèles pour maîtres ne les méprisent pas, sous prétexte qu`ils sont frères; mais qu`ils les servent d`autant mieux que ce sont des fidèles et des bien-aimés qui s`attachent à leur faire du bien. Enseigne ces choses et recommande-les. 
\verse Si quelqu`un enseigne de fausses doctrines, et ne s`attache pas aux saines paroles de notre Seigneur Jésus Christ et à la doctrine qui est selon la piété, 
\verse il est enflé d`orgueil, il ne sait rien, et il a la maladie des questions oiseuses et des disputes de mots, d`où naissent l`envie, les querelles, les calomnies, les mauvais soupçons, 
\verse les vaines discussions d`hommes corrompus d`entendement, privés de la vérité, et croyant que la piété est une source de gain. 
\verse C`est, en effet, une grande source de gain que la piété avec le contentement; 
\verse car nous n`avons rien apporté dans le monde, et il est évident que nous n`en pouvons rien emporter; 
\verse si donc nous avons la nourriture et le vêtement, cela nous suffira. 
\verse Mais ceux qui veulent s`enrichir tombent dans la tentation, dans le piège, et dans beaucoup de désirs insensés et pernicieux qui plongent les hommes dans la ruine et la perdition. 
\verse Car l`amour de l`argent est une racine de tous les maux; et quelques-uns, en étant possédés, se sont égarés loin de la foi, et se sont jetés eux-mêmes dans bien des tourments. 
\verse Pour toi, homme de Dieu, fuis ces choses, et recherche la justice, la piété, la foi, la charité, la patience, la douceur. 
\verse Combats le bon combat de la foi, saisis la vie éternelle, à laquelle tu as été appelé, et pour laquelle tu as fait une belle confession en présence d`un grand nombre de témoins. 
\verse Je te recommande, devant Dieu qui donne la vie à toutes choses, et devant Jésus Christ, qui fit une belle confession devant Ponce Pilate, de garder le commandement, 
\verse et de vivre sans tache, sans reproche, jusqu`à l`apparition de notre Seigneur Jésus Christ, 
\verse que manifestera en son temps le bienheureux et seul souverain, le roi des rois, et le Seigneur des seigneurs, 
\verse qui seul possède l`immortalité, qui habite une lumière inaccessible, que nul homme n`a vu ni ne peut voir, à qui appartiennent l`honneur et la puissance éternelle. Amen! 
\verse Recommande aux riches du présent siècle de ne pas être orgueilleux, et de ne pas mettre leur espérance dans des richesses incertaines, mais de la mettre en Dieu, qui nous donne avec abondance toutes choses pour que nous en jouissions. 
\verse Recommande-leur de faire du bien, d`être riches en bonnes oeuvres, d`avoir de la libéralité, de la générosité, 
\verse et de s`amasser ainsi pour l`avenir un trésor placé sur un fondement solide, afin de saisir la vie véritable. 
\verse O Timothée, garde le dépôt, en évitant les discours vains et profanes, 
\verse et les disputes de la fausse science dont font profession quelques-uns, qui se sont ainsi détournés de la foi. Que la grâce soit avec vous! 
