\book[Premier livre des Chroniques]{1 Chroniques}


\chapter
\verse Adam, Seth, Énosch, 
\verse Kénan, Mahalaleel, Jéred, 
\verse Hénoc, Metuschélah, Lémec, 
\verse Noé, Sem, Cham et Japhet. 
\verse Fils de Japhet: Gomer, Magog, Madaï, Javan, Tubal, Méschec et Tiras. - 
\verse Fils de Gomer: Aschkenaz, Diphat et Togarma. - 
\verse Fils de Javan: Élischa, Tarsisa, Kittim et Rodanim. 
\verse Fils de Cham: Cusch, Mitsraïm, Puth et Canaan. - 
\verse Fils de Cusch: Saba, Havila, Sabta, Raema et Sabteca. -Fils de Raema: Séba et Dedan. 
\verse Cusch engendra Nimrod; c`est lui qui commença à être puissant sur la terre. - 
\verse Mitsraïm engendra les Ludim, les Ananim, les Lehabim, les Naphtuhim, 
\verse les Patrusim, les Casluhim, d`où sont sortis les Philistins, et les Caphtorim. - 
\verse Canaan engendra Sidon, son premier-né, et Heth, 
\verse et les Jébusiens, les Amoréens, les Guirgasiens, 
\verse les Héviens, les Arkiens, les Siniens, 
\verse les Arvadiens, les Tsemariens, les Hamathiens. 
\verse Fils de Sem: Élam, Assur, Arpacschad, Lud et Aram; Uts, Hul, Guéter et Méschec. - 
\verse Arpacschad engendra Schélach; et Schélach engendra Héber. 
\verse Il naquit à Héber deux fils: le nom de l`un était Péleg, parce que de son temps la terre fut partagée, et le nom de son frère était Jokthan. 
\verse Jokthan engendra Almodad, Schéleph, Hatsarmaveth, Jérach, 
\verse Hadoram, Uzal, Dikla, 
\verse Ébal, Abimaël, Séba, Ophir, Havila et Jobab. 
\verse Tous ceux-là furent fils de Jokthan. 
\verse Sem, Arpacschad, Schélach, 
\verse Héber, Péleg, Rehu, 
\verse Serug, Nachor, Térach, 
\verse Abram, qui est Abraham. 
\verse Fils d`Abraham: Isaac et Ismaël. 
\verse Voici leur postérité. Nebajoth, premier-né d`Ismaël, Kédar, Adbeel, Mibsam, 
\verse Mischma, Duma, Massa, Hadad, Téma, 
\verse Jethur, Naphisch et Kedma. Ce sont là les fils d`Ismaël. 
\verse Fils de Ketura, concubine d`Abraham. Elle enfanta Zimran, Jokschan, Medan, Madian, Jischbak et Schuach. -Fils de Jokschan: Séba et Dedan. - 
\verse Fils de Madian: Épha, Épher, Hénoc, Abida et Eldaa. -Ce sont là tous les fils de Ketura. 
\verse Abraham engendra Isaac. Fils d`Isaac: Ésaü et Israël. 
\verse Fils d`Ésaü: Éliphaz, Reuel, Jeusch, Jaelam et Koré. - 
\verse Fils d`Éliphaz: Théman, Omar, Tsephi, Gaetham, Kenaz, Thimna et Amalek. - 
\verse Fils de Reuel: Nahath, Zérach, Schamma et Mizza. 
\verse Fils de Séir: Lothan, Schobal, Tsibeon, Ana, Dischon, Etser et Dischan. - 
\verse Fils de Lothan: Hori et Homam. Soeur de Lothan: Thimna. - 
\verse Fils de Schobal: Aljan, Manahath, Ébal, Schephi et Onam. -Fils de Tsibeon: Ajja et Ana. - 
\verse Fils d`Ana: Dischon. Fils de Dischon: Hamran, Eschban, Jithran et Keran. - 
\verse Fils d`Etser: Bilhan, Zaavan et Jaakan. -Fils de Dischan: Uts et Aran. - 
\verse Voici les rois qui ont régné dans le pays d`Édom, avant qu`un roi règne sur les enfants d`Israël. -Béla, fils de Beor; et le nom de sa ville était Dinhaba. - 
\verse Béla mourut; et Jobab, fils de Zérach, de Botsra, régnât à sa place. - 
\verse Jobab mourut; et Huscham, du pays des Thémanites, régna à sa place. - 
\verse Huscham mourut; et Hadad, fils de Bedad, régna à sa place. C`est lui qui frappa Madian dans les champs de Moab. Le nom de sa ville était Avith. - 
\verse Hadad mourut; et Samla, de Masréka, régna à sa place. - 
\verse Samla mourut; et Saül, de Rehoboth sur le fleuve, régna à sa place. - 
\verse Saül mourut; et Baal Hanan, fils d`Acbor, régna à sa place. - 
\verse Baal Hanan mourut; et Hadad régna à sa place. Le nom de sa ville était Pahi; et le nom de sa femme Mehéthabeel, fille de Mathred, fille de Mézahab. - 
\verse Hadad mourut. Les chefs d`Édom furent: le chef Thimna, le chef Alja, le chef Jetheth, 
\verse le chef Oholibama, le chef Éla, le chef Pinon, 
\verse le chef Kenaz, le chef Théman, le chef Mibtsar, 
\verse le chef Magdiel, le chef Iram. Ce sont là des chefs d`Édom. 

\chapter
\verse Voici les fils d`Israël. Ruben, Siméon, Lévi, Juda, Issacar, Zabulon, 
\verse Dan, Joseph, Benjamin, Nephthali, Gad et Aser. 
\verse Fils de Juda: Er, Onan, Schéla; ces trois lui naquirent de la fille de Schua, la Cananéenne. Er, premier-né de Juda, était méchant aux yeux de l`Éternel, qui le fit mourir. 
\verse Tamar, belle-fille de Juda, lui enfanta Pérets et Zérach. Total des fils de Juda: cinq. 
\verse Fils de Pérets: Hetsron et Hamul. 
\verse Fils de Zérach: Zimri, Éthan, Héman, Calcol et Dara. En tout: cinq. - 
\verse Fils de Carmi: Acar, qui troubla Israël lorsqu`il commit une infidélité au sujet des choses dévouées par interdit. - 
\verse Fils d`Éthan: Azaria. 
\verse Fils qui naquirent à Hetsron: Jerachmeel, Ram et Kelubaï. 
\verse Ram engendra Amminadab. Amminadab engendra Nachschon, prince des fils de Juda. 
\verse Nachschon engendra Salma. Salma engendra Boaz. 
\verse Boaz engendra Obed. Obed engendra Isaï. 
\verse Isaï engendra Éliab, son premier-né, Abinadab le second, Schimea le troisième, 
\verse Nethaneel le quatrième, Raddaï le cinquième, 
\verse Otsem le sixième, David le septième. 
\verse Leurs soeurs étaient: Tseruja et Abigaïl. Fils de Tseruja: Abischaï, Joab et Asaël, trois. 
\verse Abigaïl enfanta Amasa; le père d`Amasa fut Jéther, l`Ismaélite. 
\verse Caleb, fils de Hetsron, eut des enfants d`Azuba, sa femme, et de Jerioth. Voici les fils qu`il eut d`Azuba: Jéscher, Schobab et Ardon. 
\verse Azuba mourut; et Caleb prit Éphrath, qui lui enfanta Hur. 
\verse Hur engendra Uri, et Uri engendra Betsaleel. - 
\verse Ensuite, Hetsron alla vers la fille de Makir, père de Galaad, et il avait soixante ans lorsqu`il la prit; elle lui enfanta Segub. 
\verse Segub engendra Jaïr, qui eut vingt-trois villes dans le pays de Galaad. 
\verse Les Gueschuriens et les Syriens leur prirent les bourgs de Jaïr avec Kenath et les villes de son ressort, soixante villes. Tous ceux-là étaient fils de Makir, père de Galaad. 
\verse Après la mort de Hetsron à Caleb Éphratha, Abija, femme de Hetsron, lui enfanta Aschchur, père de Tekoa. 
\verse Les fils de Jerachmeel, premier-né de Hetsron, furent: Ram, le premier-né, Buna, Oren et Otsem, nés d`Achija. 
\verse Jerachmeel eut une autre femme, nommée Athara, qui fut mère d`Onam. - 
\verse Les fils de Ram, premier-né de Jerachmeel, furent: Maats, Jamin et Éker. - 
\verse Les fils d`Onam furent: Schammaï et Jada. Fils de Schammaï: Nadab et Abischur. 
\verse Le nom de la femme d`Abischur était Abichaïl, et elle lui enfanta Achban et Molid. 
\verse Fils de Nadab: Séled et Appaïm. Séled mourut sans fils. 
\verse Fils d`Appaïm: Jischeï. Fils de Jischeï: Schéschan. Fils de Schéschan: Achlaï. - 
\verse Fils de Jada, frère de Schammaï: Jéther et Jonathan. Jéther mourut sans fils. 
\verse Fils de Jonathan: Péleth et Zara. -Ce sont là les fils de Jerachmeel. - 
\verse Schéschan n`eut point de fils, mais il eut des filles. Schéschan avait un esclave égyptien nommé Jarcha. 
\verse Et Schéschan donna sa fille pour femme à Jarcha, son esclave, à qui elle enfanta Attaï. 
\verse Attaï engendra Nathan; Nathan engendra Zabad; 
\verse Zabad engendra Ephlal; Ephlal engendra Obed; 
\verse Obed engendra Jéhu; Jéhu engendra Azaria; 
\verse Azaria engendra Halets; Halets engendra Élasa; 
\verse Élasa engendra Sismaï; Sismaï engendra Schallum; 
\verse Schallum engendra Jekamja; Jekamja engendra Élischama. 
\verse Fils de Caleb, frère de Jerachmeel: Méscha, son premier-né, qui fut père de Ziph, et les fils de Maréscha, père d`Hébron. 
\verse Fils d`Hébron: Koré, Thappuach, Rékem et Schéma. 
\verse Schéma engendra Racham, père de Jorkeam. Rékem engendra Schammaï. 
\verse Fils de Schammaï: Maon; et Maon, père de Beth Tsur. 
\verse Épha, concubine de Caleb, enfanta Haran, Motsa et Gazez. Haran engendra Gazez. 
\verse Fils de Jahdaï: Réguem, Jotham, Guéschan, Péleth, Épha et Schaaph. 
\verse Maaca, concubine de Caleb, enfanta Schéber et Tirchana. 
\verse Elle enfanta encore Schaaph, père de Madmanna, et Scheva, père de Macbéna et père de Guibea. La fille de Caleb était Acsa. 
\verse Ceux-ci furent fils de Caleb: Schobal, fils de Hur, premier-né d`Éphrata, et père de Kirjath Jearim; 
\verse Salma, père de Bethléhem; Hareph, père de Beth Gader. 
\verse Les fils de Schobal, père de Kirjath Jearim, furent: Haroé, Hatsi Hammenuhoth. 
\verse Les familles de Kirjath Jearim furent: les Jéthriens, les Puthiens, les Schumathiens et les Mischraïens; de ces familles sont sortis les Tsoreathiens et les Eschthaoliens. 
\verse Fils de Salma: Bethléhem et les Nethophatiens, Athroth Beth Joab, Hatsi Hammanachthi, les Tsoreïns; 
\verse et les familles des scribes demeurant à Jaebets, les Thireathiens, les Schimeathiens et les Sucathiens. Ce sont les Kéniens, issus de Hamath, père de la maison de Récab. 

\chapter
\verse Voici les fils de David, qui lui naquirent à Hébron. Le premier-né, Amnon, d`Achinoam de Jizreel; le second, Daniel, d`Abigaïl de Carmel; 
\verse le troisième, Absalom, fils de Maaca, fille de Talmaï, roi de Gueschur; le quatrième, Adonija, fils de Haggith; 
\verse le cinquième, Schephatia, d`Abithal; le sixième, Jithream, d`Égla, sa femme. 
\verse Ces six lui naquirent à Hébron. Il régna là sept ans et six mois, et il régna trente-trois ans à Jérusalem. 
\verse Voici ceux qui lui naquirent à Jérusalem. Schimea, Schobab, Nathan et Salomon, quatre de Bath Schua, fille d`Ammiel; 
\verse Jibhar, Élischama, Éliphéleth, 
\verse Noga, Népheg, Japhia, Élischama, 
\verse Éliada et Éliphéleth, neuf. 
\verse Ce sont là tous les fils de David, outre les fils des concubines. Et Tamar était leur soeur. 
\verse Fils de Salomon: Roboam. Abija, son fils; Asa, son fils; Josaphat, son fils; 
\verse Joram, son fils; Achazia, son fils; Joas, son fils; 
\verse Amatsia, son fils; Azaria, son fils; Jotham, son fils; 
\verse Achaz, son fils; Ézéchias, son fils; Manassé, son fils; 
\verse Amon, son fils; Josias, son fils. 
\verse Fils de Josias: le premier-né, Jochanan; le second, Jojakim; le troisième, Sédécias; le quatrième, Schallum. 
\verse Fils de Jojakim: Jéconias, son fils; Sédécias, son fils. 
\verse Fils de Jéconias: Assir, dont le fils fut Schealthiel, 
\verse Malkiram, Pedaja, Schénatsar, Jekamia, Hoschama et Nedabia. 
\verse Fils de Pedaja: Zorobabel et Schimeï. Fils de Zorobabel: Meschullam et Hanania; Schelomith, leur soeur; 
\verse et Haschuba, Ohel, Bérékia, Hasadia, Juschab Hésed, cinq. 
\verse Fils de Hanania: Pelathia et Ésaïe; les fils de Rephaja, les fils d`Arnan, les fils d`Abdias, les fils de Schecania. 
\verse Fils de Schecania: Schemaeja. Fils de Schemaeja: Hattusch, Jigueal, Bariach, Nearia et Schaphath, six. 
\verse Fils de Nearia: Eljoénaï, Ézéchias et Azrikam, trois. 
\verse Fils d`Eljoénaï: Hodavia, Éliaschib, Pelaja, Akkub, Jochanan, Delaja et Anani, sept. 

\chapter
\verse Fils de Juda: Pérets, Hetsron, Carmi, Hur et Schobal. 
\verse Reaja, fils de Schobal, engendra Jachath; Jachath engendra Achumaï et Lahad. Ce sont les familles des Tsoreathiens. 
\verse Voici les descendants du père d`Étham: Jizreel, Jischma et Jidbasch; le nom de leur soeur était Hatselelponi. 
\verse Penuel était père de Guedor, et Ézer père de Huscha. Ce sont là les fils de Hur, premier-né d`Éphrata, père de Bethléhem. 
\verse Aschchur, père de Tekoa, eut deux femmes, Hélea et Naara. 
\verse Naara lui enfanta Achuzzam, Hépher, Thémeni et Achaschthari: ce sont là les fils de Naara. 
\verse Fils de Hélea: Tséreth, Tsochar et Ethnan. 
\verse Kots engendra Anub et Hatsobéba, et les familles d`Acharchel, fils d`Harum. 
\verse Jaebets était plus considéré que ses frères; sa mère lui donna le nom de Jaebets, en disant: C`est parce que je l`ai enfanté avec douleur. 
\verse Jaebets invoqua le Dieu d`Israël, en disant: Si tu me bénis et que tu étendes mes limites, si ta main est avec moi, et si tu me préserves du malheur, en sorte que je ne sois pas dans la souffrance!... Et Dieu accorda ce qu`il avait demandé. 
\verse Kelub, frère de Schucha, engendra Mechir, qui fut père d`Eschthon. 
\verse Eschthon engendra la maison de Rapha, Paséach, et Thechinna, père de la ville de Nachasch. Ce sont là les hommes de Réca. 
\verse Fils de Kenaz: Othniel et Seraja. Fils d`Othniel: Hathath. 
\verse Meonothaï engendra Ophra. Seraja engendra Joab, père de la vallée des ouvriers; car ils étaient ouvriers. 
\verse Fils de Caleb, fils de Jephunné: Iru, Éla et Naam, et les fils d`Éla, et Kenaz. 
\verse Fils de Jehalléleel: Ziph, Zipha, Thirja et Asareel. 
\verse Fils d`Esdras: Jéther, Méred, Épher et Jalon. La femme de Méred enfanta Miriam, Schammaï, et Jischbach, père d`Eschthemoa. 
\verse Sa femme, la Juive, enfanta Jéred, père de Guedor, Héber, père de Soco, et Jekuthiel, père de Zanoach. Ceux-là sont les fils de Bithja, fille de Pharaon, que Méred prit pour femme. 
\verse Fils de la femme d`Hodija, soeur de Nacham: le père de Kehila, le Garmien, et Eschthemoa, le Maacathien. 
\verse Fils de Simon: Amnon, Rinna, Ben Hanan et Thilon. Fils de Jischeï: Zocheth et Ben Zocheth. 
\verse Fils de Schéla, fils de Juda: Er, père de Léca, Laeda, père de Maréscha, et les familles de la maison où l`on travaille le byssus, de la maison d`Aschbéa, 
\verse et Jokim, et les hommes de Cozéba, et Joas et Saraph, qui dominèrent sur Moab, et Jaschubi Léchem. Ces choses sont anciennes. 
\verse C`étaient les potiers et les habitants des plantations et des parcs; ils demeuraient là près du roi et travaillaient pour lui. 
\verse Fils de Siméon: Nemuel, Jamin, Jarib, Zérach, Saül. 
\verse Fils de Saül: Schallum. Mibsam, son fils; Mischma, son fils. 
\verse Fils de Mischma: Hammuel, son fils. Zaccur, son fils; Schimeï, son fils. 
\verse Schimeï eut seize fils et six filles. Ses frères n`eurent pas beaucoup de fils. Et toutes leurs familles ne se multiplièrent pas autant que les fils de Juda. 
\verse Ils habitaient à Beer Schéba, à Molada, à Hatsar Schual, 
\verse à Bilha, à Etsem, à Tholad, 
\verse à Bethuel, à Horma, à Tsiklag, 
\verse à Beth Marcaboth, à Hatsar Susim, à Beth Bireï et à Schaaraïm. Ce furent là leurs villes jusqu`au règne de David, et leurs villages. 
\verse Ils avaient encore Etham, Aïn, Rimmon, Thoken et Aschan, cinq villes; 
\verse et tous les villages aux environs de ces villes, jusqu`à Baal. Voilà leurs habitations et leur généalogie. 
\verse Meschobab; Jamlec; Joscha, fils d`Amatsia; 
\verse Joël; Jéhu, fils de Joschibia, fils de Seraja, fils d`Asiel; 
\verse Eljoénaï; Jaakoba; Jeschochaja; Asaja; Adiel; Jesimiel; Benaja; 
\verse Ziza, fils de Schipheï, fils d`Allon, fils de Jedaja, fils de Schimri, fils de Schemaeja. 
\verse Ceux-là, désignés par leurs noms, étaient princes dans leurs familles, et leurs maisons paternelles prirent un grand accroissement. 
\verse Ils allèrent du côté de Guedor jusqu`à l`orient de la vallée, afin de chercher des pâturages pour leurs troupeaux. 
\verse Ils trouvèrent de gras et bons pâturages, et un pays vaste, tranquille et paisible, car ceux qui l`habitaient auparavant descendaient de Cham. 
\verse Ces hommes, inscrits par leurs noms, arrivèrent du temps d`Ézéchias, roi de Juda; ils attaquèrent leurs tentes et les Maonites qui se trouvaient là, ils les dévouèrent par interdit jusqu`à ce jour, et ils s`établirent à leur place, car il y avait là des pâturages pour leurs troupeaux. 
\verse Il y eut aussi des fils de Siméon qui allèrent à la montagne de Séir, au nombre de cinq cents hommes. Ils avaient à leur tête Pelathia, Nearia, Rephaja et Uziel, fils de Jischeï. 
\verse Ils battirent le reste des réchappés d`Amalek, et ils s`établirent là jusqu`à ce jour. 

\chapter
\verse Fils de Ruben, premier-né d`Israël. -Car il était le premier-né; mais, parce qu`il souilla la couche de son père, son droit d`aînesse fut donné aux fils de Joseph, fils d`Israël; toutefois Joseph ne dut pas être enregistré dans les généalogies comme premier-né. 
\verse Juda fut, à la vérité, puissant parmi ses frères, et de lui est issu un prince; mais le droit d`aînesse est à Joseph. 
\verse Fils de Ruben, premier-né d`Israël: Hénoc, Pallu, Hetsron et Carmi. 
\verse Fils de Joël: Schemaeja, son fils; Gog, son fils; Schimeï, son fils; 
\verse Michée, son fils; Reaja, son fils; Baal, son fils; 
\verse Beéra, son fils, que Tilgath Pilnéser, roi d`Assyrie, emmena captif: il était prince des Rubénites. 
\verse Frères de Beéra, d`après leurs familles, tels qu`ils sont enregistrés dans les généalogies selon leurs générations: le premier, Jeïel; Zacharie; 
\verse Béla, fils d`Azaz, fils de Schéma, fils de Joël. Béla habitait à Aroër, et jusqu`à Nebo et à Baal Meon; 
\verse à l`orient, il habitait jusqu`à l`entrée du désert depuis le fleuve de l`Euphrate, car leurs troupeaux étaient nombreux dans le pays de Galaad. 
\verse Du temps de Saül, ils firent la guerre aux Hagaréniens, qui tombèrent entre leurs mains; et ils habitèrent dans leurs tentes, sur tout le côté oriental de Galaad. 
\verse Les fils de Gad habitaient vis-à-vis d`eux, dans le pays de Basan, jusqu`à Salca. 
\verse Joël, le premier, Schapham, le second, Jaenaï, et Schaphath, en Basan. 
\verse Leurs frères, d`après les maisons de leurs pères: Micaël, Meschullam, Schéba, Joraï, Jaecan, Zia et Éber, sept. 
\verse Voici les fils d`Abichaïl, fils de Huri, fils de Jaroach, fils de Galaad, fils de Micaël, fils de Jeschischaï, fils de Jachdo, fils de Buz; 
\verse Achi, fils d`Abdiel, fils de Guni, était chef des maisons de leurs pères. 
\verse Ils habitaient en Galaad, en Basan, et dans les villes de leur ressort, et dans toutes les banlieues de Saron jusqu`à leurs extrémités. 
\verse Ils furent tous enregistrés dans les généalogies, du temps de Jotham, roi de Juda, et du temps de Jéroboam, roi d`Israël. 
\verse Les fils de Ruben, les Gadites et la demi-tribu de Manassé avaient de vaillants hommes, portant le bouclier et l`épée, tirant de l`arc, et exercés à la guerre, au nombre de quarante-quatre mille sept cent soixante, en état d`aller à l`armée. 
\verse Ils firent la guerre aux Hagaréniens, à Jethur, à Naphisch et à Nodab. 
\verse Ils reçurent du secours contre eux, et les Hagaréniens et tous ceux qui étaient avec eux furent livrés entre leurs mains. Car, pendant le combat, ils avaient crié à Dieu, qui les exauça, parce qu`ils s`étaient confiés en lui. 
\verse Ils prirent leurs troupeaux, cinquante mille chameaux, deux cent cinquante mille brebis, deux mille ânes, et cent mille personnes; 
\verse car il y eut beaucoup de morts, parce que le combat venait de Dieu. Et ils s`établirent à leur place jusqu`au temps où ils furent emmenés captifs. 
\verse Les fils de la demi-tribu de Manassé habitaient dans le pays, depuis Basan jusqu`à Baal Hermon et à Senir, et à la montagne d`Hermon; ils étaient nombreux. 
\verse Voici les chefs des maisons de leurs pères: Épher, Jischeï, Éliel, Azriel, Jérémie, Hodavia et Jachdiel, vaillants hommes, gens de renom, chefs des maisons de leurs pères. 
\verse Mais ils péchèrent contre le Dieu de leurs pères, et ils se prostituèrent après les dieux des peuples du pays, que Dieu avait détruits devant eux. 
\verse Le Dieu d`Israël excita l`esprit de Pul, roi d`Assyrie, et l`esprit de Tilgath Pilnéser, roi d`Assyrie, et Tilgath Pilnéser emmena captifs les Rubénites, les Gadites et la demi-tribu de Manassé, et il les conduisit à Chalach, à Chabor, à Hara, et au fleuve de Gozan, où ils sont demeurés jusqu`à ce jour. 

\chapter
\verse Fils de Lévi: Guerschom, Kehath et Merari. 
\verse Fils de Kehath: Amram, Jitsehar, Hébron et Uziel. 
\verse Fils d`Amram: Aaron et Moïse; et Marie. Fils d`Aaron: Nadab, Abihu, Éléazar et Ithamar. 
\verse Éléazar engendra Phinées; Phinées engendra Abischua; 
\verse Abischua engendra Bukki; Bukki engendra Uzzi; 
\verse Uzzi engendra Zerachja; Zerachja engendra Merajoth; 
\verse Merajoth engendra Amaria; Amaria engendra Achithub; 
\verse Achithub engendra Tsadok; Tsadok engendra Achimaats; 
\verse Achimaats engendra Azaria; Azaria engendra Jochanan; 
\verse Jochanan engendra Azaria, qui exerça le sacerdoce dans la maison que Salomon bâtit à Jérusalem; 
\verse Azaria engendra Amaria; Amaria engendra Achithub; 
\verse Achithub engendra Tsadok; Tsadok engendra Schallum; 
\verse Schallum engendra Hilkija; Hilkija engendra Azaria; 
\verse et Azaria engendra Seraja; Seraja engendra Jehotsadak, 
\verse Jehotsadak s`en alla quand l`Éternel emmena en captivité Juda et Jérusalem par Nebucadnetsar. 
\verse Fils de Lévi: Guerschom, Kehath et Merari. 
\verse Voici les noms des fils de Guerschom: Libni et Schimeï. 
\verse Fils de Kehath: Amram, Jitsehar, Hébron et Uziel. 
\verse Fils de Merari: Machli et Muschi. -Ce sont là les familles de Lévi, selon leurs pères. 
\verse De Guerschom: Libni, son fils; Jachath, son fils; Zimma, son fils; 
\verse Joach, son fils; Iddo, son fils; Zérach, son fils; Jeathraï, son fils. 
\verse Fils de Kehath: Amminadab, son fils; Koré, son fils; Assir, son fils; 
\verse Elkana, son fils; Ebjasaph, son fils; Assir, son fils; 
\verse Thachath, son fils; Uriel, son fils; Ozias, son fils; Saül, son fils. 
\verse Fils d`Elkana: Amasaï et Achimoth; 
\verse Elkana, son fils; Elkana Tsophaï, son fils; Nachath, son fils; 
\verse Éliab, son fils; Jerocham, son fils; Elkana, son fils; 
\verse et les fils de Samuel, le premier-né Vaschni et Abija. 
\verse Fils de Merari: Machli; Libni, son fils; Schimeï, son fils; Uzza, son fils; 
\verse Schimea, son fils; Hagguija, son fils; Asaja, son fils. 
\verse Voici ceux que David établit pour la direction du chant dans la maison de l`Éternel, depuis que l`arche eut un lieu de repos: 
\verse ils remplirent les fonctions de chantres devant le tabernacle, devant la tente d`assignation, jusqu`à ce que Salomon eût bâti la maison de l`Éternel à Jérusalem, et ils faisaient leur service d`après la règle qui leur était prescrite. 
\verse Voici ceux qui officiaient avec leurs fils. -D`entre les fils des Kehathites: Héman, le chantre, fils de Joël, fils de Samuel, 
\verse fils d`Elkana, fils de Jerocham, fils d`Éliel, fils de Thoach, 
\verse fils de Tsuph, fils d`Elkana, fils de Machath, fils d`Amasaï, 
\verse fils d`Elkana, fils de Joël, fils d`Azaria, fils de Sophonie, 
\verse fils de Thachath, fils d`Assir, fils d`Ebjasaph, fils de Koré, 
\verse fils de Jitsehar, fils de Kehath, fils de Lévi, fils d`Israël. - 
\verse Son frère Asaph, qui se tenait à sa droite, Asaph, fils de Bérékia, fils de Schimea, 
\verse fils de Micaël, fils de Baaséja, fils de Malkija, 
\verse fils d`Ethni, fils de Zérach, fils d`Adaja, 
\verse fils d`Éthan, fils de Zimma, fils de Schimeï, 
\verse fils de Jachath, fils de Guerschom, fils de Lévi. - 
\verse Fils de Merari, leurs frères, à la gauche; Éthan, fils de Kischi, fils d`Abdi, fils de Malluc, 
\verse fils de Haschabia, fils d`Amatsia, fils de Hilkija, 
\verse fils d`Amtsi, fils de Bani, fils de Schémer, 
\verse fils de Machli, fils de Muschi, fils de Merari, fils de Lévi. 
\verse Leurs frères, les Lévites, étaient chargés de tout le service du tabernacle, de la maison de Dieu. 
\verse Aaron et ses fils offraient les sacrifices sur l`autel des holocaustes et l`encens sur l`autel des parfums, ils remplissaient toutes les fonctions dans le lieu très saint, et faisaient l`expiation pour Israël, selon tout ce qu`avait ordonné Moïse, serviteur de Dieu. 
\verse Voici les fils d`Aaron: Éléazar, son fils; Phinées, son fils: Abischua, son fils; 
\verse Bukki, son fils; Uzzi, son fils; Zerachja, son fils; 
\verse Merajoth, son fils; Amaria, son fils; Achithub, son fils; 
\verse Tsadok, son fils; Achimaats, son fils. 
\verse Voici leurs habitations, selon leurs enclos, dans les limites qui leur furent assignées. Aux fils d`Aaron de la famille des Kehathites, indiqués les premiers par le sort, 
\verse on donna Hébron, dans le pays de Juda, et la banlieue qui l`entoure; 
\verse mais le territoire de la ville et ses villages furent accordés à Caleb, fils de Jephunné. 
\verse Aux fils d`Aaron on donna la ville de refuge Hébron, Libna et sa banlieue, Jatthir, Eschthemoa et sa banlieue, 
\verse Hilen et sa banlieue, Debir et sa banlieue, 
\verse Aschan et sa banlieue, Beth Schémesch et sa banlieue; 
\verse et de la tribu de Benjamin, Guéba et sa banlieue, Allémeth et sa banlieue, Anathoth et sa banlieue. Total de leurs villes: treize villes, d`après leurs familles. 
\verse Les autres fils de Kehath eurent par le sort dix villes des familles de la tribu d`Éphraïm, de la tribu de Dan et de la demi-tribu de Manassé. 
\verse Les fils de Guerschom, d`après leurs familles, eurent treize villes de la tribu d`Issacar, de la tribu d`Aser, de la tribu de Nephthali et de la tribu de Manassé en Basan. 
\verse Les fils de Merari, d`après leurs familles, eurent par le sort douze villes de la tribu de Ruben, de la tribu de Gad et de la tribu de Zabulon. 
\verse Les enfants d`Israël donnèrent aux Lévites les villes et leurs banlieues. 
\verse Ils donnèrent par le sort, de la tribu des fils de Juda, de la tribu des fils de Siméon et de la tribu des fils de Benjamin, ces villes qu`ils désignèrent nominativement. 
\verse Et pour les autres familles des fils de Kehath les villes de leur territoire furent de la tribu d`Éphraïm. 
\verse Ils leur donnèrent la ville de refuge Sichem et sa banlieue, dans la montagne d`Éphraïm, Guézer et sa banlieue, 
\verse Jokmeam et sa banlieue, Beth Horon et sa banlieue, 
\verse Ajalon et sa banlieue, et Gath Rimmon et sa banlieue; 
\verse et de la demi-tribu de Manassé, Aner et sa banlieue, et Bileam et sa banlieue, pour la famille des autres fils de Kehath. 
\verse On donna aux fils de Guerschom: de la famille de la demi-tribu de Manassé, Golan en Basan et sa banlieue, et Aschtaroth et sa banlieue; 
\verse de la tribu d`Issacar, Kédesch et sa banlieue, Dobrath et sa banlieue, 
\verse Ramoth et sa banlieue, et Anem et sa banlieue; 
\verse de la tribu d`Aser, Maschal et sa banlieue, Abdon et sa banlieue, 
\verse Hukok et sa banlieue, et Rehob et sa banlieue; 
\verse et de la tribu de Nephthali, Kédesch en Galilée et sa banlieue, Hammon et sa banlieue, et Kirjathaïm et sa banlieue. 
\verse On donna au reste des Lévites, aux fils de Merari: de la tribu de Zabulon, Rimmono et sa banlieue, et Thabor et sa banlieue; 
\verse et de l`autre côté du Jourdain, vis-à-vis de Jéricho, à l`orient du Jourdain: de la tribu de Ruben, Betser au désert et sa banlieue, Jahtsa et sa banlieue, 
\verse Kedémoth et sa banlieue, et Méphaath et sa banlieue; 
\verse et de la tribu de Gad, Ramoth en Galaad et sa banlieue, Mahanaïm et sa banlieue, 
\verse Hesbon et sa banlieue, et Jaezar et sa banlieue. 

\chapter
\verse Fils d`Issacar: Thola, Pua, Jaschub et Schimron, quatre. 
\verse Fils de Thola: Uzzi, Rephaja, Jeriel, Jachmaï, Jibsam et Samuel, chef des maisons de leurs pères, de Thola, vaillants hommes dans leurs générations; leur nombre, du temps de David, était de vingt-deux mille six cents. 
\verse Fils d`Uzzi: Jizrachja. Fils de Jizrachja: Micaël, Abdias, Joël, Jischija, en tout cinq chefs; 
\verse ils avaient avec eux, selon leurs générations, selon les maisons de leurs pères, trente-six mille hommes de troupes armées pour la guerre, car ils avaient beaucoup de femmes et de fils. 
\verse Leurs frères, d`après toutes les familles d`Issacar, hommes vaillants, formaient un total de quatre-vingt-sept mille, enregistrés dans les généalogies. 
\verse Fils de Benjamin: Béla, Béker et Jediaël, trois. 
\verse Fils de Béla: Etsbon, Uzzi, Uziel, Jerimoth et Iri, cinq chefs des maisons de leurs pères, hommes vaillants, et enregistrés dans les généalogies au nombre de vingt-deux mille trente-quatre. - 
\verse Fils de Béker: Zemira, Joasch, Éliézer, Eljoénaï, Omri, Jerémoth, Abija, Anathoth et Alameth, tous ceux-là fils de Béker, 
\verse et enregistrés dans les généalogies, selon leurs générations, comme chefs des maisons de leurs pères, hommes vaillants au nombre de vingt mille deux cents. - 
\verse Fils de Jediaël: Bilhan. Fils de Bilhan: Jeusch, Benjamin, Éhud, Kenaana, Zéthan, Tarsis et Achischachar, 
\verse tous ceux-là fils de Jediaël, chefs des maisons de leurs pères, hommes vaillants au nombre de dix-sept mille deux cents, en état de porter les armes et d`aller à la guerre. 
\verse Schuppim et Huppim, fils d`Ir; Huschim, fils d`Acher. 
\verse Fils de Nephthali: Jahtsiel, Guni, Jetser et Schallum, fils de Bilha. 
\verse Fils de Manassé: Asriel, qu`enfanta sa concubine syrienne; elle enfanta Makir, père de Galaad. 
\verse Makir prit une femme de Huppim et de Schuppim. Le nom de sa soeur était Maaca. Le nom du second fils était Tselophchad; et Tselophchad eut des filles. 
\verse Maaca, femme de Makir, enfanta un fils, et l`appela du nom de Péresch; le nom de son frère était Schéresch, et ses fils étaient Ulam et Rékem. 
\verse Fils d`Ulam: Bedan. Ce sont là les fils de Galaad, fils de Makir, fils de Manassé. 
\verse Sa soeur Hammoléketh enfanta Ischhod, Abiézer et Machla. 
\verse Les fils de Schemida étaient: Achjan, Sichem, Likchi et Aniam. 
\verse Fils d`Éphraïm: Schutélach; Béred, son fils; Thachath, son fils; Éleada, son fils; Thachath, son fils; 
\verse Zabad, son fils; Schutélach, son fils; Ézer et Élead. Les hommes de Gath, nés dans le pays, les tuèrent, parce qu`ils étaient descendus pour prendre leurs troupeaux. 
\verse Éphraïm, leur père, fut longtemps dans le deuil, et ses frères vinrent pour le consoler. 
\verse Puis il alla vers sa femme, et elle conçut et enfanta un fils; il l`appela du nom de Beria, parce que le malheur était dans sa maison. 
\verse Il eut pour fille Schééra, qui bâtit Beth Horon la basse et Beth Horon la haute, et Uzzen Schééra. 
\verse Réphach, son fils, et Réscheph; Thélach, son fils; Thachan, son fils; 
\verse Laedan, son fils; Ammihud, son fils; Élischama, son fils; 
\verse Nun, son fils; Josué, son fils. 
\verse Ils avaient en propriété et pour habitations Béthel et les villes de son ressort; à l`orient, Naaran; à l`occident, Guézer et les villes de son ressort, Sichem et les villes de son ressort, jusqu`à Gaza et aux villes de son ressort. 
\verse Les fils de Manassé possédaient Beth Schean et les villes de son ressort, Thaanac et les villes de son ressort, Meguiddo et les villes de son ressort, Dor et les villes de son ressort. Ce fut dans ces villes qu`habitèrent les fils de Joseph, fils d`Israël. 
\verse Fils d`Aser: Jimna, Jischva, Jischvi et Beria; et Sérach, leur soeur. 
\verse Fils de Beria: Héber et Malkiel. Malkiel fut père de Birzavith. 
\verse Et Héber engendra Japhleth, Schomer et Hotham, et Schua, leur soeur. - 
\verse Fils de Japhleth: Pasac, Bimhal et Aschvath. Ce sont là les fils de Japhleth. - 
\verse Fils de Schamer: Achi, Rohega, Hubba et Aram. - 
\verse Fils d`Hélem, son frère: Tsophach, Jimna, Schélesch et Amal. 
\verse Fils de Tsophach: Suach, Harnépher, Schual, Béri, Jimra, 
\verse Betser, Hod, Schamma, Schilscha, Jithran et Beéra. 
\verse Fils de Jéther: Jephunné, Pispa et Ara. 
\verse Fils d`Ulla: Arach, Hanniel et Ritsja. - 
\verse Tous ceux-là étaient fils d`Aser, chef des maisons de leurs pères, hommes d`élite et vaillants, chef des princes, enregistrés au nombre de vingt-six mille hommes, en état de porter les armes et d`aller à la guerre. 

\chapter
\verse Benjamin engendra Béla, son premier-né, Aschbel le second, Achrach le troisième, 
\verse Nocha le quatrième, et Rapha le cinquième. 
\verse Les fils de Béla furent: Addar, Guéra, Abihud, 
\verse Abischua, Naaman, Achoach, 
\verse Guéra, Schephuphan et Huram. 
\verse Voici les fils d`Échud, qui étaient chefs de famille parmi les habitants de Guéba, et qui les transportèrent à Manachath: 
\verse Naaman, Achija et Guéra. Guéra, qui les transporta, engendra Uzza et Achichud. 
\verse Schacharaïm eut des enfants au pays de Moab, après qu`il eut renvoyé Huschim et Baara, ses femmes. 
\verse Il eut de Hodesch, sa femme: Jobab, Tsibja, Méscha, Malcam, 
\verse Jeuts, Schocja et Mirma. Ce sont là ses fils, chefs de famille. 
\verse Il eut de Huschim: Abithub et Elpaal. 
\verse Fils d`Elpaal: Éber, Mischeam, et Schémer, qui bâtit Ono, Lod et les villes de son ressort. 
\verse Beria et Schéma, qui étaient chefs de famille parmi les habitants d`Ajalon, mirent en fuite les habitants de Gath. 
\verse Achjo, Schaschak, Jerémoth, 
\verse Zebadja, Arad, Éder, 
\verse Micaël, Jischpha et Jocha étaient fils de Beria. - 
\verse Zebadja, Meschullam, Hizki, Héber, 
\verse Jischmeraï, Jizlia et Jobab étaient fils d`Elpaal. - 
\verse Jakim, Zicri, Zabdi, 
\verse Éliénaï, Tsilthaï, Éliel, 
\verse Adaja, Beraja et Schimrath étaient fils de Schimeï. - 
\verse Jischpan, Éber, Éliel, 
\verse Abdon, Zicri, Hanan, 
\verse Hanania, Élam, Anthothija, 
\verse Jiphdeja et Penuel étaient fils de Schaschak. - 
\verse Schamscheraï, Schecharia, Athalia, 
\verse Jaaréschia, Élija et Zicri étaient fils de Jerocham. - 
\verse Ce sont là des chefs de famille, chefs selon leurs générations. Ils habitaient à Jérusalem. 
\verse Le père de Gabaon habitait à Gabaon, et le nom de sa femme était Maaca. 
\verse Abdon, son fils premier-né, puis Tsur, Kis, Baal, Nadab, 
\verse Guedor, Achjo, et Zéker. 
\verse Mikloth engendra Schimea. Ils habitaient aussi à Jérusalem près de leurs frères, avec leurs frères. - 
\verse Ner engendra Kis; Kis engendra Saül; Saül engendra Jonathan, Malki Schua, Abinadab et Eschbaal. 
\verse Fils de Jonathan: Merib Baal. Merib Baal engendra Michée. 
\verse Fils de Michée: Pithon, Mélec, Thaeréa et Achaz. 
\verse Achaz engendra Jehoadda; Jehoadda engendra Alémeth, Azmaveth et Zimri; Zimri engendra Motsa; 
\verse Motsa engendra Binea. Rapha, son fils; Éleasa, son fils; Atsel, son fils; 
\verse Atsel eut six fils, dont voici les noms: Azrikam, Bocru, Ismaël, Schearia, Abdias et Hanan. Tous ceux-là étaient fils d`Atsel. - 
\verse Fils d`Éschek, son frère: Ulam, son premier-né, Jéusch le second, et Éliphéleth le troisième. 
\verse Les fils d`Ulam furent de vaillants hommes, tirant de l`arc; et ils eurent beaucoup de fils et de petits-fils, cent cinquante. Tous ceux-là sont des fils de Benjamin. 

\chapter
\verse Tout Israël est enregistré dans les généalogies et inscrit dans le livre des rois d`Israël. Et Juda fut emmené captif à Babylone, à cause de ses infidélités. 
\verse Les premiers habitants qui demeuraient dans leurs possessions, dans leurs villes, étaient les Israélites, les sacrificateurs, les Lévites, et les Néthiniens. 
\verse A Jérusalem habitaient des fils de Juda, des fils de Benjamin, et des fils d`Éphraïm et de Manassé. - 
\verse Des fils de Pérets, fils de Juda: Uthaï, fils d`Ammihud, fils d`Omri, fils d`Imri, fils de Bani. 
\verse Des Schilonites: Asaja, le premier-né, et ses fils. 
\verse Des fils de Zérach: Jeuel, et ses frères, six cent quatre-vingt-dix. - 
\verse Des fils de Benjamin: Sallu, fils de Meschullam, fils d`Hodavia, fils d`Assenua; 
\verse Jibneja, fils de Jerocham; Éla, fils d`Uzzi, fils de Micri; Meschullam, fils de Schephathia, fils de Reuel, fils de Jibnija; 
\verse et leurs frères, selon leurs générations, neuf cent cinquante-six. Tous ces hommes étaient chefs de famille dans les maisons de leurs pères. 
\verse Des sacrificateurs: Jedaeja; Jehojarib; Jakin; 
\verse Azaria, fils de Hilkija, fils de Meschullam, fils de Tsadok, fils de Merajoth, fils d`Achithub, prince de la maison de Dieu; 
\verse Adaja, fils de Jerocham, fils de Paschhur, fils de Malkija; Maesaï, fils d`Adiel, fils de Jachzéra, fils de Meschullam, fils de Meschillémith, fils d`Immer; 
\verse et leurs frères, chefs des maisons de leurs pères, mille sept cent soixante, hommes vaillants, occupés au service de la maison de Dieu. 
\verse Des Lévites: Schemaeja, fils de Haschub, fils d`Azrikam, fils de Haschabia, des fils de Merari; 
\verse Bakbakkar; Héresch; Galal; Matthania, fils de Michée, fils de Zicri, fils d`Asaph; 
\verse Abdias, fils de Schemaeja, fils de Galal, fils de Jeduthun; Bérékia, fils d`Asa, fils d`Elkana, qui habitait dans les villages des Nethophathiens. 
\verse Et les portiers: Schallum, Akkub, Thalmon, Achiman, et leurs frères; Schallum était le chef, 
\verse et jusqu`à présent il est à la porte du roi, à l`orient. Ce sont là les portiers pour le camp des fils de Lévi. 
\verse Schallum, fils de Koré, fils d`Ébiasaph, fils de Koré, et ses frères de la maison de son père, les Koréites, remplissaient les fonctions de gardiens des seuils de la tente; leurs pères avaient gardé l`entrée du camp de l`Éternel, 
\verse et Phinées, fils d`Éléazar, avait été autrefois leur chef, et l`Éternel était avec lui. 
\verse Zacharie, fils de Meschélémia, était portier à l`entrée de la tente d`assignation. 
\verse Ils étaient en tout deux cent douze, choisis pour portiers des seuils, et enregistrés dans les généalogies d`après leurs villages; David et Samuel le voyant les avaient établis dans leurs fonctions. 
\verse Eux et leurs enfants gardaient les portes de la maison de l`Éternel, de la maison de la tente. 
\verse Il y avait des portiers aux quatre vents, à l`orient, à l`occident, au nord et au midi. 
\verse Leurs frères, qui demeuraient dans leurs villages, devaient de temps à autre venir auprès d`eux pendant sept jours. 
\verse Car ces quatre chefs des portiers, ces Lévites, étaient toujours en fonctions, et ils avaient encore la surveillance des chambres et des trésors de la maison de Dieu; 
\verse ils passaient la nuit autour de la maison de Dieu, dont ils avaient la garde, et qu`ils devaient ouvrir chaque matin. 
\verse Quelques-uns des Lévites prenaient soin des ustensiles du service, qu`ils rentraient en les comptant et sortaient en les comptant. 
\verse D`autres veillaient sur les ustensiles, sur tous les ustensiles du sanctuaire, et sur la fleur de farine, le vin, l`huile, l`encens et les aromates. 
\verse C`étaient des fils de sacrificateurs qui composaient les parfums aromatiques. 
\verse Matthithia, l`un des Lévites, premier-né de Schallum le Koréite, s`occupait des gâteaux cuits sur la plaque. 
\verse Et quelques-uns de leurs frères, parmi les Kehathites, étaient chargés de préparer pour chaque sabbat les pains de proposition. 
\verse Ce sont là les chantres, chefs de famille des Lévites, demeurant dans les chambres, exempts des autres fonctions parce qu`ils étaient à l`oeuvre jour et nuit. 
\verse Ce sont là les chefs de famille des Lévites, chefs selon leurs générations. Ils habitaient à Jérusalem. 
\verse Le père de Gabaon, Jeïel, habitait à Gabaon, et le nom de sa femme était Maaca. 
\verse Abdon, son fils premier-né, puis Tsur, Kis, Baal, Ner, Nadab, 
\verse Guedor, Achjo, Zacharie et Mikloth. 
\verse Mikloth engendra Schimeam. Ils habitaient aussi à Jérusalem près de leurs frères, avec leurs frères. - 
\verse Ner engendra Kis; Kis engendra Saül; Saül engendra Jonathan, Malki Schua, Abinadab et Eschbaal. 
\verse Fils de Jonathan: Merib Baal. Merib Baal engendra Michée. 
\verse Fils de Michée: Pithon, Mélec, et Thachréa. 
\verse Achaz engendra Jaera; Jaera engendra Alémeth, Azmaveth et Zimri; Zimri engendra Motsa; Motsa engendra Binea. 
\verse Rephaja, son fils; Éleasa, son fils; Atsel, son fils. 
\verse Atsel eut six fils, dont voici les noms: Azrikam, Bocru, Ismaël, Scheari, Abdias et Hanan. Ce sont là les fils d`Atsel. 

\chapter
\verse Les Philistins livrèrent bataille à Israël, et les hommes d`Israël prirent la fuite devant les Philistins et tombèrent morts sur la montagne de Guilboa. 
\verse Les Philistins poursuivirent Saül et ses fils, et tuèrent Jonathan, Abinadab et Malki Schua, fils de Saül. 
\verse L`effort du combat porta sur Saül; les archers l`atteignirent et le blessèrent. 
\verse Saül dit alors à celui qui portait ses armes: Tire ton épée, et transperce-m`en, de peur que ces incirconcis ne viennent me faire subir leurs outrages. Celui qui portait ses armes ne voulut pas, car il était saisi de crainte. Et Saül prit son épée, et se jeta dessus. 
\verse Celui qui portait les armes de Saül, le voyant mort, se jeta aussi sur son épée, et mourut. 
\verse Ainsi périrent Saül et ses trois fils, et toute sa maison périt en même temps. 
\verse Tous ceux d`Israël qui étaient dans la vallée, ayant vu qu`on avait fui et que Saül et ses fils étaient morts, abandonnèrent leurs villes pour prendre aussi la fuite. Et les Philistins allèrent s`y établir. 
\verse Le lendemain, les Philistins vinrent pour dépouiller les morts, et ils trouvèrent Saül et ses fils tombés sur la montagne de Guilboa. 
\verse Ils le dépouillèrent, et emportèrent sa tête et ses armes. Puis ils firent annoncer ces bonnes nouvelles par tout le pays des Philistins à leurs idoles et au peuple. 
\verse Ils mirent les armes de Saül dans la maison de leur dieu, et ils attachèrent son crâne dans le temple de Dagon. 
\verse Tout Jabès en Galaad ayant appris tout ce que les Philistins avaient fait à Saül, 
\verse tous les hommes vaillants se levèrent, prirent le corps de Saül et ceux de ses fils, et les transportèrent à Jabès. Ils enterrèrent leur os sous le térébinthe, à Jabès, et ils jeûnèrent sept jours. 
\verse Saül mourut, parce qu`il se rendit coupable d`infidélité envers l`Éternel, dont il n`observa point la parole, et parce qu`il interrogea et consulta ceux qui évoquent les morts. 
\verse Il ne consulta point l`Éternel; alors l`Éternel le fit mourir, et transféra la royauté à David, fils d`Isaï. 

\chapter
\verse Tout Israël s`assembla auprès de David à Hébron, en disant: Voici, nous sommes tes os et ta chair. 
\verse Autrefois déjà, même lorsque Saül était roi, c`était toi qui conduisais et qui ramenais Israël. L`Éternel, ton Dieu, t`a dit: Tu paîtras mon peuple d`Israël, et tu seras le chef de mon peuple d`Israël. 
\verse Ainsi tous les anciens d`Israël vinrent auprès du roi à Hébron, et David fit alliance avec eux à Hébron, devant l`Éternel. Ils oignirent David pour roi sur Israël, selon la parole de l`Éternel, prononcée par Samuel. 
\verse David marcha avec tout Israël sur Jérusalem, qui est Jebus. Là étaient les Jébusiens, habitants du pays. 
\verse Les habitants de Jebus dirent à David: Tu n`entreras point ici. Mais David s`empara de la forteresse de Sion: c`est la cité de David. 
\verse David avait dit: Quiconque battra le premier les Jébusiens sera chef et prince. Joab, fils de Tseruja, monta le premier, et il devint chef. 
\verse David s`établit dans la forteresse; c`est pourquoi on l`appela cité de David. 
\verse Il fit tout autour de la ville des constructions, depuis Millo et aux environs; et Joab répara le reste de la ville. 
\verse David devenait de plus en plus grand, et l`Éternel des armées était avec lui. 
\verse Voici les chefs des vaillants hommes qui étaient au service de David, et qui l`aidèrent avec tout Israël à assurer sa domination, afin de l`établir roi, selon la parole de l`Éternel au sujet d`Israël. 
\verse Voici, d`après leur nombre, les vaillants hommes qui étaient au service de David. Jaschobeam, fils de Hacmoni, l`un des principaux officiers. Il brandit sa lance sur trois cents hommes, qu`il fit périr en une seule fois. 
\verse Après lui, Éléazar, fils de Dodo, l`Achochite, l`un des trois guerriers. 
\verse Il était avec David à Pas Dammim, où les Philistins s`étaient rassemblés pour combattre. Il y avait là une pièce de terre remplie d`orge; et le peuple fuyait devant les Philistins. 
\verse Ils se placèrent au milieu du champ, le protégèrent, et battirent les Philistins. Et l`Éternel opéra une grande délivrance. 
\verse Trois des trente chefs descendirent auprès de David sur le rocher dans la caverne d`Adullam, lorsque le camp des Philistins était dressé dans la vallée des Rephaïm. 
\verse David était alors dans la forteresse, et il y avait un poste de Philistins à Bethléhem. 
\verse David eut un désir, et il dit: Qui me fera boire de l`eau de la citerne qui est à la porte de Bethléhem? 
\verse Alors les trois hommes passèrent au travers du camp des Philistins, et puisèrent de l`eau de la citerne qui est à la porte de Bethléhem. Ils l`apportèrent et la présentèrent à David; mais David ne voulut pas la boire, et il la répandit devant l`Éternel. 
\verse Il dit: Que mon Dieu me garde de faire cela! Boirais-je le sang de ces hommes qui sont allés au péril de leur vie? Car c`est au péril de leur vie qu`ils l`ont apportée. Et il ne voulut pas la boire. Voilà ce que firent ces trois vaillants hommes. 
\verse Abischaï, frère de Joab, était le chef des trois. Il brandit sa lance sur trois cents hommes, et les tua; et il eut du renom parmi les trois. 
\verse Il était le plus considéré des trois de la seconde série, et il fut leur chef; mais il n`égala pas les trois premiers. 
\verse Benaja, fils de Jehojada, fils d`un homme de Kabtseel, rempli de valeur et célèbre par ses exploits. Il frappa les deux lions de Moab. Il descendit au milieu d`une citerne, où il frappa un lion, un jour de neige. 
\verse Il frappa un Égyptien d`une stature de cinq coudées et ayant à la main une lance comme une ensouple de tisserand; il descendit contre lui avec un bâton, arracha la lance de la main de l`Égyptien, et s`en servit pour le tuer. 
\verse Voilà ce que fit Benaja, fils de Jehojada; et il eut du renom parmi les trois vaillants hommes. 
\verse Il était le plus considéré des trente; mais il n`égala pas les trois premiers. David l`admit dans son conseil secret. 
\verse Hommes vaillants de l`armée: Asaël, frère de Joab. Elchanan, fils de Dodo, de Bethléhem. 
\verse Schammoth, d`Haror. Hélets, de Palon. 
\verse Ira, fils d`Ikkesch, de Tekoa. Abiézer, d`Anathoth. 
\verse Sibbecaï, le Huschatite. Ilaï, d`Achoach. 
\verse Maharaï, de Nethopha. Héled, fils de Baana, de Nethopha. 
\verse Ithaï, fils de Ribaï, de Guibea des fils de Benjamin. Benaja, de Pirathon. 
\verse Huraï, de Nachalé Gaasch. Abiel, d`Araba. 
\verse Azmaveth, de Bacharum. Éliachba, de Schaalbon. 
\verse Bené Haschem, de Guizon. Jonathan, fils de Schagué, d`Harar. 
\verse Achiam, fils de Sacar, d`Harar. Éliphal, fils d`Ur. 
\verse Hépher, de Mekéra. Achija, de Palon. 
\verse Hetsro, de Carmel. Naaraï, fils d`Ezbaï. 
\verse Joël, frère de Nathan. Mibchar, fils d`Hagri. 
\verse Tsélek, l`Ammonite. Nachraï, de Béroth, qui portait les armes de Joab, fils de Tseruja. 
\verse Ira, de Jéther. Gareb, de Jéther. 
\verse Urie, le Héthien. Zabad, fils d`Achlaï. 
\verse Adina, fils de Schiza, le Rubénite, chef des Rubénites, et trente avec lui. 
\verse Hanan, fils de Maaca. Josaphat, de Mithni. 
\verse Ozias, d`Aschtharoth. Schama et Jehiel, fils de Hotham, d`Aroër. 
\verse Jediaël, fils de Schimri. Jocha, son frère, le Thitsite. 
\verse Éliel, de Machavim, Jeribaï et Joschavia, fils d`Elnaam. Jithma, le Moabite. 
\verse Éliel, Obed et Jaasiel Metsobaja. 

\chapter
\verse Voici ceux qui se rendirent auprès de David à Tsiklag, lorsqu`il était encore éloigné de la présence de Saül, fils de Kis. Ils faisaient partie des vaillants hommes qui lui prêtèrent leur secours pendant la guerre. 
\verse C`étaient des archers, lançant des pierres de la main droite et de la main gauche, et tirant des flèches avec leur arc: ils étaient de Benjamin, du nombre des frères de Saül. 
\verse Le chef Achiézer et Joas, fils de Schemaa, de Guibea; Jeziel, et Péleth, fils d`Azmaveth; Beraca; Jéhu, d`Anathoth; 
\verse Jischmaeja, de Gabaon, vaillant parmi les trente et chef des trente; Jérémie; Jachaziel; Jochanan; Jozabad, de Guedéra; 
\verse Éluzaï; Jerimoth; Bealia; Schemaria; Schephathia, de Haroph; 
\verse Elkana, Jischija, Azareel, Joézer et Jaschobeam, Koréites; 
\verse Joéla et Zebadia, fils de Jerocham, de Guedor. 
\verse Parmi les Gadites, des hommes vaillants partirent pour se rendre auprès de David dans la forteresse du désert, des soldats exercés à la guerre, armés du bouclier et de la lance, semblables à des lions, et aussi prompts que des gazelles sur les montagnes. 
\verse Ézer, le chef; Abdias, le second; Éliab, le troisième; 
\verse Mischmanna, le quatrième; Jérémie, le cinquième; 
\verse Attaï, le sixième; Éliel, le septième; 
\verse Jochanan, le huitième; Elzabad, le neuvième; 
\verse Jérémie, le dixième; Macbannaï, le onzième. 
\verse C`étaient des fils de Gad, chefs de l`armée; un seul, le plus petit, pouvait s`attaquer à cent hommes, et le plus grand à mille. 
\verse Voilà ceux qui passèrent le Jourdain au premier mois, lorsqu`il débordait sur toutes ses rives, et qui mirent en fuite tous les habitants des vallées, à l`orient et à l`occident. 
\verse Il y eut aussi des fils de Benjamin et de Juda qui se rendirent auprès de David dans la forteresse. 
\verse David sortit au-devant d`eux, et leur adressa la parole, en disant: Si vous venez à moi dans de bonnes intentions pour me secourir, mon coeur s`unira à vous; mais si c`est pour me tromper au profit de mes ennemis, quand je ne commets aucune violence, que le Dieu de nos pères le voie et qu`il fasse justice! 
\verse Amasaï, l`un des principaux officiers, fut revêtu de l`esprit, et dit: Nous sommes à toi, David, et avec toi, fils d`Isaï! Paix, paix à toi, et paix à ceux qui te secourent, car ton Dieu t`a secouru! Et David les accueillit, et les plaça parmi les chefs de la troupe. 
\verse Des hommes de Manassé se joignirent à David, lorsqu`il alla faire la guerre à Saül avec les Philistins. Mais ils ne furent pas en aide aux Philistins; car, après s`être consultés, les princes des Philistins renvoyèrent David, en disant: Il passerait du côté de son maître Saül, au péril de nos têtes. 
\verse Quand il retourna à Tsiklag, voici ceux de Manassé qui se joignirent à lui: Adnach, Jozabad, Jediaël, Micaël, Jozabad, Élihu et Tsilthaï, chefs des milliers de Manassé. 
\verse s prêtèrent leur secours à David contre la troupe [des pillards Amalécites], car ils étaient tous de vaillants hommes, et ils furent chefs dans l`armée. 
\verse Et de jour en jour des gens arrivaient auprès de David pour le secourir, jusqu`à ce qu`il eût un grand camp, comme un camp de Dieu. 
\verse Voici le nombre des hommes armés pour la guerre qui se rendirent auprès de David à Hébron, afin de lui transférer la royauté de Saül, selon l`ordre de l`Éternel. 
\verse Fils de Juda, portant le bouclier et la lance, six mille huit cents, armés pour la guerre. 
\verse Des fils de Siméon, hommes vaillants à la guerre, sept mille cent. 
\verse Des fils de Lévi, quatre mille six cents; 
\verse et Jehojada, prince d`Aaron, et avec lui trois mille sept cents; 
\verse et Tsadok, vaillant jeune homme, et la maison de son père, vingt-deux chefs. 
\verse Des fils de Benjamin, frères de Saül, trois mille; car jusqu`alors la plus grande partie d`entre eux étaient restés fidèles à la maison de Saül. 
\verse Des fils d`Éphraïm, vingt mille huit cents, hommes vaillants, gens de renom, d`après les maisons de leurs pères. 
\verse De la demi-tribu de Manassé, dix-huit mille, qui furent nominativement désignés pour aller établir roi David. 
\verse Des fils d`Issacar, ayant l`intelligence des temps pour savoir ce que devait faire Israël, deux cents chefs, et tous leurs frères sous leurs ordres. 
\verse De Zabulon, cinquante mille, en état d`aller à l`armée, munis pour le combat de toutes les armes de guerre, et prêts à livrer bataille d`un coeur résolu. 
\verse De Nephthali, mille chefs, et avec eux trente-sept mille, portant le bouclier et la lance. 
\verse Des Danites, armés pour la guerre, vingt-huit mille six cents. 
\verse D`Aser, en état d`aller à l`armée et prêts à combattre: quarante mille. 
\verse Et de l`autre côté du Jourdain, des Rubénites, des Gadites, et de la demi-tribu de Manassé, avec toutes les armes de guerre, cent vingt mille. 
\verse Tous ces hommes, gens de guerre, prêts à combattre, arrivèrent à Hébron en sincérité de coeur pour établir David roi sur tout Israël. Et tout le reste d`Israël était également unanime pour faire régner David. 
\verse Ils furent là trois jours avec David, mangeant et buvant, car leurs frères leur avaient préparé des vivres. 
\verse Et même ceux qui habitaient près d`eux jusqu`à Issacar, à Zabulon et à Nephthali, apportaient des aliments sur des ânes, sur des chameaux, sur des mulets et sur des boeufs, des mets de farine, des masses de figues sèches et de raisins secs, du vin, de l`huile, des boeufs et des brebis en abondance, car Israël était dans la joie. 

\chapter
\verse David tint conseil avec les chefs de milliers et de centaines, avec tous les princes. 
\verse Et David dit à toute l`assemblée d`Israël: Si vous le trouvez bon, et si cela vient de l`Éternel, notre Dieu, envoyons de tous côtés vers nos frères qui restent dans toutes les contrées d`Israël, et aussi vers les sacrificateurs et les Lévites dans les villes où sont leurs banlieues, afin qu`ils se réunissent à nous, 
\verse et ramenons auprès de nous l`arche de notre Dieu, car nous ne nous en sommes pas occupés du temps de Saül. 
\verse Toute l`assemblée décida de faire ainsi, car la chose parut convenable à tout le peuple. 
\verse David assembla tout Israël, depuis le Schichor d`Égypte jusqu`à l`entrée de Hamath, pour faire venir de Kirjath Jearim l`arche de Dieu. 
\verse Et David, avec tout Israël, monta à Baala, à Kirjath Jearim, qui est à Juda, pour faire monter de là l`arche de Dieu, devant laquelle est invoqué le nom de l`Éternel qui réside entre les chérubins. 
\verse Ils mirent sur un char neuf l`arche de Dieu, qu`ils emportèrent de la maison d`Abinadab: Uzza et Achjo conduisaient le char. 
\verse David et tout Israël dansaient devant Dieu de toute leur force, en chantant, et en jouant des harpes, des luths, des tambourins, des cymbales et des trompettes. 
\verse Lorsqu`ils furent arrivés à l`aire de Kidon, Uzza étendit la main pour saisir l`arche, parce que les boeufs la faisaient pencher. 
\verse La colère de l`Éternel s`enflamma contre Uzza, et l`Éternel le frappa parce qu`il avait étendu la main sur l`arche. Uzza mourut là, devant Dieu. 
\verse David fut irrité de ce que l`Éternel avait frappé Uzza d`un tel châtiment. Et ce lieu a été appelé jusqu`à ce jour Pérets Uzza. 
\verse David eut peur de Dieu en ce jour-là, et il dit: Comment ferais-je entrer chez moi l`arche de Dieu? 
\verse David ne retira pas l`arche chez lui dans la cité de David, et il la fit conduire dans la maison d`Obed Édom de Gath. 
\verse L`arche de Dieu resta trois mois dans la maison d`Obed Édom, dans sa maison. Et l`Éternel bénit la maison d`Obed Édom et tout ce qui lui appartenait. 

\chapter
\verse Hiram, roi de Tyr, envoya des messagers à David, et du bois de cèdre, et des tailleurs de pierres et des charpentiers, pour lui bâtir une maison. 
\verse David reconnut que l`Éternel l`affermissait comme roi d`Israël, et que son royaume était haut élevé, à cause de son peuple d`Israël. 
\verse David prit encore des femmes à Jérusalem, et il engendra encore des fils et des filles. 
\verse Voici les noms de ceux qui lui naquirent à Jérusalem: Schammua, Schobab, Nathan, Salomon, 
\verse Jibhar, Élischua, Elphéleth, 
\verse Noga, Népheg, Japhia, 
\verse Élischama, Beéliada et Éliphéleth. 
\verse Les Philistins apprirent que David avait été oint pour roi sur tout Israël, et ils montèrent tous à sa recherche. David, qui en fut informé, sortit au-devant d`eux. 
\verse Les Philistins arrivèrent, et se répandirent dans la vallée des Rephaïm. 
\verse David consulta Dieu, en disant: Monterai-je contre les Philistins, et les livreras-tu entre mes mains? Et l`Éternel lui dit: Monte, et je les livrerai entre tes mains. 
\verse Ils montèrent à Baal Peratsim, où David les battit. Puis il dit: Dieu a dispersé mes ennemis par ma main, comme des eaux qui s`écoulent. C`est pourquoi l`on a donné à ce lieu le nom de Baal Peratsim. 
\verse Ils laissèrent là leurs dieux, qui furent brûlés au feu d`après l`ordre de David. 
\verse Les Philistins se répandirent de nouveau dans la vallée. 
\verse David consulta encore Dieu. Et Dieu lui dit: Tu ne monteras pas après eux; détourne-toi d`eux, et tu arriveras sur eux vis-à-vis des mûriers. 
\verse Quand tu entendras un bruit de pas dans les cimes des mûriers, alors tu sortiras pour combattre, car c`est Dieu qui marche devant toi pour battre l`armée des Philistins. 
\verse David fit ce que Dieu lui avait ordonné, et l`armée des Philistins fut battue depuis Gabaon jusqu`à Guézer. 
\verse La renommée de David se répandit dans tous les pays, et l`Éternel le rendit redoutable à toutes les nations. 

\chapter
\verse David se bâtit des maisons dans la cité de David; il prépara une place à l`arche de Dieu, et dressa pour elle une tente. 
\verse Alors David dit: L`arche de Dieu ne doit être portée que par les Lévites, car l`Éternel les a choisis pour porter l`arche de Dieu et pour en faire le service à toujours. 
\verse Et David assembla tout Israël à Jérusalem pour faire monter l`arche de l`Éternel à la place qu`il lui avait préparée. 
\verse David assembla les fils d`Aaron et les Lévites: 
\verse des fils de Kehath, Uriel le chef et ses frères, cent vingt; 
\verse des fils de Merari, Asaja le chef et ses frères, deux cent vingt; 
\verse des fils de Guerschom, Joël le chef et ses frères, cent trente; 
\verse des fils d`Élitsaphan, Schemaeja le chef et ses frères, deux cents; 
\verse des fils d`Hébron, Éliel le chef et ses frères, quatre-vingts; 
\verse des fils d`Uziel, Amminadab le chef et ses frères, cent douze. 
\verse David appela les sacrificateurs Tsadok et Abiathar, et les Lévites Uriel, Asaja, Joël, Schemaeja, Éliel et Amminadab. 
\verse Il leur dit: Vous êtes les chefs de famille des Lévites; sanctifiez-vous, vous et vos frères, et faites monter à la place que je lui ai préparée l`arche de l`Éternel, du Dieu d`Israël. 
\verse Parce que vous n`y étiez pas la première fois, l`Éternel, notre Dieu, nous a frappés; car nous ne l`avons pas cherché selon la loi. 
\verse Les sacrificateurs et les Lévites se sanctifièrent pour faire monter l`arche de l`Éternel, du Dieu d`Israël. 
\verse Les fils des Lévites portèrent l`arche de Dieu sur leurs épaules avec des barres, comme Moïse l`avait ordonné d`après la parole de l`Éternel. 
\verse Et David dit aux chefs des Lévites de disposer leurs frères les chantres avec des instruments de musique, des luths, des harpes et des cymbales, qu`ils devaient faire retentir de sons éclatants en signe de réjouissance. 
\verse Les Lévites disposèrent Héman, fils de Joël; parmi ses frères, Asaph, fils de Bérékia; et parmi les fils de Merari, leurs frères, Éthan, fils de Kuschaja; 
\verse puis avec eux leurs frères du second ordre Zacharie, Ben, Jaaziel, Schemiramoth, Jehiel, Unni, Éliab, Benaja, Maaséja, Matthithia, Éliphelé et Miknéja, et Obed Édom et Jeïel, les portiers. 
\verse Les chantres Héman, Asaph et Éthan avaient des cymbales d`airain, pour les faire retentir. 
\verse Zacharie, Aziel, Schemiramoth, Jehiel, Unni, Éliab, Maaséja et Benaja avaient des luths sur alamoth; 
\verse et Matthithia, Éliphelé, Miknéja, Obed Édom, Jeïel et Azazia, avaient des harpes à huit cordes, pour conduire le chant. 
\verse Kenania, chef de musique parmi les Lévites, dirigeait la musique, car il était habile. 
\verse Bérékia et Elkana étaient portiers de l`arche. 
\verse Schebania, Josaphat, Nethaneel, Amasaï, Zacharie, Benaja et Éliézer, les sacrificateurs, sonnaient des trompettes devant l`arche de Dieu. Obed Édom et Jechija étaient portiers de l`arche. 
\verse David, les anciens d`Israël, et les chefs de milliers se mirent en route pour faire monter l`arche de l`alliance de l`Éternel depuis la maison d`Obed Édom, au milieu des réjouissances. 
\verse Ce fut avec l`assistance de Dieu que les Lévites portèrent l`arche de l`alliance de l`Éternel; et l`on sacrifia sept taureaux et sept béliers. 
\verse David était revêtu d`un manteau de byssus; il en était de même de tous les Lévites qui portaient l`arche, des chantres, et de Kenania, chef de musique parmi les chantres; et David avait sur lui un éphod de lin. 
\verse Tout Israël fit monter l`arche de l`alliance de l`Éternel avec des cris de joie, au son des clairons, des trompettes et des cymbales, et en faisant retentir les luths et les harpes. 
\verse Comme l`arche de l`alliance de l`Éternel entrait dans la cité de David, Mical, fille de Saül, regardait par la fenêtre, et voyant le roi David sauter et danser, elle le méprisa dans son coeur. 

\chapter
\verse Après qu`on eut amené l`arche de Dieu, on la plaça au milieu de la tente que David avait dressée pour elle, et l`on offrit devant Dieu des holocaustes et des sacrifices d`actions de grâces. 
\verse Quand David eut achevé d`offrir les holocaustes et les sacrifices d`actions de grâces, il bénit le peuple au nom de l`Éternel. 
\verse Puis il distribua à tous ceux d`Israël, hommes et femmes, à chacun un pain, une portion de viande et un gâteau de raisins. 
\verse Il remit à des Lévites la charge de faire le service devant l`arche de l`Éternel, d`invoquer, de louer et de célébrer l`Éternel, le Dieu d`Israël. 
\verse C`étaient: Asaph, le chef; Zacharie, le second après lui, Jeïel, Schemiramoth, Jehiel, Matthithia, Éliab, Benaja, Obed Édom et Jeïel. Ils avaient des instruments de musique, des luths et des harpes; et Asaph faisait retentir les cymbales. 
\verse Les sacrificateurs Benaja et Jachaziel sonnaient continuellement des trompettes devant l`arche de l`alliance de Dieu. 
\verse Ce fut en ce jour que David chargea pour la première fois Asaph et ses frères de célébrer les louanges de l`Éternel. 
\verse Louez l`Éternel, invoquez son nom! Faites connaître parmi les peuples ses hauts faits! 
\verse Chantez, chantez en son honneur! Parlez de toutes ses merveilles! 
\verse Glorifiez-vous de son saint nom! Que le coeur de ceux qui cherchent l`Éternel se réjouisse! 
\verse Ayez recours à l`Éternel et à son appui, Cherchez continuellement sa face! 
\verse Souvenez-vous des prodiges qu`il a faits, De ses miracles et des jugements de sa bouche, 
\verse Race d`Israël, son serviteur, Enfants de Jacob, ses élus! 
\verse L`Éternel est notre Dieu; Ses jugements s`exercent sur toute la terre. 
\verse Rappelez-vous à toujours son alliance, Ses promesses pour mille générations, 
\verse L`alliance qu`il a traitée avec Abraham, Et le serment qu`il a fait à Isaac; 
\verse Il l`a érigé pour Jacob en loi, Pour Israël en alliance éternelle, 
\verse Disant: Je te donnerai le pays de Canaan Comme l`héritage qui vous est échu. 
\verse Ils étaient alors peu nombreux, Très peu nombreux, et étrangers dans le pays, 
\verse Et ils allaient d`une nation à l`autre Et d`un royaume vers un autre peuple; 
\verse Mais il ne permit à personne de les opprimer, Et il châtia des rois à cause d`eux: 
\verse Ne touchez pas à mes oints, Et ne faites pas de mal à mes prophètes! 
\verse Chantez à l`Éternel, vous tous habitants de la terre! Annoncez de jour en jour son salut; 
\verse Racontez parmi les nations sa gloire, Parmi tous les peuples ses merveilles! 
\verse Car l`Éternel est grand et très digne de louange, Il est redoutable par-dessus tous les dieux; 
\verse Car tous les dieux des peuples sont des idoles, Et l`Éternel a fait les cieux. 
\verse La majesté et la splendeur sont devant sa face, La force et la joie sont dans sa demeure. 
\verse Familles des peuples, rendez à l`Éternel, Rendez à l`Éternel gloire et honneur! 
\verse Rendez à l`Éternel gloire pour son nom! Apportez des offrandes et venez en sa présence, Prosternez-vous devant l`Éternel avec de saints ornements! 
\verse Tremblez devant lui, vous tous habitants de la terre! Le monde est affermi, il ne chancelle point. 
\verse Que les cieux se réjouissent, et que la terre soit dans l`allégresse! Que l`on dise parmi les nations: L`Éternel règne! 
\verse Que la mer retentisse avec tout ce qu`elle contient! Que la campagne s`égaie avec tout ce qu`elle renferme! 
\verse Que les arbres des forêts poussent des cris de joie Devant l`Éternel! Car il vient pour juger la terre. 
\verse Louez l`Éternel, car il est bon, Car sa miséricorde dure à toujours! 
\verse Dites: Sauve-nous, Dieu de notre salut, Rassemble-nous, et retire-nous du milieu des nations, Afin que nous célébrions ton saint nom Et que nous mettions notre gloire à te louer! 
\verse Béni soit l`Éternel, le Dieu d`Israël, D`éternité en éternité! Et que tout le peuple dise: Amen! Louez l`Éternel! 
\verse David laissa là, devant l`arche de l`alliance de l`Éternel, Asaph et ses frères, afin qu`ils fussent continuellement de service devant l`arche, remplissant leur tâche jour par jour. 
\verse Il laissa Obed Édom et Hosa avec leurs frères, au nombre de soixante-huit, Obed Édom, fils de Jeduthun, et Hosa, comme portiers. 
\verse Il établit le sacrificateur Tsadok et les sacrificateurs, ses frères, devant le tabernacle de l`Éternel, sur le haut lieu qui était à Gabaon, 
\verse pour qu`ils offrissent continuellement à l`Éternel des holocaustes, matin et soir, sur l`autel des holocaustes, et qu`ils accomplissent tout ce qui est écrit dans la loi de l`Éternel, imposée par l`Éternel à Israël. 
\verse Auprès d`eux étaient Héman et Jeduthun, et les autres qui avaient été choisis et désignés par leurs noms pour louer l`Éternel. Car sa miséricorde dure à toujours. 
\verse Auprès d`eux étaient Héman et Jeduthun, avec des trompettes et des cymbales pour ceux qui les faisaient retentir, et avec des instruments pour les cantiques en l`honneur de Dieu. Les fils de Jeduthun étaient portiers. 
\verse Tout le peuple s`en alla chacun dans sa maison, et David s`en retourna pour bénir sa maison. 

\chapter
\verse Lorsque David fut établi dans sa maison, il dit à Nathan le prophète: Voici, j`habite dans une maison de cèdre, et l`arche de l`alliance de l`Éternel est sous une tente. 
\verse Nathan répondit à David: Fais tout ce que tu as dans le coeur, car Dieu est avec toi. 
\verse La nuit suivante, la parole de Dieu fut adressée à Nathan: 
\verse Va dire à mon serviteur David: Ainsi parle l`Éternel: Ce ne sera pas toi qui me bâtiras une maison pour que j`en fasse ma demeure. 
\verse Car je n`ai point habité dans une maison depuis le jour où j`ai fait monter Israël jusqu`à ce jour; mais j`ai été de tente en tente et de demeure en demeure. 
\verse Partout où j`ai marché avec tout Israël, ai-je dit un mot à quelqu`un des juges d`Israël à qui j`avais ordonné de paître mon peuple, ai-je dit: Pourquoi ne me bâtissez-vous pas une maison de cèdre? 
\verse Maintenant tu diras à mon serviteur David: Ainsi parle l`Éternel des armées: Je t`ai pris au pâturage, derrière les brebis, pour que tu fusses chef de mon peuple d`Israël; 
\verse j`ai été avec toi partout où tu as marché, j`ai exterminé tous tes ennemis devant toi, et j`ai rendu ton nom semblable au nom des grands qui sont sur la terre; 
\verse j`ai donné une demeure à mon peuple d`Israël, et je l`ai planté pour qu`il y soit fixé et ne soit plus agité, pour que les méchants ne le détruisent plus comme auparavant 
\verse et comme à l`époque où j`avais établi des juges sur mon peuple d`Israël. J`ai humilié tous tes ennemis. Et je t`annonce que l`Éternel te bâtira une maison. 
\verse Quand tes jours seront accomplis et que tu iras auprès de tes pères, j`élèverai ta postérité après toi, l`un de tes fils, et j`affermirai son règne. 
\verse Ce sera lui qui me bâtira une maison, et j`affermirai pour toujours son trône. 
\verse Je serai pour lui un père, et il sera pour moi un fils; et je ne lui retirerai point ma grâce, comme je l`ai retirée à celui qui t`a précédé. 
\verse Je l`établirai pour toujours dans ma maison et dans mon royaume, et son trône sera pour toujours affermi. 
\verse Nathan rapporta à David toutes ces paroles et toute cette vision. 
\verse Et le roi David alla se présenter devant l`Éternel, et dit: Qui suis-je, Éternel Dieu, et quelle est ma maison, pour que tu m`aies fait parvenir où je suis? 
\verse C`est peu de chose à tes yeux, ô Dieu! Tu parles de la maison de ton serviteur pour les temps à venir. Et tu daignes porter les regards sur moi à la manière des hommes, toi qui es élevé, Éternel Dieu! 
\verse Que pourrait te dire encore David sur la gloire accordée à ton serviteur? Tu connais ton serviteur. 
\verse O Éternel! c`est à cause de ton serviteur, et selon ton coeur, que tu as fait toutes ces grandes choses, pour les lui révéler. 
\verse O Éternel! nul n`est semblable à toi et il n`y a point d`autre Dieu que toi, d`après tout ce que nous avons entendu de nos oreilles. 
\verse Est-il sur la terre une seule nation qui soit comme ton peuple d`Israël, que Dieu est venu racheter pour en former son peuple, pour te faire un nom et pour accomplir des miracles et des prodiges, en chassant des nations devant ton peuple que tu as racheté d`Égypte? 
\verse Tu as établi ton peuple d`Israël, pour qu`il fût ton peuple à toujours; et toi, Éternel, tu es devenu son Dieu. 
\verse Maintenant, ô Éternel! que la parole que tu as prononcée sur ton serviteur et sur sa maison subsiste éternellement, et agis selon ta parole! 
\verse Qu`elle subsiste, afin que ton nom soit à jamais glorifié et que l`on dise: L`Éternel des armées, le Dieu d`Israël, est un Dieu pour Israël! Et que la maison de David, ton serviteur, soit affermie devant toi! 
\verse Car toi-même, ô mon Dieu, tu as révélé à ton serviteur que tu lui bâtirais une maison. C`est pourquoi ton serviteur a osé prier devant toi. 
\verse Maintenant, ô Éternel! tu es Dieu, et tu as annoncé cette grâce à ton serviteur. 
\verse Veuille donc bénir la maison de ton serviteur, afin qu`elle subsiste à toujours devant toi! Car ce que tu bénis, ô Éternel! est béni pour l`éternité. 

\chapter
\verse Après cela, David battit les Philistins et les humilia, et il enleva de la main des Philistins Gath et les villes de son ressort. 
\verse Il battit les Moabites, et les Moabites furent assujettis à David et lui payèrent un tribut. 
\verse David battit Hadarézer, roi de Tsoba, vers Hamath, lorsqu`il alla établir sa domination sur le fleuve de l`Euphrate. 
\verse David lui prit mille chars, sept mille cavaliers, et vingt mille hommes de pied; il coupa les jarrets à tous les chevaux de trait, et ne conserva que cent attelages. 
\verse Les Syriens de Damas vinrent au secours d`Hadarézer, roi de Tsoba, et David battit vingt-deux mille Syriens. 
\verse David mit des garnisons dans la Syrie de Damas. Et les Syriens furent assujettis à David, et lui payèrent un tribut. L`Éternel protégeait David partout où il allait. 
\verse Et David prit les boucliers d`or qu`avaient les serviteurs d`Hadarézer, et les apporta à Jérusalem. 
\verse David prit encore une grande quantité d`airain à Thibchath et à Cun, villes d`Hadarézer. Salomon en fit la mer d`airain, les colonnes et les ustensiles d`airain. 
\verse Thohu, roi de Hamath, apprit que David avait battu toute l`armée d`Hadarézer, roi de Tsoba, 
\verse et il envoya Hadoram, son fils, vers le roi David, pour le saluer, et pour le féliciter d`avoir attaqué Hadarézer et de l`avoir battu. Car Thohu était en guerre avec Hadarézer. Il envoya aussi toutes sortes de vases d`or, d`argent et d`airain. 
\verse Le roi David les consacra à l`Éternel, avec l`argent et l`or qu`il avait pris sur toutes les nations, sur Édom, sur Moab, sur les fils d`Ammon, sur les Philistins et sur Amalek. 
\verse Abischaï, fils de Tseruja, battit dans la vallée du sel dix-huit mille Édomites. 
\verse Il mit des garnisons dans Édom, et tout Édom fut assujetti à David. L`Éternel protégeait David partout où il allait. 
\verse David régna sur tout Israël, et il faisait droit et justice à tout son peuple. 
\verse Joab, fils de Tseruja, commandait l`armée; Josaphat, fils d`Achilud, était archiviste; 
\verse Tsadok, fils d`Achithub, et Abimélec, fils d`Abiathar, étaient sacrificateurs; Schavscha était secrétaire; 
\verse Benaja, fils de Jehojada, était chef des Kéréthiens et des Péléthiens; et les fils de David étaient les premiers auprès du roi. 

\chapter
\verse Après cela, Nachasch, roi des fils d`Ammon, mourut, et son fils régna à sa place. 
\verse David dit: Je montrerai de la bienveillance à Hanun, fils de Nachasch, car son père en a montré à mon égard. Et David envoya des messagers pour le consoler au sujet de son père. Lorsque les serviteurs de David arrivèrent dans le pays des fils d`Ammon auprès de Hanun, pour le consoler, 
\verse les chefs des fils d`Ammon dirent à Hanun: Penses-tu que ce soit pour honorer ton père que David t`envoie des consolateurs? N`est-ce pas pour reconnaître la ville et pour la détruire, et pour explorer le pays, que ses serviteurs sont venus auprès de toi? 
\verse Alors Hanun saisit les serviteurs de David, les fit raser, et fit couper leurs habits par le milieu jusqu`au haut des cuisses. Puis il les congédia. 
\verse David, que l`on vint informer de ce qui était arrivé à ces hommes, envoya des gens à leur rencontre, car ils étaient dans une grande confusion; et le roi leur fit dire: Restez à Jéricho jusqu`à ce que votre barbe ait repoussé, et revenez ensuite. 
\verse Les fils d`Ammon virent qu`ils s`étaient rendus odieux à David, et Hanun et les fils d`Ammon envoyèrent mille talents d`argent pour prendre à leur solde des chars et des cavaliers chez les Syriens de Mésopotamie et chez les Syriens de Maaca et de Tsoba. 
\verse Ils prirent à leur solde trente-deux mille chars et le roi de Maaca avec son peuple, lesquels vinrent camper devant Médeba. Les fils d`Ammon se rassemblèrent de leurs villes, et marchèrent au combat. 
\verse A cette nouvelle, David envoya contre eux Joab et toute l`armée, les hommes vaillants. 
\verse Les fils d`Ammon sortirent, et se rangèrent en bataille à l`entrée de la ville; les rois qui étaient venus prirent position séparément dans la campagne. 
\verse Joab vit qu`il avait à combattre par devant et par derrière. Il choisit alors sur toute l`élite d`Israël un corps, qu`il opposa aux Syriens; 
\verse et il plaça sous le commandement de son frère Abischaï le reste du peuple, pour faire face aux fils d`Ammon. 
\verse Il dit: Si les Syriens sont plus forts que moi, tu viendras à mon secours; et si les fils d`Ammon sont plus forts que toi, j`irai à ton secours. 
\verse Sois ferme, et montrons du courage pour notre peuple et pour les villes de notre Dieu, et que l`Éternel fasse ce qui lui semblera bon! 
\verse Joab, avec son peuple, s`avança pour attaquer les Syriens, et ils s`enfuirent devant lui. 
\verse Et quand les fils d`Ammon virent que les Syriens avaient pris la fuite, ils s`enfuirent aussi devant Abischaï, frère de Joab, et rentrèrent dans la ville. Et Joab revint à Jérusalem. 
\verse Les Syriens, voyant qu`ils avaient été battus par Israël, envoyèrent chercher les Syriens qui étaient de l`autre côté du fleuve; et Schophach, chef de l`armée d`Hadarézer, était à leur tête. 
\verse On l`annonça à David, qui assembla tout Israël, passa le Jourdain, marcha contre eux, et se prépara à les attaquer. David se rangea en bataille contre les Syriens. Mais les Syriens, après s`être battus avec lui, s`enfuirent devant Israël. 
\verse David leur tua les troupes de sept mille chars et quarante mille hommes de pied, et il fit mourir Schophach, chef de l`armée. 
\verse Les serviteurs d`Hadarézer, se voyant battus par Israël, firent la paix avec David et lui furent assujettis. Et les Syriens ne voulurent plus secourir les fils d`Ammon. 

\chapter
\verse L`année suivante, au temps où les rois se mettaient en campagne, Joab, à la tête d`une forte armée, alla ravager le pays des fils d`Ammon et assiéger Rabba. Mais David resta à Jérusalem. Joab battit Rabba et la détruisit. 
\verse David enleva la couronne de dessus la tête de son roi, et la trouva du poids d`un talent d`or: elle était garnie de pierres précieuses. On la mit sur la tête de David, qui emporta de la ville un très grand butin. 
\verse Il fit sortir les habitants, et il les mit en pièces avec des scies, des herses de fer et des haches; il traita de même toutes les villes des fils d`Ammon. David retourna à Jérusalem avec tout le peuple. 
\verse Après cela, il y eut une bataille à Guézer avec les Philistins. Alors Sibbecaï, le Huschatite, tua Sippaï, l`un des enfants de Rapha. Et les Philistins furent humiliés. 
\verse Il y eut encore une bataille avec les Philistins. Et Elchanan, fils de Jaïr, tua le frère de Goliath, Lachmi de Gath, qui avait une lance dont le bois était comme une ensouple de tisserand. 
\verse Il y eut encore une bataille à Gath. Il s`y trouva un homme de haute taille, qui avait six doigts à chaque main et à chaque pied, vingt-quatre en tout, et qui était aussi issu de Rapha. 
\verse Il jeta un défi à Israël; et Jonathan, fils de Schimea, frère de David, le tua. 
\verse Ces hommes étaient des enfants de Rapha à Gath. Ils périrent par la main de David et par la main de ses serviteurs. 

\chapter
\verse Satan se leva contre Israël, et il excita David à faire le dénombrement d`Israël. 
\verse Et David dit à Joab et aux chefs du peuple: Allez, faites le dénombrement d`Israël, depuis Beer Schéba jusqu`à Dan, et rapportez-le-moi, afin que je sache à combien il s`élève. 
\verse Joab répondit: Que l`Éternel rende son peuple cent fois plus nombreux! O roi mon seigneur, ne sont-ils pas tous serviteurs de mon seigneur? Mais pourquoi mon seigneur demande-t-il cela? Pourquoi faire ainsi pécher Israël? 
\verse Le roi persista dans l`ordre qu`il donnait à Joab. Et Joab partit, et parcourut tout Israël; puis il revint à Jérusalem. 
\verse Joab remit à David le rôle du dénombrement du peuple: il y avait dans tout Israël onze cent mille hommes tirant l`épée, et en Juda quatre cent soixante-dix mille hommes tirant l`épée. 
\verse Il ne fit point parmi eux le dénombrement de Lévi et de Benjamin, car l`ordre du roi lui paraissait une abomination. 
\verse Cet ordre déplut à Dieu, qui frappa Israël. 
\verse Et David dit à Dieu: J`ai commis un grand péché en faisant cela! Maintenant, daigne pardonner l`iniquité de ton serviteur, car j`ai complètement agi en insensé! 
\verse L`Éternel adressa ainsi la parole à Gad, le voyant de David: 
\verse Va dire à David: Ainsi parle l`Éternel: Je te propose trois fléaux; choisis-en un, et je t`en frapperai. 
\verse Gad alla vers David, et lui dit: Ainsi parle l`Éternel: 
\verse Accepte, ou trois années de famine, ou trois mois pendant lesquels tu seras détruit par tes adversaires et atteint par l`épée de tes ennemis, ou trois jours pendant lesquels l`épée de l`Éternel et la peste seront dans le pays et l`ange de l`Éternel portera la destruction dans tout le territoire d`Israël. Vois maintenant ce que je dois répondre à celui qui m`envoie. 
\verse David répondit à Gad: Je suis dans une grande angoisse! Oh! que je tombe entre les mains de l`Éternel, car ses compassions sont immenses; mais que je ne tombe pas entre les mains des hommes! 
\verse L`Éternel envoya la peste en Israël, et il tomba soixante-dix mille hommes d`Israël. 
\verse Dieu envoya un ange à Jérusalem pour la détruire; et comme il la détruisait, l`Éternel regarda et se repentit de ce mal, et il dit à l`ange qui détruisait: Assez! Retire maintenant ta main. L`ange de l`Éternel se tenait près de l`aire d`Ornan, le Jébusien. 
\verse David leva les yeux, et vit l`ange de l`Éternel se tenant entre la terre et le ciel et ayant à la main son épée nue tournée contre Jérusalem. Alors David et les anciens, couverts de sacs, tombèrent sur leur visage. 
\verse Et David dit à Dieu: N`est-ce pas moi qui ai ordonné le dénombrement du peuple? C`est moi qui ai péché et qui ai fait le mal; mais ces brebis, qu`ont-elles fait? Éternel, mon Dieu, que ta main soit donc sur moi et sur la maison de mon père, et qu`elle ne fasse point une plaie parmi ton peuple! 
\verse L`ange de l`Éternel dit à Gad de parler à David, afin qu`il montât pour élever un autel à l`Éternel dans l`aire d`Ornan, le Jébusien. 
\verse David monta, selon la parole que Gad avait prononcée au nom de l`Éternel. 
\verse Ornan se retourna et vit l`ange, et ses quatre fils se cachèrent avec lui: il foulait alors du froment. 
\verse Lorsque David arriva auprès d`Ornan, Ornan regarda, et il aperçut David; puis il sortit de l`aire, et se prosterna devant David, le visage contre terre. 
\verse David dit à Ornan: Cède-moi l`emplacement de l`aire pour que j`y bâtisse un autel à l`Éternel; cède-le-moi contre sa valeur en argent, afin que la plaie se retire de dessus le peuple. 
\verse Ornan répondit à David: Prends-le, et que mon seigneur le roi fasse ce qui lui semblera bon; vois, je donne les boeufs pour l`holocauste, les chars pour le bois, et le froment pour l`offrande, je donne tout cela. 
\verse Mais le roi David dit à Ornan: Non! je veux l`acheter contre sa valeur en argent, car je ne présenterai point à l`Éternel ce qui est à toi, et je n`offrirai point un holocauste qui ne me coûte rien. 
\verse Et David donna à Ornan six cents sicles d`or pour l`emplacement. 
\verse David bâtit là un autel à l`Éternel, et il offrit des holocaustes et des sacrifices d`actions de grâces. Il invoqua l`Éternel, et l`Éternel lui répondit par le feu, qui descendit du ciel sur l`autel de l`holocauste. 
\verse Alors l`Éternel parla à l`ange, qui remit son épée dans le fourreau. 
\verse A cette époque-là, David, voyant que l`Éternel l`avait exaucé dans l`aire d`Ornan, le Jébusien, y offrait des sacrifices. 
\verse Mais le tabernacle de l`Éternel, construit par Moïse au désert, et l`autel des holocaustes, étaient alors sur le haut lieu de Gabaon. 
\verse David ne pouvait pas aller devant cet autel pour chercher Dieu, parce que l`épée de l`ange de l`Éternel lui avait causé de l`épouvante. 

\chapter
\verse Et David dit: Ici sera la maison de l`Éternel Dieu, et ici sera l`autel des holocaustes pour Israël. 
\verse David fit rassembler les étrangers qui étaient dans le pays d`Israël, et il chargea des tailleurs de pierres de préparer des pierres de taille pour la construction de la maison de Dieu. 
\verse Il prépara aussi du fer en abondance pour les clous des battants des portes et pour les crampons, de l`airain en quantité telle qu`il n`était pas possible de le peser, 
\verse et des bois de cèdre sans nombre, car les Sidoniens et les Tyriens avaient amené à David des bois de cèdre en abondance. 
\verse David disait: Mon fils Salomon est jeune et d`un âge faible, et la maison qui sera bâtie à l`Éternel s`élèvera à un haut degré de renommée et de gloire dans tous les pays; c`est pourquoi je veux faire pour lui des préparatifs. Et David fit beaucoup de préparatifs avant sa mort. 
\verse David appela Salomon, son fils, et lui ordonna de bâtir une maison à l`Éternel, le Dieu d`Israël. 
\verse David dit à Salomon: Mon fils, j`avais l`intention de bâtir une maison au nom de l`Éternel, mon Dieu. 
\verse Mais la parole de l`Éternel m`a été ainsi adressée: Tu as versé beaucoup de sang, et tu as fait de grandes guerres; tu ne bâtiras pas une maison à mon nom, car tu as versé devant moi beaucoup de sang sur la terre. 
\verse Voici, il te naîtra un fils, qui sera un homme de repos, et à qui je donnerai du repos en le délivrant de tous ses ennemis d`alentour; car Salomon sera son nom, et je ferai venir sur Israël la paix et la tranquillité pendant sa vie. 
\verse Ce sera lui qui bâtira une maison à mon nom. Il sera pour moi un fils, et je serai pour lui un père; et j`affermirai pour toujours le trône de son royaume en Israël. 
\verse Maintenant, mon fils, que l`Éternel soit avec toi, afin que tu prospères et que tu bâtisses la maison de l`Éternel, ton Dieu, comme il l`a déclaré à ton égard! 
\verse Veuille seulement l`Éternel t`accorder de la sagesse et de l`intelligence, et te faire régner sur Israël dans l`observation de la loi de l`Éternel, ton Dieu! 
\verse Alors tu prospéreras, si tu as soin de mettre en pratique les lois et les ordonnances que l`Éternel a prescrites à Moïse pour Israël. Fortifie-toi et prends courage, ne crains point et ne t`effraie point. 
\verse Voici, par mes efforts, j`ai préparé pour la maison de l`Éternel cent mille talents d`or, un million de talents d`argent, et une quantité d`airain et de fer qu`il n`est pas possible de peser, car il y en a en abondance; j`ai aussi préparé du bois et des pierres, et tu en ajouteras encore. 
\verse Tu as auprès de toi un grand nombre d`ouvriers, des tailleurs de pierres, et des charpentiers, et des hommes habiles dans toute espèce d`ouvrages. 
\verse L`or, l`argent, l`airain et le fer, sont sans nombre. Lève-toi et agis, et que l`Éternel soit avec toi! 
\verse David ordonna à tous les chefs d`Israël de venir en aide à Salomon, son fils. 
\verse L`Éternel, votre Dieu, n`est-il pas avec vous, et ne vous a-t-il pas donné du repos de tous côtés? Car il a livré entre mes mains les habitants du pays, et le pays est assujetti devant l`Éternel et devant son peuple. 
\verse Appliquez maintenant votre coeur et votre âme à chercher l`Éternel, votre Dieu; levez-vous, et bâtissez le sanctuaire de l`Éternel Dieu, afin d`amener l`arche de l`alliance de l`Éternel et les ustensiles consacrés à Dieu dans la maison qui sera bâtie au nom de l`Éternel. 

\chapter
\verse David, âgé et rassasié de jours, établit Salomon, son fils, roi sur Israël. 
\verse Il assembla tous les chefs d`Israël, les sacrificateurs et les Lévites. 
\verse On fit le dénombrement des Lévites, depuis l`âge de trente ans et au-dessus; comptés par tête et par homme, ils se trouvèrent au nombre de trente-huit mille. 
\verse Et David dit: Qu`il y en ait vingt-quatre mille pour veiller aux offices de la maison de l`Éternel, six mille comme magistrats et juges, 
\verse quatre mille comme portiers, et quatre mille chargés de louer l`Éternel avec les instruments que j`ai faits pour le célébrer. 
\verse David les divisa en classes d`après les fils de Lévi, Guerschon, Kehath et Merari. 
\verse Des Guerschonites: Laedan et Schimeï. - 
\verse Fils de Laedan: le chef Jehiel, Zétham et Joël, trois. 
\verse Fils de Schimeï: Schelomith, Haziel et Haran, trois. Ce sont là les chefs des maisons paternelles de la famille de Laedan. - 
\verse Fils de Schimeï: Jachath, Zina, Jeusch et Beria. Ce sont là les quatre fils de Schimeï. 
\verse Jachath était le chef, et Zina le second; Jeusch et Beria n`eurent pas beaucoup de fils, et ils formèrent une seule maison paternelle dans le dénombrement. 
\verse Fils de Kehath: Amram, Jitsehar, Hébron et Uziel, quatre. - 
\verse Fils d`Amram: Aaron et Moïse. Aaron fut mis à part pour être sanctifié comme très saint, lui et ses fils à perpétuité, pour offrir les parfums devant l`Éternel, pour faire son service, et pour bénir à toujours en son nom. 
\verse Mais les fils de Moïse, homme de Dieu, furent comptés dans la tribu de Lévi. 
\verse Fils de Moïse: Guerschom et Éliézer. 
\verse Fils de Guerschom: Schebuel, le chef. 
\verse Et les fils d`Éliézer furent: Rechabia, le chef; Éliézer n`eut pas d`autre fils, mais les fils de Rechabia furent très nombreux. - 
\verse Fils de Jitsehar: Schelomith, le chef. - 
\verse Fils d`Hébron: Jerija, le chef; Amaria, le second; Jachaziel, le troisième; et Jekameam, le quatrième. - 
\verse Fils d`Uziel: Michée, le chef; et Jischija, le second. 
\verse Fils de Merari: Machli et Muschi. -Fils de Machli: Éléazar et Kis. 
\verse Éléazar mourut sans avoir de fils; mais il eut des filles, que prirent pour femmes les fils de Kis, leurs frères. - 
\verse Fils de Muschi: Machli, Éder et Jerémoth, trois. 
\verse Ce sont là les fils de Lévi, selon leurs maisons paternelles, les chefs des maisons paternelles, d`après le dénombrement qu`on en fit en comptant les noms par tête. Ils étaient employés au service de la maison de l`Éternel, depuis l`âge de vingt ans et au-dessus. 
\verse Car David dit: L`Éternel, le Dieu d`Israël, a donné du repos à son peuple, et il habitera pour toujours à Jérusalem; 
\verse et les Lévites n`auront plus à porter le tabernacle et tous les ustensiles pour son service. 
\verse Ce fut d`après les derniers ordres de David qu`eut lieu le dénombrement des fils de Lévi depuis l`âge de vingt ans et au-dessus. 
\verse Placés auprès des fils d`Aaron pour le service de la maison de l`Éternel, ils avaient à prendre soin des parvis et des chambres, de la purification de toutes les choses saintes, des ouvrages concernant le service de la maison de Dieu, 
\verse des pains de proposition, de la fleur de farine pour les offrandes, des galettes sans levain, des gâteaux cuits sur la plaque et des gâteaux frits, de toutes les mesures de capacité et de longueur: 
\verse ils avaient à se présenter chaque matin et chaque soir, afin de louer et de célébrer l`Éternel, 
\verse et à offrir continuellement devant l`Éternel tous les holocaustes à l`Éternel, aux sabbats, aux nouvelles lunes et aux fêtes, selon le nombre et les usages prescrits. 
\verse Ils donnaient leurs soins à la tente d`assignation, au sanctuaire, et aux fils d`Aaron, leurs frères, pour le service de la maison de l`Éternel. 

\chapter
\verse Voici les classes des fils d`Aaron. Fils d`Aaron: Nadab, Abihu, Éléazar et Ithamar. 
\verse Nadab et Abihu moururent avant leur père, sans avoir de fils; et Éléazar et Ithamar remplirent les fonctions du sacerdoce. 
\verse David divisa les fils d`Aaron en les classant pour le service qu`ils avaient à faire; Tsadok appartenait aux descendants d`Éléazar, et Achimélec aux descendants d`Ithamar. 
\verse Il se trouva parmi les fils d`Éléazar plus de chefs que parmi les fils d`Ithamar, et on en fit la division; les fils d`Éléazar avaient seize chefs de maisons paternelles, et les fils d`Ithamar huit chefs de maisons paternelles. 
\verse On les classa par le sort, les uns avec les autres, car les chefs du sanctuaire et les chefs de de Dieu étaient des fils d`Éléazar et des fils d`Ithamar. 
\verse Schemaeja, fils de Nethaneel, le secrétaire, de la tribu de Lévi, les inscrivit devant le roi et les princes, devant Tsadok, le sacrificateur, et Achimélec, fils d`Abiathar, et devant les chefs des maisons paternelles des sacrificateurs et des Lévites. On tira au sort une maison paternelle pour Éléazar, et on en tira une autre pour Ithamar. 
\verse Le premier sort échut à Jehojarib; le second, à Jedaeja; 
\verse le troisième, à Harim; le quatrième, à Seorim; 
\verse le cinquième, à Malkija; le sixième, à Mijamin; 
\verse le septième, à Hakkots; le huitième, à Abija; 
\verse le neuvième, à Josué; le dixième, à Schecania; 
\verse le onzième, à Éliaschib; le douzième, à Jakim; 
\verse le treizième, à Huppa; le quatorzième, à Jeschébeab; 
\verse le quinzième, à Bilga; le seizième, à Immer; 
\verse le dix-septième, à Hézir; le dix-huitième, à Happitsets; 
\verse le dix-neuvième, à Pethachja; le vingtième, à Ézéchiel; 
\verse le vingt et unième, à Jakin; le vingt-deuxième, à Gamul; 
\verse le vingt-troisième, à Delaja; le vingt-quatrième, à Maazia. 
\verse C`est ainsi qu`ils furent classés pour leur service, afin qu`ils entrassent dans la maison de l`Éternel en se conformant à la règle établie par Aaron, leur père, d`après les ordres que lui avait donnés l`Éternel, le Dieu d`Israël. 
\verse Voici les chefs du reste des Lévites. -Des fils d`Amram: Schubaël; des fils de Schubaël: Jechdia; 
\verse de Rechabia, des fils de Rechabia: le chef Jischija. 
\verse Des Jitseharites: Schelomoth; des fils de Schelomoth: Jachath. 
\verse Fils d`Hébron: Jerija, Amaria le second, Jachaziel le troisième, Jekameam le quatrième. 
\verse Fils d`Uziel: Michée; des fils de Michée: Schamir; 
\verse frère de Michée: Jischija; des fils de Jischija: Zacharie. - 
\verse Fils de Merari: Machli et Muschi, et les fils de Jaazija, son fils. 
\verse Fils de Merari, de Jaazija, son fils: Schoham, Zaccur et Ibri. 
\verse De Machli: Éléazar, qui n`eut point de fils; 
\verse de Kis, les fils de Kis: Jerachmeel. 
\verse Fils de Muschi: Machli, Éder et Jerimoth. Ce sont là les fils de Lévi, selon leurs maisons paternelles. 
\verse Eux aussi, comme leurs frères, les fils d`Aaron, ils tirèrent au sort devant le roi David, Tsadok et Achimélec, et les chefs des maisons paternelles des sacrificateurs et des Lévites. Il en fut ainsi pour chaque chef de maison comme pour le moindre de ses frères. 

\chapter
\verse David et les chefs de l`armée mirent à part pour le service ceux des fils d`Asaph, d`Héman et de Jeduthun qui prophétisaient en s`accompagnant de la harpe, du luth et des cymbales. Et voici le nombre de ceux qui avaient des fonctions à remplir. 
\verse Des fils d`Asaph: Zaccur, Joseph, Nethania et Aschareéla, fils d`Asaph, sous la direction d`Asaph qui prophétisait suivant les ordres du roi. 
\verse De Jeduthun, les fils de Jeduthun: Guedalia, Tseri, Ésaïe, Haschabia, Matthithia et Schimeï, six, sous la direction de leur père Jeduthun qui prophétisait avec la harpe pour louer et célébrer l`Éternel. 
\verse D`Héman, les fils d`Héman: Bukkija, Matthania, Uziel, Schebuel, Jerimoth, Hanania, Hanani, Éliatha, Guiddalthi, Romamthi Ézer, Joschbekascha, Mallothi, Hothir, Machazioth, 
\verse tous fils d`Héman, qui était voyant du roi pour révéler les paroles de Dieu et pour exalter sa puissance; Dieu avait donné à Héman quatorze fils et trois filles. 
\verse Tous ceux-là étaient sous la direction de leurs pères, pour le chant de la maison de l`Éternel, et avaient des cymbales, des luths et des harpes pour le service de la maison de Dieu. Asaph, Jeduthun et Héman recevaient les ordres du roi. 
\verse Ils étaient au nombre de deux cent quatre-vingt-huit, y compris leurs frères exercés au chant de l`Éternel, tous ceux qui étaient habiles. 
\verse Ils tirèrent au sort pour leurs fonctions, petits et grands, maîtres et disciples. 
\verse Le premier sort échut, pour Asaph, à Joseph; le second, à Guedalia, lui, ses frères et ses fils, douze; 
\verse le troisième, à Zaccur, ses fils et ses frères, douze; 
\verse le quatrième, à Jitseri, ses fils et ses frères, douze; 
\verse le cinquième, à Nethania, ses fils et ses frères, douze; 
\verse le sixième, à Bukkija, ses fils et ses frères, douze; 
\verse le septième, à Jesareéla, ses fils et ses frères, douze; 
\verse le huitième, à Ésaïe, ses fils et ses frères, douze; 
\verse le neuvième, à Matthania, ses fils et ses frères, douze; 
\verse le dixième, à Schimeï, ses fils et ses frères, douze; 
\verse le onzième, à Azareel, ses fils et ses frères, douze; 
\verse le douzième, à Haschabia, ses fils et ses frères, douze; 
\verse le treizième, à Schubaël, ses fils et ses frères, douze; 
\verse le quatorzième, à Matthithia, ses fils et ses frères, douze; 
\verse le quinzième, à Jerémoth, ses fils et ses frères, douze; 
\verse le seizième, à Hanania, ses fils et ses frères, douze; 
\verse le dix-septième, à Joschbekascha, ses fils et ses frères, douze; 
\verse le dix-huitième, à Hanani, ses fils et ses frères, douze; 
\verse le dix-neuvième, à Mallothi, ses fils et ses frères, douze; 
\verse le vingtième, à Élijatha, ses fils et ses frères, douze; 
\verse le vingt et unième, à Hothir, ses fils et ses frères, douze; 
\verse le vingt-deuxième, à Guiddalthi, ses fils et ses frères, douze; 
\verse le vingt-troisième, à Machazioth, ses fils et ses frères, douze; 
\verse le vingt-quatrième, à Romamthi Ézer, ses fils et ses frères, douze. 

\chapter
\verse Voici les classes des portiers. Des Koréites: Meschélémia, fils de Koré, d`entre les fils d`Asaph. 
\verse Fils de Meschélémia: Zacharie, le premier-né, Jediaël le second, Zebadia le troisième, Jathniel le quatrième, 
\verse Élam le cinquième, Jochanan le sixième, Eljoénaï le septième. 
\verse Fils d`Obed Édom: Schemaeja, le premier-né, Jozabad le second, Joach le troisième, Sacar le quatrième, Nethaneel le cinquième, 
\verse Ammiel le sixième, Issacar le septième, Peulthaï le huitième; car Dieu l`avait béni. 
\verse A Schemaeja, son fils, naquirent des fils qui dominèrent dans la maison de leur père, car ils étaient de vaillants hommes; 
\verse fils de Schemaeja: Othni, Rephaël, Obed, Elzabad et ses frères, hommes vaillants, Élihu et Semaeja. 
\verse Tous ceux-là étaient des fils d`Obed Édom; eux, leurs fils et leurs frères, étaient des hommes pleins de vigueur et de force pour le service, soixante deux d`Obed Édom. 
\verse Les fils et les frères de Meschélémia, hommes vaillants, étaient au nombre de dix-huit. - 
\verse Des fils de Merari: Hosa, qui avait pour fils: Schimri, le chef, établi chef par son père, quoiqu`il ne fût pas le premier-né, 
\verse Hilkija le second, Thebalia le troisième, Zacharie le quatrième. Tous les fils et les frères de Hosa étaient au nombre de treize. 
\verse A ces classes de portiers, aux chefs de ces hommes et à leurs frères, fut remise la garde pour le service de la maison de l`Éternel. 
\verse Ils tirèrent au sort pour chaque porte, petits et grands, selon leurs maisons paternelles. 
\verse Le sort échut à Schélémia pour le côté de l`orient. On tira au sort pour Zacharie, son fils, qui était un sage conseiller, et le côté du septentrion lui échut par le sort. 
\verse Le côté du midi échut à Obed Édom, et la maison des magasins à ses fils. 
\verse Le côté de l`occident échut à Schuppim et à Hosa, avec la porte Schalléketh, sur le chemin montant: une garde était vis-à-vis de l`autre. 
\verse Il y avait à l`orient six Lévites, au nord quatre par jour, au midi quatre par jour, et quatre aux magasins en deux places différentes; 
\verse du côté du faubourg, à l`occident, quatre vers le chemin, deux vers le faubourg. 
\verse Ce sont là les classes des portiers, d`entre les fils des Koréites et d`entre les fils de Merari. 
\verse L`un des Lévites, Achija, avait l`intendance des trésors de la maison de Dieu et des trésors des choses saintes. 
\verse Parmi les fils de Laedan, les fils des Guerschonites issus de Laedan, chefs des maisons paternelles de Laedan le Guerschonite, c`étaient Jehiéli, 
\verse et les fils de Jehiéli, Zétham et Joël, son frère, qui gardaient les trésors de la maison de l`Éternel. 
\verse Parmi les Amramites, les Jitseharites, les Hébronites et les Uziélites, 
\verse c`était Schebuel, fils de Guerschom, fils de Moïse, qui était intendant des trésors. 
\verse Parmi ses frères issus d`Éliézer, dont le fils fut Rechabia, dont le fils fut Ésaïe, dont le fils fut Joram, dont le fils fut Zicri, dont le fils fut Schelomith, 
\verse c`étaient Schelomith et ses frères qui gardaient tous les trésors des choses saintes qu`avaient consacrées le roi David, les chefs des maisons paternelles, les chefs de milliers et de centaines, et les chefs de l`armée: 
\verse c`était sur le butin pris à la guerre qu`ils les avaient consacrées pour l`entretien de la maison de l`Éternel. 
\verse Tout ce qui avait été consacré par Samuel, le voyant, par Saül, fils de Kis, par Abner, fils de Ner, par Joab, fils de Tseruja, toutes les choses consacrées étaient sous la garde de Schelomith et de ses frères. 
\verse Parmi les Jitseharites, Kenania et ses frères étaient employés pour les affaires extérieures, comme magistrats et juges en Israël. 
\verse Parmi les Hébronites, Haschabia et ses frères, hommes vaillants, au nombre de mille sept cents, avaient la surveillance d`Israël, de l`autre côté du Jourdain, à l`occident, pour toutes les affaires de l`Éternel et pour le service du roi. 
\verse En ce qui concerne les Hébronites, dont Jerija était le chef, on fit, la quarantième année du règne de David, des recherches à leur égard d`après leurs généalogies et leurs maisons paternelles, et l`on trouva parmi eux de vaillants hommes à Jaezer en Galaad. 
\verse Les frères de Jerija, hommes vaillants, étaient au nombre de deux mille sept cents chefs de maisons paternelles. Le roi David les établit sur les Rubénites, sur les Gadites et sur la demi-tribu de Manassé pour toutes les affaires de Dieu et pour les affaires du roi. 

\chapter
\verse Enfants d`Israël selon leur nombre, chefs de maisons paternelles, chefs de milliers et de centaines, et officiers au service du roi pour tout ce qui concernait les divisions, leur arrivée et leur départ, mois par mois, pendant tous les mois de l`année, chaque division étant de vingt-quatre mille hommes. 
\verse A la tête de la première division, pour le premier mois, était Jaschobeam, fils de Zabdiel; et il avait une division de vingt-quatre mille hommes. 
\verse Il était des fils de Pérets, et il commandait tous les chefs des troupes du premier mois. 
\verse A la tête de la division du second mois était Dodaï, l`Achochite; Mikloth était l`un des chefs de sa division; et il avait une division de vingt-quatre mille hommes. 
\verse Le chef de la troisième division, pour le troisième mois, était Benaja, fils du sacrificateur Jehojada, chef; et il avait une division de vingt-quatre mille hommes. 
\verse Ce Benaja était un héros parmi les trente et à la tête des trente; et Ammizadab, son fils, était l`un des chefs de sa division. 
\verse Le quatrième, pour le quatrième mois, était Asaël, frère de Joab, et, après lui, Zebadia, son fils; et il avait une division de vingt-quatre mille hommes. 
\verse Le cinquième, pour le cinquième mois, était le chef Schamehuth, le Jizrachite; et il avait une division de vingt-quatre mille hommes. 
\verse Le sixième, pour le sixième mois, était Ira, fils d`Ikkesch, le Tekoïte; et il avait une division de vingt-quatre mille hommes. 
\verse Le septième, pour le septième mois, était Hélets, le Pelonite, des fils d`Éphraïm; et il avait une division de vingt-quatre mille hommes. 
\verse Le huitième, pour le huitième mois, était Sibbecaï, le Huschatite, de la famille des Zérachites; et il avait une division de vingt-quatre mille hommes. 
\verse Le neuvième, pour le neuvième mois, était Abiézer, d`Anathoth, des Benjamites; et il avait une division de vingt-quatre mille hommes. 
\verse Le dixième, pour le dixième mois, était Maharaï, de Nethopha, de la famille des Zérachites; et il avait une division de vingt-quatre mille hommes. 
\verse Le onzième, pour le onzième mois, était Benaja, de Pirathon, des fils d`Éphraïm; et il avait une division de vingt-quatre mille hommes. 
\verse Le douzième, pour le douzième mois, était Heldaï, de Nethopha, de la famille d`Othniel; et il avait une division de vingt-quatre mille hommes. 
\verse Voici les chefs des tribus d`Israël. Chefs des Rubénites: Éliézer, fils de Zicri; des Siméonites: Schephathia, fils de Maaca; 
\verse des Lévites: Haschabia, fils de Kemuel; de la famille d`Aaron: Tsadok; 
\verse de Juda: Élihu, des frères de David; d`Issacar: Omri, fils de Micaël; 
\verse de Zabulon: Jischemaeja, fils d`Abdias; de Nephthali: Jerimoth, fils d`Azriel; 
\verse des fils d`Éphraïm: Hosée, fils d`Azazia; de la demi-tribu de Manassé: Joël, fils de Pedaja; 
\verse de la demi-tribu de Manassé en Galaad: Jiddo, fils de Zacharie; de Benjamin: Jaasiel, fils d`Abner; 
\verse de Dan: Azareel, fils de Jerocham. Ce sont là les chefs des tribus d`Israël. 
\verse David ne fit point le dénombrement de ceux d`Israël qui étaient âgés de vingt ans et au-dessous, car l`Éternel avait promis de multiplier Israël comme les étoiles du ciel. 
\verse Joab, fils de Tseruja, avait commencé le dénombrement, mais il ne l`acheva pas, l`Éternel s`étant irrité contre Israël à cause de ce dénombrement, qui ne fut point porté parmi ceux des Chroniques du roi David. 
\verse Azmaveth, fils d`Adiel, était préposé sur les trésors du roi; Jonathan, fils d`Ozias, sur les provisions dans les champs, les villes, les villages et les tours; 
\verse Ezri, fils de Kelub, sur les ouvriers de la campagne qui cultivaient la terre; 
\verse Schimeï, de Rama, sur les vignes; Zabdi, de Schepham, sur les provisions de vin dans les vignes; 
\verse Baal Hanan, de Guéder, sur les oliviers et les sycomores dans la plaine; Joasch, sur les provisions d`huile; 
\verse Schithraï, de Saron, sur les boeufs qui paissaient en Saron; Schaphath, fils d`Adlaï, sur les boeufs dans les vallées; 
\verse Obil, l`Ismaélite, sur les chameaux; Jechdia, de Méronoth, sur les ânesses; 
\verse Jaziz, l`Hagarénien, sur les brebis. Tous ceux-là étaient intendants des biens du roi David. 
\verse Jonathan, oncle de David, était conseiller, homme de sens et de savoir; Jehiel, fils de Hacmoni, était auprès des fils du roi; 
\verse Achitophel était conseiller du roi; Huschaï, l`Arkien, était ami du roi; 
\verse après Achitophel, Jehojada, fils de Benaja, et Abiathar, furent conseillers; Joab était chef de l`armée du roi. 

\chapter
\verse David convoqua à Jérusalem tous les chefs d`Israël, les chefs des tribus, les chefs des divisions au service du roi, les chefs de milliers et les chefs de centaines, ceux qui étaient en charge sur tous les biens et les troupeaux du roi et auprès de ses fils, les eunuques, les héros et tous les hommes vaillants. 
\verse Le roi David se leva sur ses pieds, et dit: Écoutez-moi, mes frères et mon peuple! J`avais l`intention de bâtir une maison de repos pour l`arche de l`alliance de l`Éternel et pour le marchepied de notre Dieu, et je me préparais à bâtir. 
\verse Mais Dieu m`a dit: Tu ne bâtiras pas une maison à mon nom, car tu es un homme de guerre et tu as versé du sang. 
\verse L`Éternel, le Dieu d`Israël, m`a choisi dans toute la maison de mon père, pour que je fusse roi d`Israël à toujours; car il a choisi Juda pour chef, il a choisi la maison de mon père dans la maison de Juda, et parmi les fils de mon père c`est moi qu`il a voulu faire régner sur tout Israël. 
\verse Entre tous mes fils-car l`Éternel m`a donné beaucoup de fils-il a choisi mon fils Salomon pour le faire asseoir sur le trône du royaume de l`Éternel, sur Israël. 
\verse Il m`a dit: Salomon, ton fils, bâtira ma maison et mes parvis; car je l`ai choisi pour mon fils, et je serai pour lui un père. 
\verse J`affermirai pour toujours son royaume, s`il reste attaché comme aujourd`hui à la pratique de mes commandements et de mes ordonnances. 
\verse Maintenant, aux yeux de tout Israël, de l`assemblée de l`Éternel, et en présence de notre Dieu qui vous entend, observez et prenez à coeur tous les commandements de l`Éternel, votre Dieu, afin que vous possédiez ce bon pays et que vous le laissiez en héritage à vos fils après vous à perpétuité. 
\verse Et toi, Salomon, mon fils, connais le Dieu de ton père, et sers-le d`un coeur dévoué et d`une âme bien disposée, car l`Éternel sonde tous les coeurs et pénètre tous les desseins et toutes les pensées. Si tu le cherches, il se laissera trouver par toi; mais si tu l`abandonnes, il te rejettera pour toujours. 
\verse Considère maintenant que l`Éternel t`a choisi, afin que tu bâtisses une maison qui serve de sanctuaire. Fortifie-toi et agis. 
\verse David donna à Salomon, son fils, le modèle du portique et des bâtiments, des chambres du trésor, des chambres hautes, des chambres intérieures, et de la chambre du propitiatoire. 
\verse Il lui donna le plan de tout ce qu`il avait dans l`esprit touchant les parvis de la maison de l`Éternel, et toutes les chambres à l`entour pour les trésors de la maison de Dieu et les trésors du sanctuaire, 
\verse et touchant les classes des sacrificateurs et des Lévites, tout ce qui concernait le service de la maison de l`Éternel, et tous les ustensiles pour le service de la maison de l`Éternel. 
\verse Il lui donna le modèle des ustensiles d`or, avec le poids de ce qui devait être d`or, pour tous les ustensiles de chaque service; et le modèle de tous les ustensiles d`argent, avec le poids, pour tous les ustensiles de chaque service. 
\verse Il donna le poids des chandeliers d`or et de leurs lampes d`or, avec le poids de chaque chandelier et de ses lampes; et le poids des chandeliers d`argent, avec le poids de chaque chandelier et de ses lampes, selon l`usage de chaque chandelier. 
\verse Il lui donna l`or au poids pour les tables des pains de proposition, pour chaque table; et de l`argent pour les tables d`argent. 
\verse Il lui donna le modèle des fourchettes, des bassins et des calices d`or pur; le modèle des coupes d`or, avec le poids de chaque coupe, et des coupes d`argent, avec le poids de chaque coupe; 
\verse et le modèle de l`autel des parfums en or épuré, avec le poids. Il lui donna encore le modèle du char, des chérubins d`or qui étendent leurs ailes et couvrent l`arche de l`alliance de l`Éternel. 
\verse C`est par un écrit de sa main, dit David, que l`Éternel m`a donné l`intelligence de tout cela, de tous les ouvrages de ce modèle. 
\verse David dit à Salomon, son fils: Fortifie-toi, prends courage et agis; ne crains point, et ne t`effraie point. Car l`Éternel Dieu, mon Dieu, sera avec toi; il ne te délaissera point, il ne t`abandonnera point, jusqu`à ce que tout l`ouvrage pour le service de la maison de l`Éternel soit achevé. 
\verse Voici les classes des sacrificateurs et des Lévites pour tout le service de la maison de Dieu; et voici près de toi, pour toute l`oeuvre, tous les hommes bien disposés et habiles dans toute espèce d`ouvrages, et les chefs et tout le peuple dociles à tous tes ordres. 

\chapter
\verse Le roi David dit à toute l`assemblée: Mon fils Salomon, le seul que Dieu ait choisi, est jeune et d`un âge faible, et l`ouvrage est considérable, car ce palais n`est pas pour un homme, mais il est pour l`Éternel Dieu. 
\verse J`ai mis toutes mes forces à préparer pour la maison de mon Dieu de l`or pour ce qui doit être d`or, de l`argent pour ce qui doit être d`argent, de l`airain pour ce qui doit être d`airain, du fer pour ce qui doit être de fer, et du bois pour ce qui doit être de bois, des pierres d`onyx et des pierres à enchâsser, des pierres brillantes et de diverses couleurs, toutes sortes de pierres précieuses, et du marbre blanc en quantité. 
\verse De plus, dans mon attachement pour la maison de mon Dieu, je donne à la maison de mon Dieu l`or et l`argent que je possède en propre, outre tout ce que j`ai préparé pour la maison du sanctuaire: 
\verse trois mille talents d`or, d`or d`Ophir, et sept mille talents d`argent épuré, pour en revêtir les parois des bâtiments, 
\verse l`or pour ce qui doit être d`or, et l`argent pour ce qui doit être d`argent, et pour tous les travaux qu`exécuteront les ouvriers. Qui veut encore présenter volontairement aujourd`hui ses offrandes à l`Éternel? 
\verse Les chefs des maisons paternelles, les chefs des tribus d`Israël, les chefs de milliers et de centaines, et les intendants du roi firent volontairement des offrandes. 
\verse Ils donnèrent pour le service de la maison de Dieu cinq mille talents d`or, dix mille dariques, dix mille talents d`argent, dix-huit mille talents d`airain, et cent mille talents de fer. 
\verse Ceux qui possédaient des pierres les livrèrent pour le trésor de la maison de l`Éternel entre les mains de Jehiel, le Guerschonite. 
\verse Le peuple se réjouit de leurs offrandes volontaires, car c`était avec un coeur bien disposé qu`ils les faisaient à l`Éternel; et le roi David en eut aussi une grande joie. 
\verse David bénit l`Éternel en présence de toute l`assemblée. Il dit: Béni sois-tu, d`éternité en éternité, Éternel, Dieu de notre père Israël. 
\verse A toi, Éternel, la grandeur, la force et la magnificence, l`éternité et la gloire, car tout ce qui est au ciel et sur la terre t`appartient; à toi, Éternel, le règne, car tu t`élèves souverainement au-dessus de tout! 
\verse C`est de toi que viennent la richesse et la gloire, c`est toi qui domines sur tout, c`est dans ta main que sont la force et la puissance, et c`est ta main qui a le pouvoir d`agrandir et d`affermir toutes choses. 
\verse Maintenant, ô notre Dieu, nous te louons, et nous célébrons ton nom glorieux. 
\verse Car qui suis-je et qui est mon peuple, pour que nous puissions te faire volontairement ces offrandes? Tout vient de toi, et nous recevons de ta main ce que nous t`offrons. 
\verse Nous sommes devant toi des étrangers et des habitants, comme tous nos pères; nos jours sur la terre sont comme l`ombre, et il n`y a point d`espérance. 
\verse Éternel, notre Dieu, c`est de ta main que viennent toutes ces richesses que nous avons préparées pour te bâtir une maison, à toi, à ton saint nom, et c`est à toi que tout appartient. 
\verse Je sais, ô mon Dieu, que tu sondes le coeur, et que tu aimes la droiture; aussi je t`ai fait toutes ces offrandes volontaires dans la droiture de mon coeur, et j`ai vu maintenant avec joie ton peuple qui se trouve ici t`offrir volontairement ses dons. 
\verse Éternel, Dieu d`Abraham, d`Isaac et d`Israël, nos pères, maintiens à toujours dans le coeur de ton peuple ces dispositions et ces pensées, et affermis son coeur en toi. 
\verse Donne à mon fils Salomon un coeur dévoué à l`observation de tes commandements, de tes préceptes et de tes lois, afin qu`il mette en pratique toutes ces choses, et qu`il bâtisse le palais pour lequel j`ai fait des préparatifs. 
\verse David dit à toute l`assemblée: Bénissez l`Éternel, votre Dieu! Et toute l`assemblée bénit l`Éternel, le Dieu de leurs pères. Ils s`inclinèrent et se prosternèrent devant l`Éternel et devant le roi. 
\verse Le lendemain de ce jour, ils offrirent en sacrifice et en holocauste à l`Éternel mille taureaux, mille béliers et mille agneaux, avec les libations ordinaires, et d`autres sacrifices en grand nombre pour tout Israël. 
\verse Ils mangèrent et burent ce jour-là devant l`Éternel avec une grande joie, ils proclamèrent roi pour la seconde fois Salomon, fils de David, ils l`oignirent devant l`Éternel comme chef, et ils oignirent Tsadok comme sacrificateur. 
\verse Salomon s`assit sur le trône de l`Éternel, comme roi à la place de David, son père. Il prospéra, et tout Israël lui obéit. 
\verse Tous les chefs et les héros, et même tous les fils du roi David se soumirent au roi Salomon. 
\verse L`Éternel éleva au plus haut degré Salomon sous les yeux de tout Israël, et il rendit son règne plus éclatant que ne fut celui d`aucun roi d`Israël avant lui. 
\verse David, fils d`Isaï, régna sur tout Israël. 
\verse Le temps qu`il régna sur Israël fut de quarante ans: à Hébron il régna sept ans, et à Jérusalem il régna trente-trois ans. 
\verse Il mourut dans une heureuse vieillesse, rassasié de jours, de richesse et de gloire. Et Salomon, son fils, régna à sa place. 
\verse Les actions du roi David, les premières et les dernières, sont écrites dans le livre de Samuel le voyant, dans le livre de Nathan, le prophète, et dans le livre de Gad, le prophète, 
\verse avec tout son règne et tous ses exploits, et ce qui s`est passé de son temps, soit en Israël, soit dans tous les royaumes des autres pays. 
