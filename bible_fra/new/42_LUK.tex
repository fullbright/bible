\book[Livre de Tobit]{}


\chapter
\verse Plusieurs ayant entrepris de composer un récit des événements qui se sont accomplis parmi nous, 
\verse suivant ce que nous ont transmis ceux qui ont été des témoins oculaires dès le commencement et sont devenus des ministres de la parole, 
\verse il m`a aussi semblé bon, après avoir fait des recherches exactes sur toutes ces choses depuis leur origine, de te les exposer par écrit d`une manière suivie, excellent Théophile, 
\verse afin que tu reconnaisses la certitude des enseignements que tu as reçus. 
\verse Du temps d`Hérode, roi de Judée, il y avait un sacrificateur, nommé Zacharie, de la classe d`Abia; sa femme était d`entre les filles d`Aaron, et s`appelait Élisabeth. 
\verse Tous deux étaient justes devant Dieu, observant d`une manière irréprochable tous les commandements et toutes les ordonnances du Seigneur. 
\verse Ils n`avaient point d`enfants, parce qu`Élisabeth était stérile; et ils étaient l`un et l`autre avancés en âge. 
\verse Or, pendant qu`il s`acquittait de ses fonctions devant Dieu, selon le tour de sa classe, il fut appelé par le sort, 
\verse d`après la règle du sacerdoce, à entrer dans le temple du Seigneur pour offrir le parfum. 
\verse Toute la multitude du peuple était dehors en prière, à l`heure du parfum. 
\verse Alors un ange du Seigneur apparut à Zacharie, et se tint debout à droite de l`autel des parfums. 
\verse Zacharie fut troublé en le voyant, et la frayeur s`empara de lui. 
\verse Mais l`ange lui dit: Ne crains point, Zacharie; car ta prière a été exaucée. Ta femme Élisabeth t`enfantera un fils, et tu lui donneras le nom de Jean. 
\verse Il sera pour toi un sujet de joie et d`allégresse, et plusieurs se réjouiront de sa naissance. 
\verse Car il sera grand devant le Seigneur. Il ne boira ni vin, ni liqueur enivrante, et il sera rempli de l`Esprit Saint dès le sein de sa mère; 
\verse il ramènera plusieurs des fils d`Israël au Seigneur, leur Dieu; 
\verse il marchera devant Dieu avec l`esprit et la puissance d`Élie, pour ramener les coeurs des pères vers les enfants, et les rebelles à la sagesse des justes, afin de préparer au Seigneur un peuple bien disposé. 
\verse Zacharie dit à l`ange: A quoi reconnaîtrai-je cela? Car je suis vieux, et ma femme est avancée en âge. 
\verse L`ange lui répondit: Je suis Gabriel, je me tiens devant Dieu; j`ai été envoyé pour te parler, et pour t`annoncer cette bonne nouvelle. 
\verse Et voici, tu seras muet, et tu ne pourras parler jusqu`au jour où ces choses arriveront, parce que tu n`as pas cru à mes paroles, qui s`accompliront en leur temps. 
\verse Cependant, le peuple attendait Zacharie, s`étonnant de ce qu`il restait si longtemps dans le temple. 
\verse Quand il sortit, il ne put leur parler, et ils comprirent qu`il avait eu une vision dans le temple; il leur faisait des signes, et il resta muet. 
\verse Lorsque ses jours de service furent écoulés, il s`en alla chez lui. 
\verse Quelque temps après, Élisabeth, sa femme, devint enceinte. Elle se cacha pendant cinq mois, disant: 
\verse C`est la grâce que le Seigneur m`a faite, quand il a jeté les yeux sur moi pour ôter mon opprobre parmi les hommes. 
\verse Au sixième mois, l`ange Gabriel fut envoyé par Dieu dans une ville de Galilée, appelée Nazareth, 
\verse auprès d`une vierge fiancée à un homme de la maison de David, nommé Joseph. Le nom de la vierge était Marie. 
\verse L`ange entra chez elle, et dit: Je te salue, toi à qui une grâce a été faite; le Seigneur est avec toi. 
\verse Troublée par cette parole, Marie se demandait ce que pouvait signifier une telle salutation. 
\verse L`ange lui dit: Ne crains point, Marie; car tu as trouvé grâce devant Dieu. 
\verse Et voici, tu deviendras enceinte, et tu enfanteras un fils, et tu lui donneras le nom de Jésus. 
\verse Il sera grand et sera appelé Fils du Très Haut, et le Seigneur Dieu lui donnera le trône de David, son père. 
\verse Il règnera sur la maison de Jacob éternellement, et son règne n`aura point de fin. 
\verse Marie dit à l`ange: Comment cela se fera-t-il, puisque je ne connais point d`homme? 
\verse L`ange lui répondit: Le Saint Esprit viendra sur toi, et la puissance du Très Haut te couvrira de son ombre. C`est pourquoi le saint enfant qui naîtra de toi sera appelé Fils de Dieu. 
\verse Voici, Élisabeth, ta parente, a conçu, elle aussi, un fils en sa vieillesse, et celle qui était appelée stérile est dans son sixième mois. 
\verse Car rien n`est impossible à Dieu. 
\verse Marie dit: Je suis la servante du Seigneur; qu`il me soit fait selon ta parole! Et l`ange la quitta. 
\verse Dans ce même temps, Marie se leva, et s`en alla en hâte vers les montagnes, dans une ville de Juda. 
\verse Elle entra dans la maison de Zacharie, et salua Élisabeth. 
\verse Dès qu`Élisabeth entendit la salutation de Marie, son enfant tressaillit dans son sein, et elle fut remplie du Saint Esprit. 
\verse Elle s`écria d`une voix forte: Tu es bénie entre les femmes, et le fruit de ton sein est béni. 
\verse Comment m`est-il accordé que la mère de mon Seigneur vienne auprès de moi? 
\verse Car voici, aussitôt que la voix de ta salutation a frappé mon oreille, l`enfant a tressailli d`allégresse dans mon sein. 
\verse Heureuse celle qui a cru, parce que les choses qui lui ont été dites de la part du Seigneur auront leur accomplissement. 
\verse Et Marie dit: Mon âme exalte le Seigneur, 
\verse Et mon esprit se réjouit en Dieu, mon Sauveur, 
\verse Parce qu`il a jeté les yeux sur la bassesse de sa servante. Car voici, désormais toutes les générations me diront bienheureuse, 
\verse Parce que le Tout Puissant a fait pour moi de grandes choses. Son nom est saint, 
\verse Et sa miséricorde s`étend d`âge en âge Sur ceux qui le craignent. 
\verse Il a déployé la force de son bras; Il a dispersé ceux qui avaient dans le coeur des pensées orgueilleuses. 
\verse Il a renversé les puissants de leurs trônes, Et il a élevé les humbles. 
\verse Il a rassasié de biens les affamés, Et il a renvoyé les riches à vide. 
\verse Il a secouru Israël, son serviteur, Et il s`est souvenu de sa miséricorde, - 
\verse Comme il l`avait dit à nos pères, -Envers Abraham et sa postérité pour toujours. 
\verse Marie demeura avec Élisabeth environ trois mois. Puis elle retourna chez elle. 
\verse Le temps où Élisabeth devait accoucher arriva, et elle enfanta un fils. 
\verse Ses voisins et ses parents apprirent que le Seigneur avait fait éclater envers elle sa miséricorde, et ils se réjouirent avec elle. 
\verse Le huitième jour, ils vinrent pour circoncire l`enfant, et ils l`appelaient Zacharie, du nom de son père. 
\verse Mais sa mère prit la parole, et dit: Non, il sera appelé Jean. 
\verse Ils lui dirent: Il n`y a dans ta parenté personne qui soit appelé de ce nom. 
\verse Et ils firent des signes à son père pour savoir comment il voulait qu`on l`appelle. 
\verse Zacharie demanda des tablettes, et il écrivit: Jean est son nom. Et tous furent dans l`étonnement. 
\verse Au même instant, sa bouche s`ouvrit, sa langue se délia, et il parlait, bénissant Dieu. 
\verse La crainte s`empara de tous les habitants d`alentour, et, dans toutes les montagnes de la Judée, on s`entretenait de toutes ces choses. 
\verse Tous ceux qui les apprirent les gardèrent dans leur coeur, en disant: Que sera donc cet enfant? Et la main du Seigneur était avec lui. 
\verse Zacharie, son père, fut rempli du Saint Esprit, et il prophétisa, en ces mots: 
\verse Béni soit le Seigneur, le Dieu d`Israël, De ce qu`il a visité et racheté son peuple, 
\verse Et nous a suscité un puissant Sauveur Dans la maison de David, son serviteur, 
\verse Comme il l`avait annoncé par la bouche de ses saints prophètes des temps anciens, - 
\verse Un Sauveur qui nous délivre de nos ennemis et de la main de tous ceux qui nous haïssent! 
\verse C`est ainsi qu`il manifeste sa miséricorde envers nos pères, Et se souvient de sa sainte alliance, 
\verse Selon le serment par lequel il avait juré à Abraham, notre père, 
\verse De nous permettre, après que nous serions délivrés de la main de nos ennemis, De le servir sans crainte, 
\verse En marchant devant lui dans la sainteté et dans la justice tous les jours de notre vie. 
\verse Et toi, petit enfant, tu seras appelé prophète du Très Haut; Car tu marcheras devant la face du Seigneur, pour préparer ses voies, 
\verse Afin de donner à son peuple la connaissance du salut Par le pardon de ses péchés, 
\verse Grâce aux entrailles de la miséricorde de notre Dieu, En vertu de laquelle le soleil levant nous a visités d`en haut, 
\verse Pour éclairer ceux qui sont assis dans les ténèbres et dans l`ombre de la mort, Pour diriger nos pas dans le chemin de la paix. 
\verse Or, l`enfant croissait, et se fortifiait en esprit. Et il demeura dans les déserts, jusqu`au jour où il se présenta devant Israël. 

\chapter
\verse En ce temps-là parut un édit de César Auguste, ordonnant un recensement de toute la terre. 
\verse Ce premier recensement eut lieu pendant que Quirinius était gouverneur de Syrie. 
\verse Tous allaient se faire inscrire, chacun dans sa ville. 
\verse Joseph aussi monta de la Galilée, de la ville de Nazareth, pour se rendre en Judée, dans la ville de David, appelée Bethléhem, parce qu`il était de la maison et de la famille de David, 
\verse afin de se faire inscrire avec Marie, sa fiancée, qui était enceinte. 
\verse Pendant qu`ils étaient là, le temps où Marie devait accoucher arriva, 
\verse et elle enfanta son fils premier-né. Elle l`emmaillota, et le coucha dans une crèche, parce qu`il n`y avait pas de place pour eux dans l`hôtellerie. 
\verse Il y avait, dans cette même contrée, des bergers qui passaient dans les champs les veilles de la nuit pour garder leurs troupeaux. 
\verse Et voici, un ange du Seigneur leur apparut, et la gloire du Seigneur resplendit autour d`eux. Ils furent saisis d`une grande frayeur. 
\verse Mais l`ange leur dit: Ne craignez point; car je vous annonce une bonne nouvelle, qui sera pour tout le peuple le sujet d`une grande joie: 
\verse c`est qu`aujourd`hui, dans la ville de David, il vous est né un Sauveur, qui est le Christ, le Seigneur. 
\verse Et voici à quel signe vous le reconnaîtrez: vous trouverez un enfant emmailloté et couché dans une crèche. 
\verse Et soudain il se joignit à l`ange une multitude de l`armée céleste, louant Dieu et disant: 
\verse Gloire à Dieu dans les lieux très hauts, Et paix sur la terre parmi les hommes qu`il agrée! 
\verse Lorsque les anges les eurent quittés pour retourner au ciel, les bergers se dirent les uns aux autres: Allons jusqu`à Bethléhem, et voyons ce qui est arrivé, ce que le Seigneur nous a fait connaître. 
\verse Ils y allèrent en hâte, et ils trouvèrent Marie et Joseph, et le petit enfant couché dans la crèche. 
\verse Après l`avoir vu, ils racontèrent ce qui leur avait été dit au sujet de ce petit enfant. 
\verse Tous ceux qui les entendirent furent dans l`étonnement de ce que leur disaient les bergers. 
\verse Marie gardait toutes ces choses, et les repassait dans son coeur. 
\verse Et les bergers s`en retournèrent, glorifiant et louant Dieu pour tout ce qu`ils avaient entendu et vu, et qui était conforme à ce qui leur avait été annoncé. 
\verse Le huitième jour, auquel l`enfant devait être circoncis, étant arrivé, on lui donna le nom de Jésus, nom qu`avait indiqué l`ange avant qu`il fût conçu dans le sein de sa mère. 
\verse Et, quand les jours de leur purification furent accomplis, selon la loi de Moïse, Joseph et Marie le portèrent à Jérusalem, pour le présenter au Seigneur, - 
\verse suivant ce qui est écrit dans la loi du Seigneur: Tout mâle premier-né sera consacré au Seigneur, - 
\verse et pour offrir en sacrifice deux tourterelles ou deux jeunes pigeons, comme cela est prescrit dans la loi du Seigneur. 
\verse Et voici, il y avait à Jérusalem un homme appelé Siméon. Cet homme était juste et pieux, il attendait la consolation d`Israël, et l`Esprit Saint était sur lui. 
\verse Il avait été divinement averti par le Saint Esprit qu`il ne mourrait point avant d`avoir vu le Christ du Seigneur. 
\verse Il vint au temple, poussé par l`Esprit. Et, comme les parents apportaient le petit enfant Jésus pour accomplir à son égard ce qu`ordonnait la loi, 
\verse il le reçut dans ses bras, bénit Dieu, et dit: 
\verse Maintenant, Seigneur, tu laisses ton serviteur S`en aller en paix, selon ta parole. 
\verse Car mes yeux ont vu ton salut, 
\verse Salut que tu as préparé devant tous les peuples, 
\verse Lumière pour éclairer les nations, Et gloire d`Israël, ton peuple. 
\verse Son père et sa mère étaient dans l`admiration des choses qu`on disait de lui. 
\verse Siméon les bénit, et dit à Marie, sa mère: Voici, cet enfant est destiné à amener la chute et le relèvement de plusieurs en Israël, et à devenir un signe qui provoquera la contradiction, 
\verse et à toi-même une épée te transpercera l`âme, afin que les pensées de beaucoup de coeurs soient dévoilées. 
\verse Il y avait aussi une prophétesse, Anne, fille de Phanuel, de la tribu d`Aser. Elle était fort avancée en âge, et elle avait vécu sept ans avec son mari depuis sa virginité. 
\verse Restée veuve, et âgée de quatre vingt-quatre ans, elle ne quittait pas le temple, et elle servait Dieu nuit et jour dans le jeûne et dans la prière. 
\verse Étant survenue, elle aussi, à cette même heure, elle louait Dieu, et elle parlait de Jésus à tous ceux qui attendaient la délivrance de Jérusalem. 
\verse Lorsqu`ils eurent accompli tout ce qu`ordonnait la loi du Seigneur, Joseph et Marie retournèrent en Galilée, à Nazareth, leur ville. 
\verse Or, l`enfant croissait et se fortifiait. Il était rempli de sagesse, et la grâce de Dieu était sur lui. 
\verse Les parents de Jésus allaient chaque année à Jérusalem, à la fête de Pâque. 
\verse Lorsqu`il fut âgé de douze ans, ils y montèrent, selon la coutume de la fête. 
\verse Puis, quand les jours furent écoulés, et qu`ils s`en retournèrent, l`enfant Jésus resta à Jérusalem. Son père et sa mère ne s`en aperçurent pas. 
\verse Croyant qu`il était avec leurs compagnons de voyage, ils firent une journée de chemin, et le cherchèrent parmi leurs parents et leurs connaissances. 
\verse Mais, ne l`ayant pas trouvé, ils retournèrent à Jérusalem pour le chercher. 
\verse Au bout de trois jours, ils le trouvèrent dans le temple, assis au milieu des docteurs, les écoutant et les interrogeant. 
\verse Tous ceux qui l`entendaient étaient frappés de son intelligence et de ses réponses. 
\verse Quand ses parents le virent, ils furent saisis d`étonnement, et sa mère lui dit: Mon enfant, pourquoi as-tu agi de la sorte avec nous? Voici, ton père et moi, nous te cherchions avec angoisse. 
\verse Il leur dit: Pourquoi me cherchiez-vous? Ne saviez-vous pas qu`il faut que je m`occupe des affaires de mon Père? 
\verse Mais ils ne comprirent pas ce qu`il leur disait. 
\verse Puis il descendit avec eux pour aller à Nazareth, et il leur était soumis. Sa mère gardait toutes ces choses dans son coeur. 
\verse Et Jésus croissait en sagesse, en stature, et en grâce, devant Dieu et devant les hommes. 

\chapter
\verse La quinzième année du règne de Tibère César, -lorsque Ponce Pilate était gouverneur de la Judée, Hérode tétrarque de la Galilée, son frère Philippe tétrarque de l`Iturée et du territoire de la Trachonite, Lysanias tétrarque de l`Abilène, 
\verse et du temps des souverains sacrificateurs Anne et Caïphe, -la parole de Dieu fut adressée à Jean, fils de Zacharie, dans le désert. 
\verse Et il alla dans tout le pays des environs de Jourdain, prêchant le baptême de repentance, pour la rémission des péchés, 
\verse selon ce qui est écrit dans le livre des paroles d`Ésaïe, le prophète: C`est la voix de celui qui crie dans le désert: Préparez le chemin du Seigneur, Aplanissez ses sentiers. 
\verse Toute vallée sera comblée, Toute montagne et toute colline seront abaissées; Ce qui est tortueux sera redressé, Et les chemins raboteux seront aplanis. 
\verse Et toute chair verra le salut de Dieu. 
\verse Il disait donc à ceux qui venaient en foule pour être baptisés par lui: Races de vipères, qui vous a appris à fuir la colère à venir? 
\verse Produisez donc des fruits dignes de la repentance, et ne vous mettez pas à dire en vous-mêmes: Nous avons Abraham pour père! Car je vous déclare que de ces pierres Dieu peut susciter des enfants à Abraham. 
\verse Déjà même la cognée est mise à la racine des arbres: tout arbre donc qui ne produit pas de bons fruits sera coupé et jeté au feu. 
\verse La foule l`interrogeait, disant: Que devons-nous donc faire? 
\verse Il leur répondit: Que celui qui a deux tuniques partage avec celui qui n`en a point, et que celui qui a de quoi manger agisse de même. 
\verse Il vint aussi des publicains pour être baptisés, et ils lui dirent: Maître, que devons-nous faire? 
\verse Il leur répondit: N`exigez rien au delà de ce qui vous a été ordonné. 
\verse Des soldats aussi lui demandèrent: Et nous, que devons-nous faire? Il leur répondit: Ne commettez ni extorsion ni fraude envers personne, et contentez-vous de votre solde. 
\verse Comme le peuple était dans l`attente, et que tous se demandaient en eux-même si Jean n`était pas le Christ, 
\verse il leur dit à tous: Moi, je vous baptise d`eau; mais il vient, celui qui est plus puissant que moi, et je ne suis pas digne de délier la courroie de ses souliers. Lui, il vous baptisera du Saint Esprit et de feu. 
\verse Il a son van à la main; il nettoiera son aire, et il amassera le blé dans son grenier, mais il brûlera la paille dans un feu qui ne s`éteint point. 
\verse C`est ainsi que Jean annonçait la bonne nouvelle au peuple, en lui adressant encore beaucoup d`autres exhortations. 
\verse Mais Hérode le tétrarque, étant repris par Jean au sujet d`Hérodias, femme de son frère, et pour toutes les mauvaises actions qu`il avait commises, 
\verse ajouta encore à toutes les autres celle d`enfermer Jean dans la prison. 
\verse Tout le peuple se faisant baptiser, Jésus fut aussi baptisé; et, pendant qu`il priait, le ciel s`ouvrit, 
\verse et le Saint Esprit descendit sur lui sous une forme corporelle, comme une colombe. Et une voix fit entendre du ciel ces paroles: Tu es mon Fils bien-aimé; en toi j`ai mis toute mon affection. 
\verse Jésus avait environ trente ans lorsqu`il commença son ministère, étant, comme on le croyait, fils de Joseph, fils d`Héli, 
\verse fils de Matthat, fils de Lévi, fils de Melchi, fils de Jannaï, fils de Joseph, 
\verse fils de Mattathias, fils d`Amos, fils de Nahum, fils d`Esli, fils de Naggaï, 
\verse fils de Maath, fils de Mattathias, fils de Sémeï, fils de Josech, fils de Joda, 
\verse fils de Joanan, fils de Rhésa, fils de Zorobabel, fils de Salathiel, fils de Néri, 
\verse fils de Melchi, fils d`Addi, fils de Kosam, fils d`Elmadam, fils D`Er, 
\verse fils de Jésus, fils d`Éliézer, fils de Jorim, fils de Matthat, fils de Lévi, 
\verse fils de Siméon, fils de Juda, fils de Joseph, fils de Jonam, fils d`Éliakim, 
\verse fils de Méléa, fils de Menna, fils de Mattatha, fils de Nathan, fils de David, 
\verse fils d`Isaï, fils de Jobed, fils de Booz, fils de Salmon, fils de Naasson, 
\verse fils d`Aminadab, fils d`Admin, fils d`Arni, fils d`Esrom, fils de Pharès, fils de Juda, 
\verse fils de Jacob, fils d`Isaac, fils d`Abraham, fis de Thara, fils de Nachor, 
\verse fils de Seruch, fils de Ragau, fils de Phalek, fils d`Éber, fils de Sala, 
\verse fils de Kaïnam, fils d`Arphaxad, fils de Sem, fils de Noé, fils de Lamech, 
\verse fils de Mathusala, fils d`Énoch, fils de Jared, fils de Maléléel, fils de Kaïnan, 
\verse fils d`Énos, fils de Seth, fils d`Adam, fils de Dieu. 

\chapter
\verse Jésus, rempli du Saint Esprit, revint du Jourdain, et il fut conduit par l`Esprit dans le désert, 
\verse où il fut tenté par le diable pendant quarante jours. Il ne mangea rien durant ces jours-là, et, après qu`ils furent écoulés, il eut faim. 
\verse Le diable lui dit: Si tu es Fils de Dieu, ordonne à cette pierre qu`elle devienne du pain. 
\verse Jésus lui répondit: Il est écrit: L`Homme ne vivra pas de pain seulement. 
\verse Le diable, l`ayant élevé, lui montra en un instant tous les royaumes de la terre, 
\verse et lui dit: Je te donnerai toute cette puissance, et la gloire de ces royaumes; car elle m`a été donnée, et je la donne à qui je veux. 
\verse Si donc tu te prosternes devant moi, elle sera toute à toi. 
\verse Jésus lui répondit: Il est écrit: Tu adoreras le Seigneur, ton Dieu, et tu le serviras lui seul. 
\verse Le diable le conduisit encore à Jérusalem, le plaça sur le haut du temple, et lui dit: Si tu es Fils de Dieu, jette-toi d`ici en bas; car il est écrit: 
\verse Il donnera des ordres à ses anges à ton sujet, Afin qu`ils te gardent; 
\verse et: Ils te porteront sur les mains, De peur que ton pied ne heurte contre une pierre. 
\verse Jésus lui répondit: Il es dit: Tu ne tenteras point le Seigneur, ton Dieu. 
\verse Après l`avoir tenté de toutes ces manières, le diable s`éloigna de lui jusqu`à un moment favorable. 
\verse Jésus, revêtu de la puissance de l`Esprit, retourna en Galilée, et sa renommée se répandit dans tout le pays d`alentour. 
\verse Il enseignait dans les synagogues, et il était glorifié par tous. 
\verse Il se rendit à Nazareth, où il avait été élevé, et, selon sa coutume, il entra dans la synagogue le jour du sabbat. Il se leva pour faire la lecture, 
\verse et on lui remit le livre du prophète Ésaïe. L`ayant déroulé, il trouva l`endroit où il était écrit: 
\verse L`Esprit du Seigneur est sur moi, Parce qu`il m`a oint pour annoncer une bonne nouvelle aux pauvres; Il m`a envoyé pour guérir ceux qui ont le coeur brisé, 
\verse Pour proclamer aux captifs la délivrance, Et aux aveugles le recouvrement de la vue, Pour renvoyer libres les opprimés, Pour publier une année de grâce du Seigneur. 
\verse Ensuite, il roula le livre, le remit au serviteur, et s`assit. Tous ceux qui se trouvaient dans la synagogue avaient les regards fixés sur lui. 
\verse Alors il commença à leur dire: Aujourd`hui cette parole de l`Écriture, que vous venez d`entendre, est accomplie. 
\verse Et tous lui rendaient témoignage; ils étaient étonnés des paroles de grâce qui sortaient de sa bouche, et ils disaient: N`est-ce pas le fils de Joseph? 
\verse Jésus leur dit: Sans doute vous m`appliquerez ce proverbe: Médecin, guéris-toi toi-même; et vous me direz: Fais ici, dans ta patrie, tout ce que nous avons appris que tu as fait à Capernaüm. 
\verse Mais, ajouta-t-il, je vous le dis en vérité, aucun prophète n`est bien reçu dans sa patrie. 
\verse Je vous le dis en vérité: il y avait plusieurs veuves en Israël du temps d`Élie, lorsque le ciel fut fermé trois ans et six mois et qu`il y eut une grande famine sur toute la terre; 
\verse et cependant Élie ne fut envoyé vers aucune d`elles, si ce n`est vers une femme veuve, à Sarepta, dans le pays de Sidon. 
\verse Il y avait aussi plusieurs lépreux en Israël du temps d`Élisée, le prophète; et cependant aucun d`eux ne fut purifié, si ce n`est Naaman le Syrien. 
\verse Ils furent tous remplis de colère dans la synagogue, lorsqu`ils entendirent ces choses. 
\verse Et s`étant levés, ils le chassèrent de la ville, et le menèrent jusqu`au sommet de la montagne sur laquelle leur ville était bâtie, afin de le précipiter en bas. 
\verse Mais Jésus, passant au milieu d`eux, s`en alla. 
\verse Il descendit à Capernaüm, ville de la Galilée; et il enseignait, le jour du sabbat. 
\verse On était frappé de sa doctrine; car il parlait avec autorité. 
\verse Il se trouva dans la synagogue un homme qui avait un esprit de démon impur, et qui s`écria d`une voix forte: 
\verse Ah! qu`y a-t-il entre nous et toi, Jésus de Nazareth? Tu es venu pour nous perdre. Je sais qui tu es: le Saint de Dieu. 
\verse Jésus le menaça, disant: Tais-toi, et sors de cet homme. Et le démon le jeta au milieu de l`assemblée, et sortit de lui, sans lui faire aucun mal. 
\verse Tous furent saisis de stupeur, et ils se disaient les uns aux autres: Quelle est cette parole? il commande avec autorité et puissance aux esprits impurs, et ils sortent! 
\verse Et sa renommée se répandit dans tous les lieux d`alentour. 
\verse En sortant de la synagogue, il se rendit à la maison de Simon. La belle-mère de Simon avait une violente fièvre, et ils le prièrent en sa faveur. 
\verse S`étant penché sur elle, il menaça la fièvre, et la fièvre la quitta. A l`instant elle se leva, et les servit. 
\verse Après le couché du soleil, tous ceux qui avaient des malades atteints de diverses maladies les lui amenèrent. Il imposa les mains à chacun d`eux, et il les guérit. 
\verse Des démons aussi sortirent de beaucoup de personnes, en criant et en disant: Tu es le Fils de Dieu. Mais il les menaçait et ne leur permettait pas de parler, parce qu`ils savaient qu`il était le Christ. 
\verse Dès que le jour parut, il sortit et alla dans un lieu désert. Une foule de gens se mirent à sa recherche, et arrivèrent jusqu`à lui; ils voulaient le retenir, afin qu`il ne les quittât point. 
\verse Mais il leur dit: Il faut aussi que j`annonce aux autres villes la bonne nouvelle du royaume de Dieu; car c`est pour cela que j`ai été envoyé. 
\verse Et il prêchait dans les synagogues de la Galilée. 

\chapter
\verse Comme Jésus se trouvait auprès du lac de Génésareth, et que la foule se pressait autour de lui pour entendre la parole de Dieu, 
\verse il vit au bord du lac deux barques, d`où les pêcheurs étaient descendus pour laver leurs filets. 
\verse Il monta dans l`une de ces barques, qui était à Simon, et il le pria de s`éloigner un peu de terre. Puis il s`assit, et de la barque il enseignait la foule. 
\verse Lorsqu`il eut cessé de parler, il dit à Simon: Avance en pleine eau, et jetez vos filets pour pêcher. 
\verse Simon lui répondit: Maître, nous avons travaillé toute la nuit sans rien prendre; mais, sur ta parole, je jetterai le filet. 
\verse L`ayant jeté, ils prirent une grande quantité de poissons, et leur filet se rompait. 
\verse Ils firent signe à leurs compagnons qui étaient dans l`autre barque de venir les aider. Ils vinrent et ils remplirent les deux barques, au point qu`elles enfonçaient. 
\verse Quand il vit cela, Simon Pierre tomba aux genoux de Jésus, et dit: Seigneur, retire-toi de moi, parce que je suis un homme pécheur. 
\verse Car l`épouvante l`avait saisi, lui et tous ceux qui étaient avec lui, à cause de la pêche qu`ils avaient faite. 
\verse Il en était de même de Jacques et de Jean, fils de Zébédée, les associés de Simon. Alors Jésus dit à Simon: Ne crains point; désormais tu seras pêcheur d`hommes. 
\verse Et, ayant ramené les barques à terre, ils laissèrent tout, et le suivirent. 
\verse Jésus était dans une des villes; et voici, un homme couvert de lèpre, l`ayant vu, tomba sur sa face, et lui fit cette prière: Seigneur, si tu le veux, tu peux me rendre pur. 
\verse Jésus étendit la main, le toucha, et dit: Je le veux, sois pur. Aussitôt la lèpre le quitta. 
\verse Puis il lui ordonna de n`en parler à personne. Mais, dit-il, va te montrer au sacrificateur, et offre pour ta purification ce que Moïse a prescrit, afin que cela leur serve de témoignage. 
\verse Sa renommée se répandait de plus en plus, et les gens venaient en foule pour l`entendre et pour être guéris de leurs maladies. 
\verse Et lui, il se retirait dans les déserts, et priait. 
\verse Un jour Jésus enseignait. Des pharisiens et des docteurs de la loi étaient là assis, venus de tous les villages de la Galilée, de la Judée et de Jérusalem; et la puissance du Seigneur se manifestait par des guérisons. 
\verse Et voici, des gens, portant sur un lit un homme qui était paralytique, cherchaient à le faire entrer et à le placer sous ses regards. 
\verse Comme ils ne savaient par où l`introduire, à cause de la foule, ils montèrent sur le toit, et ils le descendirent par une ouverture, avec son lit, au milieu de l`assemblée, devant Jésus. 
\verse Voyant leur foi, Jésus dit: Homme, tes péchés te sont pardonnés. 
\verse Les scribes et les pharisiens se mirent à raisonner et à dire: Qui est celui-ci, qui profère des blasphèmes? Qui peut pardonner les péchés, si ce n`est Dieu seul? 
\verse Jésus, connaissant leurs pensées, prit la parole et leur dit: Quelles pensées avez-vous dans vos coeurs? 
\verse Lequel est le plus aisé, de dire: Tes péchés te sont pardonnés, ou de dire: Lève-toi, et marche? 
\verse Or, afin que vous sachiez que le Fils de l`homme a sur la terre le pouvoir de pardonner les péchés: Je te l`ordonne, dit-il au paralytique, lève-toi, prends ton lit, et va dans ta maison. 
\verse Et, à l`instant, il se leva en leur présence, prit le lit sur lequel il était couché, et s`en alla dans sa maison, glorifiant Dieu. 
\verse Tous étaient dans l`étonnement, et glorifiaient Dieu; remplis de crainte, ils disaient: Nous avons vu aujourd`hui des choses étranges. 
\verse Après cela, Jésus sortit, et il vit un publicain, nommé Lévi, assis au lieu des péages. Il lui dit: Suis-moi. 
\verse Et, laissant tout, il se leva, et le suivit. 
\verse Lévi lui donna un grand festin dans sa maison, et beaucoup de publicains et d`autres personnes étaient à table avec eux. 
\verse Les pharisiens et les scribes murmurèrent, et dirent à ses disciples: Pourquoi mangez-vous et buvez-vous avec les publicains et les gens de mauvaise vie? 
\verse Jésus, prenant la parole, leur dit: Ce ne sont pas ceux qui se portent bien qui ont besoin de médecin, mais les malades. 
\verse Je ne suis pas venu appeler à la repentance des justes, mais des pécheurs. 
\verse Ils lui dirent: Les disciples de Jean, comme ceux des pharisiens, jeûnent fréquemment et font des prières, tandis que les tiens mangent et boivent. 
\verse Il leur répondit: Pouvez-vous faire jeûner les amis de l`époux pendant que l`époux est avec eux? 
\verse Les jours viendront où l`époux leur sera enlevé, alors ils jeûneront en ces jours-là. 
\verse Il leur dit aussi une parabole: Personne ne déchire d`un habit neuf un morceau pour le mettre à un vieil habit; car, il déchire l`habit neuf, et le morceau qu`il en a pris n`est pas assorti au vieux. 
\verse Et personne ne met du vin nouveau dans de vieilles outres; autrement, le vin nouveau fait rompre les outres, il se répand, et les outres sont perdues; 
\verse mais il faut mettre le vin nouveau dans des outres neuves. 
\verse Et personne, après avoir bu du vin vieux, ne veut du nouveau, car il dit: Le vieux est bon. 

\chapter
\verse Il arriva, un jour de sabbat appelé second-premier, que Jésus traversait des champs de blé. Ses disciples arrachaient des épis et les mangeaient, après les avoir froissés dans leurs mains. 
\verse Quelques pharisiens leur dirent: Pourquoi faites-vous ce qu`il n`est pas permis de faire pendant le sabbat? 
\verse Jésus leur répondit: N`avez-vous pas lu ce que fit David, lorsqu`il eut faim, lui et ceux qui étaient avec lui; 
\verse comment il entra dans la maison de Dieu, prit les pains de proposition, en mangea, et en donna à ceux qui étaient avec lui, bien qu`il ne soit permis qu`aux sacrificateurs de les manger? 
\verse Et il leur dit: Le Fils de l`homme est maître même du sabbat. 
\verse Il arriva, un autre jour de sabbat, que Jésus entra dans la synagogue, et qu`il enseignait. Il s`y trouvait un homme dont la main droite était sèche. 
\verse Les scribes et les pharisiens observaient Jésus, pour voir s`il ferait une guérison le jour du sabbat: c`était afin d`avoir sujet de l`accuser. 
\verse Mais il connaissait leurs pensées, et il dit à l`homme qui avait la main sèche: Lève-toi, et tiens-toi là au milieu. Il se leva, et se tint debout. 
\verse Et Jésus leur dit: Je vous demande s`il est permis, le jour du sabbat, de faire du bien ou de faire du mal, de sauver une personne ou de la tuer. 
\verse Alors, promenant ses regards sur eux tous, il dit à l`homme: Étends ta main. Il le fit, et sa main fut guérie. 
\verse Ils furent remplis de fureur, et ils se consultèrent pour savoir ce qu`ils feraient à Jésus. 
\verse En ce temps-là, Jésus se rendit sur la montagne pour prier, et il passa toute la nuit à prier Dieu. 
\verse Quand le jour parut, il appela ses disciples, et il en choisit douze, auxquels il donna le nom d`apôtres: 
\verse Simon, qu`il nomma Pierre; André, son frère; Jacques; Jean; Philippe; Barthélemy; 
\verse Matthieu; Thomas; Jacques, fils d`Alphée; Simon, appelé le zélote; 
\verse Jude, fils de Jacques; et Judas Iscariot, qui devint traître. 
\verse Il descendit avec eux, et s`arrêta sur un plateau, où se trouvaient une foule de ses disciples et une multitude de peuple de toute la Judée, de Jérusalem, et de la contrée maritime de Tyr et de Sidon. Ils étaient venus pour l`entendre, et pour être guéris de leurs maladies. 
\verse Ceux qui étaient tourmentés par des esprits impurs étaient guéris. 
\verse Et toute la foule cherchait à le toucher, parce qu`une force sortait de lui et les guérissait tous. 
\verse Alors Jésus, levant les yeux sur ses disciples, dit: Heureux vous qui êtes pauvres, car le royaume de Dieu est à vous! 
\verse Heureux vous qui avez faim maintenant, car vous serez rassasiés! Heureux vous qui pleurez maintenant, car vous serez dans la joie! 
\verse Heureux serez-vous, lorsque les hommes vous haïront, lorsqu`on vous chassera, vous outragera, et qu`on rejettera votre nom comme infâme, à cause du Fils de l`homme! 
\verse Réjouissez-vous en ce jour-là et tressaillez d`allégresse, parce que votre récompense sera grande dans le ciel; car c`est ainsi que leurs pères traitaient les prophètes. 
\verse Mais, malheur à vous, riches, car vous avez votre consolation! 
\verse Malheur à vous qui êtes rassasiés, car vous aurez faim! Malheur à vous qui riez maintenant, car vous serez dans le deuil et dans les larmes! 
\verse Malheur, lorsque tous les hommes diront du bien de vous, car c`est ainsi qu`agissaient leurs pères à l`égard des faux prophètes! 
\verse Mais je vous dis, à vous qui m`écoutez: Aimez vos ennemis, faites du bien à ceux qui vous haïssent, 
\verse bénissez ceux qui vous maudissent, priez pour ceux qui vous maltraitent. 
\verse Si quelqu`un te frappe sur une joue, présente-lui aussi l`autre. Si quelqu`un prend ton manteau, ne l`empêche pas de prendre encore ta tunique. 
\verse Donne à quiconque te demande, et ne réclame pas ton bien à celui qui s`en empare. 
\verse Ce que vous voulez que les hommes fassent pour vous, faites-le de même pour eux. 
\verse Si vous aimez ceux qui vous aiment, quel gré vous en saura-t-on? Les pécheurs aussi aiment ceux qui les aiment. 
\verse Si vous faites du bien à ceux qui vous font du bien, quel gré vous en saura-t-on? Les pécheurs aussi agissent de même. 
\verse Et si vous prêtez à ceux de qui vous espérez recevoir, quel gré vous en saura-t-on? Les pécheurs aussi prêtent aux pécheurs, afin de recevoir la pareille. 
\verse Mais aimez vos ennemis, faites du bien, et prêtez sans rien espérer. Et votre récompense sera grande, et vous serez fils du Très Haut, car il est bon pour les ingrats et pour les méchants. 
\verse Soyez donc miséricordieux, comme votre Père est miséricordieux. 
\verse Ne jugez point, et vous ne serez point jugés; ne condamnez point, et vous ne serez point condamnés; absolvez, et vous serez absous. 
\verse Donnez, et il vous sera donné: on versera dans votre sein une bonne mesure, serrée, secouée et qui déborde; car on vous mesurera avec la mesure dont vous vous serez servis. 
\verse Il leur dit aussi cette parabole: Un aveugle peut-il conduire un aveugle? Ne tomberont-ils pas tous deux dans une fosse? 
\verse Le disciple n`est pas plus que le maître; mais tout disciple accompli sera comme son maître. 
\verse Pourquoi vois-tu la paille qui est dans l`oeil de ton frère, et n`aperçois-tu pas la poutre qui est dans ton oeil? 
\verse Ou comment peux-tu dire à ton frère: Frère, laisse-moi ôter la paille qui est dans ton oeil, toi qui ne vois pas la poutre qui est dans le tien? Hypocrite, ôte premièrement la poutre de ton oeil, et alors tu verras comment ôter la paille qui est dans l`oeil de ton frère. 
\verse Ce n`est pas un bon arbre qui porte du mauvais fruit, ni un mauvais arbre qui porte du bon fruit. 
\verse Car chaque arbre se connaît à son fruit. On ne cueille pas des figues sur des épines, et l`on ne vendange pas des raisins sur des ronces. 
\verse L`homme bon tire de bonnes choses du bon trésor de son coeur, et le méchant tire de mauvaises choses de son mauvais trésor; car c`est de l`abondance du coeur que la bouche parle. 
\verse Pourquoi m`appelez-vous Seigneur, Seigneur! et ne faites-vous pas ce que je dis? 
\verse Je vous montrerai à qui est semblable tout homme qui vient à moi, entend mes paroles, et les met en pratique. 
\verse Il est semblable à un homme qui, bâtissant une maison, a creusé, creusé profondément, et a posé le fondement sur le roc. Une inondation est venue, et le torrent s`est jeté contre cette maison, sans pouvoir l`ébranler, parce qu`elle était bien bâtie. 
\verse Mais celui qui entend, et ne met pas en pratique, est semblable à un homme qui a bâti une maison sur la terre, sans fondement. Le torrent s`est jeté contre elle: aussitôt elle est tombée, et la ruine de cette maison a été grande. 

\chapter
\verse Après avoir achevé tous ces discours devant le peuple qui l`écoutait, Jésus entra dans Capernaüm. 
\verse Un centenier avait un serviteur auquel il était très attaché, et qui se trouvait malade, sur le point de mourir. 
\verse Ayant entendu parler de Jésus, il lui envoya quelques anciens des Juifs, pour le prier de venir guérir son serviteur. 
\verse Ils arrivèrent auprès de Jésus, et lui adressèrent d`instantes supplications, disant: Il mérite que tu lui accordes cela; 
\verse car il aime notre nation, et c`est lui qui a bâti notre synagogue. 
\verse Jésus, étant allé avec eux, n`était guère éloigné de la maison, quand le centenier envoya des amis pour lui dire: Seigneur, ne prends pas tant de peine; car je ne suis pas digne que tu entres sous mon toit. 
\verse C`est aussi pour cela que je ne me suis pas cru digne d`aller en personne vers toi. Mais dis un mot, et mon serviteur sera guéri. 
\verse Car, moi qui suis soumis à des supérieurs, j`ai des soldats sous mes ordres; et je dis à l`un: Va! et il va; à l`autre: Viens! et il vient; et à mon serviteur: Fais cela! et il le fait. 
\verse Lorsque Jésus entendit ces paroles, il admira le centenier, et, se tournant vers la foule qui le suivait, il dit: Je vous le dis, même en Israël je n`ai pas trouvé une aussi grande foi. 
\verse De retour à la maison, les gens envoyés par le centenier trouvèrent guéri le serviteur qui avait été malade. 
\verse Le jour suivant, Jésus alla dans une ville appelée Naïn; ses disciples et une grande foule faisaient route avec lui. 
\verse Lorsqu`il fut près de la porte de la ville, voici, on portait en terre un mort, fils unique de sa mère, qui était veuve; et il y avait avec elle beaucoup de gens de la ville. 
\verse Le Seigneur, l`ayant vue, fut ému de compassion pour elle, et lui dit: Ne pleure pas! 
\verse Il s`approcha, et toucha le cercueil. Ceux qui le portaient s`arrêtèrent. Il dit: Jeune homme, je te le dis, lève-toi! 
\verse Et le mort s`assit, et se mit à parler. Jésus le rendit à sa mère. 
\verse Tous furent saisis de crainte, et ils glorifiaient Dieu, disant: Un grand prophète a paru parmi nous, et Dieu a visité son peuple. 
\verse Cette parole sur Jésus se répandit dans toute la Judée et dans tout le pays d`alentour. 
\verse Jean fut informé de toutes ces choses par ses disciples. 
\verse Il en appela deux, et les envoya vers Jésus, pour lui dire: Es-tu celui qui doit venir, ou devons-nous en attendre un autre? 
\verse Arrivés auprès de Jésus, ils dirent: Jean Baptiste nous a envoyés vers toi, pour dire: Es-tu celui qui doit venir, ou devons-nous en attendre un autre? 
\verse A l`heure même, Jésus guérit plusieurs personnes de maladies, d`infirmités, et d`esprits malins, et il rendit la vue à plusieurs aveugles. 
\verse Et il leur répondit: Allez rapporter à Jean ce que vous avez vu et entendu: les aveugles voient, les boiteux marchent, les lépreux sont purifiés, les sourds entendent, les morts ressuscitent, la bonne nouvelle est annoncée aux pauvres. 
\verse Heureux celui pour qui je ne serai pas une occasion de chute! 
\verse Lorsque les envoyés de Jean furent partis, Jésus se mit à dire à la foule, au sujet de Jean: Qu`êtes-vous allés voir au désert? un roseau agité par le vent? 
\verse Mais, qu`êtes-vous allés voir? un homme vêtu d`habits précieux? Voici, ceux qui portent des habits magnifiques, et qui vivent dans les délices, sont dans les maisons des rois. 
\verse Qu`êtes-vous donc allés voir? un prophète? Oui, vous dis-je, et plus qu`un prophète. 
\verse C`est celui dont il est écrit: Voici, j`envoie mon messager devant ta face, Pour préparer ton chemin devant toi. 
\verse Je vous le dis, parmi ceux qui sont nés de femmes, il n`y en a point de plus grand que Jean. Cependant, le plus petit dans le royaume de Dieu est plus grand que lui. 
\verse Et tout le peuple qui l`a entendu et même les publicains ont justifié Dieu, en se faisant baptiser du baptême de Jean; 
\verse mais les pharisiens et les docteurs de la loi, en ne se faisant pas baptiser par lui, ont rendu nul à leur égard le dessein de Dieu. 
\verse A qui donc comparerai-je les hommes de cette génération, et à qui ressemblent-ils? 
\verse Ils ressemblent aux enfants assis dans la place publique, et qui, se parlant les uns aux autres, disent: Nous vous avons joué de la flûte, et vous n`avez pas dansé; nous vous avons chanté des complaintes, et vous n`avez pas pleuré. 
\verse Car Jean Baptiste est venu, ne mangeant pas de pain et ne buvant pas de vin, et vous dites: Il a un démon. 
\verse Le Fils de l`homme est venu, mangeant et buvant, et vous dites: C`est un mangeur et un buveur, un ami des publicains et des gens de mauvaise vie. 
\verse Mais la sagesse a été justifiée par tous ses enfants. 
\verse Un pharisien pria Jésus de manger avec lui. Jésus entra dans la maison du pharisien, et se mit à table. 
\verse Et voici, une femme pécheresse qui se trouvait dans la ville, ayant su qu`il était à table dans la maison du pharisien, apporta un vase d`albâtre plein de parfum, 
\verse et se tint derrière, aux pieds de Jésus. Elle pleurait; et bientôt elle lui mouilla les pieds de ses larmes, puis les essuya avec ses cheveux, les baisa, et les oignit de parfum. 
\verse Le pharisien qui l`avait invité, voyant cela, dit en lui-même: Si cet homme était prophète, il connaîtrait qui et de quelle espèce est la femme qui le touche, il connaîtrait que c`est une pécheresse. 
\verse Jésus prit la parole, et lui dit: Simon, j`ai quelque chose à te dire. -Maître, parle, répondit-il. - 
\verse Un créancier avait deux débiteurs: l`un devait cinq cents deniers, et l`autre cinquante. 
\verse Comme ils n`avaient pas de quoi payer, il leur remit à tous deux leur dette. Lequel l`aimera le plus? 
\verse Simon répondit: Celui, je pense, auquel il a le plus remis. Jésus lui dit: Tu as bien jugé. 
\verse Puis, se tournant vers la femme, il dit à Simon: Vois-tu cette femme? Je suis entré dans ta maison, et tu ne m`as point donné d`eau pour laver mes pieds; mais elle, elle les a mouillés de ses larmes, et les a essuyés avec ses cheveux. 
\verse Tu ne m`as point donné de baiser; mais elle, depuis que je suis entré, elle n`a point cessé de me baiser les pieds. 
\verse Tu n`as point versé d`huile sur ma tête; mais elle, elle a versé du parfum sur mes pieds. 
\verse C`est pourquoi, je te le dis, ses nombreux péchés ont été pardonnés: car elle a beaucoup aimé. Mais celui à qui on pardonne peu aime peu. 
\verse Et il dit à la femme: Tes péchés sont pardonnés. 
\verse Ceux qui étaient à table avec lui se mirent à dire en eux-mêmes: Qui est celui-ci, qui pardonne même les péchés? 
\verse Mais Jésus dit à la femme: Ta foi t`a sauvée, va en paix. 

\chapter
\verse Ensuite, Jésus allait de ville en ville et de village en village, prêchant et annonçant la bonne nouvelle du royaume de Dieu. 
\verse Les douze étaient avec de lui et quelques femmes qui avaient été guéries d`esprits malins et de maladies: Marie, dite de Magdala, de laquelle étaient sortis sept démons, 
\verse Jeanne, femme de Chuza, intendant d`Hérode, Susanne, et plusieurs autres, qui l`assistaient de leurs biens. 
\verse Une grande foule s`étant assemblée, et des gens étant venus de diverses villes auprès de lui, il dit cette parabole: 
\verse Un semeur sortit pour semer sa semence. Comme il semait, une partie de la semence tomba le long du chemin: elle fut foulée aux pieds, et les oiseaux du ciel la mangèrent. 
\verse Une autre partie tomba sur le roc: quand elle fut levée, elle sécha, parce qu`elle n`avait point d`humidité. 
\verse Une autre partie tomba au milieu des épines: les épines crûrent avec elle, et l`étouffèrent. 
\verse Une autre partie tomba dans la bonne terre: quand elle fut levée, elle donna du fruit au centuple. Après avoir ainsi parlé, Jésus dit à haute voix: Que celui qui a des oreilles pour entendre entende! 
\verse Ses disciples lui demandèrent ce que signifiait cette parabole. 
\verse Il répondit: Il vous a été donné de connaître les mystères du royaume de Dieu; mais pour les autres, cela leur est dit en paraboles, afin qu`en voyant ils ne voient point, et qu`en entendant ils ne comprennent point. 
\verse Voici ce que signifie cette parabole: La semence, c`est la parole de Dieu. 
\verse Ceux qui sont le long du chemin, ce sont ceux qui entendent; puis le diable vient, et enlève de leur coeur la parole, de peur qu`ils ne croient et soient sauvés. 
\verse Ceux qui sont sur le roc, ce sont ceux qui, lorsqu`ils entendent la parole, la reçoivent avec joie; mais ils n`ont point de racine, ils croient pour un temps, et ils succombent au moment de la tentation. 
\verse Ce qui est tombé parmi les épines, ce sont ceux qui, ayant entendu la parole, s`en vont, et la laissent étouffer par les soucis, les richesses et les plaisirs de la vie, et ils ne portent point de fruit qui vienne à maturité. 
\verse Ce qui est tombé dans la bonne terre, ce sont ceux qui, ayant entendu la parole avec un coeur honnête et bon, la retiennent, et portent du fruit avec persévérance. 
\verse Personne, après avoir allumé une lampe, ne la couvre d`un vase, ou ne la met sous un lit; mais il la met sur un chandelier, afin que ceux qui entrent voient la lumière. 
\verse Car il n`est rien de caché qui ne doive être découvert, rien de secret qui ne doive être connu et mis au jour. 
\verse Prenez donc garde à la manière dont vous écoutez; car on donnera à celui qui a, mais à celui qui n`a pas on ôtera même ce qu`il croit avoir. 
\verse La mère et les frères de Jésus vinrent le trouver; mais ils ne purent l`aborder, à cause de la foule. 
\verse On lui dit: Ta mère et tes frères sont dehors, et ils désirent te voir. 
\verse Mais il répondit: Ma mère et mes frères, ce sont ceux qui écoutent la parole de Dieu, et qui la mettent en pratique. 
\verse Un jour, Jésus monta dans une barque avec ses disciples. Il leur dit: Passons de l`autre côté du lac. Et ils partirent. 
\verse Pendant qu`ils naviguaient, Jésus s`endormit. Un tourbillon fondit sur le lac, la barque se remplissait d`eau, et ils étaient en péril. 
\verse Ils s`approchèrent et le réveillèrent, en disant: Maître, maître, nous périssons! S`étant réveillé, il menaça le vent et les flots, qui s`apaisèrent, et le calme revint. 
\verse Puis il leur dit: Où est votre foi? Saisis de frayeur et d`étonnement, ils se dirent les uns aux autres: Quel est donc celui-ci, qui commande même au vent et à l`eau, et à qui ils obéissent? 
\verse Ils abordèrent dans le pays des Géraséniens, qui est vis-à-vis de la Galilée. 
\verse Lorsque Jésus fut descendu à terre, il vint au-devant de lui un homme de la ville, qui était possédé de plusieurs démons. Depuis longtemps il ne portait point de vêtement, et avait sa demeure non dans une maison, mais dans les sépulcres. 
\verse Ayant vu Jésus, il poussa un cri, se jeta à ses pieds, et dit d`une voix forte: Qu`y a-t-il entre moi et toi, Jésus, Fils du Dieu Très Haut? Je t`en supplie, ne me tourmente pas. 
\verse Car Jésus commandait à l`esprit impur de sortir de cet homme, dont il s`était emparé depuis longtemps; on le gardait lié de chaînes et les fers aux pieds, mais il rompait les liens, et il était entraîné par le démon dans les déserts. 
\verse Jésus lui demanda: Quel est ton nom? Légion, répondit-il. Car plusieurs démons étaient entrés en lui. 
\verse Et ils priaient instamment Jésus de ne pas leur ordonner d`aller dans l`abîme. 
\verse Il y avait là, dans la montagne, un grand troupeau de pourceaux qui paissaient. Et les démons supplièrent Jésus de leur permettre d`entrer dans ces pourceaux. Il le leur permit. 
\verse Les démons sortirent de cet homme, entrèrent dans les pourceaux, et le troupeau se précipita des pentes escarpées dans le lac, et se noya. 
\verse Ceux qui les faisaient paître, voyant ce qui était arrivé, s`enfuirent, et répandirent la nouvelle dans la ville et dans les campagnes. 
\verse Les gens allèrent voir ce qui était arrivé. Ils vinrent auprès de Jésus, et ils trouvèrent l`homme de qui étaient sortis les démons, assis à ses pieds, vêtu, et dans son bon sens; et ils furent saisis de frayeur. 
\verse Ceux qui avaient vu ce qui s`était passé leur racontèrent comment le démoniaque avait été guéri. 
\verse Tous les habitants du pays des Géraséniens prièrent Jésus de s`éloigner d`eux, car ils étaient saisis d`une grande crainte. Jésus monta dans la barque, et s`en retourna. 
\verse L`homme de qui étaient sortis les démons lui demandait la permission de rester avec lui. Mais Jésus le renvoya, en disant: 
\verse Retourne dans ta maison, et raconte tout ce que Dieu t`a fait. Il s`en alla, et publia par toute la ville tout ce que Jésus avait fait pour lui. 
\verse A son retour, Jésus fut reçu par la foule, car tous l`attendaient. 
\verse Et voici, il vint un homme, nommé Jaïrus, qui était chef de la synagogue. Il se jeta à ses pieds, et le supplia d`entrer dans sa maison, 
\verse parce qu`il avait une fille unique d`environ douze ans qui se mourait. Pendant que Jésus y allait, il était pressé par la foule. 
\verse Or, il y avait une femme atteinte d`une perte de sang depuis douze ans, et qui avait dépensé tout son bien pour les médecins, sans qu`aucun ait pu la guérir. 
\verse Elle s`approcha par derrière, et toucha le bord du vêtement de Jésus. Au même instant la perte de sang s`arrêta. 
\verse Et Jésus dit: Qui m`a touché? Comme tous s`en défendaient, Pierre et ceux qui étaient avec lui dirent: Maître, la foule t`entoure et te presse, et tu dis: Qui m`a touché? 
\verse Mais Jésus répondit: Quelqu`un m`a touché, car j`ai connu qu`une force était sortie de moi. 
\verse La femme, se voyant découverte, vint toute tremblante se jeter à ses pieds, et déclara devant tout le peuple pourquoi elle l`avait touché, et comment elle avait été guérie à l`instant. 
\verse Jésus lui dit: Ma fille, ta foi t`a sauvée; va en paix. 
\verse Comme il parlait encore, survint de chez le chef de la synagogue quelqu`un disant: Ta fille est morte; n`importune pas le maître. 
\verse Mais Jésus, ayant entendu cela, dit au chef de la synagogue: Ne crains pas, crois seulement, et elle sera sauvée. 
\verse Lorsqu`il fut arrivé à la maison, il ne permit à personne d`entrer avec lui, si ce n`est à Pierre, à Jean et à Jacques, et au père et à la mère de l`enfant. 
\verse Tous pleuraient et se lamentaient sur elle. Alors Jésus dit: Ne pleurez pas; elle n`est pas morte, mais elle dort. 
\verse Et ils se moquaient de lui, sachant qu`elle était morte. 
\verse Mais il la saisit par la main, et dit d`une voix forte: Enfant, lève-toi. 
\verse Et son esprit revint en elle, et à l`instant elle se leva; et Jésus ordonna qu`on lui donnât à manger. 
\verse Les parents de la jeune fille furent dans l`étonnement, et il leur recommanda de ne dire à personne ce qui était arrivé. 

\chapter
\verse Jésus, ayant assemblé les douze, leur donna force et pouvoir sur tous les démons, avec la puissance de guérir les maladies. 
\verse Il les envoya prêcher le royaume de Dieu, et guérir les malades. 
\verse Ne prenez rien pour le voyage, leur dit-il, ni bâton, ni sac, ni pain, ni argent, et n`ayez pas deux tuniques. 
\verse Dans quelque maison que vous entriez, restez-y; et c`est de là que vous partirez. 
\verse Et, si les gens ne vous reçoivent pas, sortez de cette ville, et secouez la poussière de vos pieds, en témoignage contre eux. 
\verse Ils partirent, et ils allèrent de village en village, annonçant la bonne nouvelle et opérant partout des guérisons. 
\verse Hérode le tétrarque entendit parler de tout ce qui se passait, et il ne savait que penser. Car les uns disaient que Jean était ressuscité des morts; 
\verse d`autres, qu`Élie était apparu; et d`autres, qu`un des anciens prophètes était ressuscité. 
\verse Mais Hérode disait: J`ai fait décapiter Jean; qui donc est celui-ci, dont j`entends dire de telles choses? Et il cherchait à le voir. 
\verse Les apôtres, étant de retour, racontèrent à Jésus tout ce qu`ils avaient fait. Il les prit avec lui, et se retira à l`écart, du côté d`une ville appelée Bethsaïda. 
\verse Les foules, l`ayant su, le suivirent. Jésus les accueillit, et il leur parlait du royaume de Dieu; il guérit aussi ceux qui avaient besoin d`être guéris. 
\verse Comme le jour commençait à baisser, les douze s`approchèrent, et lui dirent: Renvoie la foule, afin qu`elle aille dans les villages et dans les campagnes des environs, pour se loger et pour trouver des vivres; car nous sommes ici dans un lieu désert. 
\verse Jésus leur dit: Donnez-leur vous-mêmes à manger. Mais ils répondirent: Nous n`avons que cinq pains et deux poissons, à moins que nous n`allions nous-mêmes acheter des vivres pour tout ce peuple. 
\verse Or, il y avait environ cinq mille hommes. Jésus dit à ses disciples: Faites-les asseoir par rangées de cinquante. 
\verse Ils firent ainsi, ils les firent tous asseoir. 
\verse Jésus prit les cinq pains et les deux poissons, et, levant les yeux vers le ciel, il les bénit. Puis, il les rompit, et les donna aux disciples, afin qu`ils les distribuassent à la foule. 
\verse Tous mangèrent et furent rassasiés, et l`on emporta douze paniers pleins des morceaux qui restaient. 
\verse Un jour que Jésus priait à l`écart, ayant avec lui ses disciples, il leur posa cette question: Qui dit-on que je suis? 
\verse Ils répondirent: Jean Baptiste; les autres, Élie; les autres, qu`un des anciens prophètes est ressuscité. 
\verse Et vous, leur demanda-t-il, qui dites-vous que je suis? Pierre répondit: Le Christ de Dieu. 
\verse Jésus leur recommanda sévèrement de ne le dire à personne. 
\verse Il ajouta qu`il fallait que le Fils de l`homme souffrît beaucoup, qu`il fût rejeté par les anciens, par les principaux sacrificateurs et par les scribes, qu`il fût mis à mort, et qu`il ressuscitât le troisième jour. 
\verse Puis il dit à tous: Si quelqu`un veut venir après moi, qu`il renonce à lui-même, qu`il se charge chaque jour de sa croix, et qu`il me suive. 
\verse Car celui qui voudra sauver sa vie la perdra, mais celui qui la perdra à cause de moi la sauvera. 
\verse Et que servirait-il à un homme de gagner tout le monde, s`il se détruisait ou se perdait lui-même? 
\verse Car quiconque aura honte de moi et de mes paroles, le Fils de l`homme aura honte de lui, quand il viendra dans sa gloire, et dans celle du Père et des saints anges. 
\verse Je vous le dis en vérité, quelques-uns de ceux qui sont ici ne mourront point qu`ils n`aient vu le royaume de Dieu. 
\verse Environ huit jours après qu`il eut dit ces paroles, Jésus prit avec lui Pierre, Jean et Jacques, et il monta sur la montagne pour prier. 
\verse Pendant qu`il priait, l`aspect de son visage changea, et son vêtement devint d`une éclatante blancheur. 
\verse Et voici, deux hommes s`entretenaient avec lui: c`étaient Moïse et Élie, 
\verse qui, apparaissant dans la gloire, parlaient de son départ qu`il allait accomplir à Jérusalem. 
\verse Pierre et ses compagnons étaient appesantis par le sommeil; mais, s`étant tenus éveillés, ils virent la gloire de Jésus et les deux hommes qui étaient avec lui. 
\verse Au moment où ces hommes se séparaient de Jésus, Pierre lui dit: Maître, il est bon que nous soyons ici; dressons trois tentes, une pour toi, une pour Moïse, et une pour Élie. Il ne savait ce qu`il disait. 
\verse Comme il parlait ainsi, une nuée vint les couvrir; et les disciples furent saisis de frayeur en les voyant entrer dans la nuée. 
\verse Et de la nuée sortit une voix, qui dit: Celui-ci est mon Fils élu: écoutez-le! 
\verse Quand la voix se fit entendre, Jésus se trouva seul. Les disciples gardèrent le silence, et ils ne racontèrent à personne, en ce temps-là, rien de ce qu`ils avaient vu. 
\verse Le lendemain, lorsqu`ils furent descendus de la montagne, une grande foule vint au-devant de Jésus. 
\verse Et voici, du milieu de la foule un homme s`écria: Maître, je t`en prie, porte les regards sur mon fils, car c`est mon fils unique. 
\verse Un esprit le saisit, et aussitôt il pousse des cris; et l`esprit l`agite avec violence, le fait écumer, et a de la peine à se retirer de lui, après l`avoir tout brisé. 
\verse J`ai prié tes disciples de le chasser, et ils n`ont pas pu. 
\verse Race incrédule et perverse, répondit Jésus, jusqu`à quand serai-je avec vous, et vous supporterai-je? Amène ici ton fils. 
\verse Comme il approchait, le démon le jeta par terre, et l`agita avec violence. Mais Jésus menaça l`esprit impur, guérit l`enfant, et le rendit à son père. 
\verse Et tous furent frappés de la grandeur de Dieu. Tandis que chacun était dans l`admiration de tout ce que faisait Jésus, il dit à ses disciples: 
\verse Pour vous, écoutez bien ceci: Le Fils de l`homme doit être livré entre les mains des hommes. 
\verse Mais les disciples ne comprenaient pas cette parole; elle était voilée pour eux, afin qu`ils n`en eussent pas le sens; et ils craignaient de l`interroger à ce sujet. 
\verse Or, une pensée leur vint à l`esprit, savoir lequel d`entre eux était le plus grand. 
\verse Jésus, voyant la pensée de leur coeur, prit un petit enfant, le plaça près de lui, 
\verse et leur dit: Quiconque reçoit en mon nom ce petit enfant me reçoit moi-même; et quiconque me reçoit reçoit celui qui m`a envoyé. Car celui qui est le plus petit parmi vous tous, c`est celui-là qui est grand. 
\verse Jean prit la parole, et dit: Maître, nous avons vu un homme qui chasse des démons en ton nom; et nous l`en avons empêché, parce qu`il ne nous suit pas. 
\verse Ne l`en empêchez pas, lui répondit Jésus; car qui n`est pas contre vous est pour vous. 
\verse Lorsque le temps où il devait être enlevé du monde approcha, Jésus prit la résolution de se rendre à Jérusalem. 
\verse Il envoya devant lui des messagers, qui se mirent en route et entrèrent dans un bourg des Samaritains, pour lui préparer un logement. 
\verse Mais on ne le reçut pas, parce qu`il se dirigeait sur Jérusalem. 
\verse Les disciples Jacques et Jean, voyant cela, dirent: Seigneur, veux-tu que nous commandions que le feu descende du ciel et les consume? 
\verse Jésus se tourna vers eux, et les réprimanda, disant: Vous ne savez de quel esprit vous êtes animés. 
\verse Car le Fils de l`homme est venu, non pour perdre les âmes des hommes, mais pour les sauver. Et ils allèrent dans un autre bourg. 
\verse Pendant qu`ils étaient en chemin, un homme lui dit: Seigneur, je te suivrai partout où tu iras. 
\verse Jésus lui répondit: Les renards ont des tanières, et les oiseaux du ciel ont des nids: mais le Fils de l`homme n`a pas un lieu où il puisse reposer sa tête. 
\verse Il dit à un autre: Suis-moi. Et il répondit: Seigneur, permets-moi d`aller d`abord ensevelir mon père. 
\verse Mais Jésus lui dit: Laisse les morts ensevelir leurs morts; et toi, va annoncer le royaume de Dieu. 
\verse Un autre dit: Je te suivrai, Seigneur, mais permets-moi d`aller d`abord prendre congé de ceux de ma maison. 
\verse Jésus lui répondit: Quiconque met la main à la charrue, et regarde en arrière, n`est pas propre au royaume de Dieu. 

\chapter
\verse Après cela, le Seigneur désigna encore soixante-dix autres disciples, et il les envoya deux à deux devant lui dans toutes les villes et dans tous les lieux où lui-même devait aller. 
\verse Il leur dit: La moisson est grande, mais il y a peu d`ouvriers. Priez donc le maître de la moisson d`envoyer des ouvriers dans sa moisson. 
\verse Partez; voici, je vous envoie comme des agneaux au milieu des loups. 
\verse Ne portez ni bourse, ni sac, ni souliers, et ne saluez personne en chemin. 
\verse Dans quelque maison que vous entriez, dites d`abord: Que la paix soit sur cette maison! 
\verse Et s`il se trouve là un enfant de paix, votre paix reposera sur lui; sinon, elle reviendra à vous. 
\verse Demeurez dans cette maison-là, mangeant et buvant ce qu`on vous donnera; car l`ouvrier mérite son salaire. N`allez pas de maison en maison. 
\verse Dans quelque ville que vous entriez, et où l`on vous recevra, mangez ce qui vous sera présenté, 
\verse guérissez les malades qui s`y trouveront, et dites-leur: Le royaume de Dieu s`est approché de vous. 
\verse Mais dans quelque ville que vous entriez, et où l`on ne vous recevra pas, allez dans ses rues, et dites: 
\verse Nous secouons contre vous la poussière même de votre ville qui s`est attachée à nos pieds; sachez cependant que le royaume de Dieu s`est approché. 
\verse Je vous dis qu`en ce jour Sodome sera traitée moins rigoureusement que cette ville-là. 
\verse Malheur à toi, Chorazin! malheur à toi, Bethsaïda! car, si les miracles qui ont été faits au milieu de vous avaient été faits dans Tyr et dans Sidon, il y a longtemps qu`elles se seraient repenties, en prenant le sac et la cendre. 
\verse C`est pourquoi, au jour du jugement, Tyr et Sidon seront traitées moins rigoureusement que vous. 
\verse Et toi, Capernaüm, qui as été élevée jusqu`au ciel, tu seras abaissée jusqu`au séjour des morts. 
\verse Celui qui vous écoute m`écoute, et celui qui vous rejette me rejette; et celui qui me rejette rejette celui qui m`a envoyé. 
\verse Les soixante-dix revinrent avec joie, disant: Seigneur, les démons mêmes nous sont soumis en ton nom. 
\verse Jésus leur dit: Je voyais Satan tomber du ciel comme un éclair. 
\verse Voici, je vous ai donné le pouvoir de marcher sur les serpents et les scorpions, et sur toute la puissance de l`ennemi; et rien ne pourra vous nuire. 
\verse Cependant, ne vous réjouissez pas de ce que les esprits vous sont soumis; mais réjouissez-vous de ce que vos noms sont écrits dans les cieux. 
\verse En ce moment même, Jésus tressaillit de joie par le Saint Esprit, et il dit: Je te loue, Père, Seigneur du ciel et de la terre, de ce que tu as caché ces choses aux sages et aux intelligents, et de ce que tu les as révélées aux enfants. Oui, Père, je te loue de ce que tu l`as voulu ainsi. 
\verse Toutes choses m`ont été données par mon Père, et personne ne connaît qui est le Fils, si ce n`est le Père, ni qui est le Père, si ce n`est le Fils et celui à qui le Fils veut le révéler. 
\verse Et, se tournant vers les disciples, il leur dit en particulier: Heureux les yeux qui voient ce que vous voyez! 
\verse Car je vous dis que beaucoup de prophètes et de rois ont désiré voir ce que vous voyez, et ne l`ont pas vu, entendre ce que vous entendez, et ne l`ont pas entendu. 
\verse Un docteur de la loi se leva, et dit à Jésus, pour l`éprouver: Maître, que dois-je faire pour hériter la vie éternelle? 
\verse Jésus lui dit: Qu`est-il écrit dans la loi? Qu`y lis-tu? 
\verse Il répondit: Tu aimeras le Seigneur, ton Dieu, de tout ton coeur, de toute ton âme, de toute ta force, et de toute ta pensée; et ton prochain comme toi-même. 
\verse Tu as bien répondu, lui dit Jésus; fais cela, et tu vivras. 
\verse Mais lui, voulant se justifier, dit à Jésus: Et qui est mon prochain? 
\verse Jésus reprit la parole, et dit: Un homme descendait de Jérusalem à Jéricho. Il tomba au milieu des brigands, qui le dépouillèrent, le chargèrent de coups, et s`en allèrent, le laissant à demi mort. 
\verse Un sacrificateur, qui par hasard descendait par le même chemin, ayant vu cet homme, passa outre. 
\verse Un Lévite, qui arriva aussi dans ce lieu, l`ayant vu, passa outre. 
\verse Mais un Samaritain, qui voyageait, étant venu là, fut ému de compassion lorsqu`il le vit. 
\verse Il s`approcha, et banda ses plaies, en y versant de l`huile et du vin; puis il le mit sur sa propre monture, le conduisit à une hôtellerie, et prit soin de lui. 
\verse Le lendemain, il tira deux deniers, les donna à l`hôte, et dit: Aie soin de lui, et ce que tu dépenseras de plus, je te le rendrai à mon retour. 
\verse Lequel de ces trois te semble avoir été le prochain de celui qui était tombé au milieu des brigands? 
\verse C`est celui qui a exercé la miséricorde envers lui, répondit le docteur de la loi. Et Jésus lui dit: Va, et toi, fais de même. 
\verse Comme Jésus était en chemin avec ses disciples, il entra dans un village, et une femme, nommée Marthe, le reçut dans sa maison. 
\verse Elle avait une soeur, nommée Marie, qui, s`étant assise aux pieds du Seigneur, écoutait sa parole. 
\verse Marthe, occupée à divers soins domestiques, survint et dit: Seigneur, cela ne te fait-il rien que ma soeur me laisse seule pour servir? Dis-lui donc de m`aider. 
\verse Le Seigneur lui répondit: Marthe, Marthe, tu t`inquiètes et tu t`agites pour beaucoup de choses. 
\verse Une seule chose est nécessaire. Marie a choisi la bonne part, qui ne lui sera point ôtée. 

\chapter
\verse Jésus priait un jour en un certain lieu. Lorsqu`il eut achevé, un de ses disciples lui dit: Seigneur, enseigne-nous à prier, comme Jean l`a enseigné à ses disciples. 
\verse Il leur dit: Quand vous priez, dites: Père! Que ton nom soit sanctifié; que ton règne vienne. 
\verse Donne-nous chaque jour notre pain quotidien; 
\verse pardonne-nous nos péchés, car nous aussi nous pardonnons à quiconque nous offense; et ne nous induis pas en tentation. 
\verse Il leur dit encore: Si l`un de vous a un ami, et qu`il aille le trouver au milieu de la nuit pour lui dire: Ami, prête-moi trois pains, 
\verse car un de mes amis est arrivé de voyage chez moi, et je n`ai rien à lui offrir, 
\verse et si, de l`intérieur de sa maison, cet ami lui répond: Ne m`importune pas, la porte est déjà fermée, mes enfants et moi sommes au lit, je ne puis me lever pour te donner des pains, - 
\verse je vous le dis, même s`il ne se levait pas pour les lui donner parce que c`est son ami, il se lèverait à cause de son importunité et lui donnerait tout ce dont il a besoin. 
\verse Et moi, je vous dis: Demandez, et l`on vous donnera; cherchez, et vous trouverez; frappez, et l`on vous ouvrira. 
\verse Car quiconque demande reçoit, celui qui cherche trouve, et l`on ouvre à celui qui frappe. 
\verse Quel est parmi vous le père qui donnera une pierre à son fils, s`il lui demande du pain? Ou, s`il demande un poisson, lui donnera-t-il un serpent au lieu d`un poisson? 
\verse Ou, s`il demande un oeuf, lui donnera-t-il un scorpion? 
\verse Si donc, méchants comme vous l`êtes, vous savez donner de bonnes choses à vos enfants, à combien plus forte raison le Père céleste donnera-t-il le Saint Esprit à ceux qui le lui demandent. 
\verse Jésus chassa un démon qui était muet. Lorsque le démon fut sorti, le muet parla, et la foule fut dans l`admiration. 
\verse Mais quelques-uns dirent: c`est par Béelzébul, le prince des démons, qu`il chasse les démons. 
\verse Et d`autres, pour l`éprouver, lui demandèrent un signe venant du ciel. 
\verse Comme Jésus connaissait leurs pensées, il leur dit: Tout royaume divisé contre lui-même est dévasté, et une maison s`écroule sur une autre. 
\verse Si donc Satan est divisé contre lui-même, comment son royaume subsistera-t-il, puisque vous dites que je chasse les démons par Béelzébul? 
\verse Et si moi, je chasse les démons par Béelzébul, vos fils, par qui les chassent-ils? C`est pourquoi ils seront eux-mêmes vos juges. 
\verse Mais, si c`est par le doigt de Dieu que je chasse les démons, le royaume de Dieu est donc venu vers vous. 
\verse Lorsqu`un homme fort et bien armé garde sa maison, ce qu`il possède est en sûreté. 
\verse Mais, si un plus fort que lui survient et le dompte, il lui enlève toutes les armes dans lesquelles il se confiait, et il distribue ses dépouilles. 
\verse Celui qui n`est pas avec moi est contre moi, et celui qui n`assemble pas avec moi disperse. 
\verse Lorsque l`esprit impur est sorti d`un homme, il va dans des lieux arides, pour chercher du repos. N`en trouvant point, il dit: Je retournerai dans ma maison d`où je suis sorti; 
\verse et, quand il arrive, il la trouve balayée et ornée. 
\verse Alors il s`en va, et il prend sept autres esprits plus méchants que lui; ils entrent dans la maison, s`y établissent, et la dernière condition de cet homme est pire que la première. 
\verse Tandis que Jésus parlait ainsi, une femme, élevant la voix du milieu de la foule, lui dit: Heureux le sein qui t`a porté! heureuses les mamelles qui t`ont allaité! 
\verse Et il répondit: Heureux plutôt ceux qui écoutent la parole de Dieu, et qui la gardent! 
\verse Comme le peuple s`amassait en foule, il se mit à dire: Cette génération est une génération méchante; elle demande un miracle; il ne lui sera donné d`autre miracle que celui de Jonas. 
\verse Car, de même que Jonas fut un signe pour les Ninivites, de même le Fils de l`homme en sera un pour cette génération. 
\verse La reine du Midi se lèvera, au jour du jugement, avec les hommes de cette génération et les condamnera, parce qu`elle vint des extrémités de la terre pour entendre la sagesse de Salomon; et voici, il y a ici plus que Salomon. 
\verse Les hommes de Ninive se lèveront, au jour du jugement, avec cette génération et la condamneront, parce qu`ils se repentirent à la prédication de Jonas; et voici, il y a ici plus que Jonas. 
\verse Personne n`allume une lampe pour la mettre dans un lieu caché ou sous le boisseau, mais on la met sur le chandelier, afin que ceux qui entrent voient la lumière. 
\verse Ton oeil est la lampe de ton corps. Lorsque ton oeil est en bon état, tout ton corps est éclairé; mais lorsque ton oeil est en mauvais état, ton corps est dans les ténèbres. 
\verse Prends donc garde que la lumière qui est en toi ne soit ténèbres. 
\verse Si donc tout ton corps est éclairé, n`ayant aucune partie dans les ténèbres, il sera entièrement éclairé, comme lorsque la lampe t`éclaire de sa lumière. 
\verse Pendant que Jésus parlait, un pharisien le pria de dîner chez lui. Il entra, et se mit à table. 
\verse Le pharisien vit avec étonnement qu`il ne s`était pas lavé avant le repas. 
\verse Mais le Seigneur lui dit: Vous, pharisiens, vous nettoyez le dehors de la coupe et du plat, et à l`intérieur vous êtes pleins de rapine et de méchanceté. 
\verse Insensés! celui qui a fait le dehors n`a-t-il pas fait aussi le dedans? 
\verse Donnez plutôt en aumônes ce qui est dedans, et voici, toutes choses seront pures pour vous. 
\verse Mais malheur à vous, pharisiens! parce que vous payez la dîme de la menthe, de la rue, et de toutes les herbes, et que vous négligez la justice et l`amour de Dieu: c`est là ce qu`il fallait pratiquer, sans omettre les autres choses. 
\verse Malheur à vous, pharisiens! parce que vous aimez les premiers sièges dans les synagogues, et les salutations dans les places publiques. 
\verse Malheur à vous! parce que vous êtes comme les sépulcres qui ne paraissent pas, et sur lesquels on marche sans le savoir. 
\verse Un des docteurs de la loi prit la parole, et lui dit: Maître, en parlant de la sorte, c`est aussi nous que tu outrages. 
\verse Et Jésus répondit: Malheur à vous aussi, docteurs de la loi! parce que vous chargez les hommes de fardeaux difficiles à porter, et que vous ne touchez pas vous-mêmes de l`un de vos doigts. 
\verse Malheur à vous! parce que vous bâtissez les tombeaux des prophètes, que vos pères ont tués. 
\verse Vous rendez donc témoignage aux oeuvres de vos pères, et vous les approuvez; car eux, ils ont tué les prophètes, et vous, vous bâtissez leurs tombeaux. 
\verse C`est pourquoi la sagesse de Dieu a dit: Je leur enverrai des prophètes et des apôtres; ils tueront les uns et persécuteront les autres, 
\verse afin qu`il soit demandé compte à cette génération du sang de tous les prophètes qui a été répandu depuis la création du monde, 
\verse depuis le sang d`Abel jusqu`au sang de Zacharie, tué entre l`autel et le temple; oui, je vous le dis, il en sera demandé compte à cette génération. 
\verse Malheur à vous, docteurs de la loi! parce que vous avez enlevé la clef de la science; vous n`êtes pas entrés vous-mêmes, et vous avez empêché d`entrer ceux qui le voulaient. 
\verse Quand il fut sorti de là, les scribes et les pharisiens commencèrent à le presser violemment, et à le faire parler sur beaucoup de choses, 
\verse lui tendant des pièges, pour surprendre quelque parole sortie de sa bouche. 

\chapter
\verse Sur ces entrefaites, les gens s`étant rassemblés par milliers, au point de se fouler les uns les autres, Jésus se mit à dire à ses disciples: Avant tout, gardez-vous du levain des pharisiens, qui est l`hypocrisie. 
\verse Il n`y a rien de caché qui ne doive être découvert, ni de secret qui ne doive être connu. 
\verse C`est pourquoi tout ce que vous aurez dit dans les ténèbres sera entendu dans la lumière, et ce que vous aurez dit à l`oreille dans les chambres sera prêché sur les toits. 
\verse Je vous dis, à vous qui êtes mes amis: Ne craignez pas ceux qui tuent le corps et qui, après cela, ne peuvent rien faire de plus. 
\verse Je vous montrerai qui vous devez craindre. Craignez celui qui, après avoir tué, a le pouvoir de jeter dans la géhenne; oui, je vous le dis, c`est lui que vous devez craindre. 
\verse Ne vend-on pas cinq passereaux pour deux sous? Cependant, aucun d`eux n`est oublié devant Dieu. 
\verse Et même les cheveux de votre tête sont tous comptés. Ne craignez donc point: vous valez plus que beaucoup de passereaux. 
\verse Je vous le dis, quiconque me confessera devant les hommes, le Fils de l`homme le confessera aussi devant les anges de Dieu; 
\verse mais celui qui me reniera devant les hommes sera renié devant les anges de Dieu. 
\verse Et quiconque parlera contre le Fils de l`homme, il lui sera pardonné; mais à celui qui blasphémera contre le Saint Esprit il ne sera point pardonné. 
\verse Quand on vous mènera devant les synagogues, les magistrats et les autorités, ne vous inquiétez pas de la manière dont vous vous défendrez ni de ce que vous direz; 
\verse car le Saint Esprit vous enseignera à l`heure même ce qu`il faudra dire. 
\verse Quelqu`un dit à Jésus, du milieu de la foule: Maître, dis à mon frère de partager avec moi notre héritage. 
\verse Jésus lui répondit: O homme, qui m`a établi pour être votre juge, ou pour faire vos partages? 
\verse Puis il leur dit: Gardez-vous avec soin de toute avarice; car la vie d`un homme ne dépend pas de ses biens, fût-il dans l`abondance. 
\verse Et il leur dit cette parabole: Les terres d`un homme riche avaient beaucoup rapporté. 
\verse Et il raisonnait en lui-même, disant: Que ferai-je? car je n`ai pas de place pour serrer ma récolte. 
\verse Voici, dit-il, ce que je ferai: j`abattrai mes greniers, j`en bâtirai de plus grands, j`y amasserai toute ma récolte et tous mes biens; 
\verse et je dirai à mon âme: Mon âme, tu as beaucoup de biens en réserve pour plusieurs années; repose-toi, mange, bois, et réjouis-toi. 
\verse Mais Dieu lui dit: Insensé! cette nuit même ton âme te sera redemandée; et ce que tu as préparé, pour qui cela sera-t-il? 
\verse Il en est ainsi de celui qui amasse des trésors pour lui-même, et qui n`est pas riche pour Dieu. 
\verse Jésus dit ensuite à ses disciples: C`est pourquoi je vous dis: Ne vous inquiétez pas pour votre vie de ce que vous mangerez, ni pour votre corps de quoi vous serez vêtus. 
\verse La vie est plus que la nourriture, et le corps plus que le vêtement. 
\verse Considérez les corbeaux: ils ne sèment ni ne moissonnent, ils n`ont ni cellier ni grenier; et Dieu les nourrit. Combien ne valez-vous pas plus que les oiseaux! 
\verse Qui de vous, par ses inquiétudes, peut ajouter une coudée à la durée de sa vie? 
\verse Si donc vous ne pouvez pas même la moindre chose, pourquoi vous inquiétez-vous du reste? 
\verse Considérez comment croissent les lis: ils ne travaillent ni ne filent; cependant je vous dis que Salomon même, dans toute sa gloire, n`a pas été vêtu comme l`un d`eux. 
\verse Si Dieu revêt ainsi l`herbe qui est aujourd`hui dans les champs et qui demain sera jetée au four, à combien plus forte raison ne vous vêtira-t-il pas, gens de peu de foi? 
\verse Et vous, ne cherchez pas ce que vous mangerez et ce que vous boirez, et ne soyez pas inquiets. 
\verse Car toutes ces choses, ce sont les païens du monde qui les recherchent. Votre Père sait que vous en avez besoin. 
\verse Cherchez plutôt le royaume de Dieu; et toutes ces choses vous seront données par-dessus. 
\verse Ne crains point, petit troupeau; car votre Père a trouvé bon de vous donner le royaume. 
\verse Vendez ce que vous possédez, et donnez-le en aumônes. Faites-vous des bourses qui ne s`usent point, un trésor inépuisable dans les cieux, où le voleur n`approche point, et où la teigne ne détruit point. 
\verse Car là où est votre trésor, là aussi sera votre coeur. 
\verse Que vos reins soient ceints, et vos lampes allumées. 
\verse Et vous, soyez semblables à des hommes qui attendent que leur maître revienne des noces, afin de lui ouvrir dès qu`il arrivera et frappera. 
\verse Heureux ces serviteurs que le maître, à son arrivée, trouvera veillant! Je vous le dis en vérité, il se ceindra, les fera mettre à table, et s`approchera pour les servir. 
\verse Qu`il arrive à la deuxième ou à la troisième veille, heureux ces serviteurs, s`il les trouve veillant! 
\verse Sachez-le bien, si le maître de la maison savait à quelle heure le voleur doit venir, il veillerait et ne laisserait pas percer sa maison. 
\verse Vous aussi, tenez-vous prêts, car le Fils de l`homme viendra à l`heure où vous n`y penserez pas. 
\verse Pierre lui dit: Seigneur, est-ce à nous, ou à tous, que tu adresses cette parabole? 
\verse Et le Seigneur dit: Quel est donc l`économe fidèle et prudent que le maître établira sur ses gens, pour leur donner la nourriture au temps convenable? 
\verse Heureux ce serviteur, que son maître, à son arrivée, trouvera faisant ainsi! 
\verse Je vous le dis en vérité, il l`établira sur tous ses biens. 
\verse Mais, si ce serviteur dit en lui-même: Mon maître tarde à venir; s`il se met à battre les serviteurs et les servantes, à manger, à boire et à s`enivrer, 
\verse le maître de ce serviteur viendra le jour où il ne s`y attend pas et à l`heure qu`il ne connaît pas, il le mettra en pièces, et lui donnera sa part avec les infidèles. 
\verse Le serviteur qui, ayant connu la volonté de son maître, n`a rien préparé et n`a pas agi selon sa volonté, sera battu d`un grand nombre de coups. 
\verse Mais celui qui, ne l`ayant pas connue, a fait des choses dignes de châtiment, sera battu de peu de coups. On demandera beaucoup à qui l`on a beaucoup donné, et on exigera davantage de celui à qui l`on a beaucoup confié. 
\verse Je suis venu jeter un feu sur la terre, et qu`ai-je à désirer, s`il est déjà allumé? 
\verse Il est un baptême dont je dois être baptisé, et combien il me tarde qu`il soit accompli! 
\verse Pensez-vous que je sois venu apporter la paix sur la terre? Non, vous dis-je, mais la division. 
\verse Car désormais cinq dans une maison seront divisés, trois contre deux, et deux contre trois; 
\verse le père contre le fils et le fils contre le père, la mère contre la fille et la fille contre la mère, la belle-mère contre la belle-fille et la belle-fille contre la belle-mère. 
\verse Il dit encore aux foules: Quand vous voyez un nuage se lever à l`occident, vous dites aussitôt: La pluie vient. Et il arrive ainsi. 
\verse Et quand vous voyez souffler le vent du midi, vous dites: Il fera chaud. Et cela arrive. 
\verse Hypocrites! vous savez discerner l`aspect de la terre et du ciel; comment ne discernez-vous pas ce temps-ci? 
\verse Et pourquoi ne discernez-vous pas de vous-mêmes ce qui est juste? 
\verse Lorsque tu vas avec ton adversaire devant le magistrat, tâche en chemin de te dégager de lui, de peur qu`il ne te traîne devant le juge, que le juge ne te livre à l`officier de justice, et que celui-ci ne te mette en prison. 
\verse Je te le dis, tu ne sortiras pas de là que tu n`aies payé jusqu`à dernière pite. 

\chapter
\verse En ce même temps, quelques personnes qui se trouvaient là racontaient à Jésus ce qui était arrivé à des Galiléens dont Pilate avait mêlé le sang avec celui de leurs sacrifices. 
\verse Il leur répondit: Croyez-vous que ces Galiléens fussent de plus grands pécheurs que tous les autres Galiléens, parce qu`ils ont souffert de la sorte? 
\verse Non, je vous le dis. Mais si vous ne vous repentez, vous périrez tous également. 
\verse Ou bien, ces dix-huit personnes sur qui est tombée la tour de Siloé et qu`elle a tuées, croyez-vous qu`elles fussent plus coupables que tous les autres habitants de Jérusalem? 
\verse Non, je vous le dis. Mais si vous ne vous repentez, vous périrez tous également. 
\verse Il dit aussi cette parabole: Un homme avait un figuier planté dans sa vigne. Il vint pour y chercher du fruit, et il n`en trouva point. 
\verse Alors il dit au vigneron: Voilà trois ans que je viens chercher du fruit à ce figuier, et je n`en trouve point. Coupe-le: pourquoi occupe-t-il la terre inutilement? 
\verse Le vigneron lui répondit: Seigneur, laisse-le encore cette année; je creuserai tout autour, et j`y mettrai du fumier. 
\verse Peut-être à l`avenir donnera-t-il du fruit; sinon, tu le couperas. 
\verse Jésus enseignait dans une des synagogues, le jour du sabbat. 
\verse Et voici, il y avait là une femme possédée d`un esprit qui la rendait infirme depuis dix-huit ans; elle était courbée, et ne pouvait pas du tout se redresser. 
\verse Lorsqu`il la vit, Jésus lui adressa la parole, et lui dit: Femme, tu es délivrée de ton infirmité. 
\verse Et il lui imposa les mains. A l`instant elle se redressa, et glorifia Dieu. 
\verse Mais le chef de la synagogue, indigné de ce que Jésus avait opéré cette guérison un jour de sabbat, dit à la foule: Il y a six jours pour travailler; venez donc vous faire guérir ces jours-là, et non pas le jour du sabbat. 
\verse Hypocrites! lui répondit le Seigneur, est-ce que chacun de vous, le jour du sabbat, ne détache pas de la crèche son boeuf ou son âne, pour le mener boire? 
\verse Et cette femme, qui est une fille d`Abraham, et que Satan tenait liée depuis dix-huit ans, ne fallait-il pas la délivrer de cette chaîne le jour du sabbat? 
\verse Tandis qu`il parlait ainsi, tous ses adversaires étaient confus, et la foule se réjouissait de toutes les choses glorieuses qu`il faisait. 
\verse Il dit encore: A quoi le royaume de Dieu est-il semblable, et à quoi le comparerai-je? 
\verse Il est semblable à un grain de sénevé qu`un homme a pris et jeté dans son jardin; il pousse, devient un arbre, et les oiseaux du ciel habitent dans ses branches. 
\verse Il dit encore: A quoi comparerai-je le royaume de Dieu? 
\verse Il est semblable à du levain qu`une femme a pris et mis dans trois mesures de farine, pour faire lever toute la pâte. 
\verse Jésus traversait les villes et les villages, enseignant, et faisant route vers Jérusalem. 
\verse Quelqu`un lui dit: Seigneur, n`y a-t-il que peu de gens qui soient sauvés? Il leur répondit: 
\verse Efforcez-vous d`entrer par la porte étroite. Car, je vous le dis, beaucoup chercheront à entrer, et ne le pourront pas. 
\verse Quand le maître de la maison se sera levé et aura fermé la porte, et que vous, étant dehors, vous commencerez à frapper à la porte, en disant: Seigneur, Seigneur, ouvre-nous! il vous répondra: Je ne sais d`où vous êtes. 
\verse Alors vous vous mettrez à dire: Nous avons mangé et bu devant toi, et tu as enseigné dans nos rues. 
\verse Et il répondra: Je vous le dis, je ne sais d`où vous êtes; retirez-vous de moi, vous tous, ouvriers d`iniquité. 
\verse C`est là qu`il y aura des pleurs et des grincements de dents, quand vous verrez Abraham, Isaac et Jacob, et tous les prophètes, dans le royaume de Dieu, et que vous serez jetés dehors. 
\verse Il en viendra de l`orient et de l`occident, du nord et du midi; et ils se mettront à table dans le royaume de Dieu. 
\verse Et voici, il y en a des derniers qui seront les premiers, et des premiers qui seront les derniers. 
\verse Ce même jour, quelques pharisiens vinrent lui dire: Va-t`en, pars d`ici, car Hérode veut te tuer. 
\verse Il leur répondit: Allez, et dites à ce renard: Voici, je chasse les démons et je fais des guérisons aujourd`hui et demain, et le troisième jour j`aurai fini. 
\verse Mais il faut que je marche aujourd`hui, demain, et le jour suivant; car il ne convient pas qu`un prophète périsse hors de Jérusalem. 
\verse Jérusalem, Jérusalem, qui tues les prophètes et qui lapides ceux qui te sont envoyés, combien de fois ai-je voulu rassembler tes enfants, comme une poule rassemble sa couvée sous ses ailes, et vous ne l`avez pas voulu! 
\verse Voici, votre maison vous sera laissée; mais, je vous le dis, vous ne me verrez plus, jusqu`à ce que vous disiez: Béni soit celui qui vient au nom du Seigneur! 

\chapter
\verse Jésus étant entré, un jour de sabbat, dans la maison de l`un des chefs des pharisiens, pour prendre un repas, les pharisiens l`observaient. 
\verse Et voici, un homme hydropique était devant lui. 
\verse Jésus prit la parole, et dit aux docteurs de la loi et aux pharisiens: Est-il permis, ou non, de faire une guérison le jour du sabbat? 
\verse Ils gardèrent le silence. Alors Jésus avança la main sur cet homme, le guérit, et le renvoya. 
\verse Puis il leur dit: Lequel de vous, si son fils ou son boeuf tombe dans un puits, ne l`en retirera pas aussitôt, le jour du sabbat? 
\verse Et ils ne purent rien répondre à cela. 
\verse Il adressa ensuite une parabole aux conviés, en voyant qu`ils choisissaient les premières places; et il leur dit: 
\verse Lorsque tu seras invité par quelqu`un à des noces, ne te mets pas à la première place, de peur qu`il n`y ait parmi les invités une personne plus considérable que toi, 
\verse et que celui qui vous a invités l`un et l`autre ne vienne te dire: Cède la place à cette personne-là. Tu aurais alors la honte d`aller occuper la dernière place. 
\verse Mais, lorsque tu seras invité, va te mettre à la dernière place, afin que, quand celui qui t`a invité viendra, il te dise: Mon ami, monte plus haut. Alors cela te fera honneur devant tous ceux qui seront à table avec toi. 
\verse Car quiconque s`élève sera abaissé, et quiconque s`abaisse sera élevé. 
\verse Il dit aussi à celui qui l`avait invité: Lorsque tu donnes à dîner ou à souper, n`invite pas tes amis, ni tes frères, ni tes parents, ni des voisins riches, de peur qu`ils ne t`invitent à leur tour et qu`on ne te rende la pareille. 
\verse Mais, lorsque tu donnes un festin, invite des pauvres, des estropiés, des boiteux, des aveugles. 
\verse Et tu seras heureux de ce qu`ils ne peuvent pas te rendre la pareille; car elle te sera rendue à la résurrection des justes. 
\verse Un de ceux qui étaient à table, après avoir entendu ces paroles, dit à Jésus: Heureux celui qui prendra son repas dans le royaume de Dieu! 
\verse Et Jésus lui répondit: Un homme donna un grand souper, et il invita beaucoup de gens. 
\verse A l`heure du souper, il envoya son serviteur dire aux conviés: Venez, car tout est déjà prêt. 
\verse Mais tous unanimement se mirent à s`excuser. Le premier lui dit: J`ai acheté un champ, et je suis obligé d`aller le voir; excuse-moi, je te prie. 
\verse Un autre dit: J`ai acheté cinq paires de boeufs, et je vais les essayer; excuse-moi, je te prie. 
\verse Un autre dit: Je viens de me marier, et c`est pourquoi je ne puis aller. 
\verse Le serviteur, de retour, rapporta ces choses à son maître. Alors le maître de la maison irrité dit à son serviteur: Va promptement dans les places et dans les rues de la ville, et amène ici les pauvres, les estropiés, les aveugles et les boiteux. 
\verse Le serviteur dit: Maître, ce que tu as ordonné a été fait, et il y a encore de la place. 
\verse Et le maître dit au serviteur: Va dans les chemins et le long des haies, et ceux que tu trouveras, contrains-les d`entrer, afin que ma maison soit remplie. 
\verse Car, je vous le dis, aucun de ces hommes qui avaient été invités ne goûtera de mon souper. 
\verse De grandes foules faisaient route avec Jésus. Il se retourna, et leur dit: 
\verse Si quelqu`un vient à moi, et s`il ne hait pas son père, sa mère, sa femme, ses enfants, ses frères, et ses soeurs, et même à sa propre vie, il ne peut être mon disciple. 
\verse Et quiconque ne porte pas sa croix, et ne me suis pas, ne peut être mon disciple. 
\verse Car, lequel de vous, s`il veut bâtir une tour, ne s`assied d`abord pour calculer la dépense et voir s`il a de quoi la terminer, 
\verse de peur qu`après avoir posé les fondements, il ne puisse l`achever, et que tous ceux qui le verront ne se mettent à le railler, 
\verse en disant: Cet homme a commencé à bâtir, et il n`a pu achever? 
\verse Ou quel roi, s`il va faire la guerre à un autre roi, ne s`assied d`abord pour examiner s`il peut, avec dix mille hommes, marcher à la rencontre de celui qui vient l`attaquer avec vingt mille? 
\verse S`il ne le peut, tandis que cet autre roi est encore loin, il lui envoie une ambassade pour demander la paix. 
\verse Ainsi donc, quiconque d`entre vous ne renonce pas à tout ce qu`il possède ne peut être mon disciple. 
\verse Le sel est une bonne chose; mais si le sel perd sa saveur, avec quoi l`assaisonnera-t-on? 
\verse Il n`est bon ni pour la terre, ni pour le fumier; on le jette dehors. Que celui qui a des oreilles pour entendre entende. 

\chapter
\verse Tous les publicains et les gens de mauvaise vie s`approchaient de Jésus pour l`entendre. 
\verse Et les pharisiens et les scribes murmuraient, disant: Cet homme accueille des gens de mauvaise vie, et mange avec eux. 
\verse Mais il leur dit cette parabole: 
\verse Quel homme d`entre vous, s`il a cent brebis, et qu`il en perde une, ne laisse les quatre-vingt-dix-neuf autres dans le désert pour aller après celle qui est perdue, jusqu`à ce qu`il la retrouve? 
\verse Lorsqu`il l`a retrouvée, il la met avec joie sur ses épaules, 
\verse et, de retour à la maison, il appelle ses amis et ses voisins, et leur dit: Réjouissez-vous avec moi, car j`ai retrouvé ma brebis qui était perdue. 
\verse De même, je vous le dis, il y aura plus de joie dans le ciel pour un seul pécheur qui se repent, que pour quatre-vingt-dix-neuf justes qui n`ont pas besoin de repentance. 
\verse Ou quelle femme, si elle a dix drachmes, et qu`elle en perde une, n`allume une lampe, ne balaie la maison, et ne cherche avec soin, jusqu`à ce qu`elle la retrouve? 
\verse Lorsqu`elle l`a retrouvée, elle appelle ses amies et ses voisines, et dit: Réjouissez-vous avec moi, car j`ai retrouvé la drachme que j`avais perdue. 
\verse De même, je vous le dis, il y a de la joie devant les anges de Dieu pour un seul pécheur qui se repent. 
\verse Il dit encore: Un homme avait deux fils. 
\verse Le plus jeune dit à son père: Mon père, donne-moi la part de bien qui doit me revenir. Et le père leur partagea son bien. 
\verse Peu de jours après, le plus jeune fils, ayant tout ramassé, partit pour un pays éloigné, où il dissipa son bien en vivant dans la débauche. 
\verse Lorsqu`il eut tout dépensé, une grande famine survint dans ce pays, et il commença à se trouver dans le besoin. 
\verse Il alla se mettre au service d`un des habitants du pays, qui l`envoya dans ses champs garder les pourceaux. 
\verse Il aurait bien voulu se rassasier des carouges que mangeaient les pourceaux, mais personne ne lui en donnait. 
\verse Étant rentré en lui-même, il se dit: Combien de mercenaires chez mon père ont du pain en abondance, et moi, ici, je meurs de faim! 
\verse Je me lèverai, j`irai vers mon père, et je lui dirai: Mon père, j`ai péché contre le ciel et contre toi, 
\verse je ne suis plus digne d`être appelé ton fils; traite-moi comme l`un de tes mercenaires. 
\verse Et il se leva, et alla vers son père. Comme il était encore loin, son père le vit et fut ému de compassion, il courut se jeter à son cou et le baisa. 
\verse Le fils lui dit: Mon père, j`ai péché contre le ciel et contre toi, je ne suis plus digne d`être appelé ton fils. 
\verse Mais le père dit à ses serviteurs: Apportez vite la plus belle robe, et l`en revêtez; mettez-lui un anneau au doigt, et des souliers aux pieds. 
\verse Amenez le veau gras, et tuez-le. Mangeons et réjouissons-nous; 
\verse car mon fils que voici était mort, et il est revenu à la vie; il était perdu, et il est retrouvé. Et ils commencèrent à se réjouir. 
\verse Or, le fils aîné était dans les champs. Lorsqu`il revint et approcha de la maison, il entendit la musique et les danses. 
\verse Il appela un des serviteurs, et lui demanda ce que c`était. 
\verse Ce serviteur lui dit: Ton frère est de retour, et, parce qu`il l`a retrouvé en bonne santé, ton père a tué le veau gras. 
\verse Il se mit en colère, et ne voulut pas entrer. Son père sortit, et le pria d`entrer. 
\verse Mais il répondit à son père: Voici, il y a tant d`années que je te sers, sans avoir jamais transgressé tes ordres, et jamais tu ne m`as donné un chevreau pour que je me réjouisse avec mes amis. 
\verse Et quand ton fils est arrivé, celui qui a mangé ton bien avec des prostituées, c`est pour lui que tu as tué le veau gras! 
\verse Mon enfant, lui dit le père, tu es toujours avec moi, et tout ce que j`ai est à toi; 
\verse mais il fallait bien s`égayer et se réjouir, parce que ton frère que voici était mort et qu`il est revenu à la vie, parce qu`il était perdu et qu`il est retrouvé. 

\chapter
\verse Jésus dit aussi à ses disciples: Un homme riche avait un économe, qui lui fut dénoncé comme dissipant ses biens. 
\verse Il l`appela, et lui dit: Qu`est-ce que j`entends dire de toi? Rends compte de ton administration, car tu ne pourras plus administrer mes biens. 
\verse L`économe dit en lui-même: Que ferai-je, puisque mon maître m`ôte l`administration de ses biens? Travailler à la terre? je ne le puis. Mendier? j`en ai honte. 
\verse Je sais ce que je ferai, pour qu`il y ait des gens qui me reçoivent dans leurs maisons quand je serai destitué de mon emploi. 
\verse Et, faisant venir chacun des débiteurs de son maître, il dit au premier: Combien dois-tu à mon maître? 
\verse Cent mesures d`huile, répondit-il. Et il lui dit: Prends ton billet, assieds-toi vite, et écris cinquante. 
\verse Il dit ensuite à un autre: Et toi, combien dois-tu? Cent mesures de blé, répondit-il. Et il lui dit: Prends ton billet, et écris quatre-vingts. 
\verse Le maître loua l`économe infidèle de ce qu`il avait agi prudemment. Car les enfants de ce siècle sont plus prudents à l`égard de leurs semblables que ne le sont les enfants de lumière. 
\verse Et moi, je vous dis: Faites-vous des amis avec les richesses injustes, pour qu`ils vous reçoivent dans les tabernacles éternels, quand elles viendront à vous manquer. 
\verse Celui qui est fidèle dans les moindres choses l`est aussi dans les grandes, et celui qui est injuste dans les moindres choses l`est aussi dans les grandes. 
\verse Si donc vous n`avez pas été fidèle dans les richesses injustes, qui vous confiera les véritables? 
\verse Et si vous n`avez pas été fidèles dans ce qui est à autrui, qui vous donnera ce qui est à vous? 
\verse Nul serviteur ne peut servir deux maîtres. Car, ou il haïra l`un et aimera l`autre; ou il s`attachera à l`un et méprisera l`autre. Vous ne pouvez servir Dieu et Mamon. 
\verse Les pharisiens, qui étaient avares, écoutaient aussi tout cela, et ils se moquaient de lui. 
\verse Jésus leur dit: Vous, vous cherchez à paraître justes devant les hommes, mais Dieu connaît vos coeurs; car ce qui est élevé parmi les hommes est une abomination devant Dieu. 
\verse La loi et les prophètes ont subsisté jusqu`à Jean; depuis lors, le royaume de Dieu est annoncé, et chacun use de violence pour y entrer. 
\verse Il est plus facile que le ciel et la terre passent, qu`il ne l`est qu`un seul trait de lettre de la loi vienne à tomber. 
\verse Quiconque répudie sa femme et en épouse une autre commet un adultère, et quiconque épouse une femme répudiée par son mari commet un adultère. 
\verse Il y avait un homme riche, qui était vêtu de pourpre et de fin lin, et qui chaque jour menait joyeuse et brillante vie. 
\verse Un pauvre, nommé Lazare, était couché à sa porte, couvert d`ulcères, 
\verse et désireux de se rassasier des miettes qui tombaient de la table du riche; et même les chiens venaient encore lécher ses ulcères. 
\verse Le pauvre mourut, et il fut porté par les anges dans le sein d`Abraham. Le riche mourut aussi, et il fut enseveli. 
\verse Dans le séjour des morts, il leva les yeux; et, tandis qu`il était en proie aux tourments, il vit de loin Abraham, et Lazare dans son sein. 
\verse Il s`écria: Père Abraham, aie pitié de moi, et envoie Lazare, pour qu`il trempe le bout de son doigt dans l`eau et me rafraîchisse la langue; car je souffre cruellement dans cette flamme. 
\verse Abraham répondit: Mon enfant, souviens-toi que tu as reçu tes biens pendant ta vie, et que Lazare a eu les maux pendant la sienne; maintenant il est ici consolé, et toi, tu souffres. 
\verse D`ailleurs, il y a entre nous et vous un grand abîme, afin que ceux qui voudraient passer d`ici vers vous, ou de là vers nous, ne puissent le faire. 
\verse Le riche dit: Je te prie donc, père Abraham, d`envoyer Lazare dans la maison de mon père; car j`ai cinq frères. 
\verse C`est pour qu`il leur atteste ces choses, afin qu`ils ne viennent pas aussi dans ce lieu de tourments. 
\verse Abraham répondit: Ils ont Moïse et les prophètes; qu`ils les écoutent. 
\verse Et il dit: Non, père Abraham, mais si quelqu`un des morts va vers eux, ils se repentiront. 
\verse Et Abraham lui dit: S`ils n`écoutent pas Moïse et les prophètes, ils ne se laisseront pas persuader quand même quelqu`un des morts ressusciterait. 

\chapter
\verse Jésus dit à ses disciples: Il est impossible qu`il n`arrive pas des scandales; mais malheur à celui par qui ils arrivent! 
\verse Il vaudrait mieux pour lui qu`on mît à son cou une pierre de moulin et qu`on le jetât dans la mer, que s`il scandalisait un de ces petits. 
\verse Prenez garde à vous-mêmes. Si ton frère a péché, reprends-le; et, s`il se repent, pardonne-lui. 
\verse Et s`il a péché contre toi sept fois dans un jour et que sept fois il revienne à toi, disant: Je me repens, -tu lui pardonneras. 
\verse Les apôtres dirent au Seigneur: Augmente-nous la foi. 
\verse Et le Seigneur dit: Si vous aviez de la foi comme un grain de sénevé, vous diriez à ce sycomore: Déracine-toi, et plante-toi dans la mer; et il vous obéirait. 
\verse Qui de vous, ayant un serviteur qui laboure ou paît les troupeaux, lui dira, quand il revient des champs: Approche vite, et mets-toi à table? 
\verse Ne lui dira-t-il pas au contraire: Prépare-moi à souper, ceins-toi, et sers-moi, jusqu`à ce que j`aie mangé et bu; après cela, toi, tu mangeras et boiras? 
\verse Doit-il de la reconnaissance à ce serviteur parce qu`il a fait ce qui lui était ordonné? 
\verse Vous de même, quand vous avez fait tout ce qui vous a été ordonné, dites: Nous sommes des serviteurs inutiles, nous avons fait ce que nous devions faire. 
\verse Jésus, se rendant à Jérusalem, passait entre la Samarie et la Galilée. 
\verse Comme il entrait dans un village, dix lépreux vinrent à sa rencontre. Se tenant à distance, ils élevèrent la voix, et dirent: 
\verse Jésus, maître, aie pitié de nous! 
\verse Dès qu`il les eut vus, il leur dit: Allez vous montrer aux sacrificateurs. Et, pendant qu`ils y allaient, il arriva qu`ils furent guéris. 
\verse L`un deux, se voyant guéri, revint sur ses pas, glorifiant Dieu à haute voix. 
\verse Il tomba sur sa face aux pieds de Jésus, et lui rendit grâces. C`était un Samaritain. 
\verse Jésus, prenant la parole, dit: Les dix n`ont-ils pas été guéris? Et les neuf autres, où sont-ils? 
\verse Ne s`est-il trouvé que cet étranger pour revenir et donner gloire à Dieu? 
\verse Puis il lui dit: Lève-toi, va; ta foi t`a sauvé. 
\verse Les pharisiens demandèrent à Jésus quand viendrait le royaume de Dieu. Il leur répondit: Le royaume de Dieu ne vient pas de manière à frapper les regards. 
\verse On ne dira point: Il est ici, ou: Il est là. Car voici, le royaume de Dieu est au milieu de vous. 
\verse Et il dit aux disciples: Des jours viendront où vous désirerez voir l`un des jours du Fils de l`homme, et vous ne le verrez point. 
\verse On vous dira: Il est ici, il est là. N`y allez pas, ne courez pas après. 
\verse Car, comme l`éclair resplendit et brille d`une extrémité du ciel à l`autre, ainsi sera le Fils de l`homme en son jour. 
\verse Mais il faut auparavant qu`il souffre beaucoup, et qu`il soit rejeté par cette génération. 
\verse Ce qui arriva du temps de Noé arrivera de même aux jours du Fils de l`homme. 
\verse Les hommes mangeaient, buvaient, se mariaient et mariaient leurs enfants, jusqu`au jour où Noé entra dans l`arche; le déluge vint, et les fit tous périr. 
\verse Ce qui arriva du temps de Lot arrivera pareillement. Les hommes mangeaient, buvaient, achetaient, vendaient, plantaient, bâtissaient; 
\verse mais le jour où Lot sortit de Sodome, une pluie de feu et de souffre tomba du ciel, et les fit tous périr. 
\verse Il en sera de même le jour où le Fils de l`homme paraîtra. 
\verse En ce jour-là, que celui qui sera sur le toit, et qui aura ses effets dans la maison, ne descende pas pour les prendre; et que celui qui sera dans les champs ne retourne pas non plus en arrière. 
\verse Souvenez-vous de la femme de Lot. 
\verse Celui qui cherchera à sauver sa vie la perdra, et celui qui la perdra la retrouvera. 
\verse Je vous le dis, en cette nuit-là, de deux personnes qui seront dans un même lit, l`une sera prise et l`autre laissée; 
\verse de deux femmes qui moudront ensemble, l`une sera prise et l`autre laissée. 
\verse De deux hommes qui seront dans un champ, l`un sera pris et l`autre laissé. 
\verse Les disciples lui dirent: Où sera-ce, Seigneur? Et il répondit: Où sera le corps, là s`assembleront les aigles. 

\chapter
\verse Jésus leur adressa une parabole, pour montrer qu`il faut toujours prier, et ne point se relâcher. 
\verse Il dit: Il y avait dans une ville un juge qui ne craignait point Dieu et qui n`avait d`égard pour personne. 
\verse Il y avait aussi dans cette ville une veuve qui venait lui dire: Fais-moi justice de ma partie adverse. 
\verse Pendant longtemps il refusa. Mais ensuite il dit en lui-même: Quoique je ne craigne point Dieu et que je n`aie d`égard pour personne, 
\verse néanmoins, parce que cette veuve m`importune, je lui ferai justice, afin qu`elle ne vienne pas sans cesse me rompre la tête. 
\verse Le Seigneur ajouta: Entendez ce que dit le juge inique. 
\verse Et Dieu ne fera-t-il pas justice à ses élus, qui crient à lui jour et nuit, et tardera-t-il à leur égard? 
\verse Je vous le dis, il leur fera promptement justice. Mais, quand le Fils de l`homme viendra, trouvera-t-il la foi sur la terre? 
\verse Il dit encore cette parabole, en vue de certaines personnes se persuadant qu`elles étaient justes, et ne faisant aucun cas des autres: 
\verse Deux hommes montèrent au temple pour prier; l`un était pharisien, et l`autre publicain. 
\verse Le pharisien, debout, priait ainsi en lui-même: O Dieu, je te rends grâces de ce que je ne suis pas comme le reste des hommes, qui sont ravisseurs, injustes, adultères, ou même comme ce publicain; 
\verse je jeûne deux fois la semaine, je donne la dîme de tous mes revenus. 
\verse Le publicain, se tenant à distance, n`osait même pas lever les yeux au ciel; mais il se frappait la poitrine, en disant: O Dieu, sois apaisé envers moi, qui suis un pécheur. 
\verse Je vous le dis, celui-ci descendit dans sa maison justifié, plutôt que l`autre. Car quiconque s`élève sera abaissé, et celui qui s`abaisse sera élevé. 
\verse On lui amena aussi les petits enfants, afin qu`il les touchât. Mais les disciples, voyant cela, reprenaient ceux qui les amenaient. 
\verse Et Jésus les appela, et dit: Laissez venir à moi les petits enfants, et ne les en empêchez pas; car le royaume de Dieu est pour ceux qui leur ressemblent. 
\verse Je vous le dis en vérité, quiconque ne recevra pas le royaume de Dieu comme un petit enfant n`y entrera point. 
\verse Un chef interrogea Jésus, et dit: Bon maître, que dois-je faire pour hériter la vie éternelle? 
\verse Jésus lui répondit: Pourquoi m`appelles-tu bon? Il n`y a de bon que Dieu seul. 
\verse Tu connais les commandements: Tu ne commettras point d`adultère; tu ne tueras point; tu ne déroberas point; tu ne diras point de faux témoignage; honore ton père et ta mère. 
\verse J`ai, dit-il, observé toutes ces choses dès ma jeunesse. 
\verse Jésus, ayant entendu cela, lui dit: Il te manque encore une chose: vends tout ce que tu as, distribue-le aux pauvres, et tu auras un trésor dans les cieux. Puis, viens, et suis-moi. 
\verse Lorsqu`il entendit ces paroles, il devint tout triste; car il était très riche. 
\verse Jésus, voyant qu`il était devenu tout triste, dit: Qu`il est difficile à ceux qui ont des richesses d`entrer dans le royaume de Dieu! 
\verse Car il est plus facile à un chameau de passer par le trou d`une aiguille qu`à un riche d`entrer dans le royaume de Dieu. 
\verse Ceux qui l`écoutaient dirent: Et qui peut être sauvé? 
\verse Jésus répondit: Ce qui est impossible aux hommes est possible à Dieu. 
\verse Pierre dit alors: Voici, nous avons tout quitté, et nous t`avons suivi. 
\verse Et Jésus leur dit: Je vous le dis en vérité, il n`est personne qui, ayant quitté, à cause du royaume de Dieu, sa maison, ou sa femme, ou ses frères, ou ses parents, ou ses enfants, 
\verse ne reçoive beaucoup plus dans ce siècle-ci, et, dans le siècle à venir, la vie éternelle. 
\verse Jésus prit les douze auprès de lui, et leur dit: Voici, nous montons à Jérusalem, et tout ce qui a été écrit par les prophètes au sujet du Fils de l`homme s`accomplira. 
\verse Car il sera livré aux païens; on se moquera de lui, on l`outragera, on crachera sur lui, 
\verse et, après l`avoir battu de verges, on le fera mourir; et le troisième jour il ressuscitera. 
\verse Mais ils ne comprirent rien à cela; c`était pour eux un langage caché, des paroles dont ils ne saisissaient pas le sens. 
\verse Comme Jésus approchait de Jéricho, un aveugle était assis au bord du chemin, et mendiait. 
\verse Entendant la foule passer, il demanda ce que c`était. 
\verse On lui dit: C`est Jésus de Nazareth qui passe. 
\verse Et il cria: Jésus, Fils de David, aie pitié de moi! 
\verse Ceux qui marchaient devant le reprenaient, pour le faire taire; mais il criait beaucoup plus fort: Fils de David, aie pitié de moi! 
\verse Jésus, s`étant arrêté, ordonna qu`on le lui amène; et, quand il se fut approché, 
\verse il lui demanda: Que veux-tu que je te fasse? Il répondit: Seigneur, que je recouvre la vue. 
\verse Et Jésus lui dit: Recouvre la vue; ta foi t`a sauvé. 
\verse A l`instant il recouvra la vue, et suivit Jésus, en glorifiant Dieu. Tout le peuple, voyant cela, loua Dieu. 

\chapter
\verse Jésus, étant entré dans Jéricho, traversait la ville. 
\verse Et voici, un homme riche, appelé Zachée, chef des publicains, cherchait à voir qui était Jésus; 
\verse mais il ne pouvait y parvenir, à cause de la foule, car il était de petite taille. 
\verse Il courut en avant, et monta sur un sycomore pour le voir, parce qu`il devait passer par là. 
\verse Lorsque Jésus fut arrivé à cet endroit, il leva les yeux et lui dit: Zachée, hâte-toi de descendre; car il faut que je demeure aujourd`hui dans ta maison. 
\verse Zachée se hâta de descendre, et le reçut avec joie. 
\verse Voyant cela, tous murmuraient, et disaient: Il est allé loger chez un homme pécheur. 
\verse Mais Zachée, se tenant devant le Seigneur, lui dit: Voici, Seigneur, je donne aux pauvres la moitié de mes biens, et, si j`ai fait tort de quelque chose à quelqu`un, je lui rends le quadruple. 
\verse Jésus lui dit: Le salut est entré aujourd`hui dans cette maison, parce que celui-ci est aussi un fils d`Abraham. 
\verse Car le Fils de l`homme est venu chercher et sauver ce qui était perdu. 
\verse Ils écoutaient ces choses, et Jésus ajouta une parabole, parce qu`il était près de Jérusalem, et qu`on croyait qu`à l`instant le royaume de Dieu allait paraître. 
\verse Il dit donc: Un homme de haute naissance s`en alla dans un pays lointain, pour se faire investir de l`autorité royale, et revenir ensuite. 
\verse Il appela dix de ses serviteurs, leur donna dix mines, et leur dit: Faites-les valoir jusqu`à ce que je revienne. 
\verse Mais ses concitoyens le haïssaient, et ils envoyèrent une ambassade après lui, pour dire: Nous ne voulons pas que cet homme règne sur nous. 
\verse Lorsqu`il fut de retour, après avoir été investi de l`autorité royale, il fit appeler auprès de lui les serviteurs auxquels il avait donné l`argent, afin de connaître comment chacun l`avait fait valoir. 
\verse Le premier vint, et dit: Seigneur, ta mine a rapporté dix mines. 
\verse Il lui dit: C`est bien, bon serviteur; parce que tu as été fidèle en peu de chose, reçois le gouvernement de dix villes. 
\verse Le second vint, et dit: Seigneur, ta mine a produit cinq mines. 
\verse Il lui dit: Toi aussi, sois établi sur cinq villes. 
\verse Un autre vint, et dit: Seigneur, voici ta mine, que j`ai gardée dans un linge; 
\verse car j`avais peur de toi, parce que tu es un homme sévère; tu prends ce que tu n`as pas déposé, et tu moissonnes ce que tu n`as pas semé. 
\verse Il lui dit: Je te juge sur tes paroles, méchant serviteur; tu savais que je suis un homme sévère, prenant ce que je n`ai pas déposé, et moissonnant ce que je n`ai pas semé; 
\verse pourquoi donc n`as-tu pas mis mon argent dans une banque, afin qu`à mon retour je le retirasse avec un intérêt? 
\verse Puis il dit à ceux qui étaient là: Otez-lui la mine, et donnez-la à celui qui a les dix mines. 
\verse Ils lui dirent: Seigneur, il a dix mines. - 
\verse Je vous le dis, on donnera à celui qui a, mais à celui qui n`a pas on ôtera même ce qu`il a. 
\verse Au reste, amenez ici mes ennemis, qui n`ont pas voulu que je régnasse sur eux, et tuez-les en ma présence. 
\verse Après avoir ainsi parlé, Jésus marcha devant la foule, pour monter à Jérusalem. 
\verse Lorsqu`il approcha de Bethphagé et de Béthanie, vers la montagne appelée montagne des Oliviers, Jésus envoya deux de ses disciples, 
\verse en disant: Allez au village qui est en face; quand vous y serez entrés, vous trouverez un ânon attaché, sur lequel aucun homme ne s`est jamais assis; détachez-le, et amenez-le. 
\verse Si quelqu`un vous demande: Pourquoi le détachez-vous? vous lui répondrez: Le Seigneur en a besoin. 
\verse Ceux qui étaient envoyés allèrent, et trouvèrent les choses comme Jésus leur avait dit. 
\verse Comme ils détachaient l`ânon, ses maîtres leur dirent: Pourquoi détachez-vous l`ânon? 
\verse Ils répondirent: Le Seigneur en a besoin. 
\verse Et ils amenèrent à Jésus l`ânon, sur lequel ils jetèrent leurs vêtements, et firent monter Jésus. 
\verse Quand il fut en marche, les gens étendirent leurs vêtements sur le chemin. 
\verse Et lorsque déjà il approchait de Jérusalem, vers la descente de la montagne des Oliviers, toute la multitude des disciples, saisie de joie, se mit à louer Dieu à haute voix pour tous les miracles qu`ils avaient vus. 
\verse Ils disaient: Béni soit le roi qui vient au nom du Seigneur! Paix dans le ciel, et gloire dans les lieux très hauts! 
\verse Quelques pharisiens, du milieu de la foule, dirent à Jésus: Maître, reprends tes disciples. 
\verse Et il répondit: Je vous le dis, s`ils se taisent, les pierres crieront! 
\verse Comme il approchait de la ville, Jésus, en la voyant, pleura sur elle, et dit: 
\verse Si toi aussi, au moins en ce jour qui t`est donné, tu connaissais les choses qui appartiennent à ta paix! Mais maintenant elles sont cachées à tes yeux. 
\verse Il viendra sur toi des jours où tes ennemis t`environneront de tranchées, t`enfermeront, et te serreront de toutes parts; 
\verse ils te détruiront, toi et tes enfants au milieu de toi, et ils ne laisseront pas en toi pierre sur pierre, parce que tu n`as pas connu le temps où tu as été visitée. 
\verse Il entra dans le temple, et il se mit à chasser ceux qui vendaient, 
\verse leur disant: Il est écrit: Ma maison sera une maison de prière. Mais vous, vous en avez fait une caverne de voleurs. 
\verse Il enseignait tous les jours dans le temple. Et les principaux sacrificateurs, les scribes, et les principaux du peuple cherchaient à le faire périr; 
\verse mais ils ne savaient comment s`y prendre, car tout le peuple l`écoutait avec admiration. 

\chapter
\verse Un de ces jours-là, comme Jésus enseignait le peuple dans le temple et qu`il annonçait la bonne nouvelle, les principaux sacrificateurs et les scribes, avec les anciens, survinrent, 
\verse et lui dirent: Dis-nous, par quelle autorité fais-tu ces choses, ou qui est celui qui t`a donné cette autorité? 
\verse Il leur répondit: Je vous adresserai aussi une question. 
\verse Dites-moi, le baptême de Jean venait-il du ciel, ou des hommes? 
\verse Mais ils raisonnèrent ainsi entre eux: Si nous répondons: Du ciel, il dira: Pourquoi n`avez-vous pas cru en lui? 
\verse Et si nous répondons: Des hommes, tout le peuple nous lapidera, car il est persuadé que Jean était un prophète. 
\verse Alors ils répondirent qu`ils ne savaient d`où il venait. 
\verse Et Jésus leur dit: Moi non plus, je ne vous dirai pas par quelle autorité je fais ces choses. 
\verse Il se mit ensuite à dire au peuple cette parabole: Un homme planta une vigne, l`afferma à des vignerons, et quitta pour longtemps le pays. 
\verse Au temps de la récolte, il envoya un serviteur vers les vignerons, pour qu`ils lui donnent une part du produit de la vigne. Les vignerons le battirent, et le renvoyèrent à vide. 
\verse Il envoya encore un autre serviteur; ils le battirent, l`outragèrent, et le renvoyèrent à vide. 
\verse Il en envoya encore un troisième; ils le blessèrent, et le chassèrent. 
\verse Le maître de la vigne dit: Que ferai-je? J`enverrai mon fils bien-aimé; peut-être auront-ils pour lui du respect. 
\verse Mais, quand les vignerons le virent, ils raisonnèrent entre eux, et dirent: Voici l`héritier; tuons-le, afin que l`héritage soit à nous. 
\verse Et ils le jetèrent hors de la vigne, et le tuèrent. Maintenant, que leur fera le maître de la vigne? 
\verse Il viendra, fera périr ces vignerons, et il donnera la vigne à d`autres. Lorsqu`ils eurent entendu cela, ils dirent: A Dieu ne plaise! 
\verse Mais, jetant les regards sur eux, Jésus dit: Que signifie donc ce qui est écrit: La pierre qu`ont rejetée ceux qui bâtissaient Est devenue la principale de l`angle? 
\verse Quiconque tombera sur cette pierre s`y brisera, et celui sur qui elle tombera sera écrasé. 
\verse Les principaux sacrificateurs et les scribes cherchèrent à mettre la main sur lui à l`heure même, mais ils craignirent le peuple. Ils avaient compris que c`était pour eux que Jésus avait dit cette parabole. 
\verse Ils se mirent à observer Jésus; et ils envoyèrent des gens qui feignaient d`être justes, pour lui tendre des pièges et saisir de lui quelque parole, afin de le livrer au magistrat et à l`autorité du gouverneur. 
\verse Ces gens lui posèrent cette question: Maître, nous savons que tu parles et enseignes droitement, et que tu ne regardes pas à l`apparence, mais que tu enseignes la voie de Dieu selon la vérité. 
\verse Nous est-il permis, ou non, de payer le tribut à César? 
\verse Jésus, apercevant leur ruse, leur répondit: Montrez-moi un denier. 
\verse De qui porte-t-il l`effigie et l`inscription? De César, répondirent-ils. 
\verse Alors il leur dit: Rendez donc à César ce qui est à César, et à Dieu ce qui est à Dieu. 
\verse Ils ne purent rien reprendre dans ses paroles devant le peuple; mais, étonnés de sa réponse, ils gardèrent le silence. 
\verse Quelques-uns des sadducéens, qui disent qu`il n`y a point de résurrection, s`approchèrent, et posèrent à Jésus cette question: 
\verse Maître, voici ce que Moïse nous a prescrit: Si le frère de quelqu`un meurt, ayant une femme sans avoir d`enfants, son frère épousera la femme, et suscitera une postérité à son frère. 
\verse Or, il y avait sept frères. Le premier se maria, et mourut sans enfants. 
\verse Le second et le troisième épousèrent la veuve; 
\verse il en fut de même des sept, qui moururent sans laisser d`enfants. 
\verse Enfin, la femme mourut aussi. 
\verse A la résurrection, duquel d`entre eux sera-t-elle donc la femme? Car les sept l`ont eue pour femme. 
\verse Jésus leur répondit: Les enfants de ce siècle prennent des femmes et des maris; 
\verse mais ceux qui seront trouvés dignes d`avoir part au siècle à venir et à la résurrection des morts ne prendront ni femmes ni maris. 
\verse Car ils ne pourront plus mourir, parce qu`ils seront semblables aux anges, et qu`ils seront fils de Dieu, étant fils de la résurrection. 
\verse Que les morts ressuscitent, c`est ce que Moïse a fait connaître quand, à propos du buisson, il appelle le Seigneur le Dieu d`Abraham, le Dieu d`Isaac, et le Dieu de Jacob. 
\verse Or, Dieu n`est pas Dieu des morts, mais des vivants; car pour lui tous sont vivants. 
\verse Quelques-uns des scribes, prenant la parole, dirent: Maître, tu as bien parlé. 
\verse Et ils n`osaient plus lui faire aucune question. 
\verse Jésus leur dit: Comment dit-on que le Christ est fils de David? 
\verse David lui-même dit dans le livre des Psaumes: Le Seigneur a dit à mon Seigneur: Assieds-toi à ma droite, 
\verse Jusqu`à ce que je fasse de tes ennemis ton marchepied. 
\verse David donc l`appelle Seigneur; comment est-il son fils? 
\verse Tandis que tout le peuple l`écoutait, il dit à ses disciples: 
\verse Gardez-vous des scribes, qui aiment à se promener en robes longues, et à être salués dans les places publiques; qui recherchent les premiers sièges dans les synagogues, et les premières places dans les festins; 
\verse qui dévorent les maisons des veuves, et qui font pour l`apparence de longues prières. Ils seront jugés plus sévèrement. 

\chapter
\verse Jésus, ayant levé les yeux, vit les riches qui mettaient leurs offrandes dans le tronc. 
\verse Il vit aussi une pauvre veuve, qui y mettait deux petites pièces. 
\verse Et il dit: Je vous le dis en vérité, cette pauvre veuve a mis plus que tous les autres; 
\verse car c`est de leur superflu que tous ceux-là ont mis des offrandes dans le tronc, mais elle a mis de son nécessaire, tout ce qu`elle avait pour vivre. 
\verse Comme quelques-uns parlaient des belles pierres et des offrandes qui faisaient l`ornement du temple, Jésus dit: 
\verse Les jours viendront où, de ce que vous voyez, il ne restera pas pierre sur pierre qui ne soit renversée. 
\verse Ils lui demandèrent: Maître, quand donc cela arrivera-t-il, et à quel signe connaîtra-t-on que ces choses vont arriver? 
\verse Jésus répondit: Prenez garde que vous ne soyez séduits. Car plusieurs viendront en mon nom, disant: C`est moi, et le temps approche. Ne les suivez pas. 
\verse Quand vous entendrez parler de guerres et de soulèvements, ne soyez pas effrayés, car il faut que ces choses arrivent premièrement. Mais ce ne sera pas encore la fin. 
\verse Alors il leur dit: Une nation s`élèvera contre une nation, et un royaume contre un royaume; 
\verse il y aura de grands tremblements de terre, et, en divers lieux, des pestes et des famines; il y aura des phénomènes terribles, et de grands signes dans le ciel. 
\verse Mais, avant tout cela, on mettra la main sur vous, et l`on vous persécutera; on vous livrera aux synagogues, on vous jettera en prison, on vous mènera devant des rois et devant des gouverneurs, à cause de mon nom. 
\verse Cela vous arrivera pour que vous serviez de témoignage. 
\verse Mettez-vous donc dans l`esprit de ne pas préméditer votre défense; 
\verse car je vous donnerai une bouche et une sagesse à laquelle tous vos adversaires ne pourront résister ou contredire. 
\verse Vous serez livrés même par vos parents, par vos frères, par vos proches et par vos amis, et ils feront mourir plusieurs d`entre vous. 
\verse Vous serez haïs de tous, à cause de mon nom. 
\verse Mais il ne se perdra pas un cheveu de votre tête; 
\verse par votre persévérance vous sauverez vos âmes. 
\verse Lorsque vous verrez Jérusalem investie par des armées, sachez alors que sa désolation est proche. 
\verse Alors, que ceux qui seront en Judée fuient dans les montagnes, que ceux qui seront au milieu de Jérusalem en sortent, et que ceux qui seront dans les champs n`entrent pas dans la ville. 
\verse Car ce seront des jours de vengeance, pour l`accomplissement de tout ce qui est écrit. 
\verse Malheur aux femmes qui seront enceintes et à celles qui allaiteront en ces jours-là! Car il y aura une grande détresse dans le pays, et de la colère contre ce peuple. 
\verse Ils tomberont sous le tranchant de l`épée, ils seront emmenés captifs parmi toutes les nations, et Jérusalem sera foulée aux pieds par les nations, jusqu`à ce que les temps des nations soient accomplies. 
\verse Il y aura des signes dans le soleil, dans la lune et dans les étoiles. Et sur la terre, il y aura de l`angoisse chez les nations qui ne sauront que faire, au bruit de la mer et des flots, 
\verse les hommes rendant l`âme de terreur dans l`attente de ce qui surviendra pour la terre; car les puissances des cieux seront ébranlées. 
\verse Alors on verra le Fils de l`homme venant sur une nuée avec puissance et une grande gloire. 
\verse Quand ces choses commenceront à arriver, redressez-vous et levez vos têtes, parce que votre délivrance approche. 
\verse Et il leur dit une comparaison: Voyez le figuier, et tous les arbres. 
\verse Dès qu`ils ont poussé, vous connaissez de vous-mêmes, en regardant, que déjà l`été est proche. 
\verse De même, quand vous verrez ces choses arriver, sachez que le royaume de Dieu est proche. 
\verse Je vous le dis en vérité, cette génération ne passera point, que tout cela n`arrive. 
\verse Le ciel et la terre passeront, mais mes paroles ne passeront point. 
\verse Prenez garde à vous-mêmes, de crainte que vos coeurs ne s`appesantissent par les excès du manger et du boire, et par les soucis de la vie, et que ce jour ne vienne sur vous à l`improviste; 
\verse car il viendra comme un filet sur tous ceux qui habitent sur la face de toute la terre. 
\verse Veillez donc et priez en tout temps, afin que vous ayez la force d`échapper à toutes ces choses qui arriveront, et de paraître debout devant le Fils de l`homme. 
\verse Pendant le jour, Jésus enseignait dans le temple, et il allait passer la nuit à la montagne appelée montagne des Oliviers. 
\verse Et tout le peuple, dès le matin, se rendait vers lui dans le temple pour l`écouter. 

\chapter
\verse La fête des pains sans levain, appelée la Pâque, approchait. 
\verse Les principaux sacrificateurs et les scribes cherchaient les moyens de faire mourir Jésus; car ils craignaient le peuple. 
\verse Or, Satan entra dans Judas, surnommé Iscariot, qui était du nombre des douze. 
\verse Et Judas alla s`entendre avec les principaux sacrificateurs et les chefs des gardes, sur la manière de le leur livrer. 
\verse Ils furent dans la joie, et ils convinrent de lui donner de l`argent. 
\verse Après s`être engagé, il cherchait une occasion favorable pour leur livrer Jésus à l`insu de la foule. 
\verse Le jour des pains sans levain, où l`on devait immoler la Pâque, arriva, 
\verse et Jésus envoya Pierre et Jean, en disant: Allez nous préparer la Pâque, afin que nous la mangions. 
\verse Ils lui dirent: Où veux-tu que nous la préparions? 
\verse Il leur répondit: Voici, quand vous serez entrés dans la ville, vous rencontrerez un homme portant une cruche d`eau; suivez-le dans la maison où il entrera, 
\verse et vous direz au maître de la maison: Le maître te dit: Où est le lieu où je mangerai la Pâque avec mes disciples? 
\verse Et il vous montrera une grande chambre haute, meublée: c`est là que vous préparerez la Pâque. 
\verse Ils partirent, et trouvèrent les choses comme il le leur avait dit; et ils préparèrent la Pâque. 
\verse L`heure étant venue, il se mit à table, et les apôtres avec lui. 
\verse Il leur dit: J`ai désiré vivement manger cette Pâque avec vous, avant de souffrir; 
\verse car, je vous le dis, je ne la mangerai plus, jusqu`à ce qu`elle soit accomplie dans le royaume de Dieu. 
\verse Et, ayant pris une coupe et rendu grâces, il dit: Prenez cette coupe, et distribuez-la entre vous; 
\verse car, je vous le dis, je ne boirai plus désormais du fruit de la vigne, jusqu`à ce que le royaume de Dieu soit venu. 
\verse Ensuite il prit du pain; et, après avoir rendu grâces, il le rompit, et le leur donna, en disant: Ceci est mon corps, qui est donné pour vous; faites ceci en mémoire de moi. 
\verse Il prit de même la coupe, après le souper, et la leur donna, en disant: Cette coupe est la nouvelle alliance en mon sang, qui est répandu pour vous. 
\verse Cependant voici, la main de celui qui me livre est avec moi à cette table. 
\verse Le Fils de l`homme s`en va selon ce qui est déterminé. Mais malheur à l`homme par qui il est livré! 
\verse Et ils commencèrent à se demander les uns aux autres qui était celui d`entre eux qui ferait cela. 
\verse Il s`éleva aussi parmi les apôtres une contestation: lequel d`entre eux devait être estimé le plus grand? 
\verse Jésus leur dit: Les rois des nations les maîtrisent, et ceux qui les dominent sont appelés bienfaiteurs. 
\verse Qu`il n`en soit pas de même pour vous. Mais que le plus grand parmi vous soit comme le plus petit, et celui qui gouverne comme celui qui sert. 
\verse Car quel est le plus grand, celui qui est à table, ou celui qui sert? N`est-ce pas celui qui est à table? Et moi, cependant, je suis au milieu de vous comme celui qui sert. 
\verse Vous, vous êtes ceux qui avez persévéré avec moi dans mes épreuves; 
\verse c`est pourquoi je dispose du royaume en votre faveur, comme mon Père en a disposé en ma faveur, 
\verse afin que vous mangiez et buviez à ma table dans mon royaume, et que vous soyez assis sur des trônes, pour juger les douze tribus d`Israël. 
\verse Le Seigneur dit: Simon, Simon, Satan vous a réclamés, pour vous cribler comme le froment. 
\verse Mais j`ai prié pour toi, afin que ta foi ne défaille point; et toi, quand tu seras converti, affermis tes frères. 
\verse Seigneur, lui dit Pierre, je suis prêt à aller avec toi et en prison et à la mort. 
\verse Et Jésus dit: Pierre, je te le dis, le coq ne chantera pas aujourd`hui que tu n`aies nié trois fois de me connaître. 
\verse Il leur dit encore: Quand je vous ai envoyés sans bourse, sans sac, et sans souliers, avez-vous manqué de quelque chose? Ils répondirent: De rien. 
\verse Et il leur dit: Maintenant, au contraire, que celui qui a une bourse la prenne et que celui qui a un sac le prenne également, que celui qui n`a point d`épée vende son vêtement et achète une épée. 
\verse Car, je vous le dis, il faut que cette parole qui est écrite s`accomplisse en moi: Il a été mis au nombre des malfaiteurs. Et ce qui me concerne est sur le point d`arriver. 
\verse Ils dirent: Seigneur, voici deux épées. Et il leur dit: Cela suffit. 
\verse Après être sorti, il alla, selon sa coutume, à la montagne des Oliviers. Ses disciples le suivirent. 
\verse Lorsqu`il fut arrivé dans ce lieu, il leur dit: Priez, afin que vous ne tombiez pas en tentation. 
\verse Puis il s`éloigna d`eux à la distance d`environ un jet de pierre, et, s`étant mis à genoux, il pria, 
\verse disant: Père, si tu voulais éloigner de moi cette coupe! Toutefois, que ma volonté ne se fasse pas, mais la tienne. 
\verse Alors un ange lui apparut du ciel, pour le fortifier. 
\verse Étant en agonie, il priait plus instamment, et sa sueur devint comme des grumeaux de sang, qui tombaient à terre. 
\verse Après avoir prié, il se leva, et vint vers les disciples, qu`il trouva endormis de tristesse, 
\verse et il leur dit: Pourquoi dormez-vous? Levez-vous et priez, afin que vous ne tombiez pas en tentation. 
\verse Comme il parlait encore, voici, une foule arriva; et celui qui s`appelait Judas, l`un des douze, marchait devant elle. Il s`approcha de Jésus, pour le baiser. 
\verse Et Jésus lui dit: Judas, c`est par un baiser que tu livres le Fils de l`homme! 
\verse Ceux qui étaient avec Jésus, voyant ce qui allait arriver, dirent: Seigneur, frapperons-nous de l`épée? 
\verse Et l`un d`eux frappa le serviteur du souverain sacrificateur, et lui emporta l`oreille droite. 
\verse Mais Jésus, prenant la parole, dit: Laissez, arrêtez! Et, ayant touché l`oreille de cet homme, il le guérit. 
\verse Jésus dit ensuite aux principaux sacrificateurs, aux chefs des gardes du temple, et aux anciens, qui étaient venus contre lui: Vous êtes venus, comme après un brigand, avec des épées et des bâtons. 
\verse J`étais tous les jours avec vous dans le temple, et vous n`avez pas mis la main sur moi. Mais c`est ici votre heure, et la puissance des ténèbres. 
\verse Après avoir saisi Jésus, ils l`emmenèrent, et le conduisirent dans la maison du souverain sacrificateur. Pierre suivait de loin. 
\verse Ils allumèrent du feu au milieu de la cour, et ils s`assirent. Pierre s`assit parmi eux. 
\verse Une servante, qui le vit assis devant le feu, fixa sur lui les regards, et dit: Cet homme était aussi avec lui. 
\verse Mais il le nia disant: Femme, je ne le connais pas. 
\verse Peu après, un autre, l`ayant vu, dit: Tu es aussi de ces gens-là. Et Pierre dit: Homme, je n`en suis pas. 
\verse Environ une heure plus tard, un autre insistait, disant: Certainement cet homme était aussi avec lui, car il est Galiléen. 
\verse Pierre répondit: Homme, je ne sais ce que tu dis. Au même instant, comme il parlait encore, le coq chanta. 
\verse Le Seigneur, s`étant retourné, regarda Pierre. Et Pierre se souvint de la parole que le Seigneur lui avait dite: Avant que le coq chante aujourd`hui, tu me renieras trois fois. 
\verse Et étant sorti, il pleura amèrement. 
\verse Les hommes qui tenaient Jésus se moquaient de lui, et le frappaient. 
\verse Ils lui voilèrent le visage, et ils l`interrogeaient, en disant: Devine qui t`a frappé. 
\verse Et ils proféraient contre lui beaucoup d`autres injures. 
\verse Quand le jour fut venu, le collège des anciens du peuple, les principaux sacrificateurs et les scribes, s`assemblèrent, et firent amener Jésus dans leur sanhédrin. 
\verse Ils dirent: Si tu es le Christ, dis-le nous. Jésus leur répondit: Si je vous le dis, vous ne le croirez pas; 
\verse et, si je vous interroge, vous ne répondrez pas. 
\verse Désormais le Fils de l`homme sera assis à la droite de la puissance de Dieu. 
\verse Tous dirent: Tu es donc le Fils de Dieu? Et il leur répondit: Vous le dites, je le suis. 
\verse Alors ils dirent: Qu`avons-nous encore besoin de témoignage? Nous l`avons entendu nous-mêmes de sa bouche. 

\chapter
\verse Ils se levèrent tous, et ils conduisirent Jésus devant Pilate. 
\verse Ils se mirent à l`accuser, disant: Nous avons trouvé cet homme excitant notre nation à la révolte, empêchant de payer le tribut à César, et se disant lui-même Christ, roi. 
\verse Pilate l`interrogea, en ces termes: Es-tu le roi des Juifs? Jésus lui répondit: Tu le dis. 
\verse Pilate dit aux principaux sacrificateurs et à la foule: Je ne trouve rien de coupable en cet homme. 
\verse Mais ils insistèrent, et dirent: Il soulève le peuple, en enseignant par toute la Judée, depuis la Galilée, où il a commencé, jusqu`ici. 
\verse Quand Pilate entendit parler de la Galilée, il demanda si cet homme était Galiléen; 
\verse et, ayant appris qu`il était de la juridiction d`Hérode, il le renvoya à Hérode, qui se trouvait aussi à Jérusalem en ces jours-là. 
\verse Lorsque Hérode vit Jésus, il en eut une grande joie; car depuis longtemps, il désirait le voir, à cause de ce qu`il avait entendu dire de lui, et il espérait qu`il le verrait faire quelque miracle. 
\verse Il lui adressa beaucoup de questions; mais Jésus ne lui répondit rien. 
\verse Les principaux sacrificateurs et les scribes étaient là, et l`accusaient avec violence. 
\verse Hérode, avec ses gardes, le traita avec mépris; et, après s`être moqué de lui et l`avoir revêtu d`un habit éclatant, il le renvoya à Pilate. 
\verse Ce jour même, Pilate et Hérode devinrent amis, d`ennemis qu`ils étaient auparavant. 
\verse Pilate, ayant assemblé les principaux sacrificateurs, les magistrats, et le peuple, leur dit: 
\verse Vous m`avez amené cet homme comme excitant le peuple à la révolte. Et voici, je l`ai interrogé devant vous, et je ne l`ai trouvé coupable d`aucune des choses dont vous l`accusez; 
\verse Hérode non plus, car il nous l`a renvoyé, et voici, cet homme n`a rien fait qui soit digne de mort. 
\verse Je le relâcherai donc, après l`avoir fait battre de verges. 
\verse A chaque fête, il était obligé de leur relâcher un prisonnier. 
\verse Ils s`écrièrent tous ensemble: Fais mourir celui-ci, et relâche-nous Barabbas. 
\verse Cet homme avait été mis en prison pour une sédition qui avait eu lieu dans la ville, et pour un meurtre. 
\verse Pilate leur parla de nouveau, dans l`intention de relâcher Jésus. 
\verse Et ils crièrent: Crucifie, crucifie-le! 
\verse Pilate leur dit pour la troisième fois: Quel mal a-t-il fait? Je n`ai rien trouvé en lui qui mérite la mort. Je le relâcherai donc, après l`avoir fait battre de verges. 
\verse Mais ils insistèrent à grands cris, demandant qu`il fût crucifié. Et leurs cris l`emportèrent: 
\verse Pilate prononça que ce qu`ils demandaient serait fait. 
\verse Il relâcha celui qui avait été mis en prison pour sédition et pour meurtre, et qu`ils réclamaient; et il livra Jésus à leur volonté. 
\verse Comme ils l`emmenaient, ils prirent un certain Simon de Cyrène, qui revenait des champs, et ils le chargèrent de la croix, pour qu`il la porte derrière Jésus. 
\verse Il était suivi d`une grande multitude des gens du peuple, et de femmes qui se frappaient la poitrine et se lamentaient sur lui. 
\verse Jésus se tourna vers elles, et dit: Filles de Jérusalem, ne pleurez pas sur moi; mais pleurez sur vous et sur vos enfants. 
\verse Car voici, des jours viendront où l`on dira: Heureuses les stériles, heureuses les entrailles qui n`ont point enfanté, et les mamelles qui n`ont point allaité! 
\verse Alors ils se mettront à dire aux montagnes: Tombez sur nous! Et aux collines: Couvrez-nous! 
\verse Car, si l`on fait ces choses au bois vert, qu`arrivera-t-il au bois sec? 
\verse On conduisait en même temps deux malfaiteurs, qui devaient être mis à mort avec Jésus. 
\verse Lorsqu`ils furent arrivés au lieu appelé Crâne, ils le crucifièrent là, ainsi que les deux malfaiteurs, l`un à droite, l`autre à gauche. 
\verse Jésus dit: Père, pardonne-leur, car ils ne savent ce qu`ils font. Ils se partagèrent ses vêtements, en tirant au sort. 
\verse Le peuple se tenait là, et regardait. Les magistrats se moquaient de Jésus, disant: Il a sauvé les autres; qu`il se sauve lui-même, s`il est le Christ, l`élu de Dieu! 
\verse Les soldats aussi se moquaient de lui; s`approchant et lui présentant du vinaigre, 
\verse ils disaient: Si tu es le roi des Juifs, sauve-toi toi-même! 
\verse Il y avait au-dessus de lui cette inscription: Celui-ci est le roi des Juifs. 
\verse L`un des malfaiteurs crucifiés l`injuriait, disant: N`es-tu pas le Christ? Sauve-toi toi-même, et sauve-nous! 
\verse Mais l`autre le reprenait, et disait: Ne crains-tu pas Dieu, toi qui subis la même condamnation? 
\verse Pour nous, c`est justice, car nous recevons ce qu`ont mérité nos crimes; mais celui-ci n`a rien fait de mal. 
\verse Et il dit à Jésus: Souviens-toi de moi, quand tu viendras dans ton règne. 
\verse Jésus lui répondit: Je te le dis en vérité, aujourd`hui tu seras avec moi dans le paradis. 
\verse Il était déjà environ la sixième heure, et il y eut des ténèbres sur toute la terre, jusqu`à la neuvième heure. 
\verse Le soleil s`obscurcit, et le voile du temple se déchira par le milieu. 
\verse Jésus s`écria d`une voix forte: Père, je remets mon esprit entre tes mains. Et, en disant ces paroles, il expira. 
\verse Le centenier, voyant ce qui était arrivé, glorifia Dieu, et dit: Certainement, cet homme était juste. 
\verse Et tous ceux qui assistaient en foule à ce spectacle, après avoir vu ce qui était arrivé, s`en retournèrent, se frappant la poitrine. 
\verse Tous ceux de la connaissance de Jésus, et les femmes qui l`avaient accompagné depuis la Galilée, se tenaient dans l`éloignement et regardaient ce qui se passait. 
\verse Il y avait un conseiller, nommé Joseph, homme bon et juste, 
\verse qui n`avait point participé à la décision et aux actes des autres; il était d`Arimathée, ville des Juifs, et il attendait le royaume de Dieu. 
\verse Cet homme se rendit vers Pilate, et demanda le corps de Jésus. 
\verse Il le descendit de la croix, l`enveloppa d`un linceul, et le déposa dans un sépulcre taillé dans le roc, où personne n`avait encore été mis. 
\verse C`était le jour de la préparation, et le sabbat allait commencer. 
\verse Les femmes qui étaient venues de la Galilée avec Jésus accompagnèrent Joseph, virent le sépulcre et la manière dont le corps de Jésus y fut déposé, 
\verse et, s`en étant retournées, elles préparèrent des aromates et des parfums. Puis elles se reposèrent le jour du sabbat, selon la loi. 

\chapter
\verse Le premier jour de la semaine, elles se rendirent au sépulcre de grand matin, portant les aromates qu`elles avaient préparés. 
\verse Elles trouvèrent que la pierre avait été roulée de devant le sépulcre; 
\verse et, étant entrées, elles ne trouvèrent pas le corps du Seigneur Jésus. 
\verse Comme elles ne savaient que penser de cela, voici, deux hommes leur apparurent, en habits resplendissants. 
\verse Saisies de frayeur, elles baissèrent le visage contre terre; mais ils leur dirent: Pourquoi cherchez-vous parmi les morts celui qui est vivant? 
\verse Il n`est point ici, mais il est ressuscité. Souvenez-vous de quelle manière il vous a parlé, lorsqu`il était encore en Galilée, 
\verse et qu`il disait: Il faut que le Fils de l`homme soit livré entre les mains des pécheurs, qu`il soit crucifié, et qu`il ressuscite le troisième jour. 
\verse Et elles se ressouvinrent des paroles de Jésus. 
\verse A leur retour du sépulcre, elles annoncèrent toutes ces choses aux onze, et à tous les autres. 
\verse Celles qui dirent ces choses aux apôtres étaient Marie de Magdala, Jeanne, Marie, mère de Jacques, et les autres qui étaient avec elles. 
\verse Ils tinrent ces discours pour des rêveries, et ils ne crurent pas ces femmes. 
\verse Mais Pierre se leva, et courut au sépulcre. S`étant baissé, il ne vit que les linges qui étaient à terre; puis il s`en alla chez lui, dans l`étonnement de ce qui était arrivé. 
\verse Et voici, ce même jour, deux disciples allaient à un village nommé Emmaüs, éloigné de Jérusalem de soixante stades; 
\verse et ils s`entretenaient de tout ce qui s`était passé. 
\verse Pendant qu`ils parlaient et discutaient, Jésus s`approcha, et fit route avec eux. 
\verse Mais leurs yeux étaient empêchés de le reconnaître. 
\verse Il leur dit: De quoi vous entretenez-vous en marchant, pour que vous soyez tout tristes? 
\verse L`un d`eux, nommé Cléopas, lui répondit: Es-tu le seul qui, séjournant à Jérusalem ne sache pas ce qui y est arrivé ces jours-ci? - 
\verse Quoi? leur dit-il. -Et ils lui répondirent: Ce qui est arrivé au sujet de Jésus de Nazareth, qui était un prophète puissant en oeuvres et en paroles devant Dieu et devant tout le peuple, 
\verse et comment les principaux sacrificateurs et nos magistrats l`on livré pour le faire condamner à mort et l`ont crucifié. 
\verse Nous espérions que ce serait lui qui délivrerait Israël; mais avec tout cela, voici le troisième jour que ces choses se sont passées. 
\verse Il est vrai que quelques femmes d`entre nous nous ont fort étonnés; s`étant rendues de grand matin au sépulcre 
\verse et n`ayant pas trouvé son corps, elles sont venues dire que des anges leurs sont apparus et ont annoncé qu`il est vivant. 
\verse Quelques-uns de ceux qui étaient avec nous sont allés au sépulcre, et ils ont trouvé les choses comme les femmes l`avaient dit; mais lui, ils ne l`ont point vu. 
\verse Alors Jésus leur dit: O hommes sans intelligence, et dont le coeur est lent à croire tout ce qu`ont dit les prophètes! 
\verse Ne fallait-il pas que le Christ souffrît ces choses, et qu`il entrât dans sa gloire? 
\verse Et, commençant par Moïse et par tous les prophètes, il leur expliqua dans toutes les Écritures ce qui le concernait. 
\verse Lorsqu`ils furent près du village où ils allaient, il parut vouloir aller plus loin. 
\verse Mais ils le pressèrent, en disant: Reste avec nous, car le soir approche, le jour est sur son déclin. Et il entra, pour rester avec eux. 
\verse Pendant qu`il était à table avec eux, il prit le pain; et, après avoir rendu grâces, il le rompit, et le leur donna. 
\verse Alors leurs yeux s`ouvrirent, et ils le reconnurent; mais il disparut de devant eux. 
\verse Et ils se dirent l`un à l`autre: Notre coeur ne brûlait-il pas au dedans de nous, lorsqu`il nous parlait en chemin et nous expliquait les Écritures? 
\verse Se levant à l`heure même, ils retournèrent à Jérusalem, et ils trouvèrent les onze, et ceux qui étaient avec eux, assemblés 
\verse et disant: Le Seigneur est réellement ressuscité, et il est apparu à Simon. 
\verse Et ils racontèrent ce qui leur était arrivé en chemin, et comment ils l`avaient reconnu au moment où il rompit le pain. 
\verse Tandis qu`ils parlaient de la sorte, lui-même se présenta au milieu d`eux, et leur dit: La paix soit avec vous! 
\verse Saisis de frayeur et d`épouvante, ils croyaient voir un esprit. 
\verse Mais il leur dit: Pourquoi êtes-vous troublés, et pourquoi pareilles pensées s`élèvent-elles dans vos coeurs? 
\verse Voyez mes mains et mes pieds, c`est bien moi; touchez-moi et voyez: un esprit n`a ni chair ni os, comme vous voyez que j`ai. 
\verse Et en disant cela, il leur montra ses mains et ses pieds. 
\verse Comme, dans leur joie, ils ne croyaient point encore, et qu`ils étaient dans l`étonnement, il leur dit: Avez-vous ici quelque chose à manger? 
\verse Ils lui présentèrent du poisson rôti et un rayon de miel. 
\verse Il en prit, et il mangea devant eux. 
\verse Puis il leur dit: C`est là ce que je vous disais lorsque j`étais encore avec vous, qu`il fallait que s`accomplît tout ce qui est écrit de moi dans la loi de Moïse, dans les prophètes, et dans les psaumes. 
\verse Alors il leur ouvrit l`esprit, afin qu`ils comprissent les Écritures. 
\verse Et il leur dit: Ainsi il est écrit que le Christ souffrirait, et qu`il ressusciterait des morts le troisième jour, 
\verse et que la repentance et le pardon des péchés seraient prêchés en son nom à toutes les nations, à commencer par Jérusalem. 
\verse Vous êtes témoins de ces choses. 
\verse Et voici, j`enverrai sur vous ce que mon Père a promis; mais vous, restez dans la ville jusqu`à ce que vous soyez revêtus de la puissance d`en haut. 
\verse Il les conduisit jusque vers Béthanie, et, ayant levé les mains, il les bénit. 
\verse Pendant qu`il les bénissait, il se sépara d`eux, et fut enlevé au ciel. 
\verse Pour eux, après l`avoir adoré, ils retournèrent à Jérusalem avec une grande joie; 
\verse et ils étaient continuellement dans le temple, louant et bénissant Dieu. 
