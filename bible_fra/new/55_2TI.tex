\book[Deuxième épître à Timothée]{2 Timothée}


\chapter
\verse Paul, apôtre de Jésus Christ, par la volonté de Dieu, pour annoncer la promesse de la vie qui est en Jésus Christ, 
\verse à Timothée, mon enfant bien-aimé: que la grâce, la miséricorde et la paix te soient données de la part de Dieu le Père et de Jésus Christ notre Seigneur! 
\verse Je rends grâces à Dieu, que mes ancêtres ont servi, et que je sers avec une conscience pure, de ce que nuit et jour je me souviens continuellement de toi dans mes prières, 
\verse me rappelant tes larmes, et désirant te voir afin d`être rempli de joie, 
\verse gardant le souvenir de la foi sincère qui est en toi, qui habita d`abord dans ton aïeule Loïs et dans ta mère Eunice, et qui, j`en suis persuadé, habite aussi en toi. 
\verse C`est pourquoi je t`exhorte à ranimer le don de Dieu que tu as reçu par l`imposition de mes mains. 
\verse Car ce n`est pas un esprit de timidité que Dieu nous a donné, mais un esprit de force, d`amour et de sagesse. 
\verse N`aie donc point honte du témoignage à rendre à notre Seigneur, ni de moi son prisonnier. Mais souffre avec moi pour l`Évangile, 
\verse par la puissance de Dieu qui nous a sauvés, et nous a adressé une sainte vocation, non à cause de nos oeuvres, mais selon son propre dessein, et selon la grâce qui nous a été donnée en Jésus Christ avant les temps éternels, 
\verse et qui a été manifestée maintenant par l`apparition de notre Sauveur Jésus Christ, qui a détruit la mort et a mis en évidence la vie et l`immortalité par l`Évangile. 
\verse C`est pour cet Évangile que j`ai été établi prédicateur et apôtre, chargé d`instruire les païens. 
\verse Et c`est à cause de cela que je souffre ces choses; mais j`en ai point honte, car je sais en qui j`ai cru, et je suis persuadé qu`il a la puissance de garder mon dépôt jusqu`à ce jour-là. 
\verse Retiens dans la foi et dans la charité qui est en Jésus Christ le modèle des saines paroles que tu as reçues de moi. 
\verse Garde le bon dépôt, par le Saint Esprit qui habite en nous. 
\verse Tu sais que tous ceux qui sont en Asie m`ont abandonné, entre autres Phygelle et Hermogène. 
\verse Que le Seigneur répande sa miséricorde sur la maison d`Onésiphore, car il m`a souvent consolé, et il n`a pas eu honte de mes chaînes; 
\verse au contraire, lorsqu`il est venu à Rome, il m`a cherché avec beaucoup d`empressement, et il m`a trouvé. 
\verse Que le Seigneur lui donne d`obtenir miséricorde auprès du Seigneur en ce jour-là. Tu sais mieux que personne combien de services il m`a rendus à Éphèse. 

\chapter
\verse Toi donc, mon enfant, fortifie-toi dans la grâce qui est en Jésus Christ. 
\verse Et ce que tu as entendu de moi en présence de beaucoup de témoins, confie-le à des hommes fidèles, qui soient capables de l`enseigner aussi à d`autres. 
\verse Souffre avec moi, comme un bon soldat de Jésus Christ. 
\verse Il n`est pas de soldat qui s`embarrasse des affaires de la vie, s`il veut plaire à celui qui l`a enrôlé; 
\verse et l`athlète n`est pas couronné, s`il n`a combattu suivant les règles. 
\verse Il faut que le laboureur travaille avant de recueillir les fruits. 
\verse Comprends ce que je dis, car le Seigneur te donnera de l`intelligence en toutes choses. 
\verse Souviens-toi de Jésus Christ, issu de la postérité de David, ressuscité des morts, selon mon Évangile, 
\verse pour lequel je souffre jusqu`à être lié comme un malfaiteur. Mais la parole de Dieu n`est pas liée. 
\verse C`est pourquoi je supporte tout à cause des élus, afin qu`eux aussi obtiennent le salut qui est en Jésus Christ, avec la gloire éternelle. 
\verse Cette parole est certaine: Si nous sommes morts avec lui, nous vivrons aussi avec lui; 
\verse si nous persévérons, nous régnerons aussi avec lui; si nous le renions, lui aussi nous reniera; 
\verse si nous sommes infidèles, il demeure fidèle, car il ne peut se renier lui-même. 
\verse Rappelle ces choses, en conjurant devant Dieu qu`on évite les disputes de mots, qui ne servent qu`à la ruine de ceux qui écoutent. 
\verse Efforce-toi de te présenter devant Dieu comme un homme éprouvé, un ouvrier qui n`a point à rougir, qui dispense droitement la parole de la vérité. 
\verse Évite les discours vains et profanes; car ceux qui les tiennent avanceront toujours plus dans l`impiété, et leur parole rongera comme la gangrène. 
\verse De ce nombre sont Hyménée et Philète, 
\verse qui se sont détournés de la vérité, disant que la résurrection est déjà arrivée, et qui renversent le foi de quelques uns. 
\verse Néanmoins, le solide fondement de Dieu reste debout, avec ces paroles qui lui servent de sceau: Le Seigneur connaît ceux qui lui appartiennent; et: Quiconque prononce le nom du Seigneur, qu`il s`éloigne de l`iniquité. 
\verse Dans une grande maison, il n`y a pas seulement des vases d`or et d`argent, mais il y en a aussi de bois et de terre; les uns sont des vases d`honneur, et les autres sont d`un usage vil. 
\verse Si donc quelqu`un se conserve pur, en s`abstenant de ces choses, il sera un vase d`honneur, sanctifié, utile à son maître, propre à toute bonne oeuvre. 
\verse Fuis les passions de la jeunesse, et recherche la justice, la foi, la charité, la paix, avec ceux qui invoquent le Seigneur d`un coeur pur. 
\verse Repousse les discussions folles et inutiles, sachant qu`elles font naître des querelles. 
\verse Or, il ne faut pas qu`un serviteur du Seigneur ait des querelles; il doit, au contraire, avoir de la condescendance pour tous, être propre à enseigner, doué de patience; 
\verse il doit redresser avec douceur les adversaires, dans l`espérance que Dieu leur donnera la repentance pour arriver à la connaissance de la vérité, 
\verse et que, revenus à leur bon sens, ils se dégageront des pièges du diable, qui s`est emparé d`eux pour les soumettre à sa volonté. 

\chapter
\verse Sache que, dans les derniers jours, il y aura des temps difficiles. 
\verse Car les hommes seront égoïstes, amis de l`argent, fanfarons, hautains, blasphémateurs, rebelles à leurs parents, ingrats, irréligieux, 
\verse insensibles, déloyaux, calomniateurs, intempérants, cruels, ennemis des gens de bien, 
\verse traîtres, emportés, enflés d`orgueil, aimant le plaisir plus que Dieu, 
\verse ayant l`apparence de la piété, mais reniant ce qui en fait la force. Éloigne-toi de ces hommes-là. 
\verse Il en est parmi eux qui s`introduisent dans les maisons, et qui captivent des femmes d`un esprit faible et borné, chargées de péchés, agitées par des passions de toute espèce, 
\verse apprenant toujours et ne pouvant jamais arriver à la connaissance de la vérité. 
\verse De même que Jannès et Jambrès s`opposèrent à Moïse, de même ces hommes s`opposent à la vérité, étant corrompus d`entendement, réprouvés en ce qui concerne la foi. 
\verse Mais ils ne feront pas de plus grands progrès; car leur folie sera manifeste pour tous, comme le fut celle de ces deux hommes. 
\verse Pour toi, tu as suivi de près mon enseignement, ma conduite, mes résolutions, ma foi, ma douceur, ma charité, ma constance, 
\verse mes persécutions, mes souffrances. A quelles souffrances n`ai-je pas été exposé à Antioche, à Icone, à Lystre? Quelles persécutions n`ai-je pas supportées? Et le Seigneur m`a délivré de toutes. 
\verse Or, tous ceux qui veulent vivre pieusement en Jésus Christ seront persécutés. 
\verse Mais les homme méchants et imposteurs avanceront toujours plus dans le mal, égarants les autres et égarés eux-mêmes. 
\verse Toi, demeure dans les choses que tu as apprises, et reconnues certaines, sachant de qui tu les as apprises; 
\verse dès ton enfance, tu connais les saintes lettres, qui peuvent te rendre sage à salut par la foi en Jésus Christ. 
\verse Toute Écriture est inspirée de Dieu, et utile pour enseigner, pour convaincre, pour corriger, pour instruire dans la justice, 
\verse afin que l`homme de Dieu soit accompli et propre à toute bonne oeuvre. 

\chapter
\verse Je t`en conjure devant Dieu et devant Jésus Christ, qui doit juger les vivants et les morts, et au nom de son apparition et de son royaume, 
\verse prêche la parole, insiste en toute occasion, favorable ou non, reprends, censure, exhorte, avec toute douceur et en instruisant. 
\verse Car il viendra un temps où les hommes ne supporteront pas la saine doctrine; mais, ayant la démangeaison d`entendre des choses agréables, ils se donneront une foule de docteurs selon leurs propres désires, 
\verse détourneront l`oreille de la vérité, et se tourneront vers les fables. 
\verse Mais toi, sois sobre en toutes choses, supporte les souffrances, fais l`oeuvre d`un évangéliste, remplis bien ton ministère. 
\verse Car pour moi, je sers déjà de libation, et le moment de mon départ approche. 
\verse J`ai combattu le bon combat, j`ai achevé la course, j`ai gardé la foi. 
\verse Désormais la couronne de justice m`est réservée; le Seigneur, le juste juge, me le donnera dans ce jour-là, et non seulement à moi, mais encore à tous ceux qui auront aimé son avènement. 
\verse Viens au plus tôt vers moi; 
\verse car Démas m`a abandonné, par amour pour le siècle présent, et il est parti pour Thessalonique; Crescens est allé en Galatie, Tite en Dalmatie. 
\verse Luc seul est avec moi. Prends Marc, et amène-le avec toi, car il m`est utile pour le ministère. 
\verse J`ai envoyé Tychique à Éphèse. 
\verse Quand tu viendras, apporte le manteau que j`ai laissé à Troas chez Carpus, et les livres, surtout les parchemins. 
\verse Alexandre, le forgeron, m`a fait beaucoup de mal. Le Seigneur lui rendra selon ses oeuvres. 
\verse Garde-toi aussi de lui, car il s`est fortement opposé à nos paroles. 
\verse Dans ma première défense, personne ne m`a assisté, mais tous m`ont abandonné. Que cela ne leur soit point imputé! 
\verse C`est le Seigneur qui m`a assisté et qui m`a fortifié, afin que la prédication fût accomplie par moi et que tous les païens l`entendissent. Et j`ai été délivré de la gueule du lion. 
\verse Le Seigneur me délivrera de toute oeuvre mauvaise, et il me sauvera pour me faire entrer dans son royaume céleste. A lui soit la gloire aux siècles des siècles! Amen! 
\verse Salue Prisca et Aquilas, et la famille d`Onésiphore. 
\verse Éraste est resté à Corinthe, et j`ai laissé Trophime malade à Milet. 
\verse Tâche de venir avant l`hiver. Eubulus, Pudens, Linus, Claudia, et tous les frères te saluent. 
\verse Que le Seigneur soit avec ton esprit! Que la grâce soit avec vous! 
