\book[Évangile selon Matthieu]{Matthieu}


\chapter[Évangile selon Matthieu]

\chaptermark{Évangile selon Matthieu}{}
\verse Généalogie de Jésus Christ, fils de David, fils d`Abraham. 
\verse Abraham engendra Isaac; Isaac engendra Jacob; Jacob engendra Juda et ses frères; 
\verse Juda engendra de Thamar Pharès et Zara; Pharès engendra Esrom; Esrom engendra Aram; 
\verse Aram engendra Aminadab; Aminadab engendra Naasson; Naasson engendra Salmon; 
\verse Salmon engendra Boaz de Rahab; Boaz engendra Obed de Ruth; 
\verse Obed engendra Isaï; Isaï engendra David. Le roi David engendra Salomon de la femme d`Urie; 
\verse Salomon engendra Roboam; Roboam engendra Abia; Abia engendra Asa; 
\verse Asa engendra Josaphat; Josaphat engendra Joram; Joram engendra Ozias; 
\verse Ozias engendra Joatham; Joatham engendra Achaz; Achaz engendra Ézéchias; 
\verse Ézéchias engendra Manassé; Manassé engendra Amon; Amon engendra Josias; 
\verse Josias engendra Jéchonias et ses frères, au temps de la déportation à Babylone. 
\verse Après la déportation à Babylone, Jéchonias engendra Salathiel; Salathiel engendra Zorobabel; 
\verse Zorobabel engendra Abiud; Abiud engendra Éliakim; Éliakim engendra Azor; 
\verse Azor engendra Sadok; Sadok engendra Achim; Achim engendra Éliud; 
\verse Éliud engendra Éléazar; Éléazar engendra Matthan; Matthan engendra Jacob; 
\verse Jacob engendra Joseph, l`époux de Marie, de laquelle est né Jésus, qui est appelé Christ. 
\verse Il y a donc en tout quatorze générations depuis Abraham jusqu`à David, quatorze générations depuis David jusqu`à la déportation à Babylone, et quatorze générations depuis la déportation à Babylone jusqu`au Christ. 
\verse Voici de quelle manière arriva la naissance de Jésus Christ. Marie, sa mère, ayant été fiancée à Joseph, se trouva enceinte, par la vertu du Saint Esprit, avant qu`ils eussent habité ensemble. 
\verse Joseph, son époux, qui était un homme de bien et qui ne voulait pas la diffamer, se proposa de rompre secrètement avec elle. 
\verse Comme il y pensait, voici, un ange du Seigneur lui apparut en songe, et dit: Joseph, fils de David, ne crains pas de prendre avec toi Marie, ta femme, car l`enfant qu`elle a conçu vient du Saint Esprit; 
\verse elle enfantera un fils, et tu lui donneras le nom de Jésus; c`est lui qui sauvera son peuple de ses péchés. 
\verse Tout cela arriva afin que s`accomplît ce que le Seigneur avait annoncé par le prophète: 
\verse Voici, la vierge sera enceinte, elle enfantera un fils, et on lui donnera le nom d`Emmanuel, ce qui signifie Dieu avec nous. 
\verse Joseph s`étant réveillé fit ce que l`ange du Seigneur lui avait ordonné, et il prit sa femme avec lui. 
\verse Mais il ne la connut point jusqu`à ce qu`elle eût enfanté un fils, auquel il donna le nom de Jésus. 

\chapter[Évangile selon Matthieu]

\chaptermark{Évangile selon Matthieu}{}
\verse Jésus étant né à Bethléhem en Judée, au temps du roi Hérode, voici des mages d`Orient arrivèrent à Jérusalem, 
\verse et dirent: Où est le roi des Juifs qui vient de naître? car nous avons vu son étoile en Orient, et nous sommes venus pour l`adorer. 
\verse Le roi Hérode, ayant appris cela, fut troublé, et tout Jérusalem avec lui. 
\verse Il assembla tous les principaux sacrificateurs et les scribes du peuple, et il s`informa auprès d`eux où devait naître le Christ. 
\verse Ils lui dirent: A Bethléhem en Judée; car voici ce qui a été écrit par le prophète: 
\verse Et toi, Bethléhem, terre de Juda, Tu n`es certes pas la moindre entre les principales villes de Juda, Car de toi sortira un chef Qui paîtra Israël, mon peuple. 
\verse Alors Hérode fit appeler en secret les mages, et s`enquit soigneusement auprès d`eux depuis combien de temps l`étoile brillait. 
\verse Puis il les envoya à Bethléhem, en disant: Allez, et prenez des informations exactes sur le petit enfant; quand vous l`aurez trouvé, faites-le-moi savoir, afin que j`aille aussi moi-même l`adorer. 
\verse Après avoir entendu le roi, ils partirent. Et voici, l`étoile qu`ils avaient vue en Orient marchait devant eux jusqu`à ce qu`étant arrivée au-dessus du lieu où était le petit enfant, elle s`arrêta. 
\verse Quand ils aperçurent l`étoile, ils furent saisis d`une très grande joie. 
\verse Ils entrèrent dans la maison, virent le petit enfant avec Marie, sa mère, se prosternèrent et l`adorèrent; ils ouvrirent ensuite leurs trésors, et lui offrirent en présent de l`or, de l`encens et de la myrrhe. 
\verse Puis, divinement avertis en songe de ne pas retourner vers Hérode, ils regagnèrent leur pays par un autre chemin. 
\verse Lorsqu`ils furent partis, voici, un ange du Seigneur apparut en songe à Joseph, et dit: Lève-toi, prends le petit enfant et sa mère, fuis en Égypte, et restes-y jusqu`à ce que je te parle; car Hérode cherchera le petit enfant pour le faire périr. 
\verse Joseph se leva, prit de nuit le petit enfant et sa mère, et se retira en Égypte. 
\verse Il y resta jusqu`à la mort d`Hérode, afin que s`accomplît ce que le Seigneur avait annoncé par le prophète: J`ai appelé mon fils hors d`Égypte. 
\verse Alors Hérode, voyant qu`il avait été joué par les mages, se mit dans une grande colère, et il envoya tuer tous les enfants de deux ans et au-dessous qui étaient à Bethléhem et dans tout son territoire, selon la date dont il s`était soigneusement enquis auprès des mages. 
\verse Alors s`accomplit ce qui avait été annoncé par Jérémie, le prophète: 
\verse On a entendu des cris à Rama, Des pleurs et de grandes lamentations: Rachel pleure ses enfants, Et n`a pas voulu être consolée, Parce qu`ils ne sont plus. 
\verse Quand Hérode fut mort, voici, un ange du Seigneur apparut en songe à Joseph, en Égypte, 
\verse et dit: Lève-toi, prends le petit enfant et sa mère, et va dans le pays d`Israël, car ceux qui en voulaient à la vie du petit enfant sont morts. 
\verse Joseph se leva, prit le petit enfant et sa mère, et alla dans le pays d`Israël. 
\verse Mais, ayant appris qu`Archélaüs régnait sur la Judée à la place d`Hérode, son père, il craignit de s`y rendre; et, divinement averti en songe, il se retira dans le territoire de la Galilée, 
\verse et vint demeurer dans une ville appelée Nazareth, afin que s`accomplît ce qui avait été annoncé par les prophètes: Il sera appelé Nazaréen. 

\chapter[Évangile selon Matthieu]

\chaptermark{Évangile selon Matthieu}{}
\verse En ce temps-là parut Jean Baptiste, prêchant dans le désert de Judée. 
\verse Il disait: Repentez-vous, car le royaume des cieux est proche. 
\verse Jean est celui qui avait été annoncé par Ésaïe, le prophète, lorsqu`il dit: C`est ici la voix de celui qui crie dans le désert: Préparez le chemin du Seigneur, Aplanissez ses sentiers. 
\verse Jean avait un vêtement de poils de chameau, et une ceinture de cuir autour des reins. Il se nourrissait de sauterelles et de miel sauvage. 
\verse Les habitants de Jérusalem, de toute la Judée et de tout le pays des environs du Jourdain, se rendaient auprès de lui; 
\verse et, confessant leurs péchés, ils se faisaient baptiser par lui dans le fleuve du Jourdain. 
\verse Mais, voyant venir à son baptême beaucoup de pharisiens et de sadducéens, il leur dit: Races de vipères, qui vous a appris à fuir la colère à venir? 
\verse Produisez donc du fruit digne de la repentance, 
\verse et ne prétendez pas dire en vous-mêmes: Nous avons Abraham pour père! Car je vous déclare que de ces pierres-ci Dieu peut susciter des enfants à Abraham. 
\verse Déjà la cognée est mise à la racine des arbres: tout arbre donc qui ne produit pas de bons fruits sera coupé et jeté au feu. 
\verse Moi, je vous baptise d`eau, pour vous amener à la repentance; mais celui qui vient après moi est plus puissant que moi, et je ne suis pas digne de porter ses souliers. Lui, il vous baptisera du Saint Esprit et de feu. 
\verse Il a son van à la main; il nettoiera son aire, et il amassera son blé dans le grenier, mais il brûlera la paille dans un feu qui ne s`éteint point. 
\verse Alors Jésus vint de la Galilée au Jourdain vers Jean, pour être baptisé par lui. 
\verse Mais Jean s`y opposait, en disant: C`est moi qui ai besoin d`être baptisé par toi, et tu viens à moi! 
\verse Jésus lui répondit: Laisse faire maintenant, car il est convenable que nous accomplissions ainsi tout ce qui est juste. Et Jean ne lui résista plus. 
\verse Dès que Jésus eut été baptisé, il sortit de l`eau. Et voici, les cieux s`ouvrirent, et il vit l`Esprit de Dieu descendre comme une colombe et venir sur lui. 
\verse Et voici, une voix fit entendre des cieux ces paroles: Celui-ci est mon Fils bien-aimé, en qui j`ai mis toute mon affection. 

\chapter[Évangile selon Matthieu]

\chaptermark{Évangile selon Matthieu}{}
\verse Alors Jésus fut emmené par l`Esprit dans le désert, pour être tenté par le diable. 
\verse Après avoir jeûné quarante jours et quarante nuits, il eut faim. 
\verse Le tentateur, s`étant approché, lui dit: Si tu es Fils de Dieu, ordonne que ces pierres deviennent des pains. 
\verse Jésus répondit: Il est écrit: L`homme ne vivra pas de pain seulement, mais de toute parole qui sort de la bouche de Dieu. 
\verse Le diable le transporta dans la ville sainte, le plaça sur le haut du temple, 
\verse et lui dit: Si tu es Fils de Dieu, jette-toi en bas; car il est écrit: Il donnera des ordres à ses anges à ton sujet; Et ils te porteront sur les mains, De peur que ton pied ne heurte contre une pierre. 
\verse Jésus lui dit: Il est aussi écrit: Tu ne tenteras point le Seigneur, ton Dieu. 
\verse Le diable le transporta encore sur une montagne très élevée, lui montra tous les royaumes du monde et leur gloire, 
\verse et lui dit: Je te donnerai toutes ces choses, si tu te prosternes et m`adores. 
\verse Jésus lui dit: Retire-toi, Satan! Car il est écrit: Tu adoreras le Seigneur, ton Dieu, et tu le serviras lui seul. 
\verse Alors le diable le laissa. Et voici, des anges vinrent auprès de Jésus, et le servaient. 
\verse Jésus, ayant appris que Jean avait été livré, se retira dans la Galilée. 
\verse Il quitta Nazareth, et vint demeurer à Capernaüm, située près de la mer, dans le territoire de Zabulon et de Nephthali, 
\verse afin que s`accomplît ce qui avait été annoncé par Ésaïe, le prophète: 
\verse Le peuple de Zabulon et de Nephthali, De la contrée voisine de la mer, du pays au delà du Jourdain, Et de la Galilée des Gentils, 
\verse Ce peuple, assis dans les ténèbres, A vu une grande lumière; Et sur ceux qui étaient assis dans la région et l`ombre de la mort La lumière s`est levée. 
\verse Dès ce moment Jésus commença à prêcher, et à dire: Repentez-vous, car le royaume des cieux est proche. 
\verse Comme il marchait le long de la mer de Galilée, il vit deux frères, Simon, appelé Pierre, et André, son frère, qui jetaient un filet dans la mer; car ils étaient pêcheurs. 
\verse Il leur dit: Suivez-moi, et je vous ferai pêcheurs d`hommes. 
\verse Aussitôt, ils laissèrent les filets, et le suivirent. 
\verse De là étant allé plus loin, il vit deux autres frères, Jacques, fils de Zébédée, et Jean, son frère, qui étaient dans une barque avec Zébédée, leur père, et qui réparaient leurs filets. 
\verse Il les appela, et aussitôt ils laissèrent la barque et leur père, et le suivirent. 
\verse Jésus parcourait toute la Galilée, enseignant dans les synagogues, prêchant la bonne nouvelle du royaume, et guérissant toute maladie et toute infirmité parmi le peuple. 
\verse Sa renommée se répandit dans toute la Syrie, et on lui amenait tous ceux qui souffraient de maladies et de douleurs de divers genres, des démoniaques, des lunatiques, des paralytiques; et il les guérissait. 
\verse Une grande foule le suivit, de la Galilée, de la Décapole, de Jérusalem, de la Judée, et d`au delà du Jourdain. 

\chapter[Évangile selon Matthieu]

\chaptermark{Évangile selon Matthieu}{}
\verse Voyant la foule, Jésus monta sur la montagne; et, après qu`il se fut assis, ses disciples s`approchèrent de lui. 
\verse Puis, ayant ouvert la bouche, il les enseigna, et dit: 
\verse Heureux les pauvres en esprit, car le royaume des cieux est à eux! 
\verse Heureux les affligés, car ils seront consolés! 
\verse Heureux les débonnaires, car ils hériteront la terre! 
\verse Heureux ceux qui ont faim et soif de la justice, car ils seront rassasiés! 
\verse Heureux les miséricordieux, car ils obtiendront miséricorde! 
\verse Heureux ceux qui ont le coeur pur, car ils verront Dieu! 
\verse Heureux ceux qui procurent la paix, car ils seront appelés fils de Dieu! 
\verse Heureux ceux qui sont persécutés pour la justice, car le royaume des cieux est à eux! 
\verse Heureux serez-vous, lorsqu`on vous outragera, qu`on vous persécutera et qu`on dira faussement de vous toute sorte de mal, à cause de moi. 
\verse Réjouissez-vous et soyez dans l`allégresse, parce que votre récompense sera grande dans les cieux; car c`est ainsi qu`on a persécuté les prophètes qui ont été avant vous. 
\verse Vous êtes le sel de la terre. Mais si le sel perd sa saveur, avec quoi la lui rendra-t-on? Il ne sert plus qu`à être jeté dehors, et foulé aux pieds par les hommes. 
\verse Vous êtes la lumière du monde. Une ville située sur une montagne ne peut être cachée; 
\verse et on n`allume pas une lampe pour la mettre sous le boisseau, mais on la met sur le chandelier, et elle éclaire tous ceux qui sont dans la maison. 
\verse Que votre lumière luise ainsi devant les hommes, afin qu`ils voient vos bonnes oeuvres, et qu`ils glorifient votre Père qui est dans les cieux. 
\verse Ne croyez pas que je sois venu pour abolir la loi ou les prophètes; je suis venu non pour abolir, mais pour accomplir. 
\verse Car, je vous le dis en vérité, tant que le ciel et la terre ne passeront point, il ne disparaîtra pas de la loi un seul iota ou un seul trait de lettre, jusqu`à ce que tout soit arrivé. 
\verse Celui donc qui supprimera l`un de ces plus petits commandements, et qui enseignera aux hommes à faire de même, sera appelé le plus petit dans le royaume des cieux; mais celui qui les observera, et qui enseignera à les observer, celui-là sera appelé grand dans le royaume des cieux. 
\verse Car, je vous le dis, si votre justice ne surpasse celle des scribes et des pharisiens, vous n`entrerez point dans le royaume des cieux. 
\verse Vous avez entendu qu`il a été dit aux anciens: Tu ne tueras point; celui qui tuera mérite d`être puni par les juges. 
\verse Mais moi, je vous dis que quiconque se met en colère contre son frère mérite d`être puni par les juges; que celui qui dira à son frère: Raca! mérite d`être puni par le sanhédrin; et que celui qui lui dira: Insensé! mérite d`être puni par le feu de la géhenne. 
\verse Si donc tu présentes ton offrande à l`autel, et que là tu te souviennes que ton frère a quelque chose contre toi, 
\verse laisse là ton offrande devant l`autel, et va d`abord te réconcilier avec ton frère; puis, viens présenter ton offrande. 
\verse Accorde-toi promptement avec ton adversaire, pendant que tu es en chemin avec lui, de peur qu`il ne te livre au juge, que le juge ne te livre à l`officier de justice, et que tu ne sois mis en prison. 
\verse Je te le dis en vérité, tu ne sortiras pas de là que tu n`aies payé le dernier quadrant. 
\verse Vous avez appris qu`il a été dit: Tu ne commettras point d`adultère. 
\verse Mais moi, je vous dis que quiconque regarde une femme pour la convoiter a déjà commis un adultère avec elle dans son coeur. 
\verse Si ton oeil droit est pour toi une occasion de chute, arrache-le et jette-le loin de toi; car il est avantageux pour toi qu`un seul de tes membres périsse, et que ton corps entier ne soit pas jeté dans la géhenne. 
\verse Et si ta main droite est pour toi une occasion de chute, coupe-la et jette-la loin de toi; car il est avantageux pour toi qu`un seul de tes membres périsse, et que ton corps entier n`aille pas dans la géhenne. 
\verse Il a été dit: Que celui qui répudie sa femme lui donne une lettre de divorce. 
\verse Mais moi, je vous dis que celui qui répudie sa femme, sauf pour cause d`infidélité, l`expose à devenir adultère, et que celui qui épouse une femme répudiée commet un adultère. 
\verse Vous avez encore appris qu`il a été dit aux anciens: Tu ne te parjureras point, mais tu t`acquitteras envers le Seigneur de ce que tu as déclaré par serment. 
\verse Mais moi, je vous dis de ne jurer aucunement, ni par le ciel, parce que c`est le trône de Dieu; 
\verse ni par la terre, parce que c`est son marchepied; ni par Jérusalem, parce que c`est la ville du grand roi. 
\verse Ne jure pas non plus par ta tête, car tu ne peux rendre blanc ou noir un seul cheveu. 
\verse Que votre parole soit oui, oui, non, non; ce qu`on y ajoute vient du malin. 
\verse Vous avez appris qu`il a été dit: oeil pour oeil, et dent pour dent. 
\verse Mais moi, je vous dis de ne pas résister au méchant. Si quelqu`un te frappe sur la joue droite, présente-lui aussi l`autre. 
\verse Si quelqu`un veut plaider contre toi, et prendre ta tunique, laisse-lui encore ton manteau. 
\verse Si quelqu`un te force à faire un mille, fais-en deux avec lui. 
\verse Donne à celui qui te demande, et ne te détourne pas de celui qui veut emprunter de toi. 
\verse Vous avez appris qu`il a été dit: Tu aimeras ton prochain, et tu haïras ton ennemi. 
\verse Mais moi, je vous dis: Aimez vos ennemis, bénissez ceux qui vous maudissent, faites du bien à ceux qui vous haïssent, et priez pour ceux qui vous maltraitent et qui vous persécutent, 
\verse afin que vous soyez fils de votre Père qui est dans les cieux; car il fait lever son soleil sur les méchants et sur les bons, et il fait pleuvoir sur les justes et sur les injustes. 
\verse Si vous aimez ceux qui vous aiment, quelle récompense méritez-vous? Les publicains aussi n`agissent-ils pas de même? 
\verse Et si vous saluez seulement vos frères, que faites-vous d`extraordinaire? Les païens aussi n`agissent-ils pas de même? 
\verse Soyez donc parfaits, comme votre Père céleste est parfait. 

\chapter[Évangile selon Matthieu]

\chaptermark{Évangile selon Matthieu}{}
\verse Gardez-vous de pratiquer votre justice devant les hommes, pour en être vus; autrement, vous n`aurez point de récompense auprès de votre Père qui est dans les cieux. 
\verse Lors donc que tu fais l`aumône, ne sonne pas de la trompette devant toi, comme font les hypocrites dans les synagogues et dans les rues, afin d`être glorifiés par les hommes. Je vous le dis en vérité, ils reçoivent leur récompense. 
\verse Mais quand tu fais l`aumône, que ta main gauche ne sache pas ce que fait ta droite, 
\verse afin que ton aumône se fasse en secret; et ton Père, qui voit dans le secret, te le rendra. 
\verse Lorsque vous priez, ne soyez pas comme les hypocrites, qui aiment à prier debout dans les synagogues et aux coins des rues, pour être vus des hommes. Je vous le dis en vérité, ils reçoivent leur récompense. 
\verse Mais quand tu pries, entre dans ta chambre, ferme ta porte, et prie ton Père qui est là dans le lieu secret; et ton Père, qui voit dans le secret, te le rendra. 
\verse En priant, ne multipliez pas de vaines paroles, comme les païens, qui s`imaginent qu`à force de paroles ils seront exaucés. 
\verse Ne leur ressemblez pas; car votre Père sait de quoi vous avez besoin, avant que vous le lui demandiez. 
\verse Voici donc comment vous devez prier: Notre Père qui es aux cieux! Que ton nom soit sanctifié; 
\verse que ton règne vienne; que ta volonté soit faite sur la terre comme au ciel. 
\verse Donne-nous aujourd`hui notre pain quotidien; 
\verse pardonne-nous nos offenses, comme nous aussi nous pardonnons à ceux qui nous ont offensés; 
\verse ne nous induis pas en tentation, mais délivre-nous du malin. Car c`est à toi qu`appartiennent, dans tous les siècles, le règne, la puissance et la gloire. Amen! 
\verse Si vous pardonnez aux hommes leurs offenses, votre Père céleste vous pardonnera aussi; 
\verse mais si vous ne pardonnez pas aux hommes, votre Père ne vous pardonnera pas non plus vos offenses. 
\verse Lorsque vous jeûnez, ne prenez pas un air triste, comme les hypocrites, qui se rendent le visage tout défait, pour montrer aux hommes qu`ils jeûnent. Je vous le dis en vérité, ils reçoivent leur récompense. 
\verse Mais quand tu jeûnes, parfume ta tête et lave ton visage, 
\verse afin de ne pas montrer aux hommes que tu jeûnes, mais à ton Père qui est là dans le lieu secret; et ton Père, qui voit dans le secret, te le rendra. 
\verse Ne vous amassez pas des trésors sur la terre, où la teigne et la rouille détruisent, et où les voleurs percent et dérobent; 
\verse mais amassez-vous des trésors dans le ciel, où la teigne et la rouille ne détruisent point, et où les voleurs ne percent ni ne dérobent. 
\verse Car là où est ton trésor, là aussi sera ton coeur. 
\verse L`oeil est la lampe du corps. Si ton oeil est en bon état, tout ton corps sera éclairé; 
\verse mais si ton oeil est en mauvais état, tout ton corps sera dans les ténèbres. Si donc la lumière qui est en toi est ténèbres, combien seront grandes ces ténèbres! 
\verse Nul ne peut servir deux maîtres. Car, ou il haïra l`un, et aimera l`autre; ou il s`attachera à l`un, et méprisera l`autre. Vous ne pouvez servir Dieu et Mamon. 
\verse C`est pourquoi je vous dis: Ne vous inquiétez pas pour votre vie de ce que vous mangerez, ni pour votre corps, de quoi vous serez vêtus. La vie n`est-elle pas plus que la nourriture, et le corps plus que le vêtement? 
\verse Regardez les oiseaux du ciel: ils ne sèment ni ne moissonnent, et ils n`amassent rien dans des greniers; et votre Père céleste les nourrit. Ne valez-vous pas beaucoup plus qu`eux? 
\verse Qui de vous, par ses inquiétudes, peut ajouter une coudée à la durée de sa vie? 
\verse Et pourquoi vous inquiéter au sujet du vêtement? Considérez comment croissent les lis des champs: ils ne travaillent ni ne filent; 
\verse cependant je vous dis que Salomon même, dans toute sa gloire, n`a pas été vêtu comme l`un d`eux. 
\verse Si Dieu revêt ainsi l`herbe des champs, qui existe aujourd`hui et qui demain sera jetée au four, ne vous vêtira-t-il pas à plus forte raison, gens de peu de foi? 
\verse Ne vous inquiétez donc point, et ne dites pas: Que mangerons-nous? que boirons-nous? de quoi serons-nous vêtus? 
\verse Car toutes ces choses, ce sont les païens qui les recherchent. Votre Père céleste sait que vous en avez besoin. 
\verse Cherchez premièrement le royaume et la justice de Dieu; et toutes ces choses vous seront données par-dessus. 
\verse Ne vous inquiétez donc pas du lendemain; car le lendemain aura soin de lui-même. A chaque jour suffit sa peine. 

\chapter[Évangile selon Matthieu]

\chaptermark{Évangile selon Matthieu}{}
\verse Ne jugez point, afin que vous ne soyez point jugés. 
\verse Car on vous jugera du jugement dont vous jugez, et l`on vous mesurera avec la mesure dont vous mesurez. 
\verse Pourquoi vois-tu la paille qui est dans l`oeil de ton frère, et n`aperçois-tu pas la poutre qui est dans ton oeil? 
\verse Ou comment peux-tu dire à ton frère: Laisse-moi ôter une paille de ton oeil, toi qui as une poutre dans le tien? 
\verse Hypocrite, ôte premièrement la poutre de ton oeil, et alors tu verras comment ôter la paille de l`oeil de ton frère. 
\verse Ne donnez pas les choses saintes aux chiens, et ne jetez pas vos perles devant les pourceaux, de peur qu`ils ne les foulent aux pieds, ne se retournent et ne vous déchirent. 
\verse Demandez, et l`on vous donnera; cherchez, et vous trouverez; frappez, et l`on vous ouvrira. 
\verse Car quiconque demande reçoit, celui qui cherche trouve, et l`on ouvre à celui qui frappe. 
\verse Lequel de vous donnera une pierre à son fils, s`il lui demande du pain? 
\verse Ou, s`il demande un poisson, lui donnera-t-il un serpent? 
\verse Si donc, méchants comme vous l`êtes, vous savez donner de bonnes choses à vos enfants, à combien plus forte raison votre Père qui est dans les cieux donnera-t-il de bonnes choses à ceux qui les lui demandent. 
\verse Tout ce que vous voulez que les hommes fassent pour vous, faites-le de même pour eux, car c`est la loi et les prophètes. 
\verse Entrez par la porte étroite. Car large est la porte, spacieux est le chemin qui mènent à la perdition, et il y en a beaucoup qui entrent par là. 
\verse Mais étroite est la porte, resserré le chemin qui mènent à la vie, et il y en a peu qui les trouvent. 
\verse Gardez-vous des faux prophètes. Ils viennent à vous en vêtement de brebis, mais au dedans ce sont des loups ravisseurs. 
\verse Vous les reconnaîtrez à leurs fruits. Cueille-t-on des raisins sur des épines, ou des figues sur des chardons? 
\verse Tout bon arbre porte de bons fruits, mais le mauvais arbre porte de mauvais fruits. 
\verse Un bon arbre ne peut porter de mauvais fruits, ni un mauvais arbre porter de bons fruits. 
\verse Tout arbre qui ne porte pas de bons fruits est coupé et jeté au feu. 
\verse C`est donc à leurs fruits que vous les reconnaîtrez. 
\verse Ceux qui me disent: Seigneur, Seigneur! n`entreront pas tous dans le royaume des cieux, mais celui-là seul qui fait la volonté de mon Père qui est dans les cieux. 
\verse Plusieurs me diront en ce jour-là: Seigneur, Seigneur, n`avons-nous pas prophétisé par ton nom? n`avons-nous pas chassé des démons par ton nom? et n`avons-nous pas fait beaucoup de miracles par ton nom? 
\verse Alors je leur dirai ouvertement: Je ne vous ai jamais connus, retirez-vous de moi, vous qui commettez l`iniquité. 
\verse C`est pourquoi, quiconque entend ces paroles que je dis et les met en pratique, sera semblable à un homme prudent qui a bâti sa maison sur le roc. 
\verse La pluie est tombée, les torrents sont venus, les vents ont soufflé et se sont jetés contre cette maison: elle n`est point tombée, parce qu`elle était fondée sur le roc. 
\verse Mais quiconque entend ces paroles que je dis, et ne les met pas en pratique, sera semblable à un homme insensé qui a bâti sa maison sur le sable. 
\verse La pluie est tombée, les torrents sont venus, les vents ont soufflé et ont battu cette maison: elle est tombée, et sa ruine a été grande. 
\verse Après que Jésus eut achevé ces discours, la foule fut frappée de sa doctrine; 
\verse car il enseignait comme ayant autorité, et non pas comme leurs scribes. 

\chapter[Évangile selon Matthieu]

\chaptermark{Évangile selon Matthieu}{}
\verse Lorsque Jésus fut descendu de la montagne, une grande foule le suivit. 
\verse Et voici, un lépreux s`étant approché se prosterna devant lui, et dit: Seigneur, si tu le veux, tu peux me rendre pur. 
\verse Jésus étendit la main, le toucha, et dit: Je le veux, sois pur. Aussitôt il fut purifié de sa lèpre. 
\verse Puis Jésus lui dit: Garde-toi d`en parler à personne; mais va te montrer au sacrificateur, et présente l`offrande que Moïse a prescrite, afin que cela leur serve de témoignage. 
\verse Comme Jésus entrait dans Capernaüm, un centenier l`aborda, 
\verse le priant et disant: Seigneur, mon serviteur est couché à la maison, atteint de paralysie et souffrant beaucoup. 
\verse Jésus lui dit: J`irai, et je le guérirai. 
\verse Le centenier répondit: Seigneur, je ne suis pas digne que tu entres sous mon toit; mais dis seulement un mot, et mon serviteur sera guéri. 
\verse Car, moi qui suis soumis à des supérieurs, j`ai des soldats sous mes ordres; et je dis à l`un: Va! et il va; à l`autre: Viens! et il vient; et à mon serviteur: Fais cela! et il le fait. 
\verse Après l`avoir entendu, Jésus fut dans l`étonnement, et il dit à ceux qui le suivaient: Je vous le dis en vérité, même en Israël je n`ai pas trouvé une aussi grande foi. 
\verse Or, je vous déclare que plusieurs viendront de l`orient et de l`occident, et seront à table avec Abraham, Isaac et Jacob, dans le royaume des cieux. 
\verse Mais les fils du royaume seront jetés dans les ténèbres du dehors, où il y aura des pleurs et des grincements de dents. 
\verse Puis Jésus dit au centenier: Va, qu`il te soit fait selon ta foi. Et à l`heure même le serviteur fut guéri. 
\verse Jésus se rendit ensuite à la maison de Pierre, dont il vit la belle-mère couchée et ayant la fièvre. 
\verse Il toucha sa main, et la fièvre la quitta; puis elle se leva, et le servit. 
\verse Le soir, on amena auprès de Jésus plusieurs démoniaques. Il chassa les esprits par sa parole, et il guérit tous les malades, 
\verse afin que s`accomplît ce qui avait été annoncé par Ésaïe, le prophète: Il a pris nos infirmités, et il s`est chargé de nos maladies. 
\verse Jésus, voyant une grande foule autour de lui, donna l`ordre de passer à l`autre bord. 
\verse Un scribe s`approcha, et lui dit: Maître, je te suivrai partout où tu iras. 
\verse Jésus lui répondit: Les renards ont des tanières, et les oiseaux du ciel ont des nids; mais le Fils de l`homme n`a pas où reposer sa tête. 
\verse Un autre, d`entre les disciples, lui dit: Seigneur, permets-moi d`aller d`abord ensevelir mon père. 
\verse Mais Jésus lui répondit: Suis-moi, et laisse les morts ensevelir leurs morts. 
\verse Il monta dans la barque, et ses disciples le suivirent. 
\verse Et voici, il s`éleva sur la mer une si grande tempête que la barque était couverte par les flots. Et lui, il dormait. 
\verse Les disciples s`étant approchés le réveillèrent, et dirent: Seigneur, sauve-nous, nous périssons! 
\verse Il leur dit: Pourquoi avez-vous peur, gens de peu de foi? Alors il se leva, menaça les vents et la mer, et il y eut un grand calme. 
\verse Ces hommes furent saisis d`étonnement: Quel est celui-ci, disaient-ils, à qui obéissent même les vents et la mer? 
\verse Lorsqu`il fut à l`autre bord, dans le pays des Gadaréniens, deux démoniaques, sortant des sépulcres, vinrent au-devant de lui. Ils étaient si furieux que personne n`osait passer par là. 
\verse Et voici, ils s`écrièrent: Qu`y a-t-il entre nous et toi, Fils de Dieu? Es-tu venu ici pour nous tourmenter avant le temps? 
\verse Il y avait loin d`eux un grand troupeau de pourceaux qui paissaient. 
\verse Les démons priaient Jésus, disant: Si tu nous chasses, envoie-nous dans ce troupeau de pourceaux. 
\verse Il leur dit: Allez! Ils sortirent, et entrèrent dans les pourceaux. Et voici, tout le troupeau se précipita des pentes escarpées dans la mer, et ils périrent dans les eaux. 
\verse Ceux qui les faisaient paître s`enfuirent, et allèrent dans la ville raconter tout ce qui s`était passé et ce qui était arrivé aux démoniaques. 
\verse Alors toute la ville sortit à la rencontre de Jésus; et, dès qu`ils le virent, ils le supplièrent de quitter leur territoire. 

\chapter[Évangile selon Matthieu]

\chaptermark{Évangile selon Matthieu}{}
\verse Jésus, étant monté dans une barque, traversa la mer, et alla dans sa ville. 
\verse Et voici, on lui amena un paralytique couché sur un lit. Jésus, voyant leur foi, dit au paralytique: Prends courage, mon enfant, tes péchés te sont pardonnés. 
\verse Sur quoi, quelques scribes dirent au dedans d`eux: Cet homme blasphème. 
\verse Et Jésus, connaissant leurs pensées, dit: Pourquoi avez-vous de mauvaises pensées dans vos coeurs? 
\verse Car, lequel est le plus aisé, de dire: Tes péchés sont pardonnés, ou de dire: Lève-toi, et marche? 
\verse Or, afin que vous sachiez que le Fils de l`homme a sur la terre le pouvoir de pardonner les péchés: Lève-toi, dit-il au paralytique, prends ton lit, et va dans ta maison. 
\verse Et il se leva, et s`en alla dans sa maison. 
\verse Quand la foule vit cela, elle fut saisie de crainte, et elle glorifia Dieu, qui a donné aux hommes un tel pouvoir. 
\verse De là étant allé plus loin, Jésus vit un homme assis au lieu des péages, et qui s`appelait Matthieu. Il lui dit: Suis-moi. Cet homme se leva, et le suivit. 
\verse Comme Jésus était à table dans la maison, voici, beaucoup de publicains et de gens de mauvaise vie vinrent se mettre à table avec lui et avec ses disciples. 
\verse Les pharisiens virent cela, et ils dirent à ses disciples: Pourquoi votre maître mange-t-il avec les publicains et les gens de mauvaise vie? 
\verse Ce que Jésus ayant entendu, il dit: Ce ne sont pas ceux qui se portent bien qui ont besoin de médecin, mais les malades. 
\verse Allez, et apprenez ce que signifie: Je prends plaisir à la miséricorde, et non aux sacrifices. Car je ne suis pas venu appeler des justes, mais des pécheurs. 
\verse Alors les disciples de Jean vinrent auprès de Jésus, et dirent: Pourquoi nous et les pharisiens jeûnons-nous, tandis que tes disciples ne jeûnent point? 
\verse Jésus leur répondit: Les amis de l`époux peuvent-ils s`affliger pendant que l`époux est avec eux? Les jours viendront où l`époux leur sera enlevé, et alors ils jeûneront. 
\verse Personne ne met une pièce de drap neuf à un vieil habit; car elle emporterait une partie de l`habit, et la déchirure serait pire. 
\verse On ne met pas non plus du vin nouveau dans de vieilles outres; autrement, les outres se rompent, le vin se répand, et les outres sont perdues; mais on met le vin nouveau dans des outres neuves, et le vin et les outres se conservent. 
\verse Tandis qu`il leur adressait ces paroles, voici, un chef arriva, se prosterna devant lui, et dit: Ma fille est morte il y a un instant; mais viens, impose-lui les mains, et elle vivra. 
\verse Jésus se leva, et le suivit avec ses disciples. 
\verse Et voici, une femme atteinte d`une perte de sang depuis douze ans s`approcha par derrière, et toucha le bord de son vêtement. 
\verse Car elle disait en elle-même: Si je puis seulement toucher son vêtement, je serai guérie. 
\verse Jésus se retourna, et dit, en la voyant: Prends courage, ma fille, ta foi t`a guérie. Et cette femme fut guérie à l`heure même. 
\verse Lorsque Jésus fut arrivé à la maison du chef, et qu`il vit les joueurs de flûte et la foule bruyante, 
\verse il leur dit: Retirez-vous; car la jeune fille n`est pas morte, mais elle dort. Et ils se moquaient de lui. 
\verse Quand la foule eut été renvoyée, il entra, prit la main de la jeune fille, et la jeune fille se leva. 
\verse Le bruit s`en répandit dans toute la contrée. 
\verse Étant parti de là, Jésus fut suivi par deux aveugles, qui criaient: Aie pitié de nous, Fils de David! 
\verse Lorsqu`il fut arrivé à la maison, les aveugles s`approchèrent de lui, et Jésus leur dit: Croyez-vous que je puisse faire cela? Oui, Seigneur, lui répondirent-ils. 
\verse Alors il leur toucha leurs yeux, en disant: Qu`il vous soit fait selon votre foi. 
\verse Et leurs yeux s`ouvrirent. Jésus leur fit cette recommandation sévère: Prenez garde que personne ne le sache. 
\verse Mais, dès qu`ils furent sortis, ils répandirent sa renommée dans tout le pays. 
\verse Comme ils s`en allaient, voici, on amena à Jésus un démoniaque muet. 
\verse Le démon ayant été chassé, le muet parla. Et la foule étonnée disait: Jamais pareille chose ne s`est vue en Israël. 
\verse Mais les pharisiens dirent: C`est par le prince des démons qu`il chasse les démons. 
\verse Jésus parcourait toutes les villes et les villages, enseignant dans les synagogues, prêchant la bonne nouvelle du royaume, et guérissant toute maladie et toute infirmité. 
\verse Voyant la foule, il fut ému de compassion pour elle, parce qu`elle était languissante et abattue, comme des brebis qui n`ont point de berger. 
\verse Alors il dit à ses disciples: La moisson est grande, mais il y a peu d`ouvriers. 
\verse Priez donc le maître de la moisson d`envoyer des ouvriers dans sa moisson. 

\chapter[Évangile selon Matthieu]

\chaptermark{Évangile selon Matthieu}{}
\verse Puis, ayant appelé ses douze disciples, il leur donna le pouvoir de chasser les esprits impurs, et de guérir toute maladie et toute infirmité. 
\verse Voici les noms des douze apôtres. Le premier, Simon appelé Pierre, et André, son frère; Jacques, fils de Zébédée, et Jean, son frère; 
\verse Philippe, et Barthélemy; Thomas, et Matthieu, le publicain; Jacques, fils d`Alphée, et Thaddée; 
\verse Simon le Cananite, et Judas l`Iscariot, celui qui livra Jésus. 
\verse Tels sont les douze que Jésus envoya, après leur avoir donné les instructions suivantes: N`allez pas vers les païens, et n`entrez pas dans les villes des Samaritains; 
\verse allez plutôt vers les brebis perdues de la maison d`Israël. 
\verse Allez, prêchez, et dites: Le royaume des cieux est proche. 
\verse Guérissez les malades, ressuscitez les morts, purifiez les lépreux, chassez les démons. Vous avez reçu gratuitement, donnez gratuitement. 
\verse Ne prenez ni or, ni argent, ni monnaie, dans vos ceintures; 
\verse ni sac pour le voyage, ni deux tuniques, ni souliers, ni bâton; car l`ouvrier mérite sa nourriture. 
\verse Dans quelque ville ou village que vous entriez, informez-vous s`il s`y trouve quelque homme digne de vous recevoir; et demeurez chez lui jusqu`à ce que vous partiez. 
\verse En entrant dans la maison, saluez-la; 
\verse et, si la maison en est digne, que votre paix vienne sur elle; mais si elle n`en est pas digne, que votre paix retourne à vous. 
\verse Lorsqu`on ne vous recevra pas et qu`on n`écoutera pas vos paroles, sortez de cette maison ou de cette ville et secouez la poussière de vos pieds. 
\verse Je vous le dis en vérité: au jour du jugement, le pays de Sodome et de Gomorrhe sera traité moins rigoureusement que cette ville-là. 
\verse Voici, je vous envoie comme des brebis au milieu des loups. Soyez donc prudents comme les serpents, et simples comme les colombes. 
\verse Mettez-vous en garde contre les hommes; car ils vous livreront aux tribunaux, et ils vous battront de verges dans leurs synagogues; 
\verse vous serez menés, à cause de moi, devant des gouverneurs et devant des rois, pour servir de témoignage à eux et aux païens. 
\verse Mais, quand on vous livrera, ne vous inquiétez ni de la manière dont vous parlerez ni de ce que vous direz: ce que vous aurez à dire vous sera donné à l`heure même; 
\verse car ce n`est pas vous qui parlerez, c`est l`Esprit de votre Père qui parlera en vous. 
\verse Le frère livrera son frère à la mort, et le père son enfant; les enfants se soulèveront contre leurs parents, et les feront mourir. 
\verse Vous serez haïs de tous, à cause de mon nom; mais celui qui persévérera jusqu`à la fin sera sauvé. 
\verse Quand on vous persécutera dans une ville, fuyez dans une autre. Je vous le dis en vérité, vous n`aurez pas achevé de parcourir les villes d`Israël que le Fils de l`homme sera venu. 
\verse Le disciple n`est pas plus que le maître, ni le serviteur plus que son seigneur. 
\verse Il suffit au disciple d`être traité comme son maître, et au serviteur comme son seigneur. S`ils ont appelé le maître de la maison Béelzébul, à combien plus forte raison appelleront-ils ainsi les gens de sa maison! 
\verse Ne les craignez donc point; car il n`y a rien de caché qui ne doive être découvert, ni de secret qui ne doive être connu. 
\verse Ce que je vous dis dans les ténèbres, dites-le en plein jour; et ce qui vous est dit à l`oreille, prêchez-le sur les toits. 
\verse Ne craignez pas ceux qui tuent le corps et qui ne peuvent tuer l`âme; craignez plutôt celui qui peut faire périr l`âme et le corps dans la géhenne. 
\verse Ne vend-on pas deux passereaux pour un sou? Cependant, il n`en tombe pas un à terre sans la volonté de votre Père. 
\verse Et même les cheveux de votre tête sont tous comptés. 
\verse Ne craignez donc point: vous valez plus que beaucoup de passereaux. 
\verse C`est pourquoi, quiconque me confessera devant les hommes, je le confesserai aussi devant mon Père qui est dans les cieux; 
\verse mais quiconque me reniera devant les hommes, je le renierai aussi devant mon Père qui est dans les cieux. 
\verse Ne croyez pas que je sois venu apporter la paix sur la terre; je ne suis pas venu apporter la paix, mais l`épée. 
\verse Car je suis venu mettre la division entre l`homme et son père, entre la fille et sa mère, entre la belle-fille et sa belle-mère; 
\verse et l`homme aura pour ennemis les gens de sa maison. 
\verse Celui qui aime son père ou sa mère plus que moi n`est pas digne de moi, et celui qui aime son fils ou sa fille plus que moi n`est pas digne de moi; 
\verse celui qui ne prend pas sa croix, et ne me suit pas, n`est pas digne de moi. 
\verse Celui qui conservera sa vie la perdra, et celui qui perdra sa vie à cause de moi la retrouvera. 
\verse Celui qui vous reçoit me reçoit, et celui qui me reçoit, reçoit celui qui m`a envoyé. 
\verse Celui qui reçoit un prophète en qualité de prophète recevra une récompense de prophète, et celui qui reçoit un juste en qualité de juste recevra une récompense de juste. 
\verse Et quiconque donnera seulement un verre d`eau froide à l`un de ces petits parce qu`il est mon disciple, je vous le dis en vérité, il ne perdra point sa récompense. 

\chapter[Évangile selon Matthieu]

\chaptermark{Évangile selon Matthieu}{}
\verse Lorsque Jésus eut achevé de donner ses instructions à ses douze disciples, il partit de là, pour enseigner et prêcher dans les villes du pays. 
\verse Jean, ayant entendu parler dans sa prison des oeuvres du Christ, lui fit dire par ses disciples: 
\verse Es-tu celui qui doit venir, ou devons-nous en attendre un autre? 
\verse Jésus leur répondit: Allez rapporter à Jean ce que vous entendez et ce que vous voyez: 
\verse les aveugles voient, les boiteux marchent, les lépreux sont purifiés, les sourds entendent, les morts ressuscitent, et la bonne nouvelle est annoncée aux pauvres. 
\verse Heureux celui pour qui je ne serai pas une occasion de chute! 
\verse Comme ils s`en allaient, Jésus se mit à dire à la foule, au sujet de Jean: Qu`êtes-vous allés voir au désert? un roseau agité par le vent? 
\verse Mais, qu`êtes-vous allés voir? un homme vêtu d`habits précieux? Voici, ceux qui portent des habits précieux sont dans les maisons des rois. 
\verse Qu`êtes-vous donc allés voir? un prophète? Oui, vous dis-je, et plus qu`un prophète. 
\verse Car c`est celui dont il est écrit: Voici, j`envoie mon messager devant ta face, Pour préparer ton chemin devant toi. 
\verse Je vous le dis en vérité, parmi ceux qui sont nés de femmes, il n`en a point paru de plus grand que Jean Baptiste. Cependant, le plus petit dans le royaume des cieux est plus grand que lui. 
\verse Depuis le temps de Jean Baptiste jusqu`à présent, le royaume des cieux est forcé, et ce sont les violents qui s`en s`emparent. 
\verse Car tous les prophètes et la loi ont prophétisé jusqu`à Jean; 
\verse et, si vous voulez le comprendre, c`est lui qui est l`Élie qui devait venir. 
\verse Que celui qui a des oreilles pour entendre entende. 
\verse A qui comparerai-je cette génération? Elle ressemble à des enfants assis dans des places publiques, et qui, s`adressant à d`autres enfants, 
\verse disent: Nous vous avons joué de la flûte, et vous n`avez pas dansé; nous avons chanté des complaintes, et vous ne vous êtes pas lamentés. 
\verse Car Jean est venu, ne mangeant ni ne buvant, et ils disent: Il a un démon. 
\verse Le Fils de l`homme est venu, mangeant et buvant, et ils disent: C`est un mangeur et un buveur, un ami des publicains et des gens de mauvaise vie. Mais la sagesse a été justifiée par ses oeuvres. 
\verse Alors il se mit à faire des reproches aux villes dans lesquelles avaient eu lieu la plupart de ses miracles, parce qu`elles ne s`étaient pas repenties. 
\verse Malheur à toi, Chorazin! malheur à toi, Bethsaïda! car, si les miracles qui ont été faits au milieu de vous avaient été faits dans Tyr et dans Sidon, il y a longtemps qu`elles se seraient repenties, en prenant le sac et la cendre. 
\verse C`est pourquoi je vous le dis: au jour du jugement, Tyr et Sidon seront traitées moins rigoureusement que vous. 
\verse Et toi, Capernaüm, seras-tu élevée jusqu`au ciel? Non. Tu seras abaissée jusqu`au séjour des morts; car, si les miracles qui ont été faits au milieu de toi avaient été faits dans Sodome, elle subsisterait encore aujourd`hui. 
\verse C`est pourquoi je vous le dis: au jour du jugement, le pays de Sodome sera traité moins rigoureusement que toi. 
\verse En ce temps-là, Jésus prit la parole, et dit: Je te loue, Père, Seigneur du ciel et de la terre, de ce que tu as caché ces choses aux sages et aux intelligents, et de ce que tu les as révélées aux enfants. 
\verse Oui, Père, je te loue de ce que tu l`as voulu ainsi. 
\verse Toutes choses m`ont été données par mon Père, et personne ne connaît le Fils, si ce n`est le Père; personne non plus ne connaît le Père, si ce n`est le Fils et celui à qui le Fils veut le révéler. 
\verse Venez à moi, vous tous qui êtes fatigués et chargés, et je vous donnerai du repos. 
\verse Prenez mon joug sur vous et recevez mes instructions, car je suis doux et humble de coeur; et vous trouverez du repos pour vos âmes. 
\verse Car mon joug est doux, et mon fardeau léger. 

\chapter[Évangile selon Matthieu]

\chaptermark{Évangile selon Matthieu}{}
\verse En ce temps-là, Jésus traversa des champs de blé un jour de sabbat. Ses disciples, qui avaient faim, se mirent à arracher des épis et à manger. 
\verse Les pharisiens, voyant cela, lui dirent: Voici, tes disciples font ce qu`il n`est pas permis de faire pendant le sabbat. 
\verse Mais Jésus leur répondit: N`avez-vous pas lu ce que fit David, lorsqu`il eut faim, lui et ceux qui étaient avec lui; 
\verse comment il entra dans la maison de Dieu, et mangea les pains de proposition, qu`il ne lui était pas permis de manger, non plus qu`à ceux qui étaient avec lui, et qui étaient réservés aux sacrificateurs seuls? 
\verse Ou, n`avez-vous pas lu dans la loi que, les jours de sabbat, les sacrificateurs violent le sabbat dans le temple, sans se rendre coupables? 
\verse Or, je vous le dis, il y a ici quelque chose de plus grand que le temple. 
\verse Si vous saviez ce que signifie: Je prends plaisir à la miséricorde, et non aux sacrifices, vous n`auriez pas condamné des innocents. 
\verse Car le Fils de l`homme est maître du sabbat. 
\verse Étant parti de là, Jésus entra dans la synagogue. 
\verse Et voici, il s`y trouvait un homme qui avait la main sèche. Ils demandèrent à Jésus: Est-il permis de faire une guérison les jours de sabbat? C`était afin de pouvoir l`accuser. 
\verse Il leur répondit: Lequel d`entre vous, s`il n`a qu`une brebis et qu`elle tombe dans une fosse le jour du sabbat, ne la saisira pour l`en retirer? 
\verse Combien un homme ne vaut-il pas plus qu`une brebis! Il est donc permis de faire du bien les jours de sabbat. 
\verse Alors il dit à l`homme: Étends ta main. Il l`étendit, et elle devint saine comme l`autre. 
\verse Les pharisiens sortirent, et ils se consultèrent sur les moyens de le faire périr. 
\verse Mais Jésus, l`ayant su, s`éloigna de ce lieu. Une grande foule le suivit. Il guérit tous les malades, 
\verse et il leur recommanda sévèrement de ne pas le faire connaître, 
\verse afin que s`accomplît ce qui avait été annoncé par Ésaïe, le prophète: 
\verse Voici mon serviteur que j`ai choisi, Mon bien-aimé en qui mon âme a pris plaisir. Je mettrai mon Esprit sur lui, Et il annoncera la justice aux nations. 
\verse Il ne contestera point, il ne criera point, Et personne n`entendra sa voix dans les rues. 
\verse Il ne brisera point le roseau cassé, Et il n`éteindra point le lumignon qui fume, Jusqu`à ce qu`il ait fait triompher la justice. 
\verse Et les nations espéreront en son nom. 
\verse Alors on lui amena un démoniaque aveugle et muet, et il le guérit, de sorte que le muet parlait et voyait. 
\verse Toute la foule étonnée disait: N`est-ce point là le Fils de David? 
\verse Les pharisiens, ayant entendu cela, dirent: Cet homme ne chasse les démons que par Béelzébul, prince des démons. 
\verse Comme Jésus connaissait leurs pensées, il leur dit: Tout royaume divisé contre lui-même est dévasté, et toute ville ou maison divisée contre elle-même ne peut subsister. 
\verse Si Satan chasse Satan, il est divisé contre lui-même; comment donc son royaume subsistera-t-il? 
\verse Et si moi, je chasse les démons par Béelzébul, vos fils, par qui les chassent-ils? C`est pourquoi ils seront eux-mêmes vos juges. 
\verse Mais, si c`est par l`Esprit de Dieu que je chasse les démons, le royaume de Dieu est donc venu vers vous. 
\verse Ou, comment quelqu`un peut-il entrer dans la maison d`un homme fort et piller ses biens, sans avoir auparavant lié cet homme fort? Alors seulement il pillera sa maison. 
\verse Celui qui n`est pas avec moi est contre moi, et celui qui n`assemble pas avec moi disperse. 
\verse C`est pourquoi je vous dis: Tout péché et tout blasphème sera pardonné aux hommes, mais le blasphème contre l`Esprit ne sera point pardonné. 
\verse Quiconque parlera contre le Fils de l`homme, il lui sera pardonné; mais quiconque parlera contre le Saint Esprit, il ne lui sera pardonné ni dans ce siècle ni dans le siècle à venir. 
\verse Ou dites que l`arbre est bon et que son fruit est bon, ou dites que l`arbre est mauvais et que son fruit est mauvais; car on connaît l`arbre par le fruit. 
\verse Races de vipères, comment pourriez-vous dire de bonnes choses, méchants comme vous l`êtes? Car c`est de l`abondance du coeur que la bouche parle. 
\verse L`homme bon tire de bonnes choses de son bon trésor, et l`homme méchant tire de mauvaises choses de son mauvais trésor. 
\verse Je vous le dis: au jour du jugement, les hommes rendront compte de toute parole vaine qu`ils auront proférée. 
\verse Car par tes paroles tu seras justifié, et par tes paroles tu seras condamné. 
\verse Alors quelques-uns des scribes et des pharisiens prirent la parole, et dirent: Maître, nous voudrions te voir faire un miracle. 
\verse Il leur répondit: Une génération méchante et adultère demande un miracle; il ne lui sera donné d`autre miracle que celui du prophète Jonas. 
\verse Car, de même que Jonas fut trois jours et trois nuits dans le ventre d`un grand poisson, de même le Fils de l`homme sera trois jours et trois nuits dans le sein de la terre. 
\verse Les hommes de Ninive se lèveront, au jour du jugement, avec cette génération et la condamneront, parce qu`ils se repentirent à la prédication de Jonas; et voici, il y a ici plus que Jonas. 
\verse La reine du Midi se lèvera, au jour du jugement, avec cette génération et la condamnera, parce qu`elle vint des extrémités de la terre pour entendre la sagesse de Salomon, et voici, il y a ici plus que Salomon. 
\verse Lorsque l`esprit impur est sorti d`un homme, il va par des lieux arides, cherchant du repos, et il n`en trouve point. 
\verse Alors il dit: Je retournerai dans ma maison d`où je suis sorti; et, quand il arrive, il la trouve vide, balayée et ornée. 
\verse Il s`en va, et il prend avec lui sept autres esprits plus méchants que lui; ils entrent dans la maison, s`y établissent, et la dernière condition de cet homme est pire que la première. Il en sera de même pour cette génération méchante. 
\verse Comme Jésus s`adressait encore à la foule, voici, sa mère et ses frères, qui étaient dehors, cherchèrent à lui parler. 
\verse Quelqu`un lui dit: Voici, ta mère et tes frères sont dehors, et ils cherchent à te parler. 
\verse Mais Jésus répondit à celui qui le lui disait: Qui est ma mère, et qui sont mes frères? 
\verse Puis, étendant la main sur ses disciples, il dit: Voici ma mère et mes frères. 
\verse Car, quiconque fait la volonté de mon Père qui est dans les cieux, celui-là est mon frère, et ma soeur, et ma mère. 

\chapter[Évangile selon Matthieu]

\chaptermark{Évangile selon Matthieu}{}
\verse Ce même jour, Jésus sortit de la maison, et s`assit au bord de la mer. 
\verse Une grande foule s`étant assemblée auprès de lui, il monta dans une barque, et il s`assit. Toute la foule se tenait sur le rivage. 
\verse Il leur parla en paraboles sur beaucoup de choses, et il dit: 
\verse Un semeur sortit pour semer. Comme il semait, une partie de la semence tomba le long du chemin: les oiseaux vinrent, et la mangèrent. 
\verse Une autre partie tomba dans les endroits pierreux, où elle n`avait pas beaucoup de terre: elle leva aussitôt, parce qu`elle ne trouva pas un sol profond; 
\verse mais, quand le soleil parut, elle fut brûlée et sécha, faute de racines. 
\verse Une autre partie tomba parmi les épines: les épines montèrent, et l`étouffèrent. 
\verse Une autre partie tomba dans la bonne terre: elle donna du fruit, un grain cent, un autre soixante, un autre trente. 
\verse Que celui qui a des oreilles pour entendre entende. 
\verse Les disciples s`approchèrent, et lui dirent: Pourquoi leur parles-tu en paraboles? 
\verse Jésus leur répondit: Parce qu`il vous a été donné de connaître les mystères du royaume des cieux, et que cela ne leur a pas été donné. 
\verse Car on donnera à celui qui a, et il sera dans l`abondance, mais à celui qui n`a pas on ôtera même ce qu`il a. 
\verse C`est pourquoi je leur parle en paraboles, parce qu`en voyant ils ne voient point, et qu`en entendant ils n`entendent ni ne comprennent. 
\verse Et pour eux s`accomplit cette prophétie d`Ésaïe: Vous entendrez de vos oreilles, et vous ne comprendrez point; Vous regarderez de vos yeux, et vous ne verrez point. 
\verse Car le coeur de ce peuple est devenu insensible; Ils ont endurci leurs oreilles, et ils ont fermé leurs yeux, De peur qu`ils ne voient de leurs yeux, qu`ils n`entendent de leurs oreilles, Qu`ils ne comprennent de leur coeur, Qu`ils ne se convertissent, et que je ne les guérisse. 
\verse Mais heureux sont vos yeux, parce qu`ils voient, et vos oreilles, parce qu`elles entendent! 
\verse Je vous le dis en vérité, beaucoup de prophètes et de justes ont désiré voir ce que vous voyez, et ne l`ont pas vu, entendre ce que vous entendez, et ne l`ont pas entendu. 
\verse Vous donc, écoutez ce que signifie la parabole du semeur. 
\verse Lorsqu`un homme écoute la parole du royaume et ne la comprend pas, le malin vient et enlève ce qui a été semé dans son coeur: cet homme est celui qui a reçu la semence le long du chemin. 
\verse Celui qui a reçu la semence dans les endroits pierreux, c`est celui qui entend la parole et la reçoit aussitôt avec joie; 
\verse mais il n`a pas de racines en lui-même, il manque de persistance, et, dès que survient une tribulation ou une persécution à cause de la parole, il y trouve une occasion de chute. 
\verse Celui qui a reçu la semence parmi les épines, c`est celui qui entend la parole, mais en qui les soucis du siècle et la séduction des richesses étouffent cette parole, et la rendent infructueuse. 
\verse Celui qui a reçu la semence dans la bonne terre, c`est celui qui entend la parole et la comprend; il porte du fruit, et un grain en donne cent, un autre soixante, un autre trente. 
\verse Il leur proposa une autre parabole, et il dit: Le royaume des cieux est semblable à un homme qui a semé une bonne semence dans son champ. 
\verse Mais, pendant que les gens dormaient, son ennemi vint, sema de l`ivraie parmi le blé, et s`en alla. 
\verse Lorsque l`herbe eut poussé et donné du fruit, l`ivraie parut aussi. 
\verse Les serviteurs du maître de la maison vinrent lui dire: Seigneur, n`as-tu pas semé une bonne semence dans ton champ? D`où vient donc qu`il y a de l`ivraie? 
\verse Il leur répondit: C`est un ennemi qui a fait cela. Et les serviteurs lui dirent: Veux-tu que nous allions l`arracher? 
\verse Non, dit-il, de peur qu`en arrachant l`ivraie, vous ne déraciniez en même temps le blé. 
\verse Laissez croître ensemble l`un et l`autre jusqu`à la moisson, et, à l`époque de la moisson, je dirai aux moissonneurs: Arrachez d`abord l`ivraie, et liez-la en gerbes pour la brûler, mais amassez le blé dans mon grenier. 
\verse Il leur proposa une autre parabole, et il dit: Le royaume des cieux est semblable à un grain de sénevé qu`un homme a pris et semé dans son champ. 
\verse C`est la plus petite de toutes les semences; mais, quand il a poussé, il est plus grand que les légumes et devient un arbre, de sorte que les oiseaux du ciel viennent habiter dans ses branches. 
\verse Il leur dit cette autre parabole: Le royaume des cieux est semblable à du levain qu`une femme a pris et mis dans trois mesures de farine, jusqu`à ce que la pâte soit toute levée. 
\verse Jésus dit à la foule toutes ces choses en paraboles, et il ne lui parlait point sans parabole, 
\verse afin que s`accomplît ce qui avait été annoncé par le prophète: J`ouvrirai ma bouche en paraboles, Je publierai des choses cachées depuis la création du monde. 
\verse Alors il renvoya la foule, et entra dans la maison. Ses disciples s`approchèrent de lui, et dirent: Explique-nous la parabole de l`ivraie du champ. 
\verse Il répondit: Celui qui sème la bonne semence, c`est le Fils de l`homme; 
\verse le champ, c`est le monde; la bonne semence, ce sont les fils du royaume; l`ivraie, ce sont les fils du malin; 
\verse l`ennemi qui l`a semée, c`est le diable; la moisson, c`est la fin du monde; les moissonneurs, ce sont les anges. 
\verse Or, comme on arrache l`ivraie et qu`on la jette au feu, il en sera de même à la fin du monde. 
\verse Le Fils de l`homme enverra ses anges, qui arracheront de son royaume tous les scandales et ceux qui commettent l`iniquité: 
\verse et ils les jetteront dans la fournaise ardente, où il y aura des pleurs et des grincements de dents. 
\verse Alors les justes resplendiront comme le soleil dans le royaume de leur Père. Que celui qui a des oreilles pour entendre entende. 
\verse Le royaume des cieux est encore semblable à un trésor caché dans un champ. L`homme qui l`a trouvé le cache; et, dans sa joie, il va vendre tout ce qu`il a, et achète ce champ. 
\verse Le royaume des cieux est encore semblable à un marchand qui cherche de belles perles. 
\verse Il a trouvé une perle de grand prix; et il est allé vendre tout ce qu`il avait, et l`a achetée. 
\verse Le royaume des cieux est encore semblable à un filet jeté dans la mer et ramassant des poissons de toute espèce. 
\verse Quand il est rempli, les pêcheurs le tirent; et, après s`être assis sur le rivage, ils mettent dans des vases ce qui est bon, et ils jettent ce qui est mauvais. 
\verse Il en sera de même à la fin du monde. Les anges viendront séparer les méchants d`avec les justes, 
\verse et ils les jetteront dans la fournaise ardente, où il y aura des pleurs et des grincements de dents. 
\verse Avez-vous compris toutes ces choses? -Oui, répondirent-ils. 
\verse Et il leur dit: C`est pourquoi, tout scribe instruit de ce qui regarde le royaume des cieux est semblable à un maître de maison qui tire de son trésor des choses nouvelles et des choses anciennes. 
\verse Lorsque Jésus eut achevé ces paraboles, il partit de là. 
\verse S`étant rendu dans sa patrie, il enseignait dans la synagogue, de sorte que ceux qui l`entendirent étaient étonnés et disaient: D`où lui viennent cette sagesse et ces miracles? 
\verse N`est-ce pas le fils du charpentier? n`est-ce pas Marie qui est sa mère? Jacques, Joseph, Simon et Jude, ne sont-ils pas ses frères? 
\verse et ses soeurs ne sont-elles pas toutes parmi nous? D`où lui viennent donc toutes ces choses? 
\verse Et il était pour eux une occasion de chute. Mais Jésus leur dit: Un prophète n`est méprisé que dans sa patrie et dans sa maison. 
\verse Et il ne fit pas beaucoup de miracles dans ce lieu, à cause de leur incrédulité. 

\chapter[Évangile selon Matthieu]

\chaptermark{Évangile selon Matthieu}{}
\verse En ce temps-là, Hérode le tétrarque, ayant entendu parler de Jésus, dit à ses serviteurs: C`est Jean Baptiste! 
\verse Il est ressuscité des morts, et c`est pour cela qu`il se fait par lui des miracles. 
\verse Car Hérode, qui avait fait arrêter Jean, l`avait lié et mis en prison, à cause d`Hérodias, femme de Philippe, son frère, 
\verse parce que Jean lui disait: Il ne t`est pas permis de l`avoir pour femme. 
\verse Il voulait le faire mourir, mais il craignait la foule, parce qu`elle regardait Jean comme un prophète. 
\verse Or, lorsqu`on célébra l`anniversaire de la naissance d`Hérode, la fille d`Hérodias dansa au milieu des convives, et plut à Hérode, 
\verse de sorte qu`il promit avec serment de lui donner ce qu`elle demanderait. 
\verse A l`instigation de sa mère, elle dit: Donne-moi ici, sur un plat, la tête de Jean Baptiste. 
\verse Le roi fut attristé; mais, à cause de ses serments et des convives, il commanda qu`on la lui donne, 
\verse et il envoya décapiter Jean dans la prison. 
\verse Sa tête fut apportée sur un plat, et donnée à la jeune fille, qui la porta à sa mère. 
\verse Les disciples de Jean vinrent prendre son corps, et l`ensevelirent. Et ils allèrent l`annoncer à Jésus. 
\verse A cette nouvelle, Jésus partit de là dans une barque, pour se retirer à l`écart dans un lieu désert; et la foule, l`ayant su, sortit des villes et le suivit à pied. 
\verse Quand il sortit de la barque, il vit une grande foule, et fut ému de compassion pour elle, et il guérit les malades. 
\verse Le soir étant venu, les disciples s`approchèrent de lui, et dirent: Ce lieu est désert, et l`heure est déjà avancée; renvoie la foule, afin qu`elle aille dans les villages, pour s`acheter des vivres. 
\verse Jésus leur répondit: Ils n`ont pas besoin de s`en aller; donnez-leur vous-mêmes à manger. 
\verse Mais ils lui dirent: Nous n`avons ici que cinq pains et deux poissons. 
\verse Et il dit: Apportez-les-moi. 
\verse Il fit asseoir la foule sur l`herbe, prit les cinq pains et les deux poissons, et, levant les yeux vers le ciel, il rendit grâces. Puis, il rompit les pains et les donna aux disciples, qui les distribuèrent à la foule. 
\verse Tous mangèrent et furent rassasiés, et l`on emporta douze paniers pleins des morceaux qui restaient. 
\verse Ceux qui avaient mangé étaient environ cinq mille hommes, sans les femmes et les enfants. 
\verse Aussitôt après, il obligea les disciples à monter dans la barque et à passer avant lui de l`autre côté, pendant qu`il renverrait la foule. 
\verse Quand il l`eut renvoyée, il monta sur la montagne, pour prier à l`écart; et, comme le soir était venu, il était là seul. 
\verse La barque, déjà au milieu de la mer, était battue par les flots; car le vent était contraire. 
\verse A la quatrième veille de la nuit, Jésus alla vers eux, marchant sur la mer. 
\verse Quand les disciples le virent marcher sur la mer, ils furent troublés, et dirent: C`est un fantôme! Et, dans leur frayeur, ils poussèrent des cris. 
\verse Jésus leur dit aussitôt: Rassurez-vous, c`est moi; n`ayez pas peur! 
\verse Pierre lui répondit: Seigneur, si c`est toi, ordonne que j`aille vers toi sur les eaux. 
\verse Et il dit: Viens! Pierre sortit de la barque, et marcha sur les eaux, pour aller vers Jésus. 
\verse Mais, voyant que le vent était fort, il eut peur; et, comme il commençait à enfoncer, il s`écria: Seigneur, sauve-moi! 
\verse Aussitôt Jésus étendit la main, le saisit, et lui dit: Homme de peu de foi, pourquoi as-tu douté? 
\verse Et ils montèrent dans la barque, et le vent cessa. 
\verse Ceux qui étaient dans la barque vinrent se prosterner devant Jésus, et dirent: Tu es véritablement le Fils de Dieu. 
\verse Après avoir traversé la mer, ils vinrent dans le pays de Génésareth. 
\verse Les gens de ce lieu, ayant reconnu Jésus, envoyèrent des messagers dans tous les environs, et on lui amena tous les malades. 
\verse Ils le prièrent de leur permettre seulement de toucher le bord de son vêtement. Et tous ceux qui le touchèrent furent guéris. 

\chapter[Évangile selon Matthieu]

\chaptermark{Évangile selon Matthieu}{}
\verse Alors des pharisiens et des scribes vinrent de Jérusalem auprès de Jésus, et dirent: 
\verse Pourquoi tes disciples transgressent-ils la tradition des anciens? Car ils ne se lavent pas les mains, quand ils prennent leurs repas. 
\verse Il leur répondit: Et vous, pourquoi transgressez-vous le commandement de Dieu au profit de votre tradition? 
\verse Car Dieu a dit: Honore ton père et ta mère; et: Celui qui maudira son père ou sa mère sera puni de mort. 
\verse Mais vous, vous dites: Celui qui dira à son père ou à sa mère: Ce dont j`aurais pu t`assister est une offrande à Dieu, n`est pas tenu d`honorer son père ou sa mère. 
\verse Vous annulez ainsi la parole de Dieu au profit de votre tradition. 
\verse Hypocrites, Ésaïe a bien prophétisé sur vous, quand il a dit: 
\verse Ce peuple m`honore des lèvres, Mais son coeur est éloigné de moi. 
\verse C`est en vain qu`ils m`honorent, en enseignant des préceptes qui sont des commandements d`hommes. 
\verse Ayant appelé à lui la foule, il lui dit: Écoutez, et comprenez. 
\verse Ce n`est pas ce qui entre dans la bouche qui souille l`homme; mais ce qui sort de la bouche, c`est ce qui souille l`homme. 
\verse Alors ses disciples s`approchèrent, et lui dirent: Sais-tu que les pharisiens ont été scandalisés des paroles qu`ils ont entendues? 
\verse Il répondit: Toute plante que n`a pas plantée mon Père céleste sera déracinée. 
\verse Laissez-les: ce sont des aveugles qui conduisent des aveugles; si un aveugle conduit un aveugle, ils tomberont tous deux dans une fosse. 
\verse Pierre, prenant la parole, lui dit: Explique-nous cette parabole. 
\verse Et Jésus dit: Vous aussi, êtes-vous encore sans intelligence? 
\verse Ne comprenez-vous pas que tout ce qui entre dans la bouche va dans le ventre, puis est jeté dans les lieux secrets? 
\verse Mais ce qui sort de la bouche vient du coeur, et c`est ce qui souille l`homme. 
\verse Car c`est du coeur que viennent les mauvaises pensées, les meurtres, les adultères, les impudicités, les vols, les faux témoignages, les calomnies. 
\verse Voilà les choses qui souillent l`homme; mais manger sans s`être lavé les mains, cela ne souille point l`homme. 
\verse Jésus, étant parti de là, se retira dans le territoire de Tyr et de Sidon. 
\verse Et voici, une femme cananéenne, qui venait de ces contrées, lui cria: Aie pitié de moi, Seigneur, Fils de David! Ma fille est cruellement tourmentée par le démon. 
\verse Il ne lui répondit pas un mot, et ses disciples s`approchèrent, et lui dirent avec insistance: Renvoie-la, car elle crie derrière nous. 
\verse Il répondit: Je n`ai été envoyé qu`aux brebis perdues de la maison d`Israël. 
\verse Mais elle vint se prosterner devant lui, disant: Seigneur, secours-moi! 
\verse Il répondit: Il n`est pas bien de prendre le pain des enfants, et de le jeter aux petits chiens. 
\verse Oui, Seigneur, dit-elle, mais les petits chiens mangent les miettes qui tombent de la table de leurs maîtres. 
\verse Alors Jésus lui dit: Femme, ta foi est grande; qu`il te soit fait comme tu veux. Et, à l`heure même, sa fille fut guérie. 
\verse Jésus quitta ces lieux, et vint près de la mer de Galilée. Étant monté sur la montagne, il s`y assit. 
\verse Alors s`approcha de lui une grande foule, ayant avec elle des boiteux, des aveugles, des muets, des estropiés, et beaucoup d`autres malades. On les mit à ses pieds, et il les guérit; 
\verse en sorte que la foule était dans l`admiration de voir que les muets parlaient, que les estropiés étaient guéris, que les boiteux marchaient, que les aveugles voyaient; et elle glorifiait le Dieu d`Israël. 
\verse Jésus, ayant appelé ses disciples, dit: Je suis ému de compassion pour cette foule; car voilà trois jours qu`ils sont près de moi, et ils n`ont rien à manger. Je ne veux pas les renvoyer à jeun, de peur que les forces ne leur manquent en chemin. 
\verse Les disciples lui dirent: Comment nous procurer dans ce lieu désert assez de pains pour rassasier une si grande foule? 
\verse Jésus leur demanda: Combien avez-vous de pains? Sept, répondirent-ils, et quelques petits poissons. 
\verse Alors il fit asseoir la foule par terre, 
\verse prit les sept pains et les poissons, et, après avoir rendu grâces, il les rompit et les donna à ses disciples, qui les distribuèrent à la foule. 
\verse Tous mangèrent et furent rassasiés, et l`on emporta sept corbeilles pleines des morceaux qui restaient. 
\verse Ceux qui avaient mangé étaient quatre mille hommes, sans les femmes et les enfants. 
\verse Ensuite, il renvoya la foule, monta dans la barque, et se rendit dans la contrée de Magadan. 

\chapter[Évangile selon Matthieu]

\chaptermark{Évangile selon Matthieu}{}
\verse Les pharisiens et les sadducéens abordèrent Jésus et, pour l`éprouver, lui demandèrent de leur faire voir un signe venant du ciel. 
\verse Jésus leur répondit: Le soir, vous dites: Il fera beau, car le ciel est rouge; et le matin: 
\verse Il y aura de l`orage aujourd`hui, car le ciel est d`un rouge sombre. Vous savez discerner l`aspect du ciel, et vous ne pouvez discerner les signes des temps. 
\verse Une génération méchante et adultère demande un miracle; il ne lui sera donné d`autre miracle que celui de Jonas. Puis il les quitta, et s`en alla. 
\verse Les disciples, en passant à l`autre bord, avaient oublié de prendre des pains. 
\verse Jésus leur dit: Gardez-vous avec soin du levain des pharisiens et des sadducéens. 
\verse Les disciples raisonnaient en eux-mêmes, et disaient: C`est parce que nous n`avons pas pris de pains. 
\verse Jésus, l`ayant connu, dit: Pourquoi raisonnez-vous en vous-mêmes, gens de peu de foi, sur ce que vous n`avez pas pris de pains? 
\verse Etes-vous encore sans intelligence, et ne vous rappelez-vous plus les cinq pains des cinq mille hommes et combien de paniers vous avez emportés, 
\verse ni les sept pains des quatre mille hommes et combien de corbeilles vous avez emportées? 
\verse Comment ne comprenez-vous pas que ce n`est pas au sujet de pains que je vous ai parlé? Gardez-vous du levain des pharisiens et des sadducéens. 
\verse Alors ils comprirent que ce n`était pas du levain du pain qu`il avait dit de se garder, mais de l`enseignement des pharisiens et des sadducéens. 
\verse Jésus, étant arrivé dans le territoire de Césarée de Philippe, demanda à ses disciples: Qui dit-on que je suis, moi, le Fils de l`homme? 
\verse Ils répondirent: Les uns disent que tu es Jean Baptiste; les autres, Élie; les autres, Jérémie, ou l`un des prophètes. 
\verse Et vous, leur dit-il, qui dites-vous que je suis? 
\verse Simon Pierre répondit: Tu es le Christ, le Fils du Dieu vivant. 
\verse Jésus, reprenant la parole, lui dit: Tu es heureux, Simon, fils de Jonas; car ce ne sont pas la chair et le sang qui t`ont révélé cela, mais c`est mon Père qui est dans les cieux. 
\verse Et moi, je te dis que tu es Pierre, et que sur cette pierre je bâtirai mon Église, et que les portes du séjour des morts ne prévaudront point contre elle. 
\verse Je te donnerai les clefs du royaume des cieux: ce que tu lieras sur la terre sera lié dans les cieux, et ce que tu délieras sur la terre sera délié dans les cieux. 
\verse Alors il recommanda aux disciples de ne dire à personne qu`il était le Christ. 
\verse Dès lors Jésus commença à faire connaître à ses disciples qu`il fallait qu`il allât à Jérusalem, qu`il souffrît beaucoup de la part des anciens, des principaux sacrificateurs et des scribes, qu`il fût mis à mort, et qu`il ressuscitât le troisième jour. 
\verse Pierre, l`ayant pris à part, se mit à le reprendre, et dit: A Dieu ne plaise, Seigneur! Cela ne t`arrivera pas. 
\verse Mais Jésus, se retournant, dit à Pierre: Arrière de moi, Satan! tu m`es en scandale; car tes pensées ne sont pas les pensées de Dieu, mais celles des hommes. 
\verse Alors Jésus dit à ses disciples: Si quelqu`un veut venir après moi, qu`il renonce à lui-même, qu`il se charge de sa croix, et qu`il me suive. 
\verse Car celui qui voudra sauver sa vie la perdra, mais celui qui la perdra à cause de moi la trouvera. 
\verse Et que servirait-il à un homme de gagner tout le monde, s`il perdait son âme? ou, que donnerait un homme en échange de son âme? 
\verse Car le Fils de l`homme doit venir dans la gloire de son Père, avec ses anges; et alors il rendra à chacun selon ses oeuvres. 
\verse Je vous le dis en vérité, quelques-uns de ceux qui sont ici ne mourront point, qu`ils n`aient vu le Fils de l`homme venir dans son règne. 

\chapter[Évangile selon Matthieu]

\chaptermark{Évangile selon Matthieu}{}
\verse Six jours après, Jésus prit avec lui Pierre, Jacques, et Jean, son frère, et il les conduisit à l`écart sur une haute montagne. 
\verse Il fut transfiguré devant eux; son visage resplendit comme le soleil, et ses vêtements devinrent blancs comme la lumière. 
\verse Et voici, Moïse et Élie leur apparurent, s`entretenant avec lui. 
\verse Pierre, prenant la parole, dit à Jésus: Seigneur, il est bon que nous soyons ici; si tu le veux, je dresserai ici trois tentes, une pour toi, une pour Moïse, et une pour Élie. 
\verse Comme il parlait encore, une nuée lumineuse les couvrit. Et voici, une voix fit entendre de la nuée ces paroles: Celui-ci est mon Fils bien-aimé, en qui j`ai mis toute mon affection: écoutez-le! 
\verse Lorsqu`ils entendirent cette voix, les disciples tombèrent sur leur face, et furent saisis d`une grande frayeur. 
\verse Mais Jésus, s`approchant, les toucha, et dit: Levez-vous, n`ayez pas peur! 
\verse Ils levèrent les yeux, et ne virent que Jésus seul. 
\verse Comme ils descendaient de la montagne, Jésus leur donna cet ordre: Ne parlez à personne de cette vision, jusqu`à ce que le Fils de l`homme soit ressuscité des morts. 
\verse Les disciples lui firent cette question: Pourquoi donc les scribes disent-ils qu`Élie doit venir premièrement? 
\verse Il répondit: Il est vrai qu`Élie doit venir, et rétablir toutes choses. 
\verse Mais je vous dis qu`Élie est déjà venu, qu`ils ne l`ont pas reconnu, et qu`ils l`ont traité comme ils ont voulu. De même le Fils de l`homme souffrira de leur part. 
\verse Les disciples comprirent alors qu`il leur parlait de Jean Baptiste. 
\verse Lorsqu`ils furent arrivés près de la foule, un homme vint se jeter à genoux devant Jésus, et dit: 
\verse Seigneur, aie pitié de mon fils, qui est lunatique, et qui souffre cruellement; il tombe souvent dans le feu, et souvent dans l`eau. 
\verse Je l`ai amené à tes disciples, et ils n`ont pas pu le guérir. 
\verse Race incrédule et perverse, répondit Jésus, jusques à quand serai-je avec vous? jusques à quand vous supporterai-je? Amenez-le-moi ici. 
\verse Jésus parla sévèrement au démon, qui sortit de lui, et l`enfant fut guéri à l`heure même. 
\verse Alors les disciples s`approchèrent de Jésus, et lui dirent en particulier: Pourquoi n`avons-nous pu chasser ce démon? 
\verse C`est à cause de votre incrédulité, leur dit Jésus. Je vous le dis en vérité, si vous aviez de la foi comme un grain de sénevé, vous diriez à cette montagne: Transporte-toi d`ici là, et elle se transporterait; rien ne vous serait impossible. 
\verse Mais cette sorte de démon ne sort que par la prière et par le jeûne. 
\verse Pendant qu`ils parcouraient la Galilée, Jésus leur dit: Le Fils de l`homme doit être livré entre les mains des hommes; 
\verse ils le feront mourir, et le troisième jour il ressuscitera. Ils furent profondément attristés. 
\verse Lorsqu`ils arrivèrent à Capernaüm, ceux qui percevaient les deux drachmes s`adressèrent à Pierre, et lui dirent: Votre maître ne paie-t-il pas les deux drachmes? 
\verse Oui, dit-il. Et quand il fut entré dans la maison, Jésus le prévint, et dit: Que t`en semble, Simon? Les rois de la terre, de qui perçoivent-ils des tributs ou des impôts? de leurs fils, ou des étrangers? 
\verse Il lui dit: Des étrangers. Et Jésus lui répondit: Les fils en sont donc exempts. 
\verse Mais, pour ne pas les scandaliser, va à la mer, jette l`hameçon, et tire le premier poisson qui viendra; ouvre-lui la bouche, et tu trouveras un statère. Prends-le, et donne-le-leur pour moi et pour toi. 

\chapter[Évangile selon Matthieu]

\chaptermark{Évangile selon Matthieu}{}
\verse En ce moment, les disciples s`approchèrent de Jésus, et dirent: Qui donc est le plus grand dans le royaume des cieux? 
\verse Jésus, ayant appelé un petit enfant, le plaça au milieu d`eux, 
\verse et dit: Je vous le dis en vérité, si vous ne vous convertissez et si vous ne devenez comme les petits enfants, vous n`entrerez pas dans le royaume des cieux. 
\verse C`est pourquoi, quiconque se rendra humble comme ce petit enfant sera le plus grand dans le royaume des cieux. 
\verse Et quiconque reçoit en mon nom un petit enfant comme celui-ci, me reçoit moi-même. 
\verse Mais, si quelqu`un scandalisait un de ces petits qui croient en moi, il vaudrait mieux pour lui qu`on suspendît à son cou une meule de moulin, et qu`on le jetât au fond de la mer. 
\verse Malheur au monde à cause des scandales! Car il est nécessaire qu`il arrive des scandales; mais malheur à l`homme par qui le scandale arrive! 
\verse Si ta main ou ton pied est pour toi une occasion de chute, coupe-les et jette-les loin de toi; mieux vaut pour toi entrer dans la vie boiteux ou manchot, que d`avoir deux pieds ou deux mains et d`être jeté dans le feu éternel. 
\verse Et si ton oeil est pour toi une occasion de chute, arrache-le et jette-le loin de toi; mieux vaut pour toi entrer dans la vie, n`ayant qu`un oeil, que d`avoir deux yeux et d`être jeté dans le feu de la géhenne. 
\verse Gardez-vous de mépriser un seul de ces petits; car je vous dis que leurs anges dans les cieux voient continuellement la face de mon Père qui est dans les cieux. 
\verse Car le Fils de l`homme est venu sauver ce qui était perdu. 
\verse Que vous en semble? Si un homme a cent brebis, et que l`une d`elles s`égare, ne laisse-t-il pas les quatre-vingt-dix-neuf autres sur les montagnes, pour aller chercher celle qui s`est égarée? 
\verse Et, s`il la trouve, je vous le dis en vérité, elle lui cause plus de joie que les quatre-vingt-dix-neuf qui ne se sont pas égarées. 
\verse De même, ce n`est pas la volonté de votre Père qui est dans les cieux qu`il se perde un seul de ces petits. 
\verse Si ton frère a péché, va et reprends-le entre toi et lui seul. S`il t`écoute, tu as gagné ton frère. 
\verse Mais, s`il ne t`écoute pas, prends avec toi une ou deux personnes, afin que toute l`affaire se règle sur la déclaration de deux ou de trois témoins. 
\verse S`il refuse de les écouter, dis-le à l`Église; et s`il refuse aussi d`écouter l`Église, qu`il soit pour toi comme un païen et un publicain. 
\verse Je vous le dis en vérité, tout ce que vous lierez sur la terre sera lié dans le ciel, et tout ce que vous délierez sur la terre sera délié dans le ciel. 
\verse Je vous dis encore que, si deux d`entre vous s`accordent sur la terre pour demander une chose quelconque, elle leur sera accordée par mon Père qui est dans les cieux. 
\verse Car là où deux ou trois sont assemblés en mon nom, je suis au milieu d`eux. 
\verse Alors Pierre s`approcha de lui, et dit: Seigneur, combien de fois pardonnerai-je à mon frère, lorsqu`il péchera contre moi? Sera-ce jusqu`à sept fois? 
\verse Jésus lui dit: Je ne te dis pas jusqu`à sept fois, mais jusqu`à septante fois sept fois. 
\verse C`est pourquoi, le royaume des cieux est semblable à un roi qui voulut faire rendre compte à ses serviteurs. 
\verse Quand il se mit à compter, on lui en amena un qui devait dix mille talents. 
\verse Comme il n`avait pas de quoi payer, son maître ordonna qu`il fût vendu, lui, sa femme, ses enfants, et tout ce qu`il avait, et que la dette fût acquittée. 
\verse Le serviteur, se jetant à terre, se prosterna devant lui, et dit: Seigneur, aie patience envers moi, et je te paierai tout. 
\verse Ému de compassion, le maître de ce serviteur le laissa aller, et lui remit la dette. 
\verse Après qu`il fut sorti, ce serviteur rencontra un de ses compagnons qui lui devait cent deniers. Il le saisit et l`étranglait, en disant: Paie ce que tu me dois. 
\verse Son compagnon, se jetant à terre, le suppliait, disant: Aie patience envers moi, et je te paierai. 
\verse Mais l`autre ne voulut pas, et il alla le jeter en prison, jusqu`à ce qu`il eût payé ce qu`il devait. 
\verse Ses compagnons, ayant vu ce qui était arrivé, furent profondément attristés, et ils allèrent raconter à leur maître tout ce qui s`était passé. 
\verse Alors le maître fit appeler ce serviteur, et lui dit: Méchant serviteur, je t`avais remis en entier ta dette, parce que tu m`en avais supplié; 
\verse ne devais-tu pas aussi avoir pitié de ton compagnon, comme j`ai eu pitié de toi? 
\verse Et son maître, irrité, le livra aux bourreaux, jusqu`à ce qu`il eût payé tout ce qu`il devait. 
\verse C`est ainsi que mon Père céleste vous traitera, si chacun de vous ne pardonne à son frère de tout son coeur. 

\chapter[Évangile selon Matthieu]

\chaptermark{Évangile selon Matthieu}{}
\verse Lorsque Jésus eut achevé ces discours, il quitta la Galilée, et alla dans le territoire de la Judée, au delà du Jourdain. 
\verse Une grande foule le suivit, et là il guérit les malades. 
\verse Les pharisiens l`abordèrent, et dirent, pour l`éprouver: Est-il permis à un homme de répudier sa femme pour un motif quelconque? 
\verse Il répondit: N`avez-vous pas lu que le créateur, au commencement, fit l`homme et la femme 
\verse et qu`il dit: C`est pourquoi l`homme quittera son père et sa mère, et s`attachera à sa femme, et les deux deviendront une seule chair? 
\verse Ainsi ils ne sont plus deux, mais ils sont une seule chair. Que l`homme donc ne sépare pas ce que Dieu a joint. 
\verse Pourquoi donc, lui dirent-ils, Moïse a-t-il prescrit de donner à la femme une lettre de divorce et de la répudier? 
\verse Il leur répondit: C`est à cause de la dureté de votre coeur que Moïse vous a permis de répudier vos femmes; au commencement, il n`en était pas ainsi. 
\verse Mais je vous dis que celui qui répudie sa femme, sauf pour infidélité, et qui en épouse une autre, commet un adultère. 
\verse Ses disciples lui dirent: Si telle est la condition de l`homme à l`égard de la femme, il n`est pas avantageux de se marier. 
\verse Il leur répondit: Tous ne comprennent pas cette parole, mais seulement ceux à qui cela est donné. 
\verse Car il y a des eunuques qui le sont dès le ventre de leur mère; il y en a qui le sont devenus par les hommes; et il y en a qui se sont rendus tels eux-mêmes, à cause du royaume des cieux. Que celui qui peut comprendre comprenne. 
\verse Alors on lui amena des petits enfants, afin qu`il leur imposât les mains et priât pour eux. Mais les disciples les repoussèrent. 
\verse Et Jésus dit: Laissez les petits enfants, et ne les empêchez pas de venir à moi; car le royaume des cieux est pour ceux qui leur ressemblent. 
\verse Il leur imposa les mains, et il partit de là. 
\verse Et voici, un homme s`approcha, et dit à Jésus: Maître, que dois-je faire de bon pour avoir la vie éternelle? 
\verse Il lui répondit: Pourquoi m`interroges-tu sur ce qui est bon? Un seul est le bon. Si tu veux entrer dans la vie, observe les commandements. Lesquels? lui dit-il. 
\verse Et Jésus répondit: Tu ne tueras point; tu ne commettras point d`adultère; tu ne déroberas point; tu ne diras point de faux témoignage; honore ton père et ta mère; 
\verse et: tu aimeras ton prochain comme toi-même. 
\verse Le jeune homme lui dit: J`ai observé toutes ces choses; que me manque-t-il encore? 
\verse Jésus lui dit: Si tu veux être parfait, va, vends ce que tu possèdes, donne-le aux pauvres, et tu auras un trésor dans le ciel. Puis viens, et suis-moi. 
\verse Après avoir entendu ces paroles, le jeune homme s`en alla tout triste; car il avait de grands biens. 
\verse Jésus dit à ses disciples: Je vous le dis en vérité, un riche entrera difficilement dans le royaume des cieux. 
\verse Je vous le dis encore, il est plus facile à un chameau de passer par le trou d`une aiguille qu`à un riche d`entrer dans le royaume de Dieu. 
\verse Les disciples, ayant entendu cela, furent très étonnés, et dirent: Qui peut donc être sauvé? 
\verse Jésus les regarda, et leur dit: Aux hommes cela est impossible, mais à Dieu tout est possible. 
\verse Pierre, prenant alors la parole, lui dit: Voici, nous avons tout quitté, et nous t`avons suivi; qu`en sera-t-il pour nous? 
\verse Jésus leur répondit: Je vous le dis en vérité, quand le Fils de l`homme, au renouvellement de toutes choses, sera assis sur le trône de sa gloire, vous qui m`avez suivi, vous serez de même assis sur douze trônes, et vous jugerez les douze tribus d`Israël. 
\verse Et quiconque aura quitté, à cause de mon nom, ses frères, ou ses soeurs, ou son père, ou sa mère, ou sa femme, ou ses enfants, ou ses terres, ou ses maisons, recevra le centuple, et héritera la vie éternelle. 
\verse Plusieurs des premiers seront les derniers, et plusieurs des derniers seront les premiers. 

\chapter[Évangile selon Matthieu]

\chaptermark{Évangile selon Matthieu}{}
\verse Car le royaume des cieux est semblable à un maître de maison qui sortit dès le matin, afin de louer des ouvriers pour sa vigne. 
\verse Il convint avec eux d`un denier par jour, et il les envoya à sa vigne. 
\verse Il sortit vers la troisième heure, et il en vit d`autres qui étaient sur la place sans rien faire. 
\verse Il leur dit: Allez aussi à ma vigne, et je vous donnerai ce qui sera raisonnable. 
\verse Et ils y allèrent. Il sortit de nouveau vers la sixième heure et vers la neuvième, et il fit de même. 
\verse Étant sorti vers la onzième heure, il en trouva d`autres qui étaient sur la place, et il leur dit: Pourquoi vous tenez-vous ici toute la journée sans rien faire? 
\verse Ils lui répondirent: C`est que personne ne nous a loués. Allez aussi à ma vigne, leur dit-il. 
\verse Quand le soir fut venu, le maître de la vigne dit à son intendant: Appelle les ouvriers, et paie-leur le salaire, en allant des derniers aux premiers. 
\verse Ceux de la onzième heure vinrent, et reçurent chacun un denier. 
\verse Les premiers vinrent ensuite, croyant recevoir davantage; mais ils reçurent aussi chacun un denier. 
\verse En le recevant, ils murmurèrent contre le maître de la maison, 
\verse et dirent: Ces derniers n`ont travaillé qu`une heure, et tu les traites à l`égal de nous, qui avons supporté la fatigue du jour et la chaleur. 
\verse Il répondit à l`un d`eux: Mon ami, je ne te fais pas tort; n`es-tu pas convenu avec moi d`un denier? 
\verse Prends ce qui te revient, et va-t`en. Je veux donner à ce dernier autant qu`à toi. 
\verse Ne m`est-il pas permis de faire de mon bien ce que je veux? Ou vois-tu de mauvais oeil que je sois bon? - 
\verse Ainsi les derniers seront les premiers, et les premiers seront les derniers. 
\verse Pendant que Jésus montait à Jérusalem, il prit à part les douze disciples, et il leur dit en chemin: 
\verse Voici, nous montons à Jérusalem, et le Fils de l`homme sera livré aux principaux sacrificateurs et aux scribes. Ils le condamneront à mort, 
\verse et ils le livreront aux païens, pour qu`ils se moquent de lui, le battent de verges, et le crucifient; et le troisième jour il ressuscitera. 
\verse Alors la mère des fils de Zébédée s`approcha de Jésus avec ses fils, et se prosterna, pour lui faire une demande. 
\verse Il lui dit: Que veux-tu? Ordonne, lui dit-elle, que mes deux fils, que voici, soient assis, dans ton royaume, l`un à ta droite et l`autre à ta gauche. 
\verse Jésus répondit: Vous ne savez ce que vous demandez. Pouvez-vous boire la coupe que je dois boire? Nous le pouvons, dirent-ils. 
\verse Et il leur répondit: Il est vrai que vous boirez ma coupe; mais pour ce qui est d`être assis à ma droite et à ma gauche, cela ne dépend pas de moi, et ne sera donné qu`à ceux à qui mon Père l`a réservé. 
\verse Les dix, ayant entendu cela, furent indignés contre les deux frères. 
\verse Jésus les appela, et dit: Vous savez que les chefs des nations les tyrannisent, et que les grands les asservissent. 
\verse Il n`en sera pas de même au milieu de vous. Mais quiconque veut être grand parmi vous, qu`il soit votre serviteur; 
\verse et quiconque veut être le premier parmi vous, qu`il soit votre esclave. 
\verse C`est ainsi que le Fils de l`homme est venu, non pour être servi, mais pour servir et donner sa vie comme la rançon de plusieurs. 
\verse Lorsqu`ils sortirent de Jéricho, une grande foule suivit Jésus. 
\verse Et voici, deux aveugles, assis au bord du chemin, entendirent que Jésus passait, et crièrent: Aie pitié de nous, Seigneur, Fils de David! 
\verse La foule les reprenait, pour les faire taire; mais ils crièrent plus fort: Aie pitié de nous, Seigneur, Fils de David! 
\verse Jésus s`arrêta, les appela, et dit: Que voulez-vous que je vous fasse? 
\verse Ils lui dirent: Seigneur, que nos yeux s`ouvrent. 
\verse Ému de compassion, Jésus toucha leurs yeux; et aussitôt ils recouvrèrent la vue, et le suivirent. 

\chapter[Évangile selon Matthieu]

\chaptermark{Évangile selon Matthieu}{}
\verse Lorsqu`ils approchèrent de Jérusalem, et qu`ils furent arrivés à Bethphagé, vers la montagne des Oliviers, Jésus envoya deux disciples, 
\verse en leur disant: Allez au village qui est devant vous; vous trouverez aussitôt une ânesse attachée, et un ânon avec elle; détachez-les, et amenez-les-moi. 
\verse Si, quelqu`un vous dit quelque chose, vous répondrez: Le Seigneur en a besoin. Et à l`instant il les laissera aller. 
\verse Or, ceci arriva afin que s`accomplît ce qui avait été annoncé par le prophète: 
\verse Dites à la fille de Sion: Voici, ton roi vient à toi, Plein de douceur, et monté sur un âne, Sur un ânon, le petit d`une ânesse. 
\verse Les disciples allèrent, et firent ce que Jésus leur avait ordonné. 
\verse Ils amenèrent l`ânesse et l`ânon, mirent sur eux leurs vêtements, et le firent asseoir dessus. 
\verse La plupart des gens de la foule étendirent leurs vêtements sur le chemin; d`autres coupèrent des branches d`arbres, et en jonchèrent la route. 
\verse Ceux qui précédaient et ceux qui suivaient Jésus criaient: Hosanna au Fils de David! Béni soit celui qui vient au nom du Seigneur! Hosanna dans les lieux très hauts! 
\verse Lorsqu`il entra dans Jérusalem, toute la ville fut émue, et l`on disait: Qui est celui-ci? 
\verse La foule répondait: C`est Jésus, le prophète, de Nazareth en Galilée. 
\verse Jésus entra dans le temple de Dieu. Il chassa tous ceux qui vendaient et qui achetaient dans le temple; il renversa les tables des changeurs, et les sièges des vendeurs de pigeons. 
\verse Et il leur dit: Il est écrit: Ma maison sera appelée une maison de prière. Mais vous, vous en faites une caverne de voleurs. 
\verse Des aveugles et des boiteux s`approchèrent de lui dans le temple. Et il les guérit. 
\verse Mais les principaux sacrificateurs et les scribes furent indignés, à la vue des choses merveilleuses qu`il avait faites, et des enfants qui criaient dans le temple: Hosanna au Fils de David! 
\verse Ils lui dirent: Entends-tu ce qu`ils disent? Oui, leur répondit Jésus. N`avez-vous jamais lu ces paroles: Tu as tiré des louanges de la bouche des enfants et de ceux qui sont à la mamelle? 
\verse Et, les ayant laissés, il sortit de la ville pour aller à Béthanie, où il passa la nuit. 
\verse Le matin, en retournant à la ville, il eut faim. 
\verse Voyant un figuier sur le chemin, il s`en approcha; mais il n`y trouva que des feuilles, et il lui dit: Que jamais fruit ne naisse de toi! Et à l`instant le figuier sécha. 
\verse Les disciples, qui virent cela, furent étonnés, et dirent: Comment ce figuier est-il devenu sec en un instant? 
\verse Jésus leur répondit: Je vous le dis en vérité, si vous aviez de la foi et que vous ne doutiez point, non seulement vous feriez ce qui a été fait à ce figuier, mais quand vous diriez à cette montagne: Ote-toi de là et jette-toi dans la mer, cela se ferait. 
\verse Tout ce que vous demanderez avec foi par la prière, vous le recevrez. 
\verse Jésus se rendit dans le temple, et, pendant qu`il enseignait, les principaux sacrificateurs et les anciens du peuple vinrent lui dire: Par quelle autorité fais-tu ces choses, et qui t`a donné cette autorité? 
\verse Jésus leur répondit: Je vous adresserai aussi une question; et, si vous m`y répondez, je vous dirai par quelle autorité je fais ces choses. 
\verse Le baptême de Jean, d`où venait-il? du ciel, ou des hommes? Mais ils raisonnèrent ainsi entre eux; Si nous répondons: Du ciel, il nous dira: Pourquoi donc n`avez-vous pas cru en lui? 
\verse Et si nous répondons: Des hommes, nous avons à craindre la foule, car tous tiennent Jean pour un prophète. 
\verse Alors ils répondirent à Jésus: Nous ne savons. Et il leur dit à son tour: Moi non plus, je ne vous dirai pas par quelle autorité je fais ces choses. 
\verse Que vous en semble? Un homme avait deux fils; et, s`adressant au premier, il dit: Mon enfant, va travailler aujourd`hui dans ma vigne. 
\verse Il répondit: Je ne veux pas. Ensuite, il se repentit, et il alla. 
\verse S`adressant à l`autre, il dit la même chose. Et ce fils répondit: Je veux bien, seigneur. Et il n`alla pas. 
\verse Lequel des deux a fait la volonté du père? Ils répondirent: Le premier. Et Jésus leur dit: Je vous le dis en vérité, les publicains et les prostituées vous devanceront dans le royaume de Dieu. 
\verse Car Jean est venu à vous dans la voie de la justice, et vous n`avez pas cru en lui. Mais les publicains et les prostituées ont cru en lui; et vous, qui avez vu cela, vous ne vous êtes pas ensuite repentis pour croire en lui. 
\verse Écoutez une autre parabole. Il y avait un homme, maître de maison, qui planta une vigne. Il l`entoura d`une haie, y creusa un pressoir, et bâtit une tour; puis il l`afferma à des vignerons, et quitta le pays. 
\verse Lorsque le temps de la récolte fut arrivé, il envoya ses serviteurs vers les vignerons, pour recevoir le produit de sa vigne. 
\verse Les vignerons, s`étant saisis de ses serviteurs, battirent l`un, tuèrent l`autre, et lapidèrent le troisième. 
\verse Il envoya encore d`autres serviteurs, en plus grand nombre que les premiers; et les vignerons les traitèrent de la même manière. 
\verse Enfin, il envoya vers eux son fils, en disant: Ils auront du respect pour mon fils. 
\verse Mais, quand les vignerons virent le fils, ils dirent entre eux: Voici l`héritier; venez, tuons-le, et emparons-nous de son héritage. 
\verse Et ils se saisirent de lui, le jetèrent hors de la vigne, et le tuèrent. 
\verse Maintenant, lorsque le maître de la vigne viendra, que fera-t-il à ces vignerons? 
\verse Ils lui répondirent: Il fera périr misérablement ces misérables, et il affermera la vigne à d`autres vignerons, qui lui en donneront le produit au temps de la récolte. 
\verse Jésus leur dit: N`avez-vous jamais lu dans les Écritures: La pierre qu`ont rejetée ceux qui bâtissaient Est devenue la principale de l`angle; C`est du Seigneur que cela est venu, Et c`est un prodige à nos yeux? 
\verse C`est pourquoi, je vous le dis, le royaume de Dieu vous sera enlevé, et sera donné à une nation qui en rendra les fruits. 
\verse Celui qui tombera sur cette pierre s`y brisera, et celui sur qui elle tombera sera écrasé. 
\verse Après avoir entendu ses paraboles, les principaux sacrificateurs et les pharisiens comprirent que c`était d`eux que Jésus parlait, 
\verse et ils cherchaient à se saisir de lui; mais ils craignaient la foule, parce qu`elle le tenait pour un prophète. 

\chapter[Évangile selon Matthieu]

\chaptermark{Évangile selon Matthieu}{}
\verse Jésus, prenant la parole, leur parla de nouveau en parabole, et il dit: 
\verse Le royaume des cieux est semblable à un roi qui fit des noces pour son fils. 
\verse Il envoya ses serviteurs appeler ceux qui étaient invités aux noces; mais ils ne voulurent pas venir. 
\verse Il envoya encore d`autres serviteurs, en disant: Dites aux conviés: Voici, j`ai préparé mon festin; mes boeufs et mes bêtes grasses sont tués, tout est prêt, venez aux noces. 
\verse Mais, sans s`inquiéter de l`invitation, ils s`en allèrent, celui-ci à son champ, celui-là à son trafic; 
\verse et les autres se saisirent des serviteurs, les outragèrent et les tuèrent. 
\verse Le roi fut irrité; il envoya ses troupes, fit périr ces meurtriers, et brûla leur ville. 
\verse Alors il dit à ses serviteurs: Les noces sont prêtes; mais les conviés n`en étaient pas dignes. 
\verse Allez donc dans les carrefours, et appelez aux noces tous ceux que vous trouverez. 
\verse Ces serviteurs allèrent dans les chemins, rassemblèrent tous ceux qu`ils trouvèrent, méchants et bons, et la salle des noces fut pleine de convives. 
\verse Le roi entra pour voir ceux qui étaient à table, et il aperçut là un homme qui n`avait pas revêtu un habit de noces. 
\verse Il lui dit: Mon ami, comment es-tu entré ici sans avoir un habit de noces? Cet homme eut la bouche fermée. 
\verse Alors le roi dit aux serviteurs: Liez-lui les pieds et les mains, et jetez-le dans les ténèbres du dehors, où il y aura des pleurs et des grincements de dents. 
\verse Car il y a beaucoup d`appelés, mais peu d`élus. 
\verse Alors les pharisiens allèrent se consulter sur les moyens de surprendre Jésus par ses propres paroles. 
\verse Ils envoyèrent auprès de lui leurs disciples avec les hérodiens, qui dirent: Maître, nous savons que tu es vrai, et que tu enseignes la voie de Dieu selon la vérité, sans t`inquiéter de personne, car tu ne regardes pas à l`apparence des hommes. 
\verse Dis-nous donc ce qu`il t`en semble: est-il permis, ou non, de payer le tribut à César? 
\verse Jésus, connaissant leur méchanceté, répondit: Pourquoi me tentez-vous, hypocrites? 
\verse Montrez-moi la monnaie avec laquelle on paie le tribut. Et ils lui présentèrent un denier. 
\verse Il leur demanda: De qui sont cette effigie et cette inscription? 
\verse De César, lui répondirent-ils. Alors il leur dit: Rendez donc à César ce qui est à César, et à Dieu ce qui est à Dieu. 
\verse Étonnés de ce qu`ils entendaient, ils le quittèrent, et s`en allèrent. 
\verse Le même jour, les sadducéens, qui disent qu`il n`y a point de résurrection, vinrent auprès de Jésus, et lui firent cette question: 
\verse Maître, Moïse a dit: Si quelqu`un meurt sans enfants, son frère épousera sa veuve, et suscitera une postérité à son frère. 
\verse Or, il y avait parmi nous sept frères. Le premier se maria, et mourut; et, comme il n`avait pas d`enfants, il laissa sa femme à son frère. 
\verse Il en fut de même du second, puis du troisième, jusqu`au septième. 
\verse Après eux tous, la femme mourut aussi. 
\verse A la résurrection, duquel des sept sera-t-elle donc la femme? Car tous l`ont eue. 
\verse Jésus leur répondit: Vous êtes dans l`erreur, parce que vous ne comprenez ni les Écritures, ni la puissance de Dieu. 
\verse Car, à la résurrection, les hommes ne prendront point de femmes, ni les femmes de maris, mais ils seront comme les anges de Dieu dans le ciel. 
\verse Pour ce qui est de la résurrection des morts, n`avez-vous pas lu ce que Dieu vous a dit: 
\verse Je suis le Dieu d`Abraham, le Dieu d`Isaac, et le Dieu de Jacob? Dieu n`est pas Dieu des morts, mais des vivants. 
\verse La foule, qui écoutait, fut frappée de l`enseignement de Jésus. 
\verse Les pharisiens, ayant appris qu`il avait réduit au silence les sadducéens, se rassemblèrent, 
\verse et l`un d`eux, docteur de la loi, lui fit cette question, pour l`éprouver: 
\verse Maître, quel est le plus grand commandement de la loi? 
\verse Jésus lui répondit: Tu aimeras le Seigneur, ton Dieu, de tout ton coeur, de toute ton âme, et de toute ta pensée. 
\verse C`est le premier et le plus grand commandement. 
\verse Et voici le second, qui lui est semblable: Tu aimeras ton prochain comme toi-même. 
\verse De ces deux commandements dépendent toute la loi et les prophètes. 
\verse Comme les pharisiens étaient assemblés, Jésus les interrogea, 
\verse en disant: Que pensez-vous du Christ? De qui est-il fils? Ils lui répondirent: De David. 
\verse Et Jésus leur dit: Comment donc David, animé par l`Esprit, l`appelle-t-il Seigneur, lorsqu`il dit: 
\verse Le Seigneur a dit à mon Seigneur: Assieds-toi à ma droite, Jusqu`à ce que je fasse de tes ennemis ton marchepied? 
\verse Si donc David l`appelle Seigneur, comment est-il son fils? 
\verse Nul ne put lui répondre un mot. Et, depuis ce jour, personne n`osa plus lui proposer des questions. 

\chapter[Évangile selon Matthieu]

\chaptermark{Évangile selon Matthieu}{}
\verse Alors Jésus, parlant à la foule et à ses disciples, dit: 
\verse Les scribes et les pharisiens sont assis dans la chaire de Moïse. 
\verse Faites donc et observez tout ce qu`ils vous disent; mais n`agissez pas selon leurs oeuvres. Car ils disent, et ne font pas. 
\verse Ils lient des fardeaux pesants, et les mettent sur les épaules des hommes, mais ils ne veulent pas les remuer du doigt. 
\verse Ils font toutes leurs actions pour être vus des hommes. Ainsi, ils portent de larges phylactères, et ils ont de longues franges à leurs vêtements; 
\verse ils aiment la première place dans les festins, et les premiers sièges dans les synagogues; 
\verse ils aiment à être salués dans les places publiques, et à être appelés par les hommes Rabbi, Rabbi. 
\verse Mais vous, ne vous faites pas appeler Rabbi; car un seul est votre Maître, et vous êtes tous frères. 
\verse Et n`appelez personne sur la terre votre père; car un seul est votre Père, celui qui est dans les cieux. 
\verse Ne vous faites pas appeler directeurs; car un seul est votre Directeur, le Christ. 
\verse Le plus grand parmi vous sera votre serviteur. 
\verse Quiconque s`élèvera sera abaissé, et quiconque s`abaissera sera élevé. 
\verse Malheur à vous, scribes et pharisiens hypocrites! parce que vous fermez aux hommes le royaume des cieux; vous n`y entrez pas vous-mêmes, et vous n`y laissez pas entrer ceux qui veulent entrer. 
\verse Malheur à vous, scribes et pharisiens hypocrites! parce que vous dévorez les maisons des veuves, et que vous faites pour l`apparence de longues prières; à cause de cela, vous serez jugés plus sévèrement. 
\verse Malheur à vous, scribes et pharisiens hypocrites! parce que vous courez la mer et la terre pour faire un prosélyte; et, quand il l`est devenu, vous en faites un fils de la géhenne deux fois plus que vous. 
\verse Malheur à vous, conducteurs aveugles! qui dites: Si quelqu`un jure par le temple, ce n`est rien; mais, si quelqu`un jure par l`or du temple, il est engagé. 
\verse Insensés et aveugles! lequel est le plus grand, l`or, ou le temple qui sanctifie l`or? 
\verse Si quelqu`un, dites-vous encore, jure par l`autel, ce n`est rien; mais, si quelqu`un jure par l`offrande qui est sur l`autel, il est engagé. 
\verse Aveugles! lequel est le plus grand, l`offrande, ou l`autel qui sanctifie l`offrande? 
\verse Celui qui jure par l`autel jure par l`autel et par tout ce qui est dessus; 
\verse celui qui jure par le temple jure par le temple et par celui qui l`habite; 
\verse et celui qui jure par le ciel jure par le trône de Dieu et par celui qui y est assis. 
\verse Malheur à vous, scribes et pharisiens hypocrites! parce que vous payez la dîme de la menthe, de l`aneth et du cumin, et que vous laissez ce qui est plus important dans la loi, la justice, la miséricorde et la fidélité: c`est là ce qu`il fallait pratiquer, sans négliger les autres choses. 
\verse Conducteurs aveugles! qui coulez le moucheron, et qui avalez le chameau. 
\verse Malheur à vous, scribes et pharisiens hypocrites! parce que vous nettoyez le dehors de la coupe et du plat, et qu`au dedans ils sont pleins de rapine et d`intempérance. 
\verse Pharisien aveugle! nettoie premièrement l`intérieur de la coupe et du plat, afin que l`extérieur aussi devienne net. 
\verse Malheur à vous, scribes et pharisiens hypocrites! parce que vous ressemblez à des sépulcres blanchis, qui paraissent beaux au dehors, et qui, au dedans, sont pleins d`ossements de morts et de toute espèce d`impuretés. 
\verse Vous de même, au dehors, vous paraissez justes aux hommes, mais, au dedans, vous êtes pleins d`hypocrisie et d`iniquité. 
\verse Malheur à vous, scribes et pharisiens hypocrites! parce que vous bâtissez les tombeaux des prophètes et ornez les sépulcres des justes, 
\verse et que vous dites: Si nous avions vécu du temps de nos pères, nous ne nous serions pas joints à eux pour répandre le sang des prophètes. 
\verse Vous témoignez ainsi contre vous-mêmes que vous êtes les fils de ceux qui ont tué les prophètes. 
\verse Comblez donc la mesure de vos pères. 
\verse Serpents, race de vipères! comment échapperez-vous au châtiment de la géhenne? 
\verse C`est pourquoi, voici, je vous envoie des prophètes, des sages et des scribes. Vous tuerez et crucifierez les uns, vous battrez de verges les autres dans vos synagogues, et vous les persécuterez de ville en ville, 
\verse afin que retombe sur vous tout le sang innocent répandu sur la terre, depuis le sang d`Abel le juste jusqu`au sang de Zacharie, fils de Barachie, que vous avez tué entre le temple et l`autel. 
\verse Je vous le dis en vérité, tout cela retombera sur cette génération. 
\verse Jérusalem, Jérusalem, qui tues les prophètes et qui lapides ceux qui te sont envoyés, combien de fois ai-je voulu rassembler tes enfants, comme une poule rassemble ses poussins sous ses ailes, et vous ne l`avez pas voulu! 
\verse Voici, votre maison vous sera laissée déserte; 
\verse car, je vous le dis, vous ne me verrez plus désormais, jusqu`à ce que vous disiez: Béni soit celui qui vient au nom du Seigneur! 

\chapter[Évangile selon Matthieu]

\chaptermark{Évangile selon Matthieu}{}
\verse Comme Jésus s`en allait, au sortir du temple, ses disciples s`approchèrent pour lui en faire remarquer les constructions. 
\verse Mais il leur dit: Voyez-vous tout cela? Je vous le dis en vérité, il ne restera pas ici pierre sur pierre qui ne soit renversée. 
\verse Il s`assit sur la montagne des oliviers. Et les disciples vinrent en particulier lui faire cette question: Dis-nous, quand cela arrivera-t-il, et quel sera le signe de ton avènement et de la fin du monde? 
\verse Jésus leur répondit: Prenez garde que personne ne vous séduise. 
\verse Car plusieurs viendront sous mon nom, disant: C`est moi qui suis le Christ. Et ils séduiront beaucoup de gens. 
\verse Vous entendrez parler de guerres et de bruits de guerres: gardez-vous d`être troublés, car il faut que ces choses arrivent. Mais ce ne sera pas encore la fin. 
\verse Une nation s`élèvera contre une nation, et un royaume contre un royaume, et il y aura, en divers lieux, des famines et des tremblements de terre. 
\verse Tout cela ne sera que le commencement des douleurs. 
\verse Alors on vous livrera aux tourments, et l`on vous fera mourir; et vous serez haïs de toutes les nations, à cause de mon nom. 
\verse Alors aussi plusieurs succomberont, et ils se trahiront, se haïront les uns les autres. 
\verse Plusieurs faux prophètes s`élèveront, et ils séduiront beaucoup de gens. 
\verse Et, parce que l`iniquité se sera accrue, la charité du plus grand nombre se refroidira. 
\verse Mais celui qui persévérera jusqu`à la fin sera sauvé. 
\verse Cette bonne nouvelle du royaume sera prêchée dans le monde entier, pour servir de témoignage à toutes les nations. Alors viendra la fin. 
\verse C`est pourquoi, lorsque vous verrez l`abomination de la désolation, dont a parlé le prophète Daniel, établie en lieu saint, -que celui qui lit fasse attention! - 
\verse alors, que ceux qui seront en Judée fuient dans les montagnes; 
\verse que celui qui sera sur le toit ne descende pas pour prendre ce qui est dans sa maison; 
\verse et que celui qui sera dans les champs ne retourne pas en arrière pour prendre son manteau. 
\verse Malheur aux femmes qui seront enceintes et à celles qui allaiteront en ces jours-là! 
\verse Priez pour que votre fuite n`arrive pas en hiver, ni un jour de sabbat. 
\verse Car alors, la détresse sera si grande qu`il n`y en a point eu de pareille depuis le commencement du monde jusqu`à présent, et qu`il n`y en aura jamais. 
\verse Et, si ces jours n`étaient abrégés, personne ne serait sauvé; mais, à cause des élus, ces jours seront abrégés. 
\verse Si quelqu`un vous dit alors: Le Christ est ici, ou: Il est là, ne le croyez pas. 
\verse Car il s`élèvera de faux Christs et de faux prophètes; ils feront de grands prodiges et des miracles, au point de séduire, s`il était possible, même les élus. 
\verse Voici, je vous l`ai annoncé d`avance. 
\verse Si donc on vous dit: Voici, il est dans le désert, n`y allez pas; voici, il est dans les chambres, ne le croyez pas. 
\verse Car, comme l`éclair part de l`orient et se montre jusqu`en occident, ainsi sera l`avènement du Fils de l`homme. 
\verse En quelque lieu que soit le cadavre, là s`assembleront les aigles. 
\verse Aussitôt après ces jours de détresse, le soleil s`obscurcira, la lune ne donnera plus sa lumière, les étoiles tomberont du ciel, et les puissances des cieux seront ébranlées. 
\verse Alors le signe du Fils de l`homme paraîtra dans le ciel, toutes les tribus de la terre se lamenteront, et elles verront le Fils de l`homme venant sur les nuées du ciel avec puissance et une grande gloire. 
\verse Il enverra ses anges avec la trompette retentissante, et ils rassembleront ses élus des quatre vents, depuis une extrémité des cieux jusqu`à l`autre. 
\verse Instruisez-vous par une comparaison tirée du figuier. Dès que ses branches deviennent tendres, et que les feuilles poussent, vous connaissez que l`été est proche. 
\verse De même, quand vous verrez toutes ces choses, sachez que le Fils de l`homme est proche, à la porte. 
\verse Je vous le dis en vérité, cette génération ne passera point, que tout cela n`arrive. 
\verse Le ciel et la terre passeront, mais mes paroles ne passeront point. 
\verse Pour ce qui est du jour et de l`heure, personne ne le sait, ni les anges des cieux, ni le Fils, mais le Père seul. 
\verse Ce qui arriva du temps de Noé arrivera de même à l`avènement du Fils de l`homme. 
\verse Car, dans les jours qui précédèrent le déluge, les hommes mangeaient et buvaient, se mariaient et mariaient leurs enfants, jusqu`au jour où Noé entra dans l`arche; 
\verse et ils ne se doutèrent de rien, jusqu`à ce que le déluge vînt et les emportât tous: il en sera de même à l`avènement du Fils de l`homme. 
\verse Alors, de deux hommes qui seront dans un champ, l`un sera pris et l`autre laissé; 
\verse de deux femmes qui moudront à la meule, l`une sera prise et l`autre laissée. 
\verse Veillez donc, puisque vous ne savez pas quel jour votre Seigneur viendra. 
\verse Sachez-le bien, si le maître de la maison savait à quelle veille de la nuit le voleur doit venir, il veillerait et ne laisserait pas percer sa maison. 
\verse C`est pourquoi, vous aussi, tenez-vous prêts, car le Fils de l`homme viendra à l`heure où vous n`y penserez pas. 
\verse Quel est donc le serviteur fidèle et prudent, que son maître a établi sur ses gens, pour leur donner la nourriture au temps convenable? 
\verse Heureux ce serviteur, que son maître, à son arrivée, trouvera faisant ainsi! 
\verse Je vous le dis en vérité, il l`établira sur tous ses biens. 
\verse Mais, si c`est un méchant serviteur, qui dise en lui-même: Mon maître tarde à venir, 
\verse s`il se met à battre ses compagnons, s`il mange et boit avec les ivrognes, 
\verse le maître de ce serviteur viendra le jour où il ne s`y attend pas et à l`heure qu`il ne connaît pas, 
\verse il le mettra en pièces, et lui donnera sa part avec les hypocrites: c`est là qu`il y aura des pleurs et des grincements de dents. 

\chapter[Évangile selon Matthieu]

\chaptermark{Évangile selon Matthieu}{}
\verse Alors le royaume des cieux sera semblable à dix vierges qui, ayant pris leurs lampes, allèrent à la rencontre de l`époux. 
\verse Cinq d`entre elles étaient folles, et cinq sages. 
\verse Les folles, en prenant leurs lampes, ne prirent point d`huile avec elles; 
\verse mais les sages prirent, avec leurs lampes, de l`huile dans des vases. 
\verse Comme l`époux tardait, toutes s`assoupirent et s`endormirent. 
\verse Au milieu de la nuit, on cria: Voici l`époux, allez à sa rencontre! 
\verse Alors toutes ces vierges se réveillèrent, et préparèrent leurs lampes. 
\verse Les folles dirent aux sages: Donnez-nous de votre huile, car nos lampes s`éteignent. 
\verse Les sages répondirent: Non; il n`y en aurait pas assez pour nous et pour vous; allez plutôt chez ceux qui en vendent, et achetez-en pour vous. 
\verse Pendant qu`elles allaient en acheter, l`époux arriva; celles qui étaient prêtes entrèrent avec lui dans la salle des noces, et la porte fut fermée. 
\verse Plus tard, les autres vierges vinrent, et dirent: Seigneur, Seigneur, ouvre-nous. 
\verse Mais il répondit: Je vous le dis en vérité, je ne vous connais pas. 
\verse Veillez donc, puisque vous ne savez ni le jour, ni l`heure. 
\verse Il en sera comme d`un homme qui, partant pour un voyage, appela ses serviteurs, et leur remit ses biens. 
\verse Il donna cinq talents à l`un, deux à l`autre, et un au troisième, à chacun selon sa capacité, et il partit. 
\verse Aussitôt celui qui avait reçu les cinq talents s`en alla, les fit valoir, et il gagna cinq autres talents. 
\verse De même, celui qui avait reçu les deux talents en gagna deux autres. 
\verse Celui qui n`en avait reçu qu`un alla faire un creux dans la terre, et cacha l`argent de son maître. 
\verse Longtemps après, le maître de ces serviteurs revint, et leur fit rendre compte. 
\verse Celui qui avait reçu les cinq talents s`approcha, en apportant cinq autres talents, et il dit: Seigneur, tu m`as remis cinq talents; voici, j`en ai gagné cinq autres. 
\verse Son maître lui dit: C`est bien, bon et fidèle serviteur; tu as été fidèle en peu de chose, je te confierai beaucoup; entre dans la joie de ton maître. 
\verse Celui qui avait reçu les deux talents s`approcha aussi, et il dit: Seigneur, tu m`as remis deux talents; voici, j`en ai gagné deux autres. 
\verse Son maître lui dit: C`est bien, bon et fidèle serviteur; tu as été fidèle en peu de chose, je te confierai beaucoup; entre dans la joie de ton maître. 
\verse Celui qui n`avait reçu qu`un talent s`approcha ensuite, et il dit: Seigneur, je savais que tu es un homme dur, qui moissonnes où tu n`as pas semé, et qui amasses où tu n`as pas vanné; 
\verse j`ai eu peur, et je suis allé cacher ton talent dans la terre; voici, prends ce qui est à toi. 
\verse Son maître lui répondit: Serviteur méchant et paresseux, tu savais que je moissonne où je n`ai pas semé, et que j`amasse où je n`ai pas vanné; 
\verse il te fallait donc remettre mon argent aux banquiers, et, à mon retour, j`aurais retiré ce qui est à moi avec un intérêt. 
\verse Otez-lui donc le talent, et donnez-le à celui qui a les dix talents. 
\verse Car on donnera à celui qui a, et il sera dans l`abondance, mais à celui qui n`a pas on ôtera même ce qu`il a. 
\verse Et le serviteur inutile, jetez-le dans les ténèbres du dehors, où il y aura des pleurs et des grincements de dents. 
\verse Lorsque le Fils de l`homme viendra dans sa gloire, avec tous les anges, il s`assiéra sur le trône de sa gloire. 
\verse Toutes les nations seront assemblées devant lui. Il séparera les uns d`avec les autres, comme le berger sépare les brebis d`avec les boucs; 
\verse et il mettra les brebis à sa droite, et les boucs à sa gauche. 
\verse Alors le roi dira à ceux qui seront à sa droite: Venez, vous qui êtes bénis de mon Père; prenez possession du royaume qui vous a été préparé dès la fondation du monde. 
\verse Car j`ai eu faim, et vous m`avez donné à manger; j`ai eu soif, et vous m`avez donné à boire; j`étais étranger, et vous m`avez recueilli; 
\verse j`étais nu, et vous m`avez vêtu; j`étais malade, et vous m`avez visité; j`étais en prison, et vous êtes venus vers moi. 
\verse Les justes lui répondront: Seigneur, quand t`avons-nous vu avoir faim, et t`avons-nous donné à manger; ou avoir soif, et t`avons-nous donné à boire? 
\verse Quand t`avons-nous vu étranger, et t`avons-nous recueilli; ou nu, et t`avons-nous vêtu? 
\verse Quand t`avons-nous vu malade, ou en prison, et sommes-nous allés vers toi? 
\verse Et le roi leur répondra: Je vous le dis en vérité, toutes les fois que vous avez fait ces choses à l`un de ces plus petits de mes frères, c`est à moi que vous les avez faites. 
\verse Ensuite il dira à ceux qui seront à sa gauche: Retirez-vous de moi, maudits; allez dans le feu éternel qui a été préparé pour le diable et pour ses anges. 
\verse Car j`ai eu faim, et vous ne m`avez pas donné à manger; j`ai eu soif, et vous ne m`avez pas donné à boire; 
\verse j`étais étranger, et vous ne m`avez pas recueilli; j`étais nu, et vous ne m`avez pas vêtu; j`étais malade et en prison, et vous ne m`avez pas visité. 
\verse Ils répondront aussi: Seigneur, quand t`avons-nous vu ayant faim, ou ayant soif, ou étranger, ou nu, ou malade, ou en prison, et ne t`avons-nous pas assisté? 
\verse Et il leur répondra: Je vous le dis en vérité, toutes les fois que vous n`avez pas fait ces choses à l`un de ces plus petits, c`est à moi que vous ne les avez pas faites. 
\verse Et ceux-ci iront au châtiment éternel, mais les justes à la vie éternelle. 

\chapter[Évangile selon Matthieu]

\chaptermark{Évangile selon Matthieu}{}
\verse Lorsque Jésus eut achevé tous ces discours, il dit à ses disciples: 
\verse Vous savez que la Pâque a lieu dans deux jours, et que le Fils de l`homme sera livré pour être crucifié. 
\verse Alors les principaux sacrificateurs et les anciens du peuple se réunirent dans la cour du souverain sacrificateur, appelé Caïphe; 
\verse et ils délibérèrent sur les moyens d`arrêter Jésus par ruse, et de le faire mourir. 
\verse Mais ils dirent: Que ce ne soit pas pendant la fête, afin qu`il n`y ait pas de tumulte parmi le peuple. 
\verse Comme Jésus était à Béthanie, dans la maison de Simon le lépreux, 
\verse une femme s`approcha de lui, tenant un vase d`albâtre, qui renfermait un parfum de grand prix; et, pendant qu`il était à table, elle répandit le parfum sur sa tête. 
\verse Les disciples, voyant cela, s`indignèrent, et dirent: A quoi bon cette perte? 
\verse On aurait pu vendre ce parfum très cher, et en donner le prix aux pauvres. 
\verse Jésus, s`en étant aperçu, leur dit: Pourquoi faites-vous de la peine à cette femme? Elle a fait une bonne action à mon égard; 
\verse car vous avez toujours des pauvres avec vous, mais vous ne m`avez pas toujours. 
\verse En répandant ce parfum sur mon corps, elle l`a fait pour ma sépulture. 
\verse Je vous le dis en vérité, partout où cette bonne nouvelle sera prêchée, dans le monde entier, on racontera aussi en mémoire de cette femme ce qu`elle a fait. 
\verse Alors l`un des douze, appelé Judas Iscariot, alla vers les principaux sacrificateurs, 
\verse et dit: Que voulez-vous me donner, et je vous le livrerai? Et ils lui payèrent trente pièces d`argent. 
\verse Depuis ce moment, il cherchait une occasion favorable pour livrer Jésus. 
\verse Le premier jour des pains sans levain, les disciples s`adressèrent à Jésus, pour lui dire: Où veux-tu que nous te préparions le repas de la Pâque? 
\verse Il répondit: Allez à la ville chez un tel, et vous lui direz: Le maître dit: Mon temps est proche; je ferai chez toi la Pâque avec mes disciples. 
\verse Les disciples firent ce que Jésus leur avait ordonné, et ils préparèrent la Pâque. 
\verse Le soir étant venu, il se mit à table avec les douze. 
\verse Pendant qu`ils mangeaient, il dit: Je vous le dis en vérité, l`un de vous me livrera. 
\verse Ils furent profondément attristés, et chacun se mit à lui dire: Est-ce moi, Seigneur? 
\verse Il répondit: Celui qui a mis avec moi la main dans le plat, c`est celui qui me livrera. 
\verse Le Fils de l`homme s`en va, selon ce qui est écrit de lui. Mais malheur à l`homme par qui le Fils de l`homme est livré! Mieux vaudrait pour cet homme qu`il ne fût pas né. 
\verse Judas, qui le livrait, prit la parole et dit: Est-ce moi, Rabbi? Jésus lui répondit: Tu l`as dit. 
\verse Pendant qu`ils mangeaient, Jésus prit du pain; et, après avoir rendu grâces, il le rompit, et le donna aux disciples, en disant: Prenez, mangez, ceci est mon corps. 
\verse Il prit ensuite une coupe; et, après avoir rendu grâces, il la leur donna, en disant: Buvez-en tous; 
\verse car ceci est mon sang, le sang de l`alliance, qui est répandu pour plusieurs, pour la rémission des péchés. 
\verse Je vous le dis, je ne boirai plus désormais de ce fruit de la vigne, jusqu`au jour où j`en boirai du nouveau avec vous dans le royaume de mon Père. 
\verse Après voir chanté les cantiques, ils se rendirent à la montagne des oliviers. 
\verse Alors Jésus leur dit: Je serai pour vous tous, cette nuit, une occasion de chute; car il est écrit: Je frapperai le berger, et les brebis du troupeau seront dispersées. 
\verse Mais, après que je serai ressuscité, je vous précèderai en Galilée. 
\verse Pierre, prenant la parole, lui dit: Quand tu serais pour tous une occasion de chute, tu ne le seras jamais pour moi. 
\verse Jésus lui dit: Je te le dis en vérité, cette nuit même, avant que le coq chante, tu me renieras trois fois. 
\verse Pierre lui répondit: Quand il me faudrait mourir avec toi, je ne te renierai pas. Et tous les disciples dirent la même chose. 
\verse Là-dessus, Jésus alla avec eux dans un lieu appelé Gethsémané, et il dit aux disciples: Asseyez-vous ici, pendant que je m`éloignerai pour prier. 
\verse Il prit avec lui Pierre et les deux fils de Zébédée, et il commença à éprouver de la tristesse et des angoisses. 
\verse Il leur dit alors: Mon âme est triste jusqu`à la mort; restez ici, et veillez avec moi. 
\verse Puis, ayant fait quelques pas en avant, il se jeta sur sa face, et pria ainsi: Mon Père, s`il est possible, que cette coupe s`éloigne de moi! Toutefois, non pas ce que je veux, mais ce que tu veux. 
\verse Et il vint vers les disciples, qu`il trouva endormis, et il dit à Pierre: Vous n`avez donc pu veiller une heure avec moi! 
\verse Veillez et priez, afin que vous ne tombiez pas dans la tentation; l`esprit es bien disposé, mais la chair est faible. 
\verse Il s`éloigna une seconde fois, et pria ainsi: Mon Père, s`il n`est pas possible que cette coupe s`éloigne sans que je la boive, que ta volonté soit faite! 
\verse Il revint, et les trouva encore endormis; car leurs yeux étaient appesantis. 
\verse Il les quitta, et, s`éloignant, il pria pour la troisième fois, répétant les mêmes paroles. 
\verse Puis il alla vers ses disciples, et leur dit: Vous dormez maintenant, et vous vous reposez! Voici, l`heure est proche, et le Fils de l`homme est livré aux mains des pécheurs. 
\verse Levez-vous, allons; voici, celui qui me livre s`approche. 
\verse Comme il parlait encore, voici, Judas, l`un des douze, arriva, et avec lui une foule nombreuse armée d`épées et de bâtons, envoyée par les principaux sacrificateurs et par les anciens du peuple. 
\verse Celui qui le livrait leur avait donné ce signe: Celui que je baiserai, c`est lui; saisissez-le. 
\verse Aussitôt, s`approchant de Jésus, il dit: Salut, Rabbi! Et il le baisa. 
\verse Jésus lui dit: Mon ami, ce que tu es venu faire, fais-le. Alors ces gens s`avancèrent, mirent la main sur Jésus, et le saisirent. 
\verse Et voici, un de ceux qui étaient avec Jésus étendit la main, et tira son épée; il frappa le serviteur du souverain sacrificateur, et lui emporta l`oreille. 
\verse Alors Jésus lui dit: Remets ton épée à sa place; car tous ceux qui prendront l`épée périront par l`épée. 
\verse Penses-tu que je ne puisse pas invoquer mon Père, qui me donnerait à l`instant plus de douze légions d`anges? 
\verse Comment donc s`accompliraient les Écritures, d`après lesquelles il doit en être ainsi? 
\verse En ce moment, Jésus dit à la foule: Vous êtes venus, comme après un brigand, avec des épées et des bâtons, pour vous emparer de moi. J`étais tous les jours assis parmi vous, enseignant dans le temple, et vous ne m`avez pas saisi. 
\verse Mais tout cela est arrivé afin que les écrits des prophètes fussent accomplis. Alors tous les disciples l`abandonnèrent, et prirent la fuite. 
\verse Ceux qui avaient saisi Jésus l`emmenèrent chez le souverain sacrificateur Caïphe, où les scribes et les anciens étaient assemblés. 
\verse Pierre le suivit de loin jusqu`à la cour du souverain sacrificateur, y entra, et s`assit avec les serviteurs, pour voir comment cela finirait. 
\verse Les principaux sacrificateurs et tout le sanhédrin cherchaient quelque faux témoignage contre Jésus, suffisant pour le faire mourir. 
\verse Mais ils n`en trouvèrent point, quoique plusieurs faux témoins se fussent présentés. Enfin, il en vint deux, qui dirent: 
\verse Celui-ci a dit: Je puis détruire le temple de Dieu, et le rebâtir en trois jours. 
\verse Le souverain sacrificateur se leva, et lui dit: Ne réponds-tu rien? Qu`est-ce que ces hommes déposent contre toi? 
\verse Jésus garda le silence. Et le souverain sacrificateur, prenant la parole, lui dit: Je t`adjure, par le Dieu vivant, de nous dire si tu es le Christ, le Fils de Dieu. 
\verse Jésus lui répondit: Tu l`as dit. De plus, je vous le déclare, vous verrez désormais le Fils de l`homme assis à la droite de la puissance de Dieu, et venant sur les nuées du ciel. 
\verse Alors le souverain sacrificateur déchira ses vêtements, disant: Il a blasphémé! Qu`avons-nous encore besoin de témoins? Voici, vous venez d`entendre son blasphème. Que vous en semble? 
\verse Ils répondirent: Il mérite la mort. 
\verse Là-dessus, ils lui crachèrent au visage, et lui donnèrent des coups de poing et des soufflets en disant: 
\verse Christ, prophétise; dis-nous qui t`a frappé. 
\verse Cependant, Pierre était assis dehors dans la cour. Une servante s`approcha de lui, et dit: Toi aussi, tu étais avec Jésus le Galiléen. 
\verse Mais il le nia devant tous, disant: Je ne sais ce que tu veux dire. 
\verse Comme il se dirigeait vers la porte, une autre servante le vit, et dit à ceux qui se trouvaient là; Celui-ci était aussi avec Jésus de Nazareth. 
\verse Il le nia de nouveau, avec serment: Je ne connais pas cet homme. 
\verse Peu après, ceux qui étaient là, s`étant approchés, dirent à Pierre: Certainement tu es aussi de ces gens-là, car ton langage te fait reconnaître. 
\verse Alors il se mit à faire des imprécations et à jurer: Je ne connais pas cet homme. Aussitôt le coq chanta. 
\verse Et Pierre se souvint de la parole que Jésus avait dite: Avant que le coq chante, tu me renieras trois fois. Et étant sorti, il pleura amèrement. 

\chapter[Évangile selon Matthieu]

\chaptermark{Évangile selon Matthieu}{}
\verse Dès que le matin fut venu, tous les principaux sacrificateurs et les anciens du peuple tinrent conseil contre Jésus, pour le faire mourir. 
\verse Après l`avoir lié, ils l`emmenèrent, et le livrèrent à Ponce Pilate, le gouverneur. 
\verse Alors Judas, qui l`avait livré, voyant qu`il était condamné, se repentit, et rapporta les trente pièces d`argent aux principaux sacrificateurs et aux anciens, 
\verse en disant: J`ai péché, en livrant le sang innocent. Ils répondirent: Que nous importe? Cela te regarde. 
\verse Judas jeta les pièces d`argent dans le temple, se retira, et alla se pendre. 
\verse Les principaux sacrificateurs les ramassèrent, et dirent: Il n`est pas permis de les mettre dans le trésor sacré, puisque c`est le prix du sang. 
\verse Et, après en avoir délibéré, ils achetèrent avec cet argent le champ du potier, pour la sépulture des étrangers. 
\verse C`est pourquoi ce champ a été appelé champ du sang, jusqu`à ce jour. 
\verse Alors s`accomplit ce qui avait été annoncé par Jérémie, le prophète: Ils ont pris les trente pièces d`argent, la valeur de celui qui a été estimé, qu`on a estimé de la part des enfants d`Israël; 
\verse et il les ont données pour le champ du potier, comme le Seigneur me l`avait ordonné. 
\verse Jésus comparut devant le gouverneur. Le gouverneur l`interrogea, en ces termes: Es-tu le roi des Juifs? Jésus lui répondit: Tu le dis. 
\verse Mais il ne répondit rien aux accusations des principaux sacrificateurs et des anciens. 
\verse Alors Pilate lui dit: N`entends-tu pas de combien de choses ils t`accusent? 
\verse Et Jésus ne lui donna de réponse sur aucune parole, ce qui étonna beaucoup le gouverneur. 
\verse A chaque fête, le gouverneur avait coutume de relâcher un prisonnier, celui que demandait la foule. 
\verse Ils avaient alors un prisonnier fameux, nommé Barabbas. 
\verse Comme ils étaient assemblés, Pilate leur dit: Lequel voulez-vous que je vous relâche, Barabbas, ou Jésus, qu`on appelle Christ? 
\verse Car il savait que c`était par envie qu`ils avaient livré Jésus. 
\verse Pendant qu`il était assis sur le tribunal, sa femme lui fit dire: Qu`il n`y ait rien entre toi et ce juste; car aujourd`hui j`ai beaucoup souffert en songe à cause de lui. 
\verse Les principaux sacrificateurs et les anciens persuadèrent à la foule de demander Barabbas, et de faire périr Jésus. 
\verse Le gouverneur prenant la parole, leur dit: Lequel des deux voulez-vous que je vous relâche? Ils répondirent: Barabbas. 
\verse Pilate leur dit: Que ferai-je donc de Jésus, qu`on appelle Christ? Tous répondirent: Qu`il soit crucifié! 
\verse Le gouverneur dit: Mais quel mal a-t-il fait? Et ils crièrent encore plus fort: Qu`il soit crucifié! 
\verse Pilate, voyant qu`il ne gagnait rien, mais que le tumulte augmentait, prit de l`eau, se lava les mains en présence de la foule, et dit: Je suis innocent du sang de ce juste. Cela vous regarde. 
\verse Et tout le peuple répondit: Que son sang retombe sur nous et sur nos enfants! 
\verse Alors Pilate leur relâcha Barabbas; et, après avoir fait battre de verges Jésus, il le livra pour être crucifié. 
\verse Les soldats du gouverneur conduisirent Jésus dans le prétoire, et ils assemblèrent autour de lui toute la cohorte. 
\verse Ils lui ôtèrent ses vêtements, et le couvrirent d`un manteau écarlate. 
\verse Ils tressèrent une couronne d`épines, qu`ils posèrent sur sa tête, et ils lui mirent un roseau dans la main droite; puis, s`agenouillant devant lui, ils le raillaient, en disant: Salut, roi des Juifs! 
\verse Et ils crachaient contre lui, prenaient le roseau, et frappaient sur sa tête. 
\verse Après s`être ainsi moqués de lui, ils lui ôtèrent le manteau, lui remirent ses vêtements, et l`emmenèrent pour le crucifier. 
\verse Lorsqu`ils sortirent, ils rencontrèrent un homme de Cyrène, appelé Simon, et ils le forcèrent à porter la croix de Jésus. 
\verse Arrivés au lieu nommé Golgotha, ce qui signifie lieu du crâne, 
\verse ils lui donnèrent à boire du vin mêlé de fiel; mais, quand il l`eut goûté, il ne voulut pas boire. 
\verse Après l`avoir crucifié, ils se partagèrent ses vêtements, en tirant au sort, afin que s`accomplît ce qui avait été annoncé par le prophète: Ils se sont partagé mes vêtements, et ils ont tiré au sort ma tunique. 
\verse Puis ils s`assirent, et le gardèrent. 
\verse Pour indiquer le sujet de sa condamnation, on écrivit au-dessus de sa tête: Celui-ci est Jésus, le roi des Juifs. 
\verse Avec lui furent crucifiés deux brigands, l`un à sa droite, et l`autre à sa gauche. 
\verse Les passants l`injuriaient, et secouaient la tête, 
\verse en disant: Toi qui détruis le temple, et qui le rebâtis en trois jours, sauve-toi toi-même! Si tu es le Fils de Dieu, descends de la croix! 
\verse Les principaux sacrificateurs, avec les scribes et les anciens, se moquaient aussi de lui, et disaient: 
\verse Il a sauvé les autres, et il ne peut se sauver lui-même! S`il est roi d`Israël, qu`il descende de la croix, et nous croirons en lui. 
\verse Il s`est confié en Dieu; que Dieu le délivre maintenant, s`il l`aime. Car il a dit: Je suis Fils de Dieu. 
\verse Les brigands, crucifiés avec lui, l`insultaient de la même manière. 
\verse Depuis la sixième heure jusqu`à la neuvième, il y eut des ténèbres sur toute la terre. 
\verse Et vers la neuvième heure, Jésus s`écria d`une voix forte: Éli, Éli, lama sabachthani? c`est-à-dire: Mon Dieu, mon Dieu, pourquoi m`as-tu abandonné? 
\verse Quelques-un de ceux qui étaient là, l`ayant entendu, dirent: Il appelle Élie. 
\verse Et aussitôt l`un d`eux courut prendre une éponge, qu`il remplit de vinaigre, et, l`ayant fixée à un roseau, il lui donna à boire. 
\verse Mais les autres disaient: Laisse, voyons si Élie viendra le sauver. 
\verse Jésus poussa de nouveau un grand cri, et rendit l`esprit. 
\verse Et voici, le voile du temple se déchira en deux, depuis le haut jusqu`en bas, la terre trembla, les rochers se fendirent, 
\verse les sépulcres s`ouvrirent, et plusieurs corps des saints qui étaient morts ressuscitèrent. 
\verse Étant sortis des sépulcres, après la résurrection de Jésus, ils entrèrent dans la ville sainte, et apparurent à un grand nombre de personnes. 
\verse Le centenier et ceux qui étaient avec lui pour garder Jésus, ayant vu le tremblement de terre et ce qui venait d`arriver, furent saisis d`une grande frayeur, et dirent: Assurément, cet homme était Fils de Dieu. 
\verse Il y avait là plusieurs femmes qui regardaient de loin; qui avaient accompagné Jésus depuis la Galilée, pour le servir. 
\verse Parmi elles étaient Marie de Magdala, Marie, mère de Jacques et de Joseph, et la mère des fils de Zébédée. 
\verse Le soir étant venu, arriva un homme riche d`Arimathée, nommé Joseph, lequel était aussi disciple de Jésus. 
\verse Il se rendit vers Pilate, et demanda le corps de Jésus. Et Pilate ordonna de le remettre. 
\verse Joseph prit le corps, l`enveloppa d`un linceul blanc, 
\verse et le déposa dans un sépulcre neuf, qu`il s`était fait tailler dans le roc. Puis il roula une grande pierre à l`entrée du sépulcre, et il s`en alla. 
\verse Marie de Magdala et l`autre Marie étaient là, assises vis-à-vis du sépulcre. 
\verse Le lendemain, qui était le jour après la préparation, les principaux sacrificateurs et les pharisiens allèrent ensemble auprès de Pilate, 
\verse et dirent: Seigneur, nous nous souvenons que cet imposteur a dit, quand il vivait encore: Après trois jours je ressusciterai. 
\verse Ordonne donc que le sépulcre soit gardé jusqu`au troisième jour, afin que ses disciples ne viennent pas dérober le corps, et dire au peuple: Il est ressuscité des morts. Cette dernière imposture serait pire que la première. 
\verse Pilate leur dit: Vous avez une garde; allez, gardez-le comme vous l`entendrez. 
\verse Ils s`en allèrent, et s`assurèrent du sépulcre au moyen de la garde, après avoir scellé la pierre. 

\chapter[Évangile selon Matthieu]

\chaptermark{Évangile selon Matthieu}{}
\verse Après le sabbat, à l`aube du premier jour de la semaine, Marie de Magdala et l`autre Marie allèrent voir le sépulcre. 
\verse Et voici, il y eut un grand tremblement de terre; car un ange du Seigneur descendit du ciel, vint rouler la pierre, et s`assit dessus. 
\verse Son aspect était comme l`éclair, et son vêtement blanc comme la neige. 
\verse Les gardes tremblèrent de peur, et devinrent comme morts. 
\verse Mais l`ange prit la parole, et dit aux femmes: Pour vous, ne craignez pas; car je sais que vous cherchez Jésus qui a été crucifié. 
\verse Il n`est point ici; il est ressuscité, comme il l`avait dit. Venez, voyez le lieu où il était couché, 
\verse et allez promptement dire à ses disciples qu`il est ressuscité des morts. Et voici, il vous précède en Galilée: c`est là que vous le verrez. Voici, je vous l`ai dit. 
\verse Elles s`éloignèrent promptement du sépulcre, avec crainte et avec une grande joie, et elles coururent porter la nouvelle aux disciples. 
\verse Et voici, Jésus vint à leur rencontre, et dit: Je vous salue. Elles s`approchèrent pour saisir ses pieds, et elles se prosternèrent devant lui. 
\verse Alors Jésus leur dit: Ne craignez pas; allez dire à mes frères de se rendre en Galilée: c`est là qu`ils me verront. 
\verse Pendant qu`elles étaient en chemin, quelques hommes de la garde entrèrent dans la ville, et annoncèrent aux principaux sacrificateurs tout ce qui était arrivé. 
\verse Ceux-ci, après s`être assemblés avec les anciens et avoir tenu conseil, donnèrent aux soldats une forte somme d`argent, 
\verse en disant: Dites: Ses disciples sont venus de nuit le dérober, pendant que nous dormions. 
\verse Et si le gouverneur l`apprend, nous l`apaiserons, et nous vous tirerons de peine. 
\verse Les soldats prirent l`argent, et suivirent les instructions qui leur furent données. Et ce bruit s`est répandu parmi les Juifs, jusqu`à ce jour. 
\verse Les onze disciples allèrent en Galilée, sur la montagne que Jésus leur avait désignée. 
\verse Quand ils le virent, ils se prosternèrent devant lui. Mais quelques-uns eurent des doutes. 
\verse Jésus, s`étant approché, leur parla ainsi: Tout pouvoir m`a été donné dans le ciel et sur la terre. 
\verse Allez, faites de toutes les nations des disciples, les baptisant au nom du Père, du Fils et du Saint Esprit, 
\verse et enseignez-leur à observer tout ce que je vous ai prescrit. Et voici, je suis avec vous tous les jours, jusqu`à la fin du monde. 
