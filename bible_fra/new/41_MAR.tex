\book[Deuxième livre des Macchabées]{2Macchabées}


\chapter
\verse Commencement de l`Évangile de Jésus Christ, Fils de Dieu. 
\verse Selon ce qui est écrit dans Ésaïe, le prophète: Voici, j`envoie devant toi mon messager, Qui préparera ton chemin; 
\verse C`est la voix de celui qui crie dans le désert: Préparez le chemin du Seigneur, Aplanissez ses sentiers. 
\verse Jean parut, baptisant dans le désert, et prêchant le baptême de repentance, pour la rémission des péchés. 
\verse Tout le pays de Judée et tous les habitants de Jérusalem se rendaient auprès de lui; et, confessant leurs péchés, ils se faisaient baptiser par lui dans le fleuve du Jourdain. 
\verse Jean avait un vêtement de poils de chameau, et une ceinture de cuir autour des reins. Il se nourrissait de sauterelles et de miel sauvage. 
\verse Il prêchait, disant: Il vient après moi celui qui est plus puissant que moi, et je ne suis pas digne de délier, en me baissant, la courroie de ses souliers. 
\verse Moi, je vous ai baptisés d`eau; lui, il vous baptisera du Saint Esprit. 
\verse En ce temps-là, Jésus vint de Nazareth en Galilée, et il fut baptisé par Jean dans le Jourdain. 
\verse Au moment où il sortait de l`eau, il vit les cieux s`ouvrir, et l`Esprit descendre sur lui comme une colombe. 
\verse Et une voix fit entendre des cieux ces paroles: Tu es mon Fils bien-aimé, en toi j`ai mis toute mon affection. 
\verse Aussitôt, l`Esprit poussa Jésus dans le désert, 
\verse où il passa quarante jours, tenté par Satan. Il était avec les bêtes sauvages, et les anges le servaient. 
\verse Après que Jean eut été livré, Jésus alla dans la Galilée, prêchant l`Évangile de Dieu. 
\verse Il disait: Le temps est accompli, et le royaume de Dieu est proche. Repentez-vous, et croyez à la bonne nouvelle. 
\verse Comme il passait le long de la mer de Galilée, il vit Simon et André, frère de Simon, qui jetaient un filet dans la mer; car ils étaient pêcheurs. 
\verse Jésus leur dit: Suivez-moi, et je vous ferai pêcheurs d`hommes. 
\verse Aussitôt, ils laissèrent leurs filets, et le suivirent. 
\verse Étant allé un peu plus loin, il vit Jacques, fils de Zébédée, et Jean, son frère, qui, eux aussi, étaient dans une barque et réparaient les filets. 
\verse Aussitôt, il les appela; et, laissant leur père Zébédée dans la barque avec les ouvriers, ils le suivirent. 
\verse Ils se rendirent à Capernaüm. Et, le jour du sabbat, Jésus entra d`abord dans la synagogue, et il enseigna. 
\verse Ils étaient frappés de sa doctrine; car il enseignait comme ayant autorité, et non pas comme les scribes. 
\verse Il se trouva dans leur synagogue un homme qui avait un esprit impur, et qui s`écria: 
\verse Qu`y a-t-il entre nous et toi, Jésus de Nazareth? Tu es venu pour nous perdre. Je sais qui tu es: le Saint de Dieu. 
\verse Jésus le menaça, disant: Tais-toi, et sors de cet homme. 
\verse Et l`esprit impur sortit de cet homme, en l`agitant avec violence, et en poussant un grand cri. 
\verse Tous furent saisis de stupéfaction, de sorte qu`il se demandaient les uns aux autres: Qu`est-ce que ceci? Une nouvelle doctrine! Il commande avec autorité même aux esprits impurs, et ils lui obéissent! 
\verse Et sa renommée se répandit aussitôt dans tous les lieux environnants de la Galilée. 
\verse En sortant de la synagogue, ils se rendirent avec Jacques et Jean à la maison de Simon et d`André. 
\verse La belle-mère de Simon était couchée, ayant la fièvre; et aussitôt on parla d`elle à Jésus. 
\verse S`étant approché, il la fit lever en lui prenant la main, et à l`instant la fièvre la quitta. Puis elle les servit. 
\verse Le soir, après le coucher du soleil, on lui amena tous les malades et les démoniaques. 
\verse Et toute la ville était rassemblée devant sa porte. 
\verse Il guérit beaucoup de gens qui avaient diverses maladies; il chassa aussi beaucoup de démons, et il ne permettait pas aux démons de parler, parce qu`ils le connaissaient. 
\verse Vers le matin, pendant qu`il faisait encore très sombre, il se leva, et sortit pour aller dans un lieu désert, où il pria. 
\verse Simon et ceux qui étaient avec lui se mirent à sa recherche; 
\verse et, quand ils l`eurent trouvé, ils lui dirent: Tous te cherchent. 
\verse Il leur répondit: Allons ailleurs, dans les bourgades voisines, afin que j`y prêche aussi; car c`est pour cela que je suis sorti. 
\verse Et il alla prêcher dans les synagogues, par toute la Galilée, et il chassa les démons. 
\verse Un lépreux vint à lui; et, se jetant à genoux, il lui dit d`un ton suppliant: Si tu le veux, tu peux me rendre pur. 
\verse Jésus, ému de compassion, étendit la main, le toucha, et dit: Je le veux, sois pur. 
\verse Aussitôt la lèpre le quitta, et il fut purifié. 
\verse Jésus le renvoya sur-le-champ, avec de sévères recommandations, 
\verse et lui dit: Garde-toi de rien dire à personne; mais va te montrer au sacrificateur, et offre pour ta purification ce que Moïse a prescrit, afin que cela leur serve de témoignage. 
\verse Mais cet homme, s`en étant allé, se mit à publier hautement la chose et à la divulguer, de sorte que Jésus ne pouvait plus entrer publiquement dans une ville. Il se tenait dehors, dans des lieux déserts, et l`on venait à lui de toutes parts. 

\chapter
\verse Quelques jours après, Jésus revint à Capernaüm. On apprit qu`il était à la maison, 
\verse et il s`assembla un si grand nombre de personnes que l`espace devant la porte ne pouvait plus les contenir. Il leur annonçait la parole. 
\verse Des gens vinrent à lui, amenant un paralytique porté par quatre hommes. 
\verse Comme ils ne pouvaient l`aborder, à cause de la foule, ils découvrirent le toit de la maison où il était, et ils descendirent par cette ouverture le lit sur lequel le paralytique était couché. 
\verse Jésus, voyant leur foi, dit au paralytique: Mon enfant, tes péchés sont pardonnés. 
\verse Il y avait là quelques scribes, qui étaient assis, et qui se disaient au dedans d`eux: 
\verse Comment cet homme parle-t-il ainsi? Il blasphème. Qui peut pardonner les péchés, si ce n`est Dieu seul? 
\verse Jésus, ayant aussitôt connu par son esprit ce qu`ils pensaient au dedans d`eux, leur dit: Pourquoi avez-vous de telles pensées dans vos coeurs? 
\verse Lequel est le plus aisé, de dire au paralytique: Tes péchés sont pardonnés, ou de dire: Lève-toi, prends ton lit, et marche? 
\verse Or, afin que vous sachiez que le Fils de l`homme a sur la terre le pouvoir de pardonner les péchés: 
\verse Je te l`ordonne, dit-il au paralytique, lève-toi, prends ton lit, et va dans ta maison. 
\verse Et, à l`instant, il se leva, prit son lit, et sortit en présence de tout le monde, de sorte qu`ils étaient tous dans l`étonnement et glorifiaient Dieu, disant: Nous n`avons jamais rien vu de pareil. 
\verse Jésus sortit de nouveau du côté de la mer. Toute la foule venait à lui, et il les enseignait. 
\verse En passant, il vit Lévi, fils d`Alphée, assis au bureau des péages. Il lui dit: Suis-moi. Lévi se leva, et le suivit. 
\verse Comme Jésus était à table dans la maison de Lévi, beaucoup de publicains et de gens de mauvaise vie se mirent aussi à table avec lui et avec ses disciples; car ils étaient nombreux, et l`avaient suivi. 
\verse Les scribes et les pharisiens, le voyant manger avec les publicains et les gens de mauvaise vie, dirent à ses disciples: Pourquoi mange-t-il et boit-il avec les publicains et les gens de mauvaise vie? 
\verse Ce que Jésus ayant entendu, il leur dit: Ce ne sont pas ceux qui se portent bien qui ont besoin de médecin, mais les malades. Je ne suis pas venu appeler des justes, mais des pécheurs. 
\verse Les disciples de Jean et les pharisiens jeûnaient. Ils vinrent dire à Jésus: Pourquoi les disciples de Jean et ceux des pharisiens jeûnent-ils, tandis que tes disciples ne jeûnent point? 
\verse Jésus leur répondit: Les amis de l`époux peuvent-ils jeûner pendant que l`époux est avec eux? Aussi longtemps qu`ils ont avec eux l`époux, ils ne peuvent jeûner. 
\verse Les jours viendront où l`époux leur sera enlevé, et alors ils jeûneront en ce jour-là. 
\verse Personne ne coud une pièce de drap neuf à un vieil habit; autrement, la pièce de drap neuf emporterait une partie du vieux, et la déchirure serait pire. 
\verse Et personne ne met du vin nouveau dans de vieilles outres; autrement, le vin fait rompre les outres, et le vin et les outres sont perdus; mais il faut mettre le vin nouveau dans des outres neuves. 
\verse Il arriva, un jour de sabbat, que Jésus traversa des champs de blé. Ses disciples, chemin faisant, se mirent à arracher des épis. 
\verse Les pharisiens lui dirent: Voici, pourquoi font-ils ce qui n`est pas permis pendant le sabbat? 
\verse Jésus leur répondit: N`avez-vous jamais lu ce que fit David, lorsqu`il fut dans la nécessité et qu`il eut faim, lui et ceux qui étaient avec lui; 
\verse comment il entra dans la maison de Dieu, du temps du souverain sacrificateur Abiathar, et mangea les pains de proposition, qu`il n`est permis qu`aux sacrificateurs de manger, et en donna même à ceux qui étaient avec lui! 
\verse Puis il leur dit: Le sabbat a été fait pour l`homme, et non l`homme pour le sabbat, 
\verse de sorte que le Fils de l`homme est maître même du sabbat. 

\chapter
\verse Jésus entra de nouveau dans la synagogue. Il s`y trouvait un homme qui avait la main sèche. 
\verse Ils observaient Jésus, pour voir s`il le guérirait le jour du sabbat: c`était afin de pouvoir l`accuser. 
\verse Et Jésus dit à l`homme qui avait la main sèche: Lève-toi, là au milieu. 
\verse Puis il leur dit: Est-il permis, le jour du sabbat, de faire du bien ou de faire du mal, de sauver une personne ou de la tuer? Mais ils gardèrent le silence. 
\verse Alors, promenant ses regards sur eux avec indignation, et en même temps affligé de l`endurcissement de leur coeur, il dit à l`homme: Étends ta main. Il l`étendit, et sa main fut guérie. 
\verse Les pharisiens sortirent, et aussitôt ils se consultèrent avec les hérodiens sur les moyens de le faire périr. 
\verse Jésus se retira vers la mer avec ses disciples. Une grande multitude le suivit de la Galilée; 
\verse et de la Judée, et de Jérusalem, et de l`Idumée, et d`au delà du Jourdain, et des environs de Tyr et de Sidon, une grande multitude, apprenant tout ce qu`il faisait, vint à lui. 
\verse Il chargea ses disciples de tenir toujours à sa disposition une petite barque, afin de ne pas être pressé par la foule. 
\verse Car, comme il guérissait beaucoup de gens, tous ceux qui avaient des maladies se jetaient sur lui pour le toucher. 
\verse Les esprits impurs, quand ils le voyaient, se prosternaient devant lui, et s`écriaient: Tu es le Fils de Dieu. 
\verse Mais il leur recommandait très sévèrement de ne pas le faire connaître. 
\verse Il monta ensuite sur la montagne; il appela ceux qu`il voulut, et ils vinrent auprès de lui. 
\verse Il en établit douze, pour les avoir avec lui, 
\verse et pour les envoyer prêcher avec le pouvoir de chasser les démons. 
\verse Voici les douze qu`il établit: Simon, qu`il nomma Pierre; 
\verse Jacques, fils de Zébédée, et Jean, frère de Jacques, auxquels il donna le nom de Boanergès, qui signifie fils du tonnerre; 
\verse André; Philippe; Barthélemy; Matthieu; Thomas; Jacques, fils d`Alphée; Thaddée; Simon le Cananite; 
\verse et Judas Iscariot, celui qui livra Jésus. 
\verse Ils se rendirent à la maison, et la foule s`assembla de nouveau, en sorte qu`ils ne pouvaient pas même prendre leur repas. 
\verse Les parents de Jésus, ayant appris ce qui se passait, vinrent pour se saisir de lui; car ils disaient: Il est hors de sens. 
\verse Et les scribes, qui étaient descendus de Jérusalem, dirent: Il est possédé de Béelzébul; c`est par le prince des démons qu`il chasse les démons. 
\verse Jésus les appela, et leur dit sous forme de paraboles: Comment Satan peut-il chasser Satan? 
\verse Si un royaume est divisé contre lui-même, ce royaume ne peut subsister; 
\verse et si une maison est divisée contre elle-même, cette maison ne peut subsister. 
\verse Si donc Satan se révolte contre lui-même, il est divisé, et il ne peut subsister, mais c`en est fait de lui. 
\verse Personne ne peut entrer dans la maison d`un homme fort et piller ses biens, sans avoir auparavant lié cet homme fort; alors il pillera sa maison. 
\verse Je vous le dis en vérité, tous les péchés seront pardonnés aux fils des hommes, et les blasphèmes qu`ils auront proférés; 
\verse mais quiconque blasphémera contre le Saint Esprit n`obtiendra jamais de pardon: il est coupable d`un péché éternel. 
\verse Jésus parla ainsi parce qu`ils disaient: Il est possédé d`un esprit impur. 
\verse Survinrent sa mère et ses frères, qui, se tenant dehors, l`envoyèrent appeler. 
\verse La foule était assise autour de lui, et on lui dit: Voici, ta mère et tes frères sont dehors et te demandent. 
\verse Et il répondit: Qui est ma mère, et qui sont mes frères? 
\verse Puis, jetant les regards sur ceux qui étaient assis tout autour de lui: Voici, dit-il, ma mère et mes frères. 
\verse Car, quiconque fait la volonté de Dieu, celui-là est mon frère, ma soeur, et ma mère. 

\chapter
\verse Jésus se mit de nouveau à enseigner au bord de la mer. Une grande foule s`étant assemblée auprès de lui, il monta et s`assit dans une barque, sur la mer. Toute la foule était à terre sur le rivage. 
\verse Il leur enseigna beaucoup de choses en paraboles, et il leur dit dans son enseignement: 
\verse Écoutez. Un semeur sortit pour semer. 
\verse Comme il semait, une partie de la semence tomba le long du chemin: les oiseaux vinrent, et la mangèrent. 
\verse Une autre partie tomba dans un endroit pierreux, où elle n`avait pas beaucoup de terre; elle leva aussitôt, parce qu`elle ne trouva pas un sol profond; 
\verse mais, quand le soleil parut, elle fut brûlée et sécha, faute de racines. 
\verse Une autre partie tomba parmi les épines: les épines montèrent, et l`étouffèrent, et elle ne donna point de fruit. 
\verse Une autre partie tomba dans la bonne terre: elle donna du fruit qui montait et croissait, et elle rapporta trente, soixante, et cent pour un. 
\verse Puis il dit: Que celui qui a des oreilles pour entendre entende. 
\verse Lorsqu`il fut en particulier, ceux qui l`entouraient avec les douze l`interrogèrent sur les paraboles. 
\verse Il leur dit: C`est à vous qu`a été donné le mystère du royaume de Dieu; mais pour ceux qui sont dehors tout se passe en paraboles, 
\verse afin qu`en voyant ils voient et n`aperçoivent point, et qu`en entendant ils entendent et ne comprennent point, de peur qu`ils ne se convertissent, et que les péchés ne leur soient pardonnés. 
\verse Il leur dit encore: Vous ne comprenez pas cette parabole? Comment donc comprendrez-vous toutes les paraboles? 
\verse Le semeur sème la parole. 
\verse Les uns sont le long du chemin, où la parole est semée; quand ils l`ont entendue, aussitôt Satan vient et enlève la parole qui a été semée en eux. 
\verse Les autres, pareillement, reçoivent la semence dans les endroits pierreux; quand ils entendent la parole, ils la reçoivent d`abord avec joie; 
\verse mais ils n`ont pas de racine en eux-mêmes, ils manquent de persistance, et, dès que survient une tribulation ou une persécution à cause de la parole, ils y trouvent une occasion de chute. 
\verse D`autres reçoivent la semence parmi les épines; ce sont ceux qui entendent la parole, 
\verse mais en qui les soucis du siècle, la séduction des richesses et l`invasion des autres convoitises, étouffent la parole, et la rendent infructueuse. 
\verse D`autres reçoivent la semence dans la bonne terre; ce sont ceux qui entendent la parole, la reçoivent, et portent du fruit, trente, soixante, et cent pour un. 
\verse Il leur dit encore: Apporte-t-on la lampe pour la mettre sous le boisseau, ou sous le lit? N`est-ce pas pour la mettre sur le chandelier? 
\verse Car il n`est rien de caché qui ne doive être découvert, rien de secret qui ne doive être mis au jour. 
\verse Si quelqu`un a des oreilles pour entendre, qu`il entende. 
\verse Il leur dit encore: Prenez garde à ce que vous entendez. On vous mesurera avec la mesure dont vous vous serez servis, et on y ajoutera pour vous. 
\verse Car on donnera à celui qui a; mais à celui qui n`a pas on ôtera même ce qu`il a. 
\verse Il dit encore: Il en est du royaume de Dieu comme quand un homme jette de la semence en terre; 
\verse qu`il dorme ou qu`il veille, nuit et jour, la semence germe et croît sans qu`il sache comment. 
\verse La terre produit d`elle-même, d`abord l`herbe, puis l`épi, puis le grain tout formé dans l`épi; 
\verse et, dès que le fruit est mûr, on y met la faucille, car la moisson est là. 
\verse Il dit encore: A quoi comparerons-nous le royaume de Dieu, ou par quelle parabole le représenterons-nous? 
\verse Il est semblable à un grain de sénevé, qui, lorsqu`on le sème en terre, est la plus petite de toutes les semences qui sont sur la terre; 
\verse mais, lorsqu`il a été semé, il monte, devient plus grand que tous les légumes, et pousse de grandes branches, en sorte que les oiseaux du ciel peuvent habiter sous son ombre. 
\verse C`est par beaucoup de paraboles de ce genre qu`il leur annonçait la parole, selon qu`ils étaient capables de l`entendre. 
\verse Il ne leur parlait point sans parabole; mais, en particulier, il expliquait tout à ses disciples. 
\verse Ce même jour, sur le soir, Jésus leur dit: Passons à l`autre bord. 
\verse Après avoir renvoyé la foule, ils l`emmenèrent dans la barque où il se trouvait; il y avait aussi d`autres barques avec lui. 
\verse Il s`éleva un grand tourbillon, et les flots se jetaient dans la barque, au point qu`elle se remplissait déjà. 
\verse Et lui, il dormait à la poupe sur le coussin. Ils le réveillèrent, et lui dirent: Maître, ne t`inquiètes-tu pas de ce que nous périssons? 
\verse S`étant réveillé, il menaça le vent, et dit à la mer: Silence! tais-toi! Et le vent cessa, et il y eut un grand calme. 
\verse Puis il leur dit: Pourquoi avez-vous ainsi peur? Comment n`avez-vous point de foi? 
\verse Ils furent saisis d`une grande frayeur, et ils se dirent les uns aux autres: Quel est donc celui-ci, à qui obéissent même le vent et la mer? 

\chapter
\verse Ils arrivèrent à l`autre bord de la mer, dans le pays des Gadaréniens. 
\verse Aussitôt que Jésus fut hors de la barque, il vint au-devant de lui un homme, sortant des sépulcres, et possédé d`un esprit impur. 
\verse Cet homme avait sa demeure dans les sépulcres, et personne ne pouvait plus le lier, même avec une chaîne. 
\verse Car souvent il avait eu les fers aux pieds et avait été lié de chaînes, mais il avait rompu les chaînes et brisé les fers, et personne n`avait la force de le dompter. 
\verse Il était sans cesse, nuit et jour, dans les sépulcres et sur les montagnes, criant, et se meurtrissant avec des pierres. 
\verse Ayant vu Jésus de loin, il accourut, se prosterna devant lui, 
\verse et s`écria d`une voix forte: Qu`y a-t-il entre moi et toi, Jésus, Fils du Dieu Très Haut? Je t`en conjure au nom de Dieu, ne me tourmente pas. 
\verse Car Jésus lui disait: Sors de cet homme, esprit impur! 
\verse Et, il lui demanda: Quel est ton nom? Légion est mon nom, lui répondit-il, car nous sommes plusieurs. 
\verse Et il le priait instamment de ne pas les envoyer hors du pays. 
\verse Il y avait là, vers la montagne, un grand troupeau de pourceaux qui paissaient. 
\verse Et les démons le prièrent, disant: Envoie-nous dans ces pourceaux, afin que nous entrions en eux. 
\verse Il le leur permit. Et les esprits impurs sortirent, entrèrent dans les pourceaux, et le troupeau se précipita des pentes escarpées dans la mer: il y en avait environ deux mille, et ils se noyèrent dans la mer. 
\verse Ceux qui les faisaient paître s`enfuirent, et répandirent la nouvelle dans la ville et dans les campagnes. Les gens allèrent voir ce qui était arrivé. 
\verse Ils vinrent auprès de Jésus, et ils virent le démoniaque, celui qui avait eu la légion, assis, vêtu, et dans son bon sens; et ils furent saisis de frayeur. 
\verse Ceux qui avaient vu ce qui s`était passé leur racontèrent ce qui était arrivé au démoniaque et aux pourceaux. 
\verse Alors ils se mirent à supplier Jésus de quitter leur territoire. 
\verse Comme il montait dans la barque, celui qui avait été démoniaque lui demanda la permission de rester avec lui. 
\verse Jésus ne le lui permit pas, mais il lui dit: Va dans ta maison, vers les tiens, et raconte-leur tout ce que le Seigneur t`a fait, et comment il a eu pitié de toi. 
\verse Il s`en alla, et se mit à publier dans la Décapole tout ce que Jésus avait fait pour lui. Et tous furent dans l`étonnement. 
\verse Jésus dans la barque regagna l`autre rive, où une grande foule s`assembla près de lui. Il était au bord de la mer. 
\verse Alors vint un des chefs de la synagogue, nommé Jaïrus, qui, l`ayant aperçu, se jeta à ses pieds, 
\verse et lui adressa cette instante prière: Ma petite fille est à l`extrémité, viens, impose-lui les mains, afin qu`elle soit sauvée et qu`elle vive. 
\verse Jésus s`en alla avec lui. Et une grande foule le suivait et le pressait. 
\verse Or, il y avait une femme atteinte d`une perte de sang depuis douze ans. 
\verse Elle avait beaucoup souffert entre les mains de plusieurs médecins, elle avait dépensé tout ce qu`elle possédait, et elle n`avait éprouvé aucun soulagement, mais était allée plutôt en empirant. 
\verse Ayant entendu parler de Jésus, elle vint dans la foule par derrière, et toucha son vêtement. 
\verse Car elle disait: Si je puis seulement toucher ses vêtements, je serai guérie. 
\verse Au même instant la perte de sang s`arrêta, et elle sentit dans son corps qu`elle était guérie de son mal. 
\verse Jésus connut aussitôt en lui-même qu`une force était sortie de lui; et, se retournant au milieu de la foule, il dit: Qui a touché mes vêtements? 
\verse Ses disciples lui dirent: Tu vois la foule qui te presse, et tu dis: Qui m`a touché? 
\verse Et il regardait autour de lui, pour voir celle qui avait fait cela. 
\verse La femme, effrayée et tremblante, sachant ce qui s`était passé en elle, vint se jeter à ses pieds, et lui dit toute la vérité. 
\verse Mais Jésus lui dit: Ma fille, ta foi t`a sauvée; va en paix, et sois guérie de ton mal. 
\verse Comme il parlait encore, survinrent de chez le chef de la synagogue des gens qui dirent: Ta fille est morte; pourquoi importuner davantage le maître? 
\verse Mais Jésus, sans tenir compte de ces paroles, dit au chef de la synagogue: Ne crains pas, crois seulement. 
\verse Et il ne permit à personne de l`accompagner, si ce n`est à Pierre, à Jacques, et à Jean, frère de Jacques. 
\verse Ils arrivèrent à la maison du chef de la synagogue, où Jésus vit une foule bruyante et des gens qui pleuraient et poussaient de grands cris. 
\verse Il entra, et leur dit: Pourquoi faites-vous du bruit, et pourquoi pleurez-vous? L`enfant n`est pas morte, mais elle dort. 
\verse Et ils se moquaient de lui. Alors, ayant fait sortir tout le monde, il prit avec lui le père et la mère de l`enfant, et ceux qui l`avaient accompagné, et il entra là où était l`enfant. 
\verse Il la saisit par la main, et lui dit: Talitha koumi, ce qui signifie: Jeune fille, lève-toi, je te le dis. 
\verse Aussitôt la jeune fille se leva, et se mit à marcher; car elle avait douze ans. Et ils furent dans un grand étonnement. 
\verse Jésus leur adressa de fortes recommandations, pour que personne ne sût la chose; et il dit qu`on donnât à manger à la jeune fille. 

\chapter
\verse Jésus partit de là, et se rendit dans sa patrie. Ses disciples le suivirent. 
\verse Quand le sabbat fut venu, il se mit à enseigner dans la synagogue. Beaucoup de gens qui l`entendirent étaient étonnés et disaient: D`où lui viennent ces choses? Quelle est cette sagesse qui lui a été donnée, et comment de tels miracles se font-ils par ses mains? 
\verse N`est-ce pas le charpentier, le fils de Marie, le frère de Jacques, de Joses, de Jude et de Simon? et ses soeurs ne sont-elles pas ici parmi nous? Et il était pour eux une occasion de chute. 
\verse Mais Jésus leur dit: Un prophète n`est méprisé que dans sa patrie, parmi ses parents, et dans sa maison. 
\verse Il ne put faire là aucun miracle, si ce n`est qu`il imposa les mains à quelques malades et les guérit. 
\verse Et il s`étonnait de leur incrédulité. Jésus parcourait les villages d`alentour, en enseignant. 
\verse Alors il appela les douze, et il commença à les envoyer deux à deux, en leur donnant pouvoir sur les esprits impurs. 
\verse Il leur prescrivit de ne rien prendre pour le voyage, si ce n`est un bâton; de n`avoir ni pain, ni sac, ni monnaie dans la ceinture; 
\verse de chausser des sandales, et de ne pas revêtir deux tuniques. 
\verse Puis il leur dit: Dans quelque maison que vous entriez, restez-y jusqu`à ce que vous partiez de ce lieu. 
\verse Et, s`il y a quelque part des gens qui ne vous reçoivent ni ne vous écoutent, retirez-vous de là, et secouez la poussière de vos pieds, afin que cela leur serve de témoignage. 
\verse Ils partirent, et ils prêchèrent la repentance. 
\verse Ils chassaient beaucoup de démons, et ils oignaient d`huile beaucoup de malades et les guérissaient. 
\verse Le roi Hérode entendit parler de Jésus, dont le nom était devenu célèbre, et il dit: Jean Baptiste est ressuscité des morts, et c`est pour cela qu`il se fait par lui des miracles. 
\verse D`autres disaient: C`est Élie. Et d`autres disaient: C`est un prophète comme l`un des prophètes. 
\verse Mais Hérode, en apprenant cela, disait: Ce Jean que j`ai fait décapiter, c`est lui qui est ressuscité. 
\verse Car Hérode lui-même avait fait arrêter Jean, et l`avait fait lier en prison, à cause d`Hérodias, femme de Philippe, son frère, parce qu`il l`avait épousée, 
\verse et que Jean lui disait: Il ne t`est pas permis d`avoir la femme de ton frère. 
\verse Hérodias était irritée contre Jean, et voulait le faire mourir. 
\verse Mais elle ne le pouvait; car Hérode craignait Jean, le connaissant pour un homme juste et saint; il le protégeait, et, après l`avoir entendu, il était souvent perplexe, et l`écoutait avec plaisir. 
\verse Cependant, un jour propice arriva, lorsque Hérode, à l`anniversaire de sa naissance, donna un festin à ses grands, aux chefs militaires et aux principaux de la Galilée. 
\verse La fille d`Hérodias entra dans la salle; elle dansa, et plut à Hérode et à ses convives. Le roi dit à la jeune fille: Demande-moi ce que tu voudras, et je te le donnerai. 
\verse Il ajouta avec serment: Ce que tu me demanderas, je te le donnerai, fût-ce la moitié de mon royaume. 
\verse Étant sortie, elle dit à sa mère: Que demanderais-je? Et sa mère répondit: La tête de Jean Baptiste. 
\verse Elle s`empressa de rentrer aussitôt vers le roi, et lui fit cette demande: Je veux que tu me donnes à l`instant, sur un plat, la tête de Jean Baptiste. 
\verse Le roi fut attristé; mais, à cause de ses serments et des convives, il ne voulut pas lui faire un refus. 
\verse Il envoya sur-le-champ un garde, avec ordre d`apporter la tête de Jean Baptiste. 
\verse Le garde alla décapiter Jean dans la prison, et apporta la tête sur un plat. Il la donna à la jeune fille, et la jeune fille la donna à sa mère. 
\verse Les disciples de Jean, ayant appris cela, vinrent prendre son corps, et le mirent dans un sépulcre. 
\verse Les apôtres, s`étant rassemblés auprès de Jésus, lui racontèrent tout ce qu`ils avaient fait et tout ce qu`ils avaient enseigné. 
\verse Jésus leur dit: Venez à l`écart dans un lieu désert, et reposez-vous un peu. Car il y avait beaucoup d`allants et de venants, et ils n`avaient même pas le temps de manger. 
\verse Ils partirent donc dans une barque, pour aller à l`écart dans un lieu désert. 
\verse Beaucoup de gens les virent s`en aller et les reconnurent, et de toutes les villes on accourut à pied et on les devança au lieu où ils se rendaient. 
\verse Quand il sortit de la barque, Jésus vit une grande foule, et fut ému de compassion pour eux, parce qu`ils étaient comme des brebis qui n`ont point de berger; et il se mit à leur enseigner beaucoup de choses. 
\verse Comme l`heure était déjà avancée, ses disciples s`approchèrent de lui, et dirent: Ce lieu est désert, et l`heure est déjà avancée; 
\verse renvoie-les, afin qu`ils aillent dans les campagnes et dans les villages des environs, pour s`acheter de quoi manger. 
\verse Jésus leur répondit: Donnez-leur vous-mêmes à manger. Mais ils lui dirent: Irions-nous acheter des pains pour deux cents deniers, et leur donnerions-nous à manger? 
\verse Et il leur dit: Combien avez-vous de pains? Allez voir. Ils s`en assurèrent, et répondirent: Cinq, et deux poissons. 
\verse Alors il leur commanda de les faire tous asseoir par groupes sur l`herbe verte, 
\verse et ils s`assirent par rangées de cent et de cinquante. 
\verse Il prit les cinq pains et les deux poissons et, levant les yeux vers le ciel, il rendit grâces. Puis, il rompit les pains, et les donna aux disciples, afin qu`ils les distribuassent à la foule. Il partagea aussi les deux poissons entre tous. 
\verse Tous mangèrent et furent rassasiés, 
\verse et l`on emporta douze paniers pleins de morceaux de pain et de ce qui restait des poissons. 
\verse Ceux qui avaient mangé les pains étaient cinq mille hommes. 
\verse Aussitôt après, il obligea ses disciples à monter dans la barque et à passer avant lui de l`autre côté, vers Bethsaïda, pendant que lui-même renverrait la foule. 
\verse Quand il l`eut renvoyée, il s`en alla sur la montagne, pour prier. 
\verse Le soir étant venu, la barque était au milieu de la mer, et Jésus était seul à terre. 
\verse Il vit qu`ils avaient beaucoup de peine à ramer; car le vent leur était contraire. A la quatrième veille de la nuit environ, il alla vers eux, marchant sur la mer, et il voulait les dépasser. 
\verse Quand ils le virent marcher sur la mer, ils crurent que c`étaient un fantôme, et ils poussèrent des cris; 
\verse car ils le voyaient tous, et ils étaient troublés. Aussitôt Jésus leur parla, et leur dit: Rassurez-vous, c`est moi, n`ayez pas peur! 
\verse Puis il monta vers eux dans la barque, et le vent cessa. Ils furent en eux-même tout stupéfaits et remplis d`étonnement; 
\verse car ils n`avaient pas compris le miracle des pains, parce que leur coeur était endurci. 
\verse Après avoir traversé la mer, ils vinrent dans le pays de Génésareth, et ils abordèrent. 
\verse Quand ils furent sortis de la barque, les gens, ayant aussitôt reconnu Jésus, 
\verse parcoururent tous les environs, et l`on se mit à apporter les malades sur des lits, partout où l`on apprenait qu`il était. 
\verse En quelque lieu qu`il arrivât, dans les villages, dans les villes ou dans les campagnes, on mettait les malades sur les places publiques, et on le priait de leur permettre seulement de toucher le bord de son vêtement. Et tous ceux qui le touchaient étaient guéris. 

\chapter
\verse Les pharisiens et quelques scribes, venus de Jérusalem, s`assemblèrent auprès de Jésus. 
\verse Ils virent quelques-uns de ses disciples prendre leurs repas avec des mains impures, c`est-à-dire, non lavées. 
\verse Or, les pharisiens et tous les Juifs ne mangent pas sans s`être lavé soigneusement les mains, conformément à la tradition des anciens; 
\verse et, quand ils reviennent de la place publique, ils ne mangent qu`après s`être purifiés. Ils ont encore beaucoup d`autres observances traditionnelles, comme le lavage des coupes, des cruches et des vases d`airain. 
\verse Et les pharisiens et les scribes lui demandèrent: Pourquoi tes disciples ne suivent-ils pas la tradition des anciens, mais prennent-ils leurs repas avec des mains impures? 
\verse Jésus leur répondit: Hypocrites, Ésaïe a bien prophétisé sur vous, ainsi qu`il est écrit: Ce peuple m`honore des lèvres, Mais son coeur est éloigné de moi. 
\verse C`est en vain qu`ils m`honorent, En donnant des préceptes qui sont des commandements d`hommes. 
\verse Vous abandonnez le commandement de Dieu, et vous observez la tradition des hommes. 
\verse Il leur dit encore: Vous anéantissez fort bien le commandement de Dieu, pour garder votre tradition. 
\verse Car Moïse a dit: Honore ton père et ta mère; et: Celui qui maudira son père ou sa mère sera puni de mort. 
\verse Mais vous, vous dites: Si un homme dit à son père ou à sa mère: Ce dont j`aurais pu t`assister est corban, c`est-à-dire, une offrande à Dieu, 
\verse vous ne le laissez plus rien faire pour son père ou pour sa mère, 
\verse annulant ainsi la parole de Dieu par votre tradition, que vous avez établie. Et vous faites beaucoup d`autres choses semblables. 
\verse Ensuite, ayant de nouveau appelé la foule à lui, il lui dit: Écoutez-moi tous, et comprenez. 
\verse Il n`est hors de l`homme rien qui, entrant en lui, puisse le souiller; mais ce qui sort de l`homme, c`est ce qui le souille. 
\verse Si quelqu`un a des oreilles pour entendre, qu`il entende. 
\verse Lorsqu`il fut entré dans la maison, loin de la foule, ses disciples l`interrogèrent sur cette parabole. 
\verse Il leur dit: Vous aussi, êtes-vous donc sans intelligence? Ne comprenez-vous pas que rien de ce qui du dehors entre dans l`homme ne peut le souiller? 
\verse Car cela n`entre pas dans son coeur, mais dans son ventre, puis s`en va dans les lieux secrets, qui purifient tous les aliments. 
\verse Il dit encore: Ce qui sort de l`homme, c`est ce qui souille l`homme. 
\verse Car c`est du dedans, c`est du coeur des hommes, que sortent les mauvaises pensées, les adultères, les impudicités, les meurtres, 
\verse les vols, les cupidités, les méchancetés, la fraude, le dérèglement, le regard envieux, la calomnie, l`orgueil, la folie. 
\verse Toutes ces choses mauvaises sortent du dedans, et souillent l`homme. 
\verse Jésus, étant parti de là, s`en alla dans le territoire de Tyr et de Sidon. Il entra dans une maison, désirant que personne ne le sût; mais il ne put rester caché. 
\verse Car une femme, dont la fille était possédée d`un esprit impur, entendit parler de lui, et vint se jeter à ses pieds. 
\verse Cette femme était grecque, syro-phénicienne d`origine. Elle le pria de chasser le démon hors de sa fille. Jésus lui dit: 
\verse Laisse d`abord les enfants se rassasier; car il n`est pas bien de prendre le pain des enfants, et de le jeter aux petits chiens. 
\verse Oui, Seigneur, lui répondit-elle, mais les petits chiens, sous la table, mangent les miettes des enfants. 
\verse Alors il lui dit: à cause de cette parole, va, le démon est sorti de ta fille. 
\verse Et, quand elle rentra dans sa maison, elle trouva l`enfant couchée sur le lit, le démon étant sorti. 
\verse Jésus quitta le territoire de Tyr, et revint par Sidon vers la mer de Galilée, en traversant le pays de la Décapole. 
\verse On lui amena un sourd, qui avait de la difficulté à parler, et on le pria de lui imposer les mains. 
\verse Il le prit à part loin de la foule, lui mit les doigts dans les oreilles, et lui toucha la langue avec sa propre salive; 
\verse puis, levant les yeux au ciel, il soupira, et dit: Éphphatha, c`est-à-dire, ouvre-toi. 
\verse Aussitôt ses oreilles s`ouvrirent, sa langue se délia, et il parla très bien. 
\verse Jésus leur recommanda de n`en parler à personne; mais plus il le leur recommanda, plus ils le publièrent. 
\verse Ils étaient dans le plus grand étonnement, et disaient: Il fait tout à merveille; même il fait entendre les sourds, et parler les muets. 

\chapter
\verse En ces jours-là, une foule nombreuse s`étant de nouveau réunie et n`ayant pas de quoi manger, Jésus appela les disciples, et leur dit: 
\verse Je suis ému de compassion pour cette foule; car voilà trois jours qu`ils sont près de moi, et ils n`ont rien à manger. 
\verse Si je les renvoie chez eux à jeun, les forces leur manqueront en chemin; car quelques-uns d`entre eux sont venus de loin. 
\verse Ses disciples lui répondirent: Comment pourrait-on les rassasier de pains, ici, dans un lieu désert? 
\verse Jésus leur demanda: Combien avez-vous de pains? Sept, répondirent-ils. 
\verse Alors il fit asseoir la foule par terre, prit les sept pains, et, après avoir rendu grâces, il les rompit, et les donna à ses disciples pour les distribuer; et ils les distribuèrent à la foule. 
\verse Ils avaient encore quelques petits poissons, et Jésus, ayant rendu grâces, les fit aussi distribuer. 
\verse Ils mangèrent et furent rassasiés, et l`on emporta sept corbeilles pleines des morceaux qui restaient. 
\verse Ils étaient environ quatre mille. Ensuite Jésus les renvoya. 
\verse Aussitôt il monta dans la barque avec ses disciples, et se rendit dans la contrée de Dalmanutha. 
\verse Les pharisiens survinrent, se mirent à discuter avec Jésus, et, pour l`éprouver, lui demandèrent un signe venant du ciel. 
\verse Jésus, soupirant profondément en son esprit, dit: Pourquoi cette génération demande-t-elle un signe? Je vous le dis en vérité, il ne sera point donné de signe à cette génération. 
\verse Puis il les quitta, et remonta dans la barque, pour passer sur l`autre bord. 
\verse Les disciples avaient oublié de prendre des pains; ils n`en avaient qu`un seul avec eux dans la barque. 
\verse Jésus leur fit cette recommandation: Gardez-vous avec soin du levain des pharisiens et du levain d`Hérode. 
\verse Les disciples raisonnaient entre eux, et disaient: C`est parce que nous n`avons pas de pains. 
\verse Jésus, l`ayant connu, leur dit: Pourquoi raisonnez-vous sur ce que vous n`avez pas de pains? Etes-vous encore sans intelligence, et ne comprenez-vous pas? 
\verse Avez-vous le coeur endurci? Ayant des yeux, ne voyez-vous pas? Ayant des oreilles, n`entendez-vous pas? Et n`avez-vous point de mémoire? 
\verse Quand j`ai rompu les cinq pains pour les cinq mille hommes, combien de paniers pleins de morceaux avez-vous emportés? Douze, lui répondirent-ils. 
\verse Et quand j`ai rompu les sept pains pour les quatre mille hommes, combien de corbeilles pleines de morceaux avez-vous emportées? Sept, répondirent-ils. 
\verse Et il leur dit: Ne comprenez-vous pas encore? 
\verse Ils se rendirent à Bethsaïda; et on amena vers Jésus un aveugle, qu`on le pria de toucher. 
\verse Il prit l`aveugle par la main, et le conduisit hors du village; puis il lui mit de la salive sur les yeux, lui imposa les mains, et lui demanda s`il voyait quelque chose. 
\verse Il regarda, et dit: J`aperçois les hommes, mais j`en vois comme des arbres, et qui marchent. 
\verse Jésus lui mit de nouveau les mains sur les yeux; et, quand l`aveugle regarda fixement, il fut guéri, et vit tout distinctement. 
\verse Alors Jésus le renvoya dans sa maison, en disant: N`entre pas au village. 
\verse Jésus s`en alla, avec ses disciples, dans les villages de Césarée de Philippe, et il leur posa en chemin cette question: Qui dit-on que je suis? 
\verse Ils répondirent: Jean Baptiste; les autres, Élie, les autres, l`un des prophètes. 
\verse Et vous, leur demanda-t-il, qui dites-vous que je suis? Pierre lui répondit: Tu es le Christ. 
\verse Jésus leur recommanda sévèrement de ne dire cela de lui à personne. 
\verse Alors il commença à leur apprendre qu`il fallait que le Fils de l`homme souffrît beaucoup, qu`il fût rejeté par les anciens, par les principaux sacrificateurs et par les scribes, qu`il fût mis à mort, et qu`il ressuscitât trois jours après. 
\verse Il leur disait ces choses ouvertement. Et Pierre, l`ayant pris à part, se mit à le reprendre. 
\verse Mais Jésus, se retournant et regardant ses disciples, réprimanda Pierre, et dit: Arrière de moi, Satan! car tu ne conçois pas les choses de Dieu, tu n`as que des pensées humaines. 
\verse Puis, ayant appelé la foule avec ses disciples, il leur dit: Si quelqu`un veut venir après moi, qu`il renonce à lui-même, qu`il se charge de sa croix, et qu`il me suive. 
\verse Car celui qui voudra sauver sa vie la perdra, mais celui qui perdra sa vie à cause de moi et de la bonne nouvelle la sauvera. 
\verse Et que sert-il à un homme de gagner tout le monde, s`il perd son âme? 
\verse Que donnerait un homme en échange de son âme? 
\verse Car quiconque aura honte de moi et de mes paroles au milieu de cette génération adultère et pécheresse, le Fils de l`homme aura aussi honte de lui, quand il viendra dans la gloire de son Père, avec les saints anges. 

\chapter
\verse Il leur dit encore: Je vous le dis en vérité, quelques-uns de ceux qui sont ici ne mourront point, qu`ils n`aient vu le royaume de Dieu venir avec puissance. 
\verse Six jours après, Jésus prit avec lui Pierre, Jacques et Jean, et il les conduisit seuls à l`écart sur une haute montagne. Il fut transfiguré devant eux; 
\verse ses vêtements devinrent resplendissants, et d`une telle blancheur qu`il n`est pas de foulon sur la terre qui puisse blanchir ainsi. 
\verse Élie et Moïse leur apparurent, s`entretenant avec Jésus. 
\verse Pierre, prenant la parole, dit à Jésus: Rabbi, il est bon que nous soyons ici; dressons trois tentes, une pour toi, une pour Moïse, et une pour Élie. 
\verse Car il ne savait que dire, l`effroi les ayant saisis. 
\verse Une nuée vint les couvrir, et de la nuée sortit une voix: Celui-ci est mon Fils bien-aimé: écoutez-le! 
\verse Aussitôt les disciples regardèrent tout autour, et ils ne virent que Jésus seul avec eux. 
\verse Comme ils descendaient de la montagne, Jésus leur recommanda de ne dire à personne ce qu`ils avaient vu, jusqu`à ce que le Fils de l`homme fût ressuscité des morts. 
\verse Ils retinrent cette parole, se demandant entre eux ce que c`est que ressusciter des morts. 
\verse Les disciples lui firent cette question: Pourquoi les scribes disent-ils qu`il faut qu`Élie vienne premièrement? 
\verse Il leur répondit: Élie viendra premièrement, et rétablira toutes choses. Et pourquoi est-il écrit du Fils de l`homme qu`il doit souffrir beaucoup et être méprisé? 
\verse Mais je vous dis qu`Élie est venu, et qu`ils l`ont traité comme ils ont voulu, selon qu`il est écrit de lui. 
\verse Lorsqu`ils furent arrivés près des disciples, ils virent autour d`eux une grande foule, et des scribes qui discutaient avec eux. 
\verse Dès que la foule vit Jésus, elle fut surprise, et accourut pour le saluer. 
\verse Il leur demanda: Sur quoi discutez-vous avec eux? 
\verse Et un homme de la foule lui répondit: Maître, j`ai amené auprès de toi mon fils, qui est possédé d`un esprit muet. 
\verse En quelque lieu qu`il le saisisse, il le jette par terre; l`enfant écume, grince des dents, et devient tout raide. J`ai prié tes disciples de chasser l`esprit, et ils n`ont pas pu. 
\verse Race incrédule, leur dit Jésus, jusques à quand serai-je avec vous? jusques à quand vous supporterai-je? Amenez-le-moi. On le lui amena. 
\verse Et aussitôt que l`enfant vit Jésus, l`esprit l`agita avec violence; il tomba par terre, et se roulait en écumant. 
\verse Jésus demanda au père: Combien y a-t-il de temps que cela lui arrive? Depuis son enfance, répondit-il. 
\verse Et souvent l`esprit l`a jeté dans le feu et dans l`eau pour le faire périr. Mais, si tu peux quelque chose, viens à notre secours, aie compassion de nous. 
\verse Jésus lui dit: Si tu peux!... Tout est possible à celui qui croit. 
\verse Aussitôt le père de l`enfant s`écria: Je crois! viens au secours de mon incrédulité! 
\verse Jésus, voyant accourir la foule, menaça l`esprit impur, et lui dit: Esprit muet et sourd, je te l`ordonne, sors de cet enfant, et n`y rentre plus. 
\verse Et il sortit, en poussant des cris, et en l`agitant avec une grande violence. L`enfant devint comme mort, de sorte que plusieurs disaient qu`il était mort. 
\verse Mais Jésus, l`ayant pris par la main, le fit lever. Et il se tint debout. 
\verse Quand Jésus fut entré dans la maison, ses disciples lui demandèrent en particulier: Pourquoi n`avons-nous pu chasser cet esprit? 
\verse Il leur dit: Cette espèce-là ne peut sortir que par la prière. 
\verse Ils partirent de là, et traversèrent la Galilée. Jésus ne voulait pas qu`on le sût. 
\verse Car il enseignait ses disciples, et il leur dit: Le Fils de l`homme sera livré entre les mains des hommes; ils le feront mourir, et, trois jours après qu`il aura été mis à mort, il ressuscitera. 
\verse Mais les disciples ne comprenaient pas cette parole, et ils craignaient de l`interroger. 
\verse Ils arrivèrent à Capernaüm. Lorsqu`il fut dans la maison, Jésus leur demanda: De quoi discutiez-vous en chemin? 
\verse Mais ils gardèrent le silence, car en chemin ils avaient discuté entre eux pour savoir qui était le plus grand. 
\verse Alors il s`assit, appela les douze, et leur dit: Si quelqu`un veut être le premier, il sera le dernier de tous et le serviteur de tous. 
\verse Et il prit un petit enfant, le plaça au milieu d`eux, et l`ayant pris dans ses bras, il leur dit: 
\verse Quiconque reçoit en mon nom un de ces petits enfants me reçoit moi-même; et quiconque me reçoit, reçoit non pas moi, mais celui qui m`a envoyé. 
\verse Jean lui dit: Maître, nous avons vu un homme qui chasse des démons en ton nom; et nous l`en avons empêché, parce qu`il ne nous suit pas. 
\verse Ne l`en empêchez pas, répondit Jésus, car il n`est personne qui, faisant un miracle en mon nom, puisse aussitôt après parler mal de moi. 
\verse Qui n`est pas contre nous est pour nous. 
\verse Et quiconque vous donnera à boire un verre d`eau en mon nom, parce que vous appartenez à Christ, je vous le dis en vérité, il ne perdra point sa récompense. 
\verse Mais, si quelqu`un scandalisait un de ces petits qui croient, il vaudrait mieux pour lui qu`on lui mît au cou une grosse meule de moulin, et qu`on le jetât dans la mer. 
\verse Si ta main est pour toi une occasion de chute, coupe-la; mieux vaut pour toi entrer manchot dans la vie, 
\verse que d`avoir les deux mains et d`aller dans la géhenne, dans le feu qui ne s`éteint point. 
\verse Si ton pied est pour toi une occasion de chute, coupe-le; mieux vaut pour toi entrer boiteux dans la vie, 
\verse que d`avoir les deux pieds et d`être jeté dans la géhenne, dans le feu qui ne s`éteint point. 
\verse Et si ton oeil est pour toi une occasion de chute, arrache-le; mieux vaut pour toi entrer dans le royaume de Dieu n`ayant qu`un oeil, que d`avoir deux yeux et d`être jeté dans la géhenne, 
\verse où leur ver ne meurt point, et où le feu ne s`éteint point. 
\verse Car tout homme sera salé de feu. 
\verse Le sel est une bonne chose; mais si le sel devient sans saveur, avec quoi l`assaisonnerez-vous? (9:51) Ayez du sel en vous-mêmes, et soyez en paix les uns avec les autres. 

\chapter
\verse Jésus, étant parti de là, se rendit dans le territoire de la Judée au delà du Jourdain. La foule s`assembla de nouveau près de lui, et selon sa coutume, il se mit encore à l`enseigner. 
\verse Les pharisiens l`abordèrent; et, pour l`éprouver, ils lui demandèrent s`il est permis à un homme de répudiée sa femme. 
\verse Il leur répondit: Que vous a prescrit Moïse? 
\verse Moïse, dirent-ils, a permis d`écrire une lettre de divorce et de répudier. 
\verse Et Jésus leur dit: C`est à cause de la dureté de votre coeur que Moïse vous a donné ce précepte. 
\verse Mais au commencement de la création, Dieu fit l`homme et la femme; 
\verse c`est pourquoi l`homme quittera son père et sa mère, et s`attachera à sa femme, 
\verse et les deux deviendront une seule chair. Ainsi ils ne sont plus deux, mais ils sont une seule chair. 
\verse Que l`homme donc ne sépare pas ce que Dieu a joint. 
\verse Lorsqu`ils furent dans la maison, les disciples l`interrogèrent encore là-dessus. 
\verse Il leur dit: Celui qui répudie sa femme et qui en épouse une autre, commet un adultère à son égard; 
\verse et si une femme quitte son mari et en épouse un autre, elle commet un adultère. 
\verse On lui amena des petits enfants, afin qu`il les touchât. Mais les disciples reprirent ceux qui les amenaient. 
\verse Jésus, voyant cela, fut indigné, et leur dit: Laissez venir à moi les petits enfants, et ne les en empêchez pas; car le royaume de Dieu est pour ceux qui leur ressemblent. 
\verse Je vous le dis en vérité, quiconque ne recevra pas le royaume de Dieu comme un petit enfant n`y entrera point. 
\verse Puis il les prit dans ses bras, et les bénit, en leur imposant les mains. 
\verse Comme Jésus se mettait en chemin, un homme accourut, et se jetant à genoux devant lui: Bon maître, lui demanda-t-il, que dois-je faire pour hériter la vie éternelle? 
\verse Jésus lui dit: Pourquoi m`appelles-tu bon? Il n`y a de bon que Dieu seul. 
\verse Tu connais les commandements: Tu ne commettras point d`adultère; tu ne tueras point; tu ne déroberas point; tu ne diras point de faux témoignage; tu ne feras tort à personne; honore ton père et ta mère. 
\verse Il lui répondit: Maître, j`ai observé toutes ces choses dès ma jeunesse. 
\verse Jésus, l`ayant regardé, l`aima, et lui dit: Il te manque une chose; va, vends tout ce que tu as, donne-le aux pauvres, et tu auras un trésor dans le ciel. Puis viens, et suis-moi. 
\verse Mais, affligé de cette parole, cet homme s`en alla tout triste; car il avait de grands biens. 
\verse Jésus, regardant autour de lui, dit à ses disciples: Qu`il sera difficile à ceux qui ont des richesses d`entrer dans le royaume de Dieu! 
\verse Les disciples furent étonnés de ce que Jésus parlait ainsi. Et, reprenant, il leur dit: Mes enfants, qu`il est difficile à ceux qui se confient dans les richesses d`entrer dans le royaume de Dieu! 
\verse Il est plus facile à un chameau de passer par le trou d`une aiguille qu`à un riche d`entrer dans le royaume de Dieu. 
\verse Les disciples furent encore plus étonnés, et ils se dirent les uns aux autres; Et qui peut être sauvé? 
\verse Jésus les regarda, et dit: Cela est impossible aux hommes, mais non à Dieu: car tout est possible à Dieu. 
\verse Pierre se mit à lui dire; Voici, nous avons tout quitté, et nous t`avons suivi. 
\verse Jésus répondit: Je vous le dis en vérité, il n`est personne qui, ayant quitté, à cause de moi et à cause de la bonne nouvelle, sa maison, ou ses frères, ou ses soeurs, ou sa mère, ou son père, ou ses enfants, ou ses terres, 
\verse ne reçoive au centuple, présentement dans ce siècle-ci, des maisons, des frères, des soeurs, des mères, des enfants, et des terres, avec des persécutions, et, dans le siècle à venir, la vie éternelle. 
\verse Plusieurs des premiers seront les derniers, et plusieurs des derniers seront les premiers. 
\verse Ils étaient en chemin pour monter à Jérusalem, et Jésus allait devant eux. Les disciples étaient troublés, et le suivaient avec crainte. Et Jésus prit de nouveau les douze auprès de lui, et commença à leur dire ce qui devait lui arriver: 
\verse Voici, nous montons à Jérusalem, et le Fils de l`homme sera livré aux principaux sacrificateurs et aux scribes. Ils le condamneront à mort, et ils le livreront aux païens, 
\verse qui se moqueront de lui, cracheront sur lui, le battront de verges, et le feront mourir; et, trois jours après, il ressuscitera. 
\verse Les fils de Zébédée, Jacques et Jean, s`approchèrent de Jésus, et lui dirent: Maître, nous voudrions que tu fisses pour nous ce que nous te demanderons. 
\verse Il leur dit: Que voulez-vous que je fasse pour vous? 
\verse Accorde-nous, lui dirent-ils, d`être assis l`un à ta droite et l`autre à ta gauche, quand tu seras dans ta gloire. 
\verse Jésus leur répondit: Vous ne savez ce que vous demandez. Pouvez-vous boire la coupe que je dois boire, ou être baptisés du baptême dont je dois être baptisé? Nous le pouvons, dirent-ils. 
\verse Et Jésus leur répondit: Il est vrai que vous boirez la coupe que je dois boire, et que vous serez baptisés du baptême dont je dois être baptisé; 
\verse mais pour ce qui est d`être assis à ma droite ou à ma gauche, cela ne dépend pas de moi, et ne sera donné qu`à ceux à qui cela est réservé. 
\verse Les dix, ayant entendu cela, commencèrent à s`indigner contre Jacques et Jean. 
\verse Jésus les appela, et leur dit: Vous savez que ceux qu`on regarde comme les chefs des nations les tyrannisent, et que les grands les dominent. 
\verse Il n`en est pas de même au milieu de vous. Mais quiconque veut être grand parmi vous, qu`il soit votre serviteur; 
\verse et quiconque veut être le premier parmi vous, qu`il soit l`esclave de tous. 
\verse Car le Fils de l`homme est venu, non pour être servi, mais pour servir et donner sa vie comme la rançon de plusieurs. 
\verse Ils arrivèrent à Jéricho. Et, lorsque Jésus en sortit, avec ses disciples et une assez grande foule, le fils de Timée, Bartimée, mendiant aveugle, était assis au bord du chemin. 
\verse Il entendit que c`était Jésus de Nazareth, et il se mit à crier; Fils de David, Jésus aie pitié de moi! 
\verse Plusieurs le reprenaient, pour le faire taire; mais il criait beaucoup plus fort; Fils de David, aie pitié de moi! 
\verse Jésus s`arrêta, et dit: Appelez-le. Ils appelèrent l`aveugle, en lui disant: Prends courage, lève-toi, il t`appelle. 
\verse L`aveugle jeta son manteau, et, se levant d`un bond, vint vers Jésus. 
\verse Jésus, prenant la parole, lui dit: Que veux-tu que je te fasse? Rabbouni, lui répondit l`aveugle, que je recouvre la vue. 
\verse Et Jésus lui dit: Va, ta foi t`a sauvé. (10:53) Aussitôt il recouvra la vue, et suivit Jésus dans le chemin. 

\chapter
\verse Lorsqu`ils approchèrent de Jérusalem, et qu`ils furent près de Bethphagé et de Béthanie, vers la montagne des oliviers, Jésus envoya deux de ses disciples, 
\verse en leur disant: Allez au village qui est devant vous; dès que vous y serez entrés, vous trouverez un ânon attaché, sur lequel aucun homme ne s`est encore assis; détachez-le, et amenez-le. 
\verse Si quelqu`un vous dit: Pourquoi faites-vous cela? répondez: Le Seigneur en a besoin. Et à l`instant il le laissera venir ici. 
\verse les disciples, étant allés, trouvèrent l`ânon attaché dehors près d`une porte, au contour du chemin, et ils le détachèrent. 
\verse Quelques-uns de ceux qui étaient là leur dirent: Que faites-vous? pourquoi détachez-vous cet ânon? 
\verse Ils répondirent comme Jésus l`avait dit. Et on les laissa aller. 
\verse Ils amenèrent à Jésus l`ânon, sur lequel ils jetèrent leurs vêtements, et Jésus s`assit dessus. 
\verse Beaucoup de gens étendirent leurs vêtements sur le chemin, et d`autres des branches qu`ils coupèrent dans les champs. 
\verse Ceux qui précédaient et ceux qui suivaient Jésus criaient: Hosanna! Béni soit celui qui vient au nom du Seigneur! 
\verse Béni soit le règne qui vient, le règne de David, notre père! Hosanna dans les lieux très hauts! 
\verse Jésus entra à Jérusalem, dans le temple. Quand il eut tout considéré, comme il était déjà tard, il s`en alla à Béthanie avec les douze. 
\verse Le lendemain, après qu`ils furent sortis de Béthanie, Jésus eut faim. 
\verse Apercevant de loin un figuier qui avait des feuilles, il alla voir s`il y trouverait quelque chose; et, s`en étant approché, il ne trouva que des feuilles, car ce n`était pas la saison des figues. 
\verse Prenant alors la parole, il lui dit: Que jamais personne ne mange de ton fruit! Et ses disciples l`entendirent. 
\verse Ils arrivèrent à Jérusalem, et Jésus entra dans le temple. Il se mit à chasser ceux qui vendaient et qui achetaient dans le temple; il renversa les tables des changeurs, et les sièges des vendeurs de pigeons; 
\verse et il ne laissait personne transporter aucun objet à travers le temple. 
\verse Et il enseignait et disait: N`est-il pas écrit: Ma maison sera appelée une maison de prière pour toutes les nations? Mais vous, vous en avez fait une caverne de voleurs. 
\verse Les principaux sacrificateurs et les scribes, l`ayant entendu, cherchèrent les moyens de le faire périr; car ils le craignaient, parce que toute la foule était frappée de sa doctrine. 
\verse Quand le soir fut venu, Jésus sortit de la ville. 
\verse Le matin, en passant, les disciples virent le figuier séché jusqu`aux racines. 
\verse Pierre, se rappelant ce qui s`était passé, dit à Jésus: Rabbi, regarde, le figuier que tu as maudit a séché. 
\verse Jésus prit la parole, et leur dit: Ayez foi en Dieu. 
\verse Je vous le dis en vérité, si quelqu`un dit à cette montagne: Ote-toi de là et jette-toi dans la mer, et s`il ne doute point en son coeur, mais croit que ce qu`il dit arrive, il le verra s`accomplir. 
\verse C`est pourquoi je vous dis: Tout ce que vous demanderez en priant, croyez que vous l`avez reçu, et vous le verrez s`accomplir. 
\verse Et, lorsque vous êtes debout faisant votre prière, si vous avez quelque chose contre quelqu`un, pardonnez, afin que votre Père qui est dans les cieux vous pardonne aussi vos offenses. 
\verse Mais si vous ne pardonnez pas, votre Père qui est dans les cieux ne vous pardonnera pas non plus vos offenses. 
\verse Ils se rendirent de nouveau à Jérusalem, et, pendant que Jésus se promenait dans le temple, les principaux sacrificateurs, les scribes et les anciens, vinrent à lui, 
\verse et lui dirent: Par quelle autorité fais-tu ces choses, et qui t`a donné l`autorité de les faire? 
\verse Jésus leur répondit: Je vous adresserai aussi une question; répondez-moi, et je vous dirai par quelle autorité je fais ces choses. 
\verse Le baptême de Jean venait-il du ciel, ou des hommes? Répondez-moi. 
\verse Mais ils raisonnèrent ainsi entre eux: Si nous répondons: Du ciel, il dira: Pourquoi donc n`avez-vous pas cru en lui? 
\verse Et si nous répondons: Des hommes... Ils craignaient le peuple, car tous tenaient réellement Jean pour un prophète. 
\verse Alors ils répondirent à Jésus: Nous ne savons. Et Jésus leur dit: Moi non plus, je ne vous dirai pas par quelle autorité je fais ces choses. 

\chapter
\verse Jésus se mit ensuite à leur parler en paraboles. Un homme planta une vigne. Il l`entoura d`une haie, creusa un pressoir, et bâtit une tour; puis il l`afferma à des vignerons, et quitta le pays. 
\verse Au temps de la récolte, il envoya un serviteur vers les vignerons, pour recevoir d`eux une part du produit de la vigne. 
\verse S`étant saisis de lui, ils le battirent, et le renvoyèrent à vide. 
\verse Il envoya de nouveau vers eux un autre serviteur; ils le frappèrent à la tête, et l`outragèrent. 
\verse Il en envoya un troisième, qu`ils tuèrent; puis plusieurs autres, qu`ils battirent ou tuèrent. 
\verse Il avait encore un fils bien-aimé; il l`envoya vers eux le dernier, en disant: Ils auront du respect pour mon fils. 
\verse Mais ces vignerons dirent entre eux: Voici l`héritier; venez, tuons-le, et l`héritage sera à nous. 
\verse Et ils se saisirent de lui, le tuèrent, et le jetèrent hors de la vigne. 
\verse Maintenant, que fera le maître de la vigne? Il viendra, fera périr les vignerons, et il donnera la vigne à d`autres. 
\verse N`avez-vous pas lu cette parole de l`Écriture: La pierre qu`ont rejetée ceux qui bâtissaient Est devenue la principale de l`angle; 
\verse C`est par la volonté du Seigneur qu`elle l`est devenue, Et c`est un prodige à nos yeux? 
\verse Ils cherchaient à se saisir de lui, mais ils craignaient la foule. Ils avaient compris que c`était pour eux que Jésus avait dit cette parabole. Et ils le quittèrent, et s`en allèrent. 
\verse Ils envoyèrent auprès de Jésus quelques-uns des pharisiens et des hérodiens, afin de le surprendre par ses propres paroles. 
\verse Et ils vinrent lui dire: Maître, nous savons que tu es vrai, et que tu ne t`inquiètes de personne; car tu ne regardes pas à l`apparence des hommes, et tu enseignes la voie de Dieu selon la vérité. Est-il permis, ou non, de payer le tribut à César? Devons-nous payer, ou ne pas payer? 
\verse Jésus, connaissant leur hypocrisie, leur répondit: Pourquoi me tentez-vous? Apportez-moi un denier, afin que je le voie. 
\verse Ils en apportèrent un; et Jésus leur demanda: De qui sont cette effigie et cette inscription? De César, lui répondirent-ils. 
\verse Alors il leur dit: Rendez à César ce qui est à César, et à Dieu ce qui est à Dieu. Et ils furent à son égard dans l`étonnement. 
\verse Les sadducéens, qui disent qu`il n`y a point de résurrection, vinrent auprès de Jésus, et lui firent cette question: 
\verse Maître, voici ce que Moïse nous a prescrit: Si le frère de quelqu`un meurt, et laisse une femme, sans avoir d`enfants, son frère épousera sa veuve, et suscitera une postérité à son frère. 
\verse Or, il y avait sept frères. Le premier se maria, et mourut sans laisser de postérité. 
\verse Le second prit la veuve pour femme, et mourut sans laisser de postérité. Il en fut de même du troisième, 
\verse et aucun des sept ne laissa de postérité. Après eux tous, la femme mourut aussi. 
\verse A la résurrection, duquel d`entre eux sera-t-elle la femme? Car les sept l`ont eue pour femme. 
\verse Jésus leur répondit: N`êtes-vous pas dans l`erreur, parce que vous ne comprenez ni les Écritures, ni la puissance de Dieu? 
\verse Car, à la résurrection des morts, les hommes ne prendront point de femmes, ni les femmes de maris, mais ils seront comme les anges dans les cieux. 
\verse Pour ce qui est de la résurrection des morts, n`avez-vous pas lu, dans le livre de Moïse, ce que Dieu lui dit, à propos du buisson: Je suis le Dieu d`Abraham, le Dieu d`Isaac, et le Dieu de Jacob? 
\verse Dieu n`est pas Dieu des morts, mais des vivants. Vous êtes grandement dans l`erreur. 
\verse Un des scribes, qui les avait entendus discuter, sachant que Jésus avait bien répondu aux sadducéens, s`approcha, et lui demanda: Quel est le premier de tous les commandements? 
\verse Jésus répondit: Voici le premier: Écoute, Israël, le Seigneur, notre Dieu, est l`unique Seigneur; 
\verse et: Tu aimeras le Seigneur, ton Dieu, de tout ton coeur, de toute ton âme, de toute ta pensée, et de toute ta force. 
\verse Voici le second: Tu aimeras ton prochain comme toi-même. Il n`y a pas d`autre commandement plus grand que ceux-là. 
\verse Le scribe lui dit: Bien, maître; tu as dit avec vérité que Dieu est unique, et qu`il n`y en a point d`autre que lui, 
\verse et que l`aimer de tout son coeur, de toute sa pensée, de toute son âme et de toute sa force, et aimer son prochain comme soi-même, c`est plus que tous les holocaustes et tous les sacrifices. 
\verse Jésus, voyant qu`il avait répondu avec intelligence, lui dit: Tu n`es pas loin du royaume de Dieu. Et personne n`osa plus lui proposer des questions. 
\verse Jésus, continuant à enseigner dans le temple, dit: Comment les scribes disent-ils que le Christ est fils de David? 
\verse David lui-même, animé par l`Esprit Saint, a dit: Le Seigneur a dit à mon Seigneur: Assieds-toi à ma droite, Jusqu`à ce que je fasse de tes ennemis ton marchepied. 
\verse David lui-même l`appelle Seigneur; comment donc est-il son fils? Et une grande foule l`écoutait avec plaisir. 
\verse Il leur disait dans son enseignement: Gardez-vous des scribes, qui aiment à se promener en robes longues, et à être salués dans les places publiques; 
\verse qui recherchent les premiers sièges dans les synagogues, et les premières places dans les festins; 
\verse qui dévorent les maisons des veuves, et qui font pour l`apparence de longues prières. Ils seront jugés plus sévèrement. 
\verse Jésus, s`étant assis vis-à-vis du tronc, regardait comment la foule y mettait de l`argent. Plusieurs riches mettaient beaucoup. 
\verse Il vint aussi une pauvre veuve, elle y mit deux petites pièces, faisant un quart de sou. 
\verse Alors Jésus, ayant appelé ses disciples, leur dit: Je vous le dis en vérité, cette pauvre veuve a donné plus qu`aucun de ceux qui ont mis dans le tronc; 
\verse car tous ont mis de leur superflu, mais elle a mis de son nécessaire, tout ce qu`elle possédait, tout ce qu`elle avait pour vivre. 

\chapter
\verse Lorsque Jésus sortit du temple, un de ses disciples lui dit: Maître, regarde quelles pierres, et quelles constructions! 
\verse Jésus lui répondit: Vois-tu ces grandes constructions? Il ne restera pas pierre sur pierre qui ne soit renversée. 
\verse Il s`assit sur la montagne des oliviers, en face du temple. Et Pierre, Jacques, Jean et André lui firent en particulier cette question: 
\verse Dis-nous, quand cela arrivera-t-il, et à quel signe connaîtra-t-on que toutes ces choses vont s`accomplir? 
\verse Jésus se mit alors à leur dire: Prenez garde que personne ne vous séduise. 
\verse Car plusieurs viendront sous mon nom, disant; C`est moi. Et ils séduiront beaucoup de gens. 
\verse Quand vous entendrez parler de guerres et de bruits de guerres, ne soyez pas troublés, car il faut que ces choses arrivent. Mais ce ne sera pas encore la fin. 
\verse Une nation s`élèvera contre une nation, et un royaume contre un royaume; il y aura des tremblements de terre en divers lieux, il y aura des famines. Ce ne sera que le commencement des douleurs. 
\verse Prenez garde à vous-mêmes. On vous livrera aux tribunaux, et vous serez battus de verges dans les synagogues; vous comparaîtrez devant des gouverneurs et devant des rois, à cause de moi, pour leur servir de témoignage. 
\verse Il faut premièrement que la bonne nouvelle soit prêchée à toutes les nations. 
\verse Quand on vous emmènera pour vous livrer, ne vous inquiétez pas d`avance de ce que vous aurez à dire, mais dites ce qui vous sera donné à l`heure même; car ce n`est pas vous qui parlerez, mais l`Esprit Saint. 
\verse Le frère livrera son frère à la mort, et le père son enfant; les enfants se soulèveront contre leurs parents, et les feront mourir. 
\verse Vous serez haïs de tous, à cause de mon nom, mais celui qui persévérera jusqu`à la fin sera sauvé. 
\verse Lorsque vous verrez l`abomination de la désolation établie là où elle ne doit pas être, -que celui qui lit fasse attention, -alors, que ceux qui seront en Judée fuient dans les montagnes; 
\verse que celui qui sera sur le toit ne descende pas et n`entre pas pour prendre quelque chose dans sa maison; 
\verse et que celui qui sera dans les champs ne retourne pas en arrière pour prendre son manteau. 
\verse Malheur aux femmes qui seront enceintes et à celles qui allaiteront en ces jours-là! 
\verse Priez pour que ces choses n`arrivent pas en hiver. 
\verse Car la détresse, en ces jours, sera telle qu`il n`y en a point eu de semblable depuis le commencement du monde que Dieu a créé jusqu`à présent, et qu`il n`y en aura jamais. 
\verse Et, si le Seigneur n`avait abrégé ces jours, personne ne serait sauvé; mais il les a abrégés, à cause des élus qu`il a choisis. 
\verse Si quelqu`un vous dit alors: "Le Christ est ici", ou: "Il est là", ne le croyez pas. 
\verse Car il s`élèvera de faux Christs et de faux prophètes; ils feront des prodiges et des miracles pour séduire les élus, s`il était possible. 
\verse Soyez sur vos gardes: je vous ai tout annoncé d`avance. 
\verse Mais dans ces jours, après cette détresse, le soleil s`obscurcira, la lune ne donnera plus sa lumière, 
\verse les étoiles tomberont du ciel, et les puissances qui sont dans les cieux seront ébranlées. 
\verse Alors on verra le Fils de l`homme venant sur les nuées avec une grande puissance et avec gloire. 
\verse Alors il enverra les anges, et il rassemblera les élus des quatre vents, de l`extrémité de la terre jusqu`à l`extrémité du ciel. 
\verse Instruisez-vous par une comparaison tirée du figuier. Dès que ses branches deviennent tendres, et que les feuilles poussent, vous connaissez que l`été est proche. 
\verse De même, quand vous verrez ces choses arriver, sachez que le Fils de l`homme est proche, à la porte. 
\verse Je vous le dis en vérité, cette génération ne passera point, que tout cela n`arrive. 
\verse Le ciel et la terre passeront, mais mes paroles ne passeront point. 
\verse Pour ce qui est du jour ou de l`heure, personne ne le sait, ni les anges dans le ciel, ni le Fils, mais le Père seul. 
\verse Prenez garde, veillez et priez; car vous ne savez quand ce temps viendra. 
\verse Il en sera comme d`un homme qui, partant pour un voyage, laisse sa maison, remet l`autorité à ses serviteurs, indique à chacun sa tâche, et ordonne au portier de veiller. 
\verse Veillez donc, car vous ne savez quand viendra le maître de la maison, ou le soir, ou au milieu de la nuit, ou au chant du coq, ou le matin; 
\verse craignez qu`il ne vous trouve endormis, à son arrivée soudaine. 
\verse Ce que je vous dis, je le dis à tous: Veillez. 

\chapter
\verse La fête de Pâque et des pains sans levain devait avoir lieu deux jours après. Les principaux sacrificateurs et les scribes cherchaient les moyens d`arrêter Jésus par ruse, et de le faire mourir. 
\verse Car ils disaient: Que ce ne soit pas pendant la fête, afin qu`il n`y ait pas de tumulte parmi le peuple. 
\verse Comme Jésus était à Béthanie, dans la maison de Simon le lépreux, une femme entra, pendant qu`il se trouvait à table. Elle tenait un vase d`albâtre, qui renfermait un parfum de nard pur de grand prix; et, ayant rompu le vase, elle répandit le parfum sur la tête de Jésus. 
\verse Quelques-uns exprimèrent entre eux leur indignation: A quoi bon perdre ce parfum? 
\verse On aurait pu le vendre plus de trois cents deniers, et les donner aux pauvres. Et ils s`irritaient contre cette femme. 
\verse Mais Jésus dit: Laissez-la. Pourquoi lui faites-vous de la peine? Elle a fait une bonne action à mon égard; 
\verse car vous avez toujours les pauvres avec vous, et vous pouvez leur faire du bien quand vous voulez, mais vous ne m`avez pas toujours. 
\verse Elle a fait ce qu`elle a pu; elle a d`avance embaumé mon corps pour la sépulture. 
\verse Je vous le dis en vérité, partout où la bonne nouvelle sera prêchée, dans le monde entier, on racontera aussi en mémoire de cette femme ce qu`elle a fait. 
\verse Judas Iscariot, l`un des douze, alla vers les principaux sacrificateurs, afin de leur livrer Jésus. 
\verse Après l`avoir entendu, ils furent dans la joie, et promirent de lui donner de l`argent. Et Judas cherchait une occasion favorable pour le livrer. 
\verse Le premier jour des pains sans levain, où l`on immolait la Pâque, les disciples de Jésus lui dirent: Où veux-tu que nous allions te préparer la Pâque? 
\verse Et il envoya deux de ses disciples, et leur dit: Allez à la ville; vous rencontrerez un homme portant une cruche d`eau, suivez-le. 
\verse Quelque part qu`il entre, dites au maître de la maison: Le maître dit: Où est le lieu où je mangerai la Pâque avec mes disciples? 
\verse Et il vous montrera une grande chambre haute, meublée et toute prête: c`est là que vous nous préparerez la Pâque. 
\verse Les disciples partirent, arrivèrent à la ville, et trouvèrent les choses comme il le leur avait dit; et ils préparèrent la Pâque. 
\verse Le soir étant venu, il arriva avec les douze. 
\verse Pendant qu`ils étaient à table et qu`ils mangeaient, Jésus dit: Je vous le dis en vérité, l`un de vous, qui mange avec moi, me livrera. 
\verse Ils commencèrent à s`attrister, et à lui dire, l`un après l`autre: Est-ce moi? 
\verse Il leur répondit: C`est l`un des douze, qui met avec moi la main dans le plat. 
\verse Le Fils de l`homme s`en va selon ce qui est écrit de lui. Mais malheur à l`homme par qui le Fils de l`homme est livré! Mieux vaudrait pour cet homme qu`il ne fût pas né. 
\verse Pendant qu`ils mangeaient, Jésus prit du pain; et, après avoir rendu grâces, il le rompit, et le leur donna, en disant: Prenez, ceci est mon corps. 
\verse Il prit ensuite une coupe; et, après avoir rendu grâces, il la leur donna, et ils en burent tous. 
\verse Et il leur dit: Ceci est mon sang, le sang de l`alliance, qui est répandu pour plusieurs. 
\verse Je vous le dis en vérité, je ne boirai plus jamais du fruit de la vigne, jusqu`au jour où je le boirai nouveau dans le royaume de Dieu. 
\verse Après avoir chanté les cantiques, ils se rendirent à la montagne des oliviers. 
\verse Jésus leur dit: Vous serez tous scandalisés; car il est écrit: Je frapperai le berger, et les brebis seront dispersées. 
\verse Mais, après que je serai ressuscité, je vous précéderai en Galilée. 
\verse Pierre lui dit: Quand tous seraient scandalisés, je ne serai pas scandalisé. 
\verse Et Jésus lui dit: Je te le dis en vérité, toi, aujourd`hui, cette nuit même, avant que le coq chante deux fois, tu me renieras trois fois. 
\verse Mais Pierre reprit plus fortement: Quand il me faudrait mourir avec toi, je ne te renierai pas. Et tous dirent la même chose. 
\verse Ils allèrent ensuite dans un lieu appelé Gethsémané, et Jésus dit à ses disciples: Asseyez-vous ici, pendant que je prierai. 
\verse Il prit avec lui Pierre, Jacques et Jean, et il commença à éprouver de la frayeur et des angoisses. 
\verse Il leur dit: Mon âme est triste jusqu`à la mort; restez ici, et veillez. 
\verse Puis, ayant fait quelques pas en avant, il se jeta contre terre, et pria que, s`il était possible, cette heure s`éloignât de lui. 
\verse Il disait: Abba, Père, toutes choses te sont possibles, éloigne de moi cette coupe! Toutefois, non pas ce que je veux, mais ce que tu veux. 
\verse Et il vint vers les disciples, qu`il trouva endormis, et il dit à Pierre: Simon, tu dors! Tu n`as pu veiller une heure! 
\verse Veillez et priez, afin que vous ne tombiez pas en tentation; l`esprit est bien disposé, mais la chair est faible. 
\verse Il s`éloigna de nouveau, et fit la même prière. 
\verse Il revint, et les trouva encore endormis; car leurs yeux étaient appesantis. Ils ne surent que lui répondre. 
\verse Il revint pour la troisième fois, et leur dit: Dormez maintenant, et reposez-vous! C`est assez! L`heure est venue; voici, le Fils de l`homme est livré aux mains des pécheurs. 
\verse Levez-vous, allons; voici, celui qui me livre s`approche. 
\verse Et aussitôt, comme il parlait encore, arriva Judas l`un des douze, et avec lui une foule armée d`épées et de bâtons, envoyée par les principaux sacrificateurs, par les scribes et par les anciens. 
\verse Celui qui le livrait leur avait donné ce signe: Celui que je baiserai, c`est lui; saisissez-le, et emmenez-le sûrement. 
\verse Dès qu`il fut arrivé, il s`approcha de Jésus, disant: Rabbi! Et il le baisa. 
\verse Alors ces gens mirent la main sur Jésus, et le saisirent. 
\verse Un de ceux qui étaient là, tirant l`épée, frappa le serviteur du souverain sacrificateur, et lui emporta l`oreille. 
\verse Jésus, prenant la parole, leur dit: Vous êtes venus, comme après un brigand, avec des épées et des bâtons, pour vous emparer de moi. 
\verse J`étais tous les jours parmi vous, enseignant dans le temple, et vous ne m`avez pas saisi. Mais c`est afin que les Écritures soient accomplies. 
\verse Alors tous l`abandonnèrent, et prirent la fuite. 
\verse Un jeune homme le suivait, n`ayant sur le corps qu`un drap. On se saisit de lui; 
\verse mais il lâcha son vêtement, et se sauva tout nu. 
\verse Ils emmenèrent Jésus chez le souverain sacrificateur, où s`assemblèrent tous les principaux sacrificateurs, les anciens et les scribes. 
\verse Pierre le suivit de loin jusque dans l`intérieur de la cour du souverain sacrificateur; il s`assit avec les serviteurs, et il se chauffait près du feu. 
\verse Les principaux sacrificateurs et tout le sanhédrin cherchaient un témoignage contre Jésus, pour le faire mourir, et ils n`en trouvaient point; 
\verse car plusieurs rendaient de faux témoignages contre lui, mais les témoignages ne s`accordaient pas. 
\verse Quelques-uns se levèrent, et portèrent un faux témoignage contre lui, disant: 
\verse Nous l`avons entendu dire: Je détruirai ce temple fait de main d`homme, et en trois jours j`en bâtirai un autre qui ne sera pas fait de main d`homme. 
\verse Même sur ce point-là leur témoignage ne s`accordait pas. 
\verse Alors le souverain sacrificateur, se levant au milieu de l`assemblée, interrogea Jésus, et dit: Ne réponds-tu rien? Qu`est-ce que ces gens déposent contre toi? 
\verse Jésus garda le silence, et ne répondit rien. Le souverain sacrificateur l`interrogea de nouveau, et lui dit: Es-tu le Christ, le Fils du Dieu béni? 
\verse Jésus répondit: Je le suis. Et vous verrez le Fils de l`homme assis à la droite de la puissance de Dieu, et venant sur les nuées du ciel. 
\verse Alors le souverain sacrificateur déchira ses vêtements, et dit: Qu`avons-nous encore besoin de témoins? 
\verse Vous avez entendu le blasphème. Que vous en semble? Tous le condamnèrent comme méritant la mort. 
\verse Et quelques-uns se mirent à cracher sur lui, à lui voiler le visage et à le frapper à coups de poing, en lui disant: Devine! Et les serviteurs le reçurent en lui donnant des soufflets. 
\verse Pendant que Pierre était en bas dans la cour, il vint une des servantes du souverain sacrificateur. 
\verse Voyant Pierre qui se chauffait, elle le regarda, et lui dit: Toi aussi, tu étais avec Jésus de Nazareth. 
\verse Il le nia, disant: Je ne sais pas, je ne comprends pas ce que tu veux dire. Puis il sortit pour aller dans le vestibule. Et le coq chanta. 
\verse La servante, l`ayant vu, se mit de nouveau à dire à ceux qui étaient présents: Celui-ci est de ces gens-là. Et il le nia de nouveau. 
\verse Peu après, ceux qui étaient présents dirent encore à Pierre: Certainement tu es de ces gens-là, car tu es Galiléen. 
\verse Alors il commença à faire des imprécations et à jurer: Je ne connais pas cet homme dont vous parlez. 
\verse Aussitôt, pour la seconde fois, le coq chanta. Et Pierre se souvint de la parole que Jésus lui avait dite: Avant que le coq chante deux fois, tu me renieras trois fois. Et en y réfléchissant, il pleurait. 

\chapter
\verse Dès le matin, les principaux sacrificateurs tinrent conseil avec les anciens et les scribes, et tout le sanhédrin. Après avoir lié Jésus, ils l`emmenèrent, et le livrèrent à Pilate. 
\verse Pilate l`interrogea: Es-tu le roi des Juifs? Jésus lui répondit: Tu le dis. 
\verse Les principaux sacrificateurs portaient contre lui plusieurs accusations. 
\verse Pilate l`interrogea de nouveau: Ne réponds-tu rien? Vois de combien de choses ils t`accusent. 
\verse Et Jésus ne fit plus aucune réponse, ce qui étonna Pilate. 
\verse A chaque fête, il relâchait un prisonnier, celui que demandait la foule. 
\verse Il y avait en prison un nommé Barabbas avec ses complices, pour un meurtre qu`ils avaient commis dans une sédition. 
\verse La foule, étant montée, se mit à demander ce qu`il avait coutume de leur accorder. 
\verse Pilate leur répondit: Voulez-vous que je vous relâche le roi des Juif? 
\verse Car il savait que c`était par envie que les principaux sacrificateurs l`avaient livré. 
\verse Mais les chefs des sacrificateurs excitèrent la foule, afin que Pilate leur relâchât plutôt Barabbas. 
\verse Pilate, reprenant la parole, leur dit: Que voulez-vous donc que je fasse de celui que vous appelez le roi des Juifs? 
\verse Ils crièrent de nouveau: Crucifie-le! 
\verse Pilate leur dit: Quel mal a-t-il fait? Et ils crièrent encore plus fort: Crucifie-le! 
\verse Pilate, voulant satisfaire la foule, leur relâcha Barabbas; et, après avoir fait battre de verges Jésus, il le livra pour être crucifié. 
\verse Les soldats conduisirent Jésus dans l`intérieur de la cour, c`est-à-dire, dans le prétoire, et ils assemblèrent toute la cohorte. 
\verse Ils le revêtirent de pourpre, et posèrent sur sa tête une couronne d`épines, qu`ils avaient tressée. 
\verse Puis ils se mirent à le saluer: Salut, roi des Juifs! 
\verse Et ils lui frappaient la tête avec un roseau, crachaient sur lui, et, fléchissant les genoux, ils se prosternaient devant lui. 
\verse Après s`être ainsi moqués de lui, ils lui ôtèrent la pourpre, lui remirent ses vêtements, et l`emmenèrent pour le crucifier. 
\verse Ils forcèrent à porter la croix de Jésus un passant qui revenait des champs, Simon de Cyrène, père d`Alexandre et de Rufus; 
\verse et ils conduisirent Jésus au lieu nommé Golgotha, ce qui signifie lieu du crâne. 
\verse Ils lui donnèrent à boire du vin mêlé de myrrhe, mais il ne le prit pas. 
\verse Ils le crucifièrent, et se partagèrent ses vêtements, en tirant au sort pour savoir ce que chacun aurait. 
\verse C`était la troisième heure, quand ils le crucifièrent. 
\verse L`inscription indiquant le sujet de sa condamnation portait ces mots: Le roi des Juifs. 
\verse Ils crucifièrent avec lui deux brigands, l`un à sa droite, et l`autre à sa gauche. 
\verse Ainsi fut accompli ce que dit l`Écriture: Il a été mis au nombre des malfaiteurs. 
\verse Les passants l`injuriaient, et secouaient la tête, en disant: Hé! toi qui détruis le temple, et qui le rebâtis en trois jours, 
\verse sauve-toi toi-même, en descendant de la croix! 
\verse Les principaux sacrificateurs aussi, avec les scribes, se moquaient entre eux, et disaient: Il a sauvé les autres, et il ne peut se sauver lui-même! 
\verse Que le Christ, le roi d`Israël, descende maintenant de la croix, afin que nous voyions et que nous croyions! Ceux qui étaient crucifiés avec lui l`insultaient aussi. 
\verse La sixième heure étant venue, il y eut des ténèbres sur toute la terre, jusqu`à la neuvième heure. 
\verse Et à la neuvième heure, Jésus s`écria d`une voix forte: Éloï, Éloï, lama sabachthani? ce qui signifie: Mon Dieu, mon Dieu, pourquoi m`as-tu abandonné? 
\verse Quelques-uns de ceux qui étaient là, l`ayant entendu, dirent: Voici, il appelle Élie. 
\verse Et l`un d`eux courut remplir une éponge de vinaigre, et, l`ayant fixée à un roseau, il lui donna à boire, en disant: Laissez, voyons si Élie viendra le descendre. 
\verse Mais Jésus, ayant poussé un grand cri, expira. 
\verse Le voile du temple se déchira en deux, depuis le haut jusqu`en bas. 
\verse Le centenier, qui était en face de Jésus, voyant qu`il avait expiré de la sorte, dit: Assurément, cet homme était Fils de Dieu. 
\verse Il y avait aussi des femmes qui regardaient de loin. Parmi elles étaient Marie de Magdala, Marie, mère de Jacques le mineur et de Joses, et Salomé, 
\verse qui le suivaient et le servaient lorsqu`il était en Galilée, et plusieurs autres qui étaient montées avec lui à Jérusalem. 
\verse Le soir étant venu, comme c`était la préparation, c`est-à-dire, la veille du sabbat, - 
\verse arriva Joseph d`Arimathée, conseiller de distinction, qui lui-même attendait aussi le royaume de Dieu. Il osa se rendre vers Pilate, pour demander le corps de Jésus. 
\verse Pilate s`étonna qu`il fût mort si tôt; fit venir le centenier et lui demanda s`il était mort depuis longtemps. 
\verse S`en étant assuré par le centenier, il donna le corps à Joseph. 
\verse Et Joseph, ayant acheté un linceul, descendit Jésus de la croix, l`enveloppa du linceul, et le déposa dans un sépulcre taillé dans le roc. Puis il roula une pierre à l`entrée du sépulcre. 
\verse Marie de Magdala, et Marie, mère de Joses, regardaient où on le mettait. 

\chapter
\verse Lorsque le sabbat fut passé, Marie de Magdala, Marie, mère de Jacques, et Salomé, achetèrent des aromates, afin d`aller embaumer Jésus. 
\verse Le premier jour de la semaine, elles se rendirent au sépulcre, de grand matin, comme le soleil venait de se lever. 
\verse Elles disaient entre elles: Qui nous roulera la pierre loin de l`entrée du sépulcre? 
\verse Et, levant les yeux, elles aperçurent que la pierre, qui était très grande, avait été roulée. 
\verse Elles entrèrent dans le sépulcre, virent un jeune homme assis à droite vêtu d`une robe blanche, et elles furent épouvantées. 
\verse Il leur dit: Ne vous épouvantez pas; vous cherchez Jésus de Nazareth, qui a été crucifié; il est ressuscité, il n`est point ici; voici le lieu où on l`avait mis. 
\verse Mais allez dire à ses disciples et à Pierre qu`il vous précède en Galilée: c`est là que vous le verrez, comme il vous l`a dit. 
\verse Elles sortirent du sépulcre et s`enfuirent. La peur et le trouble les avaient saisies; et elles ne dirent rien à personne, à cause de leur effroi. 
\verse Jésus, étant ressuscité le matin du premier jour de la semaine, apparut d`abord à Marie de Magdala, de laquelle il avait chassé sept démons. 
\verse Elle alla en porter la nouvelle à ceux qui avaient été avec lui, et qui s`affligeaient et pleuraient. 
\verse Quand ils entendirent qu`il vivait, et qu`elle l`avait vu, ils ne le crurent point. 
\verse Après cela, il apparut, sous une autre forme, à deux d`entre eux qui étaient en chemin pour aller à la campagne. 
\verse Ils revinrent l`annoncer aux autres, qui ne les crurent pas non plus. 
\verse Enfin, il apparut aux onze, pendant qu`ils étaient à table; et il leur reprocha leur incrédulité et la dureté de leur coeur, parce qu`ils n`avaient pas cru ceux qui l`avaient vu ressuscité. 
\verse Puis il leur dit: Allez par tout le monde, et prêchez la bonne nouvelle à toute la création. 
\verse Celui qui croira et qui sera baptisé sera sauvé, mais celui qui ne croira pas sera condamné. 
\verse Voici les miracles qui accompagneront ceux qui auront cru: en mon nom, ils chasseront les démons; ils parleront de nouvelles langues; 
\verse ils saisiront des serpents; s`ils boivent quelque breuvage mortel, il ne leur feront point de mal; ils imposeront les mains aux malades, et les malades, seront guéris. 
\verse Le Seigneur, après leur avoir parlé, fut enlevé au ciel, et il s`assit à la droite de Dieu. 
\verse Et ils s`en allèrent prêcher partout. Le Seigneur travaillait avec eux, et confirmait la parole par les miracles qui l`accompagnaient. 
