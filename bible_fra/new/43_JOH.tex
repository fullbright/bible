\book[Livre de Judith]{}


\chapter
\verse Au commencement était la Parole, et la Parole était avec Dieu, et la Parole était Dieu. 
\verse Elle était au commencement avec Dieu. 
\verse Toutes choses ont été faites par elle, et rien de ce qui a été fait n`a été fait sans elle. 
\verse En elle était la vie, et la vie était la lumière des hommes. 
\verse La lumière luit dans les ténèbres, et les ténèbres ne l`ont point reçue. 
\verse Il y eut un homme envoyé de Dieu: son nom était Jean. 
\verse Il vint pour servir de témoin, pour rendre témoignage à la lumière, afin que tous crussent par lui. 
\verse Il n`était pas la lumière, mais il parut pour rendre témoignage à la lumière. 
\verse Cette lumière était la véritable lumière, qui, en venant dans le monde, éclaire tout homme. 
\verse Elle était dans le monde, et le monde a été fait par elle, et le monde ne l`a point connue. 
\verse Elle est venue chez les siens, et les siens ne l`ont point reçue. 
\verse Mais à tous ceux qui l`ont reçue, à ceux qui croient en son nom, elle a donné le pouvoir de devenir enfants de Dieu, lesquels sont nés, 
\verse non du sang, ni de la volonté de la chair, ni de la volonté de l`homme, mais de Dieu. 
\verse Et la parole a été faite chair, et elle a habité parmi nous, pleine de grâce et de vérité; et nous avons contemplé sa gloire, une gloire comme la gloire du Fils unique venu du Père. 
\verse Jean lui a rendu témoignage, et s`est écrié: C`est celui dont j`ai dit: Celui qui vient après moi m`a précédé, car il était avant moi. 
\verse Et nous avons tous reçu de sa plénitude, et grâce pour grâce; 
\verse car la loi a été donnée par Moïse, la grâce et la vérité sont venues par Jésus Christ. 
\verse Personne n`a jamais vu Dieu; le Fils unique, qui est dans le sein du Père, est celui qui l`a fait connaître. 
\verse Voici le témoignage de Jean, lorsque les Juifs envoyèrent de Jérusalem des sacrificateurs et des Lévites, pour lui demander: Toi, qui es-tu? 
\verse Il déclara, et ne le nia point, il déclara qu`il n`était pas le Christ. 
\verse Et ils lui demandèrent: Quoi donc? es-tu Élie? Et il dit: Je ne le suis point. Es-tu le prophète? Et il répondit: Non. 
\verse Ils lui dirent alors: Qui es-tu? afin que nous donnions une réponse à ceux qui nous ont envoyés. Que dis-tu de toi-même? 
\verse Moi, dit-il, je suis la voix de celui qui crie dans le désert: Aplanissez le chemin du Seigneur, comme a dit Ésaïe, le prophète. 
\verse Ceux qui avaient été envoyés étaient des pharisiens. 
\verse Ils lui firent encore cette question: Pourquoi donc baptises-tu, si tu n`es pas le Christ, ni Élie, ni le prophète? 
\verse Jean leur répondit: Moi, je baptise d`eau, mais au milieu de vous il y a quelqu`un que vous ne connaissez pas, qui vient après moi; 
\verse je ne suis pas digne de délier la courroie de ses souliers. 
\verse Ces choses se passèrent à Béthanie, au delà du Jourdain, où Jean baptisait. 
\verse Le lendemain, il vit Jésus venant à lui, et il dit: Voici l`Agneau de Dieu, qui ôte le péché du monde. 
\verse C`est celui dont j`ai dit: Après moi vient un homme qui m`a précédé, car il était avant moi. 
\verse Je ne le connaissais pas, mais c`est afin qu`il fût manifesté à Israël que je suis venu baptiser d`eau. 
\verse Jean rendit ce témoignage: J`ai vu l`Esprit descendre du ciel comme une colombe et s`arrêter sur lui. 
\verse Je ne le connaissais pas, mais celui qui m`a envoyé baptiser d`eau, celui-là m`a dit: Celui sur qui tu verras l`Esprit descendre et s`arrêter, c`est celui qui baptise du Saint Esprit. 
\verse Et j`ai vu, et j`ai rendu témoignage qu`il est le Fils de Dieu. 
\verse Le lendemain, Jean était encore là, avec deux de ses disciples; 
\verse et, ayant regardé Jésus qui passait, il dit: Voilà l`Agneau de Dieu. 
\verse Les deux disciples l`entendirent prononcer ces paroles, et ils suivirent Jésus. 
\verse Jésus se retourna, et voyant qu`ils le suivaient, il leur dit: Que cherchez-vous? Ils lui répondirent: Rabbi (ce qui signifie Maître), où demeures-tu? 
\verse Venez, leur dit-il, et voyez. Ils allèrent, et ils virent où il demeurait; et ils restèrent auprès de lui ce jour-là. C`était environ la dixième heure. 
\verse André, frère de Simon Pierre, était l`un des deux qui avaient entendu les paroles de Jean, et qui avaient suivi Jésus. 
\verse Ce fut lui qui rencontra le premier son frère Simon, et il lui dit: Nous avons trouvé le Messie (ce qui signifie Christ). 
\verse Et il le conduisit vers Jésus. Jésus, l`ayant regardé, dit: Tu es Simon, fils de Jonas; tu seras appelé Céphas (ce qui signifie Pierre). 
\verse Le lendemain, Jésus voulut se rendre en Galilée, et il rencontra Philippe. Il lui dit: Suis-moi. 
\verse Philippe était de Bethsaïda, de la ville d`André et de Pierre. 
\verse Philippe rencontra Nathanaël, et lui dit: Nous avons trouvé celui de qui Moïse a écrit dans la loi et dont les prophètes ont parlé, Jésus de Nazareth, fils de Joseph. 
\verse Nathanaël lui dit: Peut-il venir de Nazareth quelque chose de bon? Philippe lui répondit: Viens, et vois. 
\verse Jésus, voyant venir à lui Nathanaël, dit de lui: Voici vraiment un Israélite, dans lequel il n`y a point de fraude. 
\verse D`où me connais-tu? lui dit Nathanaël. Jésus lui répondit: Avant que Philippe t`appelât, quand tu étais sous le figuier, je t`ai vu. 
\verse Nathanaël répondit et lui dit: Rabbi, tu es le Fils de Dieu, tu es le roi d`Israël. 
\verse Jésus lui répondit: Parce que je t`ai dit que je t`ai vu sous le figuier, tu crois; tu verras de plus grandes choses que celles-ci. 
\verse Et il lui dit: En vérité, en vérité, vous verrez désormais le ciel ouvert et les anges de Dieu monter et descendre sur le Fils de l`homme. 

\chapter
\verse Trois jours après, il y eut des noces à Cana en Galilée. La mère de Jésus était là, 
\verse et Jésus fut aussi invité aux noces avec ses disciples. 
\verse Le vin ayant manqué, la mère de Jésus lui dit: Ils n`ont plus de vin. 
\verse Jésus lui répondit: Femme, qu`y a-t-il entre moi et toi? Mon heure n`est pas encore venue. 
\verse Sa mère dit aux serviteurs: Faites ce qu`il vous dira. 
\verse Or, il y avait là six vases de pierre, destinés aux purifications des Juifs, et contenant chacun deux ou trois mesures. 
\verse Jésus leur dit: Remplissez d`eau ces vases. Et ils les remplirent jusqu`au bord. 
\verse Puisez maintenant, leur dit-il, et portez-en à l`ordonnateur du repas. Et ils en portèrent. 
\verse Quand l`ordonnateur du repas eut goûté l`eau changée en vin, -ne sachant d`où venait ce vin, tandis que les serviteurs, qui avaient puisé l`eau, le savaient bien, -il appela l`époux, 
\verse et lui dit: Tout homme sert d`abord le bon vin, puis le moins bon après qu`on s`est enivré; toi, tu as gardé le bon vin jusqu`à présent. 
\verse Tel fut, à Cana en Galilée, le premier des miracles que fit Jésus. Il manifesta sa gloire, et ses disciples crurent en lui. 
\verse Après cela, il descendit à Capernaüm, avec sa mère, ses frères et ses disciples, et ils n`y demeurèrent que peu de jours. 
\verse La Pâque des Juifs était proche, et Jésus monta à Jérusalem. 
\verse Il trouva dans le temple les vendeurs de boeufs, de brebis et de pigeons, et les changeurs assis. 
\verse Ayant fait un fouet avec des cordes, il les chassa tous du temple, ainsi que les brebis et les boeufs; il dispersa la monnaie des changeurs, et renversa les tables; 
\verse et il dit aux vendeurs de pigeons: Otez cela d`ici, ne faites pas de la maison de mon Père une maison de trafic. 
\verse Ses disciples se souvinrent qu`il est écrit: Le zèle de ta maison me dévore. 
\verse Les Juifs, prenant la parole, lui dirent: Quel miracle nous montres-tu, pour agir de la sorte? 
\verse Jésus leur répondit: Détruisez ce temple, et en trois jours je le relèverai. 
\verse Les Juifs dirent: Il a fallu quarante-six ans pour bâtir ce temple, et toi, en trois jours tu le relèveras! 
\verse Mais il parlait du temple de son corps. 
\verse C`est pourquoi, lorsqu`il fut ressuscité des morts, ses disciples se souvinrent qu`il avait dit cela, et ils crurent à l`Écriture et à la parole que Jésus avait dite. 
\verse Pendant que Jésus était à Jérusalem, à la fête de Pâque, plusieurs crurent en son nom, voyant les miracles qu`il faisait. 
\verse Mais Jésus ne se fiait point à eux, parce qu`il les connaissait tous, 
\verse et parce qu`il n`avait pas besoin qu`on lui rendît témoignage d`aucun homme; car il savait lui-même ce qui était dans l`homme. 

\chapter
\verse Mais il y eut un homme d`entre les pharisiens, nommé Nicodème, un chef des Juifs, 
\verse qui vint, lui, auprès de Jésus, de nuit, et lui dit: Rabbi, nous savons que tu es un docteur venu de Dieu; car personne ne peut faire ces miracles que tu fais, si Dieu n`est avec lui. 
\verse Jésus lui répondit: En vérité, en vérité, je te le dis, si un homme ne naît de nouveau, il ne peut voir le royaume de Dieu. 
\verse Nicodème lui dit: Comment un homme peut-il naître quand il est vieux? Peut-il rentrer dans le sein de sa mère et naître? 
\verse Jésus répondit: En vérité, en vérité, je te le dis, si un homme ne naît d`eau et d`Esprit, il ne peut entrer dans le royaume de Dieu. 
\verse Ce qui est né de la chair est chair, et ce qui est né de l`Esprit est Esprit. 
\verse Ne t`étonne pas que je t`aie dit: Il faut que vous naissiez de nouveau. 
\verse Le vent souffle où il veut, et tu en entends le bruit; mais tu ne sais d`où il vient, ni où il va. Il en est ainsi de tout homme qui est né de l`Esprit. 
\verse Nicodème lui dit: Comment cela peut-il se faire? 
\verse Jésus lui répondit: Tu es le docteur d`Israël, et tu ne sais pas ces choses! 
\verse En vérité, en vérité, je te le dis, nous disons ce que nous savons, et nous rendons témoignage de ce que nous avons vu; et vous ne recevez pas notre témoignage. 
\verse Si vous ne croyez pas quand je vous ai parlé des choses terrestres, comment croirez-vous quand je vous parlerai des choses célestes? 
\verse Personne n`est monté au ciel, si ce n`est celui qui est descendu du ciel, le Fils de l`homme qui est dans le ciel. 
\verse Et comme Moïse éleva le serpent dans le désert, il faut de même que le Fils de l`homme soit élevé, 
\verse afin que quiconque croit en lui ait la vie éternelle. 
\verse Car Dieu a tant aimé le monde qu`il a donné son Fils unique, afin que quiconque croit en lui ne périsse point, mais qu`il ait la vie éternelle. 
\verse Dieu, en effet, n`a pas envoyé son Fils dans le monde pour qu`il juge le monde, mais pour que le monde soit sauvé par lui. 
\verse Celui qui croit en lui n`est point jugé; mais celui qui ne croit pas est déjà jugé, parce qu`il n`a pas cru au nom du Fils unique de Dieu. 
\verse Et ce jugement c`est que, la lumière étant venue dans le monde, les hommes ont préféré les ténèbres à la lumière, parce que leurs oeuvres étaient mauvaises. 
\verse Car quiconque fait le mal hait la lumière, et ne vient point à la lumière, de peur que ses oeuvres ne soient dévoilées; 
\verse mais celui qui agit selon la vérité vient à la lumière, afin que ses oeuvres soient manifestées, parce qu`elles sont faites en Dieu. 
\verse Après cela, Jésus, accompagné de ses disciples, se rendit dans la terre de Judée; et là il demeurait avec eux, et il baptisait. 
\verse Jean aussi baptisait à Énon, près de Salim, parce qu`il y avait là beaucoup d`eau; et on y venait pour être baptisé. 
\verse Car Jean n`avait pas encore été mis en prison. 
\verse Or, il s`éleva de la part des disciples de Jean une dispute avec un Juif touchant la purification. 
\verse Ils vinrent trouver Jean, et lui dirent: Rabbi, celui qui était avec toi au delà du Jourdain, et à qui tu as rendu témoignage, voici, il baptise, et tous vont à lui. 
\verse Jean répondit: Un homme ne peut recevoir que ce qui lui a été donné du ciel. 
\verse Vous-mêmes m`êtes témoins que j`ai dit: Je ne suis pas le Christ, mais j`ai été envoyé devant lui. 
\verse Celui à qui appartient l`épouse, c`est l`époux; mais l`ami de l`époux, qui se tient là et qui l`entend, éprouve une grande joie à cause de la voix de l`époux: aussi cette joie, qui est la mienne, est parfaite. 
\verse Il faut qu`il croisse, et que je diminue. 
\verse Celui qui vient d`en haut est au-dessus de tous; celui qui est de la terre est de la terre, et il parle comme étant de la terre. Celui qui vient du ciel est au-dessus de tous, 
\verse il rend témoignage de ce qu`il a vu et entendu, et personne ne reçoit son témoignage. 
\verse Celui qui a reçu son témoignage a certifié que Dieu est vrai; 
\verse car celui que Dieu a envoyé dit les paroles de Dieu, parce que Dieu ne lui donne pas l`Esprit avec mesure. 
\verse Le Père aime le Fils, et il a remis toutes choses entre ses mains. 
\verse Celui qui croit au Fils a la vie éternelle; celui qui ne croit pas au Fils ne verra point la vie, mais la colère de Dieu demeure sur lui. 

\chapter
\verse Le Seigneur sut que les pharisiens avaient appris qu`il faisait et baptisait plus de disciples que Jean. 
\verse Toutefois Jésus ne baptisait pas lui-même, mais c`étaient ses disciples. 
\verse Alors il quitta la Judée, et retourna en Galilée. 
\verse Comme il fallait qu`il passât par la Samarie, 
\verse il arriva dans une ville de Samarie, nommée Sychar, près du champ que Jacob avait donné à Joseph, son fils. 
\verse Là se trouvait le puits de Jacob. Jésus, fatigué du voyage, était assis au bord du puits. C`était environ la sixième heure. 
\verse Une femme de Samarie vint puiser de l`eau. Jésus lui dit: Donne-moi à boire. 
\verse Car ses disciples étaient allés à la ville pour acheter des vivres. 
\verse La femme samaritaine lui dit: Comment toi, qui es Juif, me demandes-tu à boire, à moi qui suis une femme samaritaine? -Les Juifs, en effet, n`ont pas de relations avec les Samaritains. - 
\verse Jésus lui répondit: Si tu connaissais le don de Dieu et qui est celui qui te dit: Donne-moi à boire! tu lui aurais toi-même demandé à boire, et il t`aurait donné de l`eau vive. 
\verse Seigneur, lui dit la femme, tu n`as rien pour puiser, et le puits est profond; d`où aurais-tu donc cette eau vive? 
\verse Es-tu plus grand que notre père Jacob, qui nous a donné ce puits, et qui en a bu lui-même, ainsi que ses fils et ses troupeaux? 
\verse Jésus lui répondit: Quiconque boit de cette eau aura encore soif; 
\verse mais celui qui boira de l`eau que je lui donnerai n`aura jamais soif, et l`eau que je lui donnerai deviendra en lui une source d`eau qui jaillira jusque dans la vie éternelle. 
\verse La femme lui dit: Seigneur, donne-moi cette eau, afin que je n`aie plus soif, et que je ne vienne plus puiser ici. 
\verse Va, lui dit Jésus, appelle ton mari, et viens ici. 
\verse La femme répondit: Je n`ai point de mari. Jésus lui dit: Tu as eu raison de dire: Je n`ai point de mari. 
\verse Car tu as eu cinq maris, et celui que tu as maintenant n`est pas ton mari. En cela tu as dit vrai. 
\verse Seigneur, lui dit la femme, je vois que tu es prophète. 
\verse Nos pères ont adoré sur cette montagne; et vous dites, vous, que le lieu où il faut adorer est à Jérusalem. 
\verse Femme, lui dit Jésus, crois-moi, l`heure vient où ce ne sera ni sur cette montagne ni à Jérusalem que vous adorerez le Père. 
\verse Vous adorez ce que vous ne connaissez pas; nous, nous adorons ce que nous connaissons, car le salut vient des Juifs. 
\verse Mais l`heure vient, et elle est déjà venue, où les vrais adorateurs adoreront le Père en esprit et en vérité; car ce sont là les adorateurs que le Père demande. 
\verse Dieu est Esprit, et il faut que ceux qui l`adorent l`adorent en esprit et en vérité. 
\verse La femme lui dit: Je sais que le Messie doit venir (celui qu`on appelle Christ); quand il sera venu, il nous annoncera toutes choses. 
\verse Jésus lui dit: Je le suis, moi qui te parle. 
\verse Là-dessus arrivèrent ses disciples, qui furent étonnés de ce qu`il parlait avec une femme. Toutefois aucun ne dit: Que demandes-tu? ou: De quoi parles-tu avec elle? 
\verse Alors la femme, ayant laissé sa cruche, s`en alla dans la ville, et dit aux gens: 
\verse Venez voir un homme qui m`a dit tout ce que j`ai fait; ne serait-ce point le Christ? 
\verse Ils sortirent de la ville, et ils vinrent vers lui. 
\verse Pendant ce temps, les disciples le pressaient de manger, disant: Rabbi, mange. 
\verse Mais il leur dit: J`ai à manger une nourriture que vous ne connaissez pas. 
\verse Les disciples se disaient donc les uns aux autres: Quelqu`un lui aurait-il apporté à manger? 
\verse Jésus leur dit: Ma nourriture est de faire la volonté de celui qui m`a envoyé, et d`accomplir son oeuvre. 
\verse Ne dites-vous pas qu`il y a encore quatre mois jusqu`à la moisson? Voici, je vous le dis, levez les yeux, et regardez les champs qui déjà blanchissent pour la moisson. 
\verse Celui qui moissonne reçoit un salaire, et amasse des fruits pour la vie éternelle, afin que celui qui sème et celui qui moissonne se réjouissent ensemble. 
\verse Car en ceci ce qu`on dit est vrai: Autre est celui qui sème, et autre celui qui moissonne. 
\verse Je vous ai envoyés moissonner ce que vous n`avez pas travaillé; d`autres ont travaillé, et vous êtes entrés dans leur travail. 
\verse Plusieurs Samaritains de cette ville crurent en Jésus à cause de cette déclaration formelle de la femme: Il m`a dit tout ce que j`ai fait. 
\verse Aussi, quand les Samaritains vinrent le trouver, ils le prièrent de rester auprès d`eux. Et il resta là deux jours. 
\verse Un beaucoup plus grand nombre crurent à cause de sa parole; 
\verse et ils disaient à la femme: Ce n`est plus à cause de ce que tu as dit que nous croyons; car nous l`avons entendu nous-mêmes, et nous savons qu`il est vraiment le Sauveur du monde. 
\verse Après ces deux jours, Jésus partit de là, pour se rendre en Galilée; 
\verse car il avait déclaré lui-même qu`un prophète n`est pas honoré dans sa propre patrie. 
\verse Lorsqu`il arriva en Galilée, il fut bien reçu des Galiléens, qui avaient vu tout ce qu`il avait fait à Jérusalem pendant la fête; car eux aussi étaient allés à la fête. 
\verse Il retourna donc à Cana en Galilée, où il avait changé l`eau en vin. Il y avait à Capernaüm un officier du roi, dont le fils était malade. 
\verse Ayant appris que Jésus était venu de Judée en Galilée, il alla vers lui, et le pria de descendre et de guérir son fils, qui était près de mourir. 
\verse Jésus lui dit: Si vous ne voyez des miracles et des prodiges, vous ne croyez point. 
\verse L`officier du roi lui dit: Seigneur, descends avant que mon enfant meure. 
\verse Va, lui dit Jésus, ton fils vit. Et cet homme crut à la parole que Jésus lui avait dite, et il s`en alla. 
\verse Comme déjà il descendait, ses serviteurs venant à sa rencontre, lui apportèrent cette nouvelle: Ton enfant vit. 
\verse Il leur demanda à quelle heure il s`était trouvé mieux; et ils lui dirent: Hier, à la septième heure, la fièvre l`a quitté. 
\verse Le père reconnut que c`était à cette heure-là que Jésus lui avait dit: Ton fils vit. Et il crut, lui et toute sa maison. 
\verse Jésus fit encore ce second miracle lorsqu`il fut venu de Judée en Galilée. 

\chapter
\verse Après cela, il y eut une fête des Juifs, et Jésus monta à Jérusalem. 
\verse Or, à Jérusalem, près de la porte des brebis, il y a une piscine qui s`appelle en hébreu Béthesda, et qui a cinq portiques. 
\verse Sous ces portiques étaient couchés en grand nombre des malades, des aveugles, des boiteux, des paralytiques, qui attendaient le mouvement de l`eau; 
\verse car un ange descendait de temps en temps dans la piscine, et agitait l`eau; et celui qui y descendait le premier après que l`eau avait été agitée était guéri, quelle que fût sa maladie. 
\verse Là se trouvait un homme malade depuis trente-huit ans. 
\verse Jésus, l`ayant vu couché, et sachant qu`il était malade depuis longtemps, lui dit: Veux-tu être guéri? 
\verse Le malade lui répondit: Seigneur, je n`ai personne pour me jeter dans la piscine quand l`eau est agitée, et, pendant que j`y vais, un autre descend avant moi. 
\verse Lève-toi, lui dit Jésus, prends ton lit, et marche. 
\verse Aussitôt cet homme fut guéri; il prit son lit, et marcha. 
\verse C`était un jour de sabbat. Les Juifs dirent donc à celui qui avait été guéri: C`est le sabbat; il ne t`est pas permis d`emporter ton lit. 
\verse Il leur répondit: Celui qui m`a guéri m`a dit: Prends ton lit, et marche. 
\verse Ils lui demandèrent: Qui est l`homme qui t`a dit: Prends ton lit, et marche? 
\verse Mais celui qui avait été guéri ne savait pas qui c`était; car Jésus avait disparu de la foule qui était en ce lieu. 
\verse Depuis, Jésus le trouva dans le temple, et lui dit: Voici, tu as été guéri; ne pèche plus, de peur qu`il ne t`arrive quelque chose de pire. 
\verse Cet homme s`en alla, et annonça aux Juifs que c`était Jésus qui l`avait guéri. 
\verse C`est pourquoi les Juifs poursuivaient Jésus, parce qu`il faisait ces choses le jour du sabbat. 
\verse Mais Jésus leur répondit: Mon Père agit jusqu`à présent; moi aussi, j`agis. 
\verse A cause de cela, les Juifs cherchaient encore plus à le faire mourir, non seulement parce qu`il violait le sabbat, mais parce qu`il appelait Dieu son propre Père, se faisant lui-même égal à Dieu. 
\verse Jésus reprit donc la parole, et leur dit: En vérité, en vérité, je vous le dis, le Fils ne peut rien faire de lui-même, il ne fait que ce qu`il voit faire au Père; et tout ce que le Père fait, le Fils aussi le fait pareillement. 
\verse Car le Père aime le Fils, et lui montre tout ce qu`il fait; et il lui montrera des oeuvres plus grandes que celles-ci, afin que vous soyez dans l`étonnement. 
\verse Car, comme le Père ressuscite les morts et donne la vie, ainsi le Fils donne la vie à qui il veut. 
\verse Le Père ne juge personne, mais il a remis tout jugement au Fils, 
\verse afin que tous honorent le Fils comme ils honorent le Père. Celui qui n`honore pas le Fils n`honore pas le Père qui l`a envoyé. 
\verse En vérité, en vérité, je vous le dis, celui qui écoute ma parole, et qui croit à celui qui m`a envoyé, a la vie éternelle et ne vient point en jugement, mais il est passé de la mort à la vie. 
\verse En vérité, en vérité, je vous le dis, l`heure vient, et elle est déjà venue, où les morts entendront la voix du Fils de Dieu; et ceux qui l`auront entendue vivront. 
\verse Car, comme le Père a la vie en lui-même, ainsi il a donné au Fils d`avoir la vie en lui-même. 
\verse Et il lui a donné le pouvoir de juger, parce qu`il est Fils de l`homme. 
\verse Ne vous étonnez pas de cela; car l`heure vient où tous ceux qui sont dans les sépulcres entendront sa voix, et en sortiront. 
\verse Ceux qui auront fait le bien ressusciteront pour la vie, mais ceux qui auront fait le mal ressusciteront pour le jugement. 
\verse Je ne puis rien faire de moi-même: selon que j`entends, je juge; et mon jugement est juste, parce que je ne cherche pas ma volonté, mais la volonté de celui qui m`a envoyé. 
\verse Si c`est moi qui rends témoignage de moi-même, mon témoignage n`est pas vrai. 
\verse Il y en a un autre qui rend témoignage de moi, et je sais que le témoignage qu`il rend de moi est vrai. 
\verse Vous avez envoyé vers Jean, et il a rendu témoignage à la vérité. 
\verse Pour moi ce n`est pas d`un homme que je reçois le témoignage; mais je dis ceci, afin que vous soyez sauvés. 
\verse Jean était la lampe qui brûle et qui luit, et vous avez voulu vous réjouir une heure à sa lumière. 
\verse Moi, j`ai un témoignage plus grand que celui de Jean; car les oeuvres que le Père m`a donné d`accomplir, ces oeuvres mêmes que je fais, témoignent de moi que c`est le Père qui m`a envoyé. 
\verse Et le Père qui m`a envoyé a rendu lui-même témoignage de moi. Vous n`avez jamais entendu sa voix, vous n`avez point vu sa face, 
\verse et sa parole ne demeure point en vous, parce que vous ne croyez pas à celui qu`il a envoyé. 
\verse Vous sondez les Écritures, parce que vous pensez avoir en elles la vie éternelle: ce sont elles qui rendent témoignage de moi. 
\verse Et vous ne voulez pas venir à moi pour avoir la vie! 
\verse Je ne tire pas ma gloire des hommes. 
\verse Mais je sais que vous n`avez point en vous l`amour de Dieu. 
\verse Je suis venu au nom de mon Père, et vous ne me recevez pas; si un autre vient en son propre nom, vous le recevrez. 
\verse Comment pouvez-vous croire, vous qui tirez votre gloire les uns des autres, et qui ne cherchez point la gloire qui vient de Dieu seul? 
\verse Ne pensez pas que moi je vous accuserai devant le Père; celui qui vous accuse, c`est Moïse, en qui vous avez mis votre espérance. 
\verse Car si vous croyiez Moïse, vous me croiriez aussi, parce qu`il a écrit de moi. 
\verse Mais si vous ne croyez pas à ses écrits, comment croirez-vous à mes paroles? 

\chapter
\verse Après cela, Jésus s`en alla de l`autre côté de la mer de Galilée, de Tibériade. 
\verse Une grande foule le suivait, parce qu`elle voyait les miracles qu`il opérait sur les malades. 
\verse Jésus monta sur la montagne, et là il s`assit avec ses disciples. 
\verse Or, la Pâque était proche, la fête des Juifs. 
\verse Ayant levé les yeux, et voyant qu`une grande foule venait à lui, Jésus dit à Philippe: Où achèterons-nous des pains, pour que ces gens aient à manger? 
\verse Il disait cela pour l`éprouver, car il savait ce qu`il allait faire. 
\verse Philippe lui répondit: Les pains qu`on aurait pour deux cents deniers ne suffiraient pas pour que chacun en reçût un peu. 
\verse Un de ses disciples, André, frère de Simon Pierre, lui dit: 
\verse Il y a ici un jeune garçon qui a cinq pains d`orge et deux poissons; mais qu`est-ce que cela pour tant de gens? 
\verse Jésus dit: Faites-les asseoir. Il y avait dans ce lieu beaucoup d`herbe. Ils s`assirent donc, au nombre d`environ cinq mille hommes. 
\verse Jésus prit les pains, rendit grâces, et les distribua à ceux qui étaient assis; il leur donna de même des poissons, autant qu`ils en voulurent. 
\verse Lorsqu`ils furent rassasiés, il dit à ses disciples: Ramassez les morceaux qui restent, afin que rien ne se perde. 
\verse Ils les ramassèrent donc, et ils remplirent douze paniers avec les morceaux qui restèrent des cinq pains d`orge, après que tous eurent mangé. 
\verse Ces gens, ayant vu le miracle que Jésus avait fait, disaient: Celui-ci est vraiment le prophète qui doit venir dans le monde. 
\verse Et Jésus, sachant qu`ils allaient venir l`enlever pour le faire roi, se retira de nouveau sur la montagne, lui seul. 
\verse Quand le soir fut venu, ses disciples descendirent au bord de la mer. 
\verse Étant montés dans une barque, ils traversaient la mer pour se rendre à Capernaüm. Il faisait déjà nuit, et Jésus ne les avait pas encore rejoints. 
\verse Il soufflait un grand vent, et la mer était agitée. 
\verse Après avoir ramé environ vingt-cinq ou trente stades, ils virent Jésus marchant sur la mer et s`approchant de la barque. Et ils eurent peur. 
\verse Mais Jésus leur dit: C`est moi; n`ayez pas peur! 
\verse Ils voulaient donc le prendre dans la barque, et aussitôt la barque aborda au lieu où ils allaient. 
\verse La foule qui était restée de l`autre côté de la mer avait remarqué qu`il ne se trouvait là qu`une seule barque, et que Jésus n`était pas monté dans cette barque avec ses disciples, mais qu`ils étaient partis seuls. 
\verse Le lendemain, comme d`autres barques étaient arrivées de Tibériade près du lieu où ils avaient mangé le pain après que le Seigneur eut rendu grâces, 
\verse les gens de la foule, ayant vu que ni Jésus ni ses disciples n`étaient là, montèrent eux-mêmes dans ces barques et allèrent à Capernaüm à la recherche de Jésus. 
\verse Et l`ayant trouvé au delà de la mer, ils lui dirent: Rabbi, quand es-tu venu ici? 
\verse Jésus leur répondit: En vérité, en vérité, je vous le dis, vous me cherchez, non parce que vous avez vu des miracles, mais parce que vous avez mangé des pains et que vous avez été rassasiés. 
\verse Travaillez, non pour la nourriture qui périt, mais pour celle qui subsiste pour la vie éternelle, et que le Fils de l`homme vous donnera; car c`est lui que le Père, que Dieu a marqué de son sceau. 
\verse Ils lui dirent: Que devons-nous faire, pour faire les oeuvres de Dieu? 
\verse Jésus leur répondit: L`oeuvre de Dieu, c`est que vous croyiez en celui qu`il a envoyé. 
\verse Quel miracle fais-tu donc, lui dirent-ils, afin que nous le voyions, et que nous croyions en toi? Que fais-tu? 
\verse Nos pères ont mangé la manne dans le désert, selon ce qui est écrit: Il leur donna le pain du ciel à manger. 
\verse Jésus leur dit: En vérité, en vérité, je vous le dis, Moïse ne vous a pas donné le pain du ciel, mais mon Père vous donne le vrai pain du ciel; 
\verse car le pain de Dieu, c`est celui qui descend du ciel et qui donne la vie au monde. 
\verse Ils lui dirent: Seigneur, donne-nous toujours ce pain. 
\verse Jésus leur dit: Je suis le pain de vie. Celui qui vient à moi n`aura jamais faim, et celui qui croit en moi n`aura jamais soif. 
\verse Mais, je vous l`ai dit, vous m`avez vu, et vous ne croyez point. 
\verse Tous ceux que le Père me donne viendront à moi, et je ne mettrai pas dehors celui qui vient à moi; 
\verse car je suis descendu du ciel pour faire, non ma volonté, mais la volonté de celui qui m`a envoyé. 
\verse Or, la volonté de celui qui m`a envoyé, c`est que je ne perde rien de tout ce qu`il m`a donné, mais que je le ressuscite au dernier jour. 
\verse La volonté de mon Père, c`est que quiconque voit le Fils et croit en lui ait la vie éternelle; et je le ressusciterai au dernier jour. 
\verse Les Juifs murmuraient à son sujet, parce qu`il avait dit: Je suis le pain qui est descendu du ciel. 
\verse Et ils disaient: N`est-ce pas là Jésus, le fils de Joseph, celui dont nous connaissons le père et la mère? Comment donc dit-il: Je suis descendu du ciel? 
\verse Jésus leur répondit: Ne murmurez pas entre vous. 
\verse Nul ne peut venir à moi, si le Père qui m`a envoyé ne l`attire; et je le ressusciterai au dernier jour. 
\verse Il est écrit dans les prophètes: Ils seront tous enseignés de Dieu. Ainsi quiconque a entendu le Père et a reçu son enseignement vient à moi. 
\verse C`est que nul n`a vu le Père, sinon celui qui vient de Dieu; celui-là a vu le Père. 
\verse En vérité, en vérité, je vous le dis, celui qui croit en moi a la vie éternelle. 
\verse Je suis le pain de vie. 
\verse Vos pères ont mangé la manne dans le désert, et ils sont morts. 
\verse C`est ici le pain qui descend du ciel, afin que celui qui en mange ne meure point. 
\verse Je suis le pain vivant qui est descendu du ciel. Si quelqu`un mange de ce pain, il vivra éternellement; et le pain que je donnerai, c`est ma chair, que je donnerai pour la vie du monde. 
\verse Là-dessus, les Juifs disputaient entre eux, disant: Comment peut-il nous donner sa chair à manger? 
\verse Jésus leur dit: En vérité, en vérité, je vous le dis, si vous ne mangez la chair du Fils de l`homme, et si vous ne buvez son sang, vous n`avez point la vie en vous-mêmes. 
\verse Celui qui mange ma chair et qui boit mon sang a la vie éternelle; et je le ressusciterai au dernier jour. 
\verse Car ma chair est vraiment une nourriture, et mon sang est vraiment un breuvage. 
\verse Celui qui mange ma chair et qui boit mon sang demeure en moi, et je demeure en lui. 
\verse Comme le Père qui est vivant m`a envoyé, et que je vis par le Père, ainsi celui qui me mange vivra par moi. 
\verse C`est ici le pain qui est descendu du ciel. Il n`en est pas comme de vos pères qui ont mangé la manne et qui sont morts: celui qui mange ce pain vivra éternellement. 
\verse Jésus dit ces choses dans la synagogue, enseignant à Capernaüm. 
\verse Plusieurs de ses disciples, après l`avoir entendu, dirent: Cette parole est dure; qui peut l`écouter? 
\verse Jésus, sachant en lui-même que ses disciples murmuraient à ce sujet, leur dit: Cela vous scandalise-t-il? 
\verse Et si vous voyez le Fils de l`homme monter où il était auparavant?... 
\verse C`est l`esprit qui vivifie; la chair ne sert de rien. Les paroles que je vous ai dites sont esprit et vie. 
\verse Mais il en est parmi vous quelques-uns qui ne croient point. Car Jésus savait dès le commencement qui étaient ceux qui ne croyaient point, et qui était celui qui le livrerait. 
\verse Et il ajouta: C`est pourquoi je vous ai dit que nul ne peut venir à moi, si cela ne lui a été donné par le Père. 
\verse Dès ce moment, plusieurs de ses disciples se retirèrent, et ils n`allaient plus avec lui. 
\verse Jésus donc dit aux douze: Et vous, ne voulez-vous pas aussi vous en aller? 
\verse Simon Pierre lui répondit: Seigneur, à qui irions-nous? Tu as les paroles de la vie éternelle. 
\verse Et nous avons cru et nous avons connu que tu es le Christ, le Saint de Dieu. 
\verse Jésus leur répondit: N`est-ce pas moi qui vous ai choisis, vous les douze? Et l`un de vous est un démon! 
\verse Il parlait de Judas Iscariot, fils de Simon; car c`était lui qui devait le livrer, lui, l`un des douze. 

\chapter
\verse Après cela, Jésus parcourait la Galilée, car il ne voulait pas séjourner en Judée, parce que les Juifs cherchaient à le faire mourir. 
\verse Or, la fête des Juifs, la fête des Tabernacles, était proche. 
\verse Et ses frères lui dirent: Pars d`ici, et va en Judée, afin que tes disciples voient aussi les oeuvres que tu fais. 
\verse Personne n`agit en secret, lorsqu`il désire paraître: si tu fais ces choses, montre-toi toi-même au monde. 
\verse Car ses frères non plus ne croyaient pas en lui. 
\verse Jésus leur dit: Mon temps n`est pas encore venu, mais votre temps est toujours prêt. 
\verse Le monde ne peut vous haïr; moi, il me hait, parce que je rends de lui le témoignage que ses oeuvres sont mauvaises. 
\verse Montez, vous, à cette fête; pour moi, je n`y monte point, parce que mon temps n`est pas encore accompli. 
\verse Après leur avoir dit cela, il resta en Galilée. 
\verse Lorsque ses frères furent montés à la fête, il y monta aussi lui-même, non publiquement, mais comme en secret. 
\verse Les Juifs le cherchaient pendant la fête, et disaient: Où est-il? 
\verse Il y avait dans la foule grande rumeur à son sujet. Les uns disaient: C`est un homme de bien. D`autres disaient: Non, il égare la multitude. 
\verse Personne, toutefois, ne parlait librement de lui, par crainte des Juifs. 
\verse Vers le milieu de la fête, Jésus monta au temple. Et il enseignait. 
\verse Les Juifs s`étonnaient, disant: Comment connaît-il les Écritures, lui qui n`a point étudié? 
\verse Jésus leur répondit: Ma doctrine n`est pas de moi, mais de celui qui m`a envoyé. 
\verse Si quelqu`un veut faire sa volonté, il connaîtra si ma doctrine est de Dieu, ou si je parle de mon chef. 
\verse Celui qui parle de son chef cherche sa propre gloire; mais celui qui cherche la gloire de celui qui l`a envoyé, celui-là est vrai, et il n`y a point d`injustice en lui. 
\verse Moïse ne vous a-t-il pas donné la loi? Et nul de vous n`observe la loi. Pourquoi cherchez-vous à me faire mourir? 
\verse La foule répondit: Tu as un démon. Qui est-ce qui cherche à te faire mourir? 
\verse Jésus leur répondit: J`ai fait une oeuvre, et vous en êtes tous étonnés. 
\verse Moïse vous a donné la circoncision, -non qu`elle vienne de Moïse, car elle vient des patriarches, -et vous circoncisez un homme le jour du sabbat. 
\verse Si un homme reçoit la circoncision le jour du sabbat, afin que la loi de Moïse ne soit pas violée, pourquoi vous irritez-vous contre moi de ce que j`ai guéri un homme tout entier le jour du sabbat? 
\verse Ne jugez pas selon l`apparence, mais jugez selon la justice. 
\verse Quelques habitants de Jérusalem disaient: N`est-ce pas là celui qu`ils cherchent à faire mourir? 
\verse Et voici, il parle librement, et ils ne lui disent rien! Est-ce que vraiment les chefs auraient reconnu qu`il est le Christ? 
\verse Cependant celui-ci, nous savons d`où il est; mais le Christ, quand il viendra, personne ne saura d`où il est. 
\verse Et Jésus, enseignant dans le temple, s`écria: Vous me connaissez, et vous savez d`où je suis! Je ne suis pas venu de moi-même: mais celui qui m`a envoyé est vrai, et vous ne le connaissez pas. 
\verse Moi, je le connais; car je viens de lui, et c`est lui qui m`a envoyé. 
\verse Ils cherchaient donc à se saisir de lui, et personne ne mit la main sur lui, parce que son heure n`était pas encore venue. 
\verse Plusieurs parmi la foule crurent en lui, et ils disaient: Le Christ, quand il viendra, fera-t-il plus de miracles que n`en a fait celui-ci? 
\verse Les pharisiens entendirent la foule murmurant de lui ces choses. Alors les principaux sacrificateurs et les pharisiens envoyèrent des huissiers pour le saisir. 
\verse Jésus dit: Je suis encore avec vous pour un peu de temps, puis je m`en vais vers celui qui m`a envoyé. 
\verse Vous me chercherez et vous ne me trouverez pas, et vous ne pouvez venir où je serai. 
\verse Sur quoi les Juifs dirent entre eux: Où ira-t-il, que nous ne le trouvions pas? Ira-t-il parmi ceux qui sont dispersés chez les Grecs, et enseignera-t-il les Grecs? 
\verse Que signifie cette parole qu`il a dite: Vous me chercherez et vous ne me trouverez pas, et vous ne pouvez venir où je serai? 
\verse Le dernier jour, le grand jour de la fête, Jésus, se tenant debout, s`écria: Si quelqu`un a soif, qu`il vienne à moi, et qu`il boive. 
\verse Celui qui croit en moi, des fleuves d`eau vive couleront de son sein, comme dit l`Écriture. 
\verse Il dit cela de l`Esprit que devaient recevoir ceux qui croiraient en lui; car l`Esprit n`était pas encore, parce que Jésus n`avait pas encore été glorifié. 
\verse Des gens de la foule, ayant entendu ces paroles, disaient: Celui-ci est vraiment le prophète. 
\verse D`autres disaient: C`est le Christ. Et d`autres disaient: Est-ce bien de la Galilée que doit venir le Christ? 
\verse L`Écriture ne dit-elle pas que c`est de la postérité de David, et du village de Bethléhem, où était David, que le Christ doit venir? 
\verse Il y eut donc, à cause de lui, division parmi la foule. 
\verse Quelques-uns d`entre eux voulaient le saisir, mais personne ne mit la main sur lui. 
\verse Ainsi les huissiers retournèrent vers les principaux sacrificateurs et les pharisiens. Et ceux-ci leur dirent: Pourquoi ne l`avez-vous pas amené? 
\verse Les huissiers répondirent: Jamais homme n`a parlé comme cet homme. 
\verse Les pharisiens leur répliquèrent: Est-ce que vous aussi, vous avez été séduits? 
\verse Y a-t-il quelqu`un des chefs ou des pharisiens qui ait cru en lui? 
\verse Mais cette foule qui ne connaît pas la loi, ce sont des maudits! 
\verse Nicodème, qui était venu de nuit vers Jésus, et qui était l`un d`entre eux, leur dit: 
\verse Notre loi condamne-t-elle un homme avant qu`on l`entende et qu`on sache ce qu`il a fait? 
\verse Ils lui répondirent: Es-tu aussi Galiléen? Examine, et tu verras que de la Galilée il ne sort point de prophète. 
\verse Et chacun s`en retourna dans sa maison. 

\chapter
\verse Jésus se rendit à la montagne des oliviers. 
\verse Mais, dès le matin, il alla de nouveau dans le temple, et tout le peuple vint à lui. S`étant assis, il les enseignait. 
\verse Alors les scribes et les pharisiens amenèrent une femme surprise en adultère; 
\verse et, la plaçant au milieu du peuple, ils dirent à Jésus: Maître, cette femme a été surprise en flagrant délit d`adultère. 
\verse Moïse, dans la loi, nous a ordonné de lapider de telles femmes: toi donc, que dis-tu? 
\verse Ils disaient cela pour l`éprouver, afin de pouvoir l`accuser. Mais Jésus, s`étant baissé, écrivait avec le doigt sur la terre. 
\verse Comme ils continuaient à l`interroger, il se releva et leur dit: Que celui de vous qui est sans péché jette le premier la pierre contre elle. 
\verse Et s`étant de nouveau baissé, il écrivait sur la terre. 
\verse Quand ils entendirent cela, accusés par leur conscience, ils se retirèrent un à un, depuis les plus âgés jusqu`aux derniers; et Jésus resta seul avec la femme qui était là au milieu. 
\verse Alors s`étant relevé, et ne voyant plus que la femme, Jésus lui dit: Femme, où sont ceux qui t`accusaient? Personne ne t`a-t-il condamnée? 
\verse Elle répondit: Non, Seigneur. Et Jésus lui dit: Je ne te condamne pas non plus: va, et ne pèche plus. 
\verse Jésus leur parla de nouveau, et dit: Je suis la lumière du monde; celui qui me suit ne marchera pas dans les ténèbres, mais il aura la lumière de la vie. 
\verse Là-dessus, les pharisiens lui dirent: Tu rends témoignage de toi-même; ton témoignage n`est pas vrai. 
\verse Jésus leur répondit: Quoique je rende témoignage de moi-même, mon témoignage est vrai, car je sais d`où je suis venu et où je vais; mais vous, vous ne savez d`où je viens ni où je vais. 
\verse Vous jugez selon la chair; moi, je ne juge personne. 
\verse Et si je juge, mon jugement est vrai, car je ne suis pas seul; mais le Père qui m`a envoyé est avec moi. 
\verse Il est écrit dans votre loi que le témoignage de deux hommes est vrai; 
\verse je rends témoignage de moi-même, et le Père qui m`a envoyé rend témoignage de moi. 
\verse Ils lui dirent donc: Où est ton Père? Jésus répondit: Vous ne connaissez ni moi, ni mon Père. Si vous me connaissiez, vous connaîtriez aussi mon Père. 
\verse Jésus dit ces paroles, enseignant dans le temple, au lieu où était le trésor; et personne ne le saisit, parce que son heure n`était pas encore venue. 
\verse Jésus leur dit encore: Je m`en vais, et vous me chercherez, et vous mourrez dans votre péché; vous ne pouvez venir où je vais. 
\verse Sur quoi les Juifs dirent: Se tuera-t-il lui-même, puisqu`il dit: Vous ne pouvez venir où je vais? 
\verse Et il leur dit: Vous êtes d`en bas; moi, je suis d`en haut. Vous êtes de ce monde; moi, je ne suis pas de ce monde. 
\verse C`est pourquoi je vous ai dit que vous mourrez dans vos péchés; car si vous ne croyez pas ce que je suis, vous mourrez dans vos péchés. 
\verse Qui es-tu? lui dirent-ils. Jésus leur répondit: Ce que je vous dis dès le commencement. 
\verse J`ai beaucoup de choses à dire de vous et à juger en vous; mais celui qui m`a envoyé est vrai, et ce que j`ai entendu de lui, je le dis au monde. 
\verse Ils ne comprirent point qu`il leur parlait du Père. 
\verse Jésus donc leur dit: Quand vous aurez élevé le Fils de l`homme, alors vous connaîtrez ce que je suis, et que je ne fais rien de moi-même, mais que je parle selon ce que le Père m`a enseigné. 
\verse Celui qui m`a envoyé est avec moi; il ne m`a pas laissé seul, parce que je fais toujours ce qui lui est agréable. 
\verse Comme Jésus parlait ainsi, plusieurs crurent en lui. 
\verse Et il dit aux Juifs qui avaient cru en lui: Si vous demeurez dans ma parole, vous êtes vraiment mes disciples; 
\verse vous connaîtrez la vérité, et la vérité vous affranchira. 
\verse Ils lui répondirent: Nous sommes la postérité d`Abraham, et nous ne fûmes jamais esclaves de personne; comment dis-tu: Vous deviendrez libres? 
\verse En vérité, en vérité, je vous le dis, leur répliqua Jésus, quiconque se livre au péché est esclave du péché. 
\verse Or, l`esclave ne demeure pas toujours dans la maison; le fils y demeure toujours. 
\verse Si donc le Fils vous affranchit, vous serez réellement libres. 
\verse Je sais que vous êtes la postérité d`Abraham; mais vous cherchez à me faire mourir, parce que ma parole ne pénètre pas en vous. 
\verse Je dis ce que j`ai vu chez mon Père; et vous, vous faites ce que vous avez entendu de la part de votre père. 
\verse Ils lui répondirent: Notre père, c`est Abraham. Jésus leur dit: Si vous étiez enfants d`Abraham, vous feriez les oeuvres d`Abraham. 
\verse Mais maintenant vous cherchez à me faire mourir, moi qui vous ai dit la vérité que j`ai entendue de Dieu. Cela, Abraham ne l`a point fait. 
\verse Vous faites les oeuvres de votre père. Ils lui dirent: Nous ne sommes pas des enfants illégitimes; nous avons un seul Père, Dieu. 
\verse Jésus leur dit: Si Dieu était votre Père, vous m`aimeriez, car c`est de Dieu que je suis sorti et que je viens; je ne suis pas venu de moi-même, mais c`est lui qui m`a envoyé. 
\verse Pourquoi ne comprenez-vous pas mon langage? Parce que vous ne pouvez écouter ma parole. 
\verse Vous avez pour père le diable, et vous voulez accomplir les désirs de votre père. Il a été meurtrier dès le commencement, et il ne se tient pas dans la vérité, parce qu`il n`y a pas de vérité en lui. Lorsqu`il profère le mensonge, il parle de son propre fonds; car il est menteur et le père du mensonge. 
\verse Et moi, parce que je dis la vérité, vous ne me croyez pas. 
\verse Qui de vous me convaincra de péché? Si je dis la vérité, pourquoi ne me croyez-vous pas? 
\verse Celui qui est de Dieu, écoute les paroles de Dieu; vous n`écoutez pas, parce que vous n`êtes pas de Dieu. 
\verse Les Juifs lui répondirent: N`avons-nous pas raison de dire que tu es un Samaritain, et que tu as un démon? 
\verse Jésus répliqua: Je n`ai point de démon; mais j`honore mon Père, et vous m`outragez. 
\verse Je ne cherche point ma gloire; il en est un qui la cherche et qui juge. 
\verse En vérité, en vérité, je vous le dis, si quelqu`un garde ma parole, il ne verra jamais la mort. 
\verse Maintenant, lui dirent les Juifs, nous connaissons que tu as un démon. Abraham est mort, les prophètes aussi, et tu dis: Si quelqu`un garde ma parole, il ne verra jamais la mort. 
\verse Es-tu plus grand que notre père Abraham, qui est mort? Les prophètes aussi sont morts. Qui prétends-tu être? 
\verse Jésus répondit: Si je me glorifie moi-même, ma gloire n`est rien. C`est mon père qui me glorifie, lui que vous dites être votre Dieu, 
\verse et que vous ne connaissez pas. Pour moi, je le connais; et, si je disais que je ne le connais pas, je serais semblable à vous, un menteur. Mais je le connais, et je garde sa parole. 
\verse Abraham, votre père, a tressailli de joie de ce qu`il verrait mon jour: il l`a vu, et il s`est réjoui. 
\verse Les Juifs lui dirent: Tu n`as pas encore cinquante ans, et tu as vu Abraham! 
\verse Jésus leur dit: En vérité, en vérité, je vous le dis, avant qu`Abraham fût, je suis. 
\verse Là-dessus, ils prirent des pierres pour les jeter contre lui; mais Jésus se cacha, et il sortit du temple. 

\chapter
\verse Jésus vit, en passant, un homme aveugle de naissance. 
\verse Ses disciples lui firent cette question: Rabbi, qui a péché, cet homme ou ses parents, pour qu`il soit né aveugle? 
\verse Jésus répondit: Ce n`est pas que lui ou ses parents aient péché; mais c`est afin que les oeuvres de Dieu soient manifestées en lui. 
\verse Il faut que je fasse, tandis qu`il est jour, les oeuvres de celui qui m`a envoyé; la nuit vient, où personne ne peut travailler. 
\verse Pendant que je suis dans le monde, je suis la lumière du monde. 
\verse Après avoir dit cela, il cracha à terre, et fit de la boue avec sa salive. Puis il appliqua cette boue sur les yeux de l`aveugle, 
\verse et lui dit: Va, et lave-toi au réservoir de Siloé (nom qui signifie envoyé). Il y alla, se lava, et s`en retourna voyant clair. 
\verse Ses voisins et ceux qui auparavant l`avaient connu comme un mendiant disaient: N`est-ce pas là celui qui se tenait assis et qui mendiait? 
\verse Les uns disaient: C`est lui. D`autres disaient: Non, mais il lui ressemble. Et lui-même disait: C`est moi. 
\verse Ils lui dirent donc: Comment tes yeux ont-ils été ouverts? 
\verse Il répondit: L`Homme qu`on appelle Jésus a fait de la boue, a oint mes yeux, et m`a dit: Va au réservoir de Siloé, et lave-toi. J`y suis allé, je me suis lavé, et j`ai recouvré la vue. 
\verse Ils lui dirent: Où est cet homme? Il répondit: Je ne sais. 
\verse Ils menèrent vers les pharisiens celui qui avait été aveugle. 
\verse Or, c`était un jour de sabbat que Jésus avait fait de la boue, et lui avait ouvert les yeux. 
\verse De nouveau, les pharisiens aussi lui demandèrent comment il avait recouvré la vue. Et il leur dit: Il a appliqué de la boue sur mes yeux, je me suis lavé, et je vois. 
\verse Sur quoi quelques-uns des pharisiens dirent: Cet homme ne vient pas de Dieu, car il n`observe pas le sabbat. D`autres dirent: Comment un homme pécheur peut-il faire de tels miracles? 
\verse Et il y eut division parmi eux. Ils dirent encore à l`aveugle: Toi, que dis-tu de lui, sur ce qu`il t`a ouvert les yeux? Il répondit: C`est un prophète. 
\verse Les Juifs ne crurent point qu`il eût été aveugle et qu`il eût recouvré la vue jusqu`à ce qu`ils eussent fait venir ses parents. 
\verse Et ils les interrogèrent, disant: Est-ce là votre fils, que vous dites être né aveugle? Comment donc voit-il maintenant? 
\verse Ses parents répondirent: Nous savons que c`est notre fils, et qu`il est né aveugle; 
\verse mais comment il voit maintenant, ou qui lui a ouvert les yeux, c`est ce que nous ne savons. Interrogez-le lui-même, il a de l`âge, il parlera de ce qui le concerne. 
\verse Ses parents dirent cela parce qu`ils craignaient les Juifs; car les Juifs étaient déjà convenus que, si quelqu`un reconnaissait Jésus pour le Christ, il serait exclu de la synagogue. 
\verse C`est pourquoi ses parents dirent: Il a de l`âge, interrogez-le lui-même. 
\verse Les pharisiens appelèrent une seconde fois l`homme qui avait été aveugle, et ils lui dirent: Donne gloire à Dieu; nous savons que cet homme est un pécheur. 
\verse Il répondit: S`il est un pécheur, je ne sais; je sais une chose, c`est que j`étais aveugle et que maintenant je vois. 
\verse Ils lui dirent: Que t`a-t-il fait? Comment t`a-t-il ouvert les yeux? 
\verse Il leur répondit: Je vous l`ai déjà dit, et vous n`avez pas écouté; pourquoi voulez-vous l`entendre encore? Voulez-vous aussi devenir ses disciples? 
\verse Ils l`injurièrent et dirent: C`est toi qui es son disciple; nous, nous sommes disciples de Moïse. 
\verse Nous savons que Dieu a parlé à Moïse; mais celui-ci, nous ne savons d`où il est. 
\verse Cet homme leur répondit: Il est étonnant que vous ne sachiez d`où il est; et cependant il m`a ouvert les yeux. 
\verse Nous savons que Dieu n`exauce point les pécheurs; mais, si quelqu`un l`honore et fait sa volonté, c`est celui là qu`il l`exauce. 
\verse Jamais on n`a entendu dire que quelqu`un ait ouvert les yeux d`un aveugle-né. 
\verse Si cet homme ne venait pas de Dieu, il ne pourrait rien faire. 
\verse Ils lui répondirent: Tu es né tout entier dans le péché, et tu nous enseignes! Et ils le chassèrent. 
\verse Jésus apprit qu`ils l`avaient chassé; et, l`ayant rencontré, il lui dit: Crois-tu au Fils de Dieu? 
\verse Il répondit: Et qui est-il, Seigneur, afin que je croie en lui? 
\verse Tu l`as vu, lui dit Jésus, et celui qui te parle, c`est lui. 
\verse Et il dit: Je crois, Seigneur. Et il se prosterna devant lui. 
\verse Puis Jésus dit: Je suis venu dans ce monde pour un jugement, pour que ceux qui ne voient point voient, et que ceux qui voient deviennent aveugles. 
\verse Quelques pharisiens qui étaient avec lui, ayant entendu ces paroles, lui dirent: Nous aussi, sommes-nous aveugles? 
\verse Jésus leur répondit: Si vous étiez aveugles, vous n`auriez pas de péché. Mais maintenant vous dites: Nous voyons. C`est pour cela que votre péché subsiste. 

\chapter
\verse En vérité, en vérité, je vous le dis, celui qui n`entre pas par la porte dans la bergerie, mais qui y monte par ailleurs, est un voleur et un brigand. 
\verse Mais celui qui entre par la porte est le berger des brebis. 
\verse Le portier lui ouvre, et les brebis entendent sa voix; il appelle par leur nom les brebis qui lui appartiennent, et il les conduit dehors. 
\verse Lorsqu`il a fait sortir toutes ses propres brebis, il marche devant elles; et les brebis le suivent, parce qu`elles connaissent sa voix. 
\verse Elles ne suivront point un étranger; mais elles fuiront loin de lui, parce qu`elles ne connaissent pas la voix des étrangers. 
\verse Jésus leur dit cette parabole, mais ils ne comprirent pas de quoi il leur parlait. 
\verse Jésus leur dit encore: En vérité, en vérité, je vous le dis, je suis la porte des brebis. 
\verse Tous ceux qui sont venus avant moi sont des voleurs et des brigands; mais les brebis ne les ont point écoutés. 
\verse Je suis la porte. Si quelqu`un entre par moi, il sera sauvé; il entrera et il sortira, et il trouvera des pâturages. 
\verse Le voleur ne vient que pour dérober, égorger et détruire; moi, je suis venu afin que les brebis aient la vie, et qu`elles soient dans l`abondance. 
\verse Je suis le bon berger. Le bon berger donne sa vie pour ses brebis. 
\verse Mais le mercenaire, qui n`est pas le berger, et à qui n`appartiennent pas les brebis, voit venir le loup, abandonne les brebis, et prend la fuite; et le loup les ravit et les disperse. 
\verse Le mercenaire s`enfuit, parce qu`il est mercenaire, et qu`il ne se met point en peine des brebis. Je suis le bon berger. 
\verse Je connais mes brebis, et elles me connaissent, 
\verse comme le Père me connaît et comme je connais le Père; et je donne ma vie pour mes brebis. 
\verse J`ai encore d`autres brebis, qui ne sont pas de cette bergerie; celles-là, il faut que je les amène; elles entendront ma voix, et il y aura un seul troupeau, un seul berger. 
\verse Le Père m`aime, parce que je donne ma vie, afin de la reprendre. 
\verse Personne ne me l`ôte, mais je la donne de moi-même; j`ai le pouvoir de la donner, et j`ai le pouvoir de la reprendre: tel est l`ordre que j`ai reçu de mon Père. 
\verse Il y eut de nouveau, à cause de ces paroles, division parmi les Juifs. 
\verse Plusieurs d`entre eux disaient: Il a un démon, il est fou; pourquoi l`écoutez-vous? 
\verse D`autres disaient: Ce ne sont pas les paroles d`un démoniaque; un démon peut-il ouvrir les yeux des aveugles? 
\verse On célébrait à Jérusalem la fête de la Dédicace. C`était l`hiver. 
\verse Et Jésus se promenait dans le temple, sous le portique de Salomon. 
\verse Les Juifs l`entourèrent, et lui dirent: Jusques à quand tiendras-tu notre esprit en suspens? Si tu es le Christ, dis-le nous franchement. 
\verse Jésus leur répondit: Je vous l`ai dit, et vous ne croyez pas. Les oeuvres que je fais au nom de mon Père rendent témoignage de moi. 
\verse Mais vous ne croyez pas, parce que vous n`êtes pas de mes brebis. 
\verse Mes brebis entendent ma voix; je les connais, et elles me suivent. 
\verse Je leur donne la vie éternelle; et elles ne périront jamais, et personne ne les ravira de ma main. 
\verse Mon Père, qui me les a données, est plus grand que tous; et personne ne peut les ravir de la main de mon Père. 
\verse Moi et le Père nous sommes un. 
\verse Alors les Juifs prirent de nouveau des pierres pour le lapider. 
\verse Jésus leur dit: Je vous ai fait voir plusieurs bonnes oeuvres venant de mon Père: pour laquelle me lapidez-vous? 
\verse Les Juifs lui répondirent: Ce n`est point pour une bonne oeuvre que nous te lapidons, mais pour un blasphème, et parce que toi, qui es un homme, tu te fais Dieu. 
\verse Jésus leur répondit: N`est-il pas écrit dans votre loi: J`ai dit: Vous êtes des dieux? 
\verse Si elle a appelé dieux ceux à qui la parole de Dieu a été adressée, et si l`Écriture ne peut être anéantie, 
\verse celui que le Père a sanctifié et envoyé dans le monde, vous lui dites: Tu blasphèmes! Et cela parce que j`ai dit: Je suis le Fils de Dieu. 
\verse Si je ne fais pas les oeuvres de mon Père, ne me croyez pas. 
\verse Mais si je les fais, quand même vous ne me croyez point, croyez à ces oeuvres, afin que vous sachiez et reconnaissiez que le Père est en moi et que je suis dans le Père. 
\verse Là-dessus, ils cherchèrent encore à le saisir, mais il s`échappa de leurs mains. 
\verse Jésus s`en alla de nouveau au delà du Jourdain, dans le lieu où Jean avait d`abord baptisé. Et il y demeura. 
\verse Beaucoup de gens vinrent à lui, et ils disaient: Jean n`a fait aucun miracle; mais tout ce que Jean a dit de cet homme était vrai. 
\verse Et, dans ce lieu-là, plusieurs crurent en lui. 

\chapter
\verse Il y avait un homme malade, Lazare, de Béthanie, village de Marie et de Marthe, sa soeur. 
\verse C`était cette Marie qui oignit de parfum le Seigneur et qui lui essuya les pieds avec ses cheveux, et c`était son frère Lazare qui était malade. 
\verse Les soeurs envoyèrent dire à Jésus: Seigneur, voici, celui que tu aimes est malade. 
\verse Après avoir entendu cela, Jésus dit: Cette maladie n`est point à la mort; mais elle est pour la gloire de Dieu, afin que le Fils de Dieu soit glorifié par elle. 
\verse Or, Jésus aimait Marthe, et sa soeur, et Lazare. 
\verse Lors donc qu`il eut appris que Lazare était malade, il resta deux jours encore dans le lieu où il était, 
\verse et il dit ensuite aux disciples: Retournons en Judée. 
\verse Les disciples lui dirent: Rabbi, les Juifs tout récemment cherchaient à te lapider, et tu retournes en Judée! 
\verse Jésus répondit: N`y a-t-il pas douze heures au jour? Si quelqu`un marche pendant le jour, il ne bronche point, parce qu`il voit la lumière de ce monde; 
\verse mais, si quelqu`un marche pendant la nuit, il bronche, parce que la lumière n`est pas en lui. 
\verse Après ces paroles, il leur dit: Lazare, notre ami, dort; mais je vais le réveiller. 
\verse Les disciples lui dirent: Seigneur, s`il dort, il sera guéri. 
\verse Jésus avait parlé de sa mort, mais ils crurent qu`il parlait de l`assoupissement du sommeil. 
\verse Alors Jésus leur dit ouvertement: Lazare est mort. 
\verse Et, à cause de vous, afin que vous croyiez, je me réjouis de ce que je n`étais pas là. Mais allons vers lui. 
\verse Sur quoi Thomas, appelé Didyme, dit aux autres disciples: Allons aussi, afin de mourir avec lui. 
\verse Jésus, étant arrivé, trouva que Lazare était déjà depuis quatre jours dans le sépulcre. 
\verse Et, comme Béthanie était près de Jérusalem, à quinze stades environ, 
\verse beaucoup de Juifs étaient venus vers Marthe et Marie, pour les consoler de la mort de leur frère. 
\verse Lorsque Marthe apprit que Jésus arrivait, elle alla au-devant de lui, tandis que Marie se tenait assise à la maison. 
\verse Marthe dit à Jésus: Seigneur, si tu eusses été ici, mon frère ne serait pas mort. 
\verse Mais, maintenant même, je sais que tout ce que tu demanderas à Dieu, Dieu te l`accordera. 
\verse Jésus lui dit: Ton frère ressuscitera. 
\verse Je sais, lui répondit Marthe, qu`il ressuscitera à la résurrection, au dernier jour. 
\verse Jésus lui dit: Je suis la résurrection et la vie. Celui qui croit en moi vivra, quand même il serait mort; 
\verse et quiconque vit et croit en moi ne mourra jamais. Crois-tu cela? 
\verse Elle lui dit: Oui, Seigneur, je crois que tu es le Christ, le Fils de Dieu, qui devait venir dans le monde. 
\verse Ayant ainsi parlé, elle s`en alla. Puis elle appela secrètement Marie, sa soeur, et lui dit: Le maître est ici, et il te demande. 
\verse Dès que Marie eut entendu, elle se leva promptement, et alla vers lui. 
\verse Car Jésus n`était pas encore entré dans le village, mais il était dans le lieu où Marthe l`avait rencontré. 
\verse Les Juifs qui étaient avec Marie dans la maison et qui la consolaient, l`ayant vue se lever promptement et sortir, la suivirent, disant: Elle va au sépulcre, pour y pleurer. 
\verse Lorsque Marie fut arrivée là où était Jésus, et qu`elle le vit, elle tomba à ses pieds, et lui dit: Seigneur, si tu eusses été ici, mon frère ne serait pas mort. 
\verse Jésus, la voyant pleurer, elle et les Juifs qui étaient venus avec elle, frémit en son esprit, et fut tout ému. 
\verse Et il dit: Où l`avez-vous mis? Seigneur, lui répondirent-ils, viens et vois. 
\verse Jésus pleura. 
\verse Sur quoi les Juifs dirent: Voyez comme il l`aimait. 
\verse Et quelques-uns d`entre eux dirent: Lui qui a ouvert les yeux de l`aveugle, ne pouvait-il pas faire aussi que cet homme ne mourût point? 
\verse Jésus frémissant de nouveau en lui-même, se rendit au sépulcre. C`était une grotte, et une pierre était placée devant. 
\verse Jésus dit: Otez la pierre. Marthe, la soeur du mort, lui dit: Seigneur, il sent déjà, car il y a quatre jours qu`il est là. 
\verse Jésus lui dit: Ne t`ai-je pas dit que, si tu crois, tu verras la gloire de Dieu? 
\verse Ils ôtèrent donc la pierre. Et Jésus leva les yeux en haut, et dit: Père, je te rends grâces de ce que tu m`as exaucé. 
\verse Pour moi, je savais que tu m`exauces toujours; mais j`ai parlé à cause de la foule qui m`entoure, afin qu`ils croient que c`est toi qui m`as envoyé. 
\verse Ayant dit cela, il cria d`une voix forte: Lazare, sors! 
\verse Et le mort sortit, les pieds et les mains liés de bandes, et le visage enveloppé d`un linge. Jésus leur dit: Déliez-le, et laissez-le aller. 
\verse Plusieurs des Juifs qui étaient venus vers Marie, et qui virent ce que fit Jésus, crurent en lui. 
\verse Mais quelques-uns d`entre eux allèrent trouver les pharisiens, et leur dirent ce que Jésus avait fait. 
\verse Alors les principaux sacrificateurs et les pharisiens assemblèrent le sanhédrin, et dirent: Que ferons-nous? Car cet homme fait beaucoup de miracles. 
\verse Si nous le laissons faire, tous croiront en lui, et les Romains viendront détruire et notre ville et notre nation. 
\verse L`un d`eux, Caïphe, qui était souverain sacrificateur cette année-là, leur dit: Vous n`y entendez rien; 
\verse vous ne réfléchissez pas qu`il est dans votre intérêt qu`un seul homme meure pour le peuple, et que la nation entière ne périsse pas. 
\verse Or, il ne dit pas cela de lui-même; mais étant souverain sacrificateur cette année-là, il prophétisa que Jésus devait mourir pour la nation. 
\verse Et ce n`était pas pour la nation seulement; c`était aussi afin de réunir en un seul corps les enfants de Dieu dispersés. 
\verse Dès ce jour, ils résolurent de le faire mourir. 
\verse C`est pourquoi Jésus ne se montra plus ouvertement parmi les Juifs; mais il se retira dans la contrée voisine du désert, dans une ville appelée Éphraïm; et là il demeurait avec ses disciples. 
\verse La Pâque des Juifs était proche. Et beaucoup de gens du pays montèrent à Jérusalem avant la Pâque, pour se purifier. 
\verse Ils cherchaient Jésus, et ils se disaient les uns aux autres dans le temple: Que vous en semble? Ne viendra-t-il pas à la fête? 
\verse Or, les principaux sacrificateurs et les pharisiens avaient donné l`ordre que, si quelqu`un savait où il était, il le déclarât, afin qu`on se saisît de lui. 

\chapter
\verse Six jours avant la Pâque, Jésus arriva à Béthanie, où était Lazare, qu`il avait ressuscité des morts. 
\verse Là, on lui fit un souper; Marthe servait, et Lazare était un de ceux qui se trouvaient à table avec lui. 
\verse Marie, ayant pris une livre d`un parfum de nard pur de grand prix, oignit les pieds de Jésus, et elle lui essuya les pieds avec ses cheveux; et la maison fut remplie de l`odeur du parfum. 
\verse Un de ses disciples, Judas Iscariot, fils de Simon, celui qui devait le livrer, dit: 
\verse Pourquoi n`a-t-on pas vendu ce parfum trois cent deniers, pour les donner aux pauvres? 
\verse Il disait cela, non qu`il se mît en peine des pauvres, mais parce qu`il était voleur, et que, tenant la bourse, il prenait ce qu`on y mettait. 
\verse Mais Jésus dit: Laisse-la garder ce parfum pour le jour de ma sépulture. 
\verse Vous avez toujours les pauvres avec vous, mais vous ne m`avez pas toujours. 
\verse Une grande multitude de Juifs apprirent que Jésus était à Béthanie; et ils y vinrent, non pas seulement à cause de lui, mais aussi pour voir Lazare, qu`il avait ressuscité des morts. 
\verse Les principaux sacrificateurs délibérèrent de faire mourir aussi Lazare, 
\verse parce que beaucoup de Juifs se retiraient d`eux à cause de lui, et croyaient en Jésus. 
\verse Le lendemain, une foule nombreuse de gens venus à la fête ayant entendu dire que Jésus se rendait à Jérusalem, 
\verse prirent des branches de palmiers, et allèrent au-devant de lui, en criant: Hosanna! Béni soit celui qui vient au nom du Seigneur, le roi d`Israël! 
\verse Jésus trouva un ânon, et s`assit dessus, selon ce qui est écrit: 
\verse Ne crains point, fille de Sion; Voici, ton roi vient, Assis sur le petit d`une ânesse. 
\verse Ses disciples ne comprirent pas d`abord ces choses; mais, lorsque Jésus eut été glorifié, ils se souvinrent qu`elles étaient écrites de lui, et qu`il les avaient été accomplies à son égard. 
\verse Tous ceux qui étaient avec Jésus, quand il appela Lazare du sépulcre et le ressuscita des morts, lui rendaient témoignage; 
\verse et la foule vint au-devant de lui, parce qu`elle avait appris qu`il avait fait ce miracle. 
\verse Les pharisiens se dirent donc les uns aux autres: Vous voyez que vous ne gagnez rien; voici, le monde est allé après lui. 
\verse Quelques Grecs, du nombre de ceux qui étaient montés pour adorer pendant la fête, 
\verse s`adressèrent à Philippe, de Bethsaïda en Galilée, et lui dirent avec instance: Seigneur, nous voudrions voir Jésus. 
\verse Philippe alla le dire à André, puis André et Philippe le dirent à Jésus. 
\verse Jésus leur répondit: L`heure est venue où le Fils de l`homme doit être glorifié. 
\verse En vérité, en vérité, je vous le dis, si le grain de blé qui est tombé en terre ne meurt, il reste seul; mais, s`il meurt, il porte beaucoup de fruit. 
\verse Celui qui aime sa vie la perdra, et celui qui hait sa vie dans ce monde la conservera pour la vie éternelle. 
\verse Si quelqu`un me sert, qu`il me suive; et là où je suis, là aussi sera mon serviteur. Si quelqu`un me sert, le Père l`honorera. 
\verse Maintenant mon âme est troublée. Et que dirais-je?... Père, délivre-moi de cette heure?... Mais c`est pour cela que je suis venu jusqu`à cette heure. 
\verse Père, glorifie ton nom! Et une voix vint du ciel: Je l`ai glorifié, et je le glorifierai encore. 
\verse La foule qui était là, et qui avait entendu, disait que c`était un tonnerre. D`autres disaient: Un ange lui a parlé. 
\verse Jésus dit: Ce n`est pas à cause de moi que cette voix s`est fait entendre; c`est à cause de vous. 
\verse Maintenant a lieu le jugement de ce monde; maintenant le prince de ce monde sera jeté dehors. 
\verse Et moi, quand j`aurai été élevé de la terre, j`attirerai tous les hommes à moi. 
\verse En parlant ainsi, il indiquait de quelle mort il devait mourir. - 
\verse La foule lui répondit: Nous avons appris par la loi que le Christ demeure éternellement; comment donc dis-tu: Il faut que le Fils de l`homme soit élevé? Qui est ce Fils de l`homme? 
\verse Jésus leur dit: La lumière est encore pour un peu de temps au milieu de vous. Marchez, pendant que vous avez la lumière, afin que les ténèbres ne vous surprennent point: celui qui marche dans les ténèbres ne sait où il va. 
\verse Pendant que vous avez la lumière, croyez en la lumière, afin que vous soyez des enfants de lumière. Jésus dit ces choses, puis il s`en alla, et se cacha loin d`eux. 
\verse Malgré tant de miracles qu`il avait faits en leur présence, ils ne croyaient pas en lui, 
\verse afin que s`accomplît la parole qu`Ésaïe, le prophète, a prononcée: Seigneur, Qui a cru à notre prédication? Et à qui le bras du Seigneur a-t-il été révélé? 
\verse Aussi ne pouvaient-ils croire, parce qu`Ésaïe a dit encore: 
\verse Il a aveuglé leurs yeux; et il a endurci leur coeur, De peur qu`ils ne voient des yeux, Qu`ils ne comprennent du coeur, Qu`ils ne se convertissent, et que je ne les guérisse. 
\verse Ésaïe dit ces choses, lorsqu`il vit sa gloire, et qu`il parla de lui. 
\verse Cependant, même parmi les chefs, plusieurs crurent en lui; mais, à cause des pharisiens, ils n`en faisaient pas l`aveu, dans la crainte d`être exclus de la synagogue. 
\verse Car ils aimèrent la gloire des hommes plus que la gloire de Dieu. 
\verse Or, Jésus s`était écrié: Celui qui croit en moi croit, non pas en moi, mais en celui qui m`a envoyé; 
\verse et celui qui me voit voit celui qui m`a envoyé. 
\verse Je suis venu comme une lumière dans le monde, afin que quiconque croit en moi ne demeure pas dans les ténèbres. 
\verse Si quelqu`un entend mes paroles et ne les garde point, ce n`est pas moi qui le juge; car je suis venu non pour juger le monde, mais pour sauver le monde. 
\verse Celui qui me rejette et qui ne reçoit pas mes paroles a son juge; la parole que j`ai annoncée, c`est elle qui le jugera au dernier jour. 
\verse Car je n`ai point parlé de moi-même; mais le Père, qui m`a envoyé, m`a prescrit lui-même ce que je dois dire et annoncer. 
\verse Et je sais que son commandement est la vie éternelle. C`est pourquoi les choses que je dis, je les dis comme le Père me les a dites. 

\chapter
\verse Avant la fête de Pâque, Jésus, sachant que son heure était venue de passer de ce monde au Père, et ayant aimé les siens qui étaient dans le monde, mit le comble à son amour pour eux. 
\verse Pendant le souper, lorsque le diable avait déjà inspiré au coeur de Judas Iscariot, fils de Simon, le dessein de le livrer, 
\verse Jésus, qui savait que le Père avait remis toutes choses entre ses mains, qu`il était venu de Dieu, et qu`il s`en allait à Dieu, 
\verse se leva de table, ôta ses vêtements, et prit un linge, dont il se ceignit. 
\verse Ensuite il versa de l`eau dans un bassin, et il se mit à laver les pieds des disciples, et à les essuyer avec le linge dont il était ceint. 
\verse Il vint donc à Simon Pierre; et Pierre lui dit: Toi, Seigneur, tu me laves les pieds! 
\verse Jésus lui répondit: Ce que je fais, tu ne le comprends pas maintenant, mais tu le comprendras bientôt. 
\verse Pierre lui dit: Non, jamais tu ne me laveras les pieds. Jésus lui répondit: Si je ne te lave, tu n`auras point de part avec moi. 
\verse Simon Pierre lui dit: Seigneur, non seulement les pieds, mais encore les mains et la tête. 
\verse Jésus lui dit: Celui qui est lavé n`a besoin que de se laver les pieds pour être entièrement pur; et vous êtes purs, mais non pas tous. 
\verse Car il connaissait celui qui le livrait; c`est pourquoi il dit: Vous n`êtes pas tous purs. 
\verse Après qu`il leur eut lavé les pieds, et qu`il eut pris ses vêtements, il se remit à table, et leur dit: Comprenez-vous ce que je vous ai fait? 
\verse Vous m`appelez Maître et Seigneur; et vous dites bien, car je le suis. 
\verse Si donc je vous ai lavé les pieds, moi, le Seigneur et le Maître, vous devez aussi vous laver les pieds les uns aux autres; 
\verse car je vous ai donné un exemple, afin que vous fassiez comme je vous ai fait. 
\verse En vérité, en vérité, je vous le dis, le serviteur n`est pas plus grand que son seigneur, ni l`apôtre plus grand que celui qui l`a envoyé. 
\verse Si vous savez ces choses, vous êtes heureux, pourvu que vous les pratiquiez. 
\verse Ce n`est pas de vous tous que je parle; je connais ceux que j`ai choisis. Mais il faut que l`Écriture s`accomplisse: Celui qui mange avec moi le pain A levé son talon contre moi. 
\verse Dès à présent je vous le dis, avant que la chose arrive, afin que, lorsqu`elle arrivera, vous croyiez à ce que je suis. 
\verse En vérité, en vérité, je vous le dis, celui qui reçoit celui que j`aurai envoyé me reçoit, et celui qui me reçoit, reçoit celui qui m`a envoyé. 
\verse Ayant ainsi parlé, Jésus fut troublé en son esprit, et il dit expressément: En vérité, en vérité, je vous le dis, l`un de vous me livrera. 
\verse Les disciples se regardaient les uns les autres, ne sachant de qui il parlait. 
\verse Un des disciples, celui que Jésus aimait, était couché sur le sein de Jésus. 
\verse Simon Pierre lui fit signe de demander qui était celui dont parlait Jésus. 
\verse Et ce disciple, s`étant penché sur la poitrine de Jésus, lui dit: Seigneur, qui est-ce? 
\verse Jésus répondit: C`est celui à qui je donnerai le morceau trempé. Et, ayant trempé le morceau, il le donna à Judas, fils de Simon, l`Iscariot. 
\verse Dès que le morceau fut donné, Satan entra dans Judas. Jésus lui dit: Ce que tu fais, fais-le promptement. 
\verse Mais aucun de ceux qui étaient à table ne comprit pourquoi il lui disait cela; 
\verse car quelques-uns pensaient que, comme Judas avait la bourse, Jésus voulait lui dire: Achète ce dont nous avons besoin pour la fête, ou qu`il lui commandait de donner quelque chose aux pauvres. 
\verse Judas, ayant pris le morceau, se hâta de sortir. Il était nuit. 
\verse Lorsque Judas fut sorti, Jésus dit: Maintenant, le Fils de l`homme a été glorifié, et Dieu a été glorifié en lui. 
\verse Si Dieu a été glorifié en lui, Dieu aussi le glorifiera en lui-même, et il le glorifiera bientôt. 
\verse Mes petits enfants, je suis pour peu de temps encore avec vous. Vous me chercherez; et, comme j`ai dit aux Juifs: Vous ne pouvez venir où je vais, je vous le dis aussi maintenant. 
\verse Je vous donne un commandement nouveau: Aimez-vous les uns les autres; comme je vous ai aimés, vous aussi, aimez-vous les uns les autres. 
\verse A ceci tous connaîtront que vous êtes mes disciples, si vous avez de l`amour les uns pour les autres. 
\verse Simon Pierre lui dit: Seigneur, où vas-tu? Jésus répondit: Tu ne peux pas maintenant me suivre où je vais, mais tu me suivras plus tard. 
\verse Seigneur, lui dit Pierre, pourquoi ne puis-je pas te suivre maintenant? Je donnerai ma vie pour toi. 
\verse Jésus répondit: Tu donneras ta vie pour moi! En vérité, en vérité, je te le dis, le coq ne chantera pas que tu ne m`aies renié trois fois. 

\chapter
\verse Que votre coeur ne se trouble point. Croyez en Dieu, et croyez en moi. 
\verse Il y a plusieurs demeures dans la maison de mon Père. Si cela n`était pas, je vous l`aurais dit. Je vais vous préparer une place. 
\verse Et, lorsque je m`en serai allé, et que je vous aurai préparé une place, je reviendrai, et je vous prendrai avec moi, afin que là où je suis vous y soyez aussi. 
\verse Vous savez où je vais, et vous en savez le chemin. 
\verse Thomas lui dit: Seigneur, nous ne savons où tu vas; comment pouvons-nous en savoir le chemin? 
\verse Jésus lui dit: Je suis le chemin, la vérité, et la vie. Nul ne vient au Père que par moi. 
\verse Si vous me connaissiez, vous connaîtriez aussi mon Père. Et dès maintenant vous le connaissez, et vous l`avez vu. 
\verse Philippe lui dit: Seigneur, montre-nous le Père, et cela nous suffit. 
\verse Jésus lui dit: Il y a si longtemps que je suis avec vous, et tu ne m`as pas connu, Philippe! Celui qui m`a vu a vu le Père; comment dis-tu: Montre-nous le Père? 
\verse Ne crois-tu pas que je suis dans le Père, et que le Père est en moi? Les paroles que je vous dis, je ne les dis pas de moi-même; et le Père qui demeure en moi, c`est lui qui fait les oeuvres. 
\verse Croyez-moi, je suis dans le Père, et le Père est en moi; croyez du moins à cause de ces oeuvres. 
\verse En vérité, en vérité, je vous le dis, celui qui croit en moi fera aussi les oeuvres que je fais, et il en fera de plus grandes, parce que je m`en vais au Père; 
\verse et tout ce que vous demanderez en mon nom, je le ferai, afin que le Père soit glorifié dans le Fils. 
\verse Si vous demandez quelque chose en mon nom, je le ferai. 
\verse Si vous m`aimez, gardez mes commandements. 
\verse Et moi, je prierai le Père, et il vous donnera un autre consolateur, afin qu`il demeure éternellement avec vous, 
\verse l`Esprit de vérité, que le monde ne peut recevoir, parce qu`il ne le voit point et ne le connaît point; mais vous, vous le connaissez, car il demeure avec vous, et il sera en vous. 
\verse Je ne vous laisserai pas orphelins, je viendrai à vous. 
\verse Encore un peu de temps, et le monde ne me verra plus; mais vous, vous me verrez, car je vis, et vous vivrez aussi. 
\verse En ce jour-là, vous connaîtrez que je suis en mon Père, que vous êtes en moi, et que je suis en vous. 
\verse Celui qui a mes commandements et qui les garde, c`est celui qui m`aime; et celui qui m`aime sera aimé de mon Père, je l`aimerai, et je me ferai connaître à lui. 
\verse Jude, non pas l`Iscariot, lui dit: Seigneur, d`où vient que tu te feras connaître à nous, et non au monde? 
\verse Jésus lui répondit: Si quelqu`un m`aime, il gardera ma parole, et mon Père l`aimera; nous viendrons à lui, et nous ferons notre demeure chez lui. 
\verse Celui qui ne m`aime pas ne garde point mes paroles. Et la parole que vous entendez n`est pas de moi, mais du Père qui m`a envoyé. 
\verse Je vous ai dit ces choses pendant que je demeure avec vous. 
\verse Mais le consolateur, l`Esprit Saint, que le Père enverra en mon nom, vous enseignera toutes choses, et vous rappellera tout ce que je vous ai dit. 
\verse Je vous laisse la paix, je vous donne ma paix. Je ne vous donne pas comme le monde donne. Que votre coeur ne se trouble point, et ne s`alarme point. 
\verse Vous avez entendu que je vous ai dit: Je m`en vais, et je reviens vers vous. Si vous m`aimiez, vous vous réjouiriez de ce que je vais au Père; car le Père est plus grand que moi. 
\verse Et maintenant je vous ai dit ces choses avant qu`elles arrivent, afin que, lorsqu`elles arriveront, vous croyiez. 
\verse Je ne parlerai plus guère avec vous; car le prince du monde vient. Il n`a rien en moi; 
\verse mais afin que le monde sache que j`aime le Père, et que j`agis selon l`ordre que le Père m`a donné, levez-vous, partons d`ici. 

\chapter
\verse Je suis le vrai cep, et mon Père est le vigneron. 
\verse Tout sarment qui est en moi et qui ne porte pas de fruit, il le retranche; et tout sarment qui porte du fruit, il l`émonde, afin qu`il porte encore plus de fruit. 
\verse Déjà vous êtes purs, à cause de la parole que je vous ai annoncée. 
\verse Demeurez en moi, et je demeurerai en vous. Comme le sarment ne peut de lui-même porter du fruit, s`il ne demeure attaché au cep, ainsi vous ne le pouvez non plus, si vous ne demeurez en moi. 
\verse Je suis le cep, vous êtes les sarments. Celui qui demeure en moi et en qui je demeure porte beaucoup de fruit, car sans moi vous ne pouvez rien faire. 
\verse Si quelqu`un ne demeure pas en moi, il est jeté dehors, comme le sarment, et il sèche; puis on ramasse les sarments, on les jette au feu, et ils brûlent. 
\verse Si vous demeurez en moi, et que mes paroles demeurent en vous, demandez ce que vous voudrez, et cela vous sera accordé. 
\verse Si vous portez beaucoup de fruit, c`est ainsi que mon Père sera glorifié, et que vous serez mes disciples. 
\verse Comme le Père m`a aimé, je vous ai aussi aimés. Demeurez dans mon amour. 
\verse Si vous gardez mes commandements, vous demeurerez dans mon amour, de même que j`ai gardé les commandements de mon Père, et que je demeure dans son amour. 
\verse Je vous ai dit ces choses, afin que ma joie soit en vous, et que votre joie soit parfaite. 
\verse C`est ici mon commandement: Aimez-vous les uns les autres, comme je vous ai aimés. 
\verse Il n`y a pas de plus grand amour que de donner sa vie pour ses amis. 
\verse Vous êtes mes amis, si vous faites ce que je vous commande. 
\verse Je ne vous appelle plus serviteurs, parce que le serviteur ne sait pas ce que fait son maître; mais je vous ai appelés amis, parce que je vous ai fait connaître tout ce que j`ai appris de mon Père. 
\verse Ce n`est pas vous qui m`avez choisi; mais moi, je vous ai choisis, et je vous ai établis, afin que vous alliez, et que vous portiez du fruit, et que votre fruit demeure, afin que ce que vous demanderez au Père en mon nom, il vous le donne. 
\verse Ce que je vous commande, c`est de vous aimer les uns les autres. 
\verse Si le monde vous hait, sachez qu`il m`a haï avant vous. 
\verse Si vous étiez du monde, le monde aimerait ce qui est à lui; mais parce que vous n`êtes pas du monde, et que je vous ai choisis du milieu du monde, à cause de cela le monde vous hait. 
\verse Souvenez-vous de la parole que je vous ai dite: Le serviteur n`est pas plus grand que son maître. S`ils m`ont persécuté, ils vous persécuteront aussi; s`ils ont gardé ma parole, ils garderont aussi la vôtre. 
\verse Mais ils vous feront toutes ces choses à cause de mon nom, parce qu`ils ne connaissent pas celui qui m`a envoyé. 
\verse Si je n`étais pas venu et que je ne leur eusses point parlé, ils n`auraient pas de péché; mais maintenant ils n`ont aucune excuse de leur péché. 
\verse Celui qui me hait, hait aussi mon Père. 
\verse Si je n`avais pas fait parmi eux des oeuvres que nul autre n`a faites, ils n`auraient pas de péché; mais maintenant ils les ont vues, et ils ont haï et moi et mon Père. 
\verse Mais cela est arrivé afin que s`accomplît la parole qui est écrite dans leur loi: Ils m`ont haï sans cause. 
\verse Quand sera venu le consolateur, que je vous enverrai de la part du Père, l`Esprit de vérité, qui vient du Père, il rendra témoignage de moi; 
\verse et vous aussi, vous rendrez témoignage, parce que vous êtes avec moi dès le commencement. 

\chapter
\verse Je vous ai dit ces choses, afin qu`elles ne soient pas pour vous une occasion de chute. 
\verse Ils vous excluront des synagogues; et même l`heure vient où quiconque vous fera mourir croira rendre un culte à Dieu. 
\verse Et ils agiront ainsi, parce qu`ils n`ont connu ni le Père ni moi. 
\verse Je vous ai dit ces choses, afin que, lorsque l`heure sera venue, vous vous souveniez que je vous les ai dites. Je ne vous en ai pas parlé dès le commencement, parce que j`étais avec vous. 
\verse Maintenant je m`en vais vers celui qui m`a envoyé, et aucun de vous ne me demande: Où vas-tu? 
\verse Mais, parce que je vous ai dit ces choses, la tristesse a rempli votre coeur. 
\verse Cependant je vous dis la vérité: il vous est avantageux que je m`en aille, car si je ne m`en vais pas, le consolateur ne viendra pas vers vous; mais, si je m`en vais, je vous l`enverrai. 
\verse Et quand il sera venu, il convaincra le monde en ce qui concerne le péché, la justice, et le jugement: 
\verse en ce qui concerne le péché, parce qu`ils ne croient pas en moi; 
\verse la justice, parce que je vais au Père, et que vous ne me verrez plus; 
\verse le jugement, parce que le prince de ce monde est jugé. 
\verse J`ai encore beaucoup de choses à vous dire, mais vous ne pouvez pas les porter maintenant. 
\verse Quand le consolateur sera venu, l`Esprit de vérité, il vous conduira dans toute la vérité; car il ne parlera pas de lui-même, mais il dira tout ce qu`il aura entendu, et il vous annoncera les choses à venir. 
\verse Il me glorifiera, parce qu`il prendra de ce qui est à moi, et vous l`annoncera. 
\verse Tout ce que le Père a est à moi; c`est pourquoi j`ai dit qu`il prend de ce qui est à moi, et qu`il vous l`annoncera. 
\verse Encore un peu de temps, et vous ne me verrez plus; et puis encore un peu de temps, et vous me verrez, parce que je vais au Père. 
\verse Là-dessus, quelques-uns de ses disciples dirent entre eux: Que signifie ce qu`il nous dit: Encore un peu de temps, et vous ne me verrez plus; et puis encore un peu de temps, et vous me verrez? et: Parce que je vais au Père? 
\verse Ils disaient donc: Que signifie ce qu`il dit: Encore un peu de temps? Nous ne savons de quoi il parle. 
\verse Jésus, connut qu`ils voulaient l`interroger, leur dit: Vous vous questionnez les uns les autres sur ce que j`ai dit: Encore un peu de temps, et vous ne me verrez plus; et puis encore un peu de temps, et vous me verrez. 
\verse En vérité, en vérité, je vous le dis, vous pleurerez et vous vous lamenterez, et le monde se réjouira: vous serez dans la tristesse, mais votre tristesse se changera en joie. 
\verse La femme, lorsqu`elle enfante, éprouve de la tristesse, parce que son heure est venue; mais, lorsqu`elle a donné le jour à l`enfant, elle ne se souvient plus de la souffrance, à cause de la joie qu`elle a de ce qu`un homme est né dans le monde. 
\verse Vous donc aussi, vous êtes maintenant dans la tristesse; mais je vous reverrai, et votre coeur se réjouira, et nul ne vous ravira votre joie. 
\verse En ce jour-là, vous ne m`interrogerez plus sur rien. En vérité, en vérité, je vous le dis, ce que vous demanderez au Père, il vous le donnera en mon nom. 
\verse Jusqu`à présent vous n`avez rien demandé en mon nom. Demandez, et vous recevrez, afin que votre joie soit parfaite. 
\verse Je vous ai dit ces choses en paraboles. L`heure vient où je ne vous parlerai plus en paraboles, mais où je vous parlerai ouvertement du Père. 
\verse En ce jour, vous demanderez en mon nom, et je ne vous dis pas que je prierai le Père pour vous; 
\verse car le Père lui-même vous aime, parce que vous m`avez aimé, et que vous avez cru que je suis sorti de Dieu. 
\verse Je suis sorti du Père, et je suis venu dans le monde; maintenant je quitte le monde, et je vais au Père. 
\verse Ses disciples lui dirent: Voici, maintenant tu parles ouvertement, et tu n`emploies aucune parabole. 
\verse Maintenant nous savons que tu sais toutes choses, et que tu n`as pas besoin que personne t`interroge; c`est pourquoi nous croyons que tu es sorti de Dieu. 
\verse Jésus leur répondit: Vous croyez maintenant. 
\verse Voici, l`heure vient, et elle est déjà venue, où vous serez dispersés chacun de son côté, et où vous me laisserez seul; mais je ne suis pas seul, car le Père est avec moi. 
\verse Je vous ai dit ces choses, afin que vous ayez la paix en moi. Vous aurez des tribulations dans le monde; mais prenez courage, j`ai vaincu le monde. 

\chapter
\verse Après avoir ainsi parlé, Jésus leva les yeux au ciel, et dit: Père, l`heure est venue! Glorifie ton Fils, afin que ton Fils te glorifie, 
\verse selon que tu lui as donné pouvoir sur toute chair, afin qu`il accorde la vie éternelle à tous ceux que tu lui as donnés. 
\verse Or, la vie éternelle, c`est qu`ils te connaissent, toi, le seul vrai Dieu, et celui que tu as envoyé, Jésus Christ. 
\verse Je t`ai glorifié sur la terre, j`ai achevé l`oeuvre que tu m`as donnée à faire. 
\verse Et maintenant toi, Père, glorifie-moi auprès de toi-même de la gloire que j`avais auprès de toi avant que le monde fût. 
\verse J`ai fait connaître ton nom aux hommes que tu m`as donnés du milieu du monde. Ils étaient à toi, et tu me les as donnés; et ils ont gardé ta parole. 
\verse Maintenant ils ont connu que tout ce que tu m`as donné vient de toi. 
\verse Car je leur ai donné les paroles que tu m`as données; et ils les ont reçues, et ils ont vraiment connu que je suis sorti de toi, et ils ont cru que tu m`as envoyé. 
\verse C`est pour eux que je prie. Je ne prie pas pour le monde, mais pour ceux que tu m`as donnés, parce qu`ils sont à toi; - 
\verse et tout ce qui est à moi est à toi, et ce qui est à toi est à moi; -et je suis glorifié en eux. 
\verse Je ne suis plus dans le monde, et ils sont dans le monde, et je vais à toi. Père saint, garde en ton nom ceux que tu m`as donnés, afin qu`ils soient un comme nous. 
\verse Lorsque j`étais avec eux dans le monde, je les gardais en ton nom. J`ai gardé ceux que tu m`as donnés, et aucun d`eux ne s`est perdu, sinon le fils de perdition, afin que l`Écriture fût accomplie. 
\verse Et maintenant je vais à toi, et je dis ces choses dans le monde, afin qu`ils aient en eux ma joie parfaite. 
\verse Je leur ai donné ta parole; et le monde les a haïs, parce qu`ils ne sont pas du monde, comme moi je ne suis pas du monde. 
\verse Je ne te prie pas de les ôter du monde, mais de les préserver du mal. 
\verse Ils ne sont pas du monde, comme moi je ne suis pas du monde. 
\verse Sanctifie-les par ta vérité: ta parole est la vérité. 
\verse Comme tu m`as envoyé dans le monde, je les ai aussi envoyés dans le monde. 
\verse Et je me sanctifie moi-même pour eux, afin qu`eux aussi soient sanctifiés par la vérité. 
\verse Ce n`est pas pour eux seulement que je prie, mais encore pour ceux qui croiront en moi par leur parole, 
\verse afin que tous soient un, comme toi, Père, tu es en moi, et comme je suis en toi, afin qu`eux aussi soient un en nous, pour que le monde croie que tu m`as envoyé. 
\verse Je leur ai donné la gloire que tu m`as donnée, afin qu`ils soient un comme nous sommes un, - 
\verse moi en eux, et toi en moi, -afin qu`ils soient parfaitement un, et que le monde connaisse que tu m`as envoyé et que tu les as aimés comme tu m`as aimé. 
\verse Père, je veux que là où je suis ceux que tu m`as donnés soient aussi avec moi, afin qu`ils voient ma gloire, la gloire que tu m`as donnée, parce que tu m`as aimé avant la fondation du monde. 
\verse Père juste, le monde ne t`a point connu; mais moi je t`ai connu, et ceux-ci ont connu que tu m`as envoyé. 
\verse Je leur ai fait connaître ton nom, et je le leur ferai connaître, afin que l`amour dont tu m`as aimé soit en eux, et que je sois en eux. 

\chapter
\verse Lorsqu`il eut dit ces choses, Jésus alla avec ses disciples de l`autre côté du torrent du Cédron, où se trouvait un jardin, dans lequel il entra, lui et ses disciples. 
\verse Judas, qui le livrait, connaissait ce lieu, parce que Jésus et ses disciples s`y étaient souvent réunis. 
\verse Judas donc, ayant pris la cohorte, et des huissiers qu`envoyèrent les principaux sacrificateurs et les pharisiens, vint là avec des lanternes, des flambeaux et des armes. 
\verse Jésus, sachant tout ce qui devait lui arriver, s`avança, et leur dit: Qui cherchez-vous? 
\verse Ils lui répondirent: Jésus de Nazareth. Jésus leur dit: C`est moi. Et Judas, qui le livrait, était avec eux. 
\verse Lorsque Jésus leur eut dit: C`est moi, ils reculèrent et tombèrent par terre. 
\verse Il leur demanda de nouveau: Qui cherchez-vous? Et ils dirent: Jésus de Nazareth. 
\verse Jésus répondit: Je vous ai dit que c`est moi. Si donc c`est moi que vous cherchez, laissez aller ceux-ci. 
\verse Il dit cela, afin que s`accomplît la parole qu`il avait dite: Je n`ai perdu aucun de ceux que tu m`as donnés. 
\verse Simon Pierre, qui avait une épée, la tira, frappa le serviteur du souverain sacrificateur, et lui coupa l`oreille droite. Ce serviteur s`appelait Malchus. 
\verse Jésus dit à Pierre: Remets ton épée dans le fourreau. Ne boirai-je pas la coupe que le Père m`a donnée à boire? 
\verse La cohorte, le tribun, et les huissiers des Juifs, se saisirent alors de Jésus, et le lièrent. 
\verse Ils l`emmenèrent d`abord chez Anne; car il était le beau-père de Caïphe, qui était souverain sacrificateur cette année-là. 
\verse Et Caïphe était celui qui avait donné ce conseil aux Juifs: Il est avantageux qu`un seul homme meure pour le peuple. 
\verse Simon Pierre, avec un autre disciple, suivait Jésus. Ce disciple était connu du souverain sacrificateur, et il entra avec Jésus dans la cour du souverain sacrificateur; 
\verse mais Pierre resta dehors près de la porte. L`autre disciple, qui était connu du souverain sacrificateur, sortit, parla à la portière, et fit entrer Pierre. 
\verse Alors la servante, la portière, dit à Pierre: Toi aussi, n`es-tu pas des disciples de cet homme? Il dit: Je n`en suis point. 
\verse Les serviteurs et les huissiers, qui étaient là, avaient allumé un brasier, car il faisait froid, et ils se chauffaient. Pierre se tenait avec eux, et se chauffait. 
\verse Le souverain sacrificateur interrogea Jésus sur ses disciples et sur sa doctrine. 
\verse Jésus lui répondit: J`ai parlé ouvertement au monde; j`ai toujours enseigné dans la synagogue et dans le temple, où tous les Juifs s`assemblent, et je n`ai rien dit en secret. 
\verse Pourquoi m`interroges-tu? Interroge sur ce que je leur ai dit ceux qui m`ont entendu; voici, ceux-là savent ce que j`ai dit. 
\verse A ces mots, un des huissiers, qui se trouvait là, donna un soufflet à Jésus, en disant: Est-ce ainsi que tu réponds au souverain sacrificateur? 
\verse Jésus lui dit: Si j`ai mal parlé, fais voir ce que j`ai dit de mal; et si j`ai bien parlé, pourquoi me frappes-tu? 
\verse Anne l`envoya lié à Caïphe, le souverain sacrificateur. 
\verse Simon Pierre était là, et se chauffait. On lui dit: Toi aussi, n`es-tu pas de ses disciples? Il le nia, et dit: Je n`en suis point. 
\verse Un des serviteurs du souverain sacrificateur, parent de celui à qui Pierre avait coupé l`oreille, dit: Ne t`ai-je pas vu avec lui dans le jardin? 
\verse Pierre le nia de nouveau. Et aussitôt le coq chanta. 
\verse Ils conduisirent Jésus de chez Caïphe au prétoire: c`était le matin. Ils n`entrèrent point eux-mêmes dans le prétoire, afin de ne pas se souiller, et de pouvoir manger la Pâque. 
\verse Pilate sortit donc pour aller à eux, et il dit: Quelle accusation portez-vous contre cet homme? 
\verse Ils lui répondirent: Si ce n`était pas un malfaiteur, nous ne te l`aurions pas livré. 
\verse Sur quoi Pilate leur dit: Prenez-le vous-mêmes, et jugez-le selon votre loi. Les Juifs lui dirent: Il ne nous est pas permis de mettre personne à mort. 
\verse C`était afin que s`accomplît la parole que Jésus avait dite, lorsqu`il indiqua de quelle mort il devait mourir. 
\verse Pilate rentra dans le prétoire, appela Jésus, et lui dit: Es-tu le roi des Juifs? 
\verse Jésus répondit: Est-ce de toi-même que tu dis cela, ou d`autres te l`ont-ils dit de moi? 
\verse Pilate répondit: Moi, suis-je Juif? Ta nation et les principaux sacrificateurs t`ont livré à moi: qu`as-tu fait? 
\verse Mon royaume n`est pas de ce monde, répondit Jésus. Si mon royaume était de ce monde, mes serviteurs auraient combattu pour moi afin que je ne fusse pas livré aux Juifs; mais maintenant mon royaume n`est point d`ici-bas. 
\verse Pilate lui dit: Tu es donc roi? Jésus répondit: Tu le dis, je suis roi. Je suis né et je suis venu dans le monde pour rendre témoignage à la vérité. Quiconque est de la vérité écoute ma voix. 
\verse Pilate lui dit: Qu`est-ce que la vérité? Après avoir dit cela, il sortit de nouveau pour aller vers les Juifs, et il leur dit: Je ne trouve aucun crime en lui. 
\verse Mais, comme c`est parmi vous une coutume que je vous relâche quelqu`un à la fête de Pâque, voulez-vous que je vous relâche le roi des Juifs? 
\verse Alors de nouveau tous s`écrièrent: Non pas lui, mais Barabbas. Or, Barabbas était un brigand. 

\chapter
\verse Alors Pilate prit Jésus, et le fit battre de verges. 
\verse Les soldats tressèrent une couronne d`épines qu`ils posèrent sur sa tête, et ils le revêtirent d`un manteau de pourpre; puis, s`approchant de lui, 
\verse ils disaient: Salut, roi des Juifs! Et ils lui donnaient des soufflets. 
\verse Pilate sortit de nouveau, et dit aux Juifs: Voici, je vous l`amène dehors, afin que vous sachiez que je ne trouve en lui aucun crime. 
\verse Jésus sortit donc, portant la couronne d`épines et le manteau de pourpre. Et Pilate leur dit: Voici l`homme. 
\verse Lorsque les principaux sacrificateurs et les huissiers le virent, ils s`écrièrent: Crucifie! crucifie! Pilate leur dit: Prenez-le vous-mêmes, et crucifiez-le; car moi, je ne trouve point de crime en lui. 
\verse Les Juifs lui répondirent: Nous avons une loi; et, selon notre loi, il doit mourir, parce qu`il s`est fait Fils de Dieu. 
\verse Quand Pilate entendit cette parole, sa frayeur augmenta. 
\verse Il rentra dans le prétoire, et il dit à Jésus: D`où es-tu? Mais Jésus ne lui donna point de réponse. 
\verse Pilate lui dit: Est-ce à moi que tu ne parles pas? Ne sais-tu pas que j`ai le pouvoir de te crucifier, et que j`ai le pouvoir de te relâcher? 
\verse Jésus répondit: Tu n`aurais sur moi aucun pouvoir, s`il ne t`avait été donné d`en haut. C`est pourquoi celui qui me livre à toi commet un plus grand péché. 
\verse Dès ce moment, Pilate cherchait à le relâcher. Mais les Juifs criaient: Si tu le relâches, tu n`es pas ami de César. Quiconque se fait roi se déclare contre César. 
\verse Pilate, ayant entendu ces paroles, amena Jésus dehors; et il s`assit sur le tribunal, au lieu appelé le Pavé, et en hébreu Gabbatha. 
\verse C`était la préparation de la Pâque, et environ la sixième heure. Pilate dit aux Juifs: Voici votre roi. 
\verse Mais ils s`écrièrent: Ote, ôte, crucifie-le! Pilate leur dit: Crucifierai-je votre roi? Les principaux sacrificateurs répondirent: Nous n`avons de roi que César. 
\verse Alors il le leur livra pour être crucifié. Ils prirent donc Jésus, et l`emmenèrent. 
\verse Jésus, portant sa croix, arriva au lieu du crâne, qui se nomme en hébreu Golgotha. 
\verse C`est là qu`il fut crucifié, et deux autres avec lui, un de chaque côté, et Jésus au milieu. 
\verse Pilate fit une inscription, qu`il plaça sur la croix, et qui était ainsi conçue: Jésus de Nazareth, roi des Juifs. 
\verse Beaucoup de Juifs lurent cette inscription, parce que le lieu où Jésus fut crucifié était près de la ville: elle était en hébreu, en grec et en latin. 
\verse Les principaux sacrificateurs des Juifs dirent à Pilate: N`écris pas: Roi des Juifs. Mais écris qu`il a dit: Je suis roi des Juifs. 
\verse Pilate répondit: Ce que j`ai écrit, je l`ai écrit. 
\verse Les soldats, après avoir crucifié Jésus, prirent ses vêtements, et ils en firent quatre parts, une part pour chaque soldat. Ils prirent aussi sa tunique, qui était sans couture, d`un seul tissu depuis le haut jusqu`en bas. Et ils dirent entre eux: 
\verse Ne la déchirons pas, mais tirons au sort à qui elle sera. Cela arriva afin que s`accomplît cette parole de l`Écriture: Ils se sont partagé mes vêtements, Et ils ont tiré au sort ma tunique. Voilà ce que firent les soldats. 
\verse Près de la croix de Jésus se tenaient sa mère et la soeur de sa mère, Marie, femme de Clopas, et Marie de Magdala. 
\verse Jésus, voyant sa mère, et auprès d`elle le disciple qu`il aimait, dit à sa mère: Femme, voilà ton fils. 
\verse Puis il dit au disciple: Voilà ta mère. Et, dès ce moment, le disciple la prit chez lui. 
\verse Après cela, Jésus, qui savait que tout était déjà consommé, dit, afin que l`Écriture fût accomplie: J`ai soif. 
\verse Il y avait là un vase plein de vinaigre. Les soldats en remplirent une éponge, et, l`ayant fixée à une branche d`hysope, ils l`approchèrent de sa bouche. 
\verse Quand Jésus eut pris le vinaigre, il dit: Tout est accompli. Et, baissant la tête, il rendit l`esprit. 
\verse Dans la crainte que les corps ne restassent sur la croix pendant le sabbat, -car c`était la préparation, et ce jour de sabbat était un grand jour, -les Juifs demandèrent à Pilate qu`on rompît les jambes aux crucifiés, et qu`on les enlevât. 
\verse Les soldats vinrent donc, et ils rompirent les jambes au premier, puis à l`autre qui avait été crucifié avec lui. 
\verse S`étant approchés de Jésus, et le voyant déjà mort, ils ne lui rompirent pas les jambes; 
\verse mais un des soldats lui perça le côté avec une lance, et aussitôt il sortit du sang et de l`eau. 
\verse Celui qui l`a vu en a rendu témoignage, et son témoignage est vrai; et il sait qu`il dit vrai, afin que vous croyiez aussi. 
\verse Ces choses sont arrivées, afin que l`Écriture fût accomplie: Aucun de ses os ne sera brisé. 
\verse Et ailleurs l`Écriture dit encore: Ils verront celui qu`ils ont percé. 
\verse Après cela, Joseph d`Arimathée, qui était disciple de Jésus, mais en secret par crainte des Juifs, demanda à Pilate la permission de prendre le corps de Jésus. Et Pilate le permit. Il vint donc, et prit le corps de Jésus. 
\verse Nicodème, qui auparavant était allé de nuit vers Jésus, vint aussi, apportant un mélange d`environ cent livres de myrrhe et d`aloès. 
\verse Ils prirent donc le corps de Jésus, et l`enveloppèrent de bandes, avec les aromates, comme c`est la coutume d`ensevelir chez les Juifs. 
\verse Or, il y avait un jardin dans le lieu où Jésus avait été crucifié, et dans le jardin un sépulcre neuf, où personne encore n`avait été mis. 
\verse Ce fut là qu`ils déposèrent Jésus, à cause de la préparation des Juifs, parce que le sépulcre était proche. 

\chapter
\verse Le premier jour de la semaine, Marie de Magdala se rendit au sépulcre dès le matin, comme il faisait encore obscur; et elle vit que la pierre était ôtée du sépulcre. 
\verse Elle courut vers Simon Pierre et vers l`autre disciple que Jésus aimait, et leur dit: Ils ont enlevé du sépulcre le Seigneur, et nous ne savons où ils l`ont mis. 
\verse Pierre et l`autre disciple sortirent, et allèrent au sépulcre. 
\verse Ils couraient tous deux ensemble. Mais l`autre disciple courut plus vite que Pierre, et arriva le premier au sépulcre; 
\verse s`étant baissé, il vit les bandes qui étaient à terre, cependant il n`entra pas. 
\verse Simon Pierre, qui le suivait, arriva et entra dans le sépulcre; il vit les bandes qui étaient à terre, 
\verse et le linge qu`on avait mis sur la tête de Jésus, non pas avec les bandes, mais plié dans un lieu à part. 
\verse Alors l`autre disciple, qui était arrivé le premier au sépulcre, entra aussi; et il vit, et il crut. 
\verse Car ils ne comprenaient pas encore que, selon l`Écriture, Jésus devait ressusciter des morts. 
\verse Et les disciples s`en retournèrent chez eux. 
\verse Cependant Marie se tenait dehors près du sépulcre, et pleurait. Comme elle pleurait, elle se baissa pour regarder dans le sépulcre; 
\verse et elle vit deux anges vêtus de blanc, assis à la place où avait été couché le corps de Jésus, l`un à la tête, l`autre aux pieds. 
\verse Ils lui dirent: Femme, pourquoi pleures-tu? Elle leur répondit: Parce qu`ils ont enlevé mon Seigneur, et je ne sais où ils l`ont mis. 
\verse En disant cela, elle se retourna, et elle vit Jésus debout; mais elle ne savait pas que c`était Jésus. 
\verse Jésus lui dit: Femme, pourquoi pleures-tu? Qui cherches-tu? Elle, pensant que c`était le jardinier, lui dit: Seigneur, si c`est toi qui l`as emporté, dis-moi où tu l`as mis, et je le prendrai. 
\verse Jésus lui dit: Marie! Elle se retourna, et lui dit en hébreu: Rabbouni! c`est-à-dire, Maître! 
\verse Jésus lui dit: Ne me touche pas; car je ne suis pas encore monté vers mon Père. Mais va trouver mes frères, et dis-leur que je monte vers mon Père et votre Père, vers mon Dieu et votre Dieu. 
\verse Marie de Magdala alla annoncer aux disciples qu`elle avait vu le Seigneur, et qu`il lui avait dit ces choses. 
\verse Le soir de ce jour, qui était le premier de la semaine, les portes du lieu où se trouvaient les disciples étant fermées, à cause de la crainte qu`ils avaient des Juifs, Jésus vint, se présenta au milieu d`eux, et leur dit: La paix soit avec vous! 
\verse Et quand il eut dit cela, il leur montra ses mains et son côté. Les disciples furent dans la joie en voyant le Seigneur. 
\verse Jésus leur dit de nouveau: La paix soit avec vous! Comme le Père m`a envoyé, moi aussi je vous envoie. 
\verse Après ces paroles, il souffla sur eux, et leur dit: Recevez le Saint Esprit. 
\verse Ceux à qui vous pardonnerez les péchés, ils leur seront pardonnés; et ceux à qui vous les retiendrez, ils leur seront retenus. 
\verse Thomas, appelé Didyme, l`un des douze, n`était pas avec eux lorsque Jésus vint. 
\verse Les autres disciples lui dirent donc: Nous avons vu le Seigneur. Mais il leur dit: Si je ne vois dans ses mains la marque des clous, et si je ne mets mon doigt dans la marque des clous, et si je ne mets ma main dans son côté, je ne croirai point. 
\verse Huit jours après, les disciples de Jésus étaient de nouveau dans la maison, et Thomas se trouvait avec eux. Jésus vint, les portes étant fermées, se présenta au milieu d`eux, et dit: La paix soit avec vous! 
\verse Puis il dit à Thomas: Avance ici ton doigt, et regarde mes mains; avance aussi ta main, et mets-la dans mon côté; et ne sois pas incrédule, mais crois. 
\verse Thomas lui répondit: Mon Seigneur et mon Dieu! Jésus lui dit: 
\verse Parce que tu m`as vu, tu as cru. Heureux ceux qui n`ont pas vu, et qui ont cru! 
\verse Jésus a fait encore, en présence de ses disciples, beaucoup d`autres miracles, qui ne sont pas écrits dans ce livre. 
\verse Mais ces choses ont été écrites afin que vous croyiez que Jésus est le Christ, le Fils de Dieu, et qu`en croyant vous ayez la vie en son nom. 

\chapter
\verse Après cela, Jésus se montra encore aux disciples, sur les bords de la mer de Tibériade. Et voici de quelle manière il se montra. 
\verse Simon Pierre, Thomas, appelé Didyme, Nathanaël, de Cana en Galilée, les fils de Zébédée, et deux autres disciples de Jésus, étaient ensemble. 
\verse Simon Pierre leur dit: Je vais pêcher. Ils lui dirent: Nous allons aussi avec toi. Ils sortirent et montèrent dans une barque, et cette nuit-là ils ne prirent rien. 
\verse Le matin étant venu, Jésus se trouva sur le rivage; mais les disciples ne savaient pas que c`était Jésus. 
\verse Jésus leur dit: Enfants, n`avez-vous rien à manger? Ils lui répondirent: Non. 
\verse Il leur dit: Jetez le filet du côté droit de la barque, et vous trouverez. Ils le jetèrent donc, et ils ne pouvaient plus le retirer, à cause de la grande quantité de poissons. 
\verse Alors le disciple que Jésus aimait dit à Pierre: C`est le Seigneur! Et Simon Pierre, dès qu`il eut entendu que c`était le Seigneur, mit son vêtement et sa ceinture, car il était nu, et se jeta dans la mer. 
\verse Les autres disciples vinrent avec la barque, tirant le filet plein de poissons, car ils n`étaient éloignés de terre que d`environ deux cents coudées. 
\verse Lorsqu`ils furent descendus à terre, ils virent là des charbons allumés, du poisson dessus, et du pain. 
\verse Jésus leur dit: Apportez des poissons que vous venez de prendre. 
\verse Simon Pierre monta dans la barque, et tira à terre le filet plein de cent cinquante-trois grands poissons; et quoiqu`il y en eût tant, le filet ne se rompit point. 
\verse Jésus leur dit: Venez, mangez. Et aucun des disciples n`osait lui demander: Qui es-tu? sachant que c`était le Seigneur. 
\verse Jésus s`approcha, prit le pain, et leur en donna; il fit de même du poisson. 
\verse C`était déjà la troisième fois que Jésus se montrait à ses disciples depuis qu`il était ressuscité des morts. 
\verse Après qu`ils eurent mangé, Jésus dit à Simon Pierre: Simon, fils de Jonas, m`aimes-tu plus que ne m`aiment ceux-ci? Il lui répondit: Oui, Seigneur, tu sais que je t`aime. Jésus lui dit: Pais mes agneaux. 
\verse Il lui dit une seconde fois: Simon, fils de Jonas, m`aimes-tu? Pierre lui répondit: Oui, Seigneur, tu sais que je t`aime. Jésus lui dit: Pais mes brebis. 
\verse Il lui dit pour la troisième fois: Simon, fils de Jonas, m`aimes-tu? Pierre fut attristé de ce qu`il lui avait dit pour la troisième fois: M`aimes-tu? Et il lui répondit: Seigneur, tu sais toutes choses, tu sais que je t`aime. Jésus lui dit: Pais mes brebis. 
\verse En vérité, en vérité, je te le dis, quand tu étais plus jeune, tu te ceignais toi-même, et tu allais où tu voulais; mais quand tu seras vieux, tu étendras tes mains, et un autre te ceindra, et te mènera où tu ne voudras pas. 
\verse Il dit cela pour indiquer par quelle mort Pierre glorifierait Dieu. Et ayant ainsi parlé, il lui dit: Suis-moi. 
\verse Pierre, s`étant retourné, vit venir après eux le disciple que Jésus aimait, celui qui, pendant le souper, s`était penché sur la poitrine de Jésus, et avait dit: Seigneur, qui est celui qui te livre? 
\verse En le voyant, Pierre dit à Jésus: Et celui-ci, Seigneur, que lui arrivera-t-il? 
\verse Jésus lui dit: Si je veux qu`il demeure jusqu`à ce que je vienne, que t`importe? Toi, suis-moi. 
\verse Là-dessus, le bruit courut parmi les frères que ce disciple ne mourrait point. Cependant Jésus n`avait pas dit à Pierre qu`il ne mourrait point; mais: Si je veux qu`il demeure jusqu`à ce que je vienne, que t`importe? 
\verse C`est ce disciple qui rend témoignage de ces choses, et qui les a écrites. Et nous savons que son témoignage est vrai. 
\verse Jésus a fait encore beaucoup d`autres choses; si on les écrivait en détail, je ne pense pas que le monde même pût contenir les livres qu`on écrirait. 
