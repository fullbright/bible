\book[b.EST]{b.est}


\chapter
\verse C`était du temps d`Assuérus, de cet Assuérus qui régnait depuis l`Inde jusqu`en Éthiopie sur cent vingt-sept provinces; 
\verse et le roi Assuérus était alors assis sur son trône royal à Suse, dans la capitale. 
\verse La troisième année de son règne, il fit un festin à tous ses princes et à ses serviteurs; les commandants de l`armée des Perses et des Mèdes, les grands et les chefs des provinces furent réunis en sa présence. 
\verse Il montra la splendide richesse de son royaume et l`éclatante magnificence de sa grandeur pendant nombre de jours, pendant cent quatre-vingts jours. 
\verse Lorsque ces jours furent écoulés, le roi fit pour tout le peuple qui se trouvait à Suse, la capitale, depuis le plus grand jusqu`au plus petit, un festin qui dura sept jours, dans la cour du jardin de la maison royale. 
\verse Des tentures blanches, vertes et bleues, étaient attachées par des cordons de byssus et de pourpre à des anneaux d`argent et à des colonnes de marbre. Des lits d`or et d`argent reposaient sur un pavé de porphyre, de marbre, de nacre et de pierres noires. 
\verse On servait à boire dans des vases d`or, de différentes espèces, et il y avait abondance de vin royal, grâce à la libéralité du roi. 
\verse Mais on ne forçait personne à boire, car le roi avait ordonné à tous les gens de sa maison de se conformer à la volonté de chacun. 
\verse La reine Vasthi fit aussi un festin pour les femmes dans la maison royale du roi Assuérus. 
\verse Le septième jour, comme le coeur du roi était réjoui par le vin, il ordonna à Mehuman, Biztha, Harbona, Bigtha, Abagtha, Zéthar et Carcas, les sept eunuques qui servaient devant le roi Assuérus, 
\verse d`amener en sa présence la reine Vasthi, avec la couronne royale, pour montrer sa beauté aux peuples et aux grands, car elle était belle de figure. 
\verse Mais la reine Vasthi refusa de venir, quand elle reçut par les eunuques l`ordre du roi. Et le roi fut très irrité, il fut enflammé de colère. 
\verse Alors le roi s`adressa aux sages qui avaient la connaissance des temps. Car ainsi se traitaient les affaires du roi, devant tous ceux qui connaissaient les lois et le droit. 
\verse Il avait auprès de lui Carschena, Schéthar, Admatha, Tarsis, Mérès, Marsena, Memucan, sept princes de Perse et de Médie, qui voyaient la face du roi et qui occupaient le premier rang dans le royaume. 
\verse Quelle loi, dit-il, faut-il appliquer à la reine Vasthi, pour n`avoir point exécuté ce que le roi Assuérus lui a ordonné par les eunuques? 
\verse Memucan répondit devant le roi et les princes: Ce n`est pas seulement à l`égard du roi que la reine Vasthi a mal agi; c`est aussi envers tous les princes et tous les peuples qui sont dans toutes les provinces du roi Assuérus. 
\verse Car l`action de la reine parviendra à la connaissance de toutes les femmes, et les portera à mépriser leurs maris; elles diront: Le roi Assuérus avait ordonné qu`on amenât en sa présence la reine Vasthi, et elle n`y est pas allée. 
\verse Et dès ce jour les princesses de Perse et de Médie qui auront appris l`action de la reine la rapporteront à tous les chefs du roi; de là beaucoup de mépris et de colère. 
\verse Si le roi le trouve bon, qu`on publie de sa part et qu`on inscrive parmi les lois des Perses et des Mèdes, avec défense de la transgresser, une ordonnance royale d`après laquelle Vasthi ne paraîtra plus devant le roi Assuérus et le roi donnera la dignité de reine à une autre qui soit meilleure qu`elle. 
\verse L`édit du roi sera connu dans tout son royaume, quelque grand qu`il soit, et toutes les femmes rendront honneur à leurs maris, depuis le plus grand jusqu`au plus petit. 
\verse Cet avis fut approuvé du roi et des princes, et le roi agit d`après la parole de Memucan. 
\verse Il envoya des lettres à toutes les provinces du royaume, à chaque province selon son écriture et à chaque peuple selon sa langue; elles portaient que tout homme devait être le maître dans sa maison, et qu`il parlerait la langue de son peuple. 

\chapter
\verse Après ces choses, lorsque la colère du roi Assuérus se fut calmée, il pensa à Vasthi, à ce qu`elle avait fait, et à la décision qui avait été prise à son sujet. 
\verse Alors ceux qui servaient le roi dirent: Qu`on cherche pour le roi des jeunes filles, vierges et belles de figure; 
\verse que le roi établisse dans toutes les provinces de son royaume des commissaires chargés de rassembler toutes les jeunes filles, vierges et belles de figure, à Suse, la capitale, dans la maison des femmes, sous la surveillance d`Hégué, eunuque du roi et gardien des femmes, qui leur donnera les choses nécessaires pour leur toilette; 
\verse et que la jeune fille qui plaira au roi devienne reine à la place de Vasthi. Cet avis eut l`approbation du roi, et il fit ainsi. 
\verse Il y avait dans Suse, la capitale, un Juif nommé Mardochée, fils de Jaïr, fils de Schimeï, fils de Kis, homme de Benjamin, 
\verse qui avait été emmené de Jérusalem parmi les captifs déportés avec Jeconia, roi de Juda, par Nebucadnetsar, roi de Babylone. 
\verse Il élevait Hadassa, qui est Esther, fille de son oncle; car elle n`avait ni père ni mère. La jeune fille était belle de taille et belle de figure. A la mort de son père et de sa mère, Mardochée l`avait adoptée pour fille. 
\verse Lorsqu`on eut publié l`ordre du roi et son édit, et qu`un grand nombre de jeunes filles furent rassemblées à Suse, la capitale, sous la surveillance d`Hégaï, Esther fut aussi prise et conduite dans la maison du roi, sous la surveillance d`Hégaï, gardien des femmes. 
\verse La jeune fille lui plut, et trouva grâce devant lui; il s`empressa de lui fournir les choses nécessaires pour sa toilette et pour sa subsistance, lui donna sept jeunes filles choisies dans la maison du roi, et la plaça avec ses jeunes filles dans le meilleur appartement de la maison des femmes. 
\verse Esther ne fit connaître ni son peuple ni sa naissance, car Mardochée lui avait défendu d`en parler. 
\verse Et chaque jour Mardochée allait et venait devant la cour de la maison des femmes, pour savoir comment se portait Esther et comment on la traitait. 
\verse Chaque jeune fille allait à son tour vers le roi Assuérus, après avoir employé douze mois à s`acquitter de ce qui était prescrit aux femmes; pendant ce temps, elles prenaient soin de leur toilette, six mois avec de l`huile de myrrhe, et six mois avec des aromates et des parfums en usage parmi les femmes. 
\verse C`est ainsi que chaque jeune fille allait vers le roi; et, quand elle passait de la maison des femmes dans la maison du roi, on lui laissait prendre avec elle tout ce qu`elle voulait. 
\verse Elle y allait le soir; et le lendemain matin elle passait dans la seconde maison des femmes, sous la surveillance de Schaaschgaz, eunuque du roi et gardien des concubines. Elle ne retournait plus vers le roi, à moins que le roi n`en eût le désir et qu`elle ne fût appelée par son nom. 
\verse Lorsque son tour d`aller vers le roi fut arrivé, Esther, fille d`Abichaïl, oncle de Mardochée qui l`avait adoptée pour fille, ne demanda que ce qui fut désigné par Hégaï, eunuque du roi et gardien des femmes. Esther trouvait grâce aux yeux de tous ceux qui la voyaient. 
\verse Esther fut conduite auprès du roi Assuérus, dans sa maison royale, le dixième mois, qui est le mois de Tébeth, la septième année de son règne. 
\verse Le roi aima Esther plus que toutes les autres femmes, et elle obtint grâce et faveur devant lui plus que toutes les autres jeunes filles. Il mit la couronne royale sur sa tête, et la fit reine à la place de Vasthi. 
\verse Le roi donna un grand festin à tous ses princes et à ses serviteurs, un festin en l`honneur d`Esther; il accorda du repos aux provinces, et fit des présents avec une libéralité royale. 
\verse La seconde fois qu`on assembla les jeunes filles, Mardochée était assis à la porte du roi. 
\verse Esther n`avait fait connaître ni sa naissance ni son peuple, car Mardochée le lui avait défendu, et elle suivait les ordres de Mardochée aussi fidèlement qu`à l`époque où elle était sous sa tutelle. 
\verse Dans ce même temps, comme Mardochée était assis à la porte du roi, Bigthan et Théresch, deux eunuques du roi, gardes du seuil, cédèrent à un mouvement d`irritation et voulurent porter la main sur le roi Assuérus. 
\verse Mardochée eut connaissance de la chose et en informa la reine Esther, qui la redit au roi de la part de Mardochée. 
\verse Le fait ayant été vérifié et trouvé exact, les deux eunuques furent pendus à un bois. Et cela fut écrit dans le livre des Chroniques en présence du roi. 

\chapter
\verse Après ces choses, le roi Assuérus fit monter au pouvoir Haman, fils d`Hammedatha, l`Agaguite; il l`éleva en dignité et plaça son siège au-dessus de ceux de tous les chefs qui étaient auprès de lui. 
\verse Tous les serviteurs du roi, qui se tenaient à la porte du roi, fléchissaient le genou et se prosternaient devant Haman, car tel était l`ordre du roi à son égard. Mais Mardochée ne fléchissait point le genou et ne se prosternait point. 
\verse Et les serviteurs du roi, qui se tenaient à la porte du roi, dirent à Mardochée: Pourquoi transgresses-tu l`ordre du roi? 
\verse Comme ils le lui répétaient chaque jour et qu`il ne les écoutait pas, ils en firent rapport à Haman, pour voir si Mardochée persisterait dans sa résolution; car il leur avait dit qu`il était Juif. 
\verse Et Haman vit que Mardochée ne fléchissait point le genou et ne se prosternait point devant lui. Il fut rempli de fureur; 
\verse mais il dédaigna de porter la main sur Mardochée seul, car on lui avait dit de quel peuple était Mardochée, et il voulut détruire le peuple de Mardochée, tous les Juifs qui se trouvaient dans tout le royaume d`Assuérus. 
\verse Au premier mois, qui est le mois de Nisan, la douzième année du roi Assuérus, on jeta le pur, c`est-à-dire le sort, devant Haman, pour chaque jour et pour chaque mois, jusqu`au douzième mois, qui est le mois d`Adar. 
\verse Alors Haman dit au roi Assuérus: Il y a dans toutes les provinces de ton royaume un peuple dispersé et à part parmi les peuples, ayant des lois différentes de celles de tous les peuples et n`observant point les lois du roi. Il n`est pas dans l`intérêt du roi de le laisser en repos. 
\verse Si le roi le trouve bon, qu`on écrive l`ordre de les faire périr; et je pèserai dix mille talents d`argent entre les mains des fonctionnaires, pour qu`on les porte dans le trésor du roi. 
\verse Le roi ôta son anneau de la main, et le remit à Haman, fils d`Hammedatha, l`Agaguite, ennemi des Juifs. 
\verse Et le roi dit à Haman: L`argent t`est donné, et ce peuple aussi; fais-en ce que tu voudras. 
\verse Les secrétaires du roi furent appelés le treizième jour du premier mois, et l`on écrivit, suivant tout ce qui fut ordonné par Haman, aux satrapes du roi, aux gouverneurs de chaque province et aux chefs de chaque peuple, à chaque province selon son écriture et à chaque peuple selon sa langue. Ce fut au nom du roi Assuérus que l`on écrivit, et on scella avec l`anneau du roi. 
\verse Les lettres furent envoyées par les courriers dans toutes les provinces du roi, pour qu`on détruisît, qu`on tuât et qu`on fît périr tous les Juifs, jeunes et vieux, petits enfants et femmes, en un seul jour, le treizième du douzième mois, qui est le mois d`Adar, et pour que leurs biens fussent livrés au pillage. 
\verse Ces lettres renfermaient une copie de l`édit qui devait être publié dans chaque province, et invitaient tous les peuples à se tenir prêts pour ce jour-là. 
\verse Les courriers partirent en toute hâte, d`après l`ordre du roi. L`édit fut aussi publié dans Suse, la capitale; et tandis que le roi et Haman étaient à boire, la ville de Suse était dans la consternation. 

\chapter
\verse Mardochée, ayant appris tout ce qui se passait, déchira ses vêtements, s`enveloppa d`un sac et se couvrit de cendre. Puis il alla au milieu de la ville en poussant avec force des cris amers, 
\verse et se rendit jusqu`à la porte du roi, dont l`entrée était interdite à toute personne revêtue d`un sac. 
\verse Dans chaque province, partout où arrivaient l`ordre du roi et son édit, il y eut une grande désolation parmi les Juifs; ils jeûnaient, pleuraient et se lamentaient, et beaucoup se couchaient sur le sac et la cendre. 
\verse Les servantes d`Esther et ses eunuques vinrent lui annoncer cela, et la reine fut très effrayée. Elle envoya des vêtements à Mardochée pour le couvrir et lui faire ôter son sac, mais il ne les accepta pas. 
\verse Alors Esther appela Hathac, l`un des eunuques que le roi avait placés auprès d`elle, et elle le chargea d`aller demander à Mardochée ce que c`était et d`où cela venait. 
\verse Hathac se rendit vers Mardochée sur la place de la ville, devant la porte du roi. 
\verse Et Mardochée lui raconta tout ce qui lui était arrivé, et lui indiqua la somme d`argent qu`Haman avait promis de livrer au trésor du roi en retour du massacre des Juifs. 
\verse Il lui donna aussi une copie de l`édit publié dans Suse en vue de leur destruction, afin qu`il le montrât à Esther et lui fît tout connaître; et il ordonna qu`Esther se rendît chez le roi pour lui demander grâce et l`implorer en faveur de son peuple. 
\verse Hathac vint rapporter à Esther les paroles de Mardochée. 
\verse Esther chargea Hathac d`aller dire à Mardochée: 
\verse Tous les serviteurs du roi et le peuple des provinces du roi savent qu`il existe une loi portant peine de mort contre quiconque, homme ou femme, entre chez le roi, dans la cour intérieure, sans avoir été appelé; celui-là seul a la vie sauve, à qui le roi tend le sceptre d`or. Et moi, je n`ai point été appelée auprès du roi depuis trente jours. 
\verse Lorsque les paroles d`Esther eurent été rapportées à Mardochée, 
\verse Mardochée fit répondre à Esther: Ne t`imagine pas que tu échapperas seule d`entre tous les Juifs, parce que tu es dans la maison du roi; 
\verse car, si tu te tais maintenant, le secours et la délivrance surgiront d`autre part pour les Juifs, et toi et la maison de ton père vous périrez. Et qui sait si ce n`est pas pour un temps comme celui-ci que tu es parvenue à la royauté? 
\verse Esther envoya dire à Mardochée: 
\verse Va, rassemble tous les Juifs qui se trouvent à Suse, et jeûnez pour moi, sans manger ni boire pendant trois jours, ni la nuit ni le jour. Moi aussi, je jeûnerai de même avec mes servantes, puis j`entrerai chez le roi, malgré la loi; et si je dois périr, je périrai. 
\verse Mardochée s`en alla, et fit tout ce qu`Esther lui avait ordonné. 

\chapter
\verse Le troisième jour, Esther mit ses vêtements royaux et se présenta dans la cour intérieure de la maison du roi, devant la maison du roi. Le roi était assis sur son trône royal dans la maison royale, en face de l`entrée de la maison. 
\verse Lorsque le roi vit la reine Esther debout dans la cour, elle trouva grâce à ses yeux; et le roi tendit à Esther le sceptre d`or qu`il tenait à la main. Esther s`approcha, et toucha le bout du sceptre. 
\verse Le roi lui dit: Qu`as-tu, reine Esther, et que demandes-tu? Quand ce serait la moitié du royaume, elle te serait donnée. 
\verse Esther répondit: Si le roi le trouve bon, que le roi vienne aujourd`hui avec Haman au festin que je lui ai préparé. 
\verse Et le roi dit: Allez tout de suite chercher Haman, comme le désire Esther. Le roi se rendit avec Haman au festin qu`avait préparé Esther. 
\verse Et pendant qu`on buvait le vin, le roi dit à Esther: Quelle est ta demande? Elle te sera accordée. Que désires-tu? Quand ce serait la moitié du royaume, tu l`obtiendras. 
\verse Esther répondit: Voici ce que je demande et ce que je désire. 
\verse Si j`ai trouvé grâce aux yeux du roi, et s`il plaît au roi d`accorder ma demande et de satisfaire mon désir, que le roi vienne avec Haman au festin que je leur préparerai, et demain je donnerai réponse au roi selon son ordre. 
\verse Haman sortit ce jour-là, joyeux et le coeur content. Mais lorsqu`il vit, à la porte du roi, Mardochée qui ne se levait ni ne se remuait devant lui, il fut rempli de colère contre Mardochée. 
\verse Il sut néanmoins se contenir, et il alla chez lui. Puis il envoya chercher ses amis et Zéresch, sa femme. 
\verse Haman leur parla de la magnificence de ses richesses, du nombre de ses fils, de tout ce qu`avait fait le roi pour l`élever en dignité, et du rang qu`il lui avait donné au-dessus des chefs et des serviteurs du roi. 
\verse Et il ajouta: Je suis même le seul que la reine Esther ait admis avec le roi au festin qu`elle a fait, et je suis encore invité pour demain chez elle avec le roi. 
\verse Mais tout cela n`est d`aucun prix pour moi aussi longtemps que je verrai Mardochée, le Juif, assis à la porte du roi. 
\verse Zéresch, sa femme, et tous ses amis lui dirent: Qu`on prépare un bois haut de cinquante coudées, et demain matin demande au roi qu`on y pende Mardochée; puis tu iras joyeux au festin avec le roi. Cet avis plut à Haman, et il fit préparer le bois. 

\chapter
\verse Cette nuit-là, le roi ne put pas dormir, et il se fit apporter le livre des annales, les Chroniques. On les lut devant le roi, 
\verse et l`on trouva écrit ce que Mardochée avait révélé au sujet de Bigthan et de Théresch, les deux eunuques du roi, gardes du seuil, qui avaient voulu porter la main sur le roi Assuérus. 
\verse Le roi dit: Quelle marque de distinction et d`honneur Mardochée a-t-il reçue pour cela? Il n`a rien reçu, répondirent ceux qui servaient le roi. 
\verse Alors le roi dit: Qui est dans la cour? -Haman était venu dans la cour extérieure de la maison du roi, pour demander au roi de faire pendre Mardochée au bois qu`il avait préparé pour lui. - 
\verse Les serviteurs du roi lui répondirent: C`est Haman qui se tient dans la cour. Et le roi dit: Qu`il entre. 
\verse Haman entra, et le roi lui dit: Que faut-il faire pour un homme que le roi veut honorer? Haman se dit en lui-même: Quel autre que moi le roi voudrait-il honorer? 
\verse Et Haman répondit au roi: Pour un homme que le roi veut honorer, 
\verse il faut prendre le vêtement royal dont le roi se couvre et le cheval que le roi monte et sur la tête duquel se pose une couronne royale, 
\verse remettre le vêtement et le cheval à l`un des principaux chefs du roi, puis revêtir l`homme que le roi veut honorer, le promener à cheval à travers la place de la ville, et crier devant lui: C`est ainsi que l`on fait à l`homme que le roi veut honorer! 
\verse Le roi dit à Haman: Prends tout de suite le vêtement et le cheval, comme tu l`as dit, et fais ainsi pour Mardochée, le Juif, qui est assis à la porte du roi; ne néglige rien de tout ce que tu as mentionné. 
\verse Et Haman prit le vêtement et le cheval, il revêtit Mardochée, il le promena à cheval à travers la place de la ville, et il cria devant lui: C`est ainsi que l`on fait à l`homme que le roi veut honorer! 
\verse Mardochée retourna à la porte du roi, et Haman se rendit en hâte chez lui, désolé et la tête voilée. 
\verse Haman raconta à Zéresch, sa femme, et à tous ses amis, tout ce qui lui était arrivé. Et ses sages, et Zéresch, sa femme, lui dirent: Si Mardochée, devant lequel tu as commencé de tomber, est de la race des Juifs, tu ne pourras rien contre lui, mais tu tomberas devant lui. 
\verse Comme ils lui parlaient encore, les eunuques du roi arrivèrent et conduisirent aussitôt Haman au festin qu`Esther avait préparé. 

\chapter
\verse Le roi et Haman allèrent au festin chez la reine Esther. 
\verse Ce second jour, le roi dit encore à Esther, pendant qu`on buvait le vin: Quelle est ta demande, reine Esther? Elle te sera accordée. Que désires-tu? Quand ce serait la moitié du royaume, tu l`obtiendras. 
\verse La reine Esther répondit: Si j`ai trouvé grâce à tes yeux, ô roi, et si le roi le trouve bon, accorde-moi la vie, voilà ma demande, et sauve mon peuple, voilà mon désir! 
\verse Car nous sommes vendus, moi et mon peuple, pour être détruits, égorgés, anéantis. Encore si nous étions vendus pour devenir esclaves et servantes, je me tairais, mais l`ennemi ne saurait compenser le dommage fait au roi. 
\verse Le roi Assuérus prit la parole et dit à la reine Esther: Qui est-il et où est-il celui qui se propose d`agir ainsi? 
\verse Esther répondit: L`oppresseur, l`ennemi, c`est Haman, ce méchant-là! Haman fut saisi de terreur en présence du roi et de la reine. 
\verse Et le roi, dans sa colère, se leva et quitta le festin, pour aller dans le jardin du palais. Haman resta pour demander grâce de la vie à la reine Esther, car il voyait bien que sa perte était arrêtée dans l`esprit du roi. 
\verse Lorsque le roi revint du jardin du palais dans la salle du festin, il vit Haman qui s`était précipité vers le lit sur lequel était Esther, et il dit: Serait-ce encore pour faire violence à la reine, chez moi, dans le palais? Dès que cette parole fut sortie de la bouche du roi, on voila le visage d`Haman. 
\verse Et Harbona, l`un des eunuques, dit en présence du roi: Voici, le bois préparé par Haman pour Mardochée, qui a parlé pour le bien du roi, est dressé dans la maison d`Haman, à une hauteur de cinquante coudées. Le roi dit: Qu`on y pende Haman! 
\verse Et l`on pendit Haman au bois qu`il avait préparé pour Mardochée. Et la colère du roi s`apaisa. 

\chapter
\verse En ce même jour, le roi Assuérus donna à la reine Esther la maison d`Haman, l`ennemi des Juifs; et Mardochée parut devant le roi, car Esther avait fait connaître la parenté qui l`unissait à elle. 
\verse Le roi ôta son anneau, qu`il avait repris à Haman, et le donna à Mardochée; Esther, de son côté, établit Mardochée sur la maison d`Haman. 
\verse Puis Esther parla de nouveau en présence du roi. Elle se jeta à ses pieds, elle pleura, elle le supplia d`empêcher les effets de la méchanceté d`Haman, l`Agaguite, et la réussite de ses projets contre les Juifs. 
\verse Le roi tendit le sceptre d`or à Esther, qui se releva et resta debout devant le roi. 
\verse Elle dit alors: Si le roi le trouve bon et si j`ai trouvé grâce devant lui, si la chose paraît convenable au roi et si je suis agréable à ses yeux, qu`on écrive pour révoquer les lettres conçues par Haman, fils d`Hammedatha, l`Agaguite, et écrites par lui dans le but de faire périr les Juifs qui sont dans toutes les provinces du roi. 
\verse Car comment pourrais-je voir le malheur qui atteindrait mon peuple, et comment pourrais-je voir la destruction de ma race? 
\verse Le roi Assuérus dit à la reine Esther et au Juif Mardochée: Voici, j`ai donné à Esther la maison d`Haman, et il a été pendu au bois pour avoir étendu la main contre les Juifs. 
\verse Écrivez donc en faveur des Juifs comme il vous plaira, au nom du roi, et scellez avec l`anneau du roi; car une lettre écrite au nom du roi et scellée avec l`anneau du roi ne peut être révoquée. 
\verse Les secrétaires du roi furent appelés en ce temps, le vingt-troisième jour du troisième mois, qui est le mois de Sivan, et l`on écrivit, suivant tout ce qui fut ordonné par Mardochée, aux Juifs, aux satrapes, aux gouverneurs et aux chefs des cent vingt-sept provinces situées de l`Inde à l`Éthiopie, à chaque province selon son écriture, à chaque peuple selon sa langue, et aux Juifs selon leur écriture et selon leur langue. 
\verse On écrivit au nom du roi Assuérus, et l`on scella avec l`anneau du roi. On envoya les lettres par des courriers ayant pour montures des chevaux et des mulets nés de juments. 
\verse Par ces lettres, le roi donnait aux Juifs, en quelque ville qu`ils fussent, la permission de se rassembler et de défendre leur vie, de détruire, de tuer et de faire périr, avec leurs petits enfants et leurs femmes, tous ceux de chaque peuple et de chaque province qui prendraient les armes pour les attaquer, et de livrer leurs biens au pillage, 
\verse et cela en un seul jour, dans toutes les provinces du roi Assuérus, le treizième du douzième mois, qui est le mois d`Adar. 
\verse Ces lettres renfermaient une copie de l`édit qui devait être publié dans chaque province, et informaient tous les peuples que les Juifs se tiendraient prêts pour ce jour-là à se venger de leurs ennemis. 
\verse Les courriers, montés sur des chevaux et des mulets, partirent aussitôt et en toute hâte, d`après l`ordre du roi. L`édit fut aussi publié dans Suse, la capitale. 
\verse Mardochée sortit de chez le roi, avec un vêtement royal bleu et blanc, une grande couronne d`or, et un manteau de byssus et de pourpre. La ville de Suse poussait des cris et se réjouissait. 
\verse Il n`y avait pour les Juifs que bonheur et joie, allégresse et gloire. 
\verse Dans chaque province et dans chaque ville, partout où arrivaient l`ordre du roi et son édit, il y eut parmi les Juifs de la joie et de l`allégresse, des festins et des fêtes. Et beaucoup de gens d`entre les peuples du pays se firent Juifs, car la crainte des Juifs les avait saisis. 

\chapter
\verse Au douzième mois, qui est le mois d`Adar, le treizième jour du mois, jour où devaient s`exécuter l`ordre et l`édit du roi, et où les ennemis des Juifs avaient espéré dominer sur eux, ce fut le contraire qui arriva, et les Juifs dominèrent sur leurs ennemis. 
\verse Les Juifs se rassemblèrent dans leurs villes, dans toutes les provinces du roi Assuérus, pour mettre la main sur ceux qui cherchaient leur perte; et personne ne put leur résister, car la crainte qu`on avait d`eux s`était emparée de tous les peuples. 
\verse Et tous les chefs des provinces, les satrapes, les gouverneurs, les fonctionnaires du roi, soutinrent les Juifs, à cause de l`effroi que leur inspirait Mardochée. 
\verse Car Mardochée était puissant dans la maison du roi, et sa renommée se répandait dans toutes les provinces, parce qu`il devenait de plus en plus puissant. 
\verse Les Juifs frappèrent à coups d`épée tous leurs ennemis, ils les tuèrent et les firent périr; ils traitèrent comme il leur plut ceux qui leur étaient hostiles. 
\verse Dans Suse, la capitale, les Juifs tuèrent et firent périr cinq cents hommes, 
\verse et ils égorgèrent Parschandatha, Dalphon, Aspatha, 
\verse Poratha, Adalia, Aridatha, 
\verse Parmaschtha, Arizaï, Aridaï et Vajezatha, 
\verse les dix fils d`Haman, fils d`Hammedatha, l`ennemi des Juifs. Mais ils ne mirent pas la main au pillage. 
\verse Ce jour-là, le nombre de ceux qui avaient été tués dans Suse, la capitale, parvint à la connaissance du roi. 
\verse Et le roi dit à la reine Esther: Les Juifs ont tué et fait périr dans Suse, la capitale, cinq cents hommes et les dix fils d`Haman; qu`auront-ils fait dans le reste des provinces du roi? Quelle est ta demande? Elle te sera accordée. Que désires-tu encore? Tu l`obtiendras. 
\verse Esther répondit: Si le roi le trouve bon, qu`il soit permis aux Juifs qui sont à Suse d`agir encore demain selon le décret d`aujourd`hui, et que l`on pende au bois les dix fils d`Haman. 
\verse Et le roi ordonna de faire ainsi. L`édit fut publié dans Suse. On pendit les dix fils d`Haman; 
\verse et les Juifs qui se trouvaient à Suse se rassemblèrent de nouveau le quatorzième jour du mois d`Adar et tuèrent dans Suse trois cents hommes. Mais ils ne mirent pas la main au pillage. 
\verse Les autres Juifs qui étaient dans les provinces du roi se rassemblèrent et défendirent leur vie; ils se procurèrent du repos en se délivrant de leurs ennemis, et ils tuèrent soixante-quinze mille de ceux qui leur étaient hostiles. Mais ils ne mirent pas la main au pillage. 
\verse Ces choses arrivèrent le treizième jour du mois d`Adar. Les Juifs se reposèrent le quatorzième, et ils en firent un jour de festin et de joie. 
\verse Ceux qui se trouvaient à Suse, s`étant rassemblés le treizième jour et le quatorzième jour, se reposèrent le quinzième, et ils en firent un jour de festin et de joie. 
\verse C`est pourquoi les Juifs de la campagne, qui habitent des villes sans murailles, font du quatorzième jour du mois d`Adar un jour de joie, de festin et de fête, où l`on s`envoie des portions les uns aux autres. 
\verse Mardochée écrivit ces choses, et il envoya des lettres à tous les Juifs qui étaient dans toutes les provinces du roi Assuérus, auprès et au loin. 
\verse Il leur prescrivait de célébrer chaque année le quatorzième jour et le quinzième jour du mois d`Adar 
\verse comme les jours où ils avaient obtenu du repos en se délivrant de leurs ennemis, de célébrer le mois où leur tristesse avait été changée en joie et leur désolation en jour de fête, et de faire de ces jours des jours de festin et de joie où l`on s`envoie des portions les uns aux autres et où l`on distribue des dons aux indigents. 
\verse Les Juifs s`engagèrent à faire ce qu`ils avaient déjà commencé et ce que Mardochée leur écrivit. 
\verse Car Haman, fils d`Hammedatha, l`Agaguite, ennemi de tous les Juifs, avait formé le projet de les faire périr, et il avait jeté le pur, c`est-à-dire le sort, afin de les tuer et de les détruire; 
\verse mais Esther s`étant présentée devant le roi, le roi ordonna par écrit de faire retomber sur la tête d`Haman le méchant projet qu`il avait formé contre les Juifs, et de le pendre au bois, lui et ses fils. 
\verse C`est pourquoi on appela ces jours Purim, du nom de pur. D`après tout le contenu de cette lettre, d`après ce qu`ils avaient eux-mêmes vu et ce qui leur était arrivé, 
\verse les Juifs prirent pour eux, pour leur postérité, et pour tous ceux qui s`attacheraient à eux, la résolution et l`engagement irrévocables de célébrer chaque année ces deux jours, selon le mode prescrit et au temps fixé. 
\verse Ces jours devaient être rappelés et célébrés de génération en génération, dans chaque famille, dans chaque province et dans chaque ville; et ces jours de Purim ne devaient jamais être abolis au milieu des Juifs, ni le souvenir s`en effacer parmi leurs descendants. 
\verse La reine Esther, fille d`Abichaïl, et le Juif Mardochée écrivirent d`une manière pressante une seconde fois pour confirmer la lettre sur les Purim. 
\verse On envoya des lettres à tous les Juifs, dans les cent vingt-sept provinces du roi Assuérus. Elles contenaient des paroles de paix et de fidélité, 
\verse pour prescrire ces jours de Purim au temps fixé, comme le Juif Mardochée et la reine Esther les avaient établis pour eux, et comme ils les avaient établis pour eux-mêmes et pour leur postérité, à l`occasion de leur jeûne et de leurs cris. 
\verse Ainsi l`ordre d`Esther confirma l`institution des Purim, et cela fut écrit dans le livre. 

\chapter
\verse Le roi Assuérus imposa un tribut au pays et aux îles de la mer. 
\verse Tous les faits concernant sa puissance et ses exploits, et les détails sur la grandeur à laquelle le roi éleva Mardochée, ne sont-ils pas écrits dans le livre des Chroniques des rois des Mèdes et des Perses? 
\verse Car le Juif Mardochée était le premier après le roi Assuérus; considéré parmi les Juifs et aimé de la multitude de ses frères, il rechercha le bien de son peuple et parla pour le bonheur de toute sa race. 
