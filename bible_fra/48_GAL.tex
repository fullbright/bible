\book[Épître aux Galates]{Galates}


\chapter
\verse Paul, apôtre, non de la part des hommes, ni par un homme, mais par Jésus Christ et Dieu le Père, qui l`a ressuscité des morts, 
\verse et tous les frères qui sont avec moi, aux Églises de la Galatie: 
\verse que la grâce et la paix vous soient données de la part de Dieu le Père et de notre Seigneur Jésus Christ, 
\verse qui s`est donné lui-même pour nos péchés, afin de nous arracher du présent siècle mauvais, selon la volonté de notre Dieu et Père, 
\verse à qui soit la gloire aux siècles des siècles! Amen! 
\verse Je m`étonne que vous vous détourniez si promptement de celui qui vous a appelés par la grâce de Christ, pour passer à un autre Évangile. 
\verse Non pas qu`il y ait un autre Évangile, mais il y a des gens qui vous troublent, et qui veulent renverser l`Évangile de Christ. 
\verse Mais, quand nous-mêmes, quand un ange du ciel annoncerait un autre Évangile que celui que nous vous avons prêché, qu`il soit anathème! 
\verse Nous l`avons dit précédemment, et je le répète à cette heure: si quelqu`un vous annonce un autre Évangile que celui que vous avez reçu, qu`il soit anathème! 
\verse Et maintenant, est-ce la faveur des hommes que je désire, ou celle de Dieu? Est-ce que je cherche à plaire aux hommes? Si je plaisais encore aux hommes, je ne serais pas serviteur de Christ. 
\verse Je vous déclare, frères, que l`Évangile qui a été annoncé par moi n`est pas de l`homme; 
\verse car je ne l`ai ni reçu ni appris d`un homme, mais par une révélation de Jésus Christ. 
\verse Vous avez su, en effet, quelle était autrefois ma conduite dans le judaïsme, comment je persécutais à outrance et ravageais l`Église de Dieu, 
\verse et comment j`étais plus avancé dans le judaïsme que beaucoup de ceux de mon âge et de ma nation, étant animé d`un zèle excessif pour les traditions de mes pères. 
\verse Mais, lorsqu`il plut à celui qui m`avait mis à part dès le sein de ma mère, et qui m`a appelé par sa grâce, 
\verse de révéler en moi son Fils, afin que je l`annonçasse parmi les païens, aussitôt, je ne consultai ni la chair ni le sang, 
\verse et je ne montai point à Jérusalem vers ceux qui furent apôtres avant moi, mais je partis pour l`Arabie. Puis je revins encore à Damas. 
\verse Trois ans plus tard, je montai à Jérusalem pour faire la connaissance de Céphas, et je demeurai quinze jours chez lui. 
\verse Mais je ne vis aucun autre des apôtres, si ce n`est Jacques, le frère du Seigneur. 
\verse Dans ce que je vous écris, voici, devant Dieu, je ne mens point. 
\verse J`allai ensuite dans les contrées de la Syrie et de la Cilicie. 
\verse Or, j`étais inconnu de visage aux Églises de Judée qui sont en Christ; 
\verse seulement, elles avaient entendu dire: Celui qui autrefois nous persécutait annonce maintenant la foi qu`il s`efforçait alors de détruire. 
\verse Et elles glorifiaient Dieu à mon sujet. 

\chapter
\verse Quatorze ans après, je montai de nouveau à Jérusalem avec Barnabas, ayant aussi pris Tite avec moi; 
\verse et ce fut d`après une révélation que j`y montai. Je leur exposai l`Évangile que je prêche parmi les païens, je l`exposai en particulier à ceux qui sont les plus considérés, afin de ne pas courir ou avoir couru en vain. 
\verse Mais Tite, qui était avec moi, et qui était Grec, ne fut pas même contraint de se faire circoncire. 
\verse Et cela, à cause des faux frères qui s`étaient furtivement introduits et glissés parmi nous, pour épier la liberté que nous avons en Jésus Christ, avec l`intention de nous asservir. 
\verse Nous ne leur cédâmes pas un instant et nous résistâmes à leurs exigences, afin que la vérité de l`Évangile fût maintenue parmi vous. 
\verse Ceux qui sont les plus considérés-quels qu`ils aient été jadis, cela ne m`importe pas: Dieu ne fait point acception de personnes, -ceux qui sont les plus considérés ne m`imposèrent rien. 
\verse Au contraire, voyant que l`Évangile m`avait été confié pour les incirconcis, comme à Pierre pour les circoncis, - 
\verse car celui qui a fait de Pierre l`apôtre des circoncis a aussi fait de moi l`apôtre des païens, - 
\verse et ayant reconnu la grâce qui m`avait été accordée, Jacques, Céphas et Jean, qui sont regardés comme des colonnes, me donnèrent, à moi et à Barnabas, la main d`association, afin que nous allassions, nous vers les païens, et eux vers les circoncis. 
\verse Ils nous recommandèrent seulement de nous souvenir des pauvres, ce que j`ai bien eu soin de faire. 
\verse Mais lorsque Céphas vint à Antioche, je lui résistai en face, parce qu`il était répréhensible. 
\verse En effet, avant l`arrivée de quelques personnes envoyées par Jacques, il mangeait avec les païens; et, quand elles furent venues, il s`esquiva et se tint à l`écart, par crainte des circoncis. 
\verse Avec lui les autres Juifs usèrent aussi de dissimulation, en sorte que Barnabas même fut entraîné par leur hypocrisie. 
\verse Voyant qu`ils ne marchaient pas droit selon la vérité de l`Évangile, je dis à Céphas, en présence de tous: Si toi qui es Juif, tu vis à la manière des païens et non à la manière des Juifs, pourquoi forces-tu les païens à judaïser? 
\verse Nous, nous sommes Juifs de naissance, et non pécheurs d`entre les païens. 
\verse Néanmoins, sachant que ce n`est pas par les oeuvres de la loi que l`homme est justifié, mais par la foi en Jésus Christ, nous aussi nous avons cru en Jésus Christ, afin d`être justifiés par la foi en Christ et non par les oeuvres de la loi, parce que nulle chair ne sera justifiée par les oeuvres de la loi. 
\verse Mais, tandis que nous cherchons à être justifié par Christ, si nous étions aussi nous-mêmes trouvés pécheurs, Christ serait-il un ministre du péché? Loin de là! 
\verse Car, si je rebâtis les choses que j`ai détruites, je me constitue moi-même un transgresseur, 
\verse car c`est par la loi que je suis mort à la loi, afin de vivre pour Dieu. 
\verse J`ai été crucifié avec Christ; et si je vis, ce n`est plus moi qui vis, c`est Christ qui vit en moi; si je vis maintenant dans la chair, je vis dans la foi au Fils de Dieu, qui m`a aimé et qui s`est livré lui-même pour moi. 
\verse Je ne rejette pas la grâce de Dieu; car si la justice s`obtient par la loi, Christ est donc mort en vain. 

\chapter
\verse O Galates, dépourvus de sens! qui vous a fascinés, vous, aux yeux de qui Jésus Christ a été peint comme crucifié? 
\verse Voici seulement ce que je veux apprendre de vous: Est-ce par les oeuvres de la loi que vous avez reçu l`Esprit, ou par la prédication de la foi? 
\verse Etes-vous tellement dépourvus de sens? Après avoir commencé par l`Esprit, voulez-vous maintenant finir par la chair? 
\verse Avez-vous tant souffert en vain? si toutefois c`est en vain. 
\verse Celui qui vous accorde l`Esprit, et qui opère des miracles parmi vous, le fait-il donc par les oeuvres de la loi, ou par la prédication de la foi? 
\verse Comme Abraham crut à Dieu, et que cela lui fut imputé à justice, 
\verse reconnaissez donc que ce sont ceux qui ont la foi qui sont fils d`Abraham. 
\verse Aussi l`Écriture, prévoyant que Dieu justifierait les païens par la foi, a d`avance annoncé cette bonne nouvelle à Abraham: Toutes les nations seront bénies en toi! 
\verse de sorte que ceux qui croient sont bénis avec Abraham le croyant. 
\verse Car tous ceux qui s`attachent aux oeuvres de la loi sont sous la malédiction; car il est écrit: Maudit est quiconque n`observe pas tout ce qui est écrit dans le livre de la loi, et ne le met pas en pratique. 
\verse Et que nul ne soit justifié devant Dieu par la loi, cela est évident, puisqu`il est dit: Le juste vivra par la foi. 
\verse Or, la loi ne procède pas de la foi; mais elle dit: Celui qui mettra ces choses en pratique vivra par elles. 
\verse Christ nous a rachetés de la malédiction de la loi, étant devenu malédiction pour nous-car il est écrit: Maudit est quiconque est pendu au bois, - 
\verse afin que la bénédiction d`Abraham eût pour les païens son accomplissement en Jésus Christ, et que nous reçussions par la foi l`Esprit qui avait été promis. 
\verse Frères (je parle à la manière des hommes), une disposition en bonne forme, bien que faite par un homme, n`est annulée par personne, et personne n`y ajoute. 
\verse Or les promesses ont été faites à Abraham et à sa postérité. Il n`est pas dit: et aux postérités, comme s`il s`agissait de plusieurs, mais en tant qu`il s`agit d`une seule: et à ta postérité, c`est-à-dire, à Christ. 
\verse Voici ce que j`entends: une disposition, que Dieu a confirmée antérieurement, ne peut pas être annulée, et ainsi la promesse rendue vaine, par la loi survenue quatre cents trente ans plus tard. 
\verse Car si l`héritage venait de la loi, il ne viendrait plus de la promesse; or, c`est par la promesse que Dieu a fait à Abraham ce don de sa grâce. 
\verse Pourquoi donc la loi? Elle a été donnée ensuite à cause des transgressions, jusqu`à ce que vînt la postérité à qui la promesse avait été faite; elle a été promulguée par des anges, au moyen d`un médiateur. 
\verse Or, le médiateur n`est pas médiateur d`un seul, tandis que Dieu est un seul. 
\verse La loi est-elle donc contre les promesses de Dieu? Loin de là! S`il eût été donné une loi qui pût procurer la vie, la justice viendrait réellement de la loi. 
\verse Mais l`Écriture a tout renfermé sous le péché, afin que ce qui avait été promis fût donné par la foi en Jésus Christ à ceux qui croient. 
\verse Avant que la foi vînt, nous étions enfermés sous la garde de la loi, en vue de la foi qui devait être révélée. 
\verse Ainsi la loi a été comme un pédagogue pour nous conduire à Christ, afin que nous fussions justifiés par la foi. 
\verse La foi étant venue, nous ne sommes plus sous ce pédagogue. 
\verse Car vous êtes tous fils de Dieu par la foi en Jésus Christ; 
\verse vous tous, qui avez été baptisés en Christ, vous avez revêtu Christ. 
\verse Il n`y a plus ni Juif ni Grec, il n`y a plus ni esclave ni libre, il n`y a plus ni homme ni femme; car tous vous êtes un en Jésus Christ. 
\verse Et si vous êtes à Christ, vous êtes donc la postérité d`Abraham, héritiers selon la promesse. 

\chapter
\verse Or, aussi longtemps que l`héritier est enfant, je dis qu`il ne diffère en rien d`un esclave, quoiqu`il soit le maître de tout; 
\verse mais il est sous des tuteurs et des administrateurs jusqu`au temps marqué par le père. 
\verse Nous aussi, de la même manière, lorsque nous étions enfants, nous étions sous l`esclavage des rudiments du monde; 
\verse mais, lorsque les temps ont été accomplis, Dieu a envoyé son Fils, né d`une femme, né sous la loi, 
\verse afin qu`il rachetât ceux qui étaient sous la loi, afin que nous reçussions l`adoption. 
\verse Et parce que vous êtes fils, Dieu a envoyé dans nos coeurs l`Esprit de son Fils, lequel crie: Abba! Père! 
\verse Ainsi tu n`es plus esclave, mais fils; et si tu es fils, tu es aussi héritier par la grâce de Dieu. 
\verse Autrefois, ne connaissant pas Dieu, vous serviez des dieux qui ne le sont pas de leur nature; 
\verse mais à présent que vous avez connu Dieu, ou plutôt que vous avez été connus de Dieu, comment retournez-vous à ces faibles et pauvres rudiments, auxquels de nouveau vous voulez vous asservir encore? 
\verse Vous observez les jours, les mois, les temps et les années! 
\verse Je crains d`avoir inutilement travaillé pour vous. 
\verse Soyez comme moi, car moi aussi je suis comme vous. Frères, je vous en supplie. 
\verse Vous ne m`avez fait aucun tort. Vous savez que ce fut à cause d`une infirmité de la chair que je vous ai pour la première fois annoncé l`Évangile. 
\verse Et mis à l`épreuve par ma chair, vous n`avez témoigné ni mépris ni dégoût; vous m`avez, au contraire, reçu comme un ange de Dieu, comme Jésus Christ. 
\verse Où donc est l`expression de votre bonheur? Car je vous atteste que, si cela eût été possible, vous vous seriez arraché les yeux pour me les donner. 
\verse Suis-je devenu votre ennemi en vous disant la vérité? 
\verse Le zèle qu`ils ont pour vous n`est pas pur, mais ils veulent vous détacher de nous, afin que vous soyez zélés pour eux. 
\verse Il est beau d`avoir du zèle pour ce qui est bien et en tout temps, et non pas seulement quand je suis présent parmi vous. 
\verse Mes enfants, pour qui j`éprouve de nouveau les douleurs de l`enfantement, jusqu`à ce que Christ soit formé en vous, 
\verse je voudrais être maintenant auprès de vous, et changer de langage, car je suis dans l`inquiétude à votre sujet. 
\verse Dites-moi, vous qui voulez être sous la loi, n`entendez-vous point la loi? 
\verse Car il est écrit qu`Abraham eut deux fils, un de la femme esclave, et un de la femme libre. 
\verse Mais celui de l`esclave naquit selon la chair, et celui de la femme libre naquit en vertu de la promesse. 
\verse Ces choses sont allégoriques; car ces femmes sont deux alliances. L`une du mont Sinaï, enfantant pour la servitude, c`est Agar, - 
\verse car Agar, c`est le mont Sinaï en Arabie, -et elle correspond à la Jérusalem actuelle, qui est dans la servitude avec ses enfants. 
\verse Mais la Jérusalem d`en haut est libre, c`est notre mère; 
\verse car il est écrit: Réjouis-toi, stérile, toi qui n`enfantes point! Éclate et pousse des cris, toi qui n`as pas éprouvé les douleurs de l`enfantement! Car les enfants de la délaissée seront plus nombreux Que les enfants de celle qui était mariée. 
\verse Pour vous, frères, comme Isaac, vous êtes enfants de la promesse; 
\verse et de même qu`alors celui qui était né selon la chair persécutait celui qui était né selon l`Esprit, ainsi en est-il encore maintenant. 
\verse Mais que dit l`Écriture? Chasse l`esclave et son fils, car le fils de l`esclave n`héritera pas avec le fils de la femme libre. 
\verse C`est pourquoi, frères, nous ne sommes pas enfants de l`esclave, mais de la femme libre. 

\chapter
\verse C`est pour la liberté que Christ nous a affranchis. Demeurez donc fermes, et ne vous laissez pas mettre de nouveau sous le joug de la servitude. 
\verse Voici, moi Paul, je vous dis que, si vous vous faites circoncire, Christ ne vous servira de rien. 
\verse Et je proteste encore une fois à tout homme qui se fait circoncire, qu`il est tenu de pratiquer la loi tout entière. 
\verse Vous êtes séparés de Christ, vous tous qui cherchez la justification dans la loi; vous êtes déchus de la grâce. 
\verse Pour nous, c`est de la foi que nous attendons, par l`Esprit, l`espérance de la justice. 
\verse Car, en Jésus Christ, ni la circoncision ni l`incirconcision n`a de valeur, mais la foi qui est agissante par la charité. 
\verse Vous couriez bien: qui vous a arrêtés, pour vous empêcher d`obéir à la vérité? 
\verse Cette influence ne vient pas de celui qui vous appelle. 
\verse Un peu de levain fait lever toute la pâte. 
\verse J`ai cette confiance en vous, dans le Seigneur, que vous ne penserez pas autrement. Mais celui qui vous trouble, quel qu`il soit, en portera la peine. 
\verse Pour moi, frères, si je prêche encore la circoncision, pourquoi suis-je encore persécuté? Le scandale de la croix a donc disparu! 
\verse Puissent-ils être retranchés, ceux qui mettent le trouble parmi vous! 
\verse Frères, vous avez été appelés à la liberté, seulement ne faites pas de cette liberté un prétexte de vivre selon la chair; mais rendez-vous, par la charité, serviteurs les uns des autres. 
\verse Car toute la loi est accomplie dans une seule parole, dans celle-ci: Tu aimeras ton prochain comme toi-même. 
\verse Mais si vous vous mordez et vous dévorez les uns les autres, prenez garde que vous ne soyez détruits les uns par les autres. 
\verse Je dis donc: Marchez selon l`Esprit, et vous n`accomplirez pas les désirs de la chair. 
\verse Car la chair a des désirs contraires à ceux de l`Esprit, et l`Esprit en a de contraires à ceux de la chair; ils sont opposés entre eux, afin que vous ne fassiez point ce que vous voudriez. 
\verse Si vous êtes conduits par l`Esprit, vous n`êtes point sous la loi. 
\verse Or, les oeuvres de la chair sont manifestes, ce sont l`impudicité, l`impureté, la dissolution, 
\verse l`idolâtrie, la magie, les inimitiés, les querelles, les jalousies, les animosités, les disputes, les divisions, les sectes, 
\verse l`envie, l`ivrognerie, les excès de table, et les choses semblables. Je vous dis d`avance, comme je l`ai déjà dit, que ceux qui commettent de telles choses n`hériteront point le royaume de Dieu. 
\verse Mais le fruit de l`Esprit, c`est l`amour, la joie, la paix, la patience, la bonté, la bénignité, la fidélité, la douceur, la tempérance; 
\verse la loi n`est pas contre ces choses. 
\verse Ceux qui sont à Jésus Christ ont crucifié la chair avec ses passions et ses désirs. 
\verse Si nous vivons par l`Esprit, marchons aussi selon l`Esprit. 
\verse Ne cherchons pas une vaine gloire, en nous provoquant les uns les autres, en nous portant envie les uns aux autres. 

\chapter
\verse Frères, si un homme vient à être surpris en quelque faute, vous qui êtes spirituels, redressez-le avec un esprit de douceur. Prends garde à toi-même, de peur que tu ne sois aussi tenté. 
\verse Portez les fardeaux les uns des autres, et vous accomplirez ainsi la loi de Christ. 
\verse Si quelqu`un pense être quelque chose, quoiqu`il ne soit rien, il s`abuse lui-même. 
\verse Que chacun examine ses propres oeuvres, et alors il aura sujet de se glorifier pour lui seul, et non par rapport à autrui; 
\verse car chacun portera son propre fardeau. 
\verse Que celui à qui l`on enseigne la parole fasse part de tous ses biens à celui qui l`enseigne. 
\verse Ne vous y trompez pas: on ne se moque pas de Dieu. Ce qu`un homme aura semé, il le moissonnera aussi. 
\verse Celui qui sème pour sa chair moissonnera de la chair la corruption; mais celui qui sème pour l`Esprit moissonnera de l`Esprit la vie éternelle. 
\verse Ne nous lassons pas de faire le bien; car nous moissonnerons au temps convenable, si nous ne nous relâchons pas. 
\verse Ainsi donc, pendant que nous en avons l`occasion, pratiquons le bien envers tous, et surtout envers les frères en la foi. 
\verse Voyez avec quelles grandes lettres je vous ai écrit de ma propre main. 
\verse Tous ceux qui veulent se rendre agréables selon la chair vous contraignent à vous faire circoncire, uniquement afin de n`être pas persécutés pour la croix de Christ. 
\verse Car les circoncis eux-mêmes n`observent point la loi; mais ils veulent que vous soyez circoncis, pour se glorifier dans votre chair. 
\verse Pour ce qui me concerne, loin de moi la pensée de me glorifier d`autre chose que de la croix de notre Seigneur Jésus Christ, par qui le monde est crucifié pour moi, comme je le suis pour le monde! 
\verse Car ce n`est rien que d`être circoncis ou incirconcis; ce qui est quelque chose, c`est d`être une nouvelle créature. 
\verse Paix et miséricorde sur tous ceux qui suivront cette règle, et sur l`Israël de Dieu! 
\verse Que personne désormais ne me fasse de la peine, car je porte sur mon corps les marques de Jésus. 
\verse Frères, que la grâce de notre Seigneur Jésus Christ soit avec votre esprit! Amen! 
