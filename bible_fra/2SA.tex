\book[b.2SA]{b.2sa}


\chapter
\verse Après la mort de Saül, David, qui avait battu les Amalécites, était depuis deux jours revenu à Tsiklag. 
\verse Le troisième jour, un homme arriva du camp de Saül, les vêtements déchirés et la tête couverte de terre. Lorsqu`il fut en présence de David, il se jeta par terre et se prosterna. 
\verse David lui dit: D`où viens-tu? Et il lui répondit: Je me suis sauvé du camp d`Israël. 
\verse David lui dit: Que s`est-il passé? dis-moi donc! Et il répondit: Le peuple s`est enfui du champ de bataille, et un grand nombre d`hommes sont tombés et ont péri; Saül même et Jonathan, son fils, sont morts. 
\verse David dit au jeune homme qui lui apportait ces nouvelles: Comment sais-tu que Saül et Jonathan, son fils, sont morts? 
\verse Et le jeune homme qui lui apportait ces nouvelles répondit: Je me trouvais sur la montagne de Guilboa; et voici, Saül s`appuyait sur sa lance, et voici, les chars et les cavaliers étaient près de l`atteindre. 
\verse S`étant retourné, il m`aperçut et m`appela. Je dis: Me voici! 
\verse Et il me dit: Qui es-tu? Je lui répondis: Je suis Amalécite. 
\verse Et il dit: Approche donc, et donne-moi la mort; car je suis pris de vertige, quoique encore plein de vie. 
\verse Je m`approchai de lui, et je lui donnai la mort, sachant bien qu`il ne survivrait pas à sa défaite. J`ai enlevé le diadème qui était sur sa tête et le bracelet qu`il avait au bras, et je les apporte ici à mon seigneur. 
\verse David saisit ses vêtements et les déchira, et tous les hommes qui étaient auprès de lui firent de même. 
\verse Ils furent dans le deuil, pleurèrent et jeûnèrent jusqu`au soir, à cause de Saül, de Jonathan, son fils, du peuple de l`Éternel, et de la maison d`Israël, parce qu`ils étaient tombés par l`épée. 
\verse David dit au jeune homme qui lui avait apporté ces nouvelles: D`où es-tu? Et il répondit: Je suis le fils d`un étranger, d`un Amalécite. 
\verse David lui dit: Comment n`as-tu pas craint de porter la main sur l`oint de l`Éternel et de lui donner la mort? 
\verse Et David appela l`un de ses gens, et dit: Approche, et tue-le! Cet homme frappa l`Amalécite, qui mourut. 
\verse Et David lui dit: Que ton sang retombe sur ta tête, car ta bouche a déposé contre toi, puisque tu as dit: J`ai donné la mort à l`oint de l`Éternel! 
\verse Voici le cantique funèbre que David composa sur Saül et sur Jonathan, son fils, 
\verse et qu`il ordonna d`enseigner aux enfants de Juda. C`est le cantique de l`arc: il est écrit dans le livre du Juste. 
\verse L`élite d`Israël a succombé sur tes collines! Comment des héros sont-ils tombés? 
\verse Ne l`annoncez point dans Gath, N`en publiez point la nouvelle dans les rues d`Askalon, De peur que les filles des Philistins ne se réjouissent, De peur que les filles des incirconcis ne triomphent. 
\verse Montagnes de Guilboa! Qu`il n`y ait sur vous ni rosée ni pluie, Ni champs qui donnent des prémices pour les offrandes! Car là ont été jetés les boucliers des héros, Le bouclier de Saül; L`huile a cessé de les oindre. 
\verse Devant le sang des blessés, devant la graisse des plus vaillants, L`arc de Jonathan n`a jamais reculé, Et l`épée de Saül ne retournait point à vide. 
\verse Saül et Jonathan, aimables et chéris pendant leur vie, N`ont point été séparés dans leur mort; Ils étaient plus légers que les aigles, Ils étaient plus forts que les lions. 
\verse Filles d`Israël! pleurez sur Saül, Qui vous revêtait magnifiquement de cramoisi, Qui mettait des ornements d`or sur vos habits. 
\verse Comment des héros sont-ils tombés au milieu du combat? Comment Jonathan a-t-il succombé sur tes collines? 
\verse Je suis dans la douleur à cause de toi, Jonathan, mon frère! Tu faisais tout mon plaisir; Ton amour pour moi était admirable, Au-dessus de l`amour des femmes. 
\verse Comment des héros sont-ils tombés? Comment leurs armes se sont-elles perdues? 

\chapter
\verse Après cela, David consulta l`Éternel, en disant: Monterai-je dans une des villes de Juda? L`Éternel lui répondit: Monte. David dit: Où monterai-je? Et l`Éternel répondit: A Hébron. 
\verse David y monta, avec ses deux femmes, Achinoam de Jizreel, et Abigaïl de Carmel, femme de Nabal. 
\verse David fit aussi monter les gens qui étaient auprès de lui, chacun avec sa maison; et ils habitèrent dans les villes d`Hébron. 
\verse Les hommes de Juda vinrent, et là ils oignirent David pour roi sur la maison de Juda. On informa David que c`étaient les gens de Jabès en Galaad qui avaient enterré Saül. 
\verse David envoya des messagers aux gens de Jabès en Galaad, pour leur dire: Soyez bénis de l`Éternel, puisque vous avez ainsi montré de la bienveillance envers Saül, votre maître, et que vous l`avez enterré. 
\verse Et maintenant, que l`Éternel use envers vous de bonté et de fidélité. Moi aussi je vous ferai du bien, parce que vous avez agi de la sorte. 
\verse Que vos mains se fortifient, et soyez de vaillants hommes; car votre maître Saül est mort, et c`est moi que la maison de Juda a oint pour roi sur elle. 
\verse Cependant Abner, fils de Ner, chef de l`armée de Saül, prit Isch Boscheth, fils de Saül, et le fit passer à Mahanaïm. 
\verse Il l`établit roi sur Galaad, sur les Gueschuriens, sur Jizreel, sur Éphraïm, sur Benjamin, sur tout Israël. 
\verse Isch Boscheth, fils de Saül, était âgé de quarante ans, lorsqu`il devint roi d`Israël, et il régna deux ans. Il n`y eut que la maison de Juda qui resta attachée à David. 
\verse Le temps pendant lequel David régna à Hébron sur la maison de Juda fut de sept ans et six mois. 
\verse Abner, fils de Ner, et les gens d`Isch Boscheth, fils de Saül, sortirent de Mahanaïm pour marcher sur Gabaon. 
\verse Joab, fils de Tseruja, et les gens de David, se mirent aussi en marche. Ils se rencontrèrent près de l`étang de Gabaon, et ils s`arrêtèrent les uns en deçà de l`étang, et les autres au delà. 
\verse Abner dit à Joab: Que ces jeunes gens se lèvent, et qu`ils se battent devant nous! Joab répondit: Qu`ils se lèvent! 
\verse Ils se levèrent et s`avancèrent en nombre égal, douze pour Benjamin et pour Isch Boscheth, fils de Saül, et douze des gens de David. 
\verse Chacun saisissant son adversaire par la tête lui enfonça son épée dans le flanc, et ils tombèrent tous ensemble. Et l`on donna à ce lieu, qui est près de Gabaon, le nom de Helkath Hatsurim. 
\verse Il y eut en ce jour un combat très rude, dans lequel Abner et les hommes d`Israël furent battus par les gens de David. 
\verse Là se trouvaient les trois fils de Tseruja: Joab, Abischaï et Asaël. Asaël avait les pieds légers comme une gazelle des champs: 
\verse il poursuivit Abner, sans se détourner de lui pour aller à droite ou à gauche. 
\verse Abner regarda derrière lui, et dit: Est-ce toi, Asaël? Et il répondit: C`est moi. 
\verse Abner lui dit: Tire à droite ou à gauche; saisis-toi de l`un de ces jeunes gens, et prends sa dépouille. Mais Asaël ne voulut point se détourner de lui. 
\verse Abner dit encore à Asaël: Détourne-toi de moi; pourquoi te frapperais-je et t`abattrais-je en terre? comment ensuite lèverais-je le visage devant ton frère Joab? 
\verse Et Asaël refusa de se détourner. Sur quoi Abner le frappa au ventre avec l`extrémité inférieure de sa lance, et la lance sortit par derrière. Il tomba et mourut sur place. Tous ceux qui arrivaient au lieu où Asaël était tombé mort, s`y arrêtaient. 
\verse Joab et Abischaï poursuivirent Abner, et le soleil se couchait quand ils arrivèrent au coteau d`Amma, qui est en face de Guiach, sur le chemin du désert de Gabaon. 
\verse Les fils de Benjamin se rallièrent à la suite d`Abner et formèrent un corps, et ils s`arrêtèrent au sommet d`une colline. 
\verse Abner appela Joab, et dit: L`épée dévorera-t-elle toujours? Ne sais-tu pas qu`il y aura de l`amertume à la fin? Jusques à quand tarderas-tu à dire au peuple de ne plus poursuivre ses frères? 
\verse Joab répondit: Dieu est vivant! si tu n`eusses parlé, le peuple n`aurait pas cessé avant le matin de poursuivre ses frères. 
\verse Et Joab sonna de la trompette, et tout le peuple s`arrêta; ils ne poursuivirent plus Israël, et ils ne continuèrent pas à se battre. 
\verse Abner et ses gens marchèrent toute la nuit dans la plaine; ils passèrent le Jourdain, traversèrent en entier le Bithron, et arrivèrent à Mahanaïm. 
\verse Joab revint de la poursuite d`Abner, et rassembla tout le peuple; il manquait dix-neuf hommes des gens de David, et Asaël. 
\verse Mais les gens de David avaient frappé à mort trois cent soixante hommes parmi ceux de Benjamin et d`Abner. 
\verse Ils emportèrent Asaël, et l`enterrèrent dans le sépulcre de son père à Bethléhem. Joab et ses gens marchèrent toute la nuit, et le jour paraissait quand ils furent à Hébron. 

\chapter
\verse La guerre dura longtemps entre la maison de Saül et la maison de David. David devenait de plus en plus fort, et la maison de Saül allait en s`affaiblissant. 
\verse Il naquit à David des fils à Hébron. Son premier-né fut Amnon, d`Achinoam de Jizreel; 
\verse le second, Kileab, d`Abigaïl de Carmel, femme de Nabal; le troisième, Absalom, fils de Maaca, fille de Talmaï, roi de Gueschur; 
\verse le quatrième, Adonija, fils de Haggith; le cinquième, Schephathia, fils d`Abithal; 
\verse et le sixième, Jithream, d`Égla, femme de David. Ce sont là ceux qui naquirent à David à Hébron. 
\verse Pendant la guerre entre la maison de Saül et la maison de David, Abner tint ferme pour la maison de Saül. 
\verse Or Saül avait eu une concubine, nommée Ritspa, fille d`Ajja. Et Isch Boscheth dit à Abner: Pourquoi es-tu venu vers la concubine de mon père? 
\verse Abner fut très irrité des paroles d`Isch Boscheth, et il répondit: Suis-je une tête de chien, qui tienne pour Juda? Je fais aujourd`hui preuve de bienveillance envers la maison de Saül, ton père, envers ses frères et ses amis, je ne t`ai pas livré entre les mains de David, et c`est aujourd`hui que tu me reproches une faute avec cette femme? 
\verse Que Dieu traite Abner dans toute sa rigueur, si je n`agis pas avec David selon ce que l`Éternel a juré à David, 
\verse en disant qu`il ferait passer la royauté de la maison de Saül dans la sienne, et qu`il établirait le trône de David sur Israël et sur Juda depuis Dan jusqu`à Beer Schéba. 
\verse Isch Boscheth n`osa pas répliquer un seul mot à Abner, parce qu`il le craignait. 
\verse Abner envoya des messagers à David pour lui dire de sa part: A qui est le pays? Fais alliance avec moi, et voici, ma main t`aidera pour tourner vers toi tout Israël. 
\verse Il répondit: Bien! je ferai alliance avec toi; mais je te demande une chose, c`est que tu ne voies point ma face, à moins que tu n`amènes d`abord Mical, fille de Saül, en venant auprès de moi. 
\verse Et David envoya des messagers à Isch Boscheth, fils de Saül, pour lui dire: Donne-moi ma femme Mical, que j`ai fiancée pour cent prépuces de Philistins. 
\verse Isch Boscheth la fit prendre chez son mari Palthiel, fils de Laïsch. 
\verse Et son mari la suivit en pleurant jusqu`à Bachurim. Alors Abner lui dit: Va, retourne-t`en! Et il s`en retourna. 
\verse Abner eut un entretien avec les anciens d`Israël, et leur dit: Vous désiriez autrefois d`avoir David pour roi; 
\verse établissez-le maintenant, car l`Éternel a dit de lui: C`est par David, mon serviteur, que je délivrerai mon peuple d`Israël de la main des Philistins et de la main de tous ses ennemis. 
\verse Abner parla aussi à Benjamin, et il alla rapporter aux oreilles de David à Hébron ce qu`avaient résolu Israël et toute la maison de Benjamin. 
\verse Il arriva auprès de David à Hébron, accompagné de vingt hommes; et David fit un festin à Abner et à ceux qui étaient avec lui. 
\verse Abner dit à David: Je me lèverai, et je partirai pour rassembler tout Israël vers mon seigneur le roi; ils feront alliance avec toi, et tu règneras entièrement selon ton désir. David renvoya Abner, qui s`en alla en paix. 
\verse Voici, Joab et les gens de David revinrent d`une excursion, et amenèrent avec eux un grand butin. Abner n`était plus auprès de David à Hébron, car David l`avait renvoyé, et il s`en était allé en paix. 
\verse Lorsque Joab et toute sa troupe arrivèrent, on fit à Joab ce rapport: Abner, fils de Ner, est venu auprès du roi, qui l`a renvoyé, et il s`en est allé en paix. 
\verse Joab se rendit chez le roi, et dit: Qu`as-tu fait? Voici, Abner est venu vers toi; pourquoi l`as-tu renvoyé et laissé partir? 
\verse Tu connais Abner, fils de Ner! c`est pour te tromper qu`il est venu, pour épier tes démarches, et pour savoir tout ce que tu fais. 
\verse Et Joab, après avoir quitté David, envoya sur les traces d`Abner des messagers, qui le ramenèrent depuis la citerne de Sira: David n`en savait rien. 
\verse Lorsque Abner fut de retour à Hébron, Joab le tira à l`écart au milieu de la porte, comme pour lui parler en secret, et là il le frappa au ventre et le tua, pour venger la mort d`Asaël, son frère. 
\verse David l`apprit ensuite, et il dit: Je suis à jamais innocent, devant l`Éternel, du sang d`Abner, fils de Ner, et mon royaume l`est aussi. 
\verse Que ce sang retombe sur Joab et sur toute la maison de son père! Qu`il y ait toujours quelqu`un dans la maison de Joab, qui soit atteint d`un flux ou de la lèpre, ou qui s`appuie sur un bâton, ou qui tombe par l`épée, ou qui manque de pain! 
\verse Ainsi Joab et Abischaï, son frère, tuèrent Abner, parce qu`il avait donné la mort à Asaël, leur frère, à Gabaon, dans la bataille. 
\verse David dit à Joab et à tout le peuple qui était avec lui: Déchirez vos vêtements, ceignez-vous de sacs, et pleurez devant Abner! Et le roi David marcha derrière le cercueil. 
\verse On enterra Abner à Hébron. Le roi éleva la voix et pleura sur le sépulcre d`Abner, et tout le peuple pleura. 
\verse Le roi fit une complainte sur Abner, et dit: Abner devait-il mourir comme meurt un criminel? 
\verse Tu n`avais ni les mains liées, ni les pieds dans les chaînes! Tu es tombé comme on tombe devant des méchants. 
\verse Et tout le peuple pleura de nouveau sur Abner. Tout le peuple s`approcha de David pour lui faire prendre quelque nourriture, pendant qu`il était encore jour; mais David jura, en disant: Que Dieu me traite dans toute sa rigueur, si je goûte du pain ou quoi que ce soit avant le coucher du soleil! 
\verse Cela fut connu et approuvé de tout le peuple, qui trouva bon tout ce qu`avait fait le roi. 
\verse Tout le peuple et tout Israël comprirent en ce jour que ce n`était pas par ordre du roi qu`Abner, fils de Ner, avait été tué. 
\verse Le roi dit à ses serviteurs: Ne savez-vous pas qu`un chef, qu`un grand homme, est tombé aujourd`hui en Israël? 
\verse Je suis encore faible, quoique j`aie reçu l`onction royale; et ces gens, les fils de Tseruja, sont trop puissants pour moi. Que l`Éternel rende selon sa méchanceté à celui qui fait le mal! 

\chapter
\verse Lorsque le fils de Saül apprit qu`Abner était mort à Hébron, ses mains restèrent sans force, et tout Israël fut dans l`épouvante. 
\verse Le fils de Saül avait deux chefs de bandes, dont l`un s`appelait Baana et l`autre Récab; ils étaient fils de Rimmon de Beéroth, d`entre les fils de Benjamin. -Car Beéroth était regardée comme faisant partie de Benjamin, 
\verse et les Beérothiens s`étaient enfuis à Guitthaïm, où ils ont habité jusqu`à ce jour. 
\verse Jonathan, fils de Saül, avait un fils perclus des pieds; et âgé de cinq ans lorsqu`arriva de Jizreel la nouvelle de la mort de Saül et de Jonathan; sa nourrice le prit et s`enfuit, et, comme elle précipitait sa fuite, il tomba et resta boiteux; son nom était Mephiboscheth. 
\verse Or les fils de Rimmon de Beéroth, Récab et Baana, se rendirent pendant la chaleur du jour à la maison d`Isch Boscheth, qui était couché pour son repos de midi. 
\verse Ils pénétrèrent jusqu`au milieu de la maison, comme pour prendre du froment, et ils le frappèrent au ventre; puis Récab et Baana, son frère, se sauvèrent. 
\verse Ils entrèrent donc dans la maison pendant qu`il reposait sur son lit dans sa chambre à coucher, ils le frappèrent et le firent mourir, et ils lui coupèrent la tête. Ils prirent sa tête, et ils marchèrent toute la nuit au travers de la plaine. 
\verse Ils apportèrent la tête d`Isch Boscheth à David dans Hébron, et ils dirent au roi: Voici la tête d`Isch Boscheth, fils de Saül, ton ennemi, qui en voulait à ta vie; l`Éternel venge aujourd`hui le roi mon seigneur de Saül et de sa race. 
\verse David répondit à Récab et à Baana, son frère, fils de Rimmon de Beéroth: L`Éternel qui m`a délivré de tout péril est vivant! 
\verse celui qui est venu me dire: Voici, Saül est mort, et qui croyait m`annoncer une bonne nouvelle, je l`ai fait saisir et tuer à Tsiklag, pour lui donner le salaire de son message; 
\verse et quand des méchants ont assassiné un homme juste dans sa maison et sur sa couche, combien plus ne redemanderai-je pas son sang de vos mains et ne vous exterminerai-je pas de la terre? 
\verse Et David ordonna à ses gens de les tuer; ils leur coupèrent les mains et les pieds, et les pendirent au bord de l`étang d`Hébron. Ils prirent ensuite la tête d`Isch Boscheth, et l`enterrèrent dans le sépulcre d`Abner à Hébron. 

\chapter
\verse Toutes les tribus d`Israël vinrent auprès de David, à Hébron, et dirent: Voici, nous sommes tes os et ta chair. 
\verse Autrefois déjà, lorsque Saül était notre roi, c`était toi qui conduisais et qui ramenais Israël. L`Éternel t`a dit: Tu paîtras mon peuple d`Israël, et tu seras le chef d`Israël. 
\verse Ainsi tous les anciens d`Israël vinrent auprès du roi à Hébron, et le roi David fit alliance avec eux à Hébron, devant l`Éternel. Ils oignirent David pour roi sur Israël. 
\verse David était âgé de trente ans lorsqu`il devint roi, et il régna quarante ans. 
\verse A Hébron il régna sur Juda sept ans et six mois, et à Jérusalem il régna trente-trois ans sur tout Israël et Juda. 
\verse Le roi marcha avec ses gens sur Jérusalem contre les Jébusiens, habitants du pays. Ils dirent à David: Tu n`entreras point ici, car les aveugles mêmes et les boiteux te repousseront! Ce qui voulait dire: David n`entrera point ici. 
\verse Mais David s`empara de la forteresse de Sion: c`est la cité de David. 
\verse David avait dit en ce jour: Quiconque battra les Jébusiens et atteindra le canal, quiconque frappera ces boiteux et ces aveugles qui sont les ennemis de David... -C`est pourquoi l`on dit: L`aveugle et le boiteux n`entreront point dans la maison. 
\verse David s`établit dans la forteresse, qu`il appela cité de David. Il fit de tous côtés des constructions, en dehors et en dedans de Millo. 
\verse David devenait de plus en plus grand, et l`Éternel, le Dieu des armées, était avec lui. 
\verse Hiram, roi de Tyr, envoya des messagers à David, et du bois de cèdre, et des charpentiers et des tailleurs de pierres, qui bâtirent une maison pour David. 
\verse David reconnut que l`Éternel l`affermissait comme roi d`Israël, et qu`il élevait son royaume à cause de son peuple d`Israël. 
\verse David prit encore des concubines et des femmes de Jérusalem, après qu`il fut venu d`Hébron, et il lui naquit encore des fils et des filles. 
\verse Voici les noms de ceux qui lui naquirent à Jérusalem: Schammua, Schobab, Nathan, Salomon, 
\verse Jibhar, Élischua, Népheg, Japhia, 
\verse Élischama, Éliada et Éliphéleth. 
\verse Les Philistins apprirent qu`on avait oint David pour roi sur Israël, et ils montèrent tous à sa recherche. David, qui en fut informé, descendit à la forteresse. 
\verse Les Philistins arrivèrent, et se répandirent dans la vallée des Rephaïm. 
\verse David consulta l`Éternel, en disant: Monterai-je contre les Philistins? Les livreras-tu entre mes mains? Et l`Éternel dit à David: Monte, car je livrerai les Philistins entre tes mains. 
\verse David vint à Baal Peratsim, où il les battit. Puis il dit: L`Éternel a dispersé mes ennemis devant moi, comme des eaux qui s`écoulent. C`est pourquoi l`on a donné à ce lieu le nom de Baal Peratsim. 
\verse Ils laissèrent là leurs idoles, et David et ses gens les emportèrent. 
\verse Les Philistins montèrent de nouveau, et se répandirent dans la vallée des Rephaïm. 
\verse David consulta l`Éternel. Et l`Éternel dit: Tu ne monteras pas; tourne-les par derrière, et tu arriveras sur eux vis-à-vis des mûriers. 
\verse Quand tu entendras un bruit de pas dans les cimes des mûriers, alors hâte-toi, car c`est l`Éternel qui marche devant toi pour battre l`armée des Philistins. 
\verse David fit ce que l`Éternel lui avait ordonné, et il battit les Philistins depuis Guéba jusqu`à Guézer. 

\chapter
\verse David rassembla encore toute l`élite d`Israël, au nombre de trente mille hommes. 
\verse Et David, avec tout le peuple qui était auprès de lui, se mit en marche depuis Baalé Juda, pour faire monter de là l`arche de Dieu, devant laquelle est invoqué le nom de l`Éternel des armées qui réside entre les chérubins au-dessus de l`arche. 
\verse Ils mirent sur un char neuf l`arche de Dieu, et l`emportèrent de la maison d`Abinadab sur la colline; Uzza et Achjo, fils d`Abinadab, conduisaient le char neuf. 
\verse Ils l`emportèrent donc de la maison d`Abinadab sur la colline; Uzza marchait à côté de l`arche de Dieu, et Achjo allait devant l`arche. 
\verse David et toute la maison d`Israël jouaient devant l`Éternel de toutes sortes d`instruments de bois de cyprès, des harpes, des luths, des tambourins, des sistres et des cymbales. 
\verse Lorsqu`ils furent arrivés à l`aire de Nacon, Uzza étendit la main vers l`arche de Dieu et la saisit, parce que les boeufs la faisaient pencher. 
\verse La colère de l`Éternel s`enflamma contre Uzza, et Dieu le frappa sur place à cause de sa faute. Uzza mourut là, près de l`arche de Dieu. 
\verse David fut irrité de ce que l`Éternel avait frappé Uzza d`un tel châtiment. Et ce lieu a été appelé jusqu`à ce jour Pérets Uzza. 
\verse David eut peur de l`Éternel en ce jour-là, et il dit: Comment l`arche de l`Éternel entrerait-elle chez moi? 
\verse Il ne voulut pas retirer l`arche de l`Éternel chez lui dans la cité de David, et il la fit conduire dans la maison d`Obed Édom de Gath. 
\verse L`arche de l`Éternel resta trois mois dans la maison d`Obed Édom de Gath, et l`Éternel bénit Obed Édom et toute sa maison. 
\verse On vint dire au roi David: L`Éternel a béni la maison d`Obed Édom et tout ce qui est à lui, à cause de l`arche de Dieu. Et David se mit en route, et il fit monter l`arche de Dieu depuis la maison d`Obed Édom jusqu`à la cité de David, au milieu des réjouissances. 
\verse Quand ceux qui portaient l`arche de l`Éternel eurent fait six pas, on sacrifia un boeuf et un veau gras. 
\verse David dansait de toute sa force devant l`Éternel, et il était ceint d`un éphod de lin. 
\verse David et toute la maison d`Israël firent monter l`arche de l`Éternel avec des cris de joie et au son des trompettes. 
\verse Comme l`arche de l`Éternel entrait dans la cité de David, Mical, fille de Saül, regardait par la fenêtre, et, voyant le roi David sauter et danser devant l`Éternel, elle le méprisa dans son coeur. 
\verse Après qu`on eut amené l`arche de l`Éternel, on la mit à sa place au milieu de la tente que David avait dressée pour elle; et David offrit devant l`Éternel des holocaustes et des sacrifices d`actions de grâces. 
\verse Quand David eut achevé d`offrir les holocaustes et les sacrifices d`actions de grâces, il bénit le peuple au nom de l`Éternel des armées. 
\verse Puis il distribua à tout le peuple, à toute la multitude d`Israël, hommes et femmes, à chacun un pain, une portion de viande et un gâteau de raisins. Et tout le peuple s`en alla, chacun dans sa maison. 
\verse David s`en retourna pour bénir sa maison, et Mical, fille de Saül, sortit à sa rencontre. Elle dit: Quel honneur aujourd`hui pour le roi d`Israël de s`être découvert aux yeux des servantes de ses serviteurs, comme se découvrirait un homme de rien! 
\verse David répondit à Mical: C`est devant l`Éternel, qui m`a choisi de préférence à ton père et à toute sa maison pour m`établir chef sur le peuple de l`Éternel, sur Israël, c`est devant l`Éternel que j`ai dansé. 
\verse Je veux paraître encore plus vil que cela, et m`abaisser à mes propres yeux; néanmoins je serai en honneur auprès des servantes dont tu parles. 
\verse Or Mical, fille de Saül, n`eut point d`enfants jusqu`au jour de sa mort. 

\chapter
\verse Lorsque le roi habita dans sa maison, et que l`Éternel lui eut donné du repos, après l`avoir délivré de tous les ennemis qui l`entouraient, 
\verse il dit à Nathan le prophète: Vois donc! j`habite dans une maison de cèdre, et l`arche de Dieu habite au milieu d`une tente. 
\verse Nathan répondit au roi: Va, fais tout ce que tu as dans le coeur, car l`Éternel est avec toi. 
\verse La nuit suivante, la parole de l`Éternel fut adressée à Nathan: 
\verse Va dire à mon serviteur David: Ainsi parle l`Éternel: Est-ce toi qui me bâtirais une maison pour que j`en fasse ma demeure? 
\verse Mais je n`ai point habité dans une maison depuis le jour où j`ai fait monter les enfants d`Israël hors d`Égypte jusqu`à ce jour; j`ai voyagé sous une tente et dans un tabernacle. 
\verse Partout où j`ai marché avec tous les enfants d`Israël, ai-je dit un mot à quelqu`une des tribus d`Israël à qui j`avais ordonné de paître mon peuple d`Israël, ai-je dit: Pourquoi ne me bâtissez-vous pas une maison de cèdre? 
\verse Maintenant tu diras à mon serviteur David: Ainsi parle l`Éternel des armées: Je t`ai pris au pâturage, derrière les brebis, pour que tu fusses chef sur mon peuple, sur Israël; 
\verse j`ai été avec toi partout où tu as marché, j`ai exterminé tous tes ennemis devant toi, et j`ai rendu ton nom grand comme le nom des grands qui sont sur la terre; 
\verse j`ai donné une demeure à mon peuple, à Israël, et je l`ai planté pour qu`il y soit fixé et ne soit plus agité, pour que les méchants ne l`oppriment plus comme auparavant 
\verse et comme à l`époque où j`avais établi des juges sur mon peuple d`Israël. Je t`ai accordé du repos en te délivrant de tous tes ennemis. Et l`Éternel t`annonce qu`il te créera une maison. 
\verse Quand tes jours seront accomplis et que tu seras couché avec tes pères, j`élèverai ta postérité après toi, celui qui sera sorti de tes entrailles, et j`affermirai son règne. 
\verse Ce sera lui qui bâtira une maison à mon nom, et j`affermirai pour toujours le trône de son royaume. 
\verse Je serai pour lui un père, et il sera pour moi un fils. S`il fait le mal, je le châtierai avec la verge des hommes et avec les coups des enfants des hommes; 
\verse mais ma grâce ne se retirera point de lui, comme je l`ai retirée de Saül, que j`ai rejeté devant toi. 
\verse Ta maison et ton règne seront pour toujours assurés, ton trône sera pour toujours affermi. 
\verse Nathan rapporta à David toutes ces paroles et toute cette vision. 
\verse Et le roi David alla se présenter devant l`Éternel, et dit: Qui suis-je, Seigneur Éternel, et quelle est ma maison, pour que tu m`aies fait parvenir où je suis? 
\verse C`est encore peu de chose à tes yeux, Seigneur Éternel; tu parles aussi de la maison de ton serviteur pour les temps à venir. Et tu daignes instruire un homme de ces choses, Seigneur Éternel! 
\verse Que pourrait te dire de plus David? Tu connais ton serviteur, Seigneur Éternel! 
\verse A cause de ta parole, et selon ton coeur, tu as fait toutes ces grandes choses pour les révéler à ton serviteur. 
\verse Que tu es donc grand, Éternel Dieu! car nul n`est semblable à toi, et il n`y a point d`autre Dieu que toi, d`après tout ce que nous avons entendu de nos oreilles. 
\verse Est-il sur la terre une seule nation qui soit comme ton peuple, comme Israël, que Dieu est venu racheter pour en former son peuple, pour se faire un nom et pour accomplir en sa faveur, en faveur de ton pays, des miracles et des prodiges, en chassant devant ton peuple, que tu as racheté d`Égypte, des nations et leurs dieux? 
\verse Tu as affermi ton peuple d`Israël, pour qu`il fût ton peuple à toujours; et toi, Éternel, tu es devenu son Dieu. 
\verse Maintenant, Éternel Dieu, fais subsister jusque dans l`éternité la parole que tu as prononcée sur ton serviteur et sur sa maison, et agis selon ta parole. 
\verse Que ton nom soit à jamais glorifié, et que l`on dise: L`Éternel des armées est le Dieu d`Israël! Et que la maison de ton serviteur David soit affermie devant toi! 
\verse Car toi-même, Éternel des armées, Dieu d`Israël, tu t`es révélé à ton serviteur, en disant: Je te fonderai une maison! C`est pourquoi ton serviteur a pris courage pour t`adresser cette prière. 
\verse Maintenant, Seigneur Éternel, tu es Dieu, et tes paroles sont vérité, et tu as annoncé cette grâce à ton serviteur. 
\verse Veuille donc bénir la maison de ton serviteur, afin qu`elle subsiste à toujours devant toi! Car c`est toi, Seigneur Éternel, qui a parlé, et par ta bénédiction la maison de ton serviteur sera bénie éternellement. 

\chapter
\verse Après cela, David battit les Philistins et les humilia, et il enleva de la main des Philistins les rênes de leur capitale. 
\verse Il battit les Moabites, et il les mesura avec un cordeau, en les faisant coucher par terre; il en mesura deux cordeaux pour les livrer à la mort, et un plein cordeau pour leur laisser la vie. Et les Moabites furent assujettis à David, et lui payèrent un tribut. 
\verse David battit Hadadézer, fils de Rehob, roi de Tsoba, lorsqu`il alla rétablir sa domination sur le fleuve de l`Euphrate. 
\verse David lui prit mille sept cents cavaliers et vingt mille hommes de pied; il coupa les jarrets à tous les chevaux de trait, et ne conserva que cent attelages. 
\verse Les Syriens de Damas vinrent au secours d`Hadadézer, roi de Tsoba, et David battit vingt-deux mille Syriens. 
\verse David mit des garnisons dans la Syrie de Damas. Et les Syriens furent assujettis à David, et lui payèrent un tribut. L`Éternel protégeait David partout où il allait. 
\verse Et David prit les boucliers d`or qu`avaient les serviteurs d`Hadadézer, et les apporta à Jérusalem. 
\verse Le roi David prit encore une grande quantité d`airain à Béthach et à Bérothaï, villes d`Hadadézer. 
\verse Thoï, roi de Hamath, apprit que David avait battu toute l`armée d`Hadadézer, 
\verse et il envoya Joram, son fils, vers le roi David, pour le saluer, et pour le féliciter d`avoir attaqué Hadadézer et de l`avoir battu. Car Thoï était en guerre avec Hadadézer. Joram apporta des vases d`argent, des vases d`or, et des vases d`airain. 
\verse Le roi David les consacra à l`Éternel, comme il avait déjà consacré l`argent et l`or pris sur toutes les nations qu`il avait vaincues, 
\verse sur la Syrie, sur Moab, sur les fils d`Ammon, sur les Philistins, sur Amalek, et sur le butin d`Hadadézer, fils de Rehob, roi de Tsoba. 
\verse Au retour de sa victoire sur les Syriens, David se fit encore un nom, en battant dans la vallée du sel dix-huit mille Édomites. 
\verse Il mit des garnisons dans Édom, il mit des garnisons dans tout Édom. Et tout Édom fut assujetti à David. L`Éternel protégeait David partout où il allait. 
\verse David régna sur Israël, et il faisait droit et justice à tout son peuple. 
\verse Joab, fils de Tseruja, commandait l`armée; Josaphat, fils d`Achilud, était archiviste; 
\verse Tsadok, fils d`Achithub, et Achimélec, fils d`Abiathar, étaient sacrificateurs; Seraja était secrétaire; 
\verse Benaja, fils de Jehojada, était chef des Kéréthiens et des Péléthiens; et les fils de David étaient ministres d`état. 

\chapter
\verse David dit: Reste-t-il encore quelqu`un de la maison de Saül, pour que je lui fasse du bien à cause de Jonathan? 
\verse Il y avait un serviteur de la maison de Saül, nommé Tsiba, que l`on fit venir auprès de David. Le roi lui dit: Es-tu Tsiba? Et il répondit: Ton serviteur! 
\verse Le roi dit: N`y a-t-il plus personne de la maison de Saül, pour que j`use envers lui de la bonté de Dieu? Et Tsiba répondit au roi: Il y a encore un fils de Jonathan, perclus des pieds. 
\verse Le roi lui dit: Où est-il? Et Tsiba répondit au roi: Il est dans la maison de Makir, fils d`Ammiel, à Lodebar. 
\verse Le roi David l`envoya chercher dans la maison de Makir, fils d`Ammiel, à Lodebar. 
\verse Et Mephiboscheth, fils de Jonathan, fils de Saül, vint auprès de David, tomba sur sa face et se prosterna. David dit: Mephiboscheth! Et il répondit: Voici ton serviteur. 
\verse David lui dit: Ne crains point, car je veux te faire du bien à cause de Jonathan, ton père. Je te rendrai toutes les terres de Saül, ton père, et tu mangeras toujours à ma table. 
\verse Il se prosterna, et dit: Qu`est ton serviteur, pour que tu regardes un chien mort, tel que moi? 
\verse Le roi appela Tsiba, serviteur de Saül, et lui dit: Je donne au fils de ton maître tout ce qui appartenait à Saül et à toute sa maison. 
\verse Tu cultiveras pour lui les terres, toi, tes fils, et tes serviteurs, et tu feras les récoltes, afin que le fils de ton maître ait du pain à manger; et Mephiboscheth, fils de ton maître, mangera toujours à ma table. Or Tsiba avait quinze fils et vingt serviteurs. 
\verse Il dit au roi: Ton serviteur fera tout ce que le roi mon seigneur ordonne à son serviteur. Et Mephiboscheth mangea à la table de David, comme l`un des fils du roi. 
\verse Mephiboscheth avait un jeune fils, nommé Mica, et tous ceux qui demeuraient dans la maison de Tsiba étaient serviteurs de Mephiboscheth. 
\verse Mephiboscheth habitait à Jérusalem, car il mangeait toujours à la table du roi. Il était boiteux des deux pieds. 

\chapter
\verse Après cela, le roi des fils d`Ammon mourut, et Hanun, son fils, régna à sa place. 
\verse David dit: Je montrerai de la bienveillance à Hanun, fils de Nachasch, comme son père en a montré à mon égard. Et David envoya ses serviteurs pour le consoler au sujet de son père. Lorsque les serviteurs de David arrivèrent dans le pays des fils d`Ammon, 
\verse les chefs des fils d`Ammon dirent à Hanun, leur maître: Penses-tu que ce soit pour honorer ton père que David t`envoie des consolateurs? N`est-ce pas pour reconnaître et explorer la ville, et pour la détruire, qu`il envoie ses serviteurs auprès de toi? 
\verse Alors Hanun saisit les serviteurs de David, leur fit raser la moitié de la barbe, et fit couper leurs habits par le milieu jusqu`au haut des cuisses. Puis il les congédia. 
\verse David, qui fut informé, envoya des gens à leur rencontre, car ces hommes étaient dans une grande confusion; et le roi leur fit dire: Restez à Jéricho jusqu`à ce que votre barbe ait repoussé, et revenez ensuite. 
\verse Les fils d`Ammon, voyant qu`ils s`étaient rendus odieux à David, firent enrôler à leur solde vingt mille hommes de pied chez les Syriens de Beth Rehob et chez les Syriens de Tsoba, mille hommes chez le roi de Maaca, et douze mille hommes chez les gens de Tob. 
\verse A cette nouvelle, David envoya contre eux Joab et toute l`armée, les hommes vaillants. 
\verse Les fils d`Ammon sortirent, et se rangèrent en bataille à l`entrée de la porte; les Syriens de Tsoba et de Rehob, et les hommes de Tob et de Maaca, étaient à part dans la campagne. 
\verse Joab vit qu`il avait à combattre par devant et par derrière. Il choisit alors sur toute l`élite d`Israël un corps, qu`il opposa aux Syriens; 
\verse et il plaça sous le commandement de son frère Abischaï le reste du peuple, pour faire face aux fils d`Ammon. 
\verse Il dit: Si les Syriens sont plus forts que moi, tu viendras à mon secours; et si les fils d`Ammon sont plus forts que toi, j`irai te secourir. 
\verse Sois ferme, et montrons du courage pour notre peuple et pour les villes de notre Dieu, et que l`Éternel fasse ce qui lui semblera bon! 
\verse Joab, avec son peuple, s`avança pour attaquer les Syriens, et ils s`enfuirent devant lui. 
\verse Et quand les fils d`Ammon virent que les Syriens avaient pris la fuite, ils s`enfuirent aussi devant Abischaï et rentrèrent dans la ville. Joab s`éloigna des fils d`Ammon et revint à Jérusalem. 
\verse Les Syriens, voyant qu`ils avaient été battus par Israël, réunirent leurs forces. 
\verse Hadadézer envoya chercher les Syriens qui étaient de l`autre côté du fleuve; et ils arrivèrent à Hélam, ayant à leur tête Schobac, chef de l`armée d`Hadadézer. 
\verse On l`annonça à David, qui assembla tout Israël, passa le Jourdain, et vint à Hélam. Les Syriens se préparèrent à la rencontre de David, et lui livrèrent bataille. 
\verse Mais les Syriens s`enfuirent devant Israël. Et David leur tua les troupes de sept cents chars et quarante mille cavaliers; il frappa aussi le chef de leur armée, Schobac, qui mourut sur place. 
\verse Tous les rois soumis à Hadadézer, se voyant battus par Israël, firent la paix avec Israël et lui furent assujettis. Et les Syriens n`osèrent plus secourir les fils d`Ammon. 

\chapter
\verse L`année suivante, au temps où les rois se mettaient en campagne, David envoya Joab, avec ses serviteurs et tout Israël, pour détruire les fils d`Ammon et pour assiéger Rabba. Mais David resta à Jérusalem. 
\verse Un soir, David se leva de sa couche; et, comme il se promenait sur le toit de la maison royale, il aperçut de là une femme qui se baignait, et qui était très belle de figure. 
\verse David fit demander qui était cette femme, et on lui dit: N`est-ce pas Bath Schéba, fille d`Éliam, femme d`Urie, le Héthien? 
\verse Et David envoya des gens pour la chercher. Elle vint vers lui, et il coucha avec elle. Après s`être purifiée de sa souillure, elle retourna dans sa maison. 
\verse Cette femme devint enceinte, et elle fit dire à David: Je suis enceinte. 
\verse Alors David expédia cet ordre à Joab: Envoie-moi Urie, le Héthien. Et Joab envoya Urie à David. 
\verse Urie se rendit auprès de David, qui l`interrogea sur l`état de Joab, sur l`état du peuple, et sur l`état de la guerre. 
\verse Puis David dit à Urie: Descends dans ta maison, et lave tes pieds. Urie sortit de la maison royale, et il fut suivi d`un présent du roi. 
\verse Mais Urie se coucha à la porte de la maison royale, avec tous les serviteurs de son maître, et il ne descendit point dans sa maison. 
\verse On en informa David, et on lui dit: Urie n`est pas descendu dans sa maison. Et David dit à Urie: N`arrives-tu pas de voyage? Pourquoi n`es-tu pas descendu dans ta maison? 
\verse Urie répondit à David: L`arche et Israël et Juda habitent sous des tentes, mon seigneur Joab et les serviteurs de mon seigneur campent en rase campagne, et moi j`entrerais dans ma maison pour manger et boire et pour coucher avec ma femme! Aussi vrai que tu es vivant et que ton âme est vivante, je ne ferai point cela. 
\verse David dit à Urie: Reste ici encore aujourd`hui, et demain je te renverrai. Et Urie resta à Jérusalem ce jour-là et le lendemain. 
\verse David l`invita à manger et à boire en sa présence, et il l`enivra; et le soir, Urie sortit pour se mettre sur sa couche, avec les serviteurs de son maître, mais il ne descendit point dans sa maison. 
\verse Le lendemain matin, David écrivit une lettre à Joab, et l`envoya par la main d`Urie. 
\verse Il écrivit dans cette lettre: Placez Urie au plus fort du combat, et retirez-vous de lui, afin qu`il soit frappé et qu`il meure. 
\verse Joab, en assiégeant la ville, plaça Urie à l`endroit qu`il savait défendu par de vaillants soldats. 
\verse Les hommes de la ville firent une sortie et se battirent contre Joab; plusieurs tombèrent parmi le peuple, parmi les serviteurs de David, et Urie, le Héthien, fut aussi tué. 
\verse Joab envoya un messager pour faire rapport à David de tout ce qui s`était passé dans le combat. 
\verse Il donna cet ordre au messager: Quand tu auras achevé de raconter au roi tous les détails du combat, 
\verse peut-être se mettra-t-il en fureur et te dira-t-il: Pourquoi vous êtes vous approchés de la ville pour combattre? Ne savez-vous pas qu`on lance des traits du haut de la muraille? 
\verse Qui a tué Abimélec, fils de Jerubbéscheth? n`est-ce pas une femme qui lança sur lui du haut de la muraille un morceau de meule de moulin, et n`en est-il pas mort à Thébets? Pourquoi vous êtes-vous approchés de la muraille? Alors tu diras: Ton serviteur Urie, le Héthien, est mort aussi. 
\verse Le messager partit: et, à son arrivée, il fit rapport à David de tout ce que Joab lui avait ordonné. 
\verse Le messager dit à David: Ces gens ont eu sur nous l`avantage; ils avaient fait une sortie contre nous dans les champs, et nous les avons repoussés jusqu`à l`entrée de la porte; 
\verse les archers ont tiré du haut de la muraille sur tes serviteurs, et plusieurs des serviteurs du roi ont été tués, et ton serviteur Urie, le Héthien, est mort aussi. 
\verse David dit au messager: Voici ce que tu diras à Joab: Ne sois point peiné de cette affaire, car l`épée dévore tantôt l`un, tantôt l`autre; attaque vigoureusement la ville, et renverse-la. Et toi, encourage-le! 
\verse La femme d`Urie apprit que son mari était mort, et elle pleura son mari. 
\verse Quand le deuil fut passé, David l`envoya chercher et la recueillit dans sa maison. Elle devint sa femme, et lui enfanta un fils. Ce que David avait fait déplut à l`Éternel. 

\chapter
\verse L`Éternel envoya Nathan vers David. Et Nathan vint à lui, et lui dit: Il y avait dans une ville deux hommes, l`un riche et l`autre pauvre. 
\verse Le riche avait des brebis et des boeufs en très grand nombre. 
\verse Le pauvre n`avait rien du tout qu`une petite brebis, qu`il avait achetée; il la nourrissait, et elle grandissait chez lui avec ses enfants; elle mangeait de son pain, buvait dans sa coupe, dormait sur son sein, et il la regardait comme sa fille. 
\verse Un voyageur arriva chez l`homme riche. Et le riche n`a pas voulu toucher à ses brebis ou à ses boeufs, pour préparer un repas au voyageur qui était venu chez lui; il a pris la brebis du pauvre, et l`a apprêtée pour l`homme qui était venu chez lui. 
\verse La colère de David s`enflamma violemment contre cet homme, et il dit à Nathan: L`Éternel est vivant! L`homme qui a fait cela mérite la mort. 
\verse Et il rendra quatre brebis, pour avoir commis cette action et pour avoir été sans pitié. 
\verse Et Nathan dit à David: Tu es cet homme-là! Ainsi parle l`Éternel, le Dieu d`Israël: Je t`ai oint pour roi sur Israël, et je t`ai délivré de la main de Saül; 
\verse je t`ai mis en possession de la maison de ton maître, j`ai placé dans ton sein les femmes de ton maître, et je t`ai donné la maison d`Israël et de Juda. Et si cela eût été peu, j`y aurais encore ajouté. 
\verse Pourquoi donc as-tu méprisé la parole de l`Éternel, en faisant ce qui est mal à ses yeux? Tu as frappé de l`épée Urie, le Héthien; tu as pris sa femme pour en faire ta femme, et lui, tu l`as tué par l`épée des fils d`Ammon. 
\verse Maintenant, l`épée ne s`éloignera jamais de ta maison, parce que tu m`as méprisé, et parce que tu as pris la femme d`Urie, le Héthien, pour en faire ta femme. 
\verse Ainsi parle l`Éternel: Voici, je vais faire sortir de ta maison le malheur contre toi, et je vais prendre sous tes yeux tes propres femmes pour les donner à un autre, qui couchera avec elles à la vue de ce soleil. 
\verse Car tu as agi en secret; et moi, je ferai cela en présence de tout Israël et à la face du soleil. 
\verse David dit à Nathan: J`ai péché contre l`Éternel! Et Nathan dit à David: L`Éternel pardonne ton péché, tu ne mourras point. 
\verse Mais, parce que tu as fait blasphémer les ennemis de l`Éternel, en commettant cette action, le fils qui t`est né mourra. 
\verse Et Nathan s`en alla dans sa maison. L`Éternel frappa l`enfant que la femme d`Urie avait enfanté à David, et il fut dangereusement malade. 
\verse David pria Dieu pour l`enfant, et jeûna; et quand il rentra, il passa la nuit couché par terre. 
\verse Les anciens de sa maison insistèrent auprès de lui pour le faire lever de terre; mais il ne voulut point, et il ne mangea rien avec eux. 
\verse Le septième jour, l`enfant mourut. Les serviteurs de David craignaient de lui annoncer que l`enfant était mort. Car ils disaient: Voici, lorsque l`enfant vivait encore, nous lui avons parlé, et il ne nous a pas écoutés; comment oserons-nous lui dire: L`enfant est mort? Il s`affligera bien davantage. 
\verse David aperçut que ses serviteurs parlaient tout bas entre eux, et il comprit que l`enfant était mort. Il dit à ses serviteurs: L`enfant est-il mort? Et ils répondirent: Il est mort. 
\verse Alors David se leva de terre. Il se lava, s`oignit, et changea de vêtements; puis il alla dans la maison de l`Éternel, et se prosterna. De retour chez lui, il demanda qu`on lui servît à manger, et il mangea. 
\verse Ses serviteurs lui dirent: Que signifie ce que tu fais? Tandis que l`enfant vivait, tu jeûnais et tu pleurais; et maintenant que l`enfant est mort, tu te lèves et tu manges! 
\verse Il répondit: Lorsque l`enfant vivait encore, je jeûnais et je pleurais, car je disais: Qui sait si l`Éternel n`aura pas pitié de moi et si l`enfant ne vivra pas? 
\verse Maintenant qu`il est mort, pourquoi jeûnerais-je? Puis-je le faire revenir? J`irai vers lui, mais il ne reviendra pas vers moi. 
\verse David consola Bath Schéba, sa femme, et il alla auprès d`elle et coucha avec elle. Elle enfanta un fils qu`il appela Salomon, et qui fut aimé de l`Éternel. 
\verse Il le remit entre les mains de Nathan le prophète, et Nathan lui donna le nom de Jedidja, à cause de l`Éternel. 
\verse Joab, qui assiégeait Rabba des fils d`Ammon, s`empara de la ville royale, 
\verse et envoya des messagers à David pour lui dire: J`ai attaqué Rabba, et je me suis déjà emparé de la ville des eaux; 
\verse rassemble maintenant le reste du peuple, campe contre la ville, et prends-la, de peur que je ne la prenne moi-même et que la gloire ne m`en soit attribuée. 
\verse David rassembla tout le peuple, et marcha sur Rabba; il l`attaqua, et s`en rendit maître. 
\verse Il enleva la couronne de dessus la tête de son roi: elle pesait un talent d`or et était garnie de pierres précieuses. On la mit sur la tête de David, qui emporta de la ville un très grand butin. 
\verse Il fit sortir les habitants, et il les plaça sous des scies, des herses de fer et des haches de fer, et les fit passer par des fours à briques; il traita de même toutes les villes des fils d`Ammon. David retourna à Jérusalem avec tout le peuple. 

\chapter
\verse Après cela, voici ce qui arriva. Absalom, fils de David, avait une soeur qui était belle et qui s`appelait Tamar; et Amnon, fils de David, l`aima. 
\verse Amnon était tourmenté jusqu`à se rendre malade à cause de Tamar, sa soeur; car elle était vierge, et il paraissait difficile à Amnon de faire sur elle la moindre tentative. 
\verse Amnon avait un ami, nommé Jonadab, fils de Schimea, frère de David, et Jonadab était un homme très habile. 
\verse Il lui dit: Pourquoi deviens-tu, ainsi chaque matin plus maigre, toi, fils de roi? Ne veux-tu pas me le dire? Amnon lui répondit: J`aime Tamar, soeur d`Absalom, mon frère. 
\verse Jonadab lui dit: Mets-toi au lit, et fais le malade. Quand ton père viendra te voir, tu lui diras: Permets à Tamar, ma soeur, de venir pour me donner à manger; qu`elle prépare un mets sous mes yeux, afin que je le voie et que je le prenne de sa main. 
\verse Amnon se coucha, et fit le malade. Le roi vint le voir, et Amnon dit au roi: Je te prie, que Tamar, ma soeur, vienne faire deux gâteaux sous mes yeux, et que je les mange de sa main. 
\verse David envoya dire à Tamar dans l`intérieur des appartements: Va dans la maison d`Amnon, ton frère, et prépare-lui un mets. 
\verse Tamar alla dans la maison d`Amnon, son frère, qui était couché. Elle prit de la pâte, la pétrit, prépara devant lui des gâteaux, et les fit cuire; 
\verse prenant ensuite la poêle, elle les versa devant lui. Mais Amnon refusa de manger. Il dit: Faites sortir tout le monde. Et tout le monde sortit de chez lui. 
\verse Alors Amnon dit à Tamar: Apporte le mets dans la chambre, et que je le mange de ta main. Tamar prit les gâteaux qu`elle avait faits, et les porta à Amnon, son frère, dans la chambre. 
\verse Comme elle les lui présentait à manger, il la saisit et lui dit: Viens, couche avec moi, ma soeur. 
\verse Elle lui répondit: Non, mon frère, ne me déshonore pas, car on n`agit point ainsi en Israël; ne commets pas cette infamie. 
\verse Où irais-je, moi, avec ma honte? Et toi, tu serais comme l`un des infâmes en Israël. Maintenant, je te prie, parle au roi, et il ne s`opposera pas à ce que je sois à toi. 
\verse Mais il ne voulut pas l`écouter; il lui fit violence, la déshonora et coucha avec elle. 
\verse Puis Amnon eut pour elle une forte aversion, plus forte que n`avait été son amour. Et il lui dit: Lève-toi, va-t`en! 
\verse Elle lui répondit: N`augmente pas, en me chassant, le mal que tu m`as déjà fait. 
\verse Il ne voulut pas l`écouter, et appelant le garçon qui le servait, il dit: Qu`on éloigne de moi cette femme et qu`on la mette dehors. Et ferme la porte après elle! 
\verse Elle avait une tunique de plusieurs couleurs; car c`était le vêtement que portaient les filles du roi, aussi longtemps qu`elles étaient vierges. Le serviteur d`Amnon la mit dehors, et ferma la porte après elle. 
\verse Tamar répandit de la cendre sur sa tête, et déchira sa tunique bigarrée; elle mit la main sur sa tête, et s`en alla en poussant des cris. 
\verse Absalom, son frère, lui dit: Amnon, ton frère, a-t-il été avec toi? Maintenant, ma soeur, tais-toi, c`est ton frère; ne prends pas cette affaire trop à coeur. Et Tamar, désolée, demeura dans la maison d`Absalom, son frère. 
\verse Le roi David apprit toutes ces choses, et il fut très irrité. 
\verse Absalom ne parla ni en bien ni en mal avec Amnon; mais il le prit en haine, parce qu`il avait déshonoré Tamar, sa soeur. 
\verse Deux ans après, comme Absalom avait les tondeurs à Baal Hatsor, près d`Éphraïm, il invita tous les fils du roi. 
\verse Absalom alla vers le roi, et dit: Voici, ton serviteur a les tondeurs; que le roi et ses serviteurs viennent chez ton serviteur. 
\verse Et le roi dit à Absalom: Non, mon fils, nous n`irons pas tous, de peur que nous ne te soyons à charge. Absalom le pressa; mais le roi ne voulut point aller, et il le bénit. 
\verse Absalom dit: Permets du moins à Amnon, mon frère, de venir avec nous. Le roi lui répondit: Pourquoi irait-il chez toi? 
\verse Sur les instances d`Absalom, le roi laissa aller avec lui Amnon et tous ses fils. 
\verse Absalom donna cet ordre à ses serviteurs: Faites attention quand le coeur d`Amnon sera égayé par le vin et que je vous dirai: Frappez Amnon! Alors tuez-le; ne craignez point, n`est-ce pas moi qui vous l`ordonne? Soyez fermes, et montrez du courage! 
\verse Les serviteurs d`Absalom traitèrent Amnon comme Absalom l`avait ordonné. Et tous les fils du roi se levèrent, montèrent chacun sur son mulet, et s`enfuirent. 
\verse Comme ils étaient en chemin, le bruit parvint à David qu`Absalom avait tué tous les fils du roi, et qu`il n`en était pas resté un seul. 
\verse Le roi se leva, déchira ses vêtements, et se coucha par terre; et tous ses serviteurs étaient là, les vêtements déchirés. 
\verse Jonadab, fils de Schimea, frère de David, prit la parole et dit: Que mon seigneur ne pense point que tous les jeunes gens, fils du roi, ont été tués, car Amnon seul est mort; et c`est l`effet d`une résolution d`Absalom, depuis le jour où Amnon a déshonoré Tamar, sa soeur. 
\verse Que le roi mon seigneur ne se tourmente donc point dans l`idée que tous les fils du roi sont morts, car Amnon seul est mort. 
\verse Absalom prit la fuite. Or le jeune homme placé en sentinelle leva les yeux et regarda. Et voici, une grande troupe venait par le chemin qui était derrière lui, du côté de la montagne. 
\verse Jonadab dit au roi: Voici les fils du roi qui arrivent! Ainsi se confirme ce que disait ton serviteur. 
\verse Comme il achevait de parler, voici, les fils du roi arrivèrent. Ils élevèrent la voix, et pleurèrent; le roi aussi et tous ses serviteurs versèrent d`abondantes larmes. 
\verse Absalom s`était enfui, et il alla chez Talmaï, fils d`Ammihur, roi de Gueschur. Et David pleurait tous les jours son fils. 
\verse Absalom resta trois ans à Gueschur, où il était allé, après avoir pris la fuite. 
\verse Le roi David cessa de poursuivre Absalom, car il était consolé de la mort d`Amnon. 

\chapter
\verse Joab, fils de Tseruja, s`aperçut que le coeur du roi était porté pour Absalom. 
\verse Il envoya chercher à Tekoa une femme habile, et il lui dit: Montre-toi désolée, et revêts des habits de deuil; ne t`oins pas d`huile, et sois comme une femme qui depuis longtemps pleure un mort. 
\verse Tu iras ainsi vers le roi, et tu lui parleras de cette manière. Et Joab lui mit dans la bouche ce qu`elle devait dire. 
\verse La femme de Tekoa alla parler au roi. Elle tomba la face contre terre et se prosterna, et elle dit: O roi, sauve-moi! 
\verse Le roi lui dit: Qu`as-tu? Elle répondit: Oui, je suis veuve, mon mari est mort! 
\verse Ta servante avait deux fils; il se sont tous deux querellés dans les champs, et il n`y avait personne pour les séparer; l`un a frappé l`autre, et l`a tué. 
\verse Et voici, toute la famille s`est levée contre ta servante, en disant: Livre le meurtrier de son frère! Nous voulons le faire mourir, pour la vie de son frère qu`il a tué; nous voulons détruire même l`héritier! Ils éteindraient ainsi le tison qui me reste, pour ne laisser à mon mari ni nom ni survivant sur la face de la terre. 
\verse Le roi dit à la femme: Va dans ta maison. Je donnerai des ordres à ton sujet. 
\verse La femme de Tekoa dit au roi: C`est sur moi, ô roi mon seigneur, et sur la maison de mon père, que le châtiment va tomber; le roi et son trône n`auront pas à en souffrir. 
\verse Le roi dit: Si quelqu`un parle contre toi, amène-le-moi, et il ne lui arrivera plus de te toucher. 
\verse Elle dit: Que le roi se souvienne de l`Éternel, ton Dieu, afin que le vengeur du sang n`augmente pas la ruine, et qu`on ne détruise pas mon fils! Et il dit: L`Éternel est vivant! il ne tombera pas à terre un cheveu de ton fils. 
\verse La femme dit: Permets que ta servante dise un mot à mon seigneur le roi. Et il dit: Parle! 
\verse La femme dit: Pourquoi penses-tu de la sorte à l`égard du peuple de Dieu, puisqu`il résulte des paroles mêmes du roi que le roi est comme coupable en ne rappelant pas celui qu`il a proscrit? 
\verse Il nous faut certainement mourir, et nous serons comme des eaux répandues à terre et qui ne se rassemblent plus; Dieu n`ôte pas la vie, mais il désire que le fugitif ne reste pas banni de sa présence. 
\verse Maintenant, si je suis venu dire ces choses au roi mon seigneur, c`est que le peuple m`a effrayée. Et ta servante a dit: Je veux parler au roi; peut-être le roi fera-t-il ce que dira sa servante. 
\verse Oui, le roi écoutera sa servante, pour la délivrer de la main de ceux qui cherchent à nous exterminer, moi et mon fils, de l`héritage de Dieu. 
\verse Ta servante a dit: Que la parole de mon seigneur le roi me donne le repos. Car mon seigneur le roi est comme un ange de Dieu, prêt à entendre le bien et le mal. Et que l`Éternel, ton Dieu, soit avec toi! 
\verse Le roi répondit, et dit à la femme: Ne me cache pas ce que je vais te demander. Et la femme dit: Que mon seigneur le roi parle! 
\verse Le roi dit alors: La main de Joab n`est-elle pas avec toi dans tout ceci? Et la femme répondit: Aussi vrai que ton âme est vivante, ô roi mon seigneur, il n`y a rien à droite ni à gauche de tout ce que dit mon seigneur le roi. C`est, en effet, ton serviteur Joab qui m`a donné des ordres, et qui a mis dans la bouche de ta servante toutes ces paroles. 
\verse C`est pour donner à la chose une autre tournure que ton serviteur Joab a fait cela. Mais mon seigneur est aussi sage qu`un ange de Dieu, pour connaître tout ce qui se passe sur la terre. 
\verse Le roi dit à Joab: Voici, je veux bien faire cela; va donc, ramène le jeune homme Absalom. 
\verse Joab tomba la face contre terre et se prosterna, et il bénit le roi. Puis il dit: Ton serviteur connaît aujourd`hui que j`ai trouvé grâce à tes yeux, ô roi mon seigneur, puisque le roi agit selon la parole de son serviteur. 
\verse Et Joab se leva et partit pour Gueschur, et il ramena Absalom à Jérusalem. 
\verse Mais le roi dit: Qu`il se retire dans sa maison, et qu`il ne voie point ma face. Et Absalom se retira dans sa maison, et il ne vit point la face du roi. 
\verse Il n`y avait pas un homme dans tout Israël aussi renommé qu`Absalom pour sa beauté; depuis la plante du pied jusqu`au sommet de la tête, il n`y avait point en lui de défaut. 
\verse Lorsqu`il se rasait la tête, -c`était chaque année qu`il se la rasait, parce que sa chevelure lui pesait, -le poids des cheveux de sa tête était de deux cents sicles, poids du roi. 
\verse Il naquit à Absalom trois fils, et une fille nommée Tamar, qui était une femme belle de figure. 
\verse Absalom demeura deux ans à Jérusalem, sans voir la face du roi. 
\verse Il fit demander Joab, pour l`envoyer vers le roi; mais Joab ne voulut point venir auprès de lui. Il le fit demander une seconde fois; et Joab ne voulut point venir. 
\verse Absalom dit alors à ses serviteurs: Voyez, le champ de Joab est à côté du mien; il y a de l`orge; allez et mettez-y le feu. Et les serviteurs d`Absalom mirent le feu au champ. 
\verse Joab se leva et se rendit auprès d`Absalom, dans sa maison. Il lui dit: Pourquoi tes serviteurs ont-ils mis le feu au champ qui m`appartient? 
\verse Absalom répondit à Joab: Voici, je t`ai fait dire: Viens ici, et je t`enverrai vers le roi, afin que tu lui dises: Pourquoi suis-je revenu de Gueschur? Il vaudrait mieux pour moi que j`y fusse encore. Je désire maintenant voir la face du roi; et s`il y a quelque crime en moi, qu`il me fasse mourir. 
\verse Joab alla vers le roi, et lui rapporta cela. Et le roi appela Absalom, qui vint auprès de lui et se prosterna la face contre terre en sa présence. Le roi baisa Absalom. 

\chapter
\verse Après cela, Absalom se procura un char et des chevaux, et cinquante hommes qui couraient devant lui. 
\verse Il se levait de bon matin, et se tenait au bord du chemin de la porte. Et chaque fois qu`un homme ayant une contestation se rendait vers le roi pour obtenir un jugement, Absalom l`appelait, et disait: De quelle ville es-tu? Lorsqu`il avait répondu: Je suis d`une telle tribu d`Israël, 
\verse Absalom lui disait: Vois, ta cause est bonne et juste; mais personne de chez le roi ne t`écoutera. 
\verse Absalom disait: Qui m`établira juge dans le pays? Tout homme qui aurait une contestation et un procès viendrait à moi, et je lui ferais justice. 
\verse Et quand quelqu`un s`approchait pour se prosterner devant lui, il lui tendait la main, le saisissait et l`embrassait. 
\verse Absalom agissait ainsi à l`égard de tous ceux d`Israël, qui se rendaient vers le roi pour demander justice. Et Absalom gagnait le coeur des gens d`Israël. 
\verse Au bout de quarante ans, Absalom dit au roi: Permets que j`aille à Hébron, pour accomplir le voeu que j`ai fait à l`Éternel. 
\verse Car ton serviteur a fait un voeu, pendant que je demeurais à Gueschur en Syrie; j`ai dit: Si l`Éternel me ramène à Jérusalem, je servirai l`Éternel. 
\verse Le roi lui dit: Va en paix. Et Absalom se leva et partit pour Hébron. 
\verse Absalom envoya des espions dans toutes les tribus d`Israël, pour dire: Quand vous entendrez le son de la trompette, vous direz: Absalom règne à Hébron. 
\verse Deux cents hommes de Jérusalem, qui avaient été invités, accompagnèrent Absalom; et ils le firent en toute simplicité, sans rien savoir. 
\verse Pendant qu`Absalom offrait les sacrifices, il envoya chercher à la ville de Guilo Achitophel, le Guilonite, conseiller de David. La conjuration devint puissante, et le peuple était de plus en plus nombreux auprès d`Absalom. 
\verse Quelqu`un vint informer David, et lui dit: Le coeur des hommes d`Israël s`est tourné vers Absalom. 
\verse Et David dit à tous ses serviteurs qui étaient avec lui à Jérusalem: Levez-vous, fuyons, car il n`y aura point de salut pour nous devant Absalom. Hâtez-vous de partir; sinon, il ne tarderait pas à nous atteindre, et il nous précipiterait dans le malheur et frapperait la ville du tranchant de l`épée. 
\verse Les serviteurs du roi lui dirent: Tes serviteurs feront tout ce que voudra mon seigneur le roi. 
\verse Le roi sortit, et toute sa maison le suivait, et il laissa dix concubines pour garder la maison. 
\verse Le roi sortit, et tout le peuple le suivait, et ils s`arrêtèrent à la dernière maison. 
\verse Tous ses serviteurs, tous les Kéréthiens et tous les Péléthiens, passèrent à ses côtés; et tous les Gathiens, au nombre de six cents hommes, venus de Gath à sa suite, passèrent devant le roi. 
\verse Le roi dit à Ittaï de Gath: Pourquoi viendrais-tu aussi avec nous? Retourne, et reste avec le roi, car tu es étranger, et même tu as été emmené de ton pays. 
\verse Tu es arrivé d`hier, et aujourd`hui je te ferais errer avec nous çà et là, quand je ne sais moi-même où je vais! Retourne, et emmène tes frères avec toi. Que l`Éternel use envers toi de bonté et de fidélité! 
\verse Ittaï répondit au roi, et dit: L`Éternel est vivant et mon seigneur le roi est vivant! au lieu où sera mon seigneur le roi, soit pour mourir, soit pour vivre, là aussi sera ton serviteur. 
\verse David dit alors à Ittaï: Va, passe! Et Ittaï de Gath passa, avec tous ses gens et tous les enfants qui étaient avec lui. 
\verse Toute la contrée était en larmes et l`on poussait de grands cris, au passage de tout le peuple. Le roi passa le torrent de Cédron, et tout le peuple passa vis-à-vis du chemin qui mène au désert. 
\verse Tsadok était aussi là, et avec lui tous les Lévites portant l`arche de l`alliance de Dieu; et ils posèrent l`arche de Dieu, et Abiathar montait, pendant que tout le peuple achevait de sortir de la ville. 
\verse Le roi dit à Tsadok: Reporte l`arche de Dieu dans la ville. Si je trouve grâce aux yeux de l`Éternel, il me ramènera, et il me fera voir l`arche et sa demeure. 
\verse Mais s`il dit: Je ne prends point plaisir en toi! me voici, qu`il me fasse ce qui lui semblera bon. 
\verse Le roi dit encore au sacrificateur Tsadok: Comprends-tu? retourne en paix dans la ville, avec Achimaats, ton fils, et avec Jonathan, fils d`Abiathar, vos deux fils. 
\verse Voyez, j`attendrai dans les plaines du désert, jusqu`à ce qu`il m`arrive des nouvelles de votre part. 
\verse Ainsi Tsadok et Abiathar reportèrent l`arche de Dieu à Jérusalem, et ils y restèrent. 
\verse David monta la colline des oliviers. Il montait en pleurant et la tête couverte, et il marchait nu-pieds; et tous ceux qui étaient avec lui se couvrirent aussi la tête, et ils montaient en pleurant. 
\verse On vint dire à David: Achitophel est avec Absalom parmi les conjurés. Et David dit: O Éternel, réduis à néant les conseils d`Achitophel! 
\verse Lorsque David fut arrivé au sommet, où il se prosterna devant Dieu, voici, Huschaï, l`Arkien, vint au-devant de lui, la tunique déchirée et la tête couverte de terre. 
\verse David lui dit: Si tu viens avec moi, tu me seras à charge. 
\verse Et, au contraire, tu anéantiras en ma faveur les conseils d`Achitophel, si tu retournes à la ville, et que tu dises à Absalom: O roi, je serai ton serviteur; je fus autrefois le serviteur de ton père, mais je suis maintenant ton serviteur. 
\verse Les sacrificateurs Tsadok et Abiathar ne seront-ils pas là avec toi? Tout ce que tu apprendras de la maison du roi, tu le diras aux sacrificateurs Tsadok et Abiathar. 
\verse Et comme ils ont là auprès d`eux leurs deux fils, Achimaats, fils de Tsadok, et Jonathan, fils d`Abiathar, c`est par eux que vous me ferez savoir tout ce que vous aurez appris. 
\verse Huschaï, ami de David, retourna donc à la ville. Et Absalom entra dans Jérusalem. 

\chapter
\verse Lorsque David eut un peu dépassé le sommet, voici, Tsiba, serviteur de Mephiboscheth, vint au-devant de lui avec deux ânes bâtés, sur lesquels il y avait deux cents pains, cent masses de raisins secs, cent de fruits d`été, et une outre de vin. 
\verse Le roi dit à Tsiba: Que veux-tu faire de cela? Et Tsiba répondit: Les ânes serviront de monture à la maison du roi, le pain et les fruits d`été sont pour nourrir les jeunes gens, et le vin pour désaltérer ceux qui seront fatigués dans le désert. 
\verse Le roi dit: Où est le fils de ton maître? Et Tsiba répondit au roi: Voici, il est resté à Jérusalem, car il a dit: Aujourd`hui la maison d`Israël me rendra le royaume de mon père. 
\verse Le roi dit à Tsiba: Voici, tout ce qui appartient à Mephiboscheth est à toi. Et Tsiba dit: Je me prosterne! Que je trouve grâce à tes yeux, ô roi mon seigneur! 
\verse David était arrivé jusqu`à Bachurim. Et voici, il sortit de là un homme de la famille et de la maison de Saül, nommé Schimeï, fils de Guéra. Il s`avança en prononçant des malédictions, 
\verse et il jeta des pierres à David et à tous les serviteurs du roi David, tandis que tout le peuple et tous les hommes vaillants étaient à la droite et à la gauche du roi. 
\verse Schimeï parlait ainsi en le maudissant: Va-t`en, va-t`en, homme de sang, méchant homme! 
\verse L`Éternel fait retomber sur toi tout le sang de la maison de Saül, dont tu occupais le trône, et l`Éternel a livré le royaume entre les mains d`Absalom, ton fils; et te voilà malheureux comme tu le mérites, car tu es un homme de sang! 
\verse Alors Abischaï, fils de Tseruja, dit au roi: Pourquoi ce chien mort maudit-il le roi mon seigneur? Laisse-moi, je te prie, aller lui couper la tête. 
\verse Mais le roi dit: Qu`ai-je affaire avec vous, fils de Tseruja? S`il maudit, c`est que l`Éternel lui a dit: Maudis David! Qui donc lui dira: Pourquoi agis-tu ainsi? 
\verse Et David dit à Abischaï et à tous ses serviteurs: Voici, mon fils, qui est sorti de mes entrailles, en veut à ma vie; à plus forte raison ce Benjamite! Laissez-le, et qu`il maudisse, car l`Éternel le lui a dit. 
\verse Peut-être l`Éternel regardera-t-il mon affliction, et me fera-t-il du bien en retour des malédictions d`aujourd`hui. 
\verse David et ses gens continuèrent leur chemin. Et Schimeï marchait sur le flanc de la montagne près de David, et, en marchant, il maudissait, il jetait des pierres contre lui, il faisait voler la poussière. 
\verse Le roi et tout le peuple qui était avec lui arrivèrent à Ajephim, et là ils se reposèrent. 
\verse Absalom et tout le peuple, les hommes d`Israël, étaient entrés dans Jérusalem; et Achitophel était avec Absalom. 
\verse Lorsque Huschaï, l`Arkien, ami de David, fut arrivé auprès d`Absalom, il lui dit: Vive le roi! vive le roi! 
\verse Et Absalom dit à Huschaï: Voilà donc l`attachement que tu as pour ton ami! Pourquoi n`es-tu pas allé avec ton ami? 
\verse Huschaï répondit à Absalom: C`est que je veux être à celui qu`ont choisi l`Éternel et tout ce peuple et tous les hommes d`Israël, et c`est avec lui que je veux rester. 
\verse D`ailleurs, qui servirai-je? Ne sera-ce pas son fils? Comme j`ai servi ton père, ainsi je te servirai. 
\verse Absalom dit à Achitophel: Consultez ensemble; qu`avons-nous à faire? 
\verse Et Achitophel dit à Absalom: Va vers les concubines que ton père a laissées pour garder la maison; ainsi tout Israël saura que tu t`es rendu odieux à ton père, et les mains de tous ceux qui sont avec toi se fortifieront. 
\verse On dressa pour Absalom une tente sur le toit, et Absalom alla vers les concubines de son père, aux yeux de tout Israël. 
\verse Les conseils donnés en ce temps-là par Achitophel avaient autant d`autorité que si l`on eût consulté Dieu lui-même. Il en était ainsi de tous les conseils d`Achitophel, soit pour David, soit pour Absalom. 

\chapter
\verse Achitophel dit à Absalom: Laisse-moi choisir douze mille hommes! Je me lèverai, et je poursuivrai David cette nuit même. 
\verse Je le surprendrai pendant qu`il est fatigué et que ses mains sont affaiblies, je l`épouvanterai, et tout le peuple qui est avec lui s`enfuira. Je frapperai le roi seul, 
\verse et je ramènerai à toi tout le peuple; la mort de l`homme à qui tu en veux assurera le retour de tous, et tout le peuple sera en paix. 
\verse Cette parole plut à Absalom et à tous les anciens d`Israël. 
\verse Cependant Absalom dit: Appelez encore Huschaï, l`Arkien, et que nous entendions aussi ce qu`il dira. 
\verse Huschaï vint auprès d`Absalom, et Absalom lui dit: Voici comment a parlé Achitophel: devons-nous faire ce qu`il a dit, ou non? Parle, toi! 
\verse Huschaï répondit à Absalom: Pour cette fois le conseil qu`a donné Achitophel n`est pas bon. 
\verse Et Huschaï dit: Tu connais la bravoure de ton père et de ses gens, ils sont furieux comme le serait dans les champs une ourse à qui l`on aurait enlevé ses petits. Ton père est un homme de guerre, et il ne passera pas la nuit avec le peuple; 
\verse voici maintenant, il est caché dans quelque fosse ou dans quelque autre lieu. Et si, dès le commencement, il en est qui tombent sous leurs coups, on ne tardera pas à l`apprendre et l`on dira: Il y a une défaite parmi le peuple qui suit Absalom! 
\verse Alors le plus vaillant, eût-il un coeur de lion, sera saisi d`épouvante; car tout Israël sait que ton père est un héros et qu`il a des braves avec lui. 
\verse Je conseille donc que tout Israël se rassemble auprès de toi, depuis Dan jusqu`à Beer Schéba, multitude pareille au sable qui est sur le bord de la mer. Tu marcheras en personne au combat. 
\verse Nous arriverons à lui en quelque lieu que nous le trouvions, et nous tomberons sur lui comme la rosée tombe sur le sol; et pas un n`échappera, ni lui ni aucun des hommes qui sont avec lui. 
\verse S`il se retire dans une ville, tout Israël portera des cordes vers cette ville, et nous la traînerons au torrent, jusqu`à ce qu`on n`en trouve plus une pierre. 
\verse Absalom et tous les gens d`Israël dirent: Le conseil de Huschaï, l`Arkien, vaut mieux que le conseil d`Achitophel. Or l`Éternel avait résolu d`anéantir le bon conseil d`Achitophel, afin d`amener le malheur sur Absalom. 
\verse Huschaï dit aux sacrificateurs Tsadok et Abiathar: Achitophel a donné tel et tel conseil à Absalom et aux anciens d`Israël; et moi, j`ai conseillé telle et telle chose. 
\verse Maintenant, envoyez tout de suite informer David et faites-lui dire: Ne passe point la nuit dans les plaines du désert, mais va plus loin, de peur que le roi et tout le peuple qui est avec lui ne soient exposés à périr. 
\verse Jonathan et Achimaats se tenaient à En Roguel. Une servante vint leur dire d`aller informer le roi David; car ils n`osaient pas se montrer et entrer dans la ville. 
\verse Un jeune homme les aperçut, et le rapporta à Absalom. Mais ils partirent tous deux en hâte, et ils arrivèrent à Bachurim à la maison d`un homme qui avait un puits dans sa cour, et ils y descendirent. 
\verse La femme prit une couverture qu`elle étendit sur l`ouverture du puits, et elle y répandit du grain pilé pour qu`on ne se doutât de rien. 
\verse Les serviteurs d`Absalom entrèrent dans la maison auprès de cette femme, et dirent: Où sont Achimaats et Jonathan? La femme leur répondit: Ils ont passé le ruisseau. Ils cherchèrent, et ne les trouvant pas, ils retournèrent à Jérusalem. 
\verse Après leur départ, Achimaats et Jonathan remontèrent du puits et allèrent informer le roi David. Ils dirent à David: Levez-vous et hâtez-vous de passer l`eau, car Achitophel a conseillé contre vous telle chose. 
\verse David et tout le peuple qui était avec lui se levèrent et ils passèrent le Jourdain; à la lumière du matin, il n`y en avait pas un qui fût resté à l`écart, pas un qui n`eût passé le Jourdain. 
\verse Achitophel, voyant que son conseil n`était pas suivi, sella son âne et partit pour s`en aller chez lui dans sa ville. Il donna ses ordres à sa maison, et il s`étrangla. C`est ainsi qu`il mourut, et on l`enterra dans le sépulcre de son père. 
\verse David arriva à Mahanaïm. Et Absalom passa le Jourdain, lui et tous les hommes d`Israël avec lui. 
\verse Absalom mit Amasa à la tête de l`armée, en remplacement de Joab; Amasa était fils d`un homme appelé Jithra, l`Israélite, qui était allé vers Abigal, fille de Nachasch et soeur de Tseruja, mère de Joab. 
\verse Israël et Absalom campèrent dans le pays de Galaad. 
\verse Lorsque David fut arrivé à Mahanaïm, Schobi, fils de Nachasch, de Rabba des fils d`Ammon, Makir, fils d`Ammiel, de Lodebar, et Barzillaï, le Galaadite, de Roguelim, 
\verse apportèrent des lits, des bassins, des vases de terre, du froment, de l`orge, de la farine, du grain rôti, des fèves, des lentilles, des pois rôtis, 
\verse du miel, de la crème, des brebis, et des fromages de vache. Ils apportèrent ces choses à David et au peuple qui était avec lui, afin qu`ils mangeassent; car ils disaient: Ce peuple a dû souffrir de la faim, de la fatigue et de la soif, dans le désert. 

\chapter
\verse David passa en revue le peuple qui était avec lui, et il établit sur eux des chefs de milliers et des chefs de centaines. 
\verse Il plaça le tiers du peuple sous le commandement de Joab, le tiers sous celui d`Abischaï, fils de Tseruja, frère de Joab, et le tiers sous celui d`Ittaï, de Gath. Et le roi dit au peuple: Moi aussi, je veux sortir avec vous. 
\verse Mais le peuple dit: Tu ne sortiras point! Car si nous prenons la fuite, ce n`est pas sur nous que l`attention se portera; et quand la moitié d`entre nous succomberait, on n`y ferait pas attention; mais toi, tu es comme dix mille de nous, et maintenant il vaut mieux que de la ville tu puisses venir à notre secours. 
\verse Le roi leur répondit: Je ferai ce qui vous paraît bon. Et le roi se tint à côté de la porte, pendant que tout le peuple sortait par centaines et par milliers. 
\verse Le roi donna cet ordre à Joab, à Abischaï et à Ittaï: Pour l`amour de moi, doucement avec le jeune Absalom! Et tout le peuple entendit l`ordre du roi à tous les chefs au sujet d`Absalom. 
\verse Le peuple sortit dans les champs à la rencontre d`Israël, et la bataille eut lieu dans la forêt d`Éphraïm. 
\verse Là, le peuple d`Israël fut battu par les serviteurs de David, et il y eut en ce jour une grande défaite de vingt mille hommes. 
\verse Le combat s`étendit sur toute la contrée, et la forêt dévora plus de peuple ce jour-là que l`épée n`en dévora. 
\verse Absalom se trouva en présence des gens de David. Il était monté sur un mulet. Le mulet pénétra sous les branches entrelacées d`un grand térébinthe, et la tête d`Absalom fut prise au térébinthe; il demeura suspendu entre le ciel et la terre, et le mulet qui était sous lui passa outre. 
\verse Un homme ayant vu cela vint dire à Joab: Voici, j`ai vu Absalom suspendu à un térébinthe. 
\verse Et Joab dit à l`homme qui lui apporta cette nouvelle: Tu l`as vu! pourquoi donc ne l`as-tu pas abattu sur place? Je t`aurais donné dix sicles d`argent et une ceinture. 
\verse Mais cet homme dit à Joab: Quand je pèserais dans ma main mille sicles d`argent, je ne mettrais pas la main sur le fils du roi; car nous avons entendu cet ordre que le roi t`a donné, à toi, à Abischaï et à Ittaï: Prenez garde chacun au jeune Absalom! 
\verse Et si j`eusse attenté perfidement à sa vie, rien n`aurait été caché au roi, et tu aurais été toi-même contre moi. 
\verse Joab dit: Je ne m`arrêterai pas auprès de toi! Et il prit en main trois javelots, et les enfonça dans le coeur d`Absalom encore plein de vie au milieu du térébinthe. 
\verse Dix jeunes gens, qui portaient les armes de Joab, entourèrent Absalom, le frappèrent et le firent mourir. 
\verse Joab fit sonner de la trompette; et le peuple revint, cessant ainsi de poursuivre Israël, parce que Joab l`en empêcha. 
\verse Ils prirent Absalom, le jetèrent dans une grande fosse au milieu de la forêt, et mirent sur lui un très grand monceau de pierres. Tout Israël s`enfuit, chacun dans sa tente. 
\verse De son vivant, Absalom s`était fait ériger un monument dans la vallée du roi; car il disait: Je n`ai point de fils par qui le souvenir de mon nom puisse être conservé. Et il donna son propre nom au monument, qu`on appelle encore aujourd`hui monument d`Absalom. 
\verse Achimaats, fils de Tsadok, dit: Laisse-moi courir, et porter au roi la bonne nouvelle que l`Éternel lui a rendu justice en le délivrant de la main de ses ennemis. 
\verse Joab lui dit: Ce n`est pas toi qui dois porter aujourd`hui les nouvelles; tu les porteras un autre jour, mais non aujourd`hui, puisque le fils du roi est mort. 
\verse Et Joab dit à Cuschi: Va, et annonce au roi ce que tu as vu. Cuschi se prosterna devant Joab, et courut. 
\verse Achimaats, fils de Tsadok, dit encore à Joab: Quoi qu`il arrive, laisse-moi courir après Cuschi. Et Joab dit: Pourquoi veux-tu courir, mon fils? Ce n`est pas un message qui te sera profitable. 
\verse Quoi qu`il arrive, je veux courir, reprit Achimaats. Et Joab lui dit: Cours! Achimaats courut par le chemin de la plaine, et il devança Cuschi. 
\verse David était assis entre les deux portes. La sentinelle alla sur le toit de la porte vers la muraille; elle leva les yeux et regarda. Et voici, un homme courait tout seul. 
\verse La sentinelle cria, et avertit le roi. Le roi dit: S`il est seul, il apporte des nouvelles. Et cet homme arrivait toujours plus près. 
\verse La sentinelle vit un autre homme qui courait; elle cria au portier: Voici un homme qui court tout seul. Le roi dit: Il apporte aussi des nouvelles. 
\verse La sentinelle dit: La manière de courir du premier me paraît celle d`Achimaats, fils de Tsadok. Et le roi dit: C`est un homme de bien, et il apporte de bonnes nouvelles. 
\verse Achimaats cria, et il dit au roi: Tout va bien! Il se prosterna devant le roi la face contre terre, et dit: Béni soit l`Éternel, ton Dieu, qui a livré les hommes qui levaient la main contre le roi mon seigneur! 
\verse Le roi dit: Le jeune Absalom est-il en bonne santé? Achimaats répondit: J`ai aperçu un grand tumulte au moment où Joab envoya le serviteur du roi et moi ton serviteur; mais je ne sais ce que c`était. 
\verse Et le roi dit: Mets-toi là de côté. Et Achimaats se tint de côté. 
\verse Aussitôt arriva Cuschi. Et il dit: Que le roi mon seigneur apprenne la bonne nouvelle! Aujourd`hui l`Éternel t`a rendu justice en te délivrant de la main de tous ceux qui s`élevaient contre toi. 
\verse Le roi dit à Cuschi: Le jeune homme Absalom est-il en bonne santé? Cuschi répondit: Qu`ils soient comme ce jeune homme, les ennemis du roi mon seigneur et tous ceux qui s`élèvent contre toi pour te faire du mal! 
\verse Alors le roi, saisi d`émotion, monta dans la chambre au-dessus de la porte et pleura. Il disait en marchant: Mon fils Absalom! mon fils, mon fils Absalom! Que ne suis-je mort à ta place! Absalom, mon fils, mon fils! 

\chapter
\verse On vint dire à Joab: Voici, le roi pleure et se lamente à cause d`Absalom. 
\verse Et la victoire, ce jour-là, fut changée en deuil pour tout le peuple, car en ce jour le peuple entendait dire: Le roi est affligé à cause de son fils. 
\verse Ce même jour, le peuple rentra dans la ville à la dérobée, comme l`auraient fait des gens honteux d`avoir pris la fuite dans le combat. 
\verse Le roi s`était couvert le visage, et il criait à haute voix: Mon fils Absalom! Absalom, mon fils, mon fils! 
\verse Joab entra dans la chambre où était le roi, et dit: Tu couvres aujourd`hui de confusion la face de tous tes serviteurs, qui ont aujourd`hui sauvé ta vie, celle de tes fils et de tes filles, celle de tes femmes et de tes concubines. 
\verse Tu aimes ceux qui te haïssent et tu hais ceux qui t`aiment, car tu montres aujourd`hui qu`il n`y a pour toi ni chefs ni serviteurs; et je vois maintenant que, si Absalom vivait et que nous fussions tous morts en ce jour, cela serait agréable à tes yeux. 
\verse Lève-toi donc, sors, et parle au coeur de tes serviteurs! Car je jure par l`Éternel que, si tu ne sors pas, il ne restera pas un homme avec toi cette nuit; et ce sera pour toi pire que tous les malheurs qui te sont arrivés depuis ta jeunesse jusqu`à présent. 
\verse Alors le roi se leva, et il s`assit à la porte. On fit dire à tout le peuple: Voici, le roi est assis à la porte. Et tout le peuple vint devant le roi. Cependant Israël s`était enfui, chacun dans sa tente. 
\verse Et dans toutes les tribus d`Israël, tout le peuple était en contestation, disant: Le roi nous a délivrés de la main de nos ennemis, c`est lui qui nous a sauvés de la main des Philistins; et maintenant il a dû fuir du pays devant Absalom. 
\verse Or Absalom, que nous avions oint pour qu`il régnât sur nous, est mort dans la bataille: pourquoi ne parlez-vous pas de faire revenir le roi? 
\verse De son côté, le roi David envoya dire aux sacrificateurs Tsadok et Abiathar: Parlez aux anciens de Juda, et dites-leur: Pourquoi seriez-vous les derniers à ramener le roi dans sa maison? -Car ce qui se disait dans tout Israël était parvenu jusqu`au roi. 
\verse Vous êtes mes frères, vous êtes mes os et ma chair; pourquoi seriez-vous les derniers à ramener le roi? 
\verse Vous direz aussi à Amasa: N`es-tu pas mon os et ma chair? Que Dieu me traite dans toute sa rigueur, si tu ne deviens pas devant moi pour toujours chef de l`armée à la place de Joab! 
\verse David fléchit le coeur de tous ceux de Juda, comme s`ils n`eussent été qu`un seul homme; et ils envoyèrent dire au roi: Reviens, toi, et tous tes serviteurs. 
\verse Le roi revint et arriva jusqu`au Jourdain; et Juda se rendit à Guilgal, afin d`aller à la rencontre du roi et de lui faire passer le Jourdain. 
\verse Schimeï, fils de Guéra, Benjamite, qui était de Bachurim, se hâta de descendre avec ceux de Juda à la rencontre du roi David. 
\verse Il avait avec lui mille hommes de Benjamin, et Tsiba, serviteur de la maison de Saül, et les quinze fils et les vingt serviteurs de Tsiba. Ils passèrent le Jourdain à la vue du roi. 
\verse Le bateau, mis à la disposition du roi, faisait la traversée pour transporter sa maison; et au moment où le roi allait passer le Jourdain, Schimeï, fils de Guéra, se prosterna devant lui. 
\verse Et il dit au roi: Que mon seigneur ne tienne pas compte de mon iniquité, qu`il oublie que ton serviteur l`a offensé le jour où le roi mon seigneur sortait de Jérusalem, et que le roi n`y ait point égard! 
\verse Car ton serviteur reconnaît qu`il a péché. Et voici, je viens aujourd`hui le premier de toute la maison de Joseph à la rencontre du roi mon seigneur. 
\verse Alors Abischaï, fils de Tseruja, prit la parole et dit: Schimeï ne doit-il pas mourir pour avoir maudit l`oint de l`Éternel? 
\verse Mais David dit: Qu`ai-je affaire avec vous, fils de Tseruja, et pourquoi vous montrez-vous aujourd`hui mes adversaires? Aujourd`hui ferait-on mourir un homme en Israël? Ne sais-je donc pas que je règne aujourd`hui sur Israël? 
\verse Et le roi dit à Schimeï: Tu ne mourras point! Et le roi le lui jura. 
\verse Mephiboscheth, fils de Saül, descendit aussi à la rencontre du roi. Il n`avait point soigné ses pieds, ni fait sa barbe, ni lavé ses vêtements, depuis le jour où le roi s`en était allé jusqu`à celui où il revenait en paix. 
\verse Lorsqu`il se rendit au-devant du roi à Jérusalem, le roi lui dit: Pourquoi n`es-tu pas venu avec moi, Mephiboscheth? 
\verse Et il répondit: O roi mon seigneur, mon serviteur m`a trompé, car ton serviteur, qui est boiteux, avait dit: Je ferai seller mon âne, je le monterai, et j`irai avec le roi. 
\verse Et il a calomnié ton serviteur auprès de mon seigneur le roi. Mais mon seigneur le roi est comme un ange de Dieu. Fais ce qui te semblera bon. 
\verse Car tous ceux de la maison de mon père n`ont été que des gens dignes de mort devant le roi mon seigneur; et cependant tu as mis ton serviteur au nombre de ceux qui mangent à ta table. Quel droit puis-je encore avoir, et qu`ai-je à demander au roi? 
\verse Le roi lui dit: A quoi bon toutes tes paroles? Je l`ai déclaré: Toi et Tsiba, vous partagerez les terres. 
\verse Et Mephiboscheth dit au roi: Qu`il prenne même le tout, puisque le roi mon seigneur rentre en paix dans sa maison. 
\verse Barzillaï, le Galaadite, descendit de Roguelim, et passa le Jourdain avec le roi, pour l`accompagner jusqu`au delà du Jourdain. 
\verse Barzillaï était très vieux, âgé de quatre-vingts ans. Il avait entretenu le roi pendant son séjour à Mahanaïm, car c`était un homme fort riche. 
\verse Le roi dit à Barzillaï: Viens avec moi, je te nourrirai chez moi à Jérusalem. 
\verse Mais Barzillaï répondit au roi: Combien d`années vivrai-je encore, pour que je monte avec le roi à Jérusalem? 
\verse Je suis aujourd`hui âgé de quatre-vingts ans. Puis-je connaître ce qui est bon et ce qui est mauvais? Ton serviteur peut-il savourer ce qu`il mange et ce qu`il boit? Puis-je encore entendre la voix des chanteurs et des chanteuses? Et pourquoi ton serviteur serait-il encore à charge à mon seigneur le roi? 
\verse Ton serviteur ira un peu au delà du Jourdain avec le roi. Pourquoi, d`ailleurs, le roi m`accorderait-il ce bienfait? 
\verse Que ton serviteur s`en retourne, et que je meure dans ma ville, près du sépulcre de mon père et de ma mère! Mais voici ton serviteur Kimham, qui passera avec le roi mon seigneur; fais pour lui ce que tu trouveras bon. 
\verse Le roi dit: Que Kimham passe avec moi, et je ferai pour lui ce qui te plaira; tout ce que tu désireras de moi, je te l`accorderai. 
\verse Quand tout le peuple eut passé le Jourdain et que le roi l`eut aussi passé, le roi baisa Barzillaï et le bénit. Et Barzillaï retourna dans sa demeure. 
\verse Le roi se dirigea vers Guilgal, et Kimham l`accompagna. Tout le peuple de Juda et la moitié du peuple d`Israël avaient fait passer le Jourdain au roi. 
\verse Mais voici, tous les hommes d`Israël abordèrent le roi, et lui dirent: Pourquoi nos frères, les hommes de Juda, t`ont-ils enlevé, et ont-ils fait passer le Jourdain au roi, à sa maison, et à tous les gens de David? 
\verse Tous les hommes de Juda répondirent aux hommes d`Israël: C`est que le roi nous tient de plus près; et qu`y a-t-il là pour vous irriter? Avons-nous vécu aux dépens du roi? Nous a-t-il fait des présents? 
\verse Et les hommes d`Israël répondirent aux hommes de Juda: Le roi nous appartient dix fois autant, et David même plus qu`à vous. Pourquoi nous avez-vous méprisés? N`avons-nous pas été les premiers à proposer de faire revenir notre roi? Et les hommes de Juda parlèrent avec plus de violence que les hommes d`Israël. 

\chapter
\verse Il se trouvait là un méchant homme, nommé Schéba, fils de Bicri, Benjamite. Il sonna de la trompette, et dit: Point de part pour nous avec David, point d`héritage pour nous avec le fils d`Isaï! Chacun à sa tente, Israël! 
\verse Et tous les hommes d`Israël s`éloignèrent de David, et suivirent Schéba, fils de Bicri. Mais les hommes de Juda restèrent fidèles à leur roi, et l`accompagnèrent depuis le Jourdain jusqu`à Jérusalem. 
\verse David rentra dans sa maison à Jérusalem. Le roi prit les dix concubines qu`il avait laissées pour garder la maison, et il les mit dans un lieu où elles étaient séquestrées; il pourvut à leur entretien, mais il n`alla point vers elles. Et elles furent enfermées jusqu`au jour de leur mort, vivant dans un état de veuvage. 
\verse Le roi dit à Amasa: Convoque-moi d`ici à trois jours les hommes de Juda; et toi, sois ici présent. 
\verse Amasa partit pour convoquer Juda; mais il tarda au delà du temps que le roi avait fixé. 
\verse David dit alors à Abischaï: Schéba, fils de Bicri, va maintenant nous faire plus de mal qu`Absalom. Prends toi-même les serviteurs de ton maître et poursuis-le, de peur qu`il ne trouve des villes fortes et ne se dérobe à nos yeux. 
\verse Et Abischaï partit, suivi des gens de Joab, des Kéréthiens et des Péléthiens, et de tous les vaillants hommes; ils sortirent de Jérusalem, afin de poursuivre Schéba, fils de Bicri. 
\verse Lorsqu`ils furent près de la grande pierre qui est à Gabaon, Amasa arriva devant eux. Joab était ceint d`une épée par-dessus les habits dont il était revêtu; elle était attachée à ses reins dans le fourreau, d`où elle glissa, comme Joab s`avançait. 
\verse Joab dit à Amasa: Te portes-tu bien, mon frère? Et de la main droite il saisit la barbe d`Amasa pour le baiser. 
\verse Amasa ne prit point garde à l`épée qui était dans la main de Joab; et Joab l`en frappa au ventre et répandit ses entrailles à terre, sans lui porter un second coup. Et Amasa mourut. Joab et son frère Abischaï marchèrent à la poursuite de Schéba, fils de Bicri. 
\verse Un homme d`entre les gens de Joab resta près d`Amasa, et il disait: Qui veut de Joab et qui est pour David? Qu`il suive Joab! 
\verse Amasa se roulait dans le sang au milieu de la route; et cet homme, ayant vu que tout le peuple s`arrêtait, poussa Amasa hors de la route dans un champ, et jeta sur lui un vêtement, lorsqu`il vit que tous ceux qui arrivaient près de lui s`arrêtaient. 
\verse Quand il fut ôté de la route, chacun suivit Joab, afin de poursuivre Schéba, fils de Bicri. 
\verse Joab traversa toutes les tribus d`Israël dans la direction d`Abel Beth Maaca, et tous les hommes d`élite se rassemblèrent et le suivirent. 
\verse Ils vinrent assiéger Schéba dans Abel Beth Maaca, et ils élevèrent contre la ville une terrasse qui atteignait le rempart. Tout le peuple qui était avec Joab sapait la muraille pour la faire tomber. 
\verse Alors une femme habile se mit à crier de la ville: Écoutez, écoutez! Dites, je vous prie, à Joab: Approche jusqu`ici, je veux te parler! 
\verse Il s`approcha d`elle, et la femme dit: Es-tu Joab? Il répondit: Je le suis. Et elle lui dit: Écoute les paroles de ta servante. Il répondit: J`écoute. 
\verse Et elle dit: Autrefois on avait coutume de dire: Que l`on consulte Abel! Et tout se terminait ainsi. 
\verse Je suis une des villes paisibles et fidèles en Israël; et tu cherches à faire périr une ville qui est une mère en Israël! Pourquoi détruirais-tu l`héritage de l`Éternel? 
\verse Joab répondit: Loin, loin de moi la pensée de détruire et de ruiner! 
\verse La chose n`est pas ainsi. Mais un homme de la montagne d`Éphraïm, nommé Schéba, fils de Bicri, a levé la main contre le roi David; livrez-le, lui seul, et je m`éloignerai de la ville. La femme dit à Joab: Voici, sa tête te sera jetée par la muraille. 
\verse Et la femme alla vers tout le peuple avec sa sagesse; et ils coupèrent la tête à Schéba, fils de Bicri, et la jetèrent à Joab. Joab sonna de la trompette; on se dispersa loin de la ville, et chacun s`en alla dans sa tente. Et Joab retourna à Jérusalem, vers le roi. 
\verse Joab commandait toute l`armée d`Israël; Benaja, fils de Jehojada, était à la tête des Kéréthiens et des Péléthiens; 
\verse Adoram était préposé aux impôts; Josaphat, fils d`Achilud, était archiviste; 
\verse Scheja était secrétaire; Tsadok et Abiathar étaient sacrificateurs; 
\verse et Ira de Jaïr était ministre d`état de David. 

\chapter
\verse Du temps de David, il y eut une famine qui dura trois ans. David chercha la face de l`Éternel, et l`Éternel dit: C`est à cause de Saül et de sa maison sanguinaire, c`est parce qu`il a fait périr les Gabaonites. 
\verse Le roi appela les Gabaonites pour leur parler. -Les Gabaonites n`étaient point d`entre les enfants d`Israël, mais c`était un reste des Amoréens; les enfants d`Israël s`étaient liés envers eux par un serment, et néanmoins Saül avait voulu les frapper, dans son zèle pour les enfants d`Israël et de Juda. - 
\verse David dit aux Gabaonites: Que puis-je faire pour vous, et avec quoi ferai-je expiation, afin que vous bénissiez l`héritage de l`Éternel? 
\verse Les Gabaonites lui répondirent: Ce n`est pas pour nous une question d`argent et d`or avec Saül et avec sa maison, et ce n`est pas à nous qu`il appartient de faire mourir personne en Israël. Et le roi dit: Que voulez-vous donc que je fasse pour vous? 
\verse Ils répondirent au roi: Puisque cet homme nous a consumés, et qu`il avait le projet de nous détruire pour nous faire disparaître de tout le territoire d`Israël, 
\verse qu`on nous livre sept hommes d`entre ses fils, et nous les pendrons devant l`Éternel à Guibea de Saül, l`élu de l`Éternel. Et le roi dit: Je les livrerai. 
\verse Le roi épargna Mephiboscheth, fils de Jonathan, fils de Saül, à cause du serment qu`avaient fait entre eux, devant l`Éternel, David et Jonathan, fils de Saül. 
\verse Mais le roi prit les deux fils que Ritspa, fille d`Ajja, avait enfantés à Saül, Armoni et Mephiboscheth, et les cinq fils que Mérab, fille de Saül, avait enfantés à Adriel de Mehola, fils de Barzillaï; 
\verse et il les livra entre les mains des Gabaonites, qui les pendirent sur la montagne, devant l`Éternel. Tous les sept périrent ensemble; ils furent mis à mort dans les premiers jours de la moisson, au commencement de la moisson des orges. 
\verse Ritspa, fille d`Ajja, prit un sac et l`étendit sous elle contre le rocher, depuis le commencement de la moisson jusqu`à ce que la pluie du ciel tombât sur eux; et elle empêcha les oiseaux du ciel de s`approcher d`eux pendant le jour, et les bêtes des champs pendant la nuit. 
\verse On informa David de ce qu`avait fait Ritspa, fille d`Ajja, concubine de Saül. 
\verse Et David alla prendre les os de Saül et les os de Jonathan, son fils, chez les habitants de Jabès en Galaad, qui les avaient enlevés de la place de Beth Schan, où les Philistins les avaient suspendus lorsqu`ils battirent Saül à Guilboa. 
\verse Il emporta de là les os de Saül et les os de Jonathan, son fils; et l`on recueillit aussi les os de ceux qui avaient été pendus. 
\verse On enterra les os de Saül et de Jonathan, son fils, au pays de Benjamin, à Tséla, dans le sépulcre de Kis, père de Saül. Et l`on fit tout ce que le roi avait ordonné. Après cela, Dieu fut apaisé envers le pays. 
\verse Les Philistins firent encore la guerre à Israël. David descendit avec ses serviteurs, et ils combattirent les Philistins. David était fatigué. 
\verse Et Jischbi Benob, l`un des enfants de Rapha, eut la pensée de tuer David; il avait une lance du poids de trois cents sicles d`airain, et il était ceint d`une épée neuve. 
\verse Abischaï, fils de Tseruja, vint au secours de David, frappa le Philistin et le tua. Alors les gens de David jurèrent, en lui disant: Tu ne sortiras plus avec nous pour combattre, et tu n`éteindras pas la lampe d`Israël. 
\verse Il y eut encore, après cela, une bataille à Gob avec les Philistins. Alors Sibbecaï, le Huschatite, tua Saph, qui était un des enfants de Rapha. 
\verse Il y eut encore une bataille à Gob avec les Philistins. Et Elchanan, fils de Jaaré Oreguim, de Bethléhem, tua Goliath de Gath, qui avait une lance dont le bois était comme une ensouple de tisserand. 
\verse Il y eut encore une bataille à Gath. Il s`y trouva un homme de haute taille, qui avait six doigts à chaque main et à chaque pied, vingt-quatre en tout, et qui était aussi issu de Rapha. 
\verse Il jeta un défi à Israël; et Jonathan, fils de Schimea, frère de David, le tua. 
\verse Ces quatre hommes étaient des enfants de Rapha à Gath. Ils périrent par la main de David et par la main de ses serviteurs. 

\chapter
\verse David adressa à l`Éternel les paroles de ce cantique, lorsque l`Éternel l`eut délivré de la main de tous ses ennemis et de la main de Saül. 
\verse Il dit: L`Éternel est mon rocher, ma forteresse, mon libérateur. 
\verse Dieu est mon rocher, où je trouve un abri, Mon bouclier et la force qui me sauve, Ma haute retraite et mon refuge. O mon Sauveur! tu me garantis de la violence. 
\verse Je m`écrie: Loué soit l`Éternel! Et je suis délivré de mes ennemis. 
\verse Car les flots de la mort m`avaient environné, Les torrents de la destruction m`avaient épouvanté; 
\verse Les liens du sépulcre m`avaient entouré, Les filets de la mort m`avaient surpris. 
\verse Dans ma détresse, j`ai invoqué l`Éternel, J`ai invoqué mon Dieu; De son palais, il a entendu ma voix, Et mon cri est parvenu à ses oreilles. 
\verse La terre fut ébranlée et trembla, Les fondements des cieux frémirent, Et ils furent ébranlés, parce qu`il était irrité. 
\verse Il s`élevait de la fumée dans ses narines, Et un feu dévorant sortait de sa bouche: Il en jaillissait des charbons embrasés. 
\verse Il abaissa les cieux, et il descendit: Il y avait une épaisse nuée sous ses pieds. 
\verse Il était monté sur un chérubin, et il volait, Il paraissait sur les ailes du vent. 
\verse Il faisait des ténèbres une tente autour de lui, Il était enveloppé d`amas d`eaux et de sombres nuages. 
\verse De la splendeur qui le précédait S`élançaient des charbons de feu. 
\verse L`Éternel tonna des cieux, Le Très Haut fit retentir sa voix; 
\verse Il lança des flèches et dispersa mes ennemis, La foudre, et les mit en déroute. 
\verse Le lit de la mer apparut, Les fondements du monde furent découverts, Par la menace de l`Éternel, Par le bruit du souffle de ses narines. 
\verse Il étendit sa main d`en haut, il me saisit, Il me retira des grandes eaux; 
\verse Il me délivra de mon adversaire puissant, De mes ennemis qui étaient plus forts que moi. 
\verse Ils m`avaient surpris au jour de ma détresse, Mais l`Éternel fut mon appui. 
\verse Il m`a mis au large, Il m`a sauvé, parce qu`il m`aime. 
\verse L`Éternel m`a traité selon ma droiture, Il m`a rendu selon la pureté de mes mains; 
\verse Car j`ai observé les voies de l`Éternel, Et je n`ai point été coupable envers mon Dieu. 
\verse Toutes ses ordonnances ont été devant moi, Et je ne me suis point écarté de ses lois. 
\verse J`ai été sans reproche envers lui, Et je me suis tenu en garde contre mon iniquité. 
\verse Aussi l`Éternel m`a rendu selon ma droiture, Selon ma pureté devant ses yeux. 
\verse Avec celui qui est bon tu te montres bon, Avec l`homme droit tu agis selon ta droiture, 
\verse Avec celui qui est pur tu te montres pur, Et avec le pervers tu agis selon sa perversité. 
\verse Tu sauves le peuple qui s`humilie, Et de ton regard, tu abaisses les orgueilleux. 
\verse Oui, tu es ma lumière, ô Éternel! L`Éternel éclaire mes ténèbres. 
\verse Avec toi je me précipite sur une troupe en armes, Avec mon Dieu je franchis une muraille. 
\verse Les voies de Dieu sont parfaites, La parole de l`Éternel est éprouvée; Il est un bouclier pour tous ceux qui se confient en lui. 
\verse Car qui est Dieu, si ce n`est l`Éternel? Et qui est un rocher, si ce n`est notre Dieu? 
\verse C`est Dieu qui est ma puissante forteresse, Et qui me conduit dans la voie droite. 
\verse Il rend mes pieds semblables à ceux des biches, Et il me place sur mes lieux élevés. 
\verse Il exerce mes mains au combat, Et mes bras tendent l`arc d`airain. 
\verse Tu me donnes le bouclier de ton salut, Et je deviens grand par ta bonté. 
\verse Tu élargis le chemin sous mes pas, Et mes pieds ne chancellent point. 
\verse Je poursuis mes ennemis, et je les détruis; Je ne reviens pas avant de les avoir anéantis. 
\verse Je les anéantis, je les brise, et ils ne se relèvent plus; Ils tombent sous mes pieds. 
\verse Tu me ceins de force pour le combat, Tu fais plier sous moi mes adversaires. 
\verse Tu fais tourner le dos à mes ennemis devant moi, Et j`extermine ceux qui me haïssent. 
\verse Ils regardent autour d`eux, et personne pour les sauver! Ils crient à l`Éternel, et il ne leur répond pas! 
\verse Je les broie comme la poussière de la terre, Je les écrase, je les foule, comme la boue des rues. 
\verse Tu me délivres des dissensions de mon peuple; Tu me conserves pour chef des nations; Un peuple que je ne connaissais pas m`est asservi. 
\verse Les fils de l`étranger me flattent, Ils m`obéissent au premier ordre. 
\verse Les fils de l`étranger sont en défaillance, Ils tremblent hors de leurs forteresses. 
\verse Vive l`Éternel est vivant, et béni soit mon rocher! Que Dieu, le rocher de mon salut, soit exalté, 
\verse Le Dieu qui est mon vengeur, Qui m`assujettit les peuples, 
\verse Et qui me fait échapper à mes ennemis! Tu m`élèves au-dessus de mes adversaires, Tu me délivres de l`homme violent. 
\verse C`est pourquoi je te louerai parmi les nations, ô Éternel! Et je chanterai à la gloire de ton nom. 
\verse Il accorde de grandes délivrances à son roi, Et il fait miséricorde à son oint, A David, et à sa postérité, pour toujours. 

\chapter
\verse Voici les dernières paroles de David. Parole de David, fils d`Isaï, Parole de l`homme haut placé, De l`oint du Dieu de Jacob, Du chantre agréable d`Israël. 
\verse L`esprit de l`Éternel parle par moi, Et sa parole est sur ma langue. 
\verse Le Dieu d`Israël a parlé, Le rocher d`Israël m`a dit: Celui qui règne parmi les hommes avec justice, Celui qui règne dans la crainte de Dieu, 
\verse Est pareil à la lumière du matin, quand le soleil brille Et que la matinée est sans nuages; Ses rayons après la pluie font sortir de terre la verdure. 
\verse N`en est-il pas ainsi de ma maison devant Dieu, Puisqu`il a fait avec moi une alliance éternelle, En tous points bien réglée et offrant pleine sécurité? Ne fera-t-il pas germer tout mon salut et tous mes désirs? 
\verse Mais les méchants sont tous comme des épines que l`on rejette, Et que l`on ne prend pas avec la main; 
\verse Celui qui les touche s`arme d`un fer ou du bois d`une lance, Et on les brûle au feu sur place. 
\verse Voici les noms des vaillants hommes qui étaient au service de David. Joscheb Basschébeth, le Tachkemonite, l`un des principaux officiers. Il brandit sa lance sur huit cents hommes, qu`il fit périr en une seule fois. 
\verse Après lui, Éléazar, fils de Dodo, fils d`Achochi. Il était l`un des trois guerriers qui affrontèrent avec David les Philistins rassemblés pour combattre, tandis que les hommes d`Israël se retiraient sur les hauteurs. 
\verse Il se leva, et frappa les Philistins jusqu`à ce que sa main fût lasse et qu`elle restât attachée à son épée. L`Éternel opéra une grande délivrance ce jour-là. Le peuple revint après Éléazar, seulement pour prendre les dépouilles. 
\verse Après lui, Schamma, fils d`Agué, d`Harar. Les Philistins s`étaient rassemblés à Léchi. Il y avait là une pièce de terre remplie de lentilles; et le peuple fuyait devant les Philistins. 
\verse Schamma se plaça au milieu du champ, le protégea, et battit les Philistins. Et l`Éternel opéra une grande délivrance. 
\verse Trois des trente chefs descendirent au temps de la moisson et vinrent auprès de David, dans la caverne d`Adullam, lorsqu`une troupe de Philistins était campée dans la vallée des Rephaïm. 
\verse David était alors dans la forteresse, et il y avait un poste de Philistins à Bethléhem. 
\verse David eut un désir, et il dit: Qui me fera boire de l`eau de la citerne qui est à la porte de Bethléhem? 
\verse Alors les trois vaillants hommes passèrent au travers du camp des Philistins, et puisèrent de l`eau de la citerne qui est à la porte de Bethléhem. Ils l`apportèrent et la présentèrent à David; mais il ne voulut pas la boire, et il la répandit devant l`Éternel. 
\verse Il dit: Loin de moi, ô Éternel, la pensée de faire cela! Boirais-je le sang de ces hommes qui sont allés au péril de leur vie? Et il ne voulut pas la boire. Voilà ce que firent ces trois vaillants hommes. 
\verse Abischaï, frère de Joab, fils de Tseruja, était le chef des trois. Il brandit sa lance sur trois cents hommes, et les tua; et il eut du renom parmi les trois. 
\verse Il était le plus considéré des trois, et il fut leur chef; mais il n`égala pas les trois premiers. 
\verse Benaja, fils de Jehojada, fils d`un homme de Kabtseel, rempli de valeur et célèbre par ses exploits. Il frappa les deux lions de Moab. Il descendit au milieu d`une citerne, où il frappa un lion, un jour de neige. 
\verse Il frappa un Égyptien d`un aspect formidable et ayant une lance à la main; il descendit contre lui avec un bâton, arracha la lance de la main de l`Égyptien, et s`en servit pour le tuer. 
\verse Voilà ce que fit Benaja, fils de Jehojada; et il eut du renom parmi les trois vaillants hommes. 
\verse Il était le plus considéré des trente; mais il n`égala pas les trois premiers. David l`admit dans son conseil secret. 
\verse Asaël, frère de Joab, du nombre des trente. Elchanan, fils de Dodo, de Bethléhem. 
\verse Schamma, de Harod. Élika, de Harod. 
\verse Hélets, de Péleth. Ira, fils d`Ikkesch, de Tekoa. 
\verse Abiézer, d`Anathoth. Mebunnaï, de Huscha. 
\verse Tsalmon, d`Achoach. Maharaï, de Nethopha. 
\verse Héleb, fils de Baana, de Nethopha. Ittaï, fils de Ribaï, de Guibea des fils de Benjamin. 
\verse Benaja, de Pirathon. Hiddaï, de Nachalé Gaasch. 
\verse Abi Albon, d`Araba. Azmaveth, de Barchum. 
\verse Eliachba, de Schaalbon. Bené Jaschen. Jonathan. 
\verse Schamma, d`Harar. Achiam, fils de Scharar, d`Arar. 
\verse Éliphéleth, fils d`Achasbaï, fils d`un Maacathien. Éliam, fils d`Achitophel, de Guilo. 
\verse Hetsraï, de Carmel. Paaraï, d`Arab. 
\verse Jigueal, fils de Nathan, de Tsoba. Bani, de Gad. 
\verse Tsélek, l`Ammonite. Naharaï, de Beéroth, qui portait les armes de Joab, fils de Tseruja. 
\verse Ira, de Jéther. Gareb, de Jéther. 
\verse Urie, le Héthien. En tout, trente-sept. 

\chapter
\verse La colère de l`Éternel s`enflamma de nouveau contre Israël, et il excita David contre eux, en disant: Va, fais le dénombrement d`Israël et de Juda. 
\verse Et le roi dit à Joab, qui était chef de l`armée et qui se trouvait près de lui: Parcours toutes les tribus d`Israël, depuis Dan jusqu`à Beer Schéba; qu`on fasse le dénombrement du peuple, et que je sache à combien il s`élève. 
\verse Joab dit au roi: Que l`Éternel, ton Dieu, rende le peuple cent fois plus nombreux, et que les yeux du roi mon seigneur le voient! Mais pourquoi le roi mon seigneur veut-il faire cela? 
\verse Le roi persista dans l`ordre qu`il donnait à Joab et aux chefs de l`armée; et Joab et les chefs de l`armée quittèrent le roi pour faire le dénombrement du peuple d`Israël. 
\verse Ils passèrent le Jourdain, et ils campèrent à Aroër, à droite de la ville qui est au milieu de la vallée de Gad, et près de Jaezer. 
\verse Ils allèrent en Galaad et dans le pays de Thachthim Hodschi. Ils allèrent à Dan Jaan, et aux environs de Sidon. 
\verse Ils allèrent à la forteresse de Tyr, et dans toutes les villes des Héviens et des Cananéens. Ils terminèrent par le midi de Juda, à Beer Schéba. 
\verse Ils parcoururent ainsi tout le pays, et ils arrivèrent à Jérusalem au bout de neuf mois et vingt jours. 
\verse Joab remit au roi le rôle du dénombrement du peuple: il y avait en Israël huit cent mille hommes de guerre tirant l`épée, et en Juda cinq cent mille hommes. 
\verse David sentit battre son coeur, après qu`il eut ainsi fait le dénombrement du peuple. Et il dit à l`Éternel: J`ai commis un grand péché en faisant cela! Maintenant, ô Éternel, daigne pardonner l`iniquité de ton serviteur, car j`ai complètement agi en insensé! 
\verse Le lendemain, quand David se leva, la parole de l`Éternel fut ainsi adressée à Gad le prophète, le voyant de David: 
\verse Va dire à David: Ainsi parle l`Éternel: Je te propose trois fléaux; choisis-en un, et je t`en frapperai. 
\verse Gad alla vers David, et lui fit connaître la chose, en disant: Veux-tu sept années de famine dans ton pays, ou bien trois mois de fuite devant tes ennemis qui te poursuivront, ou bien trois jours de peste dans ton pays? Maintenant choisis, et vois ce que je dois répondre à celui qui m`envoie. 
\verse David répondit à Gad: Je suis dans une grande angoisse! Oh! tombons entre les mains de l`Éternel, car ses compassions sont immenses; mais que je ne tombe pas entre les mains des hommes! 
\verse L`Éternel envoya la peste en Israël, depuis le matin jusqu`au temps fixé; et, de Dan à Beer Schéba, il mourut soixante-dix mille hommes parmi le peuple. 
\verse Comme l`ange étendait la main sur Jérusalem pour la détruire, l`Éternel se repentit de ce mal, et il dit à l`ange qui faisait périr le peuple: Assez! Retire maintenant ta main. L`ange de l`Éternel était près de l`aire d`Aravna, le Jébusien. 
\verse David, voyant l`ange qui frappait parmi le peuple, dit à l`Éternel: Voici, j`ai péché! C`est moi qui suis coupable; mais ces brebis, qu`ont-elles fait? Que ta main soit donc sur moi et sur la maison de mon père! 
\verse Ce jour-là, Gad vint auprès de David, et lui dit: Monte, élève un autel à l`Éternel dans l`aire d`Aravna, le Jébusien. 
\verse David monta, selon la parole de Gad, comme l`Éternel l`avait ordonné. 
\verse Aravna regarda, et il vit le roi et ses serviteurs qui se dirigeaient vers lui; et Aravna sortit, et se prosterna devant le roi, le visage contre terre. 
\verse Aravna dit: Pourquoi mon seigneur le roi vient-il vers son serviteur? Et David répondit: Pour acheter de toi l`aire et pour y bâtir un autel à l`Éternel, afin que la plaie se retire de dessus le peuple. 
\verse Aravna dit à David: Que mon seigneur le roi prenne l`aire, et qu`il y offre les sacrifices qu`il lui plaira; vois, les boeufs seront pour l`holocauste, et les chars avec l`attelage serviront de bois. 
\verse Aravna donna le tout au roi. Et Aravna dit au roi: Que l`Éternel, ton Dieu, te soit favorable! 
\verse Mais le roi dit à Aravna: Non! Je veux l`acheter de toi à prix d`argent, et je n`offrirai point à l`Éternel, mon Dieu, des holocaustes qui ne me coûtent rien. Et David acheta l`aire et les boeufs pour cinquante sicles d`argent. 
\verse David bâtit là un autel à l`Éternel, et il offrit des holocaustes et des sacrifices d`actions de grâces. Alors l`Éternel fut apaisé envers le pays, et la plaie se retira d`Israël. 
