

\usepackage{marginfix}
\usepackage{xparse}
\usepackage{makecell}

%\renewcommand\theadalign{cl}
%\renewcommand\theadfont{\bfseries}
%\renewcommand\theadgape{\Gape[4pt]}
%\renewcommand\cellgape{\Gape[4pt]}

\NewDocumentCommand{\LORD}{}{
  \textsc{Lord}\xspace
}

\ExplSyntaxOn

\newcounter{book}
\newcounter{verse}
\setcounter{book}{0}
\bool_new:c{is_first_verse}
\bool_new:c{is_first_chap}
\DeclareDocumentCommand{\book}{omo}{
  % #1 keys
  % #2 title
  % #3 key (default #2)

  % Step the book counter
  \refstepcounter{book}

  % define book key (used in notes)
  \edef\@currbook{\IfNoValueTF{#3}{#2}{#3}}

  % Don't display page numbers on pages that list a book
  \thispagestyle{empty}

  % Typeset main title
  \par\noindent
  {\Huge #2\vspace{1ex}\newline}

  % Typeset subtitle if it exists, otherwise do nothing
  \IfValueTF{#1}{{\LARGE #1}\par\vspace{4ex}}{}

  % Reset the chapter counter
  \setcounter{chapter}{0}
  \bool_set_true:c{is_first_chap}
}

\RequirePackage{lettrine}
\DeclareDocumentCommand{\chapter}{o}{
  % #1 Title

  % Step the chapter counter
  \refstepcounter{chapter}

  % Seperate chapters with a bit of space
  \vspace{2ex plus .2ex minus .2ex}

  % Fancy chapter numbers
  \bool_if:cTF{is_first_chap}
  {
		\bool_set_false:c{is_first_chap}
		{\begin{center}Chapitre   {\Huge \arabic{chapter}} \end{center}}
  }
  {
  		% If we have a chapter title, typeset it centered on the first line
  		\IfValueTF{#1}
    		{\begin{center}#1\\Chapitre {\Huge \arabic{chapter}} \end{center}}
			{\begin{center}Chapitre {\Huge \arabic{chapter}} \end{center}}
  }
	

  % Print the chapter number
  %{\centerline{Chapitre {\Huge \arabic{chapter}} du }}

  % Reset the verse counter

  \setcounter{verse}{0}

  % Set prohibit typesetting the index of the first verse
  \bool_set_true:c{is_first_verse}
}

\DeclareDocumentCommand{\verse}{}{
  % Step the verse counter
  \refstepcounter{verse}

  % If this is the first verse, don't typeset its little index
  \bool_if:cTF{is_first_verse}
    {\bool_set_false:c{is_first_verse}}
    {\textsuperscript{\color{magenta}\arabic{verse}}}
    \noindent
  }

\RequirePackage{csquotes,xcolor}
\DeclareDocumentCommand{\NewSpeaker}{mommo}{
  % #1 csname
  % #2 arbitrary insert code before
  % #3 text color direct
  % #4 text color indirect
  % #5 arbitrary insert code after
  \NewDocumentCommand{#1}{sm}{
    % If starred, change the color and star the quote
    \IfBooleanTF{##1}
      {
        \textcolor{black}{
          \enquote*{\textcolor {#4} {##2}}
        }
      }
      {
        \textcolor{black}{
          \enquote{\textcolor {#3} {##2}}
        }
      }
    }
  }



\RequirePackage{xstring}
\newcommand{\testbook}[1]{\def\inputbook@only{#1}}
\newif\iftypeset
\typesettrue
\newcommand{\inputbook}[1]{
  \iftypeset
  \begingroup
    \cs_if_exist:NTF{\inputbook@only}
      {\IfStrEq{#1}{\inputbook@only}{\input{\bibleversion/#1}\clearpage}}
      {\input{\bibleversion/#1}\clearpage}
  \endgroup
  \fi
}

\ExplSyntaxOff

% Create a macro to group some information together in a margin note
\usepackage{zref-savepos,zref-abspage}
\usepackage{expl3}
\ExplSyntaxOn\makeatletter
\int_new:N \g_tassio_note_int
\seq_new:N\g_tassio_note_seq

\newcommand{\defi}[1]{%
  \int_gincr:N \g_tassio_note_int
  \bool_if:nTF
  {
%   \int_compare_p:n 
%    {
%     \zposy{tassionotepos\int_eval:n{\g_tassio_note_int}} 
%     =  
%     \zposy{tassionotepos\int_eval:n{\g_tassio_note_int +1}}
%    }
%   &&
   \int_compare_p:n  
   {
    \zref@extractdefault { tassionotepage\int_eval:n{\g_tassio_note_int } }{abspage}{-1}
    =
    \zref@extractdefault { tassionotepage\int_eval:n{\g_tassio_note_int +1} }{abspage}{-1}
   }
  }
   {
    %\seq_gput_left:Nn \g_tassio_note_seq {#1}
    \seq_gput_right:Nn \g_tassio_note_seq {#1}
   }
   {
    %\seq_gput_left:Nn \g_tassio_note_seq {#1}
    \seq_gput_right:Nn \g_tassio_note_seq {#1}
    \sidenote{CONCORDANCES\\ \tiny \g_tassio_note_int \seq_use:Nn \g_tassio_note_seq {,~}}%
    \seq_gclear:N \g_tassio_note_seq
   } 
  \zref@label {tassionotepage\int_use:N\g_tassio_note_int}
  \zsaveposy {tassionotepos\int_use:N\g_tassio_note_int}
  \emph{#1}
}
\ExplSyntaxOff\makeatother

% Front matter
\newcommand{\SetVersion}[1]{\def\bibleversion{#1}}


\usepackage[utf8]{inputenc}
